\documentclass[b5paper]{book}
\usepackage{geometry}
\usepackage{emptypage}

\tracinglostchars=3

\usepackage[hangul]{kotex}
\ifluatex
  \defaultfontfeatures+{Renderer=HarfBuzz}
  \registerpunctuations{`-}
  \def\hyphlatin#1{\unregisterpunctuations{`-}#1\registerpunctuations{`-}}
\else
  \hangulhyphens
  \def\hyphlatin#1{{\latinhyphens#1}}
\fi
\setmainhangulfont{Noto Serif CJK KR}
  [Script=Hangul, Language=Korean, AutoFakeSlant]
\setsanshangulfont{Noto Sans CJK KR}
  [Script=Hangul, Language=Korean, AutoFakeSlant]

\makeatletter
\def\@makechapterhead#1{%
  \vspace*{50\p@}%
  {\parindent \z@ \raggedright \normalfont
    \ifnum \c@secnumdepth >\m@ne
      \if@mainmatter
        \huge\bfseries \@chapapp\space \thechapter
        \par\nobreak
        \vskip 18\p@
      \fi
    \fi
    \interlinepenalty\@M
    \Huge \bfseries #1\par\nobreak
    \vskip 40\p@
  }}
\makeatother

\renewcommand\chaptermark[1]{\markboth{\textit{#1}}{}}

\def\para#1{\leavevmode
  \marginpar{\sffamily\footnotesize #1}\markright{#1}\ignorespaces}
\def\hanja#1{\begingroup\scriptsize #1\endgroup}
\def\latin#1{\begingroup\footnotesize #1\endgroup}
\def\hemph#1{\begingroup\sffamily\bfseries #1\endgroup}
\def\hanjalatin#1#2{\hanja{#1}\hskip.15em plus.05em minus.02em\latin{#2}}

\linespread{1.4}
\skip\footins=10pt plus 8pt minus 2pt

\includeonly{ch4}

\begin{document}

\title{고대법\\
\large 사회의 초기 역사 및 그것의 근대 관념과의 관계에 대하여}
\author{헨리 섬너 메인 지음\\
김 도현 옮김}
\date{1861 (1920)}
\maketitle

\mainmatter


\chapter{고대 법전}

우리가 알고 있는 가장 유명한 법체계는 법전과 함께 시작해서
법전과 함께 끝난다.\footnote{시작은 12표법, 끝은 로마법대전을 뜻한다.}
로마법의 해설자들은 그들의 법체계가 \wi{12표법}\latin{Twelve Decemviral Tables}에 기초하고 있다는,
따라서 성문법에 기초하고 있다는 취지의 말을 그들의 역사 내내 시종일관 해왔다.
한 가지 예외를 제외하면,\footnote{%
  사용취득(usucapio)에 관한 시민법을 말하는 듯하다.
  본서 제8장의 \hyperlink{usucapio}{사용취득}에 관한 설명 참조.}
12표법 이전으로 거슬러 올라가는 제도로 로마에서 인정된 것은 없었다.
로마법이 법전의 후예라는 이론, 영국법은 기억할 수 없는
옛 불문\hanja{不文}의 전통에 기원한다는 이론은
로마법의 발달이 왜 영국법의 발달과 달랐는지를 설명하는 주요 이론들이다.
두 이론 다 사실과 정확히 들어맞지는 않지만, 각각 대단히 중요한 결과들을 낳았다.

\para{원초적 법관념}
12표법의 공표가 법의 역사를 다루는 출발점이 될 수 없음은 말할 것도 없다.
문명화된 민족이면 거의 다 고대 로마의 법전과 비슷한 것을 가지고 있었다.
또한 로마와 헬레니즘 세계에 관한 한, 비교적 서로 가까운 시대에
그러한 법전이 두 세계에 널리 확산되었었다.
그것들은 무척 유사한 상황에서 등장했고, 우리가 아는 한
무척 유사한 원인으로 만들어졌다.
많은 법현상들이 이들 법전에 시기적으로 앞서거나 뒤따랐음은 말할 것도 없다.
적지 않은 문헌기록들이 남아있어 법의 초기 현상들을 우리에게 알려준다.
하지만 언어학이 산스크리트 문헌을 완전히 분석해내기 전까지는
우리에게 주어진 가장 좋은 인식 원천은 그리스의 호메로스의 시임에 틀림없다.
물론 이들은 실제 사건들을 기록한 역사로서가 아니라,
\wi{호메로스}가 알고 있던 사회의 상태를 기술한 것으로,
그러나 완전히 이상화시키지 않고 기술한 것으로 읽어야 할 것이다.
영웅시대의 어떤 특징이나 전사들의 용기, 신들의 힘 따위가
시적 상상력에 의해 과장됐을 수 있지만, 도덕적 또는 형이상학적 관념에 의해
그의 시가 오염되었다고 믿을 이유는 없다.
도덕이나 형이상학은 아직 의식적인 고찰의 주제가 아니었기 때문이다.
이런 점에서, 비슷한 초기 시대를 다룬다고 하면서
철학적 또는 신학적 영향 하에서 만들어진 후대의 문헌들보다
호메로스의 시가 훨씬 더 신뢰할 만하다.
법개념의 초기 형태를 발견하려는 우리에게 그것들은 더없이 소중하다.
법학자에게 이들 원초적 관념이 갖는 중요성은
지질학자에게 초기 지구의 지각이 갖는 중요성에 비할 만하다.
거기에는 후대의 법에 의해 발현될 모든 형태들이 다 담겨있을 수 있다.
조급함이나 편견으로 인해 기껏해야 피상적인 조사만 하고는 더는
아무 것도 하지 않은 탓에 오늘날 우리의 법학은 불만족스런 상태에 머물러있다.
법학자들의 탐구는 실로 물리학이나 생리학에서 관찰이 억측을 대체하기 이전
상태와 비슷한 상태에 머물러있다.
그럴듯하고 포괄적이지만 전혀 증명되지 않은 이론들, 가령
자연법이나 사회계약론 따위의 이론이 널리 인기를 구가하여
사회와 법의 원초적 역사에 대한 냉철한 연구를 압도하고 있다.
저 이론들은 진리를 가리고 있거니와,
진리가 발견될 수 있는 유일한 영역으로부터 관심이 멀어지게 할 뿐만 아니라,
일단 길들여지고 믿게 되면 후대의 법학에 참으로 엄청난 영향력을
행사할 수 있다.

\para{테미스테스}
법이나 생활규칙이라는, 이제는 무척 발달한 관념에 관련된 최초의 인식은
\wi{호메로스}가 사용한 용어 ``테미스''\latin{Themis}와
``\wi{테미스테스}''\latin{Themistes}에 담겨있다.
주지하듯이 후대 그리스의 신들 중에서 테미스는 정의의 여신으로 나타난다.
하지만 이것은 근대적인, 무척 발달된 관념의 산물이다.
<<\wi{일리아스}>>에서 제우스의 판결보조자로 묘사된 테미스는 전혀 다른 의미를 가졌다.
오늘날 원시사회의 믿을 만한 관찰자들이 밝혀놓았듯이,
인류의 유년기에 인간은
지속적인 혹은 반복적인 사건을
인격의 작용을 가정함으로써만
설명할 수 있었다.
그리하여 바람이 부는 것도 인격이었고 물론 신적 인격이었다.
태양이 뜨고 정점에 이르고 지는 것도 인격이었고 신적 인격이었다.
대지가 수확물을 내주는 것도 인격이었고 신이었다.
물리적 세계가 그러하듯이 도덕적 세계도 마찬가지였다.
왕이 분쟁에서 판결을 내릴 때, 판결은 신적 영감의 결과로 이해되었다.
왕들에게, 또는 왕중의 왕인 신들에게, 판결을 제안하는 신이
바로 \hemph{테미스}였다.
이 관념의 특이함은 복수형 표현에서 나타난다.
테미스의 복수형 \hemph{테미스테스}는 신이 판사에게 지시한
판결들 자체를 뜻했다.
왕들은 바로 꺼내 쓸 수 있는 테미스테스의 저장고를 갖고 있다고 생각되었다.
그러나 이것은 법률이 아니라
판결---게르만인들이 ``둠''\latin{doom}이라고 부르는 것에 정확히 일치한다---이었다는
점을 유의해야 한다.
그로트\latin{George Grote} 씨의 <<그리스 역사>>\latin{History of Greece}에 따르면,
``제우스나 지상의 인간 왕은 입법자가 아니라 판사였다.''
그에게는 테미스테스가 주어져 있으나,
위로부터 주어진 것이라는 믿음에 부합하게,
판결들은 어떤 일관된 원칙으로 연결되어 있다고 관념되지 못했다.
그것은 따로따로 분리된 개별적인 판결들이었다.

호메로스의 시에서도 이러한 관념은 잠시 동안의 것이었음을 알 수 있다.
단순한 구조의 고대사회에서 상황의 유사성은 오늘날보다 흔한 일이었을 테고,
유사한 소송이 연달아 제기됨에 따라 판결들도 비슷해지는 경향이 나타났을 것이다.
여기서 우리는 관습의 기원 혹은 초기 형태를 발견할 수 있거니와,
이것은 테미스테스, 즉 판결보다 나중에 등장하는 관념인 것이다.\footnote{%
  그러나 왕의 테미스테스도, 이론적으로는 신적 영감에 의한 것이라 해도,
  실제로는 당시의 관습이나 관행에 기초하였을 것이 틀림없다.
  \latin{Maine, \textit{Early Law and Custom}, 1883, p.\,163.} }
근대적 사고방식 탓에 우리는 관습의 관념이 사법적 판결에 선행하고
판결은 관습을 확인하거나 그 위반을 벌하는 것이라고 미리 단정짓는
경향이 강하지만, 관념의 역사적 발달은 내가 제시한 순서대로였음이
틀림없어 보인다.
맹아적 관습을 지칭하는 \wi{호메로스}의 용어는 때로 단수형 ``테미스''였고,
종종 ``\wi{디케}''\latin{Dike}였거니와, 그 뜻은 ``판결''과 ``관습'' 또는
``관행''을 넘나드는 것이었다.
노모스\greek{Νόμος}, 즉 `법'은 후대 그리스 사회의 정치용어로서 대단히 중요하고
유명한 것이지만, 호메로스의 시에는 등장하지 않는다.

신의 작용이라는 이러한 관념,
테미스테스를 제안하고 테미스에 인격화되어있는 신의 작용이라는 관념은
피상적인 연구로는 혼동하기 쉬운
다른 원시적 관념들과 엄격히 구분되어야 한다.
힌두의 \wi{마누법전}에 나타나는, 신이 완성된 법전을 명령한다는 관념은
훨씬 최근의 진보된 관념의 계열에 속하는 것으로 보인다.
``테미스''와 ``테미스테스''는
오랫동안 끈질기게 인간의 정신을 지배했던 관념,
신적 영향력이 모든 생활관계와 모든 사회제도를 지탱하고 지지한다는 관념과
훨씬 더 가깝다.
초기 법에서, 그리고 초기의 정치사상에서,
이러한 믿음의 징후는 모든 면에서 나타난다.
초자연적인 통치권자가 당시의 모든 주요 제도들---국가, \wi{씨족}, 가족---을
성별\hanja{聖別}하고 통합하는 것으로 관념된다.
이러한 제도들 속에서 다양한 관계로 집단을 형성하는 인간은
주기적으로 공동의 제의를 수행하고 공동의 희생물을 바칠 의무를 진다.
때로 이러한 의무는
그들이 수행하는 정화의식과 속죄의식에서
더욱 강하게 인식되거니와,
이는 의도치 않게 또는 부주의로 저지른 불경한 짓에 대해 죄를
사하여 달라는 의미를 띠는 것이었다.
고전문헌에 익숙한 독자라면 누구나,
초기 로마의 입양법과 유언법에 중대한 영향을 미쳤던
\wi{씨족제사}\hanjalatin{氏族祭祀}{sacra gentilicia}에 대해 알고 있을 것이다.
무척 진기한 원시사회의 특징들이 고정되어 남아있는
힌두 관습법에서는 지금도 거의 모든 신분법과 상속법 규칙들이
망자의 장례식에서,
즉 가\hanja{家}의 연속성에 단절이 생기는 때에,
의례를 엄정하게 거행하는 것에 달려있다.

\para{벤담의 분석}
이 단계의 법을 떠나기 전에,
특히 영국의 학자들이 유의해야 점을 지적하고자 한다.
\wi{벤담}은 <<정부론 단편>>\latin{Fragment on Government}에서,
\wi{오스틴}은 <<법학의 영역 확정>>\latin{Province of Jurisprudence Determined}에서,
법을 입법자의 \hemph{명령}으로,
그리하여 시민들에게 부과된 \hemph{의무}로,
그리고 불복종에 대해 주어지는 \hemph{제재}의 위협으로 선언한다.
나아가 법의 첫째 요소인 \hemph{명령}은 하나의 행위가 아니라
일련의 또는 다수의 동종 행위들을 지시해야 한다고 단언한다.
이렇게 여러 요소로 분리한 것은 성숙한 단계의 법학에 정확히 부합하는 것이고,
개념을 좀 무리하게 잡아늘이면 모든 시대 모든 종류의 법과
형식적으로 부합하도록 만들 수도 있을 것이다.
하지만, 오늘날에도 일반인들이 가지는 법관념이
이러한 분석과 완전히 일치한다고 주장할 수는 없다.
또한 원시적 사상의 역사를 파고들면 들수록, 이상하게도 우리는
벤담이 말한 요소들의 결합을 닮은 법의 관념으로부터 점점 멀어짐을 발견하게 된다.
확실히 인류의 유년기에는 어떠한 입법도, 어떤 뚜렷한 입법자도 생각될 수 없었다.
법은 관습의 언저리에도 도달하기 어려웠다.
법은 오히려 습관이었다.
프랑스식 표현으로 법은 ``대기 중에 퍼져 있었다''\latin{in the air}.
옳고 그름의 유일한 권위적 진술은 사건이 일어난 뒤에 내려지는 판결이었다.
위반된 법을 전제하여 내려지는 판결이 아니라,
재판의 순간에 저 위의 권력이 판사의 마음에
처음 영감을 불어넣어 내려지는 판결이었다.
물론 우리는 우리와 시간적으로 관념적으로 멀리 떨어진 사고방식을
이해하기가 무척 어렵다.
그러나 고대사회의 헌정을 더 장기간 천착하고 나면 그것은 더 설득력있게
다가올 것이다.
고대사회에서는 모든 사람이
생애 대부분을 가부장의 전제\hanja{專制} 아래서 살았으므로
그의 모든 행위는 사실상 법이 아닌 변덕에 의해 통제되었던 것이다.
생각건대 다른 나라 사람보다 영국인은
``테미스테스''가
어떤 다른 법 관념보다
선행한다는 역사적 사실을 더 쉽게 이해할 수 있을 것이다.
왜냐하면 영국법의 성격에 관한 다양한 이론들 중에서
가장 유명한, 적어도 실무에 가장 영향력 있는, 이론은 분명
판결과 선례가 규칙이나 원리나 개념구분에 선행한다는 이론이기 때문이다.
주목할 점은,
벤담 및 오스틴의 견해에서
법이 단일한 또는 단순한 명령과 구별되었듯이,
``\wi{테미스테스}''에서도 양자가 구별된다는 것이다.
진정한 법은 유사한 종류의 행위를 모든 시민에게 똑같이 명한다.
이것이야말로 대중들의 마음에 깊이 각인된 법의 성질이며,
``법''\latin{law}이라는 말이 단순히 불변성, 연속성, 유사성에도 사용되고 있는
이유이다.\footnote{%
  이런 맥락의 `법'을 우리말로는 보통 `법칙'이라고 부른다. 중력의 법칙 등. }
이에 비해 \hemph{명령}은 하나의 행위만 지시하며,
따라서 ``테미스테스''는 법보다는 명령에 더 가깝다.
그것은 따로 떨어진 하나의 사실관계에 대한 재판일 뿐이며,
전후의 판결들 간에 규칙적인 연계가 반드시 존재하지는 않는다.

\para{귀족정 시기}
영웅시대의 문헌은 ``테미스테스''와 이보다 좀 더 발달된 ``디케''라는 말로써
맹아기의 법을 우리에게 드러내보인다.
법의 역사의 다음 단계는 무척 흥미로운 시기이다.
그로트 씨의 <<역사>> 제2부 제9장은 호메로스가 묘사했던 것과는
사뭇 다른 성격의 사회가 등장하는 과정을 잘 기술하고 있다.\footnote{%
  \latinmarks
  George Grote,
  \textit{History of Greece},
  Vol.\,3,
  Boston: John P. Jewett and Company,
  1852. }
영웅시대의 왕의 권위는 부분적으로는 신에게서 부여받은 대권에,
또 부분적으로는 탁월한 힘과 용기와 지혜를 가진 데 의존했다.
점차, 왕의 신성함에 대한 관념이 약해지고 또
일련의 세습 과정에서 허약한 왕들이 배출됨에 따라
왕의 권력은 쇠퇴했고, 마침내는 귀족정으로 대체되었다.
혁명에 관한 정확한 용어를 사용할 수 있다면,
\wi{호메로스}가 여러 번 언급했던 족장들의 위원회\latin{council of chiefs}에 의해
왕의 자리가 찬탈당했다고 말할 수 있을 것이다.
여하튼 이제 유럽 각지에서 왕정 시대가 가고 과두정의 시대가 도래했다.
왕이라는 직함이 완전히 없어지지 않은 곳에서도 왕의 권위는 그저
이름에 불과했다.
라케다이몬에서처럼 그저 세습장군이거나,
아테네의 \wi{아르콘} 왕처럼 그저 관리이거나,
로마의 \wi{제사왕}\hanjalatin{祭祀王}{rex sacrificulus}처럼
그저 사제\hanja{司祭}에 불과했다.
그리스, 이탈리아, 소아시아에서
지배집단은 어디서나
가상의 혈연관계로 결합된 다수의 가\hanja{家}로
구성되었다.
애초에 그들은 모두 일종의 신성성을 주장했으나,
그들이 힘이
자칭의 신성성에 기반했던 것 같지는 않다.
민중파에 의해 일찍이 전복되어버린 경우가 있었거니와,
그렇지 않은 경우 결국 그들 모두는
오늘날 우리가 정치적 귀족이라고 부르는 것에 아주 근접해갔다.
이탈리아와 그리스 세계의 이러한 혁명에 비해,
더 먼 아시아 지역의 공동체에서의 사회 변화는
물론 시간적으로 훨씬 더 전에 일어났다.
하지만 문명화과정에서 이들 변화의 상대적 위치는 동일했고,
변화의 일반적 성격도 대단히 유사했던 것 같다.
나중에 페르시아 군주정 아래 통합되는 제 민족들이,
그리고 인도 반도 곳곳에 살았던 제 민족들이,
모두 영웅시대와 귀족정시대를 거쳤다는 여러 증거가 있다.
하지만 여기서는 군사적 귀족과 종교적 귀족이 각각 따로 성장했고,
왕의 권위도 대체로 폐기되지 않았다.
또한 서구의 역사 전개와 달리, 동양에서는
종교적 요소가 군사적^^b7정치적 요소를 압도하는 경향이 있었다.
왕과 사제집단의 틈바구니에서 군사적^^b7세속적 귀족은 보잘 것 없이
절멸당하고 파괴당하여 사라진다.
그리하여 도달한 최종 결과는
왕이 커다란 권력을, 그러나 사제계급의 특권에 의해 제한되는 권력을,
누리게 되는 것이다.
동양의 종교적 귀족과 서양의 세속적^^b7정치적 귀족이라는
이러한 차이에도 불구하고,
영웅적 왕의 시대에 이어 귀족정 시대가 도래한다는 역사적 명제는
참이라 간주해도 좋을 것이다.
전 인류에 타당할지는 모르겠으나, 적어도 인도^^b7유럽 계통 민족들에게는
두루 타당한 것이다.

\para{관습법}
법학자들이 주목할 점은
어디서나 이들 귀족이 법의 저장소이고 법의 집행자였다는 것이다.
그들은 이제 왕의 대권을 계승한 것으로 보인다.
그런데 중요한 차이가 있거니와,
그들은 매번 판결마다 직접 신의 영감을 받는다고 내세우지 않았다.
가부장적 족장의 판결이 초인간적 지시에 연결된다는 관념은
법규칙의 전부 또는 일부가 신에게서 기원한다는 주장을 통해 여기저기서 여전히
나타나고 있지만,
사고의 발달로 이제 더는 구체적인 분쟁의 해결을
인간 외적인 힘의 개입을 가지고 설명할 수 없게 되었다.
법적 과두정이 주장하는 바는 이제 법\hemph{지식}의 독점, 즉
분쟁을 해결하기 위한 법원칙을 그들만이 가진다는 것이다.
실로 우리는 \wi{관습법}\latin{customary law}의 시대에 들어선 것이다.
이제 관습이나 관례는 실체적 규칙의 집합으로 존재하고,
귀족 집단 혹은 귀족 카스트가 그것을 정확히 알고 있다고 간주된다.
옛 전거들에 따르면 과두정에 주어진 이러한 신뢰가
때로 남용되기도 했음이 분명하지만,
이를 단순한 찬탈이나 폭정의 장치로만 보아서는 안 될 것이다.
문자의 발명 이전에는, 그리고 기술이 유년기에 머물던 시절에는,
법적 특권을 가진 귀족들이야말로 민족 혹은 부족의 관습을
거의 정확하게 보존하는 유일한 현실적 방법을 구성했다.
공동체의 일부 구성원의 기억에 관습을 맡김으로써
관습의 진정성은 최대한 담보될 수 있었다.

관습법의 시대, 그리고 특권 계급에 의한 관습법의 보존은
자못 흥미를 불러 일으킨다.
당시의 법 상태는 오늘날의 법률용어나 일상용어에도 그 흔적을 남기고 있다.
그리하여
카스트이든, 귀족이든, 사제 지파든, 신관단\latin{sacerdotal college}이든,
특권을 가진 소수만이 알고 있는 법은 진정한 불문법이다.
이것을 제외하면 세상에는 불문법이 존재하지 않는다.
영국 판례법이 흔히 불문법이라 불리고 있고, 또 어떤 영국 학자들은
영국법을 법전으로 편찬하면 불문법이 성문법으로
대체---그들이 비판적인 취지에서 그러나 사뭇 진지하게 사용하는 용어로는,
개종---될 것이라고 주장한다.
물론 영국 보통법을 마땅히 불문법이라고 칭해도 좋을 시기가 한때 있었음이
분명하다.
영국의 옛 판사들은 변호사나 일반인은 온전히 알 수 없는
규칙, 원리, 개념구분 등을 알고 있다고 내세웠다.
그들이 독점한다고 주장한 법의 전부가 진정 불문법이었는지는 무척 의문스럽다.
하지만, 어쨌든 판사들에게만 알려진 민사 및 형사 규칙들이 한때 상당히 있었다고
가정하더라도, 오늘날에는 그것은 더 이상 불문법이 아니다.
웨스트민스터 홀의 법원들이
연감\latin{yearbook} 등에 기록된 선례에 따라 판결을 내리기 시작하면서,
그들의 법은 성문법이 되었다.\footnote{메인의 이러한 성문법 개념은
  오늘날 통용되는 개념과 다르다는 데 주의할 것. 우리는 판례법, 관습법,
  조리법 등을 모두 불문법으로 분류한다.
  메인이 연감에 기록된 옛 보통법 판례의 성문법성을 주장하는 것은
  이를 일종의 `고대법전'으로 간주하기 위해서인 듯하다.}
오늘날 영국의 법규칙은 우선 인쇄된 선례의 사실관계로부터 분리되고,
특정 판사의 성향, 꼼꼼함, 지식에 따라 어떤 언어의 형식으로 만들어진 후,
해당 사건의 사실관계에 적용되는 것이다.
그러나 이 과정의 어느 단계에서도 성문법과 구별되는 성질은 나타나지 않는다.
그것은 성문의 판례법인 것이다.
법전법과 다른 점은 단지 쓰여진 방식이 다르다는 것뿐이다.

\para{12표법}
관습법의 시대로부터 이제 우리는 법제사에 뚜렷이 획을 긋는 다른
시대로 진입하게 된다.
그것은 \index{법전 시대}법전\latin{code} 시대로,
로마의 \wi{12표법}으로 대표되는 고대 법전의 시대다.
그리스에서, 이탈리아에서, 그리스화된 서아시아 해안 지역에서,
이들 법전은 모두 어디서나 동일한 시기에 등장했다.
여기서 동일한 시기란
시간적으로 동시라는 뜻이 아니라,
각 공동체의 상대적 진화 단계에서 유사한 시기를 점한다는 뜻이다.
내가 언급한 지역 어디서나 법은 판자\latin{tablets}에 새겨져 대중에게 공표되었고,
그리하여 특권 귀족의 기억 속에 저장된 관행들을 대체했다.
오늘날의 법전편찬이라는 것에 가까운 어떤 세련된 숙려가
내가 말한 변화에 조금이라도 들어있었다고 생각해서는 안 된다.
고대 법전은 애초에 문자 기술의 발견과 확산에 의해 도입된 것이 분명하다.
물론 귀족들이 법지식의 독점을 남용했음에 틀림없고,
어쨌든 그들에 의한 배타적 법 전유\hanja{專有}가 서구에서 보편적으로 등장하기 시작한
민중 운동의 성공에 커다란 장애가 되었던 것은 사실이다.
하지만, 비록 민주적 감정이 법전의 확산에 도움을 주었을지라도,
대체로 법전은 문자 발명의 직접적 산물이었음이 확실하다.
일군의 사람들의 기억이
비록 반복적 사용에 의해 강화된다 할지라도,
그러한 기억보다는
글자가 새겨진 판자가 법의 저장소로서 더 훌륭했고,
법의 정확한 보존을 더 잘 담보했다.

로마의 법전은 내가 묘사한 그러한 유형의 법전에 속한다.
그것의 가치는 조화로운 분류라든가 표현의 간결성과 명확성 따위에
있는 것이 아니라, 그 공개성, 즉 무엇을 하고 무엇을 하지 말아야 할 지에 관한
지식을 모든 사람들에게 제공하는 데 있었다.
물론 로마의 \wi{12표법}은 어느 정도 체계성을 보여주긴 하지만,
아마도 이는 후기 그리스의 발달된 입법기술을 갖춘 그리스인들의 도움을 받아
12표법이 기초되었다는 전승\hanja{傳承}에 의해 설명할 수 있을 것이다.
하지만 아테네의 솔론 법전의 남아있는 단편들은
체계가 별로 없었음을 보여주며, 아마도 드라콘의 입법은 더욱 그러했을 것이다.
또한 동^^b7서양을 막론하고 이들 법전의 유물들은
종교적, 시민적, 그리고 단순한 도덕적 명령들이
그 성질의 차이를 고려하지 않은 채 무질서하게 혼재되어 있었음을 보여준다.
이는 법 외의 다른 분야의 초기 사상에 관해 우리가 알고 있는 것과 일치한다.
법과 도덕의 분리, 법과 종교의 분리는 정신의 진화에서
분명히 더 후대의 단계에 속하는 것이다.

\para{마누법전}
그러나, 현대인의 눈에 이들 법전이 아무리 이상하게 보일지라도,
고대사회에서 이 법전들의 중요성은 이루 다 말할 수 없을 정도이다.
문제는---이는 각 공동체의 장래에 큰 영향을 미치게 되는 것인데---도대체
법전이 있어야 하는가 아닌가가 아니었다.
대부분의 고대사회는 어쨌거나 조만간 법전을 가지게 되기 때문이거니와,
봉건제에 의해 만들어진 법제사의 큰 단절이 없었다면
모든 근대법은 이들 원천 중 하나 이상으로
기원을 소급할 수 있었을지도 모른다.
오히려 인류 역사의 전기\hanja{轉機}는
사회 진화의 어느 시기, 어느 단계에서 그들의 법이 성문화되었는가와 관련된다.
서양에서는 각 나라의 평민적^^b7민중적 요소가 과두제의 독점을 성공적으로
공격했고, 국가 역사의 비교적 초기에 거의 보편적으로 법전을 획득했다.
하지만 전술했듯이 동양에서는 군사적^^b7정치적 귀족이 아니라
종교적 귀족이 지배 귀족이 되어 권력을 장악하는 경향이 있었다.
그런데 몇몇 경우 서구에 비해 아시아 나라들은 그 물리적 조건으로 인해
개별 공동체가 더 커지고 인구도 더 많아지는 경향이 있었다.
그리고 어떤 제도가 적용되는 공간이 크면 클수록
그 제도의 완고함과 생명력이 더 커진다는 것은 널리 알려진 사회법칙이다.
원인이야 어찌되었든, 동양사회의 법전은 서구에 비해
상대적으로 훨씬 늦게 획득되고, 그리하여 사뭇 다른 성격을 띠게 된다.
아시아의 종교적 귀족들은 스스로 참고하기 위해서든, 기억의 괴로움을
덜기 위해서든, 후계자의 교육을 위해서든, 어쨌거나
그들의 법지식을 종국에는 법전의 형태로 구체화하기에 이른다.
그러나 자신들의 영향력을 확대하고 공고히하려는 유혹이 너무나 강해서
이에 저항하기 어려웠을 것이다. 즉,
법지식을 완전히 독점하고 있었기에 그들은
법전화를 되도록 미룰 수 있었을 것이다.
그들의 법전은 실제로 행해지는 규칙이 아니라,
준수하는 것이 마땅하다고 사제집단이 생각한 규칙들을 모은 것이다.
\wi{마누법전}\latin{Laws of Manu}이라 불리는 힌두법전은 브라만들이 집성한 것으로,
물론 인도인들이 실제로 준수한 것들을 다수 간직하고는 있지만,
오늘날 최고 가는 동양학자들의 견해에 따르면
전체적으로 그것은 인도에서 실제로 행해지던 규칙들의 집합이 아니다.
대체로 그것은 브라만들이 보기에 법\hemph{이어야 할} 것들을
이상적으로 그려놓은 것이다.
인간의 본성을 감안할 때, 그리고 그 저자들의 특별한 동기를 감안할 때,
마누법전 같은 것이 아주 오래 전의 것인양 내세워지고
그 완전한 형태로 신에게서 유래한 것이라 주장되는 것은 당연한 일에 속한다.
힌두 신화에 따르면 마누는 최고 신의 화신이다.
하지만 그의 이름이 붙어있는 법전은, 비록 정확한 연대는 알 수 없지만,
힌두법의 진화 과정 중에 상대적으로 최근의 산물이다.

\para{타락}
\wi{12표법} 등의 법전이 그것을 획득한 사회에 가져다준 주요 이점은
특권 귀족들의 기만적 행태에 대한 보호막을,
그리고 국가 제도의 자연적 타락에 대한 보호막을 제공한 것이었다.
로마의 법전은 단순히 로마 인민의 기존 관습을 언어로 선언한 것이었다.
그것은 로마의 문명화 과정에서 상대적으로 무척 이른 시기에 법전화된 것이었고,
시민적 책무와 종교적 의무가 착종되어 있던 지적 상태를 아직
로마 사회가 거의 벗어나지 못했을 때에 공표된 것이었다.
그런데 이와 달리 여전히 관습을 준행하는 미개한 사회는
문명의 진보에 전적으로 치명적일 수 있는 어떤 특별한 위험에 노출된다.
공동체가 그 유년기에, 원시적 단계에 채택한 관행들은
대체로 그 물질적^^b7정신적 복리의 증진에 가장 적합한 경우가 일반적이다.
새로운 사회적 필요가 새로운 관행을 낳을 때까지 그것들이 순수하게 보존된다면
사회의 상승적 행진은 거의 확실해진다.
하지만 불행하게도 불문\hanja{不文}의 관행에 기초한 작동에는 그것에 위협이 되는
어떤 발전 법칙이 존재한다.
관습을 준수하는 대중들은 그 유용성의 진정한 근거를 알지 못한 채
당연한 듯 관습을 준수하거니와,
따라서 그들은 불가피하게 준수의 미신적 근거를 발명해낸다.
그리하여 합리적인 관행이 비합리적인 관행을 낳는다는 표현으로
간단히 묘사될 만한 어떤 과정이 시작된다.
유추\hanja{類推}는 성숙기 법학에서는 무엇보다 유용한 도구이지만,
유년기에는 무엇보다 위험한 덫이 된다.
어떤 합당한 이유로 애초에 특정한 하나의 행위에만 국한되던 명령과 금지가
동일한 유\hanja{類}의 다른 모든 행위들에도 적용되기 시작한다.
하나의 행위가 야기하는 신의 분노에 두려움을 느낀 인간은
그것과 조금밖에 비슷하지 않은 다른 행위에 관해서도
자연스레 공포를 느끼기 때문이다.
위생상의 이유로 어떤 음식이 금지되면,
그럴듯한 유추에 때로 의존하여
그 금지는 유사한 다른 모든 음식에도 확장된다.
또한, 일반적 청결을 보증하는 현명한 규칙 하나가 이윽고
판에 박힌 의례적 세정\hanja{洗淨}행위의 기나긴 목록을 명령하게 된다.
또한, 역사 과정의 특정한 위기 시에 국가의 존립을 위해 잠시 필요했던
계급의 구분이 인류의 제도 중에 가장 재앙적이고 파멸적인 것---카스트---으로
타락한다.
힌두법의 운명은 실로 로마 법전의 가치를 보여주는 척도다.
민족학은 로마인과 인도인이 원래 동일한 계통에서 발원했음을 알려준다.
사실 그들의 최초의 관습으로 여겨지는 것들 간에는
대단히 큰 유사성이 있다.
오늘날에도 힌두법의 밑바탕에는 선견지명과 건전한 판단이 깔려있다.
하지만 비합리적인 모방으로 인해 잔인하고 부조리한 거대한 장치가
힌두법에 접목되었다.
로마인들은 그들의 법전에 의해 이러한 타락으로부터 보호될 수 있었다.
그것은 그들의 관행이 아직 건강했을 때 편찬되었거니와,
만약 백년 후였다면 너무 늦었을지도 모른다.
힌두법은 그 대부분이 성문화되었다.
그러나,
산스크리트어로 전해지는 집성들은 일응 오래된 것이긴 하지만,
해악이 작용한 연후에 작성되었다는 풍부한 증거를 담고 있다.
만약 12표법이 공표되지 않았다면 로마인들도 인도인들처럼
허약하고 타락한 문명으로 전락할 운명이었을지에 관해
물론 우리는 아무 것도 말할 수 없다.
하지만 한 가지 확실한 점은 그들의 법전과 \hemph{더불어}
로마인들은 저 불행한 운명으로부터 벗어날 수 있었다는 것이다.



\chapter{법적의제}

원시법이 법전에 구체화되면서, 자생적 발달이라고 할 만한 것은
종말을 맞았다.
이후로는 법 내부에 변화가 일어난다면 그것은 의도적으로 일어난,
그리고 외부로부터 영향받은 변화인 것이다.
어떤 민족이나 부족의 관습이
가부장적 왕에 의해 선언된 이후 마침내
성문화되어 공표되기까지 그 긴 시간---몇몇 경우에는 장구한 기간---동안
전혀 변함 없이 유지된다는 것은 상상할 수 없는 일이다.
또한 그 변화의 어떤 부분도 의도적으로 일어난 부분이 전혀 없다고
단정하는 것도 옳지만은 않을 것이다.
그러나,
이 기간의 법발달에 대해 우리가 아는 바가 별로 없긴 하지만,
변화를 가져옴에 있어 미리 계획된 목적이 차지하는 몫은
극히 작았을 것이라고 가정해도 무리가 없다.
초창기의 관행에 일어난 그러한 혁신은,
오늘날 우리의 정신 조건을 가지고는 도저히 이해할 수 없는 감정과 사고양식에 의해
주어졌던 것 같다.
하지만 \index{법전 시대}법전과 더불어 새로운 시대가 시작된다.
법전 시대 이래, 법변동의 경로 어디를 추적하더라도
그것이 의식적인 개선 노력에 기인한다는 것을,
적어도 원시 시대에 목표했던 것과는 다른 목표를 달성하려는 노력에
기인한다는 것을 발견할 수 있다.

\para{진보의 희귀성}
언뜻 보면, 법전 시대 이후의 법의 역사에서 어떤 믿을 만한 명제를
이끌어내는 것은 불가능해보인다.
대상 영역이 너무 넓다.
충분히 많은 수의 현상을 관찰했는가,
또 관찰한 것을 정확하게 이해했는가, 따위에 대해 우리는 확신을 가질 수 없다.
그러나,
\index{정체된 사회|see{진보하는 사회}}정체된 사회\latin{stationary society}와
\wi{진보하는 사회}\latin{progressive society} 간의 구별이
법전 시대 이후
나타나기 시작했음을
감안하면, 우리의 과업이 불가능해 보이지는 않는다.
우리의 관심대상은 진보하는 사회에 국한되거니와,
그것들의 숫자가 무척 적다는 점이 무엇보다 두드러진다.
압도적인 증거에도 불구하고, 서유럽 시민의 한 사람으로서
그를 둘러싸고 있는 문명이 세계 역사에서 희귀한 예외에 불과하다는 사실을
완전히 체감하기란 결코 쉬운 일이 아니다.
전체 인류에 대한 진보적 민족의 관계를 또렷이 직시한다면
우리가 공유하는 사상의 풍조, 우리들의 모든 희망, 두려움, 생각이
크게 바뀔 수 있을 것이다.
의심할 여지 없이, 인류의 대부분은
문명제도들을 항구적 기록으로 구체화하여 외면적 완성을 이룩한 순간 이후로
그 문명제도들을 개선하려는 일말의 욕구조차
보여준 적이 없었다.
때로 어떤 관행이 폭력적으로 전복되어 다른 관행에 자리는 내주는 경우는 있었다.
곳에 따라 원시 법전은,
초자연적 기원을 내세우며 대폭 확대되기도 했고,
종교적 주석가들의 왜곡을 거치며 놀랄 만한 형태로 뒤틀려지기도 했다.
하지만 이 세상의 아주 작은 한 지역을 제외하면
법체계의 지속적^^b7점진적 개량 같은 것은 찾아볼 수 없었다.
물질문명은 있었지만, 문명이 법을 확장시키기보다는
법이 문명발달의 족쇄로 작용했다.
인류의 원시적 상태를 연구함으로써 우리는
어떤 문명이 그 발달을 멈춘 지점에 관하여 단서를 얻을 수 있을 것이다.
브라만 지배의 인도는 모든 인간사회가 경험한 단계, 즉
법규칙과 종교규칙이 아직 구별되지 않던 단계를 넘어서지 못했음을 알 수 있다.
그런 사회의 구성원들은 종교적 명령의 위반을 세속적 형벌로 처벌해야 한다고
믿었고, 세속적 의무의 위반을 신의 교정\hanja{矯正}에 맡겨야 한다고 믿었다.
중국은 이 지점을 넘어서긴 했으나,
진보는 거기서 정체되었으니,
민사법이 중국인들의 관념의 한계에 갇혀있었기 때문이다.
하지만 정체된 사회와 진보하는 사회의 차이는
커다란 비밀에 싸여있고 우리는 그 비밀을 여전히 탐구해야 한다.
지난 장의 끝부분에서 나는 이 비밀에 대한 부분적 설명을 시도한 바 있다.
덧붙여 유의할 점은 인류에게 있어 정체된 사회가 일반원칙이고
진보하는 사회가 예외임을 정확히 인식하지 않으면 우리의 탐구는
성공할 수 없으리라는 것이다.
그리고 또 하나의 성공조건은 모든 주요 단계마다 로마법에 대한 정확한 지식이
불가결 요구된다는 것이다.
우리가 알고 있는 모든 인간제도들 중에서 로마법은 가장 긴 역사를 가지고 있다.
로마법이 경험한 모든 변화의 성격을 우리는 비교적 잘 알고 있다.
시작부터 종말에 이르기까지 그것은 보다 나은 방향으로,
혹은 변화의 설계자들이 보기에 더 낫다고 여겨졌던 방향으로,
변화하며 진보했다.
로마법이 개선되어 나가는 동안,
인류의 나머지 부분들은 사상과 행위의 진전이 눈에 띄게 느려졌고,
정체상태로 빠질 위험에 끊임없이 노출되었다.

\para{진보적인 법}
이하에서 나는 진보하는 사회에 국한하여 논의를 전개하겠다.
이들 사회에서는 사회의 필요와 사회의 여론이 대체로 법에 선행한다고
말할 수 있다.
그것들 간의 간격이 끊임없이 메워지는 경향을 보이지만,
그 간격은 항상 또 다시 되살아난다.
법은 안정을 추구하지만, 지금 우리가 말하고 있는 사회는 진보하는 사회인 것이다.
인민의 행복은 이 틈새가 얼마나 신속하게 좁혀지는가에 달려있다.

법이 사회와 조화되도록 하는 장치에 관하여 유용한 명제 하나를
개진하고자 한다.
이러한 장치에는 세 가지가 있거니와,
\wi{법적의제}\latin{legal fictions}, \wi{형평법}\latin{equity},
그리고 \wi{입법}\latin{legislation}이 그것이다.
이들 간의 역사적 순서는 내가 제시한 대로이다.
때로는 이들 중 두 가지가 동시에 작용하기도 하고, 또
이들 중 하나의 영향을 받지 않은 법체계도 존재한다.
그러나 내가 아는 한 이들의 등장 순서가 뒤바뀐 사례는 존재하지 않는다.
이들 중 하나인 형평법은 그 초기 역사가 어디서나 모호했고,
따라서 어떤 이는 시민법을 개혁하는 단발적인 법률들이 형평에 의한 재판보다
더 오래됐다고 생각할지도 모른다.
나는 형평법에 의한 구제가 입법에 의한 구제보다 어디서나 더 먼저였다고 믿는다.
그러나 만약 이것이 완전히 진리가 아니라면,
그들의 순서에 관한 명제를,
초창기 법의 변화에 그들이 지속적이고 실질적인
영향력을 행사한 기간에
국한해야 할 필요성이 있을 수는 있다.

\para{의제의 용도}
나는 ``의제''라는 단어를 영국 법률가들이 익히 사용하고 있는 것보다
훨씬 더 넓은 의미에서 사용한다. 또한 로마인들이 ``의제''\latin{fictiones}라는
말에 부여했던 것보다 훨씬 더 포괄적인 의미로 사용한다.
고대 로마법에서 의제\latin{fictio}는 기실 소송변론상의 용어였으니,
원고 측의 거짓 진술로서 피고가 이를 부인하는 것이 허용되지 않는 것을 뜻한다.
가령 원고가 사실은 외인\hanja{外人}이면서 로마시민이라고 진술하는 것이 그 예다.
이러한 로마법상 ``의제''의 목적은 말할 것도 없이 재판권을 부여하기 위한
것이었다.
\hypertarget{commonlawfiction}{따라서} 이는 영국의 왕좌법원\hanjalatin{王座法院}{Queen's Bench}이나
재무법원\latin{Exchequer}의 영장\latin{writ}에 담긴
진술---피고가 국왕의 감옥에 구금되어 있다는 진술, 혹은
원고가 왕의 채무자인데 피고의 채무불이행으로 인해
자신의 채무를 이행할 수 없다는
진술---과 무척 흡사하거니와,
이로써 이들 법원은 민소법원\hanjalatin{民訴法院}{Common Pleas}의 재판권을 빼앗아올 수 있는
것이다.
그러나 내가 사용하는 ``\wi{법적의제}''라는 표현은
법규칙의 문언은 그대로인 채 그 실제적 작용이 바뀐 변화의 사실을
숨기거나 숨기는 데 영향을 주는 일체의 가정\hanja{假定}을 총칭한다.
따라서, 앞서 인용한 로마법과 영국법의 의제 사례들뿐만 아니라
그 이외의 것도 여기에 포함된다. 나는 영국의 판례법과 로마의
\wi{법학자의 해답}\latin{responsa prudentium}도 의제에 기초한 것으로 보기 때문이다.
이들 두 가지에 대해서는 조금 있다 설명할 것이다.
이들 두 경우, \hemph{사실}로는 법이 완전히 변화했으나,
\hemph{의제}적으로는 법이 예전 그대로 동일한 것이다.
모든 형태의 의제가 왜 사회의 유년기와 특히 친화성이 있는지는 어렵지 않게
이해할 수 있다.
의제는 가끔 등장하는 개선 욕구를 충족시키면서도
변화에 대한 상존하는 미신적 거부감을 거스르지 않기 때문이다.
사회진화의 특정 단계에서 의제는 법의 엄격함을 극복하는 유용한 수단이 된다.
실로 그중 하나인, 인위적인 가족관계 형성을 가능케 하는
\wi{입양}\hanja{入養}이라는 의제가 없었다면, 어떻게 사회가 그 요람기를 벗어나
문명을 향한 첫걸음을 뗄 수 있었을지 상상하기 어렵다.
그러므로 우리는 \wi{벤담}이 법적의제에 대해 퍼부은 조롱과 비난에 마음상할
필요가 없다.
의제를 속임수에 불과하다고 욕하는 것은 법의 역사적 발달에서
의제가 수행한 특수한 역할에 대한 무지를 드러낼 뿐이다.
하지만 동시에, 의제의 유용성을 인식하면서 우리 법체계에 의제가
확고하게 뿌리내려야 한다고 주장하는 일부 논자들에게 동조하는 것도 똑같이
어리석은 일이 될 것이다.
영국 법률가들의 관념에 심각한 충격을 주지 않고
그들의 언어에 중대한 변화를 초래하지 않는 한
내다버릴 수 없는 몇몇 의제들이 여전히 강력한 영향력을 영국법에 행사하고 있다.
하지만 법적의제와 같은 거친 장치로써 어떤 유익한 결과를 도모하는 것이
우리에게는 어울리지 않는다는 것도 틀림없이 일반적 진리일 것이다.
법을 더 이해하기 어렵게 만들거나 조화로운 질서의 형성을 더 어렵게 만드는
어떠한 변칙도 무고하지 않다고 나는 생각한다.
그런데 여러 장애 중에서도 법적의제야말로 체계적인 분류에 가장 큰
장애가 된다.
저 법규칙은 여전히 법체계에 들러붙어 있으나,
그것은 껍질에 불과하다.
저 규칙은 이미 오래 전에 쇠퇴했고, 새로운 규칙이 껍질 아래 몸을 숨기고 있다.
그리하여 실제 작동하는 규칙을 그 진정한 장소에 분류해야 할지,
아니면 그 외관상의 장소에 분류해야 할지 알기 어려운 상황이 발생하거니와,
어느 선택지를 택할 지를 두고 여러 부류의 학자들 간에 의견이
갈라지는 일이 발생할 것이다.
영국법이 질서있는 분류를 채택하려 한다면,
최근의 몇몇 입법적 개선에도 불구하고 여전히 영국법에 널리 퍼져있는
법적의제들을 뿌리뽑지 않으면 안 될 것이다.

\para{형평법}
사회적 필요에 법이 적응하는 또 다른 수단은 내가 형평법이라 부르는 것이다.
여기서 \wi{형평법}이란 초창기 시민법에 병존하는 법체계로서
독자적인 원리에 기초하고 있고 그 원리에 내재한 우월한 신성함에 기대어
시민법을 넘어선다고 주장되는 것을 말한다.
로마 \wi{법무관}\latin{praetor}들의 형평법이든,
영국 챈슬러\latin{chancellor}들의 형평법이든,
형평법은
개방적이고 공공연하게 기존 법에 간섭한다는 점에서
각각의 경우 그것에 선행했던 의제들과 차이가 있다.
한편, 형평법은 법 개선의 동인으로 나중에 등장하는 입법과도 다르다.
형평법의 권위는
법 바깥의 어떤 사람이나 집단의 대권\hanja{大權}이 아니라,
법을 천명하는 정무관의 대권이 아니라,
모든 법이 따라야 한다고 여겨지는 법원리의 특별한 성격에
근거하고 있다는 점에서 그 차이가 있는 것이다.
초창기 법보다 더 높은 신성함을 가지고 있고
외부 기관의 승인과 무관하게 효력을 주장하는
일련의 원리들이라는 이러한 관념은
법적의제가 처음 등장했던 사고 단계보다 더 발달된 단계에 속한다.

\para{입법}
전제군주의 형태로든, 의회의 형태로든,
전체 사회를 대표한다고 간주되는 입법기관의 법제정인 \wi{입법}은
법 개선 수단 중에서 마지막 것이다.
입법과 법적의제의 차이는 형평법과 법적의제의 차이와 동일하다.
입법은
그 권위가 외부의 기구나 사람에게서 나온다는 점에서
형평법과도 구별된다.
입법의 구속력은 그것의 법원리와 무관하다.
현실적으로는 여론에 의한 제약이 있다 하더라도,
이론적으로 입법기관은 스스로가 원하는 바를 공동체 구성원들에게
의무로 부과할 권한을 가진다.
입법기관이 자의적 변덕에서 하는 입법을 막을 것은 아무 것도 없다.
만약 형평이 어떤 선악의 기준을 뜻하는 말로 사용되고
법제정이 어쩌다 이러한 기준에 맞추어 행해진다면,
그러한 입법은 형평에 의해 지시된 것이라 할 수 있을 것이다.
하지만 이런 경우에도 법제정의 구속력은 입법기관의 권위에 빚지고 있는 것이지,
입법기관의 행위 근거가 된 원리의 권위에 빚지고 있는 것이 아니다.
그리하여 입법이 기술적 의미의 용어인 형평법 규칙과 다른 점은,
후자는 최고의 신성함을 내세우며 군주나 의회의 협찬이 없더라도
즉각 법원에 받아들여질 것을 요청한다는 데 있다.
이런 차이에 주목해야 할 더 큰 이유는,
어떤 벤담 학도는 법적의제, 형평법, 제정법을 뭉뚱그려
이 모두를 입법이라는 단일 범주로 포괄하려 할 것이기 때문이다.
이 모두가 \hemph{법창조}\latin{lawmaking}에 관한 것으로,
그것들 간 차이는 단지 새 법이 만들어지는 장치의 차이일 뿐이라고 그는
말할 것이다.
이것은 분명 진실이고 우리는 이것을 망각해서는 안 된다.
하지만 그렇다고 해서 입법과 같은 무척이나 편리한 용어를
특수한 의미로 사용해서는 안 될 이유가 되지는 못한다.
입법과 형평법은 대중의 정신에서, 그리고 대부분의 법률가들의 정신에서,
서로 분리되어 있다.
특히, 중요한 실제적 결과의 차이가 뒤따른다면,
아무리 인습적이라 해도 양자의 차이를 무시하는 것은 결코 정당화될 수 없다.

\para{법적의제}
거의 모든 발달된 법체계에서 \hemph{법적의제}의 사례들을 선별하기란
쉬운 일일 것이며, 그것들은 즉시 법적의제의 진정한 성질을 현대의 관찰자들에게
드러낼 것이다.
하지만 이제부터 내가 다루려는 두 가지 경우에는
거기에 사용된 수단의 본질이 그리 쉽게 드러나지 않는다.
이들 의제의 최초 창시자들은 아마 혁신을 의도하지 않았을 것이며,
혁신의 의심을 사기는 더더욱 바라지 않았을 것이다.
게다가 그러한 혁신의 과정에 의제가 들어있음을 부인하는
사람들이 늘 있고 또 있어왔거니와,
전래의 인습적 언어가 그들의 부인\hanja{否認}을 실증한다.
그러므로 \wi{법적의제}의 광범위한 확산을 보여주는,
그리고 법체계를 변화시키면서도 그 변화를 감추는 이중적 역할의
효율적 수행을 보여주는 사례로서 이보다 더 나은 것들은 없을 것이다.

\para{사법적 입법}
이론적으로는 조금도 기존 법을 바꿀 힘이 없는 장치가
법을 확대하고 수정하고 개선해나가는 것에
우리 영국인들은 아주 익숙하다.
이러한 사실상의 입법이 작동하는 과정은 감지될 수 없는 것이 아니라
인정되지 않을 뿐이다.
판례들에 담겨있고 판결집들에 기록돼있는 우리 법체계의 방대한 부분에 대해
우리는 습관적으로 이중적 언어를 사용하고 이중의 모순적인 관념들을 구사한다.
어떤 사실관계가 영국 법원에 제소되면
판사와 변호사들 간의 모든 논쟁은
옛 법원리 외의 어떤 법원리도,
오래된 개념구분 외의 어떤 개념구분도
적용될 필요가 없고 적용될 수도 없다는
가정 하에 진행된다.
계쟁 분쟁의 사실관계를 포섭하는 기존의 법규칙이 어딘가에 존재하며,
설령 그러한 규칙이 발견되지 않더라도 인내, 지식, 통찰력을 발휘하면
얼마든지 그것을 찾아낼 수 있다는 믿음을 극히 당연한 것으로 받아들인다.
그러나 일단 판결이 내려지고 기록되고 나면, 우리는 무의식적으로 혹은 은밀하게
새로운 언어, 새로운 사고 맥락으로 넘어간다.
이제 우리는 새로운 판결로 법이 수정\hemph{되었다}고 인정한다.
적용가능한 법규칙이, 흔히 쓰이는 부정확한 표현을 사용하자면,
보다 유연해졌다고 믿는다.
실제로 법규칙은 변경되었다.
선례에 새로운 것이 첨가되었고, 선례들을 비교하여 얻어지는 법원리는
일련의 판례들에서 하나의 사례를 제외했을 때 얻어지는 것과는 다른 것이 되었다.
옛 규칙이 폐지되고 새로운 것으로 대체되었다는 사실을 우리는 받아들이기 어려운데,
선례에서 얻어지는 법적 공식을 정확한 언어로 표현하는 습관을 갖고 있지 못하여,
변화의 광채가 강렬하고 눈부신 것이 아닌 한 쉽게 포착하지 못하기 때문이다.
진기한 변종 판결 앞에서 영국 법률가들이 침묵으로 일관하는
이유를 여기서 장황하게 늘어놓을 생각은 없다.
아마도, 구름 속이든\latin{in nubibis} 혹은
판사의 마음 속이든\latin{in gremio magistratuum} 어딘가에
완전하고 일관되고 체계적인 영국법이 존재한다는, 그리하여 상상할 수 있는 어떤 상황에도
적용할 수 있는 풍부한 법원리의 체계가 존재한다는 것이 전래의 교리였다는 것은
발견할 수 있을 것이다.
처음에는 이 이론이 지금보다 훨씬 더 철저히 신봉되었으며,
실제로 그럴 만한 근거가 더 충분했다.
13세기 판사들은 변호사나 일반 대중에게는 알려지지 않은
법의 보고\hanja{寶庫}를 이용할 수 있었으니,
그들은 은밀히 당대의 로마법과 \wi{교회법} 집성들로부터, 항상 현명하게는 아닐지라도,
자유롭게 빌려왔다고 믿을 만한 이유가 있다.
하지만 웨스트민스터 홀의 법원들이 판결을 양산하여 실체법 체계의 토대가 마련되자,
이 저장고는 폐쇄되었다.
그리하여 수 세기 동안 영국의 법률가들은, 형평법과 제정법이 아닌 한 아무 것도
이미 형성된 이 토대에 첨가된 것이 없다는 역설적인 명제를 전승시켜온 것이다.
우리는 우리 법원들이 입법을 한다는 것을 인정하지 않는다.
우리는 우리 법원들이 결코 입법을 한 적이 없다고 생각한다.
그럼에도 불구하고 우리는 영국의 보통법 규칙들이, 형평법법원\latin{Court of Chancery}과
의회로부터 약간의 도움을 받아, 현대사회의 복잡한 이해관계에 충분히 대처할 수 있다고 주장한다.

\para{법학자의 해답}
방금 언급한 특징에 있어서 우리 판례법과 무척 가깝고 교훈적인 유사성을 가진 법체계가
로마에서는 ``법에 식견 있는 자의 답변''이란 뜻의 \wi{법학자의 해답}\latin{responsa prudentium}이었다.
이들 해답은 로마법의 발달 시기에 따라 상당히 다른 형태를 띠었지만,
전 시기에 걸쳐 어떤 권위 있는 성문의 문헌들을 해설하는 주석임에는 변함이 없었고,
처음에는 오로지 12표법에 대한 해석 의견의 모음이었다.
우리와 마찬가지로, 모든 법적 언어는 이 옛 법전의 텍스트가 불변이라는 가정에 기초했다.
거기에 명시적인 규칙이 있었다.
그것은 어떤 주석이나 주해보다 위에 있었고, 어떤 해석도, 설령 위대한 해석자의 것이라 해도,
거룩한 텍스트에 호소하여 수정될 수 있음을 누구도 공공연히 부인할 수 없었다.
하지만 사실 저명한 법학자의 이름을 달고 있는 해답집은
적어도 우리의 판결집에 버금가는 권위를 누렸고,
12표법의 규정을 지속적으로 수정하고 확장하고 제한하고 사실상 뒤엎었다.
새로운 법학의 형성기 동안 법학의 저술가들은 법전의 문구에 꼼꼼한 충실함을 내세웠다.
단지 그것을 설명하고 독해하고 그 의미를 온전히 드러낼 뿐이었다.
그러다 결국 그들은, 텍스트를 이어붙이고,
실제로 발생한 사실관계에 법을 적응시키고,
일어날 법한 사실관계에 법의 적용가능성을 탐구하고,
다른 성문 문헌에서 도출한 해석 원리를 가져오는 등에 의해
12표법의 편찬자들은 꿈도 꾸지 못했던, 실로 12표법에서는 거의 혹은 전혀 찾아볼 수 없는
사뭇 다양한 법원리들을 이끌어냈다.
법학자들의 저술은 모두 법전과 일치한다는 근거에서 존중받을 자격을 주장했으나,
그것의 상대적 권위는 저술을 발표한 특정 법학자의 명성에 크게 좌우되었다.
널리 알려진 위대한 학자의 이름은 입법기관의 법제정에 버금가는 구속력을 해답집에 부여했다.
그리고 그러한 저서가 이번에는 한층 더 나아간 법학 발달의 새로운 토대로 작용했다.
하지만 초기 법학자들의 해답은 오늘날처럼 저자에 의해 출간된 것이 아니었다.
그것은 그의 학생들이 기록하고 편집한 것이어서,
대개는 어떤 체계적인 분류법에 따라 배열된 것이 아니었다.
이렇게 출간에 있어 학생들이 한 역할은 특히 주목할 필요가 있거니와,
그들이 스승에게 행한 봉사는 학생 교육에 대한 스승의 충실성에 의해 보상받는 것이 일반적이기 때문이다.
후대에 가서 이 의무의 결실로 인정받게 되는
\wi{법학제요}\hanjalatin{法學提要}{Institutes}, 즉
주해서\latin{Commentary}라 불리는 교육용 저술들은
로마법 체계의 사뭇 중요한 특징을 이루는 것이다.
법학자들이 대중들에게 개념의 분류와 전문용어의 개선을 제안한 것은
이러한 법학제요 형태의 작품에서였지, 훈련된 법률가들을 겨냥한 저서에서가 아니었다.

로마의 \wi{법학자의 해답}과 그것의 영국적 대응물을 비교할 때,
로마법학의 이 부분이 가지는 권위는 \hemph{판사직}\latin{bench}이 아니라
\hemph{변호사직}\latin{bar}에서 유래한다는 점에 주의해야 한다.
로마에서 법원의 결정은, 개별 사건을 종결짓는 것이었지만,
어떤 장래를 향한 권위도 가지지 못하였고, 다만
해당 사건을 잠시 담당하게 된 정무관의 전문직업적 명성에 의해 주어지는
권위만 누릴 뿐이었다.
사실 공화정기 동안 로마는 영국의 왕좌법원\latin{Bench}이나
신성로마제국의 제실법원\hanjalatin{帝室法院}{Chamber},
프랑스왕국의 파를르망\latin{Parliament} 비슷한 제도를 전혀 알지 못했다.
각자 맡은 분야의 사법적 기능을 그때그때 담당하는 정무관\latin{magistrate}들은 있었지만,
정무관의 임기는 1년에 불과했기에, 그것은 상설 법관이라기보다는
정상급 변호사들이 돌아가면서 잠깐씩 맡는 순환 공직에 가까웠다.\footnote{%
  기실
  법무관(praetor) 등 정무관은 변호사---오늘날의 전문직 법률가로서의
  변호사가 아니라 웅변가(orator)였음---나 법학자일 수도 있지만
  그냥 정치가인 경우도 많았다.}
우리 눈에는 무척 이상하게 보이는 이러한 제도의 기원에 대해 다양한 견해가 있을 수 있지만,
사실 그것은 현대 우리의 제도보다 고대사회의 정신에,
개별 신분집단들로 분열되지만 아무리 배타적이더라도
그들 간에 전문직업적 상하관계는 허용하지 않는 정신에, 더 잘 부합했다.

이 체제는 그로부터 기대할 법한 효과를 가져오지 못했다는 점에 주목할 필요가 있다.
가령 그것은 로마법을 \hemph{대중화}하지 못했다.
비록 법학의 확산과 권위 있는 해설에 인위적인 장벽을 두지는 않았으나,
몇몇 그리스 공화국에서처럼 법학을 습득하는 데 필요한 지적 노력을 완화해주지 못했다.
오히려, 어떤 다른 원인들이 작동하지 않았더라면, 후대의 지배적 법체계들처럼 로마의 법학도
사소한 데 치중하고 기술적이고 배우기 어려운 학문이 되었을 확률이 상당히 컸다.
또한, 훨씬 더 마땅히 발견될 법한 어떤 결과도 전혀 나타나지 않은 듯하다.
로마 공화정이 무너지기 전까지 법학자들은 명확하게 정의되지 않은 집단을 형성하고 있었던 것이다.
또한 그 숫자도 틀림없이 큰 폭으로 오르내렸을 것이다.
그럼에도 불구하고, 주어진 사례에 대해서 어떤 사람의 의견이 그들 세대에서 결정적인 권위를
누렸는지는 의심의 여지가 거의 없었던 것 같다.
여러 라틴어 문헌에 전해지는, 정상급 법학자들의 일상 업무에 관한
생생한 묘사---이른 아침부터
시골에서 올라온 고객들이 그의 대기실에 몰려들고,
공책을 든 학생들은 그의 주변에 둘러서서 위대한
법률가의 답변을 기록한다---는 일정 기간에 국한해 본다면
한 두 명의 저명한 이름을 거의 혹은 전혀 벗어나지 않는다.
또한 고객들과 변호사의 직접적인 접촉 덕분에,
로마 사람들은 전문가들의 명성의 오르내림을 즉각적으로 알고 있었던 듯하다.
저 유명한 키케로의 <<무레나를 위한 변론>>\latin{Pro Muraena}을 비롯한 풍부한 증거가
있거니와, 법정에서의 성공에 대한 일반인들의 존경은 과도하면 과도했지 부족하지 않았다.

의심할 여지 없이,
로마법의 발달을 추동한 수단에 관한 전술한 특징은
그것의 우수성, 즉 일찍부터 법원리가 풍부했던 것의 원천이었다.
법원리의 성장과 풍부함은 부분적으로는 법해설자들 간의 경쟁에 의해
촉진되었으니, 국왕이나 국가에 의해 부여된 사법권의 담지자인
법관직이 존재하는 곳에서는 이러한 것이 작동할 수 없는 것이다.
하지만 주된 동력은 말할 것도 없이 사법판결의 대상이 되는 사례의
무제한적 증가에 있었다.
시골 고객들을 당혹케했던 사실관계들이 법학자의 해답이나 사법판결의
토대가 되었을 뿐만 아니라, 똑똑한 학생들이 제기한 가상의 사례들도
그에 못지 않았다.
실제 사례든 가상의 사례든, 모든 사실관계는 자격에 있어 차이가 없었다.
법학자들로서는 그의 고객의 사건을 재판하는 정무관이 그의 의견을
퇴짜놓는다고 해도 아무 문제가 되지 않았다.
오히려 정무관이 법지식에 있어서나 전문직업적 평판에 있어서 자신보다
위에 있는 것이 문제였다.
그렇다고 해서 법학자들이 고객의 이익에 무심했다는 말은 아니다.
고객들은, 초기에는 저명한 법률가들의 선거인단이었고
후기에는 돈을 벌게 해주는 사람들이었기 때문이다.
그러나 야망을 충족시키는 주된 길은 동료 집단의 평판을 통해서였던 것이다.
전술한 이러한 체제 하에서 평판을 확보하는 좋은 방법은 각 사례를,
법정에서 승리하기 위한 고립된 사건으로 접근하는 것이 아니라,
어떤 포괄적 법원리나 법규칙의 예시의 하나로 바라보는 것이다.
있을 수 있는 사례를 제시하거나 발명해내는 데 아무런 제약이 없었던 것도
큰 영향력을 발휘했을 것임에 틀림없다.
데이터를 마음껏 증가시킬 수 있는 곳에서는
일반적 규칙을 진화시키는 능력이 대폭 증대된다.
우리의 사법체계에서는 판사들이 자기 앞에 놓인,
혹은 그의 전임자들 앞에 놓였던 사실관계를 벗어날 수가 없다.
따라서 재판의 대상이 된 각 사실관계는,
프랑스 식으로 표현하면, 일종의 성별\hanja{聖別}이 이루어진다.
실제 사건이든 가상의 사례든, 다른 모든 사건들과 구별되는 성질을 가지는 것이다.
하지만 로마에서는, 전술한 바에서 짐작할 수 있듯이
판사집단\latin{bench}이나 판사들의 법원\latin{chamber}
같은 것이 전혀 없었고,\footnote{%
  여기서 판사(judge)는
  법무관이 방식서(formula)에서 지시한 바에 따라
  사실관계를 심리하고 판결을 내리는
  심판인(iudex)을 뜻하는 듯하다.
  로마법 발달의 주역은 심판인들이 아니라 법학자들과 법무관들이었다. }
따라서 어떤 사실관계도 다른 사실관계보다 더 특별한 가치를 지니지 않았다.
어떤 어려운 사안이 법학자의 의견을 요청하는 경우,
뛰어나 유추 감각을 지닌 이는 거리낌없이 그것과 어떤 특징을 공유하는
모든 상상할 수 있는 사례들을 즉시 인용하고 고려할 수 있었다.
고객에게 주어진 실무적 조언이 무엇이든 간에,
학생들의 공책에 씌어진 해답\latin{responsum}은 분명
숭고한 법원리로 규율되는, 또는 포괄적인 법규칙에 포섭되는,
그러한 사실관계를 고려하였을 것이다.
우리에게는 이러한 것이 한 번도 가능한 적이 없었다.
그리고 영국법에 가해진 수많은 비판 속에서
영국법이 선언되는 양식에 대한 비판은 잊혀져버린 것 같다고 인정하지 않을 수 없다.
우리 법원이 법원리를 선언하는 데 인색한 것은
우리 판사들의 기질 탓보다는
우리에게 선례가,
다른 법체계들을 알지 못하는 이들에게는 많아 보일지 모르나,
상대적으로 부족한 데 더 큰 원인이 있는 듯하다.
법원리의 풍부함에 있어 여러 근대 유럽대륙의 국가들에 비해
우리가 대단히 빈약한 것이 사실이다.
하지만 그들은 민사법 제도의 기초로 로마법을 채택했음을 기억해야 한다.
그들은 로마법의 파편들을 가지고 그들의 성채를 건설했다.
그러나 그밖의 재료나 솜씨에 있어서는 영국 법원이 건설한 구조보다
우월할 것이 별로 많지 않다.

\para{이후의 로마법}
로마 공화정기는 로마법학에 그 특징이 각인된 시기였다.
로마법학의 초기 동안 법학자의 해답이 법발달의 주역이었다.
그러나 공화정의 몰락이 다가오면서 해답들은 더 이상의 확장을 저해하는
형태를 띠기 시작한 것으로 보인다.
그것들은 이제 체계화되어갔고 단순한 모음집이 되어갔다.
신관\hanjalatin{神官}{pontifex}이었던
무키우스 스카이볼라\latin{Q. Mucius Scaevola}는
시민법 전체의 매뉴얼을 출간했다고 한다.
키케로의 저술들은
능동적인 법 혁신 수단들에 대비되는 낡아빠진 방법들에 대한 염증이
커지고 있었음을 보여준다.\footnote{%
  가령 키케로, <<법률론>>, 1.14. }
사실 이때쯤이면 다른 요인들도 법에 영향을 미치게 된다.
\wi{법무관}이 매년 선포하는 \index{고시|see{법무관}}고시\hanjalatin{告示}{edict}는
이제 법개혁의 주된 동력으로 인정받고 있었다.
코르넬리우스 술라\latin{L. Cornelius Sylla}는
코르넬리우스 법\latin{Leges Corneliae}이라 불리는 일련의 위대한 법률들을
제정함으로써 직접적 \wi{입법}에 의해 얼마나 빨리 개선이 이루어질 수 있는지
잘 보여주었다.
\wi{법학자의 해답}에 최종 일격을 가한 것은 아우구스투스였다.
제출된 사안에 대해 구속력 있는 해답을 줄 수 있는 권리를
몇몇 정상급 법학자들에게만 부여한 것이다.
이 변화는, 비록 근대적 관념에 가까이 다가가는 것이기는 하나,
확실히 법전문직의 성격 및 그것이 로마법에 미친 영향의 성질을
근본적으로 바꾸어놓았다.
법학의 영원하고 위대한 등불이 되는
또 다른 일군의 법학자들이 후대에 등장하지만,
울피아누스, 파울루스, \wi{가이우스}, 파피니아누스는 해답의 저자들이 아니었다.
그들의 저술은 법의 특정한 분야, 특히 법무관의 고시에 대해 쓴
본격적인 전문법학서적이었다.

\para{로마의 제정법}
로마의 형평법 및 이것을 로마법에 만들어넣은 법무관 고시에 대해서는
다음 장에서 살펴볼 것이다.
제정법에 대해서는, 공화정 시기에는 수가 많지 않았으나
제정기에는 양산되었다는 점만 말해두고자 한다.
국가의 청년기나 유년기에는 사법\hanja{私法}의 일반적 개혁에
입법기관이 동원되는 경우가 드물다.
민중의 요구사항은 법을 변화시키는
것---이것은 실제 가치보다 높게 평가받는 경향이 있다---이 아니라
재판이 깨끗하고 완전하고 수월하게 진행되는 데 있었다.
입법기관에 대한 호소는 대체로 어떤 큰 권력남용을 제거해달라든가,
해결하기 어려운 신분 간의 혹은 권문세족 간의 다툼에 대해
결정을 내려달라는 정도에 불과했다.
로마인들은 대규모의 법률 제정과
큰 내란 뒤의 사회 안정 사이에
어떤 연관성이 있다고 생각했던 듯하다.
술라는 코르넬리우스 법들로써 공화국의 재건을 알렸다.
율리우스 카이사르는 방대한 양의 제정법을 추가하려는 계획을 가지고 있었다.
아우구스투스는 율리우스 법\latin{Leges Juliae}이라 불리는
매우 중요한 일군의 법률을 통과시켰다.
후대의 황제들 가운데 가장 적극적으로 \wi{칙법}\latin{constitution}을 공포한 이는,
콘스탄티누스처럼, 세상을 재조정하는 데 관심을 가졌던 황제들이었다.
로마에서 제정법의 진정한 시대는 제정기에 비로소 시작된다.
황제들의 법제정은,
처음에는 민중의 지지에 의해 제정되는 척 치장했으나
나중에는 황제의 대권에서 유래하는 것이라고 공공연히 인정되는데,
아우구스투스의 권력이 공고해진 이후 유스티니아누스 법전의 공표에 이르기까지
점점 더 그 양이 증가하였다.
이미 제2대 황제 치세 때에 오늘날 우리 모두에게 친숙한 법상태 및
법집행 양태와 상당히 비슷해졌다고 할 수 있다.
제정법이 등장하고 한정된 인원의 법해설자단\hanja{團}이 등장했다.
얼마 후에는 상설 상소 법원과 공인된 주해을 모은 주해집이 여기에 추가될 것이다.
그리하여 우리는 오늘날의 관념에 가까이 다가가게 된다.


\chapter{자연법과 형평법}

내재적인 탁월함을 가진
일군의 법원리가
낡은 법을 대체한다는 이론은
로마에서도 영국에서도 아주 일찍부터 통용되었다.
어떤 법체계에서도 발견되는
이러한 원리들을 앞 장에서 우리는 형평법\latin{equity}이라고 불렀다.
곧 살펴보겠지만, 이 용어는
로마 법학자들이 이러한 법변화 작인\hanja{作因}을 지칭하는
여러 명칭 가운데 하나, 오직 하나에 불과했다.
영국에서는 형평법법원\latin{Court of Chancery}의 법이
형평법이라는 이름으로 불리고 있거니와,
이것은 별도의 논저를 통해서만 제대로 논의될 수 있을 것이다.
그것의 구조는 대단히 복잡하고, 여러 다양한 원천에 기원을 두고 있다.
초기 챈슬러들은 성직자들이었기에 그들은 교회법으로부터
형평법의 바탕이 되는 법원리들을 이끌어냈다.
후대의 챈슬러들은
세속 사건에 적용할 수 있는 법규칙이 교회법보다 더 풍부한 로마법을
자주 원용했다.
그들의 판결문 중에는, 비록 출처는 밝히고 있지 않으나
로마법대전에서 따온 텍스트 전체가 토씨 하나 바뀌지 않고
들어가 있는 경우가 적지 않다.
더 최근에는, 특히 18세기 후반에는,
네덜란드 공법학자들의 법학 및 윤리학의 혼합체계가 영국 법률가들에 의해
널리 연구되었거니와,
이들 연구는
탈보 경\latin{Lord Talbot}에서 엘던 경\latin{Lord Eldon}에 이르는
챈슬러들의 형평법법원 판결에 큰 영향을 끼쳤다.
이렇게 다양한 기원의 요소들로 구성된 형평법 체계는,
보통법의 유추적용과 정합성을 가져야 한다는 요청으로 인해
그 성장이 크게 제한되었으나,
상대적으로 새로운 법원리들을 기술하는 일에 언제나 응답해왔다.
그 법원리들은 내재적인 윤리적 탁월함에 있어 영국의 옛 법을 능가한다고
주장되었다.

\para{로마의 형평법}
로마의 형평법은 구조가 훨씬 단순했고, 발달과정도 보다 쉽게 추적할 수 있다.
그것의 성질과 역사는 주의깊게 살펴볼 가치가 있다.
그것은 인간의 사고에 심대한 영향을 끼친 개념들을 창조하였고,
인간의 사고를 통해 인류의 운명에도 심대한 영향을 주었다.

로마인들은 그들의 법체계가 두 부분으로 구성된다고 보았다.
유스티니아누스 황제의 명에 의해 편찬된 법학제요는 이렇게 말한다.
``법과 관습에 의해 규율되는 모든 민족들은, 부분적으로는 그들 자신의
고유한 법에 의해, 부분적으로는 모든 인류에 공통되는 법에 의해,
통치된다. 당해 인민이 제정한 법은 그 민족의 시민법\latin{civil law}이라
불리고, 자연이성\latin{natural reason}이 모든 인류에게 지시한 법은
모든 민족이 사용하기 때문에
만민법\hanjalatin{萬民法}{Law of Nations}이라 불린다.''
여기서 ``자연이성이 모든 인류에게 지시한 법''은 법무관의 고시가
로마법에 엮어넣은 요소를 의미했다.
다른 곳에서는 이것을 단순히 자연법\latin{ius naturale}이라고 불렀는데,
자연법은 자연이성뿐 아니라 자연적 형평\latin{naturalis aequitas}에 의해서도
명령된다고 여겨졌다.
나는 여기서 만민법, 자연법, 형평법이라는 유명한 표현들의 기원을 탐구할 것이고,
또 이들이 지시하는 관념들의 상호 관련성을 탐구할 것이다.

\para{만민법}
로마의 역사를 조금만 살펴보아도,
여러 다른 이름으로 불리우며
로마의 영토 안에
살고 있는 외인\hanja{外人}들의 존재에 의해 공화국의 운명이
좌우되었음을 알고 놀라게 된다.
이러한 이주의 원인은 후대에 분명히 드러나게 되거니와,
왜 모든 민족의 사람들이 세상의 주인인 도시로 몰려드는지를
짐작하기란 그다지 어렵지 않다.
그러나 외인들과 준\hanja{準}시민\latin{denizen}들의 대규모 존재는
로마의 초기 역사에서도 그 기록이 발견된다.\footnote{여기서 `준시민'은
`라틴인'을 뜻하는 듯하다.}
말할 것도 없이,
다수의 약탈적 부족들로 구성된 고대 이탈리아의 사회적 불안정성은
사람들로 하여금, 공동체와 그 구성원들을 외적으로부터 보호할 수 있는
강력한 힘을 가진 공동체의 영토에 몰려가 살도록 하는 유인을 제공했다.
그 보호가 과중한 세금, 선거권 박탈, 사회적 신분 저하의 대가로
주어지는 것이라 할지라도 말이다.
하지만 이러한 설명은 불완전하며,
활발한 상거래 관계를 고려에 넣어야만 완전해질 수 있을 것이다.
이러한 상거래 관계는 공화국의 군사적 전승\hanja{傳承}에는
별로 반영되어 있지 않지만,
분명 로마는 카르타고와, 그리고 이탈리아 안에서,
선사시대부터 상거래 관계를 유지해온 것으로 보인다.
그 원인이야 무엇이든 간에,
국가 내에 외인들의 존재는
로마의 전체 역사 과정을 결정했으며,
그 모든 역사단계는 완고한 국수주의와 이방인 인구 간의 갈등의 이야기를
크게 벗어나지 않는다.
이와 같은 것이 현대에는 발견되지 않거니와,
우선 현대 유럽 국가들은 다수 국민들이 너무 많다고 여길 정도의
외국이민을 거의 혹은 전혀 받아들이지 않아왔기 때문이며,
또한
국왕이나 주권기구에 대한 충성으로 결합되는 현대국가들은
상당한 규모의 이민자 집단도
신속하게 흡수하기 때문이다.
고대 세계는 이러한 신속한 흡수를 알지 못했다.
고대사회에서 국가의 최초 시민들은 언제나 스스로를 혈연의 친족관계에 의해
결합되어 있다고 생각했고,
특권의 평등을 주장하는 것은 그들의 생래적 권리를 찬탈하려는 것이라
여기며 분개했다.
로마 공화정 초기에는
공법 영역은 물론이고 사법 영역에서도
외인들의 철저한 배제가 만연했다.
외인이나 준\hanja{準}시민은
국가 영역에 해당하는 어떠한 제도에도 참여할 수 없었다.
그들은 로마시민법\latin{Quiritarian Law}의 혜택도 누릴 수 없었다.
그들은 초창기 로마인들의 물권이전방식이자 계약방식이었던
구속행위\hanjalatin{拘束行爲}{nexum}의 당사자가 될 수 없었다.
그들은 문명의 유년기로 그 기원이 거슬러올라가는 소송방식인
신성도금소송\hanjalatin{神聖賭金訴訟}{sacramental action}도 제기할 수 없었다.
그럼에도 불구하고 로마의 이익도 로마의 안전도 그들이 법적 보호를 박탈당하는
상태를 허용하지 않았다.
어떤 고대 공동체도 약간의 평화교란으로도 전복될 수 있는 위험을 안고 있었다.
그리하여 단순한 자기보존의 본능에서 로마인들은
외인들의 권리와 의무를 조정하는 방법을 고안해냈거니와,
그렇지 않았다면---그리고 이것은 고대 세계에서는
진짜로 중대한 위험요인이었는데---외인들의 무장봉기가 일어났을 것이기 때문이다.
더욱이 로마 역사의 어느 시기에도 외인들의 상거래가 완전히 무시된 적은
한 번도 없었다.
따라서,
당사자 모두가 외인인 분쟁이나 시민과 외인 간의 분쟁에 대해
재판권을 처음 인정한 것은
반쯤은 치안을 위한 조치였을 것이고, 반쯤은 상거래의 지속을 위해서였을 것이다.
이러한 재판권의 인정은, 재판의 대상이 된 문제들을 해결할 어떤 법원리들을
발견해야할 필요성을 즉시 불러왔다. 그리고
로마 법률가들이 이들 대상에 적용한 법원리들은 그 시대의
두드러진 성격을 반영한 것이었다.
전술했듯이 그들은 이들 새로운 사건에 로마 시민법을 적용하기를 거부했다.
그들이 거부한 이유는 분명, 외인인 당사자의 출신 모국의 법을 적용하는 것은
일종의 체면손상이라고 여겼기 때문일 것이다.
그들이 채택한 방법은 로마를 비롯하여
그 이주민들이 태어난 다른 이탈리아 공동체들에
공통되는 법규칙을 찾아내 적용하는 것이었다.
다시 말해서, 그들은
모든 민족들에 공통되는 법, 즉 만민법\latin{ius gentium}의 원시적이고
문자적인 의미에 합치하는 법체계를 만들어냈다.
실로 만법법은 옛 이탈리아 부족들의 관습 가운데 공통된 요소의 총합이었다.
이들 부족이 로마인들이 관찰할 수 있었던 \hemph{모든 민족들}이었고,
로마의 영역에 지속적으로 이주민 무리를 보낸 민족들이었던 것이다.
어떤 특정 관행이 개별 민족들의 대다수에서 공통적으로 발견되면,
모든 민족들에 공통되는 법, 즉 만민법으로 선언되었다.
그리하여,
비록 물건의 양도는 로마 인근의 여러 다른 국가들에서 각기 다른 형식으로
수행되었으나, 그 실제적 이전인 인도\hanjalatin{引渡}{tradition}, 즉
양도할 목적물을 교부하는 것은 그들 모두에서 의례행위의 일부를 구성했다.
예컨대 인도는 로마 특유의 양도방식인
악취행위\hanjalatin{握取行爲}{mancipation}의 일부분을, 비록
부차적인 부분에 불과했지만, 구성했다.
따라서,
법학자들이 관찰할 수 있었던 양도행위 방식들의 유일한
공통요소였을 인도는
만민법, 즉 모든 민족들에 공통되는 법의 규칙으로 선언되었다.
다른 수많은 관찰들도 마찬가지 방법으로 심사대상이 되었다.
공통의 대상을 가진 관찰들 모두에서
어떤 공통의 성질이 발견되면,
이러한 성질은 만민법에 속하는 것으로 분류되었던 것이다.
따라서 만민법은,
여러 이탈리아 부족들의 지배적 제도들에 공통적이라고
관찰로써 확인된,
그러한 법규칙과 법원리의 총체였다.

만민법의 기원에 관한 이러한 서술은
로마 법률가들이 만민법을 특별히 존중했을 것이라는 오해에 대한
좋은 방패막이 될 것이다.
만민법은 부분적으로는 일체의 외국법에 대한 경멸의 결과였으며,
부분적으로는 그들 고유의 시민법의 혜택을 외인들에게 주기를 꺼려하는
마음의 결과였다.
물론, 로마 법학자들이 수행했던 역할을 오늘날의 우리가 수행한다면,
우리는 만민법에 대해 사뭇 다르게 접근했을 것이다.
우리라면 그렇게 다양한 관행들을 관통하는 배경적 요소로 판별된 것에 대해
어떤 탁월성이나 우선성을 부여할 것이다.
우리라면 그렇게 보편적인 법규칙과 법원리에 어떤 존중심을 가질 것이다.
우리라면 그 공통의 요소를 당해 거래의 본질이라고 말할 것이다.
그리고 공동체마다 서로 다른, 나머지 의례적 장치들은
우연적이고 부수적인 것으로 폄하할 것이다.
혹은, 우리가 비교하고 있는 민족들이 한때 어떤 위대한 공통의 제도를
따랐고 만민법은 그것의 재현\hanja{再現}이라고 추론할 것이다.
그리고 개별 국가들의 복잡다기한 관행들은 한때 원시국가를 규율했던
보다 단순한 법제의 타락이고 퇴폐일 뿐이라고 추론할 것이다.
하지만 근대적 관념이 이끌어낸 이러한 결과들은
초기 로마인들이 본능적으로 감지하고 있던 것들과
거의 항상 정반대이다.
우리가 존중하고 칭송하는 것을 그들을 싫어하고 질시하고 두려워한다.
그들의 법 중에 그들이 애정했던 부분은
오늘날의 학자라면 모두 우연적이고 일시적인 것으로 무시할 것들뿐이다.
악취행위의 장엄한 몸짓, 언어계약\latin{verbal contract}의 정연한 질문과 답변,
변론과 소송절차의 한없는 형식주의 등등.
만민법은 단지 정치적 필요 때문에 어쩔 수 없이 용인한 법체계에 불과했다.
그들은 외인들을 사랑하지 않았듯이 만민법도 사랑하지 않았다.
만민법은 외인들의 법제도에서 추출한 것이고 외인들의 이익을 위한 것에 불과했던
것이다.
만민법이 그들의 존중을 받기 위해서는 근본적인 혁명이 필요했다.
그 일이 실제 발생했을 때 그것은 너무나 근본적이었으니,
만민법에 대한 현대적 평가가 방금 언급한 그들의 것과 다른 진정한 이유는
현대의 법학과 현대의 철학이
이 주제에 관한 후대 법학자들의 성숙한 관념을
물려받았기 때문이다.
시민법에 붙은 비천한 부속물에서
만민법은 일약 모든 법이 따라야할 위대한, 그러나 아직은 발달 중에 있는,
전범\hanja{典範}으로 간주되는 시대가 도래했다.
그 결정적 전기는
로마인들이
모든 민족들에 공통인 법의 실무적 집행에
그리스의 자연법이론을
적용하기 시작하면서 도래했다.

\para{자연법}
자연법\latin{ius naturale}은 만민법을 특정한 이론의 관점에서 바라본 것에
지나지 않는다.
법률가의 특징인 분류 성향에 따라
법학자 울피아누스가
이 둘을 구분하려는 애처로운 시도를 했지만,\footnote{%
  만민법과 자연법이 다른 것은 쉽게 알 수 있거니와,
  자연법은 모든 동물에 공통적인 법이지만 만민법은 인간들 사이에서만
  공통적인 법이다. Dig.\,1.1.1.4.}
훨씬 높이 평가되는 가이우스의 말에 따르면, 그리고
앞서 인용한 법학제요의 문구에 따르면,
이들 표현은 의심의 여지 없이 사실상 서로 바꾸어 쓸 수 있는
것들이었다.\footnote{다만 노예제도에 관한 한 자연법과 만민법은
서로 분기했다. 노예제도는 고대 모든 민족들에서 발견할 수 있었으나,
자연법상으로는 모든 인간이 자유롭게 태어났다고 여겨졌다. Inst.\,1.2.2.}
그들 간의 차이는 순전히 역사적인 것이었으며
본질적인 구별은 성립될 수 없었다.
만민법\latin{ius gentium}, 즉 모든 민족에 공통인 법과
\hemph{국제법}\latin{international law} 간의 혼동은 전적으로 근대적인 것임은
부연할 필요조차 없다.
국제법의 고전적 표현은
선전강화법\hanjalatin{宣戰講和法}{jus feciale},
즉 협상과 외교에 관한 법이었다.
하지만 만민법의 의미에 관한 모호한 인상은
독립 국가들 간의 관계가 자연법의 지배를 받는다는 오늘날의 이론을
낳는 데 크게 기여했을 것임에 틀림없다.

여기서
자연과 자연법에 관한 그리스인들의 관념을 살펴볼 필요가 생긴다.
퓌시스\latin{physis}는 라틴어로 나투라\latin{natura}, 우리말로는
자연\latin{nature}이라 번역되는데,
확실히 원래는 물질적인 우주를 뜻하는 말이었다.
그러나 그것은 현대적 언어로 표현하기 힘든---고대와 현대의 지적인 거리가
그만큼 멀다---어떤 관점에서 사고된
물질적 우주였다.
자연은 어떤 근원적인 요소 또는 근원적인 법칙의 결과로서의
물리적 세계를 의미했다.
초기 그리스 철학자들은
창조의 구성을 어떤 단일한 원리의 발현으로 설명하곤 했거니와,
그 원리를 그들은 운동, 힘, 불, 습기, 생성 등으로 다양하게 주장했다.
가장 단순한고 가장 고대적인 의미의 자연은 다름 아니라
이렇게 어떤 원리의 발현으로
간주된 물리적 우주였다.
후대에 이르러 그리스인들은, 위대한 그리스 지식인들이 벗어났던 길을 되돌려,
자연 개념의 \hemph{물리적} 세계에 \hemph{정신적} 세계를 추가했다.
자연이라는 말이 확장되어 가시적인 피조물뿐만 아니라 인간의 사상, 관찰, 소망까지
포괄하게 된 것이다.
그럼에도 불구하고 여전히, \hemph{자연}이라는 단어로 그들이 이해한 것은
그저 인간사회의 정신적 현상만이 아니라,
이러한 현상이 어떤 일반적이고 단순한 법칙으로 환원된다는 것까지 포함했다.

\para{스토아 철학}
초기 그리스 이론가들은 물리적 우주가 우연의 장난으로 단순한 원시적 형태에서
오늘날의 이질적인 복잡한 상태로 변화했다고 생각했다.
마찬가지로 이제 그들의 지적인 후손들도 만약 불행한 사고가 없었다면
인류는 보다 단순한 행위규칙과 보다 고난이 덜한 삶에 만족하며
살았을 것이라고 상상했다.
\hemph{자연}에 따라 사는 것이 인간이 창조된 목적이자
탁월한 인간이 달성해야할 목적으로 간주되기 시작했다.
\hemph{자연}에 따라 사는 것은 난잡한 습관과 저속한 것에의 탐닉을 넘어서는
고차원적인 행위법칙으로 고양되었고, 자제와 극기만이
이것을 따를 수 있게 해 준다고 생각되었다.
이 명제---자연에 따라 사는 것---가 저 스토아 철학의 핵심 신조였던 것은
너무도 유명하다.
그리스의 정복과 더불어 이 철학은 즉시 로마 사회로 흘러들어갔다.
이 철학에는 로마의 엘리트 계급을 사로잡는 매력이 있었다. 그들은, 적어도 이론적으로는,
고대 이탈리아 민족의 단순한 습관을 고수했고
외국풍의 혁신에 굴복하기를 경멸했던 것이다.
이런 사람들은 자연에 따른 삶이라는 스토아의 명제에 즉각 매료되었다.
세상을 약탈하고 가장 사치스런 민족의 대명사가 된 저 제국의 수도에 만연했던
무절제한 방종에 비추어볼 때, 참으로 감사한 매료요, 생각건대 참으로 고귀한 매료였다.
새로운 그리스 철학의 사도 무리의 맨 앞 열은,
역사적으로 증명할 수는 없을지라도, 로마 법률가들이 차지하고 있었음이 거의 확실하다.
여러 증거로 추정컨대,
로마 공화국에는 사실상 두 종류의 전문직만 있었거니와,
군인들은 일반적으로 변혁을 추진하는 당파에 속했고,
법률가들은 일반적으로 변혁에 저항하는 당파의 선두에 서 있었다.

\para{법무관의 고시}
법률가들과 스토아 철학의 결합은 수 세기에 걸쳐 지속되었다.
저명한 법학자들의 몇몇 초기 이름들은 스토아주의와 관련되어 있다.
나중에는 안토니누스 황조\latin{Antonine Caesars} 시대로 널리 합의되어 있는
로마법학의 황금기가 도래하거니와,
이 시기 황제들은 저 철학을 생활의 규칙으로 삼았던 유명한 사도들이었다.
특정 전문직 구성원들 사이에 이 신조가 장기간 확산됨에 따라
그들이 실무에 활용하고 영향을 끼쳤던 학문도 영향을 받지 않을 수 없었다.
저 스토아적 신조를 열쇠말로 사용하지 않으면
로마 법학자들이 남긴 유산에 속하는 몇몇 견해들은 거의 이해가 불가능해진다.
그러나 그렇다고 해서,
스토아주의가 로마법에 끼친 영향을,
스토아 교리에서 기원했다고 생각되는 법규칙의 숫자를 세어 측정하는 것은,
매우 흔하지만 심각한 오류에 해당한다.
스토아주의의 강점은,
때로 거부감을 불러일으키는 어처구니없는 행위준칙들에 있는 것이 아니라,
정념에의 저항을 가르치는 모호하지만 위대한 원리에 들어있다고 널리 인정되어왔다.
마찬가지로, 스토아주의로 대표되는 그리스 철학이 법학에 끼친 영향도
그것이 로마법에 기여한 여러 특정 견해들의 숫자가 아니라
그것이 가져다준 단일한 근본적인 전제에서 찾아야 한다.
자연이라는 단어가 로마인들이 일상적으로 사용하는 말이 되면서,
로마 법률가들 사이에서는
옛 만민법이 사실은 잃어버린 자연의 법전이라는 믿음이
점차 확산되어갔다.
또한 만민법 원리에 기초하여 고시법\hanja{告示法}을 형성함으로써
쇠퇴하기 시작한 법을 법무관들이 점차 다시 되살리고 있다는 믿음도 확산되어갔다.
이러한 믿음으로부터,
고시를 통해 가능한 한 많이 시민법을 대체하는 것이,
원시상태의 인간에게 자연이 가르쳐준 제도들을 가능한 한 많이 되살리는 것이,
법무관의 의무라는 생각이 즉각 추론되어 나온다.
물론 이러한 방법으로 법을 개선하는 데는 많은 장애가 따른다.
법전문직 내에서도 극복해야할 편견들이 있었고,
로마인들의 습관도 꽤나 끈질겨서 단순한 철학 이론에 당장 굴복하지는 않았다.
법무관들이 고시를 가지고 몇몇 법기술적 변칙들과 싸운 간접적인 방법들을 통해
우리는 그들이 신중하게 준수해야만 했던 것들을 엿볼 수 있다.
또한 유스티니아누스 시대에 이르기까지도 고법\hanja{古法}의 일부는
이러한 영향력에 완고하게 저항했던 것이다.
하지만 법 개선에 있어 로마인들의 진보는
자연법이론의 자극이 주어지자마자 신속하게 전개되었다.
단순화와 일반화의 관념이 자연의 개념에 밀접히 연관되어 있었다.
그리하여 단순성, 조화성, 명료성이 좋은 법체계의 특징으로 간주되었고,
복잡한 언어, 복잡한 의례, 무의미한 장애물들은
모두 사라져갔다.
로마법을 기존의 모습으로 되살리는 데는
유스티니아누스의 강력한 의지와 흔치않은 기회가 필요했지만,
로마법의 기초 계획도는 그의 제국 개혁이 착수되기 오래 전에
이미 수립되어 있었던 것이다.

\para{형평법의 기원}
옛 만민법과 자연법이 만나는 접점은 무엇인가?
나는 본래적 의미의 형평\latin{aequitas}을 통해
이 둘이 만나고 결합된다고 생각한다.
여기서 우리는 형평법\latin{equity}이라는 유명한 용어가
법학에 처음 등장함을 보게 된다.
이처럼 그 기원이 멀고 역사가 오래된 표현을 탐구할 때에는,
가능한 한,
일견 어렴풋이 개념의 그림자만 보여주는
단순한 은유나 상징을 파고드는 것이
언제나 가장 안전할 것이다.
흔히들 라틴어의 `형평'이 그리스어 `이소테스'\latin{isotes}와 동의어라고
하는데, 후자는 평등한 또는 비례적인 분배의 원리를 뜻한다.
숫자나 물리적 양을 평등하게 나누는 것은 분명 우리의 정의\hanja{正義} 관념과
밀접히 연관되어 있다.
인간의 정신에서 이처럼 강고하게 결합되어 있는 관념 연관을 찾기란 쉽지 않으며
가장 깊이있는 사상가들의 고된 작업을 통해서도 이것을 분리하기가 쉽지 않다.
하지만 이들의 연관을 역사적으로 추적해보면,
아주 초기의 사상에서는 이것이 나타나지 않거니와,
오히려 상대적으로 후대의 철학의 산물인 것으로 보인다.
또한 주목할 점은, 그리스 민주주의가 자랑하는
법의 ``평등''\latin{equality}---칼리스트라토스\latin{Callistratus}의
아름다운 권주가에 따르면
하르모디오스\latin{Harmodius}와 아리스토게이톤\latin{Aristogiton}이
아테네인들에게 주었다고 전해지는 그 평등---이
로마인들의 ``형평''\latin{equity}과 거의 공통점이 없다는 것이다.
전자는 시민들 사이에, 그 시민들의 계급이 비록 낮다고 할지라도,
시민법의 집행이 평등해야 한다는 의미이다.
후자는 시민이 아닌 자가 포함된 계급에게는 시민법이 아닌 법이
적용될 수 있다는 의미이다.
전자는 폭군을 배제한다는 뜻이고, 후자는 외인을, 경우에 따라서는 노예를,
포함한다는 뜻이다.
전반적으로 보아, 여기서 방향을 약간 틀어 로마인들의 ``형평''이란 단어의
기원을 살펴볼 필요가 있겠다.
라틴어 ``아이쿠스''\latin{aequus}는 그리스어 ``이소스''\latin{isos}보다
\hemph{평평하게 하기}\latin{levelling}라는 의미를 더 명백하게 가진다.\footnote{%
  `aequus'는 `평평한' `고른'이란 뜻으로, 라틴어 `aequitas'(형평)의 어원이 된다.}
이러한 평평하게 하는 경향은 정확히 만민법의 성격이었다.
초기 로마인들에게 만민법은 상당히 충격적이었을 것이다.
순수한 로마시민법은 사람과 물건에 대해 여러 가지 자의적인 분류를 두고 있었지만,
여러 민족들의 관습에서 일반화된 만민법은 로마시민법상의 구분을
알지 못했다.
예컨대 옛 로마법은 ``종족''\hanjalatin{宗族}{agnatic}인 친족과
``혈족''\hanjalatin{血族}{cognatic}인 친족을 근본적으로 구분했다.
전자는 공통의 가부장권\hanja{家父長權}에 복속하는 가족관계를
지칭하고,\footnote{정확히 말하면 `종족'은
나와 상대방(이들은 여자라도 상관없다)을 이어주는 가계도상의 연결점들이
모두 남자인 경우의 혈족관계를 의미한다. 남계혈족(男系血族)과 같은 뜻이다.}
후자는 \paren{오늘날의 관념에 일치하는 것으로} 단순히 공동의 혈통으로
결합된 가족관계를 지칭한다.
이러한 구분은 ``모든 민족들에 공통인 법''에서는 존재하지 않았다.
또한 ``악취물''\hanjalatin{握取物}{things \textit{mancipi}}과
``비악취물''\hanjalatin{非握取物}{things \textit{nec mancipi}}이라는
물건 분류의 옛 방식도 존재하지 않았다.
따라서 구분과 경계의 부재는 형평\latin{aequitas}으로 묘사되는 만민법의
특징이라 할 수 있다.
나는 이 형평이라는 단어가 처음에는 단지
이러한 끊임없는 \hemph{평평하게 하기}, 즉
울퉁불퉁함의 제거를 뜻했다고 생각한다.
이것은 외인 당사자가 개재된 사건에 법무관법이 적용될 때면 지속적으로 일어났다.
처음에는 이 표현에 어떤 윤리적 의미도 들어있지 않았을 것이다.
또한 초기 로마인들은 이러한 과정을 무척 싫어했을 것이라고
추정하지 않을 이유도 전혀 없다.

\para{형평과 평등}
한편, 형평이란 말로써 로마인들이 이해한 만민법의 특징은
최초로 생생하게 감지된 가상의 자연상태의 성격과 완전히 일치했다.
자연은 처음에는 물리적 세계의, 나중에는 정신적 세계의, 균형잡힌 질서였고,
질서에 대한 최초의 관념은 분명 직선, 평면, 측정된 거리 같은 것과
관련되어 있었다.
인간 정신의 눈이 가상의 자연상태의 윤곽을 그려내려 할 때든,
혹은 ``모든 민족들에 공통인 법''의 실제 집행을 바라보고 받아들일 때든,
그 정신의 눈 앞에는 이러한 종류의 그림 혹은 상징이 무의식적으로 그려졌을 것이다.
그리고 원시적 사고에 대한 우리의 모든 지식으로 판단하건대,
이러한 관념적 유사성은 이들 두 개념 간의 동일성에 대한 믿음을
불러일으켰을 것이다.
그런데 당시,
만민법은 로마에서 예전에 거의 혹은 전혀 권위를 인정받지 못하던 것이었으나,
자연법이론은 철학적 권위의 위신을 두른 채 들어왔을 뿐 아니라,
그것도 역사가 더 깊고 더 축복받은 민족의 것이라는 매력까지 품고 있었다.
이러한 관점의 차이가
옛 법원리의 작동과 새로운 이론의 결과를 동시에 기술하는 저 용어의 위엄에
어떤 영향을 주었을까는 쉽게 이해할 수 있다.
어떤 과정을 ``평평하게 하기''라고 묘사하는 것과
``변칙적인 것의 교정''이라고 부르는 것 사이에는
현대인들이 듣기에도 큰 차이가 있으나, 그러나
그 은유는 완전히 똑같은 것이다.
또한 형평이 저 그리스 이론을 암시하는 것으로 이해되자,
이제 `이소테스'라는 그리스적 관념이 형평 개념을 둘러싸기 시작했음에 틀림없다.
키케로의 언어가 이런 일이 실제 일어났음을 보여주고 있거니와,
이는 형평 개념의 변용의 첫 번째 단계였던 것이다.
그리고 그때 이후 등장한 거의 모든 윤리체계는 이 형평 개념을 전승해왔다.

\para{영구고시록}
처음에는 모든 민족들에 공통인 법과 관련되고 나중에는 자연법과 관련되는
법원리와 법개념들이 차츰 로마법에 흡수되어간 형식적 도구에 대해
몇 마디 말해둘 것이 있다.
타르퀴니우스 왕조 축출 사건으로 대변되는 로마 역사상 최초의 위기 시에
많은 고대국가의 초기 연대기에 나타나는 것과 유사한 변화가 일어났지만,
이는 오늘날 우리가 혁명이라고 부르는 정치적 변화와는 거의 공통점이 없는
것이었다.
왕정이 계속 유지되었다고 하는 것이 보다 정확한 기술일 것이다.
지금까지 한 사람의 수중에 집중되었던 권력이
다수의 선출직 관리들 사이에 분할되었으나,
왕이라는 명칭은 나중에
제사왕\hanjalatin{祭祀王}{rex sacrorum; rex sacrificulus}이라고
불리게 되는 사람에게 주어져 그대로 유지되었다.
변화의 일부로서 최고 사법관직의 기존 임무는 당시 국가의 최고 관리였던
법무관\latin{praetor}에게 부여되었다. 또한
이러한 임무와 더불어
법과 입법에 관한 불명확한 대권\latin{大權}도 그에게 이전되었거니와,
이는 고대의 통치자들이라면 누구나 가졌던 것이지만
한때 그들이 누렸던 가부장적이고 영웅적인 권위와의 희미한 연관성은
사라지고 없었다.
로마의 상황으로 인해 이렇게 이전된 기능 중에 보다 불명확한 부분이
더 큰 중요성을 가졌는데,
법기술상 본래의 로마인으로 분류할 수 없지만
그러나 로마의 법역 안에 상주하고 있는 사람들을 다루는 어려운 문제를
안겨준 재판들이
공화국의 수립 이후
지속적으로 제기되기 시작했기 때문이다.
이러한 사람들 간의 쟁송 및 이러한 사람들과 생래적 시민들 간의 쟁송은
법무관이 이러한 재판업무를 떠맡지 않았다면
로마법상 구제수단이 전혀 주어질 수 없는 것들이었다.
또한 곧이어 상거래가 확산되면서 로마 시민들과 자칭 외인들 사이에 발생한
보다 중대한 분쟁들에 대해서도 법무관이 대처하지 않으면 안 되었다.
제1차 포에니 전쟁을 전후하여 로마 법원에 이러한 소송이 대폭 증가하자,
후에 외인법무관\latin{praetor peregrinus}이라 불리게 되는,
이런 종류의 사건만 전담하는 특별한 법무관이 임명되기에 이른다.
한편, 압제의 부활에 대한 로마 인민들의 두려움으로 인해,
업무영역이 확장되는 경향을 가진 모든 정무관은
매년 임기 초에
자신이 맡은 업무를 앞으로 어떻게 수행할지를 선언하는
고시\hanjalatin{告示}{edict}를 공표할 의무가 부과되었다.
다른 정무관들과 함께 법무관도 이 규칙의 적용대상이었다.
그런데 해마다 따로 다수의 법원칙들을 고안해내는 것은 사실상 불가능하므로
법무관은 전임자의 고시를 거의 답습하여 재공표하고,
다만 그때그때의 상황에 따라 혹은 자신의 법적 견해에 따라
약간의 추가와 변경을 가하는 데 그쳤던 듯하다.
그리하여 장기간 매년 반복되는 법무관의 선포는
영구고시\latin{edictum perpetuum}라는 이름을 얻게 되었으니,
이는 \hemph{지속적인} 또는 \hemph{중단없는} 고시라는 뜻이다.
이것이 너무나 오랫동안 계속되자,
그리고 아마도 그 무질서해질 수밖에 없는 구조에 대한 염증 때문에,
하드리아누스 황제 재위기에 정무관직에 있었던
살비우스 율리아누스\latin{Salvius Julianus}의 임기 때에 이르러
더 이상의 확장이 중단되게 된다.
그리하여 이 법무관의 고시는 형평법의 총체였거니와,
아마도 새롭고 체계적인 질서를 갖추었을 것이다.
이후 로마법에서 이 영구고시록은 단순히
율리아누스 고시\latin{Edict of Julianus}로 흔히 인용되곤 했다.

고시의 특수한 메커니즘을 고찰하는 영국인의 머리에 떠오르는
첫 번째 의문은 이런 것이리라: 법무관의 이러한 포괄적 권한을 통제하는
제약\hanja{制約}은 무엇이었을까? 어떻게 그렇게나 불명확한 권한이
기존의 사회상황 및 법상태와 조화될 수 있었을까?
이에 대한 답변은 우리의 영국법이 운용되는 상황을 면밀히 관찰함으로써만
주어질 수 있을 것이다.
법무관은 그 자신이 법학자이거나, 아니면 법학자인 조언자들의 수중에 있는
사람임을 상기할 필요가 있다.
또한 로마 법률가라면 누구나 저 위대한 사법정무관직에 취임하거나 아니면
그 직을 통제할 날을 손꼽아 기다렸을 것이다.
그 사이 기간동안 그의 취향, 감정, 편견, 그리고 계몽의 정도는
불가피 그의 동료집단의 그것이었으며,
또한 후에 공직에 취임하거나 통제하게 될 때의 그의 자질은
그가 전문직으로서 실무와 연구에서 얻었던 것이었다.
영국의 챈슬러도 정확히 동일한 훈련을 거치며, 또한 동일한 종류의 자질을 가지고
챈슬러직을 수행한다.
그가 공직에 취임할 때는, 공직을 떠나기까지 어느 정도는
그가 법을 변경하리라는 것이 확실하다.
하지만 공직을 물러나고 그가 내린 판결들이 판례집에 수록되기
전에는, 그가 전임자에게서 물려받은 법원리를 얼마나 더 분명히 밝히고
또 새로운 것을 추가했는지 우리는 알 수 없다.
로마법에 대한 법무관의 영향도 단지 그 영향의 정도가 확인되는 시기에 있어서만
차이가 날 따름이었다.
전술했듯이 법무관의 임기는 1년에 불과했다.
또한 임기 동안 그가 내린 결정은, 물론 소송당사자들에게는 불가역적인 것이었으나,
장래에 대해 구속력을 갖지 않았다.
따라서 그가 계획하는 변화를 선포하는 순간은 당연히
법무관직에 취임하는 순간일 수밖에 없었다.
그리하여 임기 시작 시에 그는,
후에 영국의 챈슬러가 부지불식간에 그리고 때로는 몰래 행하는 것을
공개적이고 명시적으로 수행했다.
그의 외관상의 재량에 대한 통제는 영국 판사들에 대한 통제와
하등 다를 것이 없었다.
이론상으로는 양자의 권한에 거의 아무런 제한이 없는 것처럼 보이나,
실제적으로는 로마의 법무관도 영국의 챈슬러도
초기 훈련 과정에서 습득한 선이해에 의해, 그리고
전문직 그룹의 여론이라는 강력한 제약에 의해
엄격하게 한계지워진다.
이러한 제약의 엄격함은 직접 경험한 사람들만이 실감할 수 있는 것이다.
부연하건대, 움직임이 허락된 공간의 경계선, 넘어서는 안 되는 그 경계선은
영국만큼이나 로마에서도 분명히 그어져 있었다.
영국의 판사들은 고립된 사실관계에 대한 공표된 판결들의 유사성을 따라야 한다.
로마에서는, 법무관의 개입이 처음에는 국가의 안전이라는 단순한 고려에 의해
지배되었기에 아주 초기에는 제거하고자 하는 문제의 곤란함 정도에 비례하여
개입이 이루어졌을 것이다.
후에, 법학자의 해답에 의해 법원리에 대한 애호가 확산되자,
법무관은 그러한 근본원리들을 더 폭넓게
적용하기 위한 수단으로 그의 고시를 이용했을 것이 틀림없다.
이때 그와 나머지 실무 법학자들, 그의 동시대인들은
법의 저변에 놓여있는 그 원리들을 발견했다고 믿었다.
더 시간이 흐른 후에,
법무관은 이제 전적으로 그리스 철학이론의 영향력 하에서
행동했거니와, 이 이론은 특정한 진화의 방향으로 그를 이끄는 동시에
그 방향으로 가도록 그를 한계지웠다.

\para{그후 로마 형평법}
살비우스 율리아누스의 조치는 그 성격이 큰 논쟁의 대상이 되었다.
그 성격이 어떠하든 간에, 그것이 고시에 미친 효과는 사뭇 명백했다.
고시는 이제 해마다 확장되기를 그쳤고, 이후로
로마의 형평법은 하드리아누스 황제 치세와 알렉산데르 세베루스 황제 치세 사이에
활발한게 저술활동을 펼친 일련의 위대한 법학자들에 의해 발달하게 된다.
그들이 이룩한 경탄스런 체계의 일부가
유스티니아누스의 학설휘찬\latin{Pandects}에
남아있거니와, 이를 통해 우리는 그들의 작품이 로마법의 모든 영역에 관한
논저의 형태를,
그러나 주로 고시에 대한 주해서의 형태를, 띠고 있었음을 알 수 있다.
실로 이 시대의 어떤 법학자가 어떤 주제 하나를 다루었다 할지라도
언제나 그는 형평법의 해설자로 불릴 만하다.
고시에 담긴 법원리들은 이 시대가 끝나기 전에 로마법학의 모든 영역에
침투해 들어갔다.
로마의 형평법은, 비록 시민법과 완전히 동떨어진 경우에도,
언제나 동일한 법원에 의해 재판되었다는 점을 잊지 말아야 한다.
법무관은 형평법 수석판사인 동시에 보통법 수석판사이기도 했다.
그리하여 고시가 어떤 형평법규칙을 발달시키면,
법무관의 법원은 바로 옛 시민법규칙을 대체하여 혹은 그것과 병행하여
그 형평법규칙을 적용하기 시작했다. 이것은
입법기관의 명시적 법제정 없이 시민법이 직^^b7간접적으로 폐지되는 결과를 낳았다.
물론 이것은 시민법과 형평법의 완전한 통합에는 전혀 이르지 못하는 것이었다.
이 통합은 후에 유스티니아누스의 개혁에 의해 비로소 이루어진다.
두 영역의 법이 법기술상 분리되어 있다는 사실은
일말의 혼동와 일말의 불편함을 낳았다. 또한
시민법 법리 가운데 아주 완고한 것들은 고시의 선포자들도 그 해설자들도
감히 건드리지 못하는 것들이 있었다.
하지만 법학 분야 중에
형평법의 영향이 다소간 휩쓸고 지나가지 않은 구석은 하나도 없었다.
그것은 법학자들에게 일반화를 위한 자료를,
해석의 방법을, 근본원리들의 해명을 제공했다. 또한
입법자에 의한 개입이 거의 없는,
오히려 입법의 적용에 중대한 통제를 가하는,
다량의 요건규칙들도 제공했다.

법학자의 시대는 알렉산데르 세베루스 황제와 더불어 종말을 고한다.
하드리아누스로부터 이 황제에 이르기까지 법의 발달은,
오늘날 대부분의 대륙법계 국가들에서와 마찬가지로,
부분적으로는 공인된 주해에 의해,
부분적으로는 직접적인 입법에 의해 이루어졌다.
하지만 알렉산데르 세베루스의 치세에 로마 형평법의 성장력은 소진되었고,
법학자들의 잇따른 등장도 마감되었다.
로마법의 나머지 역사는 황제의 칙법의 역사이고,
종국에는 오늘날 로마법의 거창한 집적물로 남겨진 것을 편찬하려는
시도의 역사이다.
이런 종류의 실험 중에 최후의 그리고 가장 칭송받는 것으로
유스티니아누스 황제의 로마법대전\latin{Corpus Juris}이 우리에게 전해지고 있다.

\para{영국과 로마의 형평법}
영국과 로마의 형평법을 집요하게 비교하고 대비시키는 것이 지루하게 느껴질 수도
있겠다. 하지만 그들의 공통점 두 가지는 언급해둘 가치가 있다.
첫째는 이렇게 말할 수 있을 것이다:
그 둘은 모두, 모든 이러한 체계가 그러하듯이,
형평법이 처음 개입했을 때의 옛 보통법의 상태와 정확히 같은 상태에
이르는 경향이 있었다.
최초에 도입된 도덕적 원리들이 모든 정당한 결과들을 낳으며 역할을 다한 후,
그들에 기초한 체계가 굳어지고, 더 이상 확장이 안 되고,
보통법이라 부르는 아주 엄격한 규칙체계와 마찬가지로
도덕적 진보에 뒤처지기 시작하는
시기가 반드시 도래한다.
로마에서는 그 시기가 알렉산데르 세베루스 재위기에 도래했다.
그후, 전체 로마 세계가 정신적 혁명에 휩싸였지만, 로마의 형평법은
더 이상 확장되지 못했다.
영국의 법제사에서는 동일한 시기가 엘던 경\latin{Lord Eldon}이
챈슬러직에 있을 때 도달했다.
간접적인 입법에 의해 형평법을 확장시키는 대신, 그는
형평법을 설명하고 조화시키는 데만 평생을 바친 최초의 형평법 판사였다.
법제사의 교훈이 영국에서 좀 더 잘 이해되었더라다면,
엘던 경의 업적은
당대 법률가들 사이의 평판보다
한편으로는 덜 과장되었을 것이고,
다른 한편으로는 더 나은 평가를 받았을 것이다.
실무적 결과에 영향을 주는 또 다른 오해도 또한 불식되어야 한다.
영국의 형평법이 도덕 규칙들에 기초한 체계임을
영국 법률가라면 누구나 쉽게 이해한다.
하지만 이 규칙들이---현재가 아니라---수 세기 전 과거의 도덕임은 잊고 있다.
그동안 너무 많이 적용되어 능력이 거의 소진될 지경에 이르렀음은 잊고 있다.
그것들이 물론 오늘날의 도덕적 신조와 크게 다르지는 않다 할지라도
오늘날의 그것을 따라잡지 못하는 것일 수 있음은 잊고 있다.
이 주제에 관한 불완전한, 그러나 널리 받아들여지고 있는, 이론들이
서로 상반되는 종류의 오류를 생산해왔다.
형평법에 관한 논저의 저자들 다수는
현재 상태의 체계의 완전성에 매료되어
명시적^^b7묵시적으로 역설적인 주장을 펼치고 있거니와,
형평법의 창시자들이 처음 그 기초를 다졌을 때 이미 현재와 같은 고정된 형태를
만들어냈다는 주장이 그것이다.
또한 다른 이들은---법정 변론에서 자주 들리는 불평인데---형평법법원이
강제하는 도덕 규칙들이 오늘날의 윤리 기준에 미치지 못한다고 불평하고 있거니와,
그들은 영국 형평법의 창시자들이 옛 보통법에 대해
행하던 역할과 정확히 동일한 역할을
지금의 챈슬러들이
수행해주기를 형평법에 대해 요구하고 있는 것이다.
하지만 이것은 법의 발달이 진행되는 순서를 거꾸로 뒤집는 것이다.
형평법에는 자신만의 장소와 시간이 있다.
나는 다른 수단이 있음을, 에너지만 주어진다면 그 수단이 성공할 것임을,
앞서 지적한 바 있다.

영국과 로마의 형평법의 또 하나의 주목할만한 성격은
형법법이 보통법이나 시민법보다 우월하다는 주장이 처음
개진될 때, 이 주장이 근거했던 가정들이 모두 허위였다는 점이다.
개인이든 집단이든 인간에게 있어 도덕적 진보를 실제적 현실로
받아들이는 것만큼 싫은 것이 없다.
이 거부감이 개인에게서는 일관성이라는 의심스런 덕목을 과장되이 존중하는
모습으로 통상 나타난다.
전체 사회의 수준에서도 집단적 여론의 움직임은 너무나 명백해서 무시할 수 없고
대체로 너무나 뚜렷이 더 좋은 것을 향하기에 대놓고 비난할 수 없으나
그것을 주요한 현상으로 인정하기를 꺼리는 경향이 강하게 존재하거니와,
보통은 잃어버린 완전성의 회복---인류가 타락하기 이전 상태로의 점진적
회귀---으로 설명된다.
이렇게 도덕적 진보의 목표를 앞을 바라보는 데서가 아니라
뒤를 돌아보는 데서 찾는 경향은, 전술했듯이,
고대 로마법에 가장 심각하고 영속적인 영향을 주었다.
로마 법학자들은, 법무관에 의한 법 발달을 설명하기 위해서,
실정법에 의해 통치되는 국가들이 조직되기 이전에 존재한
인간의 자연상태---자연적 사회---의 이론을 그리스로부터 빌려왔다.
한편, 영국에서는, 당시 영국인들의 취향에 특별히 부합하는 관념으로
보통법에 대한 형평법의 우월성 주장을 설명했거니와,
국왕이 갖는 가부장적 권위의 당연한 결과로서 국왕에게는
사법\hanja{司法}을 감독할 일반적 권리가 있다는 것이 그것이었다.
동일한 견해가
``형평법은 국왕의 양심에서 유래한다''는
옛 법리에서
보다 고풍스런 형태로
등장했으니, 이는
실제로는 공동체의 도덕 기준에 진보가 일어난 것을
주권자의 내면적 도덕 감각의 상승으로 치환시키고 있는 것이다.
영국 헌정의 발달로 후에 이러한 이론은 부적합한 것이 되었지만,
실은 형평법법원의 재판권이 확고하게 자리잡음에 따라 공식적인 대체물을
고안할 가치가 없어졌던 것이다.
오늘날 형평법 교재들에서 발견되는 이론들은 참으로 다양하지만,
하나같이 유지될 수 없는 이론들뿐이다.
그 대부분은 자연법에 기초한 로마법 이론의 변용이거니와,
이는
자연적 정의와 시민적 정의의 구분으로써
형평법법원의 재판권에 대한 논의를 시작하는
저술가들에 의해
실제로 그대로
채용되고 있다.


\chapter{자연법의 근대사}

지금까지의 논의로부터, 로마법의 변화를 가져온 저 이론은 어떤 철학적 엄밀성을
주장한 것이 아니었음을 알 수 있을 것이다.
그것은 사실 일종의 ``혼합적 사고양식''이었거니와,
이러한 사고양식은 오늘날 최고의 정신을 제외한 모든 인간 정신의 유아기적 사고의
특징으로 인식되고 있으며 또한 현 시대의 정신에서도 어렵지 않게
발견할 수 있는 것이다.
자연법 이론은 과거와 현재를 혼동했다.
논리적으로는, 그것은 한때 자연법에 의해 통치되었던 자연상태를 상정한다.
하지만 로마의 법학자들은 그러한 자연상태의 존재를 분명하게 그리고
자신있게 말하지 않았다. 사실 황금시대를 상상하는 시적인 표현을
제외하면 고대인들은 그러한 상태에 대해 거의 언급하지 않았다.
실무적 목적에서는, 자연법은 현재에 속하는 어떤 것이고,
기존의 제도와 얽혀있는 어떤 것이며, 유능한 관찰자에 의해
기존 제도와 구분될 수 있는 어떤 것이다.
자연의 명령을 이와 함께 섞여있는 조잡한 요소들로부터 분리하는 기준은
단순성과 조화성의 감각이었다.
하지만 이들 더 세련된 요소가 애초 존중받은 것은
단순성과 조화성 때문이 아니라,
자연의 원초적 지배의 후예라는 데에 있었다.
이러한 혼동은 근대의 법학자들에 의해서도 성공적으로 해명되지 못했다.
실로
로마 법률가들이 받아야 할 비난보다 오히려
오늘날의 자연법사상이 인식의 불명료성을 훨씬 더 많이 노정하고 있으며
언어의 절망적인 모호성에 의해 더 많이 오염되어 있다.
이 주제에 관한 저자들 몇몇은, 자연법법전은 미래에 존재하는 것이고
모든 시민법들이 지향해야 할 목표라고 주장함으로써,
이러한 근본적인 난제를 피해가려고 시도하나,
이는 옛 이론이 근거하고 있던 가정을 순서만 뒤집는 것이거나,
아니면 서로 양립할 수 없는 두 이론을 뒤섞는 것에 불과할 것이다.
과거가 아니라 미래에서 완전성을 찾는 경향은 기독교에 의해
이 세상에 도입된 것이다.
사회의 진보가 더 나쁜 것에서 더 좋은 것으로 필연적으로 진행된다는 믿음은
고대 문헌에서는 거의 혹은 전혀 발견되지 않는다.

하지만 그 철학적 결함에 비해 이 이론이 인류에게 미친 영향은 훨씬 더 심대했다.
만약 자연법의 믿음이 고대세계에 보편적으로 퍼지지 않았다면
어떤 사상사적 전환이, 또 그에 따른 인류사적 전환이, 일어났을까는
실로 말하기가 쉽지 않다.

\para{자연법}
법, 그리고 법에 의해 결합되는 사회는 그 유아기에
두 가지 위험에 특히 취약하다.
하나는 법이 너무 빨리 발달할 수 있다는 것이다.
진보적인 그리스 공동체들에서 이런 일이 발생했거니와,
이들 공동체는 놀라운 능력으로 불편한 소송절차와 불필요한 법률용어의
질곡을 벗어던졌고, 곧이어 엄격한 규칙과 법규정들에 미신적 가치를 부여하는 일을
그만두었다.
이것으로 그 공동체의 시민들이 누린 직접적 혜택은 상당히 컸지만,
그것은 인류의 궁극적 이익에 기여하지는 못했다.
민족성의 드문 자질 중 하나는,
보다 높은 이상에 법을 일치시키려는 희망을 잃지 않으면서도,
법 자체의 적용과 운용에 있어
추상적 사법\hanja{司法}을 구현하는 데는 지속적으로 실패하는 능력이다.
유연성과 탄력성에 뛰어난
그리스의 지식인들은
엄격한 법형식의 틀 속에 스스로를 가둘 수가 없었던 것이다.
우리가 비교적 소상히 알고 있는 아테네의 인민법원을 두고 판단하건대,
그리스의 법원은 법률문제와 사실문제를 혼동하는 경향을 강하게 나타냈다.
아리스토텔레스의 <<수사학>>\latin{Treatise on Rhetoric}에 남아있는
웅변가\latin{orator}들과 법정 표현들의 흔적을 보건대,
순수한 법률문제의 변론은
판사들에게 영향을 줄 수 있는 모든 것을 끊임없이 고려하면서 이루어졌다.
이런 방식으로는 지속가능한 법학체계가 만들어질 수 없다.
특정 사건의 사실관계에 대한 완벽한 이상적인 결정에 성문법 규칙이 방해되는
경우라면 언제나 그 성문법 규칙을 완화하는 데 거리낌이 없었던 공동체는,
설령 후대에 어떤 법원리들을 물려준다 하더라도
오직 당대에 지배적이었던 옳고 그름의 관념에 기초한 것들만 물려줄 수 있을 뿐이다.
이러한 법은 후대의 보다 발달된 관념에 어울릴만한 틀을 전혀 제공할 수 없다.
기껏해야 그 법을 둘러싼 문명의 볼완전성을 드러내는 철학이 될 수 있을 뿐이다.

국가 사회 중에 그들의 법이
이러한 때이른 성숙과 때아닌 해체의 위험에 의해 위협받은 곳은 많지 않다.
로마인들이 이러한 위협에 심각하게 노출된 적이 있었는지는 모르겠으나,
어쨌든 그들의 자연법 이론에는 적절한 보호장치가 들어있었다.
분명 법학자들은 시민법을 점진적으로 흡수하는 체계로 자연법을 관념했으며,
시민법이 폐지되지 않는 한 자연법이 시민법을 대체할 수는 없다고 생각했다.
특정 소송사건을 감독하는 판사들이 자연법의 호소에 압도당할 정도로
그렇게 자연법이 신성하다는 인상은 유포되지 않았다.
이러한 관념의 가치와 유용성은 완벽한 유형의 법이 인간 정신의 눈 앞에
펼치지지 못하게 한 것이었고, 그러한 법에 무한히 가까이 다가갈 수 있다는
희망을 품지 못하게 한 것이었으며, 또한 아직 자연법에 조응하지 못한
기존 법이 부과한 의무를 실무가나 시민들이 거부하지 못하게 한 것이었다.
무엇보다 중요한 점은 이 모범적인 체계가,
후대에 인간의 희망을 꺾어놓았던 다른 많은 체계들과 달리,
결코 상상의 산물이 아니었다는 것이다.
그것은 결코 허황된 원리에 기초한 것으로 관념되지 않았다.
그것은 기존 법의 저변에 존재하며 기존 법을 통하여 추구되어야 한다고 생각되었다.
한마디로 그것의 기능은 구제수단을 제공하는 데 있었지, 혁명적이거나
무정부적인 것이 아니었다.
그리고 정확히 이 점에 있어 불행하게도 근대 자연법 사상은 고대의 그것을
닮지 않은 경우가 많다.

\para{벤담주의}
유아기의 사회에 나타나는 또 하나의 취약성은 훨씬 더 많은 민족들의 진보를
방해하고 가로막았다.
원시법의 엄격성은 대개 일찍이 종교와의 관련성 및 동일시에 의해 등장했거니와,
대부분의 민족들은 그 관행이 처음 체계적인 형태로 굳어질 당시 그들이
가지고 있던 인생관과 행위관에 얽매여왔다.
놀라운 운명으로 이러한 재난을 벗어난 민족이 한 둘 있거니와,
이들 나무줄기에 접목하여 몇몇 근대사회가 기름진 곳이 될 수 있었다.
하지만 여전히 세계의 더 많은 곳에서는 최초 입법자가 그려놓은 기본계획을
추종하는 것이 법의 완성이라고 여겨지고 있다.
만약 그런 곳에서 지식인이 법을 공부했다면, 한결같이 그들은
고대 텍스트에서 그 문자적 의미를 눈에 띄게 벗어나지 않고 이끌어낸 결론의
미묘한 고집스러움을 자랑스러워했을 것이다.
만약 자연법 이론이 비범한 탁월함을 로마법에 주지 않았더라면
로마인들의 법이 인도인들의 법에 비해 우월하다고 할 것이
무엇이 있는지 나는 모르겠다.
이 유일한 예외적인 사례에서, 다른 여러 이유들로 인류에 막대한 영향을 끼치게 될
한 사회의 눈 앞에서, 단순성과 조화성이 이상적이고 가장 완전한 법의 성질로
나타났던 것이다.
진보를 추구함에 있어 어떤 뚜렷한 목표를 가진다는 것이
한 민족이나 전문직업군에게 가지는 중요성은 아무리 강조해도 지나치지 않다.
지난 30년 동안 영국에서 벤담이 가졌던 막대한 영향력의 비밀은
이 나라 앞에 그러한 목표를 성공적으로 제시한 데 있다.
그는 우리에게 개혁의 뚜렷한 원칙을 제시했다.
지난 세기의 영국 법률가들은 아마도 명민했기에
영국법이 인간 이성의 완성이라는 흔해빠진 역설적 표현에 눈멀지는 않았겠지만,
일을 추진해나갈 다른 원리가 없었기 때문에 영국법을 믿는 척 행동했다.
벤담은 다른 모든 목표 위에 공동체의 선\hanja{善}을 두었고,
그리하여 오랫동안 밖으로 빠져나갈 길만 찾고 있던 흐름에 나갈 길을 열어주었다.

우리가 기술해온 관념을 벤담주의의 고대적 대응물이라고 부른다면
그것은 그리 훌륭한 비교가 못 될 것이다.
로마인의 이론은 저 영국인의 이론과 마찬가지 방향으로 인간의 노력을 이끌었다.
그것의 실천적 결과도 공동체의 일반적 선\hanja{善}을 꾸준히 추구해온
일군의 법개혁가들이 달성한 것과 크게 다르지 않았다.
하지만 그것이 벤담의 원리를
의식적으로 예견한 것이었다고 보는 것은 잘못일 것이다.
로마인들의 대중문헌이나 법학문헌에서, 분명
구제수단을 제공하는 입법의 목표로
때로 인류의 행복이 제시되곤 했지만,
자연법이라는 돋보이는 주장에 주어진 끊임없는 칭송에 비하면
벤담의 원리를 보여주는 증언은 거의 없거나 희미하다는 점을
주목해야 한다.
로마 법학자들이 기꺼이 수용한 것은 인간에 대한 사랑 같은 것이 아니라
단순성과 조화성---그들이 ``전아''\hanjalatin{典雅}{elegance}하다고 부르며
강조했던 것---의 감각이었다.
그들의 노력이 보다 정밀한 철학의 조언을 받아들인 자들의 노력과
우연히 일치했다는 것은 인류에게는 행운이었다.

\para{프랑스 법률가들}
자연법의 근대사로 전환하면, 우리는 그것의 영향력이 막대하다는 것은
말하기 쉬워도 그 영향이 좋은 것인지 나쁜 것인지를 자신있게
말하기는 어렵다는 것을 알고 있다.
근대 자연법 이론에서 나왔다고 할 수 있는 신조와 제도들은
우리 시대의 가장 첨예한 논쟁의 대상이거니와,
지난 백년 동안 프랑스가 서구 세계에 확산시킨 법, 정치, 사회에 관한
특수한 이념들 대부분이 자연법 이론에 그 원천을 두고 있다고 말할 때
이것이 잘 드러난다.
프랑스 역사에서 법률가들의 역할은,
그리고 프랑스 사상에서 법사상의 비중은,
언제나 대단히 컸다.
근대 유럽의 법학이 발흥한 곳은 사실 프랑스가 아니라 이탈리아였지만,
이탈리아로 유학하고 돌아와 전 유럽대륙에 건설된,
그리고 {\small(허사로 돌아갔지만)} 우리 영국에 건설이 시도된,
학자군 중에서 프랑스에 건설된 학자군이 그 나라의 운명에 가장 큰
영향을 끼쳤다.
프랑스의 법률가들은 즉각 카페 왕조 및 발루아 왕조의 왕들과
강력한 동맹관계를 형성했다.
프랑스의 왕권이 여러 소국과 속령들의 집합체에서 떨어져나와
결국 그 위에 성장하게 된 것은 무력에 의한 것인 동시에
법률가들이 왕의 대권을 옹호해주고 봉건적 세습규칙을 해석해 준 데에도 기인했다.
왕과 법률가들의 동맹으로 프랑스 왕들이
강력한 봉건제후들, 귀족들, 교회와의 투쟁과정에서 누린 이점은
중세를 거슬러 올라가 당시 유럽을 지배하던 이념을 고려하지 않으면
제대로 평가할 수 없다.
우선 일반화를 향한 강한 열정이 있었고 모든 일반적 명제를 향한 찬양이 높았다.
그리하여 법의 분야에서는, 여러 지방에서 관습적으로 사용되던 고립된
다수의 법규칙들을 하나로
포괄하고 요약하는 모든 일반적 공식\latin{formula}에 대한 무의식적 존중이 있었다.
이러한 공식은 로마법대전이나 표준주석\latin{the Glosses}\footnote{%
  주석학파를 집대성한
  아쿠르시우스(Accursius)의 표준주석(glossa ordinaria)을 말하는 듯.}에
익숙한 실무가라면 물론 얼마든지 제공할 수 있었다.
하지만 법률가들의 권력을 사뭇 증대시킨 또 다른 원인도 있었다.
우리가 말하는 이 시대에는 쓰여진 법 텍스트가 가진는 권위의 정도와 성질에 관한
관념이 보편적으로 퍼져있었다.
대체로 ``이것이 쓰여진 법이다''\latin{Ita scriptum est}라는 우선권을 가진
주장은 모든 항변을 침묵시키기에 충분했다.
우리 시대의 학자라면 인용된 공식을 조바심내며 조사하고,
그 출처를 따져묻고, {\small(필요하다면)} 인용이 들어있는 법령집이
지방관습을 대체할 만한 권위를 가지고 있지 않다고 부인하겠지만,
그 시대의 법학자들은 규칙의 적용가능성을 의문시하거나
기껏해야 학설휘찬이나 교회법에서 반대명제를 인용하는 것 외에
다른 것은 거의 시도하지 않았다.
법논쟁의 이러한 중요한 측면에 대해 사람들이 주저하는 관념을 가졌다는 것을
염두에 두는 것은 무척 중요하거니와,
그것은 법률가들이 국왕에게 힘을 실어준 것을 설명하는 데 도움이 될 뿐만 아니라,
몇몇 흥미로운 역사적 문제를 해명하는 데도 도움을 주기 때문이다.
위조된 교령\hanja{敎令}들\latin{forged decretals}\footnote{%
  `콘스탄티누스의 기증'을 포함한 `이시도르 위서'%
  \hyphlatin{(Pseudo-Isidore)}를 말한다.}을
만든 저자의 동기와 그의 특별한 성공은 이런 맥락에서
더 잘 이해될 수 있는 것이다.
보다 덜 흥미로운 예를 들자면, 브랙턴\latin{Bracton}의 표절을 이해하는 데도,
비록 부분적이지만, 도움을 준다.
헨리3세 시대의 저 영국 법률가는
순수한 영국법의 집성을 당대 영국인들에게 내놓을 수 있었는데,
이는 편제의 전부와 내용의 1/3을 로마법대전에서 직접 빌려온 논저였다.
로마법의 체계적인 연구가 공식적으로는 금지된 나라에서
이러한 작업을 감행했을 것이니, 이는 법학의 역사에서 영원히 풀리지 않는
수수께끼의 하나일 것이다.
그러나, 텍스트의 출처에 대한 고려는 차치하고라도,
쓰여진 텍스트가 가지는 구속력에 관한 당대의 여론 상황만 감안해도
우리의 놀라움은 다소간 완화된다.

프랑스의 왕들이 주권 확립을 위한 긴 투쟁을 성공적으로 종결지은 때,
즉 대체로 발루아^^b7앙굴렘 왕조의 재위기에 이르러,
프랑스 법률가들의 상황은 사뭇 특수한 것이었고 이 상태는
프랑스혁명 발발 시까지 지속된다.
한편으로, 그들은 프랑스에서 가장 식자층에 속했고
대단히 강력한 권세를 누리는 계급을 형성했다.
그들은 봉건귀족들과 나란히 특권계급의 신분을 가졌으며,
프랑스 전역에 걸쳐 분포한 기구를 통해 그들의 영향력을 행사했거니와,
이 기구는 국왕의 특허로 각지에 설립되어
폭넓은 명시적 권한과 더 폭넓은 묵시적 권리를 행사했다.
변호사, 판사, 입법자의 권한을 모두 가진 그들은 유럽 전역의 다른 동료집단들을
훨씬 능가하는 권력을 누렸다.
그들의 재판 기술, 표현의 능란함, 유추와 조화에 대한 세련된 감각,
그리고 {\small(가장 뛰어난 인물들로 판단하건대)}
그들의 정의관에 대한 열정적 헌신
따위는 그들 중에 특출난 재능을 보였던 다양한 인물들만큼이나 두드러졌다.
이들 인물들의 다양성은
퀴자\latin{Cujas}와 몽테스키외, 다그소\latin{D'Aguesseau}와
뒤물랭\latin{Dumoulin}처럼 서로 대척점에 위치한 이들 사이의
전 영역을 아우르는 것이었다.
하지만, 다른 한편으로, 그들이 집행해야 했던 법체계는
그들이 훈련받은 법학의 정신과는 현저히 다른 것이었다.
상당 부분 그들의 노력으로 만들어진 당대의 프랑스는
유럽의 다른 어떤 나라 이상으로 법의 변칙성과 불일치라는 저주에 휩싸여 있었다.
프랑스를 가로지르는 큰 구획선이 그 나라를
성문법지역\latin{Pays de Droit Écrit}과
관습법지역\latin{Pays de Droit Coutumier}으로 갈라놓고 있었으니,
전자는 쓰여진 로마법을 그들 법의 토대로 받아들이고 있었고,
후자는 지방적 관습과 조화되는 한도 내에서
단지 표현의 일반적 양식으로만, 그리고
법적 추론의 수단으로만, 로마법을 인정하고 있었다.
이러한 분열 아래에는 계속적으로 하위 분열이 존재했다.
관습법지역에서는 지방\latin{province}마다,
군현\latin{county}마다,
성읍\latin{municipality}마다 그 관습의 성질이 달랐다.
성문법지역에서는
로마법 층위 위에 봉건규칙들의 층위가 대단히 잡다한 양상으로
펼쳐져 있었다.
이러한 혼란상은 영국에는 존재한 적이 없었다.
독일에서는 존재했지만, 그것은 이 나라의 정치적^^b7종교적 분열상에
어울리는 것이었기에 한탄의 대상도, 심지어 감지의 대상도 되지 못했다.
국왕의 중앙권위가 부단히 강화되고 있었음에도,
행정의 통일성을 달성하기 위한 노력이 빠르게 진행되고 있었음에도,
인민들 사이에의 뜨거운 민족정신이 발달하고 있었음에도,
법의 비상한 다양성이 특별한 변화없이 지속된 곳은 프랑스가 유일했다.
이러한 현저한 대비\hanja{對比}는 여러 가지 심각한 결과를 낳았는데,
그중에서 첫째로 꼽아야 할 것은 프랑스 법률가들의 정신에 미친 영향일 것이다.
그들의 사변적 의견과 지성적 성향은 그들의 이해관계나 직업적 관행과
크게 상반되는 것이었다.
단순성과 통일성에 기초한 완전한 법에 대해 민감하게 느끼고 이를 완전히
수용하고 있음에도 불구하고,
그들은 프랑스법을 감싸고 있는 악덕을 근절 불가능하다고 믿었거나
혹은 그렇게 믿는 듯이 보였다.
실제로 그들은 덜 계몽된 프랑스인들 사이에서는 볼 수 없는 고집스러움으로
악습의 개혁에 저항하곤 했다.
그러나 이러한 자기모순을 조화시키는 길이 있었다.
그들은 열렬한 자연법론자들이었다.
그 자연법은 지방 간의 경계를, 성읍 간의 경계를 뛰어넘는 것이었고
귀족과 도시민\latin{burgess}의 구별을, 도시민과 농민의 구별을
알지 못하는 것이었으며
명료성, 단순성, 체계성에 최고의 지위를 부여하는 것이었으나,
그 신봉자들에게 어떤 특정한 진보도 의무지우지 않는 것이었고
존경받고 돈벌이되는 법기술을 직접 위협하지도 않는 것이었다.
자연법은 프랑스의 보통법이 되었다고 말할 수 있을 것이다.
아니면, 여하튼 자연법의 존엄과 가치는
모든 프랑스 법실무가들이 한결같이 승인하는 유일한 신조였던 것이다.
혁명 이전의 법률가들의 언어에서 자연법의 찬미는 자못 무조건적이었다.
특히,
순수한 로마법을 폄하하는 것을 의무로 여기곤 했던
관습법 연구자들이,
학설휘찬과 칙법휘찬만 존중하는 당대 로마법 학자들보다
훨씬 더 자연과 자연법을 열성적으로 이야기했던 것이다.
옛 프랑스 관습법의 최고 권위자였던 뒤물랭은
자연법에 대해서 몇몇 과도한 진술들을 하였거니와,\footnote{메인의 `고대법'에
대한 폴록의 주석에 따르면,
뒤물랭의 저술에서 이러한 자연법의 찬사는 찾을 수 없었다고 한다.}
그의 찬사가 담고 있는 특유의 수사학적 전환은
그것이 로마시대의 법학자들의 조심성과는 사뭇 거리가 먼 것이었음을 알려준다.
자연법의 가설은 이제 법실무를 이끄는 이론이 아니라
사변적 믿음의 규약이 되었다.
그리하여, 곧 언급하겠지만, 자연법의 최근의 변화에서는
지지자들에게 가장 약하게 존중받던 부분이 가장 강하게 존중받는 지위로
올라섰던 것이다.

\para{루소와 그의 이론}
18세기가 절반이 지났을 때 자연법의 역사에서 가장 중대한 시기가 도래한다.
그 이론과 결과에 대한 논의가 법전문가들 사이에서만 계속되었다면,
자연법이 누리던 신망은 감소되어갈 가능성이 있었다.
바로 이때 <<법의 정신>>\latin{Esprit des Lois}이 등장했던 것이다.
아무런 심사없이 무사통과되던 가정\hanja{假定}들에 극히 격렬하게 반발하는
특징을 조금은 과장되게 보여주는,
그러면서도 기존의 편견과 타협하려는 욕망의 흔적을
조금은 모호하게 보여주는,
몽테스키외의 이 책은,
그 모든 결함에도 불구하고,
자연법이 한 순간도 발붙인 적이 없었던 저 역사적 방법\latin{historical method}에
기초하여 논의를 전개한다.
이 책의 인기만큼이나 그 사상적 영향력도 마땅히 컸어야 했으나,
실은 꽃피울 시간이 허락되지 않았으니,
이 책이 파괴하고자 했던 반대가설이 갑자기 법정에서 거리로 뛰쳐나가
과거 법정이나 대학을 뒤흔들었던 때보다 훨씬 더 격렬한 논쟁의
한복판에 서게 된 것이다.
논쟁을 새롭게 촉발시킨 저 비범한 인물은
배운 것도 없고, 그리 유덕하지도 않으며, 강한 성격의 소유자도 아니었으나,
그럼에도 불구하고 생동하는 상상력과
그의 단점 대부분을 용서하고도 남을
인간에 대한 진정어린 불타는 사랑의 힘에 의해
역사에 지울 수 없는 각인을 남겼다.
1749년부터 1762년 사이에 루소가 출간한 문헌들만큼
인간의 정신에, 지성적 사고의 모든 형태와 색조에,
막대한 영향력을 행사한 예는
우리 시대에는 전혀 볼 수 없으며, 실로 세계 전체의 역사를 통틀어 한 두 번
있을까 말까 한 것이다.
그것은 벨\latin{Pierre Bayle}과
부분적으로 영국의 로크에 의해
시작되고 볼테르에 의해 완성된 저 순수한 우상파괴적인 노력 이래
처음으로 인간의 믿음의 건축물을 새로 건설하려는 시도였다.
그것은, 단지 파괴만 일삼는 노력에 비해 모든 건설적인 노력이
언제나 갖는 우월성 외에도,
사변적인 문제에 관하여 과거의 모든 지식의 건전성을 의심하는
거의 보편적인 회의주의의 한 복판에서 출현했다는 매우 큰 장점을 지닌다.
루소의 사상 가운데 핵심적인 상징은,
사회계약의 체약자라는 영국적 옷을 입은 존재이든,
일체의 역사적 특성에서 벗어난 벌거벗은 존재이든,
가상의 자연상태에서 살고 있는 인간인 점에는 변함이 없다.
이러한 이상적 상황 하의 상상적 존재에 어울리지 않는
모든 법과 제도는 원초적 완전성으로부터 타락한 것으로 낙인찍힌다.
저 자연의 피조물이 지배했던 세상의 모습에 가까이 다가가는 모든 사회적 변화는
칭송받을 만하고 어떤 대가를 지불하더라도 달성할 가치가 있다.
이 이론은 여전히 로마 법률가들의 그것과 일치하거니와,
자연상태를 채우고 있는 일련의 환영들 중에서
로마 법학자들을 매료시킨 단순성과 조화성을 제외한 일체의
속성과 특징들은 받아들여지지 않기 때문이다.
하지만 이 이론은 말하자면 아래위가 뒤집힌 이론이다.
이제 `자연법'이 아니라 `자연상태'가 숙고의 제일가는 주제이기 때문이다.
로마인들은 기존 제도들을 주의깊게 관찰하면 그중 일부는,
그들이 어렴풋이 실감했던 저 자연의 지배의 흔적을 이미 보여주고 있는 것으로,
혹은 사법\hanja{司法}적 정화를 거쳐서 보여줄 수 있는 것으로,
골라낼 수 있다고 생각했다.
루소의 믿음은 완전한 사회질서는 오직 자연상태를 고려함으로써만
도출될 수 있다는 것이었으니, 그 사회질서는
현실세계의 상태와 전적으로 무관한 것이고 그것과 전혀 닮지 않은 것이었다.
두 견해 간의 큰 차이는, 하나는
이상적인 과거와 닮지 않았다는 이유로 현재를 몹시 그리고 대체로 부정하는 반면,
다른 하나는 과거 못지 않게 현재도 필요한 것으로 보고 현재를 무시하거나
비난하려 하지 않는다는 데 있다.
자연상태에 기초하여 건설된 저 정치철학, 예술철학, 교육철학,
윤리학, 사회철학을 여기서 일일이 분석할 필요는 없을 것이다.
이 철학은 지금도 여러 나라의 저급한 사상가들을 매혹하는 힘을 가지고 있으며,
또한 의심할 바 없이 역사적 방법에 기초한 탐구를 방해하는
대부분의 선입견들을 낳은 다소 먼 조상이기도 하다.
하지만 오늘날 수준 높은 지성인들 사이에는 이 철학에 대한 불신이 무척 깊어,
사변적 오류의 비상한 생명력에 익숙한 이들도 놀라게 할 정도이다.
아마 오늘날 가장 빈번하게 제기되는 질문은 이 철학의 가치가 무엇이냐의 문제가
아니라, 백 년 전에 이 철학이 지배적 영향력을 가졌던 원인이
무엇이었냐의 문제일 것이다.
생각건대, 그에 대한 답은 간단하다.
고대법에만 관심을 두었을 때 초래될 법한 오해를 교정하는 데에
가장 적합했을 지난 세기의 연구분야는 종교에 관한 연구였다.
그러나 그리스 종교는, 당시에 이해된 바로는, 가공의 신화에 불과한 것으로
타기시되었다.
동양의 종교는, 설령 관심을 받았을지라도, 허황된 우주생성론에
빠져있는 것으로 여겨졌다.
연구의 가치가 있는 원시적 기록은 단 하나밖에 없었다.
바로 유대인들의 초기 역사였다.
하지만 이것에 관한 연구는 당대의 편견으로 인해 저지당했다.
루소 학파와 볼테르 학파의 공통점이 하나 있다면 그것은
일체의 고대종교에 대한, 무엇보다 히브리 민족의 종교에 대한,
철저한 경멸이라 할 것이다.
주지하듯이 당대의 지식인들에게는, 모세에게서 유래했다는 제도들이
실은 신이 명령한 것도 아니고, 그렇다고 모세 이후에 성문화된 것도 아니며,
오히려 그 제도들과 모세오경 전체가
바빌론 유수에서 귀환한 후에 위조된 것에 불과하다고
주장하는 것이 일종의 명예로운 일이었다.
그리하여 사변적 망상을 방지하는 담보장치 하나를 이용할 수 없게 된
프랑스 철학자들은, 성직자들의 미신이라 여겨진 것에서 탈출하려는 열망에서,
법률가들의 미신에로 앞뒤 가리지 않고 뛰어들었던 것이다.

\para{프랑스의 자연법}
자연상태 가설에 기초한 철학에 대한 존중이 일반적으로 감소했다고는 하나,
보다 거칠고 보다 쉽게 만져지는 측면에 관한 한,
뒷마당에서는 여전히 그것의 설득력과 인기와 권력이 사라지지 않고 있다고
해야할 것이다.
전술했듯이 여전히 그것은 역사적 방법의 적대자이다.
역사적 탐구 방법에 {\small(종교적 반대는 제쳐놓고)} 저항하거나
이를 비난하려 하는 자들은
대체로 사회나 개인의 비역사적^^b7자연적 상태에 대한 의식적^^b7무의식적
믿음에 기인하는 선입견이나 악의에 찬 편견에 의해 영향받고 있음을 알 수 있다.
그러나 자연의 교리와 자연법의 교리가 에너지를 잃지 않고 있는 것은
주로 정치적^^b7사회적 경향과의 동맹관계 덕분이다.
저 교리들은 이러한 경향의 일부를 자극했고, 다른 일부는 실제로 만들었으며,
대다수의 경향에게는 표현과 형식도 제공했다.
그것들은 프랑스로부터 다른 문명세계로 지속적으로 퍼져나가는 뚜렷한 관념이
되었고, 그리하여 문명을 변화시키는 일반적 사상체계의 일부가 되었다.
물론 이것이 인류의 운명에 행사하는 영향력의 가치는 우리 시대에
뜨거운 논쟁의 대상이 되고 있고, 또한 본 논저가 논의대상으로 삼는
목적이기도 하다.
하지만, 자연상태 이론이 최대의 정치적 중요성을 가졌던 때를 돌이켜볼 때,
제1차 프랑스혁명이 무수히 낳았던 엄청난 실망감의 원인을 제공하는 데
그것이 크게
기여했음을 부인할 이는 거의 없을 것이다.
그것은 당대에 거의 보편적이었던 악한 정신적 습관, 즉
실정법에 대한 경멸, 경험에 대한 조급증,
다른 모든 추론에 앞선 선험\latin{à priori}의 우선성 등을 낳거나
강하게 자극했다.
또한 이 철학이 생각이 짧고 관찰이 부족한 사람들의 마음을
사로잡아가는 것에 비례하여, 그것은 확실히 무정부주의적으로 되는 경향을 보였다.
뒤몽\latin{Dumont}이 벤담을 위해 출간하였으며,
특별히 프랑스적인 오류를 폭로한 벤담의 문건을 담고 있는
<<무정부적 궤변>>\latin{Sophismes Anachiques}\footnote{%
이 책은 프랑스어로 먼저 출간되었으며,
벤담 사후에 ``무정부적 오류''(Anarchical Fallacies)라는 제목으로
영어판이 나왔다. 부제는
``프랑스 제헌의회가 선포한 인간과 시민의 권리 선언에 대한 검토''다.}의
얼마나 많은 부분이
프랑스어로 번안된 로마인들의 가설에서 유래한 것인지, 그리고
그 가설을 참조하지 않고는 이해되지 않는 것인지를 알면 놀라지 않을 수 없다.
또한 이 점에 관하여 혁명의 절정기에 발간된 모니퇴르\latin{Moniteur}지를
찾아보면 흥미로울 것이다.
자연법과 자연상태에 대한 호소는 시대가 어두워질수록 점점 짙어졌다.
제헌의회 시절에는 비교적 드문 편이었으나,
입법의회 시절에는 훨씬 빈번하였으며,
음모와 전쟁에 관한 논쟁으로 시끄러웠던 국민공회 시절에는
항시적으로 나타났다.

\para{인간의 평등}
자연법 이론이
근대사회에 끼친 영향을 여실히 보여주는,
그리고 이러한 영향이 얼마나 소진되기 어려운가를 보여주는
한 가지 예가 있다.
생각건대,
인간의 근본적 평등의 원리가
자연법이라는 가정\hanja{假定}에
빚지고 있음에는 의문의 여지가 없다.
바로 이 ``모든 인간은 평등하다''야말로
시간의 흐르면서 법적 명제가 정치적 명제가 된
대표적 예인 것이다.
안토니누스 황조 시대의 로마의 법학자들은
``모든 인간은 자연적으로 평등하다''\latin{omnes homines naturâ
aequales sunt}라고 단언했지만,
그들의 눈에 이것은 어디까지나 법적 공리\hanja{公理}였다.
그들이 의도한 것은
가설적인 자연법 아래에서는, 그리고 실정법이 그것에 근접하는 한에서는,
로마 시민법이 가지고 있던 사람의 신분 간의 자의적 구별이
법적으로 사라진다는 것이었다.
이 규칙은 로마의 실무가들에게 엄청나게 중요했거니와,
로마법이 자연의 법전을 따른다고 생각되는 때면 언제나
시민과 외인\hanja{外人}, 자유인과 노예,
종족\hanja{宗族}과 혈족\hanja{血族} 간의 구별이
로마의 법정에서 사라졌다.
이와 같이 명확히 자기 의견을 밝힌 로마의 법학자들은
시민법이 사변적 법유형에 비해 모자란다고 해서 사회제도를
결코 비난하지 않았고, 자연의 질서에 완전히 일치하는
인간사회가 이 세상에 존재할 수 있으리라고 믿지도 않았다.
하지만 인간의 평등 원리가 근대의 옷을 입고 등장했을 때
그 옷은 확실히 전혀 새로운 색조의 의미를 띠고 있었다.
로마 법학자들이 ``평등하다''\latin{aequales sunt}라고 썼을 때는
그 의미가 글자 그대로였지만,
근대 로마법 학자가 ``모든 인간은 평등하다''라고 썼을 때
그 의미는 ``모든 인간은 평등해야 한다\latin{ought to be equal}''였던 것이다.
자연법은 시민법과 공존하는 것이고 차츰 시민법을 흡수하는 것이라는
로마 특유의 자연법 관념은 이제 확실히 망각되었거나
적어도 이해할 수 없는 것이 되었다.
기껏해야 인간 제도의 기원, 구성, 발달에 관한 이론을 말하던 단어들이
인류가 겪고 있는 기존의 커다란 해악을 지칭하는 표현이 되기 시작했다.
일찍이 14세기 초에, 인간의 출생시 상태에 관해 말하는 당시의 언어는,
분명 울피아누스나 그의 동료들의 언어를 그대로 따라하려 했으나,
실은 완전히 다른 형태와 의미를 띠게 되었다.
왕실의 농노들을 해방시킨
완고왕\hanja{頑固王} 루이의 유명한 왕령의 서문은
로마인들의 귀에는 생경하게 들렸을 것이다.
``자연법에 따르면 모든 사람은 자유롭게
태어나야 한다\latin{ought to be born free}.
그런데 아주 오래 전에 우리 왕국에 도입되어
지금까지 이어지고 있는 관행과 관습으로 인하여,
그리고 어쩌면 그들 선조들이 저지른 범죄행위로 인하여,
우리 평민들 가운데 다수가 예속상태에 떨어져 있다. 그리하여 우리는 \ldots'' 등.
이것은 법규칙이 아니라 정치적 도그마의 선언이었다.
그리고 이때부터 인간의 평등을 프랑스 법률가들은
그것이 마치 그들 학문의 저장고에 보관되어온 정치적 진리인 양 말해왔다.
자연법의 가설에서 연역되어 나온 모든 다른 것과 마찬가지로,
그리고 자연법 그 자체에 대한 믿음과 마찬가지로,
그것은 맥없이 승인되었고 여론이나 실무에 거의 영향을 끼치지 못했다.
그러나 그것이
법률가들의 점유에서 벗어나 18세기 문필가들과
그들에게 감화된 대중들의 점유로 넘어가면서,
이들의 신념을 표현하는 가장 두드러진 교리가 되었고
나아가 모든 신념을 요약하는 교리로 간주되었다.
하지만 그것이 1789년 사건 이후 마침내 권력을 획득하게 된 것은
프랑스 안에서의 인기에만 기인한 것이 아니었을 것이다.
18세기 중엽에 그것은 미국으로 건너갔던 것이다.
당시 미국 법률가들, 특히 버지니아 주의 법률가들은
당대 영국인들의 것과는 사뭇 다른 지식 계통을 가지고 있었던 것으로 보인다.
대륙 유럽의 법문헌들에서 유래한 것일 수밖에 없는 것들이
다수 포함되어 있었던 것이다.
제퍼슨의 저술을 조금만 들여다보더라도
프랑스에서 유행하던 반쯤은 법적이고 반쯤은 대중적인 견해들로부터
그가 강하게 영향받고 있었음을 알 수 있을 것이다.
의심할 여지 없이, 미국에서의 일련의 사건들을 이끌었던
그와 기타 식민지 법률가들은
프랑스 법률가들의 특유한 관념에 공감하였고,
``모든 인간은 평등하게 태어난다''라는 특히 프랑스적인 가정\hanja{假定}을
영국인들에게 보다 친숙한 ``모든 인간은 자유롭게 태어난다''는 가정과
결합하였으니,
이는 독립선언문의 첫 몇 줄에 잘 나타나있다.
독립선언문의 이 문장은 우리가 다루는 교리의 역사에서
가장 중요한 문장 중의 하나이다.
미국의 법률가들은 이렇게 인간의 근본적 평등을 무엇보다 강하게 긍인함으로써
그들 조국의 정치적 운동에 추동력을 부여했다.
영국에서는 영향력이 덜 하였으나, 영국은 아직
그 힘을 다 써버린 상태가 아니다.
그런데 그밖에도 그들은 저 교리를 수용한 본국인 프랑스에 그것을 되돌려주어
훨씬 더 큰 에너지를 만들어냈고 일반적 수용과 존중이 훨씬 더 강하게
주장될 수 있도록 했다.
제1차 제헌의회의 보다 신중한 정치가들조차
저 울피아누스의 명제를,
마치 그것이 인류의 직관과 본능에 동시에 기초하고 있다는 듯이,
반복하여 외쳤다.
``1789년의 원리들'' 가운데 그것은 가장 덜 공격받은 것이고,
근대의 여론에 가장 큰 영향을 끼친 것이며,
여러 사회의 헌정과 여러 국가의 정치에 가장 근본적인 변화를
약속하고 있는 것이다.

\para{국제법}
자연법의 가장 큰 기여는 근대 국제법과 근대 전쟁법을 탄생시키는 데서 수행되었다.
하지만 여기서는 이 영역에 대한 자연법의 영향을
그 중요성에 비해 훨씬 소략하게 고려하는 것으로 만족할 수밖에 없다.

국제법의 기초를 이루는 공준\hanja{公準} 중에,
또는 국제법의 초기 건설자들에서 유래한 상징을 많이 담고 있는 공준 중에,
무척 중요한 것들이 두 세 가지 있다.
그 중 첫 번째는 결정가능한 자연법이 존재한다는 입장이다.
그로티우스와 그 후계자들은 로마인들에게서 직접 이 가정을 가져왔으나,
결정의 양상에 관해서는
로마 법학자들과 차이가 크고 또한 그들 상호간에도 차이가 크다.
문예부흥 이후 대거 등장한 공법학자\latin{publicist}들의 대다수는
자연과 자연법에 대한 정의\hanja{定義}를 다루기 쉽게 새로 제공하려는
야심을 가지고 있었다.
공법학자들의 긴 행렬이 이어지면서 저 개념에는 첨가물이 대거 덧붙여졌거니와,
이는 윤리학의 거의 모든 이론들에서 따온 관념의 조각들로 이루어진 것이었으니,
이제 윤리학이 공법의 학파들을 장악하기에 이르렀음에 틀림없다.
자연상태의 필수적 성격으로부터 자연의 법전을 도출해내려는
그 모든 노력에도 불구하고, 그 결과물이란 것이
로마 법률가들의 진술을 묻지도 따지지도 말고 그대로 수용했더라면
얻을 수 있었을 결과물과 별반 다르지 않다는 점은
저 개념이 본디 역사적 성격의 개념이란 것에 대한 뚜렷한 증거인 것이다.
조약에 관한 국제법을 제쳐둔다면,\footnote{`약정은 지켜져야 한다'는
그로티우스의 계약법 이론은 로마법보다는 오히려 교회법의 영향을
더 강하게 받았다.}
국제법 체계의 얼마나 많은 부분이 순수한 로마법으로 만들어져 있는지
놀라울 지경이다.
로마 법학자들의 법리가 만민법\latin{ius gentium}과 조화된다고 생각되면
어디서나,
그것이 아무리 순수히 로마적 기원을 가진 것이라 할지라도,
공법학자들은 그것을 빌려올 구실을 발견했던 것이다.
또한 우리는
이렇게 파생된 이론들이
원래의 관념이 가지고 있던 약점을
그대로 안고 있다는 점도 관찰할 수 있다.
대부분의 공법학자들의 사고양식은 여전히 ``혼합적''인 것이었다.
이들의 저술을 연구함에 있어서 항상 부딪치는 큰 어려움은
그들이 논하는 것이 법인지 아니면 도덕인지,
그들이 기술하는 국제관계의 상태가 현실의 것인지 아니면 이상적인 것인지,
그들이 진술하는 것이 존재에 관한 것인지 아니면
그들이 생각하는 바람직한 당위에 관한 것이지를 판가름하는 일이다.

자연법이 국가들 사이에서\latin{inter se} 구속력을 가진다는 가정이
국제법의 근저에 놓여있는 두 번째 공준이다.
이 원리에 대한 일련의 주장과 수용은 근대 법학의 유아기로 거슬러올라가며
추적할 수 있을 것인데,
일견 그것은 로마인들의 가르침에서 직접 추론한 것으로 보인다.
사회의 국가적 상태와 자연적 상태의 차이는
전자에는 입법자가 뚜렷이 존재하지만 후자에는 없다는 것이므로,
만약 다수의 \hemph{단위들}\latin{units}이
어떤 공통의 주권자나 정치적 상급자에게도
복종하지 않는다고 인정되면 그들은 자연법의 지배 상태로 되돌아간다고 볼 수 있다.
국가들이 바로 그러한 단위들이다.
국가의 독립성 가설은 공통의 입법자 관념을 배제하거니와,
몇몇 이론에 의하면,
따라서 국가들은 자연의 원시적 질서에 복종한다는 관념이 도출되는 것이다.
그에 대한 대안은 독립된 공동체들 간에는 어떠한 법도 존재하지 않는다는 것이나,
이러한 무법\hanja{無法} 상태야말로
로마 법학자들의 성정이 끔찍히도 싫어했던 진공상태인 것이다.
물론, 로마 법률가들은 시민법이 추방당한 어떤 영역에 맞딱뜨리면
즉시 그 빈 공간을 자연의 명령으로 채워넣었을 것이라고 추정할 만한
외견상의 이유는 존재한다.
하지만 어떤 결론이 우리의 눈에 아무리 확실하고 자명해 보일지라도,
역사의 어느 순간에
실제로 그러한 결론이 도출되었을 것이라고 가정하는 것은 위험한 일이다.
현존하는 로마법 텍스트 가운데,
로마 법학자들이 자연법을 독립된 국가들 간에 구속력을 갖는 것으로
믿었다는 증거는, 내가 알기로는, 전혀 발견된 바 없다.
로마 제국의 시민들은,
자기들 국가의 통치영역이 문명의 영역과 경계를 같이 한다고 생각했기에,
국가들이 모두 동등하게 자연법에 복종한다는 것은,
설령 그런 생각을 해봤다 할지라도,
기껏해야 유별난 사변의 극단적 결과 쯤으로 치부했을 것이 분명하다.
사실 근대 국제법은,
로마법의 후손임에는 틀림없으나,
비정상적인 계통을 거쳐 로마법에 연결될 뿐이라고 해야할 것이다.
로마 법학의 근대 초기 해석자들은
만민법\latin{ius gentium}의 뜻을 잘못 이해하여,
로마인들이 국제 거래를 규율하는 법체계를 그들에게 물려주었다고
서슴없이 믿었다.
이 ``만민법''\latin{law of nations}은 처음에는 강력한 경쟁상대들과
권위를 두고 싸워야 했고,
유럽의 상황은 오랫동안 그것의 보편적 수용을 방해했다.
그러나 차츰 서구 세계는 로마법 학자\latin{civilian}들의 저 이론에
보다 우호적인 형태로 재편되어갔고,
상황의 변화와 더불어 경쟁적 이론들의 신용은 땅에 떨어졌다.
마침내, 특별한 행운이 겹쳐,
아얄라\latin{Balthazar Ayala}와 그로티우스는 그것에 대한 유럽의 열광적인 동의를
얻어낼 수 있었으니, 다양한 유형의 장엄한 계약이 체결될 때마다 이 동의는
계속해서 갱신되어갔다.
승리의 주역이라 할 수 있는 저 위인들이
그것을 완전히 새로운 기초 위에 놓으려 시도했음은 말할 필요도 없거니와,
이러한 재배치 과정에서 그 구조를
많이 바꾸었음---그러나 일반적으로 알려진 것보다는 훨씬 덜 바꾸었다---도
의문의 여지가 없다.
안토니누스 황조 시대 법학자들이 만민법과 자연법이 동일하다고 말한 것에
착안하여
그로티우스는 그의 직접적 선학들 및 후학들과 더불어
자연법에 특별한 권위를 부여하였으니,
그 권위는 만약 ``만민법''이 당시 모호한 의미를 갖지 않았다면
아마 결코 주장될 수 없었을 것이다.
그들은 자연법이 국가들의 법전임을 스스럼없이 주장했다.
그리하여
오로지 자연 개념에 대한 숙고로부터 도출된 것으로 여겨진 규칙들을
국제법 체계에 접목시키는 과정이 시작되었고,
이 과정은 거의 우리 시대까지 지속되고 있다.
또한 이는 인류에게 대단히 중요한 현실적 결과 하나를 낳았으니,
그것은 근대 초기 유럽의 역사에 전혀 알려지지 않은 것은 아니나
그로티우스 학파의 법리가 지배적 위치를 차지하면서 비로소
명백하게 그리고 보편적으로 인식된 것이다.
만약 국가들의 사회가 자연법의 지배를 받는다면,
그 사회를 구성하는 원자들은 절대적으로 평등해야 한다.
자연의 홀\hanja{笏} 아래서 모든 인간이 평등하듯이,
국가들 간의 상태가 일종의 자연상태라면 국가들도 평등하다.
크기와 힘이 서로 다르더라도 독립된 공동체들은
국제법의 관점에서 모두 평등하다는 이 명제는,
시대마다의 정치적 경향에 의해 위협받아온 것도 사실이지만,
대체로 인류의 행복에 기여해왔다.
문예부흥 이후 공법학자들이
자연의 존엄하신 주장으로부터 국제법을 도출하지 않았더라면,
저 법리는 결코 굳건한 반석 위에 설 수 없었을 것이다.

전체적으로 볼 때, 전술했듯이,
단순히 로마 만민법이라는 고대 지층에서 가져온 요소들에 비해
그로티우스 시대 이래 국제법에 추가된 것이
얼마나 작은 비율인지 놀라울 정도이다.
영토의 취득은 언제나 국가의 야심을 자극해왔거니와,
이러한 취득을 규율하는 규칙들은,
그 야심이 너무나 자주 불러오는 전쟁을 억제하는 규칙들과 더불어,
만민법상의\latin{jure gentium} 물건의 취득 방식에 관한 로마법을
단순히 옮겨적은 것에 지나지 않는다.
앞서 설명했듯이,
옛날 법학자들은
로마 인근의 여러 부족들을 관찰하여 그들 사이에 지배적인 관행에서
공통의 요소를 추출함으로써
이러한 취득 방식을
획득했다.
그 기원에 따라
``모든 민족들에 공통적인 법''으로 분류된 이 방식들을
후대의 법률가들은
그 단순성에 착목하여 자연법이라는 보다 최근의 개념과 어울린다고 생각했다.
그리하여 그들은 근대 만민법\latin{law of nations}으로 이어지는
길을 열었으니, 결과적으로
\hemph{영토}\latin{dominion}와 그것의 성격, 한계, 취득방식 및
안전하게 지키는 방식에 관한 국제법 분야는
순수한 로마 물권법---즉,
안토니누스 황조 시대 법학자들이 자연상태와의 모종의 일치를 보인다고
생각했던 바로 그 로마 물권법---인 것이다.
국제법의 이 분야가 적용될 수 있으려면,
주권자들 사이의 관계가
로마의 소유권자 집단의 성원들처럼 될 필요가
있었다.\footnote{로마에서는
원칙적으로 가부장(pater familias)들만이 소유권자가 될 수 있였다.}
이것이 국제법전의 초입에 놓여있는 또 하나의 공준인 것이다.
또한 이것은 근대 유럽 역사의 첫 몇 세기 동안은 지지받지 못한 공준이었다.
이것은 두 개의 명제로 분해될 수 있거니와,
``주권은 영토적이다'' 즉,
지구 표면의 한정된 부분에 대한 소유권을 갖는다는 명제와,
``주권자들 사이에서는 주권자가 당해 국가의 영토의,
\hemph{최고}\latin{paramount} 소유자가 아니라,\footnote{봉건제 하의
중층소유권 이론을 부정한다는 의미이다.}
\hemph{절대적}\latin{absolute} 소유자로 간주된다''는 명제가
그것이다.

오늘날 국제법 학자들은
형평과 상식에 기초한
그들의 국제법 법리들이
근대 문명의 모든 단계에서 쉽게 추론되어 나올 수 있다고
암묵적으로 전제한다.
이 전제는, 국제법 이론의 몇몇 현실적 결함을 감추고 있기는 하지만,
근대 역사의 대부분의 시기에 있어 결코 주장될 수 없는 전제이다.
국가들의 문제에 관하여 만민법\latin{ius gentium}의 권위가
전혀 도전받지 않았다는 것은 사실이 아니다.
오히려 그것은 오랫동안 몇몇 경쟁적인 이론들과 투쟁해야 했다.
또한 주권의 영토적 성격이 항상 인정되어왔다는 것도 사실이 아니다.
로마의 영토가 해체된 이후 오랫동안 인간의 정신은 그러한 이론과
조화될 수 없는 관념에 의해 지배되었던 것이다.
사물의 옛 질서와 그것에 기초한 견해가 쇠퇴하고 나서야,
새로운 유럽과 그것에 부합하는 새로운 관념이 등장하고 나서야,
국제법의 저 두 가지 공준이 보편적으로 받아들여질 수 있었다.

\para{영토주권}
근대사라고 불리는 것의 대부분의 기간 동안
``영토주권''\latin{territorial sovereignty}이라는 관념이
존재하지 않았다는 점을 명심할 필요가 있다.
주권은 지구의 일부분 또는 보다 세분된 영역에 대한 영유권과
아무런 관련이 없었던 것이다.
이 세계는 너무나 오랜 기간 동안 제국 로마의 그림자 아래 존재해왔기 때문에,
제국의 영토로 편입된 광대한 지역이 한때는,
외부의 간섭으로부터 면제되고 국가 간에 평등한 권리를 요구하는
다수의 독립국가들로 나뉘어져 있었다는 사실을 망각해버렸다.
만족\hanja{蠻族}들의 침입이 진정된 이후,
주권의 개념에는 다음과 같이 양면성이 있었던 것으로 보인다.
우선 그것은 ``부족주권''\hyphlatin{tribe-sovereignty}이라고 부를 수 있는
형태를 띠고 있었다.
물론
프랑크 족, 부르군드 족, 반달 족, 롬바르드 족, 서고트 족은
그들이 차지한 영토의 주인이었거니와,
이는 몇몇 지역의 지리적 명칭으로도 남아있다.
하지만 그들은 영토적 점유에 기초한 어떤 권리도 주장하지 않았으니,
사실 영토적 점유를 중요하게 여기지도 않았다.
그들은 삼림과 초원에서 가져온 전통을 계속 유지했던 것으로 보이며,
여전히 가부장적 사회의 유목 무리로서
단지 생계 수단을 제공하는 토지 위에 잠시
캠프를 치고 있을 뿐이라는 견해를 가지고 있었던 듯하다.
알프스 너머 갈리아 지방의 일부와 게르마니아 지방의 일부는
이제 프랑크 족이 사실상 지배하는 나라---오늘날의 프랑스---가 되었다.
하지만 클로비스의 후손인 메로빙거 왕조의 군장\hanja{君長}들은
프랑스의 왕이 아니었다. 그들은 프랑크 족의 왕이었던 것이다.
영토적 권리를 뜻하는 용어가 알려져 있지 않았던 것은 아니나,
처음에는 단지 부족이 점유한 땅의 \hemph{일부}를
통치하는 통치자를 지칭하는 편리한 수단의 하나로만 사용되었던 듯하다.
부족 \hemph{전체}의 왕은 그의 백성들의 왕이었지,
그의 백성들이 살고 있는 여러 토지의 왕은 아니었다.
이러한 특수한 주권 관념에 대한
대안으로---이 논점은 매우 중요한데---보편적 지배의 관념이
존재했던 것으로 보인다.
군주가 부족원들의 군장이라는 관계를 청산하고
자기자신을 위해 새로운 주권 형태를 만들어내고자 원했을 때,
받아들일 만한 선례로서 그에게 주어진 것은 로마 황제들의 지배형태였다.
흔히 쓰이는 인용구를 차용하자면, 그는
``황제가 아니면 아무 것도 아닌''\latin{aut Caesar aut nullus} 것이
된 것이다.\footnote{`전부 아니면 전무'(all or nothing)의 뜻으로 종종 쓰인다.}
비잔틴 황제의 완전한 대권\hanja{大權}을 주장하거나, 아니면 아무런 정치적 지위를
갖지 않는다는 것이다.
우리 시대에는 새로운 왕조가 폐위된 왕조의 기존의 권리를 지워버리고자 할 때,
\hemph{영토}가 아닌 \hemph{인민}을 지칭하는 용어를 사용한다.
그리하여 오늘날에는 프랑스인의 황제들과 왕들이 존재하고,
벨기에인의 왕이 존재한다.
그러나 우리가 다루는 저 시대에는 다른 대안이 사용되었다.
더 이상 부족의 왕으로 불리고 싶지 않은 군장은 세계의 황제를 자처해야 했다.
따라서, 세습 궁재\hanja{宮宰}들은 그들이 이미 오래 전부터 사실상
무력화시켰던 국왕들과 더 이상 타협하고 싶지 않았을 때,
스스로를 단순히 프랑크 족의 왕이라고 부르길 원하지 않았다.
이 호칭은 폐위된 메로빙거 왕조에 속하던 것이기 때문이다.
그렇다고 프랑스의 왕이라는 호칭도 쓸 수 없었다.
이 호칭은,
비록 알려져 있지 않은 것은 아니었으나, 존엄성을 갖지 못하던 것이기 때문이다.
그리하여 그들은 보편 제국을 지향하는 호칭을 사용했다.
그들의 동기는 크게 오해의 대상이 되었다.
최근의 프랑스 학자들은 샤를마뉴를 시대를 앞서간 인물로 그려내는 것을
당연시하거니와,
계획을 추진하는 에너지에 있어서는 물론이고
그의 계획의 성격 또한 그러하다는 것이다.
어떤 사람이 그의 시대를 앞서갈 수 있는지의 여부는 차치하고라도,
한 가지 분명한 것은, 무한한 영토를 추구했던 샤를마뉴는
그 시대의 특유한 관념이 그에게 허락한 유일한 길을 따랐을 뿐이라는 점이다.
지성을 중시하는 그의 탁월한 능력에는 이론\hanja{異論}이 없지만,
이는 그의 행위 때문에 그러한 것이지, 그의 이론 때문에 그러한 것이 아니다.

이러한 독특한 견해는 샤를마뉴의 세 명의 손자들 사이에서
상속재산이 분할되었을 때에도 그대로 유지되었다.
대머리 샤를, 루이, 그리고 로타르는 이론적으로는 여전히
로마 제국의---이 용어를 사용하는 것이 적절하다면---황제들이었다.
동로마황제와 서로마황제가 각각 법적으로는 세계 전체의 황제이지만
사실은 그 절반씩을 통치했던 것처럼,
저 세 명의 카롤링거 황제들도 권력은 제한되어 있지만
법적 타이틀은 무제한적이라 여겼던 것으로 보인다.
비만왕\hanja{肥滿王} 샤를의 죽음으로 또다시 분할이 이루어진 이후에도
이러한 주권의 보편성 관념은
오랫동안 황제의 지위와 관련되어 있었고,
실로 게르만 제국이 존속하는 한 그것과 완전히 단절될 수 없었다.
영토주권---주권을 지구 표면의 한정된 부분의 점유와 관련짓는 견해---은
명백히 봉건제도\latin{feudalism}의 자손, 그것도 뒤늦은 자손이었다.
이는 선험적으로 예상할 수 있거니와,
봉건제도는 역사상 최초로 인적\hanja{人的} 의무를, 그리고 결과적으로
인적 권리를, 토지 소유와 연결지었던 것이다.
그것의 기원과 법적 성격에 관한 적절한 견해가 무엇이건 간에,
봉건 구조를 생생하게 묘사하는 가장 좋은 방법은
그 밑바닥부터 시작하는 것이다.
우선 봉신\hanja{封臣}의 역무를 설정하고 제한하는 한 조각 토지에 대한
봉신의 관계를 고려하고,
그 다음 차츰 상위의 수봉\hanja{受封}관계로 올라가면서 원의 반경을 좁혀나가,
마침내 체제의 정점에 이르는 방법인 것이다.
암흑시대 후기 동안 그 정점이 정확히 어디에 위치했는지는 확인하기가 쉽지 않다.
아마도, 부족주권의 관념이 실제 쇠퇴한 곳이라면 어디서나,
그 정점은 서구 세계의 황제로 여겨지던 자들에게 언제나 주어졌을 것이다.
그러나 머지않아 제국의 권위가 먹혀드는 영역이 대폭 축소되자,
그리고 황제들이 얼마 남지 않은 그들 권력을
독일 지역과 북이탈리아 지역에 집중시키자,
과거 카롤링거 제국의 나머지 모든 지역에서 최고 봉건 수장들은 사실상
상급자가 없는 상태가 되었다.
차츰 그들은 새로운 상황에 적응해갔고,
불입\hanjalatin{不入}{immunity}의 사실상태는
마침내 종속\hanja{從屬}의 법이론을 가려버렸다.
그러나 이러한 변화가 쉽게 일어나기 어려웠을 것을 알려주는 여러 징후가 존재한다.
사실, 사물의 본성상 어딘가에 최고 권력이 반드시 존재해야 한다는 관념 탓에,
세속적 최고성을 로마 교황청에 부여하는 경향이 점점 커지고 있었던 것이다.
관념 혁명의 최초의 단계는
프랑스의 카페 왕조에 의해 완성된다.
그 이전까지는,
이제 카롤링거 제국에서 갈라져나온 몇몇 대\hanja{大}영지의 보유자들이
스스로를 공작이나 백작이 아닌 왕으로 자처하기 시작하고 있었다.
그런데 파리와 그 인근에 한정된 영토를 가진 저 봉건군주가
옛 왕가로부터 \hemph{프랑스인의 왕}이라는 타이틀을 찬탈하면서
중요한 변화가 일어나기 시작했다.
위그 카페와 그 후계자들은 전혀 새로운 의미의 왕들이었으니,
남작의 그의 영지에 대한 관계, 봉신의 그의 자유보유지에 대한 관계와
동일한 관계에서 프랑스 토지에 대한 주권자였던 것이다.
비록 오랫동안 저 옛 부족적 호칭이 통치왕가의 공식 라틴어 호칭으로 남아있었으나,
고유어 호칭에서는 빠르게 \hemph{프랑스의 왕}으로 변모되어갔다.
프랑스에서의 국왕 지위의 형식은 다른 곳에서 동일한 방향으로 일어나고 있던
변화를 뚜렷이 촉진시키는 결과를 가져왔다.
앵글로색슨 왕가들의 왕은 부족적 군장과 영토적 주권자 사이의
중간지대에 머물러 있었는데,
노르만 왕조 군주들의 권력은 프랑스 왕의 그것을 본받아 명백히
영토적 주권자의 모습을 띠었다.
이후 건설되거나 공고화된 모든 영토는 이러한 후대의 모델에 입각하여 형성되었다.
스페인, 나폴리, 그리고 이탈리아 자유도시들의 폐허 위에 건설된
군주국들은 모두 영토적 주권을 가진 통치자들의 지배에 놓였다.
부연컨대, 베네치아가 이 견해에서 저 견해로 옮겨가면서 점진적으로
타락해간 것만큼 이상한 일도 별로 없을 것이다.
해외정복을 시작할 때의 베네치아 공화국은
다수의 피지배 속주들을 통치하는
로마 국가 체제의 예시의 하나로
스스로를
간주했었다.
그로부터 한 세기가 지난 후의 베네치아는
이탈리아와 에게해의 점유지들에 대해
봉건영주의 권리를 주장하는 주권체로 보여지기를
바라게 되었던 것이다.

\para{국제법}
주권이라는 주제에 관한 대중의 관념이 이러한 근본적인 변화를 겪고 있던 동안,
우리가 오늘날 국제법이라고 부르는 것을 대신하던 체계는
오늘날의 그것과 형식에 있어서도 달랐고 원리에 있어서도 불일치했다.
유럽 중에서도 로마^^b7게르만 제국에 속하는 드넓은 지역에서의
국가들 간의 연합관계는 제국칙령이라는 복잡하고 불완전한 메커니즘에 의해
규율되었다.
우리에게는 놀랍게 보일지 몰라도, 당시 독일지역 법률가들 사이에서는
국가들 간의 관계가 제국 내부에서든 바깥에서든
만민법\latin{ius gentium}에 의해서가 아니라,
황제를 중심으로 하는 순수한 로마법에 의해
규율되어야 한다는 생각이 널리 선호되었다.
이 법리는, 우리의 예상과는 달리,
제국 바깥의 나라들에서도
그다지 확신에 찬 거부의 대상이 되지 못했다.
그러나 실제적으로는, 유럽의 나머지 지역에서는
봉건제의 지배복종관계가 공법의 대체물을 제공하고 있었다.
그리고 봉건제가 쇠퇴하거나 모호해지자 그 배후에서는,
적어도 이론적으로는 최고 규율권력이 교회 수장의 권위에 속한다는 이론이
모습을 드러냈다.
하지만 봉건권력도 교회권력도 15세기에 이르면, 아니 이미 14세기부터,
빠르게 쇠퇴하고 있었음이 확실하다.
또한
오늘날의 전쟁의 구실이나 동맹의 동기로 공언된 것들을 살펴보면,
조금씩 옛 원리들이 추방되고 있었고,
그 대신 나중에 아얄라와 그로티우스에 의해 조화되고 공고화될 견해들이
비록 조용하고 느리지만
괄목할만한 진전을 이루고 있었음을 알 수 있다.
저 모든 권위가 융합되었다면 모종의 국제관계 체계가 진화되어 나올 수 있었을까,
그리고 그 체계는 그로티우스의 체계와 중대한 차이를 갖는 것이었을까 따위는
오늘날 우리로서는 알 수 없거니와, 실로 종교개혁이 한 가지를 제외하고는
모든 잠재적 가능성을 파괴해버렸기 때문이다.
독일지역에서 시작된 종교개혁으로 제국의 제후들은 건널 수 없는 깊은 골을
사이에 두고 분열되었고, 최고 권력의 황제라도
이 골을 메울 수가 없었다.
비록 황제가 중립적이었다 할지라도 그러했을진대,
하물며 황제는 종교개혁에 반대하는 교회의 입장에 동조해야 했다.
교황도 동일한 곤경에 처해 있었음은 말할 나위도 없다.
그리하여 분쟁당사자들 사이에서 중재의 역할을 담당해야 할 저 두 권위는
그들 스스로가 국가들 간의 분열에서 한 쪽 당파를 대표하는 수장들이 되어버렸다.
이미 허약해진 봉건제도는 공법관계의 원리로서의 신망을 잃어버려,
종교적 당파성에 대항할만한 어떠한 안정적 결합도 제공할 수 없었다.
거의 카오스에 가까운 이러한 공법의 상황 하에서,
로마 법학자들이 지지했을 법한 그러한 국가체체에 관한 견해만이
유일하게 남게 된 것이다.
그로티우스가 보여준 저 견해의 겉모습과 조화성과 탁월성은
사실 당대의 지식인이라면 누구나 알고 있었다.
그러나 <<전쟁과 평화의 법>>\latin{De Jure Belli et Pacis}의 경이로움은
그것이 신속하고도 완전하게 그리고 보편적으로 성공을 거두었다는 데 있다.
30년전쟁이 가져온 전율, 고삐풀린 군인들의 방종이 불러온 무한한 공포와 연민,
이런 것들도 분명 그것의 성공을 어느 정도 설명할 수 있겠지만,
이것만으로 충분한 설명이 되지는 못한다.
만약 그로티우스의 저 위대한 저서에서 스케치된 국제관계의 건축물의 설계도가
이론적으로 완벽한 모습을 띠지 않았다면,
저 저서는 법률가들에 의해 버림받고
정치인들과 군인들에 의해 무시당했을 것이라는 점은
당대의 관념을 깊이 천착해들어가지 않더라도 쉽게 알 수 있다.

\para{그로티우스의 체계}
말할 것도 없이,
그로티우스의 체계가 갖는 사변적 완벽함은 우리가 논의해온
영토주권의 관념과 밀접하게 관련되어 있다.
국제법의 이론은 국가들이 그들 상호간의 관계에서는 자연상태에 있다고 가정한다.
그러나
그 근본전제에 의하면,
자연적 사회를 구성하는 원자들은
상호간에 고립되어 있어야 하고
독립되어 있어야 한다.
만약
약하게라도 그리고 가끔씩이라도
그들을 연결시켜주는 상위 권력이 존재하여
공통의 주권자임을 주장한다면,
바로 그 공통의 주권자 개념으로부터 실정법 관념이 도입될 것이고
자연법 관념은 배제될 것이다.
따라서 제국 수장의 보편적 영주권\hanja{領主權}이
순 이론적으로라도 받아들여졌다면, 그로티우스의 노력은 헛수고가 되었을 것이다.
근대 공법이론과 지금까지 그 발달과정을 서술해온 주권개념 간의 접점이
이것만 있는 것은 아니다.
전술한대로, 국제법의 분야 중에는
그 전부를 로마 물권법에서 가져온 분야들이 있다.
이로써 무엇을 추론할 수 있는가?
주권의 관념에 내가 서술했던 변화가 일어나지 않았다면---주권이
지구의 한정된 부분에 대한 소유권과 관련하여 관념되지 않았다면,
다시 말해, 영토주권이 되지 않았다면---그로티우스 이론의 세 부분은
적용 불가능한 것이 되었으리라는 것이다.\footnote{그로티우스의
<<전쟁과 평화의 법>>은 모두 세 권으로 구성된다.}



\end{document}
