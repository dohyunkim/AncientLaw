\documentclass[b5paper]{book}
\usepackage{geometry}
\usepackage{emptypage}
\usepackage[hangul]{kotex}
\usepackage{hyperref}

\ifluatex
  \defaultfontfeatures+{Renderer=HarfBuzz}
  \registerpunctuations{`-,`[,`]}
\else
  \hangulhyphens
\fi
\setmainhangulfont{Noto Serif CJK KR}
  [Script=Hangul, Language=Korean, AutoFakeSlant]
\setsanshangulfont{Noto Sans CJK KR}
  [Script=Hangul, Language=Korean, AutoFakeSlant]

\makeatletter
\def\@makechapterhead#1{%
  \vspace*{50\p@}%
  {\parindent \z@ \raggedright \normalfont
    \ifnum \c@secnumdepth >\m@ne
      \if@mainmatter
        \huge\bfseries \@chapapp\space \thechapter
        \par\nobreak
        \vskip 18\p@
      \fi
    \fi
    \interlinepenalty\@M
    \linespread{1.26}%
    \Huge \bfseries #1\par\nobreak
    \vskip 40\p@
  }}
\makeatother

\renewcommand\chaptermark[1]{\markboth{\textit{#1}}{}}

\def\para#1{\leavevmode
  \marginpar{\linespread{1.1}\sffamily\footnotesize \raggedright
    #1}\markright{#1}\ignorespaces}
\def\hanja#1{\begingroup\scriptsize #1\endgroup}
\ifluatex
  \def\latin#1{\ifnum\lastskip=0 \penalty50 \hskip0pt plus.25pt minus.15pt\fi
    \begingroup\hangulpunctuations=0
    \footnotesize #1\hangulpunctuations=1 \endgroup}
\else
  \def\latin#1{\ifnum\lastskip=0 \penalty50 \hskip.5pt plus.4pt minus.2pt\fi
    \begingroup\latinmarks\footnotesize #1\endgroup}
\fi
\def\hemph#1{\begingroup\sffamily\bfseries #1\endgroup}
\def\hanjalatin#1#2{\hanja{#1}\hskip.1em plus.05em minus.02em\latin{#2}}
\def\paren#1{\begingroup\small(#1)\endgroup}

\tracinglostchars=3

\linespread{1.4}
\skip\footins=8pt plus 8pt minus 4pt
\footnotesep=9.5pt

%\includeonly{ch9}

\begin{document}

\title{고대법\\
\large 그것은 사회의 초기 역사와 어떤 관련이 있으며\\
근대 관념에 대해서는 어떤 관계를 가지는가}
\author{헨리 섬너 메인 지음\\
김 도현 옮김}
\date{1906 (1861, 1884)}

\frontmatter

\maketitle
\tableofcontents

\mainmatter


\chapter{고대 법전}

우리가 알고 있는 가장 유명한 법체계는 법전과 함께 시작해서
법전과 함께 끝난다.\footnote{시작은 12표법, 끝은 로마법대전을 뜻한다.}
로마법의 해설자들은 그들의 법체계가 \wi{12표법}\latin{Twelve Decemviral Tables}에 기초하고 있다는,
따라서 성문법에 기초하고 있다는 취지의 말을 그들의 역사 내내 시종일관 해왔다.
한 가지 예외를 제외하면,\footnote{%
  사용취득(usucapio)에 관한 시민법을 말하는 듯하다.
  본서 제8장의 \hyperlink{usucapio}{사용취득}에 관한 설명 참조.}
12표법 이전으로 거슬러 올라가는 제도로 로마에서 인정된 것은 없었다.
로마법이 법전의 후예라는 이론, 영국법은 기억할 수 없는
옛 불문\hanja{不文}의 전통에 기원한다는 이론은
로마법의 발달이 왜 영국법의 발달과 달랐는지를 설명하는 주요 이론들이다.
두 이론 다 사실과 정확히 들어맞지는 않지만, 각각 대단히 중요한 결과들을 낳았다.

\para{원초적 법관념}
12표법의 공표가 법의 역사를 다루는 출발점이 될 수 없음은 말할 것도 없다.
문명화된 민족이면 거의 다 고대 로마의 법전과 비슷한 것을 가지고 있었다.
또한 로마와 헬레니즘 세계에 관한 한, 비교적 서로 가까운 시대에
그러한 법전이 두 세계에 널리 확산되었었다.
그것들은 무척 유사한 상황에서 등장했고, 우리가 아는 한
무척 유사한 원인으로 만들어졌다.
많은 법현상들이 이들 법전에 시기적으로 앞서거나 뒤따랐음은 말할 것도 없다.
적지 않은 문헌기록들이 남아있어 법의 초기 현상들을 우리에게 알려준다.
하지만 언어학이 산스크리트 문헌을 완전히 분석해내기 전까지는
우리에게 주어진 가장 좋은 인식 원천은 그리스의 호메로스의 시임에 틀림없다.
물론 이들은 실제 사건들을 기록한 역사로서가 아니라,
\wi{호메로스}가 알고 있던 사회의 상태를 기술한 것으로,
그러나 완전히 이상화시키지 않고 기술한 것으로 읽어야 할 것이다.
영웅시대의 어떤 특징이나 전사들의 용기, 신들의 힘 따위가
시적 상상력에 의해 과장됐을 수 있지만, 도덕적 또는 형이상학적 관념에 의해
그의 시가 오염되었다고 믿을 이유는 없다.
도덕이나 형이상학은 아직 의식적인 고찰의 주제가 아니었기 때문이다.
이런 점에서, 비슷한 초기 시대를 다룬다고 하면서
철학적 또는 신학적 영향 하에서 만들어진 후대의 문헌들보다
호메로스의 시가 훨씬 더 신뢰할 만하다.
법개념의 초기 형태를 발견하려는 우리에게 그것들은 더없이 소중하다.
법학자에게 이들 원초적 관념이 갖는 중요성은
지질학자에게 초기 지구의 지각이 갖는 중요성에 비할 만하다.
거기에는 후대의 법에 의해 발현될 모든 형태들이 다 담겨있을 수 있다.
조급함이나 편견으로 인해 기껏해야 피상적인 조사만 하고는 더는
아무 것도 하지 않은 탓에 오늘날 우리의 법학은 불만족스런 상태에 머물러있다.
법학자들의 탐구는 실로 물리학이나 생리학에서 관찰이 억측을 대체하기 이전
상태와 비슷한 상태에 머물러있다.
그럴듯하고 포괄적이지만 전혀 증명되지 않은 이론들, 가령
자연법이나 사회계약론 따위의 이론이 널리 인기를 구가하여
사회와 법의 원초적 역사에 대한 냉철한 연구를 압도하고 있다.
저 이론들은 진리를 가리고 있거니와,
진리가 발견될 수 있는 유일한 영역으로부터 관심이 멀어지게 할 뿐만 아니라,
일단 길들여지고 믿게 되면 후대의 법학에 참으로 엄청난 영향력을
행사할 수 있다.

\para{테미스테스}
법이나 생활규칙이라는, 이제는 무척 발달한 관념에 관련된 최초의 인식은
\wi{호메로스}가 사용한 용어 ``테미스''\latin{Themis}와
``\wi{테미스테스}''\latin{Themistes}에 담겨있다.
주지하듯이 후대 그리스의 신들 중에서 테미스는 정의의 여신으로 나타난다.
하지만 이것은 근대적인, 무척 발달된 관념의 산물이다.
<<\wi{일리아스}>>에서 제우스의 판결보조자로 묘사된 테미스는 전혀 다른 의미를 가졌다.
오늘날 원시사회의 믿을 만한 관찰자들이 밝혀놓았듯이,
인류의 유년기에 인간은
지속적인 혹은 반복적인 사건을
인격의 작용을 가정함으로써만
설명할 수 있었다.
그리하여 바람이 부는 것도 인격이었고 물론 신적 인격이었다.
태양이 뜨고 정점에 이르고 지는 것도 인격이었고 신적 인격이었다.
대지가 수확물을 내주는 것도 인격이었고 신이었다.
물리적 세계가 그러하듯이 도덕적 세계도 마찬가지였다.
왕이 분쟁에서 판결을 내릴 때, 판결은 신적 영감의 결과로 이해되었다.
왕들에게, 또는 왕중의 왕인 신들에게, 판결을 제안하는 신이
바로 \hemph{테미스}였다.
이 관념의 특이함은 복수형 표현에서 나타난다.
테미스의 복수형 \hemph{테미스테스}는 신이 판사에게 지시한
판결들 자체를 뜻했다.
왕들은 바로 꺼내 쓸 수 있는 테미스테스의 저장고를 갖고 있다고 생각되었다.
그러나 이것은 법률이 아니라
판결---게르만인들이 ``둠''\latin{doom}이라고 부르는 것에 정확히 일치한다---이었다는
점을 유의해야 한다.
그로트\latin{George Grote} 씨의 <<그리스 역사>>\latin{History of Greece}에 따르면,
``제우스나 지상의 인간 왕은 입법자가 아니라 판사였다.''
그에게는 테미스테스가 주어져 있으나,
위로부터 주어진 것이라는 믿음에 부합하게,
판결들은 어떤 일관된 원칙으로 연결되어 있다고 관념되지 못했다.
그것은 따로따로 분리된 개별적인 판결들이었다.

호메로스의 시에서도 이러한 관념은 잠시 동안의 것이었음을 알 수 있다.
단순한 구조의 고대사회에서 상황의 유사성은 오늘날보다 흔한 일이었을 테고,
유사한 소송이 연달아 제기됨에 따라 판결들도 비슷해지는 경향이 나타났을 것이다.
여기서 우리는 관습의 기원 혹은 초기 형태를 발견할 수 있거니와,
이것은 테미스테스, 즉 판결보다 나중에 등장하는 관념인 것이다.\footnote{%
  그러나 왕의 테미스테스도, 이론적으로는 신적 영감에 의한 것이라 해도,
  실제로는 당시의 관습이나 관행에 기초하였을 것이 틀림없다.
  \latin{Maine, \textit{Early Law and Custom}, p.\,163.} }
근대적 사고방식 탓에 우리는 관습의 관념이 사법적 판결에 선행하고
판결은 관습을 확인하거나 그 위반을 벌하는 것이라고 미리 단정짓는
경향이 강하지만, 관념의 역사적 발달은 내가 제시한 순서대로였음이
틀림없어 보인다.
맹아적 관습을 지칭하는 \wi{호메로스}의 용어는 때로 단수형 ``테미스''였고,
종종 ``\wi{디케}''\latin{Dike}였거니와, 그 뜻은 ``판결''과 ``관습'' 또는
``관행''을 넘나드는 것이었다.
노모스\greek{Νόμος}, 즉 `법'은 후대 그리스 사회의 정치용어로서 대단히 중요하고
유명한 것이지만, 호메로스의 시에는 등장하지 않는다.

신의 작용이라는 이러한 관념,
테미스테스를 제안하고 테미스에 인격화되어있는 신의 작용이라는 관념은
피상적인 연구로는 혼동하기 쉬운
다른 원시적 관념들과 엄격히 구분되어야 한다.
힌두의 \wi{마누법전}에 나타나는, 신이 완성된 법전을 명령한다는 관념은
훨씬 최근의 진보된 관념의 계열에 속하는 것으로 보인다.
``테미스''와 ``테미스테스''는
오랫동안 끈질기게 인간의 정신을 지배했던 관념,
신적 영향력이 모든 생활관계와 모든 사회제도를 지탱하고 지지한다는 관념과
훨씬 더 가깝다.
초기 법에서, 그리고 초기의 정치사상에서,
이러한 믿음의 징후는 모든 면에서 나타난다.
초자연적인 통치권자가 당시의 모든 주요 제도들---국가, \wi{씨족}, 가족---을
성별\hanja{聖別}하고 통합하는 것으로 관념된다.
이러한 제도들 속에서 다양한 관계로 집단을 형성하는 인간은
주기적으로 공동의 제의를 수행하고 공동의 희생물을 바칠 의무를 진다.
때로 이러한 의무는
그들이 수행하는 정화의식과 속죄의식에서
더욱 강하게 인식되거니와,
이는 의도치 않게 또는 부주의로 저지른 불경한 짓에 대해 죄를
사하여 달라는 의미를 띠는 것이었다.
고전문헌에 익숙한 독자라면 누구나,
초기 로마의 입양법과 유언법에 중대한 영향을 미쳤던
\wi{씨족제사}\hanjalatin{氏族祭祀}{sacra gentilicia}에 대해 알고 있을 것이다.
무척 진기한 원시사회의 특징들이 고정되어 남아있는
힌두 관습법에서는 지금도 거의 모든 신분법과 상속법 규칙들이
망자의 장례식에서,
즉 가\hanja{家}의 연속성에 단절이 생기는 때에,
의례를 엄정하게 거행하는 것에 달려있다.

\para{벤담의 분석}
이 단계의 법을 떠나기 전에,
특히 영국의 학자들이 유의해야 점을 지적하고자 한다.
\wi{벤담}은 <<정부론 단편>>\latin{Fragment on Government}에서,
\wi{오스틴}은 <<법학의 영역 확정>>\latin{Province of Jurisprudence Determined}에서,
법을 입법자의 \hemph{명령}으로,
그리하여 시민들에게 부과된 \hemph{의무}로,
그리고 불복종에 대해 주어지는 \hemph{제재}의 위협으로 선언한다.
나아가 법의 첫째 요소인 \hemph{명령}은 하나의 행위가 아니라
일련의 또는 다수의 동종 행위들을 지시해야 한다고 단언한다.
이렇게 여러 요소로 분리한 것은 성숙한 단계의 법에 정확히 부합하는 것이고,
개념에 약간의 변형을 가하면 모든 시대 모든 종류의 법과
형식적으로 부합하도록 만들 수 있을 것이다.
하지만, 오늘날에도 일반인들이 가지는 법관념이
이러한 분석과 완전히 일치한다고 주장할 수는 없다.
또한 원시적 사상의 역사를 파고들면 들수록, 이상하게도 우리는
벤담이 말한 요소들의 결합을 닮은 법의 관념으로부터 점점 멀어짐을 발견하게 된다.
확실히 인류의 유년기에는 어떠한 입법도, 어떤 뚜렷한 입법자도 생각될 수 없었다.
법은 관습의 언저리에도 도달하기 어려웠다.
법은 오히려 습관이었다.
프랑스식 표현으로 법은 ``대기 중에 퍼져 있었다''\latin{in the air}.
옳고 그름의 유일한 권위적 진술은 사건이 일어난 뒤에 내려지는 판결이었다.
위반된 법을 전제하여 내려지는 판결이 아니라,
재판의 순간에 저 위의 권력이 판사의 마음에
처음 영감을 불어넣어 내려지는 판결이었다.
물론 우리는 우리와 시간적으로 관념적으로 멀리 떨어진 사고방식을
이해하기가 무척 어렵다.
그러나 고대사회의 헌정을 더 장기간 천착하고 나면 그것은 더 설득력있게
다가올 것이다.
고대사회에서는 모든 사람이
생애 대부분을 가부장의 전제\hanja{專制} 아래서 살았으므로
그의 모든 행위는 사실상 법이 아닌 변덕에 의해 통제되었던 것이다.
생각건대 다른 나라 사람보다 영국인은
``테미스테스''가
어떤 다른 법 관념보다
선행한다는 역사적 사실을 더 쉽게 이해할 수 있을 것이다.
왜냐하면 영국법의 성격에 관한 다양한 이론들 중에서
가장 유명한, 적어도 실무에 가장 영향력 있는, 이론은 분명
판결과 선례가 규칙이나 원리나 개념구분에 선행한다는 이론이기 때문이다.
주목할 점은 ``\wi{테미스테스}''도,
벤담 및 오스틴의 견해와 같이,
법을 단일한 또는 단순한 명령과 구별하는 성질을 가진다는 것이다.
진정한 법은 유사한 종류의 행위를 모든 시민에게 똑같이 명한다.
이것이야말로 대중들의 마음에 깊이 각인된 법의 성질이며,
``법''\latin{law}이라는 말이 단순히 불변성, 연속성, 유사성에도 사용되고 있는
이유이다.\footnote{이런 맥락의 `법'을 우리말로는 보통 `법칙'이라고 부른다.
  중력의 법칙 등.}
\hemph{명령}은 하나의 행위만 지시하며,
따라서 ``테미스테스''는 법보다는 명령에 더 가깝다.
그것은 따로 떨어진 하나의 사실관계에 대한 재판일 뿐이며,
전후의 판결들 간에 규칙적인 연계가 반드시 존재하지는 않는다.

\para{귀족정 시기}
영웅시대의 문헌은 ``테미스테스''와 이보다 좀 더 발달된 ``디케''라는 말로써
맹아기의 법을 우리에게 드러내보인다.
법의 역사의 다음 단계는 무척 흥미로운 시기이다.
그로트 씨의 <<역사>> 제2부 제9장은 호메로스가 묘사했던 것과는
사뭇 다른 성격의 사회가 등장하는 과정을 잘 기술하고 있다.\footnote{%
  \latinmarks
  George Grote,
  \textit{History of Greece},
  Vol.\,3,
  Boston: John P. Jewett and Company,
  1852. }
영웅시대의 왕의 권위는 부분적으로는 신에게서 부여받은 대권에,
또 부분적으로는 탁월한 힘과 용기와 지혜를 가진 데 의존했다.
점차, 왕의 신성함에 대한 관념이 약해지고 또
일련의 세습 과정에서 허약한 왕들이 배출됨에 따라
왕의 권력은 쇠퇴했고, 마침내는 귀족정으로 대체되었다.
혁명에 관한 정확한 용어를 사용할 수 있다면,
\wi{호메로스}가 여러 번 언급했던 족장들의 위원회\latin{council of chiefs}에 의해
왕의 자리가 찬탈당했다고 말할 수 있을 것이다.
여하튼 이제 유럽 각지에서 왕정 시대가 가고 과두정의 시대가 도래했다.
왕이라는 직함이 완전히 없어지지 않은 곳에서도 왕의 권위는 그저
이름에 불과했다.
라케다이몬에서처럼 그저 세습장군이거나,
아테네의 \wi{아르콘} 왕처럼 그저 관리이거나,
로마의 \wi{제사왕}\hanjalatin{祭祀王}{rex sacrificulus}처럼
그저 사제\hanja{司祭}에 불과했다.
그리스, 이탈리아, 소아시아에서는 어디서나
지배집단은 가상의 혈족관계로 결합된 다수의 가\hanja{家}로
구성되었다.
애초에 그들은 모두 일종의 신성성을 주장했으나,
자칭의 신성성이 그들에게 그다지 힘이 되었던 것 같지는 않다.
민중파에 의해 일찍이 전복되어버린 경우가 있었는데,
그렇지 않은 경우 결국 그들 모두는
오늘날 우리가 정치적 귀족이라고 부르는 것에 아주 근접해갔다.
이탈리아와 그리스 세계의 이러한 혁명에 비해,
더 먼 아시아 지역의 공동체에서의 사회 변화는
물론 시간적으로 훨씬 더 전에 일어났다.
하지만 문명화과정에서 이들 변화의 상대적 위치는 동일했고,
변화의 일반적 성격도 대단히 유사했던 것 같다.
나중에 페르시아 군주정 아래 통합되는 제 민족들이,
그리고 인도 반도 곳곳에 살았던 제 민족들이,
모두 영웅시대와 귀족정시대를 거쳤다는 여러 증거가 있다.
하지만 여기서는 군사적 과두제와 종교적 과두제가 각각 독립적으로 성장했고,
왕의 권위도 대체로 폐기되지 않았다.
또한 서구의 역사 전개와 달리, 동양에서는
종교적 요소가 군사적^^b7정치적 요소를 압도하는 경향이 있었다.
왕과 사제집단의 틈바구니에서 군사적^^b7세속적 귀족은 보잘 것 없이
절멸당하고 파괴당하여 사라진다.
그리하여 도달한 최종 결과는
왕이 커다란 권력을, 그러나 사제계급의 특권에 의해 제한되는 권력을,
누리게 되는 것이다.
동양의 종교적 귀족과 서양의 세속적^^b7정치적 귀족이라는
이러한 차이에도 불구하고,
영웅적 왕의 시대에 이어 귀족정 시대가 도래한다는 역사적 명제는
참이라 간주해도 좋을 것이다.
전 인류에 타당할지는 모르겠으나, 적어도 인도^^b7유럽 계통 민족들에게는
두루 타당한 것이다.

\para{관습법}
법학자들이 주목할 점은
어디서나 이들 귀족이 법의 저장소이고 법의 집행자였다는 것이다.
그들은 이제 왕의 대권을 계승한 것으로 보인다.
그런데 중요한 차이가 있거니와,
그들은 매번 판결마다 직접 신의 영감을 받는다고 내세우지 않았다.
가부장적 족장의 판결이 초인간적 지시에 연결된다는 관념은
법규칙의 전부 또는 일부가 신에게서 기원한다는 주장을 통해 여기저기서 여전히
나타나고 있지만,
사고의 발달로 이제 더는 구체적인 분쟁의 해결을
인간 외적인 힘의 개입을 가지고 설명할 수 없게 되었다.
법적 과두정이 주장하는 바는 이제 법\hemph{지식}의 독점, 즉
분쟁을 해결하기 위한 법원칙을 그들만이 가진다는 것이다.
실로 우리는 \wi{관습법}\latin{customary law}의 시대에 들어선 것이다.
이제 관습이나 관례는 실체적 규칙의 집합으로 존재하고,
귀족 집단 혹은 귀족 카스트가 그것을 정확히 알고 있다고 간주된다.
옛 전거들에 따르면 과두정에 주어진 이러한 신뢰가
때로 남용되기도 했음이 분명하지만,
이를 단순한 찬탈이나 폭정의 장치로만 보아서는 안 될 것이다.
문자의 발명 이전에는, 그리고 기술이 유년기에 머물던 시절에는,
법적 특권을 가진 귀족들이야말로 민족 혹은 부족의 관습을
거의 정확하게 보존하는 유일한 현실적 방법을 구성했다.
공동체의 일부 구성원의 기억에 관습을 맡김으로써
관습의 진정성은 최대한 담보될 수 있었다.

관습법의 시대, 그리고 특권 계급에 의한 관습법의 보존은
자못 흥미를 불러 일으킨다.
당시의 법 상태는 오늘날의 법률용어나 일상용어에도 그 흔적을 남기고 있다.
그리하여
카스트이든, 귀족이든, 사제 지파든, 신관단\latin{sacerdotal college}이든,
특권을 가진 소수만이 알고 있는 법은 진정한 불문법이다.
이것을 제외하면 세상에는 불문법이 존재하지 않는다.
영국 판례법이 흔히 불문법이라 불리고 있고, 또 어떤 영국 학자들은
영국법을 법전으로 편찬하면 불문법이 성문법으로
대체---그들이 비판적인 취지에서, 그러나 사뭇 진지하게 사용하는 용어로는,
개종---될 것이라고 주장한다.
물론 영국 보통법을 마땅히 불문법이라고 칭해도 좋을 시기가 한때 있었음이
분명하다.
영국의 옛 판사들은 변호사나 일반인은 온전히 알 수 없는
규칙, 원리, 개념구분 등을 알고 있다고 내세웠다.
그들이 독점한다고 주장한 법의 전부가 진정 불문법이었는지는 무척 의문스럽다.
하지만, 어쨌든 판사들에게만 알려진 민사 및 형사 규칙들이 한때 상당히 있었다고
가정하더라도, 오늘날에는 그것은 더 이상 불문법이 아니다.
웨스트민스터 홀의 법원들이
연감\latin{yearbook} 등에 기록된 선례에 따라 판결을 내리기 시작하면서,
그들의 법은 성문법이 되었다.\footnote{메인의 이러한 성문법 개념은
  오늘날 통용되는 개념과 다르다는 데 주의할 것. 우리는 판례법, 관습법,
  조리법 등을 모두 불문법으로 분류한다.
  메인이 연감에 기록된 옛 보통법 판례의 성문법성을 주장하는 것은
  이를 일종의 `고대법전'으로 간주하기 위해서인 듯하다.}
오늘날 영국의 법규칙은 우선 인쇄된 선례의 사실관계로부터 분리되고,
특정 판사의 성향, 꼼꼼함, 지식에 따라 달라지는 어떤 언어의 형식으로 만들어진 후,
해당 사건의 사실관계에 적용되는 것이다.
그러나 이 과정의 어느 단계에서도 성문법과 구별되는 성질은 나타나지 않는다.
그것은 성문의 판례법인 것이다.
법전법과 다른 점은 단지 쓰여진 방식이 다르다는 것뿐이다.

\para{12표법}
관습법의 시대로부터 이제 우리는 법제사에 뚜렷이 획을 긋는 다른
시대로 진입하게 된다.
그것은 \index{법전 시대}법전\latin{code} 시대로,
로마의 \wi{12표법}으로 대표되는 고대 법전의 시대다.
그리스에서, 이탈리아에서, 그리스화된 서아시아 해안 지역에서,
이들 법전은 모두 어디서나 동일한 시기에 등장했다.
여기서 동일한 시기란
시간적으로 동시라는 뜻이 아니라,
각 공동체의 상대적 진화 단계에서 유사한 시기를 점한다는 뜻이다.
내가 언급한 지역 어디서나 법은 판자\latin{tablets}에 새겨져 대중에게 공표되었고,
그리하여 특권 귀족의 기억 속에 저장된 관행들을 대체했다.
오늘날의 법전편찬이라는 것에 가까운 어떤 세련된 숙려가
내가 말한 변화에 조금이라도 들어있었다고 생각해서는 안 된다.
고대 법전은 애초에 문자 기술의 발견과 확산에 의해 도입된 것이 분명하다.
물론 귀족들이 법지식의 독점을 남용했음에 틀림없고,
어쨌든 그들에 의한 배타적 법 전유\hanja{專有}가 서구에서 보편적으로 등장하기 시작한
민중 운동의 성공에 커다란 장애가 되었던 것은 사실이다.
하지만, 비록 민주적 감정이 법전의 확산에 도움을 주었을지라도,
대체로 법전은 문자 발명의 직접적 산물이었음이 확실하다.
일군의 사람들의 기억이
비록 반복적 사용에 의해 강화된다 할지라도,
그러한 기억보다는
글자가 새겨진 판자가 법의 저장소로서 더 훌륭했고,
법의 정확한 보존을 더 잘 담보했다.

로마의 법전은 내가 묘사하는 그러한 유형의 법전에 속한다.
그것의 가치는 조화로운 분류라든가 표현의 간결성과 명확성 따위에
있는 것이 아니라, 그 공개성, 즉 무엇을 하고 무엇을 하지 말아야 할 지에 관한
지식을 모든 사람들에게 제공하는 데 있었다.
물론 로마의 \wi{12표법}은 어느 정도 체계성을 보여주긴 하지만,
아마도 이는 후기 그리스의 발달된 입법기술을 갖춘 그리스인들의 도움을 받아
12표법이 기초되었다는 전승\hanja{傳承}에 의해 설명할 수 있을 것이다.
하지만 아테네의 솔론 법전의 남아있는 단편들은
체계가 별로 없었음을 보여주며, 아마도 드라콘의 입법은 더욱 그러했을 것이다.
또한 동^^b7서양을 막론하고 이들 법전의 유물들은
종교적, 시민적, 그리고 단순한 도덕적 명령들이
그 성질의 차이를 고려하지 않은 채 무질서하게 혼재되어 있었음을 보여준다.
이는 법 외의 다른 분야의 초기 사상에 관해 우리가 알고 있는 것과 일치한다.
법과 도덕의 분리, 법과 종교의 분리는 정신의 진화에서
분명히 더 후대의 단계에 속하는 것이다.

\para{마누법전}
그러나, 현대인의 눈에 이들 법전이 아무리 이상하게 보일지라도,
고대사회에서 이 법전들의 중요성은 이루 다 말할 수 없을 정도이다.
문제는---이는 각 공동체의 장래에 큰 영향을 미치게 되는 것인데---도대체
법전이 있어야 하는가 아닌가가 아니었다.
대부분의 고대사회는 어쨌거나 조만간 법전을 가지게 되기 때문이거니와,
봉건제에 의해 만들어진 법제사의 큰 단절이 없었다면
모든 근대법은 이들 원천 중 하나 이상으로
기원을 소급할 수 있었을지도 모른다.
오히려 인류 역사의 전환점은
사회 진화의 어느 시기, 어느 단계에서 그들의 법이 성문화되었는가와 관련된다.
서구에서는 각 나라의 평민적^^b7민중적 요소가 과두제의 독점을 성공적으로
공격했고, 국가 역사의 비교적 초기에 거의 보편적으로 법전을 획득했다.
하지만 전술했듯이 동양에서는 군사적^^b7정치적 귀족이 아니라
종교적 귀족이 지배 귀족이 되어 이들이 권력을 장악하는 경향이 있었다.
그런데 몇몇 경우 서구에 비해 아시아 나라들은 그 물리적 조건으로 인해
개별 공동체가 더 커지고 인구도 더 많아지는 경향이 있었다.
그리고 어떤 제도가 적용되는 공간이 크면 클수록
그 제도의 완고함과 생명력이 더 커진다는 것은 널리 알려진 사회법칙이다.
원인이야 어찌되었든, 동양사회의 법전은 서구에 비해
상대적으로 훨씬 늦게 획득되고, 그리하여 사뭇 다른 성격을 띠게 된다.
아시아의 종교적 귀족들은 스스로 참고하기 위해서든, 기억의 괴로움을
덜기 위해서든, 후계자의 교육을 위해서든, 어쨌거나
그들의 법지식을 종국에는 법전의 형태로 구체화하기에 이른다.
그러나 자신들의 영향력을 확대하고 공고히하려는 유혹이 너무나 강해서
이에 저항하기 어려웠을 것이다. 즉,
법지식을 완전히 독점하고 있었기에 그들은
법전화를 되도록 미룰 수 있었을 것이다.
그들의 법전은 실제로 행해지는 규칙이 아니라,
준수하는 것이 마땅하다고 사제집단이 생각한 규칙들을 모은 것이다.
\wi{마누법전}\latin{Laws of Manu}이라 불리는 힌두법전은 브라만들이 집성한 것으로,
물론 인도인들이 실제로 준수한 것들을 다수 간직하고는 있지만,
오늘날 최고 가는 동양학자들의 견해에 따르면
전체적으로 볼 때 그것은 인도에서 실제로 행해지던 규칙들의 집합이 아니다.
대체로 그것은 브라만들이 보기에 법\hemph{이어야 할} 것들을
이상적으로 그려놓은 것이다.
인간의 본성을 감안할 때, 그리고 그 저자들의 특별한 동기를 감안할 때,
마누법전 같은 것이 아주 오래 전의 것인양 내세워지고
그 완전한 형태로 신에게서 유래한 것이라 주장되는 것은 당연한 일에 속한다.
힌두 신화에 따르면 마누는 최고 신의 화신인 것이다.
하지만 그의 이름이 붙어있는 법전은, 비록 정확한 연대는 알 수 없지만,
힌두법의 진화 과정 중 상대적으로 최근의 산물이다.

\para{타락}
\wi{12표법} 등의 법전이 그것을 획득한 사회에 가져다준 주요 이점은
특권 귀족들의 기만적 행태에 대한 보호막을,
그리고 국가 제도의 자연적 타락에 대한 보호막을 제공한 것이었다.
로마의 법전은 단순히 로마 인민의 기존 관습을 언어로 선언한 것이었다.
그것은 로마의 문명화 과정에서 상대적으로 무척 이른 시기에 법전화된 것이었고,
시민적 책무와 종교적 의무가 착종되어 있던 지적 상태를 아직
로마 사회가 거의 벗어나지 못했을 때에 공표된 것이었다.
그런데 이와 달리 여전히 관습을 준행하는 미개한 사회는
문명의 진보에 전적으로 치명적일 수 있는 어떤 특별한 위험에 노출된다.
공동체가 그 유년기에, 원시적 단계에 채택한 관행들은
대체로 그 물질적^^b7정신적 복리의 증진에 가장 적합한 경우가 일반적이다.
새로운 사회적 필요가 새로운 관행을 낳을 때까지 그것들이 순수하게 보존된다면
사회의 상승적 행진은 거의 확실해진다.
하지만 불행하게도 불문\hanja{不文}의 관행에 기초한 작동에는 그것에 위협이 되는
어떤 발전 법칙이 존재한다.
관습을 준수하는 대중들은 그 유용성의 진정한 근거를 알지 못한 채
당연한 듯 관습을 준수하거니와,
따라서 그들은 불가피하게 준수의 미신적 근거를 발명한다.
그리하여 합리적인 관행이 비합리적인 관행을 낳는다는 표현으로
간결하게 묘사될 만한 어떤 과정이 시작된다.
유추\hanja{類推}는 성숙기 법학에서는 무엇보다 유용한 도구이지만,
유년기에는 무엇보다 위험한 덫이 된다.
어떤 합당한 이유로 애초에 특정한 하나의 행위에만 국한되던 명령과 금지가
동일한 유\hanja{類}의 다른 모든 행위들에도 적용되기 시작한다.
하나의 행위가 야기하는 신의 분노에 두려움을 느낀 인간은
그것과 조금밖에 비슷하지 않은 다른 행위에 관해서도
자연스레 공포를 느끼기 때문이다.
위생상의 이유로 어떤 음식이 금지되면,
그럴듯한 유추에 때로 의존하여
그 금지는 유사한 다른 모든 음식에도 확장된다.
또한, 일반적 청결을 보증하는 현명한 규칙 하나가 이윽고
판에 박힌 의례적 세정\hanja{洗淨}행위의 기나긴 목록을 명령하게 된다.
또한, 역사 과정의 특정한 위기 시에 국가의 존립을 위해 잠시 필요했던
계급의 구분이 인류의 제도 중에 가장 재앙적이고 파멸적인 것---카스트---으로
타락한다.
힌두법의 운명은 실로 로마 법전의 가치를 보여주는 척도다.
민족학은 로마인과 인도인이 원래 동일한 계통에서 발원했음을 알려준다.
사실 그들의 최초의 관습으로 여겨지는 것들 간에는
대단히 큰 유사성이 있다.
오늘날에도 힌두법의 밑바탕에는 선견지명과 건전한 판단이 깔려있다.
하지만 비합리적인 모방으로 인해 잔인하고 부조리한 거대한 장치가
힌두법에 접목되었다.
로마인들은 그들의 법전에 의해 이러한 타락으로부터 보호될 수 있었다.
그것은 그들의 관행이 아직 건강했을 때 편찬되었거니와,
만약 백년 후였다면 너무 늦었을지도 모른다.
힌두법은 그 대부분이 성문화되었다.
그러나,
산스크리트어로 전해지는 집성들은 일응 오래된 것이긴 하지만,
해악이 작용한 연후에 작성되었다는 풍부한 증거가 그것들 속에 들어있다.
만약 12표법이 공표되지 않았다면 로마인들도 인도인들처럼
허약하고 타락한 문명으로 전락할 운명이었을지에 관해
물론 우리는 아무 것도 말할 수 없다.
하지만 한 가지 확실한 점은 그들의 법전과 \hemph{더불어}
로마인들은 저 불행한 운명으로부터 벗어날 수 있었다는 것이다.



\chapter{법적 의제}

원시법이 법전에 구체화되면서, 자생적 발달이라고 할 만한 것은
종말을 맞았다.
이후로는 법 내부에 변화가 일어난다면 그것은 의도적으로 일어난,
그리고 외부로부터 영향받은 변화인 것이다.
가부장적 왕에 의해 선언된 후
성문화되어 공표되기까지 그 긴 시간---몇몇 경우에는 장구한 기간---동안
어떤 민족이나 부족의 관습이
전혀 변함 없이 유지된다는 것은 상상할 수 없는 일이다.
또한 그 변화의 어떤 부분도 의도적으로 일어난 부분이 전혀 없다고
단정하는 것도 옳지만은 않을 것이다.
그러나,
이 기간의 법발달에 대해 우리가 아는 바가 별로 없긴 하지만,
변화를 가져옴에 있어 미리 계획된 목적이 차지하는 몫은
극히 작았을 것이라고 가정해도 무리가 없다.
초창기의 관행에 일어난 그러한 혁신은,
오늘날 우리의 정신 조건을 가지고는 도저히 이해할 수 없는 감정과 사고양식에 의해
주어졌던 것 같다.
하지만 법전과 더불어 새로운 시대가 시작된다.
법전 시대 이래, 법변동의 경로 어디를 추적하더라도
그것이 의식적인 개선 노력에 기인한다는 것을,
적어도 원시 시대에 목표했던 것과는 다른 목표를 달성하려는 노력에
기인한다는 것을 발견할 수 있다.

\para{진보의 희귀성}
언뜻 보면, 법전 시대 이후의 법의 역사에서 어떤 믿을 만한 명제를
이끌어내는 것은 불가능해보인다.
대상 영역이 너무 넓다.
충분히 많은 수의 현상을 관찰했는가,
또 관찰한 것을 정확하게 이해했는가, 따위에 대해 우리는 확신을 가질 수 없다.
그러나,
법전 시대 이후 정체된 사회\latin{stationary society}와
진보하는 사회\latin{progressive society} 간의 구별이 나타나기 시작했음을
감안하면, 우리의 과업이 불가능해 보이지는 않는다.
우리의 관심대상은 진보하는 사회에 국한되거니와,
그것들의 숫자가 무척 적다는 점이 무엇보다 두드러진다.
압도적인 증거에도 불구하고, 서유럽 시민의 한 사람으로서
그를 둘러싸고 있는 문명이 세계 역사에서 희귀한 예외에 불과하다는 사실을
완전히 체감하기란 결코 쉬운 일이 아니다.
전체 인류에 대한 진보적 민족의 관계를 또렷이 직시한다면
우리가 공유하는 사상의 풍조, 우리들의 모든 희망, 두려움, 생각이
크게 바뀔 수 있을 것이다.
의심할 여지 없이,
문명제도들을 항구적 기록으로 구체화하여 외면적 완성을 이룩한 순간 이후로
인류의 대부분은 그 문명제도들을 개선하려는 일말의 욕구조차
보여준 적이 없었다.
때로 어떤 관행이 폭력적으로 전복되어 다른 관행에 자리는 내주는 경우는 있었다.
곳에 따라 원시 법전은,
초자연적 기원을 내세우며 대폭 확대되기도 했고,
종교적 주석가들의 왜곡을 거치며 놀랄 만한 형태로 뒤틀려지기도 했다.
하지만 이 세상의 아주 작은 한 지역을 제외하면
법체계의 지속적^^b7점진적 개량 같은 것은 찾아볼 수 없었다.



\chapter{자연법과 형평법}

본질적인 탁월함을 가진
일군의 법원리가
낡은 법을 대체한다는 이론은
로마에서도 영국에서도 아주 일찍부터 통용되었다.
어떤 법체계에서도 발견되는
이러한 원리들을 앞 장에서 형평법\latin{equity}이라고 불렀다.
곧 살펴보겠지만, 이 용어는
로마 법학자들이 이러한 법변화 작인\hanja{作因}을 지칭하는
여러 명칭 가운데 하나, 오직 하나에 불과했다.


\chapter{자연법의 근대사}

지금까지의 논의로부터, 로마법의 변화를 가져온 저 이론은 어떤 철학적 엄밀성을
주장한 것이 아니었음을 알 수 있을 것이다.
그것은 사실 일종의 ``혼합적 사고양식''이었거니와,
이러한 사고양식은 오늘날 최고의 정신을 제외한 모든 인간 정신의 유아기적 사고의
특징으로 인식되고 있으며 또한 현대인의 정신에서도 어렵지 않게
발견할 수 있는 것이다.
자연법이론은 과거와 현재를 혼동했다.
논리적으로는, 그것은 한때 자연법에 의해 통치되었던 자연상태를 상정한다.
하지만 로마의 법학자들은 그러한 자연상태의 존재를 분명하게 그리고
자신있게 말하지 않았다. 사실 황금시대를 상상하는 시적인 표현을
제외하면 고대인들은 그러한 상태에 대해 거의 언급하지 않았다.
실무적 목적에서는, 자연법은 현재에 속하는 어떤 것이고,
기존의 제도와 얽혀있는 어떤 것이며, 유능한 관찰자에 의해
기존 제도와 구분될 수 있는 어떤 것이다.
자연의 명령을 이와 함께 섞여있는 조잡한 요소들로부터 분리하는 기준은
단순성과 조화성의 감각이었다.
하지만 이들 더 세련된 요소가 애초 존중받은 것은
단순성과 조화성 때문이 아니라,
자연의 원초적 지배의 후예라는 데에 있었다.
이러한 혼동은 현대의 법학자들에 의해서도 성공적으로 설명되지 못했다.
실로
로마 법률가들이 받아야 할 비난보다 오히려
오늘날의 자연법사상이 인식의 불명료성을 훨씬 더 많이 노정하고 있으며
언어의 절망적인 모호성에 의해 더 많이 오염되어 있다.
이 주제에 관한 저자들 몇몇은, 자연법법전은 미래에 존재하는 것이고
모든 시민법들이 지향해야 할 목표라고 주장함으로써,
이러한 근본적인 난제를 피해가려고 시도하나,
이는 옛 이론이 근거하고 있던 가정을 순서만 뒤집는 것이거나,
아니면 서로 양립할 수 없는 두 이론을 뒤섞는 것에 불과할 것이다.
과거가 아니라 미래에서 완전성을 찾는 경향은 기독교에 의해
이 세상에 도입된 것이다.
사회의 진보가 더 나쁜 것에서 더 좋은 것으로 필연적으로 진행된다는 믿음은
고대 문헌에서는 거의 혹은 전혀 발견되지 않는다.

하지만 그 철학적 결함에 비해 이 이론이 인류에게 미친 영향은 훨씬 더 심대했다.
만약 자연법의 믿음이 고대세계에 보편적으로 퍼지지 않았다면
어떤 사상사적 전환이, 또 그에 따른 인류사적 전환이, 일어났을까는
실로 말하기가 쉽지 않다.

\para{자연법}
법, 그리고 법에 의해 결합되는 사회는 그 유아기에
두 가지 위험에 특히 취약하다.
하나는 법이 너무 빨리 발달할 수 있다는 것이다.
진보적인 그리스 공동체들에서 이런 일이 발생했거니와,
이들 공동체는 놀라운 능력으로 불편한 소송절차와 불필요한 법률용어의
질곡을 벗어던졌고, 곧이어 엄격한 규칙과 법규정들에 미신적 가치를 부여하는 일을
그만두었다.
이것으로 그 공동체의 시민들이 누린 직접적 혜택은 상당히 컸지만,
그것은 인류의 궁극적 이익에 기여하지는 못했다.
민족성의 드문 자질 중 하나는,
보다 높은 이상에 법을 일치시키려는 희망을 잃지 않으면서도,
법 자체의 적용과 운용에 있어
추상적 사법\hanja{司法}을 구현하는 데는 지속적으로 실패하는 능력이다.
유연성과 탄력성에 뛰어난
그리스의 지식인들은
엄격한 법형식의 틀 속에 스스로를 가둘 수가 없었던 것이다.
우리가 비교적 소상히 알고 있는 아테네의 인민법원을 두고 판단하건대,
그리스의 법원은 법률문제와 사실문제를 혼동하는 경향을 강하게 나타냈다.
아리스토텔레스의 <<수사학>>\latin{Treatise on Rhetoric}에 남아있는
웅변가\latin{orator}들과 법정 표현들의 흔적을 보건대,
순수한 법률문제의 변론은
판사들에게 영향을 줄 수 있는 모든 것을 끊임없이 고려하면서 이루어졌다.
이런 방식으로는 지속가능한 법학체계가 만들어질 수 없다.
특정 사건의 사실관계에 대한 완벽한 이상적인 결정에 성문법 규칙이 방해되는
경우라면 언제나 그 성문법 규칙을 완화하는 데 거리낌이 없었던 공동체는,
설령 후대에 어떤 법원리들을 물려준다 하더라도
오직 당대에 지배적이었던 옳고 그름의 관념에 기초한 것들만 물려줄 수 있을 뿐이다.
이러한 법은 후대의 보다 발달된 관념에 어울릴만한 틀을 전혀 제공할 수 없다.
기껏해야 그 법을 둘러싼 문명의 볼완전성을 드러내는 철학이 될 수 있을 뿐이다.

국가 사회 중에 그들의 법이
이러한 때이른 성숙과 때아닌 해체의 위험에 의해 위협받은 곳은 많지 않다.
로마인들이 이러한 위협에 심각하게 노출된 적이 있었는지는 모르겠으나,
어쨌든 그들은 자연법이론이라는 적절한 보호장치를 가지고 있었다.
분명 법학자들은 시민법을 점진적으로 흡수하는 체계로 자연법을 관념했으며,
시민법이 폐지되지 않는 한 자연법이 시민법을 대체할 수는 없다고 생각했다.
특정 소송사건을 감독하는 판사들이 자연법의 호소에 압도당할 정도로
그렇게 자연법이 신성하다는 인상은 유포되지 않았다.
이러한 관념의 가치와 유용성은 완벽한 유형의 법이 인간 정신의 눈 앞에
펼치지지 못하게 한 것이었고, 그러한 법에 무한히 가까이 다가갈 수 있다는
희망을 품지 못하게 한 것이었으며, 또한 아직 자연법에 조응하지 못한
기존 법이 부과한 의무를 실무가나 시민들이 거부하지 못하게 한 것이었다.
무엇보다 중요한 점은 이 모범적인 체계가,
후대에 인간의 희망을 꺾어놓았던 다른 많은 체계들과 달리,
결코 상상의 산물이 아니었다는 것이다.
그것은 결코 허황된 원리에 기초한 것으로 관념되지 않았다.
그것은 기존 법의 저변에 존재하며 기존 법을 통하여 추구되어야 한다고 생각되었다.
한마디로 그것의 기능은 구제수단을 제공하는 데 있었지, 혁명적이거나
무정부적인 것이 아니었다.
그리고 정확히 이 점에 있어 불행하게도 근대 자연법 사상은 고대의 그것을
닮지 않은 경우가 많다.

\para{벤담주의}
유아기의 사회에 나타나는 또 하나의 취약성은 훨씬 더 많은 민족들의 진보를
방해하고 가로막았다.
원시법의 엄격성은 대개 일찍이 종교와의 관련성 및 동일시에 의해 등장했거니와,
대부분의 민족들은 그 관행이 처음 체계적인 형태로 굳어질 당시 그들이
가지고 있던 인생관과 행위관에 얽매여왔다.
놀라운 운명으로 이러한 재난을 벗어난 민족이 한 둘 있거니와,
이들 나무줄기에 접목하여 몇몇 근대사회가 기름진 곳이 될 수 있었다.
하지만 여전히 세계의 더 많은 곳에서는 최초 입법자가 그려놓은 기본계획을
추종하는 것이 법의 완성이라고 여겨지고 있다.
만약 그런 곳에서 지식인이 법을 공부했다면, 한결같이 그들은
고대 텍스트에서 그 문자적 의미를 눈에 띄게 벗어나지 않고 이끌어낸 결론의
미묘한 고집스러움을 자랑스러워했을 것이다.
만약 자연법이론이 비범한 탁월함을 로마법에 주지 않았더라면
로마인들의 법이 인도인들의 법에 비해 우월하다고 할 것이
무엇이 있는지 나는 모르겠다.
이 유일한 예외적인 사례에서, 다른 여러 이유들로 인류에 막대한 영향을 끼치게 될
한 사회의 눈 앞에, 단순성과 조화성이 이상적이고 가장 완전한 법의 성질로
나타났던 것이다.
진보를 추구함에 있어 어떤 뚜렷한 목표를 가진다는 것이
한 민족이나 전문직업군에게 가지는 중요성은 아무리 강조해도 지나치지 않다.
지난 30년 동안 영국에서 벤담이 가졌던 막대한 영향력의 비밀은
이 나라 앞에 그러한 목표를 성공적으로 제시한 데 있다.
그는 우리에게 개혁의 뚜렷한 원칙을 제시했다.
지난 세기의 영국 법률가들은 아마도 명민했기에
영국법이 인간 이성의 완성이라는 흔해빠진 역설적 표현에 눈멀지는 않았겠지만,
일을 추진해나갈 다른 원리가 없었기 때문에 그것을 믿는 척 행동했다.
벤담은 다른 모든 목표 위에 공동체의 선\hanja{善}을 두었고,
그리하여 오랫동안 밖으로 빠져나갈 길만 찾고 있던 흐름에 나갈 길을 열어주었다.

우리가 기술해온 관념을 벤담주의의 고대적 대응물이라고 부른다면
그것은 그리 훌륭한 비교가 못 될 것이다.
로마인의 이론은 저 영국인의 이론과 마찬가지 방향으로 인간의 노력을 이끌었다.
그것의 실천적 결과도 공동체의 일반적 선\hanja{善}을 꾸준히 추구해온
일군의 법개혁가들이 달성한 것과 크게 다르지 않았다.
하지만 그것이 벤담의 원리를
의식적으로 예견한 것이었다고 보는 것은 잘못일 것이다.
로마인들의 대중문헌이나 법학문헌에서 분명 인류의 행복은
구제수단을 제공하는 입법의 목표로 때로 제시되곤 했지만,
자연법이라는 돋보이는 주장에 주어진 끊임없는 칭송에 비하면
벤담의 원리를 보여주는 증언은 거의 없거나 희미하다는 점을
주목해야 한다.
로마 법학자들이 기꺼이 수용한 것은 인간에 대한 사랑 같은 것이 아니라
단순성과 조화성---그들이 ``전아''\hanjalatin{典雅}{elegance}하다고 부르며
강조했던 것---의 감각이었다.
그들의 노력이 보다 정밀한 철학의 조언을 받아들인 자들의 노력과
우연히 일치했다는 것은 인류에게는 행운이었다.

\para{프랑스 법률가들}
자연법의 근대사로 전환하면, 우리는 그것의 영향력이 막대하다는 것은
말하기 쉬워도 그 영향이 좋은 것인지 나쁜 것인지를 자신있게
말하기는 어렵다는 것을 알고 있다.
근대 자연법 이론에서 나왔다고 할 수 있는 신조와 제도들은
우리 시대의 가장 첨예한 논쟁의 대상이거니와,
지난 백년 동안 프랑스가 서구 세계에 확산시킨 법, 정치, 사회에 관한
특수한 이념들 대부분이 자연법이론에 그 원천을 두고 있다고 말할 때
이것이 잘 드러난다.
프랑스 역사에서 법률가들의 역할은,
그리고 프랑스 사상에서 법사상의 비중은,
언제나 대단히 컸다.
근대 유럽의 법학이 발흥한 곳은 사실 프랑스가 아니라 이탈리아였지만,
이탈리아로 유학하고 돌아와 전 유럽대륙에 건설된,
그리고 {\small(허사로 돌아갔지만)} 우리 영국에 건설이 시도된,
학자군 중에서 프랑스에 건설된 학자군이 그 나라의 운명에 가장 큰
영향을 끼쳤다.
프랑스의 법률가들은 즉각 카페 왕조 및 발루아 왕조의 왕들과
강력한 동맹관계를 형성했다.
프랑스의 왕권이 여러 소국과 속령들의 집합체에서 떨어져나와
결국 그 위에 성장하게 된 것은 무력에 의한 것인 동시에
법률가들이 왕의 대권을 옹호해주고 봉건적 세습규칙을 해석해 준 데에도 기인했다.
왕과 법률가들의 동맹으로 프랑스 왕들이
강한 봉건제후들, 귀족들, 교회와의 투쟁과정에서 누린 이점은
중세를 거슬러 올라가 당시 유럽을 지배하던 이념을 고려하지 않으면
제대로 평가할 수 없다.
우선 일반화를 향한 강한 열정이 있었고 모든 일반적 명제를 향한 찬양이 높았다.
그리하여 법의 분야에서는, 여러 지방에서 관습적으로 사용되던 고립된
다수의 법규칙들을 하나로
포괄하고 요약하는 모든 일반적 공식\latin{formula}에 대한 무의식적 존중이 있었다.
이러한 공식은 로마법대전이나 표준주석\latin{the Glosses}\footnote{%
  주석학파를 집대성한
  아쿠르시우스(Accursius)의 표준주석(glossa ordinaria)을 말하는 듯.}에
익숙한 실무가라면 물론 얼마든지 제공할 수 있었다.
하지만 법률가들의 권력을 사뭇 증대시킨 또 다른 원인도 있었다.
우리가 말하는 이 시대에는 쓰여진 법 텍스트가 가진는 권위의 정도와 성질에 관한
관념이 보편적으로 퍼져있었다.
대체로 ``이것이 쓰여진 법이다''\latin{Ita scriptum est}라는 우선권을 가진
주장은 모든 항변을 침묵시키기에 충분했다.
우리 시대의 학자라면 인용된 공식을 조바심내며 조사하고,
그 출처를 따져묻고, {\small(필요하다면)} 인용이 들어있는 법령집이
지방관습을 대체할 만한 권위를 가지고 있지 않다고 부인하겠지만,
그 시대의 법학자들은 규칙의 적용가능성을 의문시하거나
기껏해야 학설휘찬이나 교회법에서 반대명제를 인용하는 것 외에
다른 것은 거의 시도하지 않았다.
법논쟁의 이러한 중요한 측면에 대해 사람들이 주저하는 관념을 가졌다는 것을
염두에 두는 것은 무척 중요하거니와,
그것은 법률가들이 국왕에게 힘을 실어준 것을 설명하는 데 도움이 될 뿐만 아니라,
몇몇 흥미로운 역사적 문제를 해명하는 데도 도움을 주기 때문이다.
위조된 교령\hanja{敎令}들\latin{forged decretals}\footnote{%
  `콘스탄티누스의 기증'을 포함한 `이시도르 위서'%
  \hyphlatin{(Pseudo-Isidore)}를 말한다.}을
만든 저자의 동기와 그의 특별한 성공은 이런 맥락에서
더 잘 이해될 수 있는 것이다.
보다 덜 흥미로운 예를 들자면, 브랙턴\latin{Bracton}의 표절을 이해하는 데도,
비록 부분적이지만, 도움을 준다.
헨리3세 시대의 저 영국 법률가는
순수한 영국법의 집성을 당대 영국인들에게 내놓을 수 있었는데,
이는 편제의 전부와 내용의 1/3을 로마법대전에서 직접 빌려온 논저였다.
로마법의 체계적인 연구가 공식적으로는 금지된 나라에서
이러한 작업을 감행했을 것이니, 이는 법학의 역사에서 영원히 풀리지 않는
수수께끼의 하나일 것이다.
그러나, 텍스트의 출처에 대한 고려는 차치하고라도,
쓰여진 텍스트가 가지는 구속력에 관한 당대의 여론 상황만 감안해도
우리의 놀라움은 다소간 완화된다.

프랑스의 왕들이 주권 확립을 위한 긴 투쟁을 성공적으로 종결지은 때,
즉 대체로 발루아^^b7앙굴렘 왕조의 재위기에 이르러,
프랑스 법률가들의 상황은 사뭇 특수한 것이었고 이 상태는
프랑스혁명 발발 시까지 지속된다.
한편으로, 그들은 프랑스에서 가장 식자층에 속했고
대단히 강력한 권세를 누리는 계급을 형성했다.
그들은 봉건귀족들과 나란히 특권계급의 신분을 가졌으며,
프랑스 전역에 걸쳐 분포한 조직을 통해 그들의 영향력을 행사했거니와,
이들 전문직업인이 속해있는 이 기구는
국왕의 특허로 각지에 설립되어
폭넓은 명시적 권한과 더 폭넓은 묵시적 권리를 행사했다.
변호사, 판사, 입법자의 권한을 모두 가진 그들은 유럽 전역의 다른 동료집단들을
훨씬 능가하는 권력을 누렸다.


















\chapter{원시사회와 고대법}

법이라는 주제를 과학적으로 다룰 필요성은 현 시대 들어
완전히 망각된 적이 없거니와,
다양한 재능을 가진 인재들이
이러한 필요성의 인식 하에 논문들을 제출해왔다.
그러나, 생각건대,
지금까지 과학의 자리를 대신 차지하고 있던 것은
대체로 일군의 추측이었다는 것에는 의심의 여지가 별로 없다.
앞의 두 장에서 살펴보았던 로마 법률가들의 추측이 바로 그런 것들이다.
이렇게 추정적 자연상태 이론과
그것에 어울리는 법원리의 체계를 인정하고 수용하는 일련의 명시적 진술들이
이것들을 발명한 시대로부터 오늘날에 이르기까지 거의 중단없이 지속되어왔다.
근대 법학을 기초놓은 주석학파\latin{Glossators}의 주석에서도,
이들을 계승한 스콜라주의 법학자들의 저술에서도 그것들은 등장한다.
교회법학자들의 법리에서도 쉽게 눈에 띈다.
문예부흥기에 쏟아져나온 놀라울 정도로 박학다식한
로마법 학자\latin{civilian}들에서는
그것들이 전면에 부상한다.
그로티우스와 그 후계자들은 그것들에 명료함과 그럴듯함뿐만 아니라
실무적 중요성도 부여했다.
그것들은 블랙스톤의 저서의 서론 장들에서도 읽을 수 있거니와,
이는 뷔를라마키\hyphlatin{Jean-Jacques Burlamaqui}의 저서에서
글자 그대로 옮겨적은 것이다.
오늘날 법학도와 실무가들을 위해 출간된 교재들의 첫 머리를 장식하고 있는
법의 제1원리에 관한 논의는 언제나 저 로마인들의 가설을
재진술하고 있는 것에 불과하다.
그러나 이들 추측의 고유한 형식에서뿐만 아니라
그것들을 감싸고 있는 위장된 겉모습에서도
우리는 그것들이 인간 정신에 뒤섞어넣은 미묘함을 잘 파악할 수 있다.
로크의 사회계약론에서 법의 기원에 관한 이론은
그 로마적 유래를 거의 숨기지 않거니와,
실로 고대의 견해가 근대인들에게 매력적으로 보이려면
어떤 모습을 갖추어야 하는지를 알려준다.
한편, 동일한 주제에 대한 홉스의 이론은
로마인들과 그 후예들이 생각했던 자연법의 현실성을
부인하기 위해 의도적으로 고안된 것이다.
그러나 영국의 정치인들을 오랫동안 적대적 진영으로 양분했던
이들 두 이론은 양자 모두 인류의 비역사적이고 검증불가능한 상태를
근본적 전제로 삼는다는 점에서 서로 닮아있다.
물론 로크와 홉스는 사회 이전 상태의 성격에 대해서, 그리고
그 상태로부터 우리가 알고 있는 사회 상태로 이월하는 계기가 되는
비정상적 행위가 어떤 것이냐에 대해서, 서로 의견을 달리한다.
하지만 원시상태의 사람과 사회상태의 사람 사이에
이들을 갈라놓는 커다란 틈이 있다는 생각에는 일치하거니와,
이 관념이 의식적으로든 무의식적으로든
로마인들에게서 빌려온 것이라는 점에는 의문의 여지가 없다.
사실 법현상을 이들 이론가들이 생각한 방식대로---즉, 하나의 거대한
복합체로---파악한다면, {\small(그럴듯하게 해석되면)}
모든 것을 조화시킬 수 있는 영리한 추측에 의지하여
우리가 우리 스스로 설정한 과업을 자주 회피하게 되더라도,
아니면 절망에 빠져 체계화의 노력을 때로 포기하게 되더라도,
그것은 놀라운 일이 아닐 것이다.

\para{몽테스키외}
로마인들의 법리와 동일한 사변적 기초를 가지는 법이론으로부터
두 명의 유명인사는 제외함이 마땅하다.
그중 첫 번째는 몽테스키외라는 위대한 이름과 관련된 인물이다.
<<법의 정신>>의 첫 부분에는 다소 모호한 표현들이 나오는데,
저자가 당대의 지배적 견해에 공개적으로 도전장을 제출하기를 꺼려했기
때문이라고 여겨진다.
하지만 저 책의 일반적 흐름은 확실히 그 주제에 관한 이전의 어떤 관념과도
결별하는 모습을 보여준다.
흔히들 지적된대로,
방대한 조사를 통해 가상의 법체계들로부터 끌어모은 다양한 사례들 속에는,
상스럽고 생경하고 외설스런 습속과 제도들을 특별히 강조함으로써
문명사회의 독자들을 놀라게 하려는 갈망이 뚜렷이 엿보인다.
그것의 일관된 주장은 법이 기후, 지리적 위치, 우연, 기망 따위의
산물---용인할만한 항구성을 가지고 작용하는 것을 제외한 모든 원인의
결실---이라는 것이다.
실로 몽테스키외는 인간의 본성을 전적으로 유연한 것으로,
외부의 영향을 수동적으로 재생산하고 외부에서 주어진 충동에 묵묵히 복종하는
존재로, 보는 듯하다.
바로 여기에 그의 체계가 체계로서 실패할 수 밖에 없는 오류가 있다.
그는 인간 본성의 안정성을 지나치게 평가절하한다.
그는 인류가 상속받은 자질을,
각 세대가 윗 세대에게서 물려받고 약간의 변경을 주어 다음 세대에
전달하는 자질을,
거의 혹은 완전히 무시한다.
물론, <<법의 정신>>에서 지적된 저 변경 원인들에 대한 적절한 고려가
없는 한, 사회현상도, 그리고 결과적으로 법현상도, 제대로 설명할 수 없다는
것은 틀림없는 진실이다.
그러나 몽테스키외는 그 원인들의 숫자와 힘을 지나치게 과대평가한 듯하다.
그가 나열하고 있는 비정상적 현상들은 거짓된 보고서나
잘못된 해석에 기초한 것이었음이 그후 밝혀졌다.
또한 나머지 것들 중에서도 상당수는 인간 본성의 가변성이 아니라
항구성을 증명하는 것들이니,
그것들은 인류의 이전 단계의 유산이며,
다른 경우라면 받았을 영향력을 끈질기게 거부해온
결과이기 때문이다.
진실은 인간의 정신, 도덕, 신체의 구조에서 안정적 부분이 대부분을
차지한다는 것이다.
그것이 변화에 저항하는 힘은 충분히 커서,
비록 세계의 일부 지역에서 인간 사회의 다양한 변이는 분명 존재하지만,
변화는 그것의 양, 성격, 일반적 방향성을 확인할 수 없을 정도로
그렇게 빠르게 일어나지도 광범위하게 일어나지도 않는다.
우리는
현재 우리가 가지고 있는 지식만을 이용하여 진리에 접근할 수밖에 없지만,
그렇다고
진리가 너무 멀리 있으므로, 혹은 {\small(같은 말이지만)}
장래에 너무 많은 수정이 필요하게 될 것이므로,
그것이 쓸모없고 배울 바가 없다고 생각할 필요는 없는 것이다.

\para{벤담}
주목의 대상이 되어온 또 하나의 이론은 벤담의 역사이론이다.




\chapter{유언상속법의 초기 역사}

\para{교회의 영향}
역사적 연구방법이
법에 관한
기존에 널리 퍼진 연구방법에 비해 우수하다는 것을
증명하는 시도가
영국에서
행해진다면,
유언\latin{testament; will}법보다 더 좋은 예를 보여주는 법분야는 없을 것이다.
그러한 능력은 유언법의 긴 역사와 오랜 지속성에 빚지고 있다.
역사의 초창기의 유언법에서
우리는
아주 유년기의 사회상태를 발견하거니와,
그것은 어느 정도 노력을 기울여야만
그 고대적 형태를 깨달을 수 있는 개념들로 둘러싸여있다.
반면, 진보의 반대편 극인 지금의 우리는
동일한 개념들이
현대적인 용어와 사고습관에 의해 감추어진 것에
불과한
법관념들 가운데에 서있거니와,
따라서
우리의 일상적 정신에 속하는 관념들을 분석하고 조사할 필요성을
인식하는 또 다른 종류의 어려움에 처한다.


\chapter{고대와 근대의 유언 및 상속에 관한 관념}


\chapter{물권법의 초기 역사}

로마의 법학제요 저서들\footnote{가이우스의 법학제요와
유스티니아누스의 법학제요를 일컫는다.}은
소유권의 여러 형태들과 변종들을 정의한 후,
자연법상의 물건취득 방식들에 대하여 논한다.
법제사를 잘 모르는 이들은
취득의 이러한 ``자연법상의 방식들''이
일견
사변적으로나 실무적으로나 큰 관심의 대상이 아닐 것이라고
생각하기 쉽다.
야생동물을 덫으로 잡거나 사냥해서 죽이는 것,
토양이 강물에 의해 충적되어 부지불식간에
내 땅에 부합\hanja{附合}하는 것,
나무가 내 땅에 뿌리를 내리는 것 따위를
로마 법률가들은 모두 \hemph{자연적으로} 취득한다고 말했다.
옛 법학자들은
그들 주위의 여러 작은 사회의 관행에서
이들 취득이
보편적으로 인정되는 것을 분명 관찰했을 것이다.
후대의 법률가들은
이들이 옛 만민법\latin{jus gentium}에 분류되어 있고
단순명쾌하게 기술\hanja{記述}되어 있는 것을 보았고, 그리하여
이들에게
자연법의 자리를
내주었을 것이다.
이들에게 부여된 존엄성은 근대에 이르러 점점 커져,
이제는 원래 가졌던 중요성을 훨씬 능가하는 것이 되었다.
자연법 이론은 이들을 가장 즐기는 음식으로 삼았고,
실무에 사뭇 심각한 영향력을 행사할 수 있도록 만들었다.

\para{선점}
이러한 ``자연법적 취득방식들'' 가운데
한 가지만은 반드시 짚고 넘어갈 필요가 있거니와,
선점\hanjalatin{先占}{occupatio}이 그것이다.
선점은
취득 당시 누구의 물건도 아닌 것을
\paren{법기술적 정의\hanja{定義}가 이어진다}
당신의 물건으로 삼고자 하는 의사로써
점유하는 것을 말한다.
로마 법률가들이 무주물\hanjalatin{無主物}{res nullius}---소유주가
없거나 있어본 적이 없는 물건---이라 불렀던
것이 무엇인지는 열거함으로써만 알 수 있을 뿐이다.
소유주가 \hemph{있어본 적이 없는} 물건에는
야생 동물, 물고기, 야생 조류\hanja{鳥類}, 최초로 캐낸 보석,
새로 발견했거나 경작된 적 없는 토지 따위가 속한다.
소유주가 \hemph{없는} 물건에는
포기된 동산, 버려진 토지,
\paren{특이한 그러나 가공스러운 항목인데}
적\hanja{敵}이 소유한 물건 따위가 속한다.
이 모든 것들은
자기 것으로 삼으려는 의사---일정한 경우 이 의사는
특정한 행위에 의해 명시적으로 드러나야 한다---를 가지고서 처음 점유한
\hemph{선점자}가 완전한 소유권\latin{dominion}을 취득한다.
생각건대,
선점 관행의 보편성으로 인해
한 세대의 로마 법률가들이 그것을 모든 민족에 공통인 법으로
자리매김한 것,
그리고 그 단순성으로 인해
다음 세대의 로마 법률가들이 그것을 자연법에 귀속시킨 것은
그리 어렵지 않게 이해할 수 있다.
그러나 근대 법사\hanja{法史}에서 그것이 누린 행운은
선험적인 고찰로는 얼른 이해되지가 않는다.
로마법의 선점 원리, 그리고 이를 둘러싸고 로마 법학자들이 전개한 법규칙들은
근대 국제법 중에서도
전쟁시 포획에 관한 법과
새로 발견한 땅에 대한 주권 획득에 관한 법의
원천이 되었다.
또한 소유권의 기원에 관한 어떤 이론의 근거가 되었거니와,
이 이론은 대중적으로 인기있는 이론인 동시에,
다수의 위대한 사변적 법학자들이
이러저러한 형태로
널리 수긍하고 있는 이론이다.

\para{적의 소유물, 발견의 법리}
방금 나는 로마법의 선점 원리가
전쟁시 포획에 관한 국제법의 흐름을 결정했다고 말했다.
전쟁시 포획법의 법규칙들은,
적대관계의 발발에 의해 국가들은 일종의 자연상태로 환원되고
이렇게 만들어진 의제적\latin{artificial} 자연상태 하에서
교전국 간에는
사적 소유권 제도가
중지된다는
가정\hanja{假定}에 기초한다.
후기의 국제법 학자들은
그들이 설명하는 법체계에서도
사적 소유권이 어떤 의미에서는 인정된다는 주장을
유지하려고 했기 때문에,
적의 재산이 무주물이라는 가설은 그들에게
정도를 벗어난 충격적인 것으로 여겨졌고,
따라서 그들은 이 가설을 단지 법적인 의제\latin{fiction}에 불과하다고
내세우는 신중함을 보였다.
그러나 자연법이 만민법에 그 기원을 두고 있음을 잘 아는 우리는
어떻게 적의 재산이 무주물로 취급되고 그리하여
최초의 점유자에 의해 취득될 수 있었는지 금방 이해할 수 있다.
고대적 형태의 전쟁을 수행하는 사람들은
승전으로 정복군의 군대가 해산되고
해산된 군인들이 무차별적인 약탈을 자행했을 때
저 관념을 자동적으로 떠올렸을 것이다.
하지만
이때 포획자가 취득하도록 허용된 재산은
원래는 동산에 국한하였을 것으로 보인다.
우리는
고대 이탈리아에서
피정복 국가의 토지에 대한 소유권의 취득에 관해서는
전혀 다른 규칙이 지배했음을 별도의 전거를 통해 알고 있다.
따라서 토지에 대해 선점의 원리가 적용되기 시작한 것은
\paren{항상 어려운 문제이지만}
만민법이 자연법으로 전환되는 시기였을 것으로,
그리하여 황금시대의 법학자들이 행한 일반화의 결과였을 것으로,
짐작할 수 있다.
이에 관한 법리는 유스티니아누스의 학설휘찬에 보존되어 있거니와,
그것은 모든 종류의 적의 재산은 교전 상대방에게 무주물이라는,
그리고 포획자가 그것을 자기 것으로 만드는 선점은 자연법상의 제도라는,
무제한적 주장으로 나아간다.
이러한 명제로부터 국제법이 이끌어낸 규칙들은
때로 군인들의 만행과 탐욕을 필요 이상으로 부추긴다고
비판받았지만,
생각건대 이 비판은
전쟁의 역사를 잘 모르는 사람들에 의해,
그리하여 어떤 종류의 규칙이든 규칙에 대한 복종을 명하는 것이
얼마나 위대한 업적인지 잘 모르는 사람들에 의해 가해진 비판이다.
선점에 관한 로마법 원리가 전쟁시 포획에 관한 근대법에 수용되어 들어왔을 때,
그 남용을 제한하고 정밀함을 부여하는
많은 부수적인 법규칙들도 함께 들어왔으니,
만약 그로티우스의 저서가 권위를 획득한 후에 수행된 전쟁들을
그 이전의 전쟁들과 비교해본다면,
로마법의 규칙들이 수용되자마자 이제 전쟁은 그나마 어느 정도
인내할 만한 성질의 것이 되었음을 알 수 있을 것이다.
선점에 관한 로마법이 근대 만민법\latin{law of nations}에
어떤 해로운 영향을 끼쳤다고 비난받아야 한다면,
해로운 영향을 입었다고 자신 있게 말할 수 있는
분야는
근대 만민법의 다른 영역에 존재한다.
보석의 발견에 로마인들이 적용한 원리를 새로운 땅의 발견에도 적용함으로써,
공법학자\latin{publicist}들은
원래 기대되는 용도와 전혀 맞지 않는 곳에다 억지로
어떤 법리를
가져다 썼다.
15, 16세기의 위대한 항해자들의 발견으로 극히 중요한 것으로 부상한
저 법리는
문제를 해결하기보다는 오히려 야기시켰다.
확실성이 무엇보다 요청되는 두 가지 사항에 관하여
커다란 불확실성이 존재한다는 것이 당장 드러났거니와,
하나는 발견자가 주권자를 위해 취득한 영토의 범위에 관한 것이고,
다른 하나는 `집지'\hanjalatin{執持}{adprehensio},
즉 주권적 점유의 확보\latin{assumption}에
필요한 행위가 무엇이냐에
관한 것이다.
더욱이,
약간의 행운의 결과치고 엄청난 이득을 가져다주는 저 원리는
유럽의 가장 모험적인 몇몇 국가들, 즉 네덜란드, 영국, 그리고 포르투갈에 의해
본능적으로 거부되었던 것이다.
우리 영국인들은,
저 국제법 규칙을 대놓고 부인하지는 않았지만,
실제로는
멕시코만 이남의 아메리카 대륙을 전부 독점한다는 스페인의 주장을
결코 받아들이지 않았다.
오하이오강 유역과 미시시피강 유역을 독점한다는 프랑스왕의 주장도 마찬가지였다.
엘리자베쓰 1세의 등극부터 찰스 2세의 등극에 이르기까지
아메리카의 수역\hanja{水域}에는 완전한 평화가 깃든 날이
하루도 없었다고 할 수 있고,
프랑스왕의 영토에 대한
뉴잉글랜드의 식민가들의
잠식은
그로부터도 한 세기 이상 계속되었다.
저 법리의 적용을 둘러싼 혼란상에 충격을 받은 벤담은
아조레스 제도 서쪽 100리그\latin{league} 지점에 그은 선을 기준으로
스페인과 포르투갈 간에
이 세상의 미발견된 땅을
나누어갖도록 한
저 유명한 알렉산데르 6세 교황의 칙서를 짐짓 칭송하기까지 했다.\footnote{%
  1493년 알렉산데르 교황의 칙서는
  아조레스 제도 서쪽 100리그 지점 자오선의 서쪽을
  아라곤^^b7카스티야 왕국에 주었다.
  이에 불만을 품은 포르투갈은 스페인과의 협상 끝에
  다음 해인 1494년 저 유명한 `토르데시야스 조약'을 맺어
  교황의 자오선을 조금 더 서쪽으로 옮겼다.
}
그의 칭송이 일견 생뚱맞아 보이기는 하지만,
손으로 잡을 수 있는 귀중품의 취득 요건으로
로마 법학자들이 내건 조건을
어떤 군주의 신민이
수행했다고 해서 그 군주에게
대륙의 절반을 내주는 공법학자들의 법규칙보다
과연
저 알렉산데르 교황의 조치가
원칙적으로 더 불합리한 것인지는 의문의 대상일 수 있다.

\para{소유권의 기원}
본 저서의 주제를 연구하는 모든 사람에게
선점은
그것이
사변적 법학에
사적 소유권의 기원에 관한 가상의 설명을
제공하고 있다는 점에서
특히 관심의 대상이다.
애초 공유의 대상이었던 대지와 그 열매가
개인적 소유권의 대상으로 허용되는 과정이
선점이 이루어지는 과정과 동일하다고 한때 널리 믿어졌다.
자연법에 관한 고대적 관념과 근대적 관념 간의 미묘한 차이를 포착한다면,
이러한 가정\hanja{假定}을 이끌어내는 사고방식을
그리 어렵지 않게 이해할 수 있다.
로마 법률가들은 선점을 자연법적 물건취득의 한 방식이라고 주장했고,
만약 인류가 자연의 제도 하에 살고 있다면
선점도 인류의 관행의 일부일 것이라고 그들은 분명 믿었을 것이다.
인류가 실제로 그러한 상태에서 살았던 적이 있다고 그들이 과연 믿었는지는,
전술한 바처럼, 남아있는 자료로는 확인하기가 어렵다.
그러나 확실히 그들은
소유권 제도가 인류의 존재만큼 오래된 것은 아니라고 생각했던 것으로 보이며,
이런 생각은 시대를 막론하고 상당한 설득력을 가지는 것이다.
그들의 모든 도그마를 유보 없이 수용한 근대법학은
가상의 자연상태를 강조하는 열성에 있어서만큼은
그들보다 훨씬 멀리 나아갔다.
그리하여 근대법학은
대지와 그 열매가 한때 무주물이었다는 명제를
수용했을 뿐만 아니라,
자연에 대한 특유한 견해로 인해
국가사회가 형성되기 오래 전부터
인류가 무주물의 선점을 실제로 관행했었다고
서슴없이
가정하기에 이르렀다.
그리고 이로부터
원시 시대의 ``누구의 것도 아닌 물건''\latin{no man's goods}이
역사 시대의 개인의 사적 소유권으로 되는 과정이
바로 선점이었다는 추론이
즉시
도출되었다.
이런 이론을
이런저런 형태로
지지하는 법학자들을 일일이 열거하는 것은
지루한 일이 될 터이고,
그다지 필요하지도 않을 것이다.
언제나
당대의 평균적 의견의 충실한 지표 역할을 하는
블랙스톤이 그의 저서 제2권 제1장에서
그것을 잘 요약해놓았기 때문이다.

\para{블랙스톤의 이론}
그는 이렇게 쓰고 있다.
``대지와 대지 위의 모든 것은 창조주의 직접적 증여로서
인류 공동의 재산이었다.
물론
최초의 시기에도
물건의 공유성은
물건의 본질에만 적용될 수 있을 뿐이었고,
그것의 사용에까지 확장될 수 없었다.
왜냐하면, 자연법과 이성법에 따르면,
물건을 처음 사용하기 시작한 사람은
일종의 일시적 소유권을 취득하고
그것을 계속 사용하고 있는 동안은 그 일시적 소유권도 계속되기 때문이다.
보다 정확히 말하자면,
점유 행위가 지속되는 동안은 점유권도 지속되는 것이다.
그리하여 토지는 공유였고,
토지의 그 어떤 일부도 특정인의 영구적 소유권의 대상일 수 없었으나,
누군가가
휴식을 위해, 그늘을 위해, 또는 다른 이유로
특정 장소를 선점\latin{occupation}하면,
그는 당분간 일종의 소유권을 취득하고,
그에게서 강제로 그 소유권을 빼앗는 것은 부정의하고
자연법에 반하는 일이 될 것이다. 하지만
그가 사용이나 점유\latin{occupation}를 그치는 순간,
다른 사람이 그 장소를 차지하는 것은 아무런 부정의가 아니다.''
그리고 이렇게 주장을 이어간다.
``인류의 인구가 증가하면서,
보다 영구적인 소유권 관념이 필요하게 되었고,
개인에게
일시적인 사용을 넘어
물건의 본질을 사용할 수 있도록
허용할 필요가 생겨났다.''



\chapter{계약법의 초기 역사}

우리가 속한 시대에 관한 명제로,
오늘날의 사회가
지난 시대의 사회와 차이나는 주요 특징은
계약법이 차지하는 영역이 대폭 증가했다는 데 있다는
주장만큼
일견 쉽게 수긍할 수 있을 법한 것도 없을 것이다.
이 명제가 근거하고 있는 현상들 중 일부는
대단히 빈번하게 선택되어 관심과 논평과 칭송의 대상이 되고 있다.
%우리들 중에서
옛 법이 사람의 출생에 따라 그의 사회적 지위를
불가역적으로 고정시켰던 수많은 사안들에서
근대법은 합의에 의해 그 스스로 자신의 지위를 만들어갈 수 있도록
허용하고 있음을
알아차리지 못할 정도로
무감한 사람은
별로 없을 것이다.
실로 이 원칙에 대한 예외로 남아있는 소수의 몇몇 것들은
열정적 분노에 찬 비난을 지속적으로 받고 있다.
가령 흑인\wi{노예제}를 둘러싸고 여전히 진행 중인 열띤 논쟁에서
실로 다투어지고 있는 논점은
노예제가 지난 시대의 제도가 아니냐는 것,
그리고
근대적 도덕성에 부합하는
고용주와 노동자 간의 관계는
오직 계약에 의해 정해지는 관계뿐이지 않겠느냐는 것이다.
과거와 현재 간의 이러한 차이의 인정은
현대의 가장 유명한 사변적 논의의 핵심으로 우리를 끌고 들어간다.
확실히,
명령법\latin{imperative law}이
한때 장악하고 있던 영역의 많은 부분을
포기하지 않았다면,
그리고
최근까지 허용되지 않던 자유를 누리며
사람들이 스스로의 행위규칙을 정하도록 허용하지 않았다면,
도덕에 관한 연구 분야 중에
우리 시대에 비약적인 진보를 보인 유일한 분야인
정치경제학\latin{political economy}은
생활 현실에 부응하지 못하고 실패할 것이다.
정치경제학의 훈련을 받은 사람들의 대다수가
실로 가지고 있는 선입견은
그들 학문이 의지하고 있는 일반적 진리가
보편적인 것이 될 권리가 있다고 보는 것이다.
그리하여 그들이 그것을 학문으로 적용할 때면,
그들의 노력은 대개 계약법의 영역을 확장하고
명령법의 영역은 축소하는 방향을 지향하거니와,
단지 계약의 이행을 강제하는 데 필요한 한에서만
명령법을 용인하는 것이다.
이러한 관념의 영향을 받은 사상가들이 불러일으킨 충격은
바야흐로 서구 세계에서 사뭇 강력하게 느껴지기 시작하고 있다.
입법은
발견과 발명과 축적된 부\hanja{富}의 사용에 관한 사람들의 행동을
따라잡을 능력이 없음을
거의 자백했다.
가장 덜 진보된 공동체의 법조차
점점 단지 껍데기에 불과한 것이 되어가고 있거니와,
그 아래에는
지속적으로 변화하는 계약적 규칙들의 연합이 존재하여,
여기에 법이 개입하는 경우는
약간의 근본원리들의 준수를 강제하거나
신의\hanja{信義} 위반을 벌하기 위해 소환되는 경우 외에는
거의 없는 실정이다.

\para{계약의 강제}
법현상을 고려해야 하는 것인 한
사회 탐구는 그 상황이 매우 낙후되어 있는지라,
사회의 진보에 관하여 널리 통용되는
통속적인 견해에서 저 진리가
발견되지 않더라도 놀라울 것이 없다.
이들 통속적인 견해는
우리의 신념보다는 우리의 편견에 더 잘 부응한다.
도덕의 진보를 인정하기를 꺼리는 강한 경향성은
계약의 기초가 되는 미덕을 의문시할 때
특히 더 강력해지는 듯하다.
우리들 중 다수는
신의와 성실이 옛날보다 오늘날에 더 널리 퍼져있음을,
또는 적어도 고대 세계의 충실성에 비견할 만한 풍속이 오늘날에도 존재함을,
인정하는 것에 대한
거의 본능적인 거부감을 가지고 있다.
때로 이러한 선입견은
예전에는 들려오지 않던
사기행각이 만연함을 보면서,
그리고 이들 범죄가 가져오는 커다란 혼란과 충격을 보면서
더욱 강화된다.
그러나 바로 이러한 사기행위의 범죄성으로부터 우리는,
그것을 범죄로 취급할 수 있기 위해서는
우선
그것이 위반하는 도덕적 의무가 더 크게 성장해야 한다는 것을
뚜렷이 알 수 있다.
다수가 믿고 따르는 신뢰가 있어야만
소수의 신뢰 위반도 생길 수 있는 법이므로,
아주 큰 부정직의 사례들이 발생한다면 이는
다수의 평균적 거래에서는 성실한 정직이 지배적이어서
예외적인 경우 범죄자들에게 기회가 주어졌다고
결론짓지 않을 수 없는 것이다.
계약법에서 형법으로 눈을 돌려
법에 반영된 도덕의 역사를 읽어야 한다면,
우리는 그것을 오독\hanja{誤讀}하지 않도록 주의해야 한다.
로마 고법\hanja{古法}에서 부정직한 행위로 취급된 형태는
\wi{절도}가 유일했다.
이 글을 쓰는 순간,
영국 형법에 추가된 최신 영역은
수탁자\hanjalatin{受託者}{trustee}의 사기행위를 처벌대상으로 삼으려는 것이다.
이러한 대비에서 얻을 수 있는 올바른 추론은
원시 로마인들이 우리보다 더 높은 도덕성을 지녔다는 것이 아니다.
오히려 그들 시대에서 우리 시대로 시간이 흐르면서
사뭇 미개한 도덕성으로부터 대단히 세련된 도덕성으로
도덕성 관념이 진보했다는 것을 알 수 있는 것이다.
소유권만을 신성한 것으로 여기던 관념에서
단지 일방적 신뢰의 수여만으로 발생하는 권리까지도
형법에 의해 보호되는 권리로 보는 관념으로 진보가 이루어진 것이다.

\para{사회계약}
이 점에 관하여 법학자들의 정연한 이론이라고 해서 대중들의 의견보다 더
진리에 가까운 것도 아니다.
로마 법률가들의 견해부터 말하자면,
그것은 도덕과 법의 진보에 관한 참된 역사와 일치하지 않았다.
계약 당사자들이 약속한 신의가 유일하게 중요한 요소인
계약의 한 유형을 그들은 \wi{만민법}상의\latin{juris gentium} 계약이라고
지칭했거니와,\footnote{%
  `낙성계약'(contractus consensu)을 말하고 있다.
  }
이 유형의 계약은 로마법에 나중에야 편입되어 들어간 것이 확실함에도
불구하고,
그들이 사용한 표현으로부터 어떤 확정적 의미를 추출해보면
그들은 그것을 로마법이 인정하는 다른 유형의 계약, 즉
법기술적 방식요건이 하나만 잘못되어도 오늘날의 착오나 사기만큼이나
계약의무의 성립에 치명적이었던 다른 유형의 계약들보다
더 오래된 것으로 보았음을 알 수 있다.
하지만 그들이 말하는 옛 것은 모호하고 희미한 것이었고
현재를 통해서만 이해될 수 있는 것이었다.
그리하여 ``만민법\latin{law of nations}상의 계약''을
자연상태의 사람들 사이의 계약으로 간주하게 된 것은
로마 법률가들의 언어가
그러한 사고양식에 진입하는 열쇠를 이미 상실해버린 시대의 언어로 된
이후의 일이었다.
\wi{루소}는 법률가들의 오류와 대중들의 오류를 모두 이어받았다.
관심을 끈 첫 작품이자
그를 한 분야의 선구자로 만든 의견이 사뭇 기탄없이 개진된 논문인
예술과 학문이 도덕에 끼친 영향을 논하는 논문에서,\footnote{%
  <<학문예술론>>(Discours sur les sciences et les arts)을 말한다. }
그는 고대 페르시아인들이 지녔던 정직함과 신의성실이야말로
문명의 등장과 더불어 점차 망각되어간 원시적 순수성의 특징이라고
누차 지적하고 있다.
그리고 나중에 그는
그의 모든 사변\hanja{思辨}의 토대를
원초적 사회계약의 교리에서 발견하게 된다.
<<사회계약론>>은 우리가 논하고 있는 오류를 지닌 것 가운데 가장 체계적인 형태이다.
비록 정치적 열정에 의해 그 중요성이 고양되었지만
이 이론은 법률가들의 사변으로부터
모든 수액\hanja{樹液}을 채취한 이론이다.
처음 이 이론에 감화된 영국의 저명인사들은
주로 정치적 유용성의 면에서 그것의 가치를 높이 평가한 것이 사실이지만,
뒤에서 설명하겠으나
만약 정치가들이 법적인 용어로 논쟁을 해오지 않았더라면
영국인들은 결코 이 이론에 다가서지 못했을 것이다.
그리하여 이 이론을 주창한 영국인 학자들도
그들로부터 그것을 물려받은 프랑스인들에게 강한 호소력을 가졌던
저 사변적 깊이를 모르지 않았다.
그들의 저서는 이 이론이 정치적 현상뿐만 아니라
사회적 현상까지 모두 설명할 수 있다고 그들이 인식했음을 보여준다.
사람들이 준수하는 실정규칙 가운데
계약\latin{contract}으로 만들어진 것이 점점 많아지고
명령법\latin{imperative law}으로 만들어진 것이 점점 줄어들고 있는 현상,
그들 시대에도 이미 현저하게 나타나고 있던 이 현상을
그들은 관찰을 통해 알고 있었다.
그러나 법학의 저 두 구성부분의 역사적 관계에 대해서는
그들은 무지했거나 주의를 게을리했다.
그리하여
그들은 모든 법은 계약에서 기원한다는 이론을 창안하였거니와, 이는
모든 법의 기원을 단일한 원천에 둠으로써 그들의 사변적 취향을
만족시키기 위한 것이었으며,
또한
명령법은 신에게서 기원한다는 교리\footnote{%
  필머(Robert Filmer)로 대표되는 왕권신수설을 말하는 듯하다.
}를 피하려는 견해에서 나온 것이었다.
한 단계 더 사고가 진보한다면,
그들은 기꺼이
그들의 이론을
어떤 기발한 가설이나 편리한 언어 공식\hanja{公式}에 불과했다고
치부했을 것이다.
그러나 당시는 법적 미신\hanja{迷信}이 지배하던 시대였다.
자연상태에 관한 논의는 그것이 역설적이 아니라고 여겨지는 한 계속되었고,
따라서
사회계약을 역사적 사실로 주장함으로써
법의 계약적 기원이라는 거짓 현실과 확신을
쉽게 심어줄 수 있었던 것으로 보인다.

\para{몽테스키외의 혈거인}
우리 세대는 이러한 잘못된 법이론을 떨쳐버렸다.
그것은 부분적으로는 저 이론이 속했던 지적 상태를 벗어났기 때문이고,
또 부분적으로는 그러한 주제를 이론화하는 일을 거의 그만두었기 때문이다.
오늘날 적극적으로 연구를 수행하는 학자들이 선호하는 작업은,
그리고 사회의 기원에 관한 우리 선조들의 사변에 대해 답할 수 있는 작업은,
사회의 존재를 있는 그대로, 사회의 운동을 운동하는 그대로 분석하는 것이다.
그러나 역사의 도움을 받지 못하면,
이런 분석은 단순한 호기심의 충족으로 전락하기 일쑤이거니와,
특히
연구자가 익숙해있는 사회상태와는 자못 다른 사회상태를 이해하는 데
장애물로 작용할 공산이 크다.
우리 시대의 도덕성을 가지고 다른 시대의 사람들을 판단하는 잘못은
현대사회라는 기계장치의 바퀴 하나, 볼트 하나까지
원초적 사회에 그 대응물이 있을 것이라고 가정하는 잘못에 견줄 만하다.
이러한 인상\hanja{印象}은
근대적 양식으로 쓰여진 역사학 저술들에서
사뭇 다양하게 가지를 치고 있으며
사뭇 미묘하게 모습을 숨기고 있다.
그러나
나는
\wi{몽테스키외}의
<<페르시아인의 편지>>에 삽입된
혈거인\hanjalatin{穴居人}{Troglodytes}의 우화\footnote{%
  \latinmarks
  Montesquieu, \textit{Persian Letters}, 11--14.
}에 대해
흔히 주어지는 찬사에서
법학 영역에서의 그러한 인상의 흔적을 발견한다.
혈거인들은 계약을 항상 위반하는 사람들이었으며, 그래서 완전히 멸망해버렸다.
만약 이 이야기에 저자가 의도한 도덕이 담겨있고,
그것이
금세기와 지난 세기를 위협해온 반사회적 이단\hanja{異端}을
폭로하기 위해 사용되었다면,
그것은 전혀 나무랄 데가 없는 것이다.
그러나
성숙한 문명이 보여주는 것과 같은 정도로
약속과 합의에 신성함을 부여하지 않는 한
어떤 사회도 결속을 유지할 수 없다는
주장이
저 이야기로부터
추론되어 나온다면,
그것은 법사\hanja{法史}의 어떤 건전한 이해와도 상반되는 치명적인 오류가
될 것이다.
사실,
혈거인들은 계약적 의무를 아주 조금 준수함으로써
번성할 수 있었고 강력한 국가를 건설할 수 있었던 것이다.
원시사회의 헌정\hanja{憲政}에 관하여
무엇보다 먼저 이해해야 할 것은
개인은 자신을 위해 권리나 의무를 거의 혹은 전혀 만들지 못한다는 점이다.
개인이 지켜야 할 규칙은 우선은 출생에 따르는 지위에서 나오고,
다음으로는 그가 속하는 가\hanja{家}의 수장이 그에게 부과하는
명령에서 나온다.
이러한 체제는 계약을 위한 여지를 거의 남겨두지 않는다.
동일한 가\hanja{家}의 구성원들은
\paren{증거로부터 해석하건대}
서로 간에 전혀 계약을 체결할 수 없으며,
가\hanja{家}는 그 구성원이 가를 구속시키려고 맺은 계약을
무시할 수 있는 권리를 가진다.
물론 가와 가 사이, 가부장과 가부장 사이의 계약은 있을 수 있지만,
그 거래는 물건의 양도와 마찬가지 성격을 지니므로
수많은 방식요건들이 부과되어
실행에 있어
사소한 실수라도 계약의무의 성립에 치명적인 것이 된다.
타인의 말을 신뢰하는 것에서 생겨나는 적극적 의무는
진보된 문명이 아주 나중에야 성취하게 되는 것이다.

\para{초기 로마의 계약들}
어떤 고대법도, 다른 어떤 전거도,
계약의 개념을 전혀 알지 못하는 사회가 있음을 보여주지 못한다.
그러나 이 개념이 처음 나타났을 때
그것은 분명 아주 원시적이었을 것이다.
어떤 믿을 만한 원시 기록에서도
약속을 유효하게 만드는 인간의 정신이 아직 미숙하였음을,
그리고
노골적인 배신행위가 비난 없이, 때로는 칭송의 대상으로, 언급되고 있음을
읽을  수 있다.
가령 \wi{호메로스}의 문헌에서
오뒷세우스의 기망적인 교활함은
네스토르의 현려\hanja{賢慮}, 헥토르의 지조,
아킬레우스의 용기와 동급의 미덕으로 나타난다.
고대법은 계약의 원시적 형태가 그것의 성숙한 형태로부터
멀리 떨어져있었음을 훨씬 더 분명히 보여준다.
처음에는 단순히 약속의 이행을 강제하기 위해
법이 개입하지는 않았던 것으로 보인다.
법이 제재로써 강제하는 것은 단순한 약속이 아니라,
엄숙한 의례\hanja{儀禮}를 수반하는 약속이었다.
요식성\hanja{要式性}은 약속과 마찬가지로 중요했을 뿐만 아니라,
어쩌면 약속 이상으로 훨씬 더 중요했다.
성숙한 법학이 구두\hanja{口頭}의 승인\hanja{承認}이 행해진 상황에 적용하는
섬세한 분석이
고대법에서는
그것의 실행에 수반되는 말과 몸짓에 전가된 듯하다.
사소한 방식\latin{form} 하나라도 빠뜨리거나 잘못 실행되면 어떠한 서약도
강제될 수 없었다.
한편, 방식이 정확히 준수되었음이 입증된다면,
사기나 강박으로 약속하였다는 항변은 아무 소용이 없었다.
법제사에서는
이러한 고대적 관념으로부터 우리에게 친숙한 계약 관념으로의 이행이
명백히 드러난다.
처음에는 의례의 한 두 단계가 건너뛸 수 있는 것이 되고,
그후 일정 조건 하에서 다른 것들도 단순화되거나 생략이 허용되며,
마침내 몇몇 특수한 계약들이 다른 것들로부터 분리되어
방식의 구애를 받지 않고 체결할 수 있게 되거니와,
이들 특수한 계약은
사회적 거래의 활동성과 에너지가
이에
의존하는 계약인 것이다.
서서히, 그러나 사뭇 명백하게,
법기술적 요소들로부터 심적\hanja{心的}인 요소가 분리되어 나오고,
차츰 법학자들의 관심을 한몸에 받는 유일한 요소가 된다.
외부적 행위를 통해 표현되는
이러한 심적 요소를 로마인들은
`\wi{약정}'\hanjalatin{約定}{pact; convention}이라 불렀다.
그리고 약정이 계약의 핵심으로 인정되자,
곧이어
방식과 의례의 껍질을 부수어버리는 것이
진보하는 법의 경향성이 된다.
그후 방식들은 진정성을 보증하는 한에서만,
그리고 주의와 숙고를 담보하는 한에서만
보존될 뿐인 것으로 된다.
이로써 계약의 관념은 완전한 발달을 보이게 되거니와,
로마법의 용어를 사용하자면,
계약은 약정에 흡수되어버리는 것이다.

\para{양도와 계약}
로마법이 보여주는 이러한 변화 과정의 역사는 자못 시사적이다.
로마법의 여명기에
계약에 해당되는 말로 쓰인 용어는
고대 라틴어를 연구하는 학자들에게는 무척 익숙한 용어이다.
그것은 바로 넥숨\latin{nexum}, 즉 \wi{구속행위}\hanja{拘束行爲}로서,
이 계약의 당사자들은 `피구속자'\hanja{被拘束者}들\latin{nexi}이라 불렸다.
이 표현들은 그 근저에 놓인 은유의 이례적인 지속성으로 인해
특히 주목할 필요가 있다.
계약관계에 놓인 사람들이 강력한 \hemph{속박}\latin{bond}
또는 \hemph{사슬}\latin{chain}로 연결되어 있다는
관념은 마지막까지 계속해서 로마계약법에 영향을 주었고,
거기서 흘러나와 근대적 관념에도 섞여들어갔다.
그렇다면 이 구속행위 혹은 속박이란 무엇을 의미하는 것이었을까?
라틴어에 관한 고문헌을 통해 우리에게 전해진 바에 따르면
구속행위는 ``구리와 저울로써 행해지는
모든 것''\latin{omne quod geritur per aes et libram}이라고
정의되어 있거니와,\footnote{%
  \latinmarks
  Varro, \textit{De Lingua Latina}, 7.105.
  }
이 단어들은 상당히 큰 혼란을 불러있으켰다.
구리와 저울은
\wi{악취행위}에 수반되는 것들로 잘 알려져있다.
악취행위는
앞 장에서 서술한 고법\hanja{古法}상의 엄숙한 행위로서,
로마 물권법에서 높은 등급의 물건의 소유권이
한 사람에게서 다른 사람에게 양도되는 방식이었다.
이렇게 악취행위는 \hemph{양도}\latin{conveyance}의 방식이기에
어려운 문제가 부상하게 된다.
위에 인용한 저 정의는
계약과 양도를 혼동하고 있거니와,
법철학에서는 이 두 가지가 단지 구분될 뿐만 아니라
사실상 서로 대립하는 것이기 때문이다.
성숙한 법학의 분석가들은
물\hanja{物}에 대한 직접적 권리\latin{jus in re},
대세적\hanja{對世的} 권리\latin{right \textit{in rem}},
``온 세상에 대하여 주장할 수 있는'' 권리,
즉 물권\hanjalatin{物權}{proprietary right}과
물\hanja{物}에 대한 간접적 권리\latin{jus ad rem},
대인적\hanja{對人的} 권리\latin{right \textit{in personam}},
``특정인이나 특정집단에 대하여 주장할 수 있는'' 권리,
즉 채권\hanjalatin{債權}{obligation}을
날카롭게 구별한다.
그런데 양도는 물권을 이전하고, 계약은 채권을 창설한다.
어떻게 이 두 가지가 동일한 이름 아래, 동일한 일반개념 아래
포섭될 수 있다는 말인가?
다른 유사한 난제들과 마찬가지로 이 문제도
미발달된 사회의 정신적 상태에
진보된 지적 단계에 특별히 속하는 능력을,
현실에서는 혼재되어 있는 것을 사변적 관념들로 구별하는 능력을,
끼워맞추려는
오류 탓에 발생한 것이다.
여기서
우리는
양도와 계약이 현실적으로 혼재되어 있는 사회상태에 관하여
오인하지 말아야 한다는 시사를 받는다.
계약과 양도에 관하여 독자적인 실무관행이 채택되기 전까지는
저 개념들 간의 차이는 인식될 수 없었던 것이다.

\para{구속행위}
로마 고법\hanja{古法}에 관한 우리의 지식으로부터
법의 여명기에 법적 개념과 법적 용어가 어떻게 변해갔는지
그 변화의 양상에 대한 약간의 관념을 얻을 수 있을 것이다.
이 변화는 일반적인 것에서 특수적인 것으로의 변화라고 할 수 있다.
다시 말해 고법상의 개념과 고법상의 용어는 점진적 특수화의 과정을
겪었던 것이다.
고법상의 개념은 하나가 아니라 다수의 근대적 개념에 대응된다.
고법상의 법기술적 표현은 근대법이 여러 개의 이름으로 나누어놓은
다수의 것들을 지칭한다.
하지만 법사\hanja{法史}의 다음 단계에 이르면,
하위 개념들이 점차 서로 분리되어,
예전의 일반적 이름은 특수적 명칭들로 바뀌어가는 것이다.
그렇다고 옛 개념이 사라지는 것은 아니고,
단지 원래 포섭하던 관념의 일부만 포섭하게 된다.
그리하여 예전의 법기술적 이름은 여전히 존재하지만,
한때 수행했던 기능들 중에 하나만 수행할 뿐이다.
이러한 현상의 예로는 여러 가지를 들 수 있겠다.
가령 여러 종류의 \wi{가부장권}은 한때
그 성격이 모두 동일했고,
따라서 하나의 이름으로 불렸을 것이 틀림없다.
존속친\hanja{尊屬親}에 의해 행사되던 가부장권은
가족에 대해 행사되든 물질적 재산에 대해 행사되든---양떼나 소떼, 노예,
자식, 아내를 불문하고---모두 동일했다.
그것의 옛 로마식 명칭에 대해 완전히 확신할 수는 없지만,
가부장\hemph{권}\latin{power}을 지칭하는 여러 명칭들에
\hemph{마누스}\latin{manus}라는 단어가 들어가 있는 것으로 볼 때,
옛 일반적 명칭은 `마누스'였을 것으로 믿을 만한 근거는 충분해보인다.\footnote{%
  `마누스'는 흔히 `수권'(手權)으로 번역되나 여기서는 본문의 의미맥락상
  원어를 살렸다.
  이하 관련 단어들도 마찬가지다.
}
그러나 로마법이 좀 더 진보하면서,
저 이름도 저 관념도 특수화되었다.
\wi{가부장권}은
그것이 행사되는 대상에 따라
단어에서도 개념에서도 분화되어갔다.
물건이나 노예에 대해 행사될 때는
`도미니움'\latin{dominium},
자식들에 대해서는 `포테스타스'\latin{potestas},
존속친에 의해 다른 사람의 권력에 제공된 자유인에 대해서는
`만키피움'\latin{mancipium}이 되었고,
아내에 대해서는 여전히 `마누스'로 남았다.\footnote{%
  \latin{Gai.\,1.49} 참조. }
여기서 알 수 있듯이,
원래의 단어가 전혀 쓰이지 않게 된 것이 아니라,
과거에 지칭하던 권력행사 중 특수한 한 가지 권력행사에 국한하게 된 것이다.
이 사례를 모범삼아 계약과 양도 간의 역사적 결합관계의 성질에 대해서도
이해를 도모할 수 있을 것이다.
처음에는 모든 엄숙한 거래에 오직 하나의 엄숙한 의례\hanja{儀禮}만
존재했을 것이니,
로마에서는 그것의 명칭이 `\wi{구속행위}'\latin{nexum}였던 것으로 보인다.
물건의 양도에 사용되던 바로 그 방식이
계약의 체결에도 사용되었던 것으로 보인다.
그러나 양도 관념으로부터 계약 관념이 분리되어 나오는 데는
그다지 긴 시간이 필요치 않았다.
그리하여 이중\hanja{二重}의 변화가 일어났다.
``구리와 저울에 의한'' 거래가
물건의 이전을 의도하는 경우에는
`\wi{악취행위}'\latin{mancipation}라는 새롭고 특수한 이름으로 불리게 된다.
옛 이름인 `구속행위'는 여전히 동일한 의례절차를 지칭하지만,
이제
오직 계약을 엄숙하게 체결하는 특수한 목적에만 쓰이게 된다.

\para{변화}
두 세 가지 법개념이 고대에는 하나로 혼재되어있었다고 해서,
거기에 포함된 관념 중 하나가 다른 것들보다 더 오래된 것이 아니라는
말은 아니다.
혹은 그 하나가, 다른 것들이 형성된 후,
이것들보다 크게 우세하거나 우선하지 않는다는 말도 아니다.
하나의 법개념이 오래 계속해서 여러 법개념들을 포괄할 수 있는 이유는,
그리고 하나의 법기술적 용어가 여러 용어들을 대신할 수 있는 이유는,
원시사회의 법에 실무관행의 변화가 일어나더라도
오랫동안 사람들은 그것에 주목하거나 이름붙일
필요를 느끼지 못하기 때문일 것이 분명하다.
비록, 전술했듯이,
처음에는 가부장권에 행사대상에 따른 구별이 없었다 할지라도,
자식들에 대한 권력이 옛 \wi{가부장권}의 근본이었다고
나는 믿어 의심치 않는다.
또한
`\wi{구속행위}'라는 말의 최초의 사용은,
그리고 그것을 사용했던 사람들이 주로 염두에 둔 것은,
물건의 양도에 엄숙한 형식을 부여하려는 것이었음을
나는 의심치 않는다.
구속행위가 원래의 기능으로부터 아주 조금 벗어나기 시작했을 때
그것은 바로 계약의 체결에 사용되었을 것이나,
아주 조금의 변화였기에 그 변화는 오랫동안 인정되거나 감지되지 못했다.
새로운 것을 원한다는 것을 사람들이 자각하지 못했기 때문에
옛 이름은 그대로 남았다.
아무도 수고스럽게 새로운 것을 검토해볼 필요를 느끼지 못했기 때문에
옛 관념은 그대로 남았다.
우리는 이러한 과정의 사례를 유언법의 역사에서 명료하게 살펴본 바 있다.
유언은 처음에는 단순히 재산의 양도였다.
점차 이러한 특수한 양도와 다른 모든 양도 간에 커다란 실무상의 차이가
나타나고 나서야 비로소
이들이
서로 다른 것으로 간주되기 시작했고,
그러고도 수 세기가 흐른 뒤에야
법의 개량에 나선 사람들이
허울뿐인 악취행위에 붙어있던 복잡한 절차를
청소했고 마침내
유언에 있어
유언자의 명시적 의사 외에는 다른 어떤 것도 중요하지 않다는
합의가 이루어졌던 것이다.
유언법의 초기 역사만큼의
절대적 확신을 가지고
계약법의 초기 역사를 추적할 수가 없다는 것은 유감스런 일이지만,
구속행위가 새로운 사용에 놓여짐으로써
계약이 처음 등장했고
이어서
이 실험의 중차대한 실무적 결과로써
계약이 독자적 거래형태로 승인되었다는 것을 암시하는
힌트마저 얻을 수 없는 것은 아니다.
다음과 같은 과정을 대체로 따랐을 것이라는 추측이,
그러나 그리 억지스럽지만은 않은 추측이 가능하다.
구속행위의 통상적인 방식에 의해
일정한 대금을 받고 매매가 행해진다고 가정하자.
매도인은 처분하고자 하는 목적물---가령 노예 한 명---을 가지고 온다.
매수인은 매매대금을 해당하는 구리 덩어리를 가지고 참석한다.
필수적 보조인인 저울소지자\latin{libripens}도 저울을 들고 나와있다.
노예는 정해진 요식절차에 따라 매수인에게 건네진다.
저울소지자는 구리 조각을 저울에 달고는 매도인에게 넘겨준다.
이러한 거래행위가 지속되는 한 그것은 `\wi{구속행위}'이고,
당사자들은 `피구속자'들\latin{nexi}이다.
그러나 그것이 완료되자마자,
구속행위는 끝나고,
매도인과 매수인도 그들의 일시적 관계에서 유래하는 이름으로 불리기를 그친다.
이제 여기서
거래의 역사를 한 걸음 진척시켜보자.
노예는 양도되었으나,
대금은 지불되지 않았다고 가정해보자.
\hemph{이} 경우,
매도인에 관한 한 구속행위는 종료된다.
이미 자기 물건을 넘겨주었으므로 그는 더 이상 `피구속자'\latin{nexus}가 아니다.
그러나 매수인에 관해서는 구속행위가 계속된다.
매수인 쪽에서는 거래가 끝나지 않았고 그는 여전히 `피구속자'로 남는다.
따라서 동일한 용어가 물권을 이전하는 양도를 기술\hanja{記述}함과 동시에
미지불된 매매대금에 관한 채무자의 채무도 기술하고 있음을 알 수 있다.
다시 한 걸음 더 나아가,
완전히 형식적인 거래, 즉 아무 것도 건네지지 \hemph{않고}
아무 것도 지불되지 \hemph{않는} 절차를 상상해보자.
우리는 사뭇 발달된 상거래 행위의 하나, 바로
\hemph{미이행}\hanjalatin{未履行}{executory} \hemph{매매계약}에
도달하게 되는 것이다.

\para{양도와 계약}
대중적 견해에서나 전문가적 견해에서나
\hemph{계약}이라는 것이
오랫동안 \hemph{미완의 양도}\latin{incomplete conveyance}라고
간주된 것이 사실이라면,
이 사실은 여러 모로 의미심장하다.
자연상태의 인류에 관한 지난 세기의 사변적 이론을
``원시사회에서는 물권은 아무 것도 아니었고 채권이 모든 것이었다''는
교리로 요약하는 것이 그다지 부당하지는 않을 것이다.
그러나 이제 우리는
저 명제를 거꾸로 뒤집으면
그것이 오히려 진실에 가깝다는 것을 알게 되었다.
다른 한편,
역사적으로 보면,
양도와 계약의 원시적 결합은
학자들과 법률가들에게 특별히 수수께끼로 여겨지곤 했던 어떤 것을
설명할 수 있을 것이다.
초기 고대법은 어디서나 \hemph{채무자들}을 무척 가혹하게 처우했으며,
\hemph{채권자들}에게는 막강한 권한을 주었다는 수수께끼 말이다.
구속행위가 채무자에게는 인위적으로 긴 시간 동안 지속되었음을 알고 나면,
대중들과 법이 바라보는 그가 지위가 어떤 것이었을지
더 잘 이해할 수 있는 것이다.
그의 채무상태는 틀림없이 비정상적인 것으로 여겨졌을 것이고,
지불의 해태\hanja{懈怠}는 일반적으로
간교한 책략이자 엄격법의 왜곡으로 비춰졌을 것이다.
반대로,
거래에서의 자신의 의무를 성실하게 완수한 사람은
특별한 호의로써 대우받았을 것이니,
엄격법에 따르면 연장되거나 지체되어서는 안 될
어떤 절차를 강제로 완성시킬 권한을 그에게 주는 것보다
더 당연한 일은 없어보인다.

\para{로마법의 합의 분석}
따라서 \wi{구속행위}는 원래 재산의 양도를 의미했지만,
부지불식간에 계약도 의미하게 되었고,
구속행위 개념과 계약 관념 간의 결합이 오랫동안 지속되었기에
마침내
\wi{악취행위}\latin{mancipatio}라는 특별한 용어가
진정한 구속행위, 즉 실제로 재산이 양도되는 거래를 지칭하는 데
사용되게 되었다.
그리하여 계약은 이제 양도와 분리되었고
이로써 계약법 역사의 첫 단계가 마무리되었으나,
계약당사자의 약속이 이를 둘러싼 요식성보다 더 신성\hanja{神聖}하게 평가되는
단계에 이르기까지는 아직 한참 멀리 떨어져있었다.
그 사이 기간 동안 진행된 변화의 성격을 알아보려면,
지금 우리가 다루고 있는 주제의 범위를 살짝 넘어갈 필요가 있거니와,
바로 로마 법학자들이 합의\latin{agreement}를 어떻게 분석했는지
살펴보는 것이다.
그들의 재능이 만들어낸 가장 아름다운 기념비인
이 분석에 관하여, 나는
그것이 \wi{채권채무관계}\latin{obligation}를 \wi{약정}\latin{pact}으로부터
이론적으로 분리하는 데 기초하고 있다는 것 이상을 말할
필요를 느끼지 않는다.
\wi{벤담}과 \wi{오스틴} 씨는 이렇게 주장했다.
``계약의 주요 성질은 다음 두 가지이다:
첫째,
하기로 약속하는 작위를 하겠다는,
또는
하지 않기로 약속하는 부작위를 하지 않겠다는
\wi{낙약자}\hanja{諾約者}의
\hemph{의사}의 표시.
둘째,
이 주어진 약속을 낙약자가 이행할 것이라는 데 대한
\wi{요약자}\hanja{要約者}의
\hemph{기대}의 표시.''
이것은
로마 법률가들의 법리와 거의 동일하지만,
그러나 로마 법률가들은
이러한 ``표시''의 결과를 `계약'이 아니라
`\wi{약정}'으로 보았다.
약정은 개인들 간의 합의로 맺어지는 약속의 최종 산물이지만,
그렇다고 그것이 바로 계약인 것은 아니다.
약정이 계약이 되는가 여부는
법이 그것에 채권채무관계를 덧붙이느냐 여부에 달려있다.
계약이란 `약정' \hemph{더하기} `채권채무관계'인 것이다.
약정이 채권채무관계의 옷을 입고 있지 않는 한,
\index{나약정}%
그것은 \hemph{나}약정\hanja{裸約定}, 즉 \hemph{벌거벗은} 약정이라 불렸다.

\para{로마법의 채권채무관계}
\wi{채권채무관계}\latin{obligation}란 무엇인가?
로마 법률가들의 정의에 따르면
``누군가에게 급부\hanja{給付}를 할 것이 필연적으로 강제되는
법의 사슬''\latin{juris vinculum, quo
necessitate adstringimur alicujus solvendae rei}이다.\footnote{%
  \latin{Inst.\,3.13.pr.} }
이 정의는
채권채무관계를 \wi{구속행위}와 연결짓거니와,
이들의 배경에 놓인 공통의 은유를 통해서 그러하다.
또한 이 정의는
특정 개념의 계보를 사뭇 명료하게 보여준다.
채권채무관계는 ``속박'' 또는 ``사슬''이거니와,
이로써
사람들 혹은 사람들의 집단들은
어떤 자발적 행위의 결과로서
법에 의해 하나로 결속되는 것이다.
채권채무관계를 이끌어내는 행위들은 주로
합의와 위법행위, 즉
계약과 불법행위라는 표제 아래 분류되지만,
정확하게 분류되기 힘든 여러 다른 행위들도 유사한 결과를 낳는다.\footnote{%
  `준(準)계약'과 `준(準)불법행위'를 말한다.
  준계약은 오늘날의 부당이득, 사무관리 등의 법리에 해당한다.
  \latin{Inst.\,3.27.}
  또한 절도(furtum), 강도(rapina), 재산침해(아퀼리우스법\,Lex Aquilia),
  인격침해(iniuria) 등 통상의 불법행위에 해당하지 않지만,
  재판을 자기 것으로 만든 심판인,
  무언가를 집 밖으로 내던지고 쏟아부은 경우,
  무언가를 위험하게 세워두고 걸어놓은 경우,
  선박주인^^b7여관주인^^b7마구간주인이 피용인의 불법행위에 책임을 지는 인수(引受)
  등은 준불법행위로 취급되었다. \latin{Inst.\,4.5.} }
하지만 유의할 점은
어떤 도덕적 필요성도 \wi{약정}을 바로 채권채무관계로 만들지는 못한다는 것이다.
약정에 채권채무관계의 힘을 완전히 부여하는 것은 법이다.
이 점 더욱 유의할 필요가 있거니와,
도덕적 또는 형이상학적 이론을 지지하는 근대 대륙법학자들에 의해
때로 이와 다른 법리가 주창되어왔기 때문이다.
`법의 사슬'\latin{vinculum juris}이라는 이미지는
로마 계약법과 불법행위법의 모든 부분을 물들이고 있고 지배하고 있다.
법은 당사자들을 속박하거니와,
이 \hemph{사슬}은 `변제'\hanjalatin{辨濟}{solutio}라고 불리는 과정을
통해서만 풀 수 있다.
`변제'라는 표현도 은유적이거니와,
``지불''\latin{payment}이라는 일상용어는 가끔씩 그리고 우연히
여기에 들어맞을 뿐이다.
이들 은유적 이미지의 일관성은
이것이 없었다면 혼란을 초래했을
로마법 용어의 특별한 의미를 이해할 수 있게 해준다.
즉, ``\wi{채권채무관계}''\latin{obligation}는 의무뿐만 아니라
권리도 의미하는 것이다.
이를테면 빌린 돈을 지불할 의무뿐만 아니라
빌려준 돈을 지불받을 권리도 의미한다.
실로 로마인들의 눈 앞에는 ``법의 사슬''의 전체 그림이
펼쳐져있었으며,
사슬의 한쪽 끝을 다른 쪽 끝보다 더 많이 혹은 더 적게
바라보지 않았다.

\para{약정과 계약}
발달된 로마법에서는
\wi{약정}이 체결되자마자 거의 모든 경우
즉시
채권채무관계의 관\hanja{冠}이 씌워지고, 따라서 계약이 된다.
이것은 분명 계약법이 지향하는 결과이다.
그러나 우리의 탐구의 목적을 위해서는
그 중간 단계, 즉 채권채무관계가 되려면 완전한 합의 이상의 어떤 것이
요구되는 단계에 주목해야 한다.
이 단계는
네 종류의 계약---\wi{언어계약}, \wi{문서계약}, \wi{요물계약}, \wi{낙성계약}---을 구분한,
저 유명한 로마법상의 분류가 사용되던 시기와 일치한다.
이 시기 동안 법에 의해 강제된 약속은 저 네 가지에 국한되었다.
채권채무관계를 약정과 분리하는 이론을 알고 있다면
저 네 가지 항목의 의미는
쉽게 이해될 수 있다.
사실,
계약들의 각 항목은
계약당사자들의 단순한 합의 이외에 어떤 요식성이 필요한가에 따라
이름붙여진 것이다.
언어\latin{verbal}계약에서는 약정이 체결되는 순간
일정한 방식의 말들이 발설되어야 ``법의 사슬''이 부여된다.
문서\latin{literal}계약에서는
원장\hanja{元帳}, 즉 회계장부에 기입되어야
\wi{약정}에 \wi{채권채무관계}의 옷이 입혀진다.
요물\latin{real}계약의 경우
예비적 약속의 목적물인
물건의 \wi{인도}\hanja{引渡}가 있어야 동일한 결과가 뒤따른다.
요컨대,
이 모든 경우
계약당사자들 간에는 의사합치가 있어야 하지만,
거기에만 그친다면 그들은 서로에게 \hemph{채권채무}를 갖지 못하고,
따라서 이행을 강제할 수도,
신의\hanja{信義} 위반을 이유로 배상을 청구할 수도 없다.
그러나 그들이 어떤 정해진 요식성을 충족시킨다면,
계약은 바로 체결되고,
그 계약의 이름은 그들이 채택한 특정한 방식에 따라 붙여지는 것이다.
이러한 관행에 대한 예외는 조금 있다 살펴보겠다.

\para{언어계약}
나는 네 가지 계약들을 역사적 순서에 따라 열거했으나,
로마의 \wi{법학제요} 저자들이 이 순서를 반드시 따른 것은 아니다.
\wi{언어계약}이 네 가지 중에 가장 오래된 것임에는 의심의 여지가 없다.
그것은 원시적 \wi{구속행위}의 후손으로 알려진 것 중에 가장 먼저 나타났다.\footnote{%
  <<고대법>>에 대한 폴록의 주석에 따르면,
  문답계약의 기원은 구속행위(nexum)가 아니라
  선서(oath)와 같은 초기의 종교적인 채권채무관계에서 찾는 것이
  오늘날
  일반적이라 한다.
  }
언어계약에 속하는 몇몇 종\hanja{種}이 옛날에는 사용되었으나,
가장 중요한 것이자 우리의 전거들이 다루었던 유일한 것은
질문과 답변으로 이루어진 `\wi{문답계약}'\latin{stipulation}이다.
\wi{요약자}가 질문을 하고 \wi{낙약자}가 답변을 한다.
이러한 질문과 답변이야말로, 전술했듯이,
원시적 관념이
당사자들 간의 단순한 합의를 넘어 추가적으로 요청하는 요소인 것이다.
이들은 채권채무관계가 부여되기 위한 매개체이다.
옛 구속행위는
보다 성숙한 법학에게 무엇보다
계약당사자들을 결속시키는 사슬의 개념을 물려주었으니,
이것이 이제 \wi{채권채무관계}가 되었다.
그것은 또한 약속에 수반하면서 약속을 성별\hanja{聖別}하는
의례행위의 개념도 물려주었으니,
이 의례행위가 이제 질문과 답변으로 변형된 것이다.
초기 구속행위의 특징이었던 엄숙한 양도행위가
어떻게 단순한 질문과 답변으로 전환되었을까 하는 것은
이와 유사한 로마 유언법의 역사가 우리에게 가르쳐준 바가 없었다면
더욱 미스테리로 남았을 것이다.
유언법의 역사를 돌아보면,
실질적 관심 대상에 직접 관련되는 절차 부분\footnote{%
  `양도'와 대비되는 `언명'(nuncupatio)을 말하는 듯하다.
}으로부터
어떻게
형식적 양도가
처음 분리되었는지를,
그리고 어떻게 그후 이것을 완전히 생략하게 되었는지를
이해할 수 있다.
그렇다면
\wi{문답계약}의 질문과 답변은 단순화된 형태의 \wi{구속행위}였을 것이 분명하고,
따라서
그것은 오랫동안 법기술적 형태의 성질을 가졌을 것이라고 쉽게 추정할 수 있다.
옛 로마 법률가들이 문답계약을 옹호했던 이유를
합의에 임한 당사자들에게 숙고하고 성찰할 기회를 제공하는
유용성때문이라고 본다면 이는 잘못일 것이다.
물론 이런 유\hanja{類}의 가치가 있었고 점점 중요하게
인식되었다는 것을 부인할 수는 없다.
그러나, 우리의 전거들에 나타난 증거에 비추어볼 때,
계약에 관련된 그것의 기능은 처음에는 형식적이고 의례적인 것이었다.
문답계약을 구성할 수 있는 오래된 질문과 답변은
특정한 경우에 적합한 법기술적 용어들을 사용한
질문과 답변에만 국한되었고,
아무 질문이나 답변이라고 해서 다 되는 것이 아니었다.

\para{문답계약}
그러나,
비록 \wi{문답계약}이 유용한 안전장치로 인식되기 이전에
엄숙한 형식으로 인식되었다고 이해하는 것이
계약법의 역사를 올바르게 평가하는 데 필수적이라 할지라도,
다른 한편
그것의 현실적 유용성에 눈을 감아버리는 것도
잘못된 일일 것이다.
언어계약은, 비록 고법\hanja{古法}상에서 누리던 중요성을
상당 부분 상실해갔지만, 로마법의 마지막 시기까지 계속 살아남았다.
당연한 말이겠지만,
로마법의 어떤 제도도
어떤 현실적 유용성이 없었다면
그렇게 오래 유지될 수 없었을 것이다.
놀랍게도
어떤 영국 학자의 말에 따르면,
로마인들은
초창기부터도
숙고없이 서둘러 계약을 맺는 것에 대한 방비가 거의 없어도
괘념치 않았다고 한다.
그러나 문답계약을 면밀히 조사해보면,
그리고 서면증거를 만들기 쉽지 않았던 당시의 사회상태를 감안하면,
\wi{문답계약}의 질문과 답변은,
만약 그것이 실제 기여한 목적을 위해 의도적으로 고안되었다면,
그야말로 천재적인 방책이었다고 평가해도 좋다고 생각한다.
\hemph{\wi{요약자}}\latin{promisee}가
계약의 모든 조항을 질문의 형태로 만들어 질문하면,
\hemph{\wi{낙약자}}\latin{promisor}가 답변을 한다.
``당신은 이러이러한 노예를 이러이러한 장소에서 이러이러한 날짜에
나에게 인도할 것을 약속하는가?''
``약속하노라.''
잠깐만 생각해보면,
이렇게 질문 형태로 구성되는 \wi{채권채무관계}는
당사자들의 자연스런 입장을 거꾸로 뒤집고,
대화의 통상적인 흐름을 깨뜨리는 효과를 가져와,
위험한 약속에 빠지지 않도록 주의를 환기시키는 기능을 함을 알 수 있다.
우리 영국인들이 행하는 구두의 약속은
오직 약속자\latin{promisor}의 말로 구성되는 것이 일반적이다.\footnote{%
  원어로는 똑같이 `promisor'이지만,
  로마법의 맥락에서는 `낙약자'(promissor)로,
  영국법의 맥락에서는 `약속자'로 번역하고 있음을 유의할 것.
  마찬가지로 `promisee'는 로마법의 맥락에서는 `요약자'(stipulator)로,
  영국법의 맥락에서는 `수약자'로 번역된다.
  사실 `요약자'니 `낙약자'니 하는 우리의 법률용어는
  바로 로마법의 문답계약에서 유래하는 말이다.
  }
옛 로마법에서는 또 하나의 단계가 반드시 요구된다.
합의가 이루어지고 나면
\wi{요약자}가 엄숙한 질문의 형태로 이 합의의 모든 조항들을 요약해야 하는 것이다.
재판에서 증거로 제출되는 것은,
구속력 없는 약속 자체가 \hemph{아니라},
바로 이 질문과 그에 대한 답변인 것이다.
일견 사소해보이는 이 차이가
계약법 용어에 얼마나 큰 차이를 만들어내는지는
로마법 입문자들이 입문 후 금세 깨닫게 되는 것이니,
그들은 첫 번째 걸림돌을 거의 언제나 여기서 만나고 있다.
우리가 영어로 어떤 계약에 관해 말하면서
그것을 편의상 한쪽 당사자와 결부시키는 경우---가령
어떤 계약의 당사자에 대해 일반적으로 말하려는 경우---우리의
말이 지시하는 것은 언제나
\hemph{약속자}\latin{promis\textit{or}} 쪽이다.
그러나 로마법의 일반적 언어는 방향이 반대이다.
로마법은 계약을 언제나, 이런 용어를 쓸 수 있다면,
\hemph{수약자}\hanjalatin{受約者}{promis\textit{ee}} 쪽에서 바라본다.
계약의 당사자 중에서
주로 언급되는 대상은 언제나 \wi{요약자}\latin{stipulator}, 즉
질문을 하는 사람이다.
하지만
\wi{문답계약}의 유용성이 자못 생생하게 드러나는 예를
라틴 희극작가들의 희곡의 몇몇 페이지에서도 찾아볼 수 있다.
이 대목들이 나오는 장면 전체를 읽어보면
\paren{예컨대, Plautus, \textit{Pseudolus}, 1막 1장;
4막 6장; \textit{Trinummus}, 5막 2장},
질문하는 것이 계약에 임한 사람의 주의를 얼마나 많이 이끌어내는지,
그리고
즉흥적인 합의에 이르지 않게 할 가능성이 얼마나 커지는지
알 수 있을 것이다.

\para{문서계약}
\wi{문서계약}에서 \wi{약정}에 \wi{채권채무관계}가 입혀지기 위해 필요한
요식행위는,
채무액이 확정될 수 있는 경우,
채무의 총액을
원장\hanja{元帳}의 차변\hanja{借邊}에 기입하는 것이었다.
이 계약은 로마의 가\hanja{家}의 관행에 의해 설명될 수 있거니와,
고대에는 그것이 체계적인 성격을 띠었고 무척 규칙적으로 장부작성이
이루어졌던 것이다.
로마 고법\hanja{古法}과 관련하여
가령 노예의 \wi{특유재산}\latin{peculium}의 성격 같은
몇몇 작은 문제들이 있거니와,
이는
로마의 가\hanja{家}가 가부장에게 엄격히 책임지는 다수의 사람들로
구성되었고,
가의 수입과 지출의 모든 항목은,
일단 일지\hanja{日誌}에 기재된 후,
정해진 시기에
가의 총\hanja{總}원장에 이기\hanja{移記}되었음을 상기할 때
비로소 해소될 수 있다.
하지만 문서계약에 관해 남아있는 기술\hanja{記述}에는
몇 가지 모호한 점이 있거니와,
사실
나중에는
장부작성의 습관이
보편적이지 않게 되었고,
``문서계약''이라는 표현은 이제 원래 가졌던 의미와 완전히
다른 형태의 계약을 지칭하게 되었던 것이다.
따라서 우리는
초기의 \wi{문서계약}에서
채권채무관계가 단순히 채권자 측의 기입만으로 성립했는지,
아니면
그것이 법적 효력을 가지려면
채무자의 동일한 행위, 즉 채무자 측 장부의 상응하는 기입도 필요했는지
말할 수 있는 입장에 있지 않다.
하지만
이 계약의 경우
한 가지 조건만 충족되면 다른 모든 요식성은 필요치 않다는
핵심적 성격만은 확실히 알려져있다.
이것은 계약법의 역사에서 또 한 걸음의 진전이었던 것이다.

\para{요물계약}
역사적 순서에 따라 다음에 등장하는 계약인 \wi{요물계약}은
윤리 개념의 큰 진전을 보여준다.
어떤 합의가 물건의 \wi{인도}를 목적으로 한다면---이는
대다수의 단순한 계약을 포괄한다---그 인도가 실제로 행해지는 즉시
\wi{채권채무관계}가 성립하는 것이다.\footnote{%
  요물계약에는 소비대차(mutuum),
  사용대차(commodatum), 임치(depositum), 입질(pignus) 등이 속했다.
  현행 민법에서는 입질(질권설정계약)을 제외하면
  모두 낙성계약으로 되어있다. }
이러한 결과는 초기 계약 관념에 큰 혁신을 의미했다.
의심의 여지 없이 초창기에는,
계약의 당사자가 자신의 합의에 문답계약의 옷을 입히지 못했다면,
계약의 이행으로써 무엇을 했던지 간에
법은 아무 것도 인정해주지 않을 것이기 때문이다.
공식적으로 \hemph{\wi{문답계약}}을 체결하지 않았다면
돈을 빌려주었더라도 빌린 돈을 갚으라고 소송할 수 없었던 것이다.
그러나 요물계약에서는
일방의 이행이 상대방에게 법적 의무를---분명 윤리적 근거에서---부과한다.
그리하여 도덕적 고려가 계약법의 요소로 처음으로 등장한 것이다.
요물계약이
앞서 살펴본 두 가지와 다른 점은,
법기술적 방식이나 로마의 가\hanja{家}의 습관에 대한 존중이 아니라,
도덕적 고려에 기초한다는 데 있다.

\para{낙성계약}
이제 네 번째 유형, 즉 \wi{낙성계약}에 이르렀거니와,
이것은 가장 흥미롭고 가장 중요한 유형이다.
여기에는 네 가지 계약들이 속했고, 그 이름은 다음과 같다:
위임\latin{mandatum}, 조합\latin{societas},
매매\latin{emtio-venditio}, 임약\hanjalatin{賃約}{locatio-conductio}.\footnote{%
  `임약'은 우리 민법의 `임대차' `고용' `도급'을 포괄하는 개념이다. }
몇 페이지 앞에서
계약이란 약정에 채권채무관계가 덧붙여진 것이라고 말한 후,
나는
약정이 채권채무관계로 되기 위해 법이 요구하는
어떤 행위나 요식성에 관해 이야기했다.
나는 일반적 표현의 장점을 활용해 이 말을 하였으나,
적극적인 것 외에 소극적인 것까지 포괄하는 것으로 이해하지 않으면
전적으로 옳은 말이 되지는 못한다.
기실,
낙성계약의 특이성은 \wi{약정} 외에 그 어떤 요식성도
\hemph{전혀} 요구되지 않는다는 것이기 때문이다.
낙성계약에 관하여 많은 옹호될 수 없는 것들이,
더 많은 모호한 것들이 주장되어왔거니와,
심지어 낙성계약에서는 당사자들의 \hemph{동의}\latin{consent}가
다른 어떤 합의 유형들보다 더 강하게 주어진다는 주장까지 있었다.
그러나 저 `낙성'\hanjalatin{諾成}{consensual}이라는 용어는
여기서는
단지 \hemph{합의}\latin{consensus}만 있으면 바로 \wi{채권채무관계}가
부가된다는 것을 의미할 뿐이다.
합의, 즉 당사자들의 상호 동의는
약정에 있어 최종의 그리고 최고의 요소이다.
당사자들의 동의가 이 요소를 제공하자마자
\hemph{즉시} 계약이 성립하는 것은
매매, 조합, 위임, 임약의 네 가지 표제 중 하나에 속하는
합의의 특유한 성질이다.
이 합의는 바로 채권채무관계를 끌고 들어오기에,
특정 종류의 거래에서 그것이 행하는 기능은
다른 종류의 계약에서 물건이, 문답의 언어가,
문서, 즉 장부에의 기입이 행하는 기능과 정확히 일치한다.
따라서 `낙성'은
조금도 이상할 것이 없는 용어이며,
`요물' `언어' `문서'에 정확히 대응한다.

일상생활에서
가장 흔하고 가장 중요한 계약은, 말할 것도 없이,
네 가지 이름의 계약을 포괄하는 \wi{낙성계약}이다.
어떤 공동체든 집단생활의 대단히 큰 부분이
사고 파는 거래, 세\hanja{貰}놓고 세드는 거래,
공동사업을 위해 사람들이 결합하는 거래,
업무처리를 타인에게 맡기는 거래로 구성된다.
바로 그 때문에
다른 많은 사회들과 마찬가지로 로마도
이들 거래에서 법기술적 장애물을 제거하여,
사회적 운동의 효율적 동력이
가능한 한
방해받지 않도록 했을 것이다.
물론 이러한 동기는 로마에만 국한된 것이 아니었다.
로마인들과 이웃 민족들 간의 거래는
우리가 말한 계약들이 어디서나 \hemph{낙성계약},
즉 상호 동의의 의사표시만으로 구속력이 부여되는 계약이
되어가는 것을 관찰할 수 있는
풍부한 기회를
로마인들에게
제공했을 것이다.
그리하여 그들의 통상적인 관행에 따라
로마인들은 이들 계약을
\wi{만민법}상의\latin{juris gentium} 계약으로 분류했다.
하지만 나는
아주 초기부터 이렇게 불리지는 않았다고 생각한다.
만민법\latin{jus gentium}이라는 관념은
외인\hanja{外人}담당법무관\latin{praetor peregrinus}이 임명되기 오래 전부터
로마 법률가들의 정신에 이미 들어있었다.
그러나 다른 이탈리아 공동체들의 계약 관행에 로마인들이 익숙해진 것은
수많은 거래가 일상적으로 이루어지면서부터일 것이고,
이러한 거래는 이탈리아가 완전히 평정되어
로마의 패권이 확고해지고 나서
비로소 대규모로 이루어질 수 있었을 것이다.
하지만, 비록
낙성계약이 가장 늦게 도입된 것일 확률이 무척 크다고 할지라도,
그리고
`만민법상의'\latin{juris gentium}라는 수식어가 그것의 뒤늦은 도입을
나타내는 것이라 하더라도,
낙성계약을 ``만민법''\latin{law of nations}에 귀속시키는
바로 이 표현이 근대에 들어서는
그것이 아득한 옛날의 것이라는 관념을 만들어냈다.
``만민법''\latin{law of nations}이
``자연법''\latin{law of nature}으로
전환되자,
저 표현은
낙성계약이 자연상태에 가장 부합하는 합의 유형임을
의미한다고 여겨졌던 것이다.
그리하여 문명의 나이가 어릴수록
계약의 형태는 더 단순할 것이라는 이상한 믿음이 형성되었다.

\para{자연법적 채무와 시민법적 채무}
전술했듯이 낙성계약에 속하는 계약들은 그 수가 대단히 제한적이었다.
그러나
낙성계약으로 대표되는 계약법 발달의 단계가
계약에 관한 모든 근대적 관념의 출발점이었음은 의심할 여지가 없다.
이제 합의를 구성하는 의사\latin{will} 작용은
다른 것들과 완전히 분리되어 독립적 고찰의 대상이 되었다.
계약 관념에서 방식은 완전히 제거되었고,
외적 행위는 오직 내적 의사의 징표로만 간주되었다.
더욱이 \wi{낙성계약}은 \wi{만민법}\latin{jus gentium}으로 분류되었으니,
이러한 분류는
얼마 안 가
낙성계약이야말로
자연에 의해 승인되고 자연의 법전에 포함된
계약을 대표하는 합의 유형이라는 추론을 형성시켰다.
이 지점에 이르러, 우리는
로마 법률가들의 몇몇 유명한 법리와 구분들을 만나게 된다.
그중 하나가 자연법상의 채무\latin{natural obligation}와
\wi{시민법}상의 채무\latin{civil obligation}의 구분이다.
지적으로 완전히 성숙한 어떤 사람이
자신의 의사에 기해
어떤 약속을 맺었다면,
비록 필요한 방식을 다 갖추지 못했더라도,
또는 어떤 법기술적 장애로 인해
유효한 계약을 체결할 법적 능력이 결여되어 있었다 할지라도,
그는 \hemph{\wi{자연채무}}\latin{natural obligation}를 진다.
법은
\paren{바로 이것이 저 구분이 의미하는 바이다}
이런 채무를 강제하지 않는다.
그러나 법이 이런 채무를 전혀 인정하지 않은 것은 아니다.
\hemph{자연채무}는
단순히 무효인 채무와는 여러 모로 달랐거니와,
특히 계약 체결 능력을 사후적으로 취득한다면
시민법적으로도 인정될 수 있었던 것이다.\footnote{%
  가령 노예가 진 빚은 자연채무였다. 따라서 노예가 해방된 뒤
  스스로 빚을 갚으면 반환받을 수 없었다.
  \latin{D.\,12.6.13.pr.}
  마찬가지로 후견인의 조성(助成) 없이 미성숙자가 돈을 빌려 이득을 본 경우,
  그가 성숙기에 달한 후 갚으면 반환받을 수 없다.
  \latin{D.\,12.6.13.1.}
  }
법학자들의 또 하나의 특유의 법리는 약정이
계약의 법기술적 요소로부터 분리된 후에 비로소 등장할 수 있었다.
그들에 따르면,
계약만이 \hemph{소권}\hanjalatin{訴權}{action}을 근거지울 수 있었지만,
단순한 \wi{약정}은 \hemph{\wi{항변권}}\hanjalatin{抗辯權}{plea}의 기초가 될 수 있었다.
그리하여,
누구도
계약을 성립시키는 데 필요한
방식을 갖추지 못한
합의에 기초하여 소송을 제기할 수 없었지만,
유효한 계약에 기초한 주장이라도
단순한 약정 상태에 불과한 다른 합의가 있었음을 입증함으로써
이를 물리칠 수 있었다.
가령 금전채무의 회수를 구하는 소송은
채무 면제나 유예의
단순한 비공식적 합의를 입증하여 이에 대항할 수 있었다.

\para{계약법의 변화}
방금 언급한 법리는 법무관들이 궁극적 혁신을 이루어내는 데
방해물로 작용했을 것이다.
그들의 자연법 이론은
낙성계약과 이를 포함하는 약정 일반에 대해
특별한 호의를 갖도록 그들을 이끌었을 것이 틀림없다.
그러나 그들이 즉시
모든 약정에 낙성계약의 효력을 확대적용하는 모험을 감행한 것은 아니었다.
그들은 로마법 초기부터 그들에게 주어졌던
소송절차에 대한 감독권한을 이용하였으니,
방식을 갖추지 못한 계약에 근거한 소송은 여전히 허가하지 않았지만,
합의에 관한 새로운 이론을 적극 활용하여
이후의 발달 단계로 향하는 길을 텄다.\footnote{%
  이른바 `법무관법상의 약정'(pacta praetoria)을 말하는 듯하다.
  특정 기일에 자기 또는 타인의 기존 채무를 변제하겠다는 약정,
  중재인, 은행업자, 선박주인^^b7여관주인^^b7마구간주인 등의 인수(引受)약정
  따위가 여기에 속한다. }
그러나
여기까지 나아가자
계속 더 나아가는 것이 불가피해졌다.
고대 계약법의 혁명이 달성된 것은,
어느 해인가 \wi{법무관}의 고시\hanja{告示}가
계약의 옷을 입지 못한 \wi{약정}이라도
그 약정이 \wi{대가관계}\latin{consideration}\paren{원인\latin{causa}}에
기초하는 것인 한
\wi{형평법}상의 소송을 허가하겠노라고 공표했을 때였다.\footnote{%
  이른바 `무명요물계약'(無名要物契約)을 말한다.
  쌍무성(synallagma) 있는 계약의 당사자라면
  자신의 급부를 이행한 경우---따라서 일종의 요물계약이었다---상대방의
  이행을 소구할 수 있었다.
  혹은 자신의 급부를 돌려달라는 부당이득반환청구소송을 제기할 수 있었다.
  로마 법학자들에 따르면 무명요물계약은 다음 네 가지 유형을 포괄했다.
  `네가 주도록 내가 준다'(do ut des)
  `네가 하도록 내가 준다'(do ut facias)
  `네가 주도록 내가 한다'(facio ut des)
  `네가 하도록 내가 한다'(facio ut facias).
  결국 `주는 채무'과 `하는 채무'가 쌍무적으로 견련되는 모든 유형의 약정에
  소권이 주어질 수 있었다.
  \latin{D.\,19.5.5.} }
이런 종류의 약정은 발달된 로마법에서는 항상 강제되었다.
이 원리는
\wi{낙성계약}의 원리가 그 합당한 결과에 도달한 것에 불과했다.
사실, 로마인들의 법기술적 용어가 그들의 법이론만큼이나 유연했다면,
법무관에 의해 강제된 이들 약정은
새로운 계약, 새로운 낙성계약이라고 불렸을지도 모른다.
하지만 법용어는 가장 바뀌기 어려운 법이어서,
형평법적으로 강제된 저 약정들은
여전히 `\wi{법무관법상의 약정}'\latin{praetorian pacts}이라고만 지칭되었다.
만약 약정에 \wi{대가관계}가 없다면,
\index{나약정}%
새로운 법에서도 계속 \hemph{나}약정\hanja{裸約定}이었음을
유의해야 한다.
이것에 법적 효력을 부여하려면,
문답계약을 통해 언어계약으로 전환시켜야 했다.

\para{계약법의 진화}
수많은 오해에 대한 방패막이로서 큰 중요성을 가지기에
계약법의 역사를
이렇게
길게 논하고 있는 것도 이해해주시리라 믿는다.
그것은 하나의 중대한 법관념으로부터
다른 중대한 법관념으로의 관념들의 행진을 완전히 설명해준다.
우리는 구속행위로부터 시작하였으니,
여기서는 계약과 양도가 혼재되어 있고,
합의에 수반되는 요식성이 합의 자체보다 훨씬 중요하다.
구속행위 다음에 오는 \wi{문답계약}은 옛 의례행위의 단순화된 방식이다.
그 다음의 \wi{문서계약}에서는
로마의 가\hanja{家}의 엄격한 관행에 의해
합의가 입증되기만 하면 다른 모든 요식성은 포기된다.
\wi{요물계약}에서는 도덕적 의무가 처음으로 인정되니,
계약의 일부 이행을 수령하거나 묵인한 자는
방식의 흠결을 이유로 계약을 부인하는 것이 금지된다.
끝으로 \wi{낙성계약}이 등장함으로써,
계약 당사자들의 내심의 의사만 고려대상이 되고
외적 격식은 내적 의사의 증거로서만 의미를 갖는다.
조야한 개념에서 세련된 개념으로 나아가는 로마법의 이러한 관념의 진보가
계약에 관한 인간 사고의 필연적 진보과정을 얼마나 예시하고 있는지는
물론 확인할 수 없다.
로마를 제외한 다른 모든 고대사회는
계약법이 너무 부족하여 정보를 얻을 수 없거나,
아니면 계약법 자체가 아예 없다.
또한 근대법은 철저히 로마법의 관념을 이어받은 것이어서
가르침을 구할 만한 비교대상이 되지 못한다.
하지만 우리가 살펴본 고대 로마계약법의 역사에는
억지스럽거나 놀랍거나 불가해한 것이 전혀 없기에,
어느 정도는 그것이
다른 고대사회의 계약법 개념의 역사에도 통용된다고
보아도 불합리하지 않을 것이다.
그러나 로마법의 진보가 다른 법체계의 진보를 대표하더라도,
그것은 어느 정도까지만 그렇다는 것이지 그 이상은 아니다.
자연법 이론은 로마법에만 있었다.
`법의 사슬'이라는 관념도, 내가 알고 있는 한,
로마법에만 있었다.
로마의 성숙한 계약법과 불법행위법의 많은 특징들은
이 두 가지  관념이 따로 혹은 함께 작용한 결과이거니와,
따라서 특정한 사회 하나만의 산물인 것이다.
이 후대의 법관념이 갖는 중요성은,
어떤 상황에서도 진보적 사고의 필연적 결과를 대표한다는 데
있는 것이 아니라,
근대 세계의 지적 기질에 엄청나게 큰 영향력을 행사했다는 데 있다.

\para{로마제국의 법적 사고, 동방과 서방의 관념}
로마법, 특히 로마계약법이
다양한 학문의
사고양식, 추론과정, 전문용어에 기여한 것보다
더 대단한 일이 또 있는지 모르겠다.
자연과학을 제외하고,
근대인의 지적 욕구를 자극한 대상 중에
로마법이라는 여과장치를 통과하지 않은 것은 거의 없다.
순수한 형이상학은 물론 로마보다는 그리스의 후예이지만,
정치학, 도덕철학, 심지어 신학까지,
모두가 로마법에서 표현수단을 발견했을 뿐만 아니라,
거기서 학문 발달을 위한 깊이있는 탐구가 배양되는 거점도 발견했다.\footnote{%
  이 단락부터 이 챕터의 거의 끝까지는
  정치학, 도덕철학(윤리학), 신학에 끼친
  로마법의 영향, 특히 로마계약법의 영향에 대한 논의가 이어진다.
  }
이런 현상을 설명하기 위해,
말과 관념 간의 불가사의한 관계를 논할 필요는 전혀 없을 것이며,
또한
적절한 언어의 저장고와 적절한 추론의 장치가 미리 주어지지 않으면
인간 정신은 어떠한 사고 주제도 다룰 수 없었다는 것을 설명할
필요도 전혀 없을 것이다.
동방과 서방의 철학적 관심이 분리되었을 때,
서방 사상의 기초자들은 라틴어로 말하고 라틴어로 사고하는 사회에
속해있었다는 것만 말해도 충분할 것이다.
그런데
서방에서는
철학적 목적을 충족하는 정확성을 갖는 유일한 언어가
로마법의 언어였거니와,
비속\hanja{卑俗}라틴어가 불길한 만족\hanja{蠻族}들의 방언으로 전락해가는 동안,
로마법의 언어는 특별한 행운으로
아우구스투스 시대의 순수함을 거의 그대로 보존할 수 있었다.
로마법이 언어의 정확성을 위한 유일한 수단이었다면,
사고의 정확성과 명석함과 깊이를 위한 유일한 수단은 더더욱 로마법이었다.
서방에서는
적어도 3백녁 동안 철학과 학문이 자리잡지 못하고 있었다.
비록 형이상학과 형이상학적 신학이 로마인 백성들의 정신적 에너지를
독점하고 있었지만,
이러한 열렬한 탐구에 사용된 언어는 오직 그리스어였고,
그러한 탐구의 무대는 동로마제국이었다.
사실,
때로 동로마의 논쟁들의 결과는 사뭇 중요해져서
그것에 찬성하고 반대하는 모든 이들의 의견이 기록되어졌다.
그후 이러한 동방의 논쟁의 결과가 서방에 소개되었으니,
대체로 그것은
감흥없이 그리고 저항없이 받아들여졌다.
그러는 동안,
가장 근면한 자에게도 어렵고,
가장 명석한 자에게도 멀고,
가장 치밀한 자에게도 까다로운
학문 분야 하나가 서방의 식자층 사이에서
매력을 잃지 않고 있었다.
아프리카, 에스파니아, 갈리아, 북이탈리아의 교양있는 시민들에게
그것은 법학, 오직 법학이었으니,
그것은 시와 역사, 철학과 학문을 대신하는 것이었다.
서방 사상의 초기 성과의 명백히 법적인 양상에는
불가사의한 것이 거의 없었으므로,
그것이 다른 의미로 다가왔다면 오히려 놀라운 일이었을 것이다.
나로서는
어떤 새로운 요소의 존재로 인해 생겨난
서방과 동방의 관념의 차이가,
서방의 신학과 동방의 신학의 차이가,
그동안 거의 주목을 받지 못했다는 것이
놀라울 따름이다.
이렇게 법학의 영향이 강력해지기 시작하기 때문에,
콘스탄티노플의 건설과 이후 서로마제국과 동로마제국의 분리가
철학의 역사에서 획을 긋는 사건이 되는 것이다.
그러나 대륙의 사상가들은
분명 이 중차대한 국면의 중요성을 인식하기 어려운 위치에 있으니,
그들은 로마법에서 유래한 관념들에 친숙하고
그것이 일상적 관념들에 섞여들어가 있기 때문이다.
반면, 영국인들은 놀라울 정도로 그것을 알지 못하니,
근대 지식의 가장 풍부한 원천,
로마 문명의 유일한 지적 성과로부터 스스로를 유폐한 것이다.
동시에,
고전기 로마법에 친숙해지는 데 많은 노력을 들이는
영국인이라면,
지금까지 영국인들이 이 분야에 무심했다는 바로 그 사실 덕분에,
내가 감히 내놓는 주장의 가치에 관해
프랑스인이나 독일인보다
더 나은 판관이 될 수 있을 것이다.
로마인들이 실제 관용한 로마법이 무엇인지 아는 사람은,
그리고 초기 서방의 신학과 철학이 그 이전의 사상 국면과 어떻게
달랐는지 아는 사람은,
사변\hanja{思辨}을 지배하기 시작한 새로운 요소가 무엇이었는지
선언할 수 있는 위치에 있다고 할 것이다.

\para{준계약}
로마법의 여러 영역 중에 다른 학분 분야에 가장 큰 영향을 끼친 것은
채권법, 혹은, 거의 같은 말이지만, 계약법과 불법행위법이었다.
로마인들도
이 법영역에 속하는 용어가 감당하게 될 역할을 모르지 않았거니와,
그것은 그들이 특유의 \hemph{준}\hanjalatin{準}{quasi}이라는
수식어를 `\wi{준계약}'\latin{quasi-contract}과
`\wi{준불법행위}'\latin{quasi-delict} 같은 표현에 사용한 것을 보면 알 수 있다.
여기서 ``준''이라는 말은 분류를 위한 용어일 뿐이다.
흔히 영국의 학자들은 준계약을 \hemph{묵시적}\latin{implied} 계약과 동일시해왔으나,
이는 잘못이다.
묵시적 계약은 진짜로 계약이지만 준계약은 그렇지 않기 때문이다.
명시적 계약에서 말로써 상징되는 것과 동일한 요소가
묵시적 계약에서는 행위와 상황에 의해 상징되거니와,
어떤 이가
어느 쪽 상징집합을 사용하든
합의의 이론에 관한 한
아무런 차이가 없다.
그러나 준계약은 계약이 아니다.
이 유형에 해당하는 가장 흔한 사례는
한 사람이 다른 사람에게 착오로 돈을 지불한 경우 두 사람 간의 관계에서 발견된다.
법은, 도덕의 관점에서,
반환할 채무를 수령자에게 지운다.
그러나 그 성질은 계약에 기초하는 것이 아니니,
계약의 본질적 요소인 \wi{약정}\latin{convention}이 결여되어 있기 때문이다.
로마법의 어떤 용어에 붙는
``준''이라는 말은 그것이 지시하는 개념이
비교대상이 되는 개념과 강한 외관상의 유비\hanja{類比} 혹은 유사성으로
연결되어 있다는 것을 의미한다.
두 개념이 동일하다거나, 혹은 동일한 유\hanja{類}에 속한다는 말이 아니다.
오히려
그것들 간의 동일성을 부정하는 의미가 들어있다.
그러나 그것들은 충분히 유사해서
하나가 다른 하나의 속편\hanja{續篇}으로
분류되고,
하나의 법영역의 용어를 다른 법영역에도 쓸 수 있어서,
그렇지 않으면 불완전하게 표현될 수밖에 없는 법규칙의 진술에
과도한 왜곡 없이 사용할 수 있다는 뜻이다.

\para{준계약과 사회계약}
진짜 계약인 묵시적 계약과 계약이 아닌 \wi{준계약} 간의 혼동이
정치적 권리와 의무의 원천을
통치자와 피치자 간의 원초적 계약에서
찾는
저 유명한 오류와 공통점이 많다는 예리한 지적이 있다.
이 이론이 확립되기 오래 전부터도
주권자와 백성들 간에 존재한다고 여겨지는
권리와 의무의 상호성을 기술하는 데
로마계약법의 용어가 자주 사용되어왔다.
무조건적인 복종을 요구할 수 있는 왕의 권리를 적극적으로 내세우는
격률들---신약성서에서 기원한다고 주장되었으나 실은
황제들의 전제정에 대한 기억이 지속된 데서 유래한 격률들---은
세상에 가득했지만,
피치자들이 갖는 상응하는 권리의 인식은,
만약
아직 제대로 발달하지 못한 관념을 암시하는 언어를
로마채권법이
제공해주지 않았다면,
그것을 표현할 수단이 전혀 없었을 것이다.
왕의 특권과 백성들에 대한 그의 의무 간의 대립은
서양 역사가 시작된 이래 한번도 잊혀진 적이 없다고 믿지만,
봉건제도가 굳건히 존속하는 동안은
아주 예외적인 극소수의 사상가를 제외하고는
이 문제에 관심을 두지 않았다.
\wi{봉건제}의 명백한 관습에 의해
유럽의 대부분 주권자들의 터무니없는 이론적 주장이
효과적으로 통제되었기 때문이다.
하지만,
주지하듯이,
봉건체제의 붕괴로 중세의 헌정질서가 혼란에 빠지자,
그리고
종교개혁으로 교황의 권위가 추락하자,
\wi{왕권신수설}\hanja{王權神授說}은
과거에 한번도 누려보지 못한
중요한 이론의 지위로
급부상했다.
이 이론이 얻은 인기는
로마법 용어에 상시 의존하는 경향을 심화시켰고,
원래 신학적 옷을 입고 있던 논쟁은
점점
법적인 논쟁의 분위기를 띠어갔다.
그리하여 여론의 역사에서 반복적으로 나타나던 현상 하나가 등장했다.
군주의 권력을 옹호하는 주장이 필머\latin{Filmer}의 교리로
확립되자,
피치자의 권리를 방어하는 데 사용되었던,
계약법에서 빌려온
용어가
왕과 신민 간의 원초적 계약이 실재한다는 이론으로
구체화되었던 것이다.
이 이론은 처음에는 영국인들의 손에서,
나중에는 특히 프랑스인들의 손에서,
모든 사회현상과 법현상을 포괄적으로 설명하는 이론으로 확장되었다.
그러나
정치학과 법학의 진정한 결합은
후자가 전자에게
특유의 유연한 용어를 제공한 것이 전부였다.
로마계약법은,
주권자와 백성의 관계에 대해서도,
보다 소박한 영역에서
``\wi{준계약}''의 \wi{채권채무관계}로 묶인 사람들의 관계에 대해 수행하던 것과
정확히 똑같은 기능을 수행했다.
그것은
정치조직이라는 주제에 관하여 수시로 형성되고 있던 관념들에
사뭇 잘 들어맞는
일군의 용어와 문장들을 제공했다.
원초적 계약의 교리는,
비록 부당한 것이지만,
휴얼\latin{William Whewell} 박사의
찬사보다 더 높은 찬사를 받을 수는 없을 것이다.
``그것은 도덕적 진리를 표현하는 \hemph{유용한} 형식일 것이다.''\footnote{%
  \latinmarks
  William Whewell,
  \textit{The Elements of Morality Including Polity},
  Vol.\,2,
  London: John W. Parker, 1848,
  p.\,113.
  }

\para{윤리학과 로마법}
우선
정치적 주제에 관한 법적 용어가
원초적 계약의 발명에
광범위하게 사용되어 들어간 것, 그리고
이후 이 가정\hanja{假定}이 강력한 영향력을 행사한 것은
정치학에는
용어와 개념이
왜 그렇게 풍부한가를 넉넉히 설명할 수 있거니와,
그것은 오로지 로마법의 산물이었다.
\wi{도덕철학}\latin{moral philosophy}에서 용어와 개념이 풍부한 것에는
조금 다른 설명이 주어져야 한다.
정치적 사변\hanja{思辨}에 비해
윤리학 저술들에서는 로마법의 기여가
훨씬 더 직접적이었으며,
윤리학 저자들도
그 은혜의 크기를 훨씬 더 잘 알고 있었다.
내가 도덕철학이 로마법에 크게 빚졌다고 말하는 것은
칸트에 의해 도덕철학의 역사에 단절이 일어나기 이전의
도덕철학을 대상으로 하는 것임을 알아야 한다.
그것은 인간의 행위를 규율하는 규칙들과
그 규칙들의 적절한 해석,
그리고 그 규칙들의 한계를
다루는 학문이었다.
비판철학이 등장한 이후
도덕철학은 옛 의미를 완전히 상실했거니와,
로마 가톨릭 신학자들이 여전히 가꾸고 있는
\wi{결의론}\hanjalatin{決疑論}{casuistry}에서
저급한 형태로 보존되어 있는 것을 제외하면,
도덕철학은 거의 예외 없이 존재론의 한 분야로 간주되고 있는 듯하다.
형이상학에 흡수되기 이전의 도덕철학,
규칙 자체보다 규칙의 근본원리가 더 중요하게 고려되기 이전의 도덕철학을
이해하는
현대 영국 학자는
내가 알기로,
휴얼 박사를 제외하면,
한 사람도 없다.
하지만,
오랫동안
윤리학은
실천적 행위준칙을 다루어왔기에,
그것은 어느 정도 로마법에 물들어 있었다.
근대 사상의 다른 모든 주요 분야들과 마찬가지로,
원래
그것은
신학과 한 몸이었다.
처음에는 `\wi{도덕신학}'\latin{moral theology}이라 불렸고
지금도 로마 가톨릭 신학자들 사이에서는 이렇게 불리고 있는
이 학문은 분명, 그 저자들도 잘 알고 있었듯이,
행위의 원리를
교회체계로부터
가져오는 것, 그리고
이를 표현하고 전개하는 데 법학의 언어와 방법을 사용하는 것으로
구성되어 있었다.
이런 과정이 지속되면서,
사고\hanja{思考}의 운송수단에 불과했어야 할 법학이
사고 그 자체에도 자신의 색깔을 전달하는 일이 불가피하게 일어났다.
법개념들과의 접촉에서 얻은 이러한 색조는
근대 세계의 초창기 윤리학 문헌에서 쉽게 감지할 수 있거니와,
생각건대
만약 계약법이 없었더라면
도덕적 의무를
신국\hanja{神國}의 시민의 공적 의무로만
바라보려는
저자들의
경향을,
철저한 상호성과 권리^^b7의무의 확고한 결속에 기초하는 계약법이
건강한 방향으로 교정했음에
틀림없다.
그러나 \wi{도덕신학}에서 로마법이 차지하는 비중은
스페인의 도덕론자들이 이 학문을 키우면서부터
눈에 띄게 줄어들게 된다.\footnote{%
  이른바 살라망카 학파를 말하고 있는 듯하다.
  }
박사들에 의해 주석에 주석이 달리는 법학적 방법으로 발달되던
도덕신학이 자신만의 용어를 스스로 만들어냈다.
또한
학파들의 도덕 논쟁에서 상당 부분 흡수한 것이 분명한
아리스토텔레스적 추론과 표현의 특색들이
로마법에 정통한 사람이라면 결코 틀릴 수 없는
사고와 언어의 특수한 문체를 대신하게 된다.
스페인 학파의 도덕신학자들이 계속해서 신망을 유지했다면
윤리학에서 법학적 요소는 하찮은 수준으로 쪼그라들었을 것이다.
그러나
그들의 영향력은
다음 세대 로마 가톨릭 저술가들이
이 학문 영역에서
그들의 성과를 이용한 방식에 의해
거의 전적으로 파괴되어버렸다.
\wi{결의론}\hanja{決疑論}으로 전락한
도덕신학은
유럽의 사변\hanja{思辨}을 선도\hanja{先導}하는 자들의
관심을 상실했으며,
전적으로 프로테스탄트의 손에 들어간
새로운 \wi{도덕철학}은
\wi{도덕신학}자들이 추종하던 길을 크게 벗어났다.
결과적으로 윤리학에 대한 로마법의 영향은 대폭 증가했다.

\para{그로티우스 학파, 결의론의 쇠퇴}
``종교개혁 이후,\origfootnote{%
  이 인용문은
  1856년 <<케임브리지 논문집>>(Cambridge Essays)에 기고한
  저자의 논문의 일부를 약간의 가필을 거쳐 가져온 것이다. }
이 학문 영역에서는
사상을 달리하는 두 개의 큰 학파 간의 대립이 나타났다.
둘 중 더 영향력 있는 쪽은 애초
\wi{결의론}자\latin{casuist}로 우리에게 알려진 분파 혹은 학파였거니와,
그들 모두는 로마 가톨릭 교회를 신앙하였고,
그들의 거의 모두는 이런저런 가톨릭 수도회에 소속되어 있었다.
다른 쪽은
<<전쟁과 평화의 법>>의 위대한 저자
후고 \wi{그로티우스}의 지적 후예라는 공통점을 갖는
일군의 학자들이었다.
후자 쪽의 거의 모두는 종교개혁의 추종자들이었으니,
그들이 공식적^^b7공개적으로 결의론자들과 갈등하였다고 할 수는 없을지라도,
그들 체계의 기원과 대상은 결의론자들의 것과 근본적으로 달랐다.
이러한 차이는 주목할 필요가 있거니와,
이들 양 체계의 사상 영역에 끼친 로마법의 영향 문제와 관련되기 때문이다.
그로티우스의 저 저서는,
비록 모든 페이지마다 순수한 윤리학 문제를 다루고 있지만,
또한 비록 수많은 공식적인 윤리학 저서의 직^^b7간접적 선조이지만,
주지하듯이 \wi{도덕철학}에 관한 논저임을 자처하지는 않는다.
그것은 자연의 법\latin{law of nature},
즉 자연법\latin{natural law}을 명확히 하려는 시도이다.
자연법이라는 개념이 로마 법학자들의 배타적 창안이었는지의
문제를 따질 필요 없이,
그로티우스 본인이 스스로 인정한 것에 근거하여,
실정법의 어느 부분을 자연법의 일부로 보아야 하는가에 관한
로마 법학의 언명은,
그것이 오류가 아닌 한,
언제나
사뭇 깊은 존경을 받으며 수용되었다고
볼 수 있다.
그리하여 그로티우스의 체계는
로마법과 근본적으로 얽혀있는 것이다.
이러한 연결로 인해 불가피---저자가 법학으로 교육받았던 것의
결과이기도 하겠지만---단락마다
법기술적 용어가 자유자재로 구사되고 있고,
추론과
정의\hanja{定義}와 예시의 방식도 마찬가지이다.
이들이 어디서 유래했는지 출처를 모르는 독자들은 틀림없이,
때로는 그 논증의 의미를 이해하기 어렵고,
거의 항상은 그 논증의 힘과 설득력을 파악하기 어려울 것이다.
다른 한편,
결의론은 로마법에서 빌려온 것이 거의 없고,
무엇이 도덕적이냐의 견해도 그로티우스의 그것과 공통점이 없다.
\wi{결의론}의 이름 아래 유명해진, 혹은 악명높아진, 옳고 그름에 관한 저 모든 철학은
대죄\hanjalatin{大罪}{mortal sin}와
소죄\hanjalatin{小罪}{venial sin} 간의 구분에 기초한다.
어떤 행위를 대죄로 판정하는 끔찍한 결과를 피하려는 자연스런 염려와,
프로테스탄티즘과 대결하고 있는 로마 가톨릭 교회에게서
부담스런 이론의 짐을 덜어주려는, 역시 이해할 만한, 열망에서,
결의론 철학의 저자들은
비도덕적 행위를 가능하면 대죄의 영역에서 제외하여
소죄의 영역에 편입시키려는 복잡한 행위기준의 체계를 발명하게 되었다.
이러한 실험의 결과는 일반 역사의 영역이다.
주지하듯이 결의론의 저 구분은,
사제들의 영적\hanja{靈的} 통제가 다종다양한 인간성에 부응할 수 있도록 만들어,
실로
군주^^b7정치인^^b7장군들에 대한
사제들의
영향력을
종교개혁 이전에는 들어본 적이 없는 수준으로
키워주었으니, 이는
프로테스탄티즘의 초기 성공을 견제하고 축소시킨
저 반\hanja{反}종교개혁에 크게 기여했던 것이다.
그러나 무언가를 세우려는 것이 아니라 피하려는 시도로,
원리를 발견하는 것이 아니라 공준\hanja{公準}을 피하려는 시도로,
옳고 그름의 본성을 정하는 것이 아니라
특정 본성의 무엇이 그르지 않은지를 정하려는 시도로
출발한 결의론은
교묘한 복잡함을 더해간 결과,
행위의 도덕적 성격을 감소시키고
인간의 도덕적 본능을 배반하는 지경에 이르렀으니,
마침내 그것에 거역하여 인류의 양심이 일거에 들고일어나
그 체계와 그 박사들을 모두 공통의 파멸로 몰아넣었다.
오래도록 유예되었던 일격이 \wi{파스칼}의
<<시골 친구에게 보내는 편지>>\latin{Provincial Letters}에서
가해졌다.
이 주목할 만한 저서가 등장한 이후,
조금이라도 영향력이나 신망이 있는 윤리학자라면
자신의 사변을 공공연히 \wi{결의론}에 기초하여 전개할 수는 없게 되었다.
윤리학의 전 영역은 이제 전적으로
그로티우스의 추종자들의 손에 남겨지게 되었다.
그리하여 지금도 윤리학은,
때로는 \wi{그로티우스} 이론의 흠의 원인으로 평가되기도 했고
때로는 그의 이론에 최고의 상찬을 가져다주기도 했던
로마법과의 연루의 흔적을
비상한 정도로 보여주고 있다.
그로티우스 시대 이래 많은 연구자들이 그의 원리를 수정했고,
비판철학의 등장 이후로는 많은 이들이 그의 원리를 포기했지만,
그의 근본 가정\hanja{假定}으로부터 가장 멀리 떠나온 사람들조차
그의 진술 방법, 그의 사고 순서, 그의 예시 방식의 많은 것을
물려받았다.
그런데 이런 것들은 로마법에 무지한 사람들에게는 거의 혹은 전혀
이해될 수 없는 것들이다.''

\para{형이상학과 로마법}
전술했듯이,
자연과학을 제외하면,
형이상학만큼 로마법의 영향을 적게 받은 학문 영역도 없다.
그 이유는 형이상학적 주제의 논의는 언제나 그리스어로 이루어졌다는 데 있다.
정확히 말하면 처음에는 순수한 그리스어로,
나중에는 그리스어 개념을 표현하기 위해 특별히 만들어진 라틴어 방언으로
이루어졌던 것이다.
현대 언어들은 이 라틴어 방언을 채용함으로써,
혹은 그것의 형성기의 과정을 모방함으로써,
비로소
형이상학적 탐구에 적합한 언어가 될 수 있었다.
근대에 들어 형이상학적 논의에 항상 사용되어온 용어의 출처는
라틴어로 번역된 아리스토텔레스였거니와,
그것이 아랍어판을 번역한 것이든 아니든,
번역자의 의도는
라틴어 문헌에서 유사한 표현을 찾는 것이 아니라,
그리스 철학 관념을 표현하는 일군의 용어들을
라틴어 어근으로부터
새롭게 구성하는 것이었다.
이러한 과정에 로마법 용어가 줄 수 있는 영향은 거의 없었다.
기껏해야 몇몇 라틴어 법률용어가 변형된 형태로
형이상학의 언어에 포함되었을 뿐이다.
동시에 언급하고 싶은 점은,
서유럽을 자못 크게 뒤흔든 형이상학의 문제는 어느 것이든,
그 언어는 몰라도,
그 사상은 법학적 기원을 드러낸다는 것이다.
사변\hanja{思辨}의 역사에서 아마도 가장 인상적인 것은,
그리스어를 말하는 민족은
자유의지\latin{free will}와 필연성\latin{necessity}\footnote{%
  `necessity'는 법률용어인 `긴급피난'으로 번역될 수도 있다.
}이라는 중대한 문제로 심각하게 고민해본 적이 없다는 사실일 것이다.
나는 이 문제를 간략하게라도 감히 설명할 생각이 전혀 없다.
그러나 그리스인들도, 그리고 그리스어로 말하고 생각하는 어떤 사회도,
법철학을 생산할 일말의 능력도 보여준 적이 없다는 사실은
이와 무관하지 않다고 생각된다.
법학은 로마인들의 산물이며,
자유의지의 문제는 형이상학적 개념을 법적인 측면에서 숙고할 때 등장하는
문제이다.
어떻게 해서 이 문제가
불변의 사건 연쇄는 필연적\latin{necessary} 관계와 동일한 것인지 어떤지의
문제로 되었는가?
내가 말할 수 있는 것은,
로마법의 경향은,
시간이 갈수록 강해진 그 경향은,
법적 원인과 법적 효과가
흔들림없는
필연성으로 결합된다고
보았다는 것뿐이다.
앞서 인용한 채권채무관계의 정의가 이러한 경향의 현저한 사례이다:
``누군가에게 급부\hanja{給付}를 할 것이 필연적으로 강제되는
법의 사슬''\latin{juris vinculum quo necessitate adstringimur alicujus
solvendae rei}.

\para{교회에서의 로마법}
그런데 자유의지의 문제는
철학이기 이전에 신학의 문제였으며,
그 용어가 법학의 영향을 받았다면
그것은
법학이 신학에 의해 감지되어 받아들여졌기 때문일 것이다.
여기서 내가 제시하는 주요 논점은 한번도 만족스럽게 해명된 적이 없는 것이다.
우리가 확인해야 할 점은,
법학이
신학적 원리에 접근하는 매개체로
기능했는가,
특유의 언어를, 특유의 추론양식을, 여러 세상사에 대한 특유의 해결책을
제시함으로써 법학은 신학적 사변이 흘러나오고 확장되어가는
새로운 통로를 열었는가 하는 것이다.
답을 구하기 위해서는,
초기에 신학이 흡수한 지적 양식\hanja{糧食}이 무엇이었는가에 관해
최고의 학자들 간에 이미 합의된 것을 상기할 필요가 있다.
기독교 교회의 초창기 언어는 그리스어였으며,
기독교 교회가 초기에 대처한 문제들도
후기 그리스 철학이 그 길을 닦아놓았던 문제였음이
널리 인정되고 있다.
신의 위격\hanjalatin{位格}{persons},
신의 본체\latin{substance},
신의 본성\latin{natures} 같은 심오한 논쟁에 인간 정신이 참여할 수 있도록
해주는 언어와 관념의 유일한 창고는
그리스의 형이상학적 문헌들이었다.
라틴어와 소박한 라틴 철학은 이러한 임무를 감당할 능력이 사뭇 모자랐고,
따라서 서방, 즉 라틴어를 사용하는 유럽 지역은
동방의 성과를 따지지도 검토하지도 않고 그대로 받아들였다.
밀만\latin{Henry Hart Milman} 주임사제에 따르면,
``라틴 기독교는 자신의 협소하고 빈약한 어휘로는 적절하게 표현하기 어려운
저 신조를 받아들였다.
그런데 로마와 서방의 동의는 어디까지나
동방 신학자들의
심오한 신학에 의해 형성된 교리체계를 수동적으로 묵인한 것이었을 뿐,
신학적 난제들을 스스로 열성적으로 그리고 독창적으로 검토한 것이 아니었다.
라틴 교회는 아타나시우스\latin{Athanasius}의 제자였으며
충성스런 지지자였다.''\footnote{%
  \latinmarks
  Henry Hart Milman, \textit{History of Latin Christianity},
  London: John Murray, 1854, p.\,61. }
그러나 동로마와 서로마의 분리가 더욱 확고해지고
라틴어를 쓰는 서로마제국이 스스로의 지적 삶을 살기 시작하면서,
동방에 대한 존경은 갑자기
동방적 사변\hanja{思辨}에는 전적으로 생경한
다수의 문제들에 관한 격론으로 변모했다.
``그리스 신학이
\paren{밀만, <<라틴 기독교>>, 서문, 5쪽}
훨씬 세련된 섬세함으로 삼위일체와 그리스도의 본성을 계속 정의해가는 동안''
``끝없는 논쟁이 여전히 길게 이어지고
허약해진 공동체로부터 이런저런 분파들이 계속 분리되어나가는 동안''\footnote{%
  위의 책, p.\,5. }
서방 교회는
새로운 종류의 논쟁들에 열정적으로 뛰어들었으니,
이는 그때부터 지금까지 라틴 교파에 속하는 사람들이라면 한시도
관심을 놓지 않았던 것들이다.
원죄와 그것의 대물림,
인간의 진 빚과 그것의 대속\hanja{代贖},
속죄\latin{Atonement}의 필연성과 충분성,
특히 자유의지와 신의 섭리 간의 표면적 대립관계,
서방은
이런 것들을
동방이 특정한 신경\hanja{信經}의 조항을 두고 논쟁했던 것 못지않게
가열차게 논쟁하기 시작했다.
그렇다면
그리스어를 쓰는 지역과
라틴어를 쓰는 지역 간에
신학적 문제의 종류가 서로 그렇게 달랐던 까닭은 무엇이었을까?
교회사가\hanja{史家}들은
동방 기독교를 갈라놓았던 문제들보다
새로운 문제들이
더 ``실제적인''\latin{practical},
즉 전적으로 사변적이지는 않은 것이었다고 말하여
어느 정도 정답에 가까이 다가갔으나,
내가 아는 한 어느 누구도 정답에 도달하지는 못했다.
나는
두 신학체계 간의 차이는,
동방에서 서방으로 넘어오면서
신학적 사변의 풍토도
그리스의 형이상학에서 로마법으로 바뀌었다는 사실로
설명된다고
서슴없이 주장하고 싶다.
이들 논쟁이 압도적으로 중요한 논쟁으로 부상하기
수 세기 전부터
서로마인들은 그들의 지적 활동을 전적으로 법학에 쏟아부었다.
그들은
세상사가 조합해낼 수 있는 온갖 상황에
특유의 법원리들을
적용하는 일에 몰두해왔다.
다른 어떤 업무나 취미도
그 일에서 그들의 관심을 멀어지게 할 수 없었으며,
그것을 수행하기 위해 그들은
정확하고 풍부한 어휘,
엄격한 추론방법,
경험에 의해 어느 정도 실증된 일반적 행위 명제의 저장고,
그리고 엄정한 \wi{도덕철학}을
보유하고 있었다.
기독교 기록에 나타난 문제들 중에서
그들에게 친숙한 사고 유형에 가까운 것들을
그들이
발견하지 못한다는 것은 불가능한 일일 것이다.
또한 그것들을 취급하는 방식을
그들의 법학적 습관에서 가져오지 않는다는 것도 불가능한 일일 것이다.
로마법에 대한 지식이 충분해서
로마의 형법체계를,
계약과 불법행위로 성립되는 로마의 채권채무관계 이론을,
채무 및 그것을 부담하고 소멸시키고 이전하는 방식에 관한
로마인의 견해를,
포괄적 승계에 의해 개인의 존재가 계속 이어진다는 로마인의 관념을
이해할 수 있는 사람이라면 거의 누구나,
서방 신학의 저 문제들과 잘 어울리는 것으로 드러난 사고의 틀이 어디서 온 것인지,
이들 문제를 진술하는 용어가 어디서 온 것인지,
그 문제의 해결책에 사용된 추론의 유형이 어디서 온 것인지
자신있게 말할 수 있을 것이다.
다만,
서방 사상에 작용하여 들어간 로마법은
옛 로마시의 고법\hanja{古法}도 아니고,
비잔틴 황제들에 의해 잘려나가 축약된 법도 아니며,
그렇다고 근대의 사변적 교리의 기생\hanja{寄生}적 과대성장 속에 거의 파묻힌,
근대 대륙법이라고 불리는 법규칙의 덩어리도 아니었다는
점만은 유념해야 한다.
나는 바로 안토니누스 황조 시대의 위대한 법학자들이
일구어낸 법철학을 말하고 있거니와,
그것은 유스티니아누스의 학설휘찬\latin{Pandects}을 통해 지금도 부분적으로
재구성할 수 있는 것이다.
그 체계의 흠을 굳이 들라면,
인간의 법이 추구할 수 있을 것으로 보이는 한계를 넘어선
고도의 우아함, 확실함, 정확함을
목표로 했다는 점
정도가 아닐까 한다.

\para{로마에서 법학의 우위}
영국인들이 자진해서 고백하는,
때로 부끄러워하기는커녕 자랑스러워하는,
로마법에 대한 무지로 인해,
다수의 저명하고 신망있는 영국 학자들조차
제정기 로마의 지적 상태에 관해 도저히 지지할 수 없는
역설적 주장을 내놓는 특이한 결과가 생겨났다.
아우구스투스 시대가 마감된 때부터
기독교 신앙에 대한 대중적 관심이 일어나기 전까지
문명 세계의 정신적 에너지가 마비상태에 빠졌다는 명제가,
그 명제의 주장에 아무런 무모함도 없다는 듯이,
서슴없이
주장되어왔다.
하지만
인간 정신이 보유한 모든 힘과 능력을 사용할 수 있도록 하는
사고 영역에는 두 가지---아마도 자연과학을 제외하면
이 두 가지뿐일 것이다---가 있다.
하나는 형이상학으로,
인간 정신이 스스로 즐겨 작동하는 한 한계가 없는 영역이고,
다른 하나는 법학으로,
인간사의 일들과 외연을 같이하는 영역이다.
전술한 바로 그 시기 동안,
그리스어를 말하는 지역에서는 전자가,
라틴어를 말하는 지역에서는 후자가,
몰두의 대상이었다.
알렉산드리아와 동방에서의 사변의 결실에 대해서는 모르겠으나,
로마와 서방은
다른 모든 지적 훈련의 부재를 보상하고도 남을 만한
직업 하나를 수중에 쥐고 있었다고
자신있게 말할 수 있다.
또한 우리가 아는 한,
그것이 이룩한 성취는
그것을 만드는 데 들어간 지속적이고도 배타적인 노력에
충분히
값하는 것이었다.
어쩌면
전문 법률가가 아니라면
법학이 흡수할 수 있는 개인의 지적 능력이 얼마나 큰지
완전히 이해할 수 없을지도 모른다.
그러나 일반인이라도
로마의 집단지성 가운데 이례적으로 큰 몫이
어째서 법학에 의해
독점되었는지
어렵지 않게 이해할 수 있을 것이다.
``장기적으로 볼 때,\origfootnote{%
  앞의 1856년도 <<케임브리지 논문집>>. }
어떤 공동체의
법학적 능숙함은
다른 어떤 학문 분야의 진보와도 동일한 조건에 달려있다.
그중 중요한 것은 한 나라의 지식인 중에 거기에 투입되는 비율과
투입되는 시간의 길이이다.
그런데
학문을 진보시키고 완성시키는 데 기여하는
직^^b7간접적인 원인들이 모두 함께,
12표법부터 두 제국의 분리에 이르는 기간 내내
로마의 법학에 대해 작용하였으니,
그것은 불규칙적이거나 간헐적이 아니라
꾸준히 힘이 증가하고 지속적으로 수가 많아지는 양상이었다.
초창기의 지적 훈련이 법의 연구에 바쳐지고 있는 젊은 나라를 상상해보라.
일반화를 위한 의식적 노력이 행해지면서,
일상생활의 관심은 가장 먼저
그것을 일반적 규칙과 포괄적 공식에 포섭하는 것이 된다.
젊은 공동체의 모든 에너지가 바쳐지고 있는 이 분야의 인기는
처음에는 무제한적이다.
하지만 시간이 흐르면서 그것도 시들해진다.
법학이 인간 정신을 독점하는 상황도 깨져간다.
위대한 로마 법학자의 대기실에 아침부터 몰려들던 고객들도 줄어든다.
영국의 법조원\hanjalatin{法曹院}{inns of court}의 학생 수도
수천명대에서 수백명대로 줄어든다.
예술, 문학, 과학, 정치가 그 나라의 지식인 중의 일정 몫을 가져간다.
법실무는 전문가 그룹 내의 것으로 국한된다.
그러나 쪼그라들거나 하찮은 것이 되지는 않거니와,
보수\hanja{報酬}의 측면에서도 그들의 학문의 고유한 매력의 측면에서도
여전히 사람들을 끌어들인다.
이러한 변화의 과정은 영국보다 로마에서 더 현저하게 나타났다.
공화정 말기에 이르면
군대를 통솔하는 특별한 재능을 제외하면
모든 재능 있는 사람들은
법학을 공부한다.
그러나,
영국의 엘리자베스 1세 시대가 그러했듯이,
아우구스투스 시대와 더불어
지성의 진보는 새로운 단계를 맞이한다.
주지하듯이 시와 산문에 있어 그 시대의 업적은 대단했지만,
장식용에 불과한 문학의 번영 외에도
자연과학을 정복하려는 새로운 경향도 막 등장하려 했음에
유의해야 한다.
하지만 이 시기는 로마 국가의 정신의 역사가
그후 추구되어온 정신 진보의 일반적인 경로와 달라지는 시기였다.
이른바, 그러나 정확한 묘사인, 로마 문학의 짧은 수명은
여러 가지 요인으로 갑자기 종말을 맞았거니와,
여기서 그 요인들을 분석하는 것은,
비록 부분적으로 추적가능하다 할지라도,
적합치 않을 것이다.
고대 지식인들은 급격히
옛 상태로 되돌아갔고,
로마인들이 철학과 시를 유치한 민족의 장난감으로 경멸하던 시절만큼이나
배타적으로 법학이 다시
재능 있는 사람들에게 적합한 영역으로 각광받았다.
제정기 동안,
법학 분야에 적합한
내적 능력을 가진 사람들을 끌어들인
외적 요인을 이해하기 위해서는
그의 앞에 놓인 직업의 선택지를 생각해보는 것이
가장 적절할 것이다.
그는 수사학 교사,
전선의 사령관,
또는 온갖 찬사를 쏟아내는 전문 작가가 될 수 있었다.
하지만 그에게 열려있는 그밖의 활동영역으로는
법실무에 종사하는 것이 유일했다.
\hemph{이것}을 통하여 그는
부, 명예, 관직에 접근할 수 있었고,
황제의 자문단\latin{council chamber}---어쩌면 황제의 자리 자체---에도
오를 수 있었다.''

\para{서방 신학에서의 로마법}
법학이 갖는 장점이 그렇게 컸기에
제국의 모든 지역에, 심지어 형이상학이 번성한 지역에도,
법학교들이 존재했다.
비록 황제의 거처가 비잔티움으로 옮아가
동방에서 법학이 발달할 뚜렷한 계기가 되었음에도,
법학은 거기서 경쟁관계에 있는 다른 학문들을 결코 몰아내지 못했다.
법학의 언어는 라틴어였으니,
제국의 동부에서는 외래 방언이었던 것이다.
오직 서방에서만
법학이 야심과 포부를 가진 사람들의 정신적 양식이었을 뿐만 아니라
지적 활동의 유일한 자양분이기도 했다.
로마의 식자층 사이에서는 그리스 철학이
일시적인 유행 이상의 것이 되지 못했다.
동방에 새로운 수도가 건설되고 그후 제국이 둘로 갈라지자,
서방은
그리스적 사변으로부터
더없이 결정적으로
결별하게 된다.
이제 그리스의 문하생에서 벗어나
독자적으로 신학을 궁구하기 시작하자,
그들의 신학은 법적인 관념에 물들고 법적인 용어로 표현되었다.
확실히 서방 신학에서 이러한 법학적 토대는 대단히 깊은 것이다.
그후 아리스토텔레스 철학이라는 새로운 그리스적 이론이
서방에 유입되었고 서방의 고유한 원리들을 거의 전부 덮어버렸다.
그러나
종교개혁 이후 서방은
그것의 영향력을
부분적으로 떨쳐버렸고,
그 자리에 즉각 법학을 가져다 앉혔다.
칼뱅\latin{Calvin}의 종교체계와
아르미니우스파\latin{Arminians}의 종교체계 중
어느 것이 더 법학적 성격이 강한 지는 판가름하기 어렵다.

\para{계약법과 봉건제도}
로마인들이 생산한 이러한 계약법이
근대 계약법에 끼친 막대한 영향력은
성숙한 법학의 역사에 해당하므로
본 논저의 논의대상을 벗어난다.
그것은
볼로냐 대학이 근대 유럽 법학의 기초를 다지면서
비로소
감지되기 시작했다.
그러나
제국이 몰락하기 전에 이미
로마인들에 의해
계약 개념이
완전히 발달했다는 사실은
그보다 훨씬 이른 시기에
중요한 의미를 갖게 된다.
누차 강조했듯이
\wi{봉건제}도는 옛 만족\hanja{蠻族}들의 관습과 로마법이
결합한 것이었다.
다른 설명은 지지될 수 없거나 심지어 이해조차 불가능하다.
봉건시대 초창기의 사회 형태는
원시 문명의 사람들이 어디서나 보여주는
결합의 형태와 별반 다르지 않았다.
봉건관계는 일종의 유기적으로 완전히 결합된 동료관계로서,
재산적 권리와 신분적 권리가 불가분 혼재되어 있었다.
그것은 인도의 \wi{촌락공동체}와 많은 공통점을 가지며,
스코틀랜드 산악지대의 씨족과도 많은 공통점을 가진다.
그러나 그것은 여러 문명의 초기에 자발적으로 형성된 결합관계와는 다른
특수한 성질도 가진다.
실로 원시적 공동체는 명시적 규칙이 아니라 감정에 의해,
아니 어쩌면 본능에 의해 결합된다.
또한 동료관계에 새로 들어오는 자는
이러한 본능에 부합하게
짐짓
자연적 혈연관계를 공유한다고 내세움으로써 편입되는 것이다.
그러나 초창기의 봉건적 공동체는 단순한 감정에 의해
결합되는 것도 아니었고
의제\latin{fiction}에 의해 충원되는 것도 아니었다.
그들을 결속시키는 것은 계약이었으니,
그들은 계약을 맺음으로써 새로운 성원을 얻었던 것이다.
주군과 가신의 관계는 원래 명시적 계약을 통해 설정되었다.
\hemph{\wi{충성서약}}\latin{commendation}이나
\hemph{\wi{수봉}}\hanjalatin{受封}{infeudation}을 통해
동료관계에 편입되려는 자는
그가 받아들여지는 조건을 분명히 알 수 있었다.
따라서 \wi{봉건제}도가 원시 민족들의 순수한 관행과 다른
주된 차이점은 계약이 차지하는 부분에 의해서인 것이다.
주군은 가부장의 성격을 다분히 가지고 있었으나,
그의 대권\hanja{大權}은
수봉시 합의된 명시적 조건에서 기인하는 다양하게 설정된 관습에 의해
제한되었다.
그리하여 봉건사회를 진정한 원시 공동체로 분류할 수 없는
주요 차이들이 발생하게 된다.
봉건사회는 훨씬 더 지속적이었고 훨씬 더 다양했다.
명시적 규칙은 본능적 습관에 의해 파괴되기 어렵다는 점에서
그것은
훨씬 더 지속적이었다.
봉건사회의 기초인 계약은
세부적인 상황에 따라
그리고
토지를 맡기거나 양여하는 자의 원하는 바에 따라
얼마든지 달라질 수 있다는 점에서
그것은
훨씬 더 다양했다.
이 마지막 점은
근대 사회의 기원에 관한 오늘날 우리의 통속적인 견해가
얼마나 잘못된 것인지를 알려주는 데 도움이 될 수 있다.
근대 문명의 불규칙하고 다양한 모습이
게르만 민족들의 지나치게 많은 변칙적인 풍속 탓이라고 하면서,
이를 지루하리만치 틀에 박힌 로마제국의 그것과 대비시키는 일이 흔히 있다.
그러나 진실은
로마제국이 이 모든 불규칙성의 원인인 저 법개념을
근대 사회에 물려주었다는 데에 있다.
만족\hanja{蠻族}들의 관습과 제도들의 가장 두드러진 특징 하나를 들자면,
그것은
그것들이 무척 단조로웠다는 것이다.


\chapter{불법행위법 및 형법의 초기 역사}

\para{고대 법전에서의 형법}
앵글로색슨 선조들의 법전을 포함한
튜턴족의 법전들은
초기의 법들 간의 비중을 정확히 알 수 있는 상태로
우리에게 전해지는
유일한 원시 세속법 법전들이다.
로마와 그리스 법전들의 현존하는 단편들은
그것들의 일반적 성격을 알려주기에는 충분하지만,
그 부분들 간의 정확한 양이나 비율을 파악하기에는 부족한 상태로 남아있다.
그럼에도 불구하고 전반적으로 보아
우리에게 전해지는 모든 고대법 집성들은
성숙한 법체계와 크게 다른 한 가지 특징을 가지고 있다.
민법 대비 형법의 비율이 대단히 큰 차이를 보이는 것이다.
게르만 법전들에서는 형법에 비해 민법 부분의 비중이 아주 작다.
드라콘의 법전이 규정한 포악한 형벌에 관한 전승\hanja{傳承}을 보더라도
이것 또한 마찬가지 성격이었던 것으로 보인다.
뛰어난 법적 재능과 애초 부드러운 풍속을 가졌던 사회의 작품인
12표법만이 유일하게
근대법과 비슷한 정도로 민법의 우위를 보여주지만,
불법행위의 구제방식이 차지하는 상대적 비중이
아주 크지는 않더라도 상당히 큰 편이다.
생각건대
오래된 법전일수록 형법이 더 풍부하고 더 상세하다고
주장할 수 있겠다.
이 현상은
처음으로 그들의 법을 성문화한 공동체들에 만연했던 폭력때문이라고
흔히
인식되어왔고 설명되어왔거니와,
대체로 정확하다고 할 수 있다.
입법자들은
그들 법전의 부분들의 비율을
야만적 생활에서 발생하는 특정 종류의 사건들의 빈도에 맞추었다는 것이다.
하지만
이런 설명은 불완전하다고 생각한다.
옛 집성들에서 민법이 상대적으로 황무지인 것은
이 논저에서 다룬 고대법의 다른 성격들과 밀접히 관련된다는 점을 기억해야 한다.
문명사회의 민법 부분의 십중팔구는
신분법, 물권 및 상속법, 그리고 계약법으로 이루어져있다.
그러나
이들 법분야는
사회적 결속의 유년기로 거슬러올라갈수록
더욱 좁은 범위로 축소될 수밖에 없음이 명백하다.
신분법\latin{law of status}에 다름아닌 인법\hanjalatin{人法}{law of persons}은
모든 형태의 신분이 가부장권에 함께 복속해 들어가있는 한,
아내가 남편에 대해,
아들이 아버지에 대해,
미성숙의 피후견인이 종친\hanja{宗親}인 후견인에 대해
아무런 권리도 갖지 않는 한,
아주 좁은 범위로 축소될 것이다.
마찬가지로
물건과 상속에 관한 법도
부동산과 동산이 가족 내에서 대물림되는 한,
설령 분배되더라도 가족 내에서 분배되는 한,
결코 풍부할 수 없을 것이다.
그러나 고대 민법의 가장 큰 빈틈은
언제나 계약법의 부재에 기인할 것이다.
몇몇 원시 법전은 계약법이 아예 없고,
다른 법전들에서는
선서\hanjalatin{宣誓}{oath}에 관한 복잡한 법이 계약법을 대신하고 있어서
계약 관련
도덕관념의 미성숙을 보여주고 있을 뿐이다.
그런데 이에 상응하는,
형법의 빈곤을 가져올 만한 이유는 없다.
따라서,
설령 제 민족들의 유년기가 언제나 무제약적 폭력의 시기였다고 말하는 것이
위험하다 할지라도,
왜 근대법의 민법 대비 형법의 관계가 고대 법전에서는 역전되는 것인지
우리는 여전히 이해할 수 있는 것이다.

\para{범죄와 불법행위}
나는 후대에 비해 원시법이 형법,
즉 \hemph{범죄}법\latin{\textit{criminal} law}에
큰 비중을 둔다고 말했다.
이 표현은 편의상 사용한 것이고,
실은
고대 법전을 살펴보면
비상한 양을 차지하는 저 법이
진정한 범죄법은 아님이 드러난다.
문명사회의 법은
국가나 공동체에 대한 침해와
개인에 대한 침해를 구분하는 데 일치하고 있다.
이렇게 구분된 두 종류의 침해를 우리는,
법학에서 항상 일관되게 이들 용어가 사용되고 있다고 자신할 수는 없지만,
범죄\latin{crimes; \textit{crimina}}와
불법행위\latin{wrongs; \textit{delicta}}라고
부를 수 있을 것이다.
그렇다면 고대 공동체의 형법은
범죄\latin{crimes}법이 아니라,
불법행위\latin{wrongs}---영국법 용어로는 토트\latin{torts}---법인 것이다.
피해자는 가해자를 상대로 통상의 민사소송을 제기하고,
승소하면 금전배상의 형태로 전보\hanja{塡補}받는다.
가이우스의 주해서 중에
12표법에 기초한 형법을 취급하는 부분을 펼쳐보면,
로마법이 인정하는 민사 불법행위의 첫 머리에
\hemph{절도}\latin{furtum}가 나오는 것을 볼 수 있다.
우리가 익히 \hemph{범죄}로만 취급된다고 여기는 침해가
\hemph{불법행위}로만 취급되고 있는 것이다.
절도뿐만 아니라
폭행 및 모욕\latin{assault}\footnote{%
  저자는 로마법상의 인격침해(iniuria)를 영어의 `assault'로 옮기고 있는 듯하다.
  로마법의 `인격침해'는 신체적 폭행뿐 아니라
  모욕, 명예훼손 등 인격적 침해도
  포괄하므로 본문에서는 `폭행 및 모욕'으로 번역했다.
}과 강도도
로마 법학자들은 영국법상의 불법침해\latin{trespass},
문서명예훼손\latin{libel}, 구두명예훼손\latin{slander}과
마찬가지로 취급한다.\footnote{%
  이들 세 가지 영미법상의 법개념들은 모두
  불법행위(tort)에 속하는 소송형식들이었다.}
이들 모두가 채권채무관계, 즉 `법의 사슬'\latin{vinculum juris}을
가져오고, 이들 모두가 금전배상으로 전보되는 것이다.
하지만 이 특징은 게르만 부족들의 법전에서 더욱 뚜렷하게 나타난다.
예외 없이 그것들은
살인에 대한 금전배상의 방대한 체계를 기술하고 있고,
거의 예외 없이
기타 덜 중대한 침해에 대한 방대한 배상체계를 기술하고 있다.
켐블\latin{John Mitchell Kemble} 씨에 따르면
``앵글로색슨법에서는 \paren{<<앵글로색슨>>, 1.177}
모든 자유인의 생명에 그의 신분에 따라 금액이 매겨져있었다.
또한 사람의 신체에게 가해질 수 있는 모든 상해에 대해,
그리고 그의 시민권, 명예, 평온에 대해 가해질 수 있는 거의 모든 침해에 대해
그에 상응하는 금액이 매겨져있었다.
그 금액은 우발적인 상황에 따라 가중된다.''\footnote{%
  John Mitchell Kemble,
  \textit{The Saxons in England: A History of the English Commonwealth
  Till the Period of the Norman Conquest},
  Vol. 1,
  London: Longman, Brown, Green, and Longmans, 1849,
  pp.\,276f.}
이들 배상금은 중요한 수입원이 되었을 것이 분명하고,
매우 복잡한 규칙이 그것에 대한 권리와 책임을 규율하고 있거니와,
전술했듯이, 귀속되는 사람의 사망으로 그것이 면책되지 않는다면,
어떤 특정한 상속규칙에 따라 상속되는 것이 일반적이다.
따라서,
국가가 아니라 개인을 피해자로 보는 것이
\hemph{불법행위}법의 기준이라면,
법의 유년기에는
형법이 아니라
불법행위법에 의존하여
시민들이
폭력이나 사기로부터 보호받았다고 주장할 수 있는 것이다.

\para{불법행위와 종교적 죄}
그리하여 원시법에서 불법행위는 방대한 양을 차지하고 있다.
또한 종교적 죄\latin{sin}도 원시법에 알려져있었음을 첨언해야 할 것이다.
튜턴족 법전들에 관해서는 이런 주장을 굳이 할 필요도 없거니와,
현존하는 이들 법전은 기독교도 입법자들에 의해
편찬되고 개정되었기 때문이다.
그러나 사실 비\hanja{非}기독교적인 옛 법전들도
일정한 작위 유형과 일정한 부작위 유형에 대해
신의 지시와 명령을 위반했다는 이유로 형벌을 부과한다.
아테네의 아레오파고스\latin{Areopagus} 원로회의가 집행한 법은
아마도 어떤 특별한 종교법전이었을 것이다.
로마에서도 일찍이
신관\hanjalatin{神官}{pontifical}법이
간통, 독신\hanja{瀆神}, 그리고 어쩌면 살인도
처벌했던 것이다.
따라서 아테네와 로마 국가에서는 \hemph{종교적 죄}를 벌하는 법이 있었다.
또한 \hemph{불법행위}를 벌하는 법도 있었다.
신에 대한 침해라는 개념이 전자의 법을 만들었고,
이웃에 대한 침해라는 개념이 후자의 법을 만들었다.
그러나 국가 또는 전체 공동체에 대한 침해라는 관념은
초기에는 진정한 의미의 형법을 만들지 못했다.

\para{범죄의 개념}
그렇다고 해서
국가에 대한 침해라는 간단하고도 기초적인 개념이 원시사회에
부재했다고 생각해서는 안 된다.
그보다는,
이 개념이 실현되는 특별한 방식이
초기에 형법의 성장을 가로막는 원인이었다고 보인다.
여하튼
로마 공동체는 자신이 침해당했다고 생각되면
개인이 침해당한 경우를 유추하여
완전히 똑같은 결과를 강제했거니와,
국가는 어떤 특별한 행위로써 개인 침해자를 응징했던 것이다.
즉, 로마 공동체의 유년기에는
국가의 안전과 이익을 중대하게 침해하는 모든 행위는
입법기관의 개별 입법에 의해 처벌했다.
그리고 이것이 범죄\latin{crimen}의 초창기 개념이었거니와,
국가가 사건을 민사법원이나 종교법정에 맡기지 아니하고
침해자에 대한 특별법\latin{privilegium}을 만들어 처벌할 정도로
중대한 문제를 야기하는 행위가
바로 그것이다.
따라서 모든 기소는
`형벌법안'\latin{bill of pains and penalties}의
형태를 띠었고,
\hemph{범죄자}의 재판은
정해진 규칙이나 정해진 절차로부터 전적으로 독립된
전적으로 특별하고 전적으로 비정규적인 절차였다.
결과적으로
재판을 담당하는 법원이 주권자인 국가 자신인 까닭에,
또한
미리 어떤 행위 유형을 지시하거나 금지하는 것이 불가능한 까닭에,
이 시대에는 형\hemph{법}이 존재하지 않았던 것이다.
그 절차는 통상적으로 법률이 통과되는 형태와 동일했다.
동일한 사람들에 의해 발의되었고
동일한 엄숙한 절차에 의해 진행되었다.
나중에 법원 및 사법관의 조직을 갖춘 통상의 형법이
들어서고 나서도,
옛 절차는,
이론상 모순될 것이 없다는 점에서 짐작할 수 있듯이,
여전히
가용한 상태로
엄연히
남아있었음을 유념해야 한다.
이러한 수단을 사용하는 것이 다분히 경원시되었음에도,
로마 인민은
이 권한을 항상 보유했고 이를 이용해
특별법으로 대역죄를 처벌하곤 했다.
고전학자들에게는
정확히 동양\hanja{同樣}으로
아테네의 형벌법안인
에이산겔리아\greek{εἰσαγγελία}가
정규 법원의 설치 이후에도 계속 유지되었음을 굳이 상기시킬 필요가 없을 것이다.
또한
주지하듯이
튜턴족의 자유민들이 입법을 위해 집회했을 때
그들은
특별히 사악한 범죄
또는
높은 신분의 범죄자가 저지른 범죄를 처벌할 권한도 행사했다.
앵글로색슨의 위테나게모트\latin{Witenagemot}\footnote{%
  `현자(賢者)들의 모임'이라는 뜻으로 앵글로색슨 왕국들에서
  일종의 국왕평의회(curia regis)의 역할을 했다.
}의 형사 재판권도
동일한 성격을 가졌다.






\end{document}
