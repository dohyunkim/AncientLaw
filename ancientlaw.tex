\documentclass[b5paper]{book}
\usepackage{geometry}
\usepackage{emptypage}
\usepackage[hangul]{kotex}
\usepackage{hyperref}

\ifluatex
  \defaultfontfeatures+{Renderer=HarfBuzz}
  \registerpunctuations{`-,`[,`]}
\else
  \hangulhyphens
\fi
\setmainhangulfont{Noto Serif CJK KR}
  [Script=Hangul, Language=Korean, AutoFakeSlant]
\setsanshangulfont{Noto Sans CJK KR}
  [Script=Hangul, Language=Korean, AutoFakeSlant]

\makeatletter
\def\@makechapterhead#1{%
  \vspace*{50\p@}%
  {\parindent \z@ \raggedright \normalfont
    \ifnum \c@secnumdepth >\m@ne
      \if@mainmatter
        \huge\bfseries \@chapapp\space \thechapter
        \par\nobreak
        \vskip 18\p@
      \fi
    \fi
    \interlinepenalty\@M
    \linespread{1.26}%
    \Huge \bfseries #1\par\nobreak
    \vskip 40\p@
  }}
\makeatother

\renewcommand\chaptermark[1]{\markboth{\textit{#1}}{}}

\def\para#1{\leavevmode
  \marginpar{\linespread{1.1}\sffamily\footnotesize \raggedright
    #1}\markright{#1}\ignorespaces}
\def\hanja#1{\begingroup\scriptsize #1\endgroup}
\ifluatex
  \def\latin#1{\ifnum\lastskip=0 \penalty50 \hskip0pt plus.25pt minus.15pt\fi
    \begingroup\hangulpunctuations=0
    \footnotesize #1\hangulpunctuations=1 \endgroup}
\else
  \def\latin#1{\ifnum\lastskip=0 \penalty50 \hskip.5pt plus.4pt minus.2pt\fi
    \begingroup\latinmarks\footnotesize #1\endgroup}
\fi
\def\hemph#1{\begingroup\sffamily\bfseries #1\endgroup}
\def\hanjalatin#1#2{\hanja{#1}\hskip.1em plus.05em minus.02em\latin{#2}}
\def\paren#1{\begingroup\small(#1)\endgroup}

\tracinglostchars=3

\linespread{1.4}
\skip\footins=8pt plus 8pt minus 4pt
\footnotesep=9.5pt

%\includeonly{ch8}

\begin{document}

\title{고대법\\
\large 그것은 사회의 초기 역사와 어떤 관련이 있으며\\
근대 관념에 대해서는 어떤 관계를 가지는가}
\author{헨리 섬너 메인 지음\\
김 도현 옮김}
\date{1920 (1861)}

\frontmatter

\maketitle
\tableofcontents

\mainmatter


\chapter{고대 법전}

우리가 알고 있는 가장 유명한 법체계는 법전과 함께 시작해서
법전과 함께 끝난다.\footnote{시작은 12표법, 끝은 로마법대전을 뜻한다.}
로마법의 해설자들은 그들의 법체계가 \wi{12표법}\latin{Twelve Decemviral Tables}에 기초하고 있다는,
따라서 성문법에 기초하고 있다는 취지의 말을 그들의 역사 내내 시종일관 해왔다.
한 가지 예외를 제외하면,\footnote{%
  사용취득(usucapio)에 관한 시민법을 말하는 듯하다.
  본서 제8장의 \hyperlink{usucapio}{사용취득}에 관한 설명 참조.}
12표법 이전으로 거슬러 올라가는 제도로 로마에서 인정된 것은 없었다.
로마법이 법전의 후예라는 이론, 영국법은 기억할 수 없는
옛 불문\hanja{不文}의 전통에 기원한다는 이론은
로마법의 발달이 왜 영국법의 발달과 달랐는지를 설명하는 주요 이론들이다.
두 이론 다 사실과 정확히 들어맞지는 않지만, 각각 대단히 중요한 결과들을 낳았다.

\para{원초적 법관념}
12표법의 공표가 법의 역사를 다루는 출발점이 될 수 없음은 말할 것도 없다.
문명화된 민족이면 거의 다 고대 로마의 법전과 비슷한 것을 가지고 있었다.
또한 로마와 헬레니즘 세계에 관한 한, 비교적 서로 가까운 시대에
그러한 법전이 두 세계에 널리 확산되었었다.
그것들은 무척 유사한 상황에서 등장했고, 우리가 아는 한
무척 유사한 원인으로 만들어졌다.
많은 법현상들이 이들 법전에 시기적으로 앞서거나 뒤따랐음은 말할 것도 없다.
적지 않은 문헌기록들이 남아있어 법의 초기 현상들을 우리에게 알려준다.
하지만 언어학이 산스크리트 문헌을 완전히 분석해내기 전까지는
우리에게 주어진 가장 좋은 인식 원천은 그리스의 호메로스의 시임에 틀림없다.
물론 이들은 실제 사건들을 기록한 역사로서가 아니라,
\wi{호메로스}가 알고 있던 사회의 상태를 기술한 것으로,
그러나 완전히 이상화시키지 않고 기술한 것으로 읽어야 할 것이다.
영웅시대의 어떤 특징이나 전사들의 용기, 신들의 힘 따위가
시적 상상력에 의해 과장됐을 수 있지만, 도덕적 또는 형이상학적 관념에 의해
그의 시가 오염되었다고 믿을 이유는 없다.
도덕이나 형이상학은 아직 의식적인 고찰의 주제가 아니었기 때문이다.
이런 점에서, 비슷한 초기 시대를 다룬다고 하면서
철학적 또는 신학적 영향 하에서 만들어진 후대의 문헌들보다
호메로스의 시가 훨씬 더 신뢰할 만하다.
법개념의 초기 형태를 발견하려는 우리에게 그것들은 더없이 소중하다.
법학자에게 이들 원초적 관념이 갖는 중요성은
지질학자에게 초기 지구의 지각이 갖는 중요성에 비할 만하다.
거기에는 후대의 법에 의해 발현될 모든 형태들이 다 담겨있을 수 있다.
조급함이나 편견으로 인해 기껏해야 피상적인 조사만 하고는 더는
아무 것도 하지 않은 탓에 오늘날 우리의 법학은 불만족스런 상태에 머물러있다.
법학자들의 탐구는 실로 물리학이나 생리학에서 관찰이 억측을 대체하기 이전
상태와 비슷한 상태에 머물러있다.
그럴듯하고 포괄적이지만 전혀 증명되지 않은 이론들, 가령
자연법이나 사회계약론 따위의 이론이 널리 인기를 구가하여
사회와 법의 원초적 역사에 대한 냉철한 연구를 압도하고 있다.
저 이론들은 진리를 가리고 있거니와,
진리가 발견될 수 있는 유일한 영역으로부터 관심이 멀어지게 할 뿐만 아니라,
일단 길들여지고 믿게 되면 후대의 법학에 참으로 엄청난 영향력을
행사할 수 있다.

\para{테미스테스}
법이나 생활규칙이라는, 이제는 무척 발달한 관념에 관련된 최초의 인식은
\wi{호메로스}가 사용한 용어 ``테미스''\latin{Themis}와
``\wi{테미스테스}''\latin{Themistes}에 담겨있다.
주지하듯이 후대 그리스의 신들 중에서 테미스는 정의의 여신으로 나타난다.
하지만 이것은 근대적인, 무척 발달된 관념의 산물이다.
<<\wi{일리아스}>>에서 제우스의 판결보조자로 묘사된 테미스는 전혀 다른 의미를 가졌다.
오늘날 원시사회의 믿을 만한 관찰자들이 밝혀놓았듯이,
인류의 유년기에 인간은
지속적인 혹은 반복적인 사건을
인격의 작용을 가정함으로써만
설명할 수 있었다.
그리하여 바람이 부는 것도 인격이었고 물론 신적 인격이었다.
태양이 뜨고 정점에 이르고 지는 것도 인격이었고 신적 인격이었다.
대지가 수확물을 내주는 것도 인격이었고 신이었다.
물리적 세계가 그러하듯이 도덕적 세계도 마찬가지였다.
왕이 분쟁에서 판결을 내릴 때, 판결은 신적 영감의 결과로 이해되었다.
왕들에게, 또는 왕중의 왕인 신들에게, 판결을 제안하는 신이
바로 \hemph{테미스}였다.
이 관념의 특이함은 복수형 표현에서 나타난다.
테미스의 복수형 \hemph{테미스테스}는 신이 판사에게 지시한
판결들 자체를 뜻했다.
왕들은 바로 꺼내 쓸 수 있는 테미스테스의 저장고를 갖고 있다고 생각되었다.
그러나 이것은 법률이 아니라
판결---게르만인들이 ``둠''\latin{doom}이라고 부르는 것에 정확히 일치한다---이었다는
점을 유의해야 한다.
그로트\latin{George Grote} 씨의 <<그리스 역사>>\latin{History of Greece}에 따르면,
``제우스나 지상의 인간 왕은 입법자가 아니라 판사였다.''
그에게는 테미스테스가 주어져 있으나,
위로부터 주어진 것이라는 믿음에 부합하게,
판결들은 어떤 일관된 원칙으로 연결되어 있다고 관념되지 못했다.
그것은 따로따로 분리된 개별적인 판결들이었다.

호메로스의 시에서도 이러한 관념은 잠시 동안의 것이었음을 알 수 있다.
단순한 구조의 고대사회에서 상황의 유사성은 오늘날보다 흔한 일이었을 테고,
유사한 소송이 연달아 제기됨에 따라 판결들도 비슷해지는 경향이 나타났을 것이다.
여기서 우리는 관습의 기원 혹은 초기 형태를 발견할 수 있거니와,
이것은 테미스테스, 즉 판결보다 나중에 등장하는 관념인 것이다.\footnote{%
  그러나 왕의 테미스테스도, 이론적으로는 신적 영감에 의한 것이라 해도,
  실제로는 당시의 관습이나 관행에 기초하였을 것이 틀림없다.
  \latin{Maine, \textit{Early Law and Custom}, 1883, p.\,163.} }
근대적 사고방식 탓에 우리는 관습의 관념이 사법적 판결에 선행하고
판결은 관습을 확인하거나 그 위반을 벌하는 것이라고 미리 단정짓는
경향이 강하지만, 관념의 역사적 발달은 내가 제시한 순서대로였음이
틀림없어 보인다.
맹아적 관습을 지칭하는 \wi{호메로스}의 용어는 때로 단수형 ``테미스''였고,
종종 ``\wi{디케}''\latin{Dike}였거니와, 그 뜻은 ``판결''과 ``관습'' 또는
``관행''을 넘나드는 것이었다.
노모스\greek{Νόμος}, 즉 `법'은 후대 그리스 사회의 정치용어로서 대단히 중요하고
유명한 것이지만, 호메로스의 시에는 등장하지 않는다.

신의 작용이라는 이러한 관념,
테미스테스를 제안하고 테미스에 인격화되어있는 신의 작용이라는 관념은
피상적인 연구로는 혼동하기 쉬운
다른 원시적 관념들과 엄격히 구분되어야 한다.
힌두의 \wi{마누법전}에 나타나는, 신이 완성된 법전을 명령한다는 관념은
훨씬 최근의 진보된 관념의 계열에 속하는 것으로 보인다.
``테미스''와 ``테미스테스''는
오랫동안 끈질기게 인간의 정신을 지배했던 관념,
신적 영향력이 모든 생활관계와 모든 사회제도를 지탱하고 지지한다는 관념과
훨씬 더 가깝다.
초기 법에서, 그리고 초기의 정치사상에서,
이러한 믿음의 징후는 모든 면에서 나타난다.
초자연적인 통치권자가 당시의 모든 주요 제도들---국가, \wi{씨족}, 가족---을
성별\hanja{聖別}하고 통합하는 것으로 관념된다.
이러한 제도들 속에서 다양한 관계로 집단을 형성하는 인간은
주기적으로 공동의 제의를 수행하고 공동의 희생물을 바칠 의무를 진다.
때로 이러한 의무는
그들이 수행하는 정화의식과 속죄의식에서
더욱 강하게 인식되거니와,
이는 의도치 않게 또는 부주의로 저지른 불경한 짓에 대해 죄를
사하여 달라는 의미를 띠는 것이었다.
고전문헌에 익숙한 독자라면 누구나,
초기 로마의 입양법과 유언법에 중대한 영향을 미쳤던
\wi{씨족제사}\hanjalatin{氏族祭祀}{sacra gentilicia}에 대해 알고 있을 것이다.
무척 진기한 원시사회의 특징들이 고정되어 남아있는
힌두 관습법에서는 지금도 거의 모든 신분법과 상속법 규칙들이
망자의 장례식에서,
즉 가\hanja{家}의 연속성에 단절이 생기는 때에,
의례를 엄정하게 거행하는 것에 달려있다.

\para{벤담의 분석}
이 단계의 법을 떠나기 전에,
특히 영국의 학자들이 유의해야 점을 지적하고자 한다.
\wi{벤담}은 <<정부론 단편>>\latin{Fragment on Government}에서,
\wi{오스틴}은 <<법학의 영역 확정>>\latin{Province of Jurisprudence Determined}에서,
법을 입법자의 \hemph{명령}으로,
그리하여 시민들에게 부과된 \hemph{의무}로,
그리고 불복종에 대해 주어지는 \hemph{제재}의 위협으로 선언한다.
나아가 법의 첫째 요소인 \hemph{명령}은 하나의 행위가 아니라
일련의 또는 다수의 동종 행위들을 지시해야 한다고 단언한다.
이렇게 여러 요소로 분리한 것은 성숙한 단계의 법학에 정확히 부합하는 것이고,
개념을 좀 무리하게 잡아늘이면 모든 시대 모든 종류의 법과
형식적으로 부합하도록 만들 수도 있을 것이다.
하지만, 오늘날에도 일반인들이 가지는 법관념이
이러한 분석과 완전히 일치한다고 주장할 수는 없다.
또한 원시적 사상의 역사를 파고들면 들수록, 이상하게도 우리는
벤담이 말한 요소들의 결합을 닮은 법의 관념으로부터 점점 멀어짐을 발견하게 된다.
확실히 인류의 유년기에는 어떠한 입법도, 어떤 뚜렷한 입법자도 생각될 수 없었다.
법은 관습의 언저리에도 도달하기 어려웠다.
법은 오히려 습관이었다.
프랑스식 표현으로 법은 ``대기 중에 퍼져 있었다''\latin{in the air}.
옳고 그름의 유일한 권위적 진술은 사건이 일어난 뒤에 내려지는 판결이었다.
위반된 법을 전제하여 내려지는 판결이 아니라,
재판의 순간에 저 위의 권력이 판사의 마음에
처음 영감을 불어넣어 내려지는 판결이었다.
물론 우리는 우리와 시간적으로 관념적으로 멀리 떨어진 사고방식을
이해하기가 무척 어렵다.
그러나 고대사회의 헌정을 더 장기간 천착하고 나면 그것은 더 설득력있게
다가올 것이다.
고대사회에서는 모든 사람이
생애 대부분을 가부장의 전제\hanja{專制} 아래서 살았으므로
그의 모든 행위는 사실상 법이 아닌 변덕에 의해 통제되었던 것이다.
생각건대 다른 나라 사람보다 영국인은
``테미스테스''가
어떤 다른 법 관념보다
선행한다는 역사적 사실을 더 쉽게 이해할 수 있을 것이다.
왜냐하면 영국법의 성격에 관한 다양한 이론들 중에서
가장 유명한, 적어도 실무에 가장 영향력 있는, 이론은 분명
판결과 선례가 규칙이나 원리나 개념구분에 선행한다는 이론이기 때문이다.
주목할 점은,
벤담 및 오스틴의 견해에서
법이 단일한 또는 단순한 명령과 구별되었듯이,
``\wi{테미스테스}''에서도 양자가 구별된다는 것이다.
진정한 법은 유사한 종류의 행위를 모든 시민에게 똑같이 명한다.
이것이야말로 대중들의 마음에 깊이 각인된 법의 성질이며,
``법''\latin{law}이라는 말이 단순히 불변성, 연속성, 유사성에도 사용되고 있는
이유이다.\footnote{%
  이런 맥락의 `법'을 우리말로는 보통 `법칙'이라고 부른다. 중력의 법칙 등. }
이에 비해 \hemph{명령}은 하나의 행위만 지시하며,
따라서 ``테미스테스''는 법보다는 명령에 더 가깝다.
그것은 따로 떨어진 하나의 사실관계에 대한 재판일 뿐이며,
전후의 판결들 간에 규칙적인 연계가 반드시 존재하지는 않는다.

\para{귀족정 시기}
영웅시대의 문헌은 ``테미스테스''와 이보다 좀 더 발달된 ``디케''라는 말로써
맹아기의 법을 우리에게 드러내보인다.
법의 역사의 다음 단계는 무척 흥미로운 시기이다.
그로트 씨의 <<역사>> 제2부 제9장은 호메로스가 묘사했던 것과는
사뭇 다른 성격의 사회가 등장하는 과정을 잘 기술하고 있다.\footnote{%
  \latinmarks
  George Grote,
  \textit{History of Greece},
  Vol.\,3,
  Boston: John P. Jewett and Company,
  1852. }
영웅시대의 왕의 권위는 부분적으로는 신에게서 부여받은 대권에,
또 부분적으로는 탁월한 힘과 용기와 지혜를 가진 데 의존했다.
점차, 왕의 신성함에 대한 관념이 약해지고 또
일련의 세습 과정에서 허약한 왕들이 배출됨에 따라
왕의 권력은 쇠퇴했고, 마침내는 귀족정으로 대체되었다.
혁명에 관한 정확한 용어를 사용할 수 있다면,
\wi{호메로스}가 여러 번 언급했던 족장들의 위원회\latin{council of chiefs}에 의해
왕의 자리가 찬탈당했다고 말할 수 있을 것이다.
여하튼 이제 유럽 각지에서 왕정 시대가 가고 과두정의 시대가 도래했다.
왕이라는 직함이 완전히 없어지지 않은 곳에서도 왕의 권위는 그저
이름에 불과했다.
라케다이몬에서처럼 그저 세습장군이거나,
아테네의 \wi{아르콘} 왕처럼 그저 관리이거나,
로마의 \wi{제사왕}\hanjalatin{祭祀王}{rex sacrificulus}처럼
그저 사제\hanja{司祭}에 불과했다.
그리스, 이탈리아, 소아시아에서
지배집단은 어디서나
가상의 혈연관계로 결합된 다수의 가\hanja{家}로
구성되었다.
애초에 그들은 모두 일종의 신성성을 주장했으나,
그들이 힘이
자칭의 신성성에 기반했던 것 같지는 않다.
민중파에 의해 일찍이 전복되어버린 경우가 있었거니와,
그렇지 않은 경우 결국 그들 모두는
오늘날 우리가 정치적 귀족이라고 부르는 것에 아주 근접해갔다.
이탈리아와 그리스 세계의 이러한 혁명에 비해,
더 먼 아시아 지역의 공동체에서의 사회 변화는
물론 시간적으로 훨씬 더 전에 일어났다.
하지만 문명화과정에서 이들 변화의 상대적 위치는 동일했고,
변화의 일반적 성격도 대단히 유사했던 것 같다.
나중에 페르시아 군주정 아래 통합되는 제 민족들이,
그리고 인도 반도 곳곳에 살았던 제 민족들이,
모두 영웅시대와 귀족정시대를 거쳤다는 여러 증거가 있다.
하지만 여기서는 군사적 귀족과 종교적 귀족이 각각 따로 성장했고,
왕의 권위도 대체로 폐기되지 않았다.
또한 서구의 역사 전개와 달리, 동양에서는
종교적 요소가 군사적^^b7정치적 요소를 압도하는 경향이 있었다.
왕과 사제집단의 틈바구니에서 군사적^^b7세속적 귀족은 보잘 것 없이
절멸당하고 파괴당하여 사라진다.
그리하여 도달한 최종 결과는
왕이 커다란 권력을, 그러나 사제계급의 특권에 의해 제한되는 권력을,
누리게 되는 것이다.
동양의 종교적 귀족과 서양의 세속적^^b7정치적 귀족이라는
이러한 차이에도 불구하고,
영웅적 왕의 시대에 이어 귀족정 시대가 도래한다는 역사적 명제는
참이라 간주해도 좋을 것이다.
전 인류에 타당할지는 모르겠으나, 적어도 인도^^b7유럽 계통 민족들에게는
두루 타당한 것이다.

\para{관습법}
법학자들이 주목할 점은
어디서나 이들 귀족이 법의 저장소이고 법의 집행자였다는 것이다.
그들은 이제 왕의 대권을 계승한 것으로 보인다.
그런데 중요한 차이가 있거니와,
그들은 매번 판결마다 직접 신의 영감을 받는다고 내세우지 않았다.
가부장적 족장의 판결이 초인간적 지시에 연결된다는 관념은
법규칙의 전부 또는 일부가 신에게서 기원한다는 주장을 통해 여기저기서 여전히
나타나고 있지만,
사고의 발달로 이제 더는 구체적인 분쟁의 해결을
인간 외적인 힘의 개입을 가지고 설명할 수 없게 되었다.
법적 과두정이 주장하는 바는 이제 법\hemph{지식}의 독점, 즉
분쟁을 해결하기 위한 법원칙을 그들만이 가진다는 것이다.
실로 우리는 \wi{관습법}\latin{customary law}의 시대에 들어선 것이다.
이제 관습이나 관례는 실체적 규칙의 집합으로 존재하고,
귀족 집단 혹은 귀족 카스트가 그것을 정확히 알고 있다고 간주된다.
옛 전거들에 따르면 과두정에 주어진 이러한 신뢰가
때로 남용되기도 했음이 분명하지만,
이를 단순한 찬탈이나 폭정의 장치로만 보아서는 안 될 것이다.
문자의 발명 이전에는, 그리고 기술이 유년기에 머물던 시절에는,
법적 특권을 가진 귀족들이야말로 민족 혹은 부족의 관습을
거의 정확하게 보존하는 유일한 현실적 방법을 구성했다.
공동체의 일부 구성원의 기억에 관습을 맡김으로써
관습의 진정성은 최대한 담보될 수 있었다.

관습법의 시대, 그리고 특권 계급에 의한 관습법의 보존은
자못 흥미를 불러 일으킨다.
당시의 법 상태는 오늘날의 법률용어나 일상용어에도 그 흔적을 남기고 있다.
그리하여
카스트이든, 귀족이든, 사제 지파든, 신관단\latin{sacerdotal college}이든,
특권을 가진 소수만이 알고 있는 법은 진정한 불문법이다.
이것을 제외하면 세상에는 불문법이 존재하지 않는다.
영국 판례법이 흔히 불문법이라 불리고 있고, 또 어떤 영국 학자들은
영국법을 법전으로 편찬하면 불문법이 성문법으로
대체---그들이 비판적인 취지에서 그러나 사뭇 진지하게 사용하는 용어로는,
개종---될 것이라고 주장한다.
물론 영국 보통법을 마땅히 불문법이라고 칭해도 좋을 시기가 한때 있었음이
분명하다.
영국의 옛 판사들은 변호사나 일반인은 온전히 알 수 없는
규칙, 원리, 개념구분 등을 알고 있다고 내세웠다.
그들이 독점한다고 주장한 법의 전부가 진정 불문법이었는지는 무척 의문스럽다.
하지만, 어쨌든 판사들에게만 알려진 민사 및 형사 규칙들이 한때 상당히 있었다고
가정하더라도, 오늘날에는 그것은 더 이상 불문법이 아니다.
웨스트민스터 홀의 법원들이
연감\latin{yearbook} 등에 기록된 선례에 따라 판결을 내리기 시작하면서,
그들의 법은 성문법이 되었다.\footnote{메인의 이러한 성문법 개념은
  오늘날 통용되는 개념과 다르다는 데 주의할 것. 우리는 판례법, 관습법,
  조리법 등을 모두 불문법으로 분류한다.
  메인이 연감에 기록된 옛 보통법 판례의 성문법성을 주장하는 것은
  이를 일종의 `고대법전'으로 간주하기 위해서인 듯하다.}
오늘날 영국의 법규칙은 우선 인쇄된 선례의 사실관계로부터 분리되고,
특정 판사의 성향, 꼼꼼함, 지식에 따라 어떤 언어의 형식으로 만들어진 후,
해당 사건의 사실관계에 적용되는 것이다.
그러나 이 과정의 어느 단계에서도 성문법과 구별되는 성질은 나타나지 않는다.
그것은 성문의 판례법인 것이다.
법전법과 다른 점은 단지 쓰여진 방식이 다르다는 것뿐이다.

\para{12표법}
관습법의 시대로부터 이제 우리는 법제사에 뚜렷이 획을 긋는 다른
시대로 진입하게 된다.
그것은 \index{법전 시대}법전\latin{code} 시대로,
로마의 \wi{12표법}으로 대표되는 고대 법전의 시대다.
그리스에서, 이탈리아에서, 그리스화된 서아시아 해안 지역에서,
이들 법전은 모두 어디서나 동일한 시기에 등장했다.
여기서 동일한 시기란
시간적으로 동시라는 뜻이 아니라,
각 공동체의 상대적 진화 단계에서 유사한 시기를 점한다는 뜻이다.
내가 언급한 지역 어디서나 법은 판자\latin{tablets}에 새겨져 대중에게 공표되었고,
그리하여 특권 귀족의 기억 속에 저장된 관행들을 대체했다.
오늘날의 법전편찬이라는 것에 가까운 어떤 세련된 숙려가
내가 말한 변화에 조금이라도 들어있었다고 생각해서는 안 된다.
고대 법전은 애초에 문자 기술의 발견과 확산에 의해 도입된 것이 분명하다.
물론 귀족들이 법지식의 독점을 남용했음에 틀림없고,
어쨌든 그들에 의한 배타적 법 전유\hanja{專有}가 서구에서 보편적으로 등장하기 시작한
민중 운동의 성공에 커다란 장애가 되었던 것은 사실이다.
하지만, 비록 민주적 감정이 법전의 확산에 도움을 주었을지라도,
대체로 법전은 문자 발명의 직접적 산물이었음이 확실하다.
일군의 사람들의 기억이
비록 반복적 사용에 의해 강화된다 할지라도,
그러한 기억보다는
글자가 새겨진 판자가 법의 저장소로서 더 훌륭했고,
법의 정확한 보존을 더 잘 담보했다.

로마의 법전은 내가 묘사한 그러한 유형의 법전에 속한다.
그것의 가치는 조화로운 분류라든가 표현의 간결성과 명확성 따위에
있는 것이 아니라, 그 공개성, 즉 무엇을 하고 무엇을 하지 말아야 할 지에 관한
지식을 모든 사람들에게 제공하는 데 있었다.
물론 로마의 \wi{12표법}은 어느 정도 체계성을 보여주긴 하지만,
아마도 이는 후기 그리스의 발달된 입법기술을 갖춘 그리스인들의 도움을 받아
12표법이 기초되었다는 전승\hanja{傳承}에 의해 설명할 수 있을 것이다.
하지만 아테네의 솔론 법전의 남아있는 단편들은
체계가 별로 없었음을 보여주며, 아마도 드라콘의 입법은 더욱 그러했을 것이다.
또한 동^^b7서양을 막론하고 이들 법전의 유물들은
종교적, 시민적, 그리고 단순한 도덕적 명령들이
그 성질의 차이를 고려하지 않은 채 무질서하게 혼재되어 있었음을 보여준다.
이는 법 외의 다른 분야의 초기 사상에 관해 우리가 알고 있는 것과 일치한다.
법과 도덕의 분리, 법과 종교의 분리는 정신의 진화에서
분명히 더 후대의 단계에 속하는 것이다.

\para{마누법전}
그러나, 현대인의 눈에 이들 법전이 아무리 이상하게 보일지라도,
고대사회에서 이 법전들의 중요성은 이루 다 말할 수 없을 정도이다.
문제는---이는 각 공동체의 장래에 큰 영향을 미치게 되는 것인데---도대체
법전이 있어야 하는가 아닌가가 아니었다.
대부분의 고대사회는 어쨌거나 조만간 법전을 가지게 되기 때문이거니와,
봉건제에 의해 만들어진 법제사의 큰 단절이 없었다면
모든 근대법은 이들 원천 중 하나 이상으로
기원을 소급할 수 있었을지도 모른다.
오히려 인류 역사의 전기\hanja{轉機}는
사회 진화의 어느 시기, 어느 단계에서 그들의 법이 성문화되었는가와 관련된다.
서양에서는 각 나라의 평민적^^b7민중적 요소가 과두제의 독점을 성공적으로
공격했고, 국가 역사의 비교적 초기에 거의 보편적으로 법전을 획득했다.
하지만 전술했듯이 동양에서는 군사적^^b7정치적 귀족이 아니라
종교적 귀족이 지배 귀족이 되어 권력을 장악하는 경향이 있었다.
그런데 몇몇 경우 서구에 비해 아시아 나라들은 그 물리적 조건으로 인해
개별 공동체가 더 커지고 인구도 더 많아지는 경향이 있었다.
그리고 어떤 제도가 적용되는 공간이 크면 클수록
그 제도의 완고함과 생명력이 더 커진다는 것은 널리 알려진 사회법칙이다.
원인이야 어찌되었든, 동양사회의 법전은 서구에 비해
상대적으로 훨씬 늦게 획득되고, 그리하여 사뭇 다른 성격을 띠게 된다.
아시아의 종교적 귀족들은 스스로 참고하기 위해서든, 기억의 괴로움을
덜기 위해서든, 후계자의 교육을 위해서든, 어쨌거나
그들의 법지식을 종국에는 법전의 형태로 구체화하기에 이른다.
그러나 자신들의 영향력을 확대하고 공고히하려는 유혹이 너무나 강해서
이에 저항하기 어려웠을 것이다. 즉,
법지식을 완전히 독점하고 있었기에 그들은
법전화를 되도록 미룰 수 있었을 것이다.
그들의 법전은 실제로 행해지는 규칙이 아니라,
준수하는 것이 마땅하다고 사제집단이 생각한 규칙들을 모은 것이다.
\wi{마누법전}\latin{Laws of Manu}이라 불리는 힌두법전은 브라만들이 집성한 것으로,
물론 인도인들이 실제로 준수한 것들을 다수 간직하고는 있지만,
오늘날 최고 가는 동양학자들의 견해에 따르면
전체적으로 그것은 인도에서 실제로 행해지던 규칙들의 집합이 아니다.
대체로 그것은 브라만들이 보기에 법\hemph{이어야 할} 것들을
이상적으로 그려놓은 것이다.
인간의 본성을 감안할 때, 그리고 그 저자들의 특별한 동기를 감안할 때,
마누법전 같은 것이 아주 오래 전의 것인양 내세워지고
그 완전한 형태로 신에게서 유래한 것이라 주장되는 것은 당연한 일에 속한다.
힌두 신화에 따르면 마누는 최고 신의 화신이다.
하지만 그의 이름이 붙어있는 법전은, 비록 정확한 연대는 알 수 없지만,
힌두법의 진화 과정 중에 상대적으로 최근의 산물이다.

\para{타락}
\wi{12표법} 등의 법전이 그것을 획득한 사회에 가져다준 주요 이점은
특권 귀족들의 기만적 행태에 대한 보호막을,
그리고 국가 제도의 자연적 타락에 대한 보호막을 제공한 것이었다.
로마의 법전은 단순히 로마 인민의 기존 관습을 언어로 선언한 것이었다.
그것은 로마의 문명화 과정에서 상대적으로 무척 이른 시기에 법전화된 것이었고,
시민적 책무와 종교적 의무가 착종되어 있던 지적 상태를 아직
로마 사회가 거의 벗어나지 못했을 때에 공표된 것이었다.
그런데 이와 달리 여전히 관습을 준행하는 미개한 사회는
문명의 진보에 전적으로 치명적일 수 있는 어떤 특별한 위험에 노출된다.
공동체가 그 유년기에, 원시적 단계에 채택한 관행들은
대체로 그 물질적^^b7정신적 복리의 증진에 가장 적합한 경우가 일반적이다.
새로운 사회적 필요가 새로운 관행을 낳을 때까지 그것들이 순수하게 보존된다면
사회의 상승적 행진은 거의 확실해진다.
하지만 불행하게도 불문\hanja{不文}의 관행에 기초한 작동에는 그것에 위협이 되는
어떤 발전 법칙이 존재한다.
관습을 준수하는 대중들은 그 유용성의 진정한 근거를 알지 못한 채
당연한 듯 관습을 준수하거니와,
따라서 그들은 불가피하게 준수의 미신적 근거를 발명해낸다.
그리하여 합리적인 관행이 비합리적인 관행을 낳는다는 표현으로
간단히 묘사될 만한 어떤 과정이 시작된다.
유추\hanja{類推}는 성숙기 법학에서는 무엇보다 유용한 도구이지만,
유년기에는 무엇보다 위험한 덫이 된다.
어떤 합당한 이유로 애초에 특정한 하나의 행위에만 국한되던 명령과 금지가
동일한 유\hanja{類}의 다른 모든 행위들에도 적용되기 시작한다.
하나의 행위가 야기하는 신의 분노에 두려움을 느낀 인간은
그것과 조금밖에 비슷하지 않은 다른 행위에 관해서도
자연스레 공포를 느끼기 때문이다.
위생상의 이유로 어떤 음식이 금지되면,
그럴듯한 유추에 때로 의존하여
그 금지는 유사한 다른 모든 음식에도 확장된다.
또한, 일반적 청결을 보증하는 현명한 규칙 하나가 이윽고
판에 박힌 의례적 세정\hanja{洗淨}행위의 기나긴 목록을 명령하게 된다.
또한, 역사 과정의 특정한 위기 시에 국가의 존립을 위해 잠시 필요했던
계급의 구분이 인류의 제도 중에 가장 재앙적이고 파멸적인 것---카스트---으로
타락한다.
힌두법의 운명은 실로 로마 법전의 가치를 보여주는 척도다.
민족학은 로마인과 인도인이 원래 동일한 계통에서 발원했음을 알려준다.
사실 그들의 최초의 관습으로 여겨지는 것들 간에는
대단히 큰 유사성이 있다.
오늘날에도 힌두법의 밑바탕에는 선견지명과 건전한 판단이 깔려있다.
하지만 비합리적인 모방으로 인해 잔인하고 부조리한 거대한 장치가
힌두법에 접목되었다.
로마인들은 그들의 법전에 의해 이러한 타락으로부터 보호될 수 있었다.
그것은 그들의 관행이 아직 건강했을 때 편찬되었거니와,
만약 백년 후였다면 너무 늦었을지도 모른다.
힌두법은 그 대부분이 성문화되었다.
그러나,
산스크리트어로 전해지는 집성들은 일응 오래된 것이긴 하지만,
해악이 작용한 연후에 작성되었다는 풍부한 증거를 담고 있다.
만약 12표법이 공표되지 않았다면 로마인들도 인도인들처럼
허약하고 타락한 문명으로 전락할 운명이었을지에 관해
물론 우리는 아무 것도 말할 수 없다.
하지만 한 가지 확실한 점은 그들의 법전과 \hemph{더불어}
로마인들은 저 불행한 운명으로부터 벗어날 수 있었다는 것이다.



\chapter{법적의제}

원시법이 법전에 구체화되면서, 자생적 발달이라고 할 만한 것은
종말을 맞았다.
이후로는 법 내부에 변화가 일어난다면 그것은 의도적으로 일어난,
그리고 외부로부터 영향받은 변화인 것이다.
어떤 민족이나 부족의 관습이
가부장적 왕에 의해 선언된 이후 마침내
성문화되어 공표되기까지 그 긴 시간---몇몇 경우에는 장구한 기간---동안
전혀 변함 없이 유지된다는 것은 상상할 수 없는 일이다.
또한 그 변화의 어떤 부분도 의도적으로 일어난 부분이 전혀 없다고
단정하는 것도 옳지만은 않을 것이다.
그러나,
이 기간의 법발달에 대해 우리가 아는 바가 별로 없긴 하지만,
변화를 가져옴에 있어 미리 계획된 목적이 차지하는 몫은
극히 작았을 것이라고 가정해도 무리가 없다.
초창기의 관행에 일어난 그러한 혁신은,
오늘날 우리의 정신 조건을 가지고는 도저히 이해할 수 없는 감정과 사고양식에 의해
주어졌던 것 같다.
하지만 \index{법전 시대}법전과 더불어 새로운 시대가 시작된다.
법전 시대 이래, 법변동의 경로 어디를 추적하더라도
그것이 의식적인 개선 노력에 기인한다는 것을,
적어도 원시 시대에 목표했던 것과는 다른 목표를 달성하려는 노력에
기인한다는 것을 발견할 수 있다.

\para{진보의 희귀성}
언뜻 보면, 법전 시대 이후의 법의 역사에서 어떤 믿을 만한 명제를
이끌어내는 것은 불가능해보인다.
대상 영역이 너무 넓다.
충분히 많은 수의 현상을 관찰했는가,
또 관찰한 것을 정확하게 이해했는가, 따위에 대해 우리는 확신을 가질 수 없다.
그러나,
\index{정체된 사회|see{진보하는 사회}}정체된 사회\latin{stationary society}와
\wi{진보하는 사회}\latin{progressive society} 간의 구별이
법전 시대 이후
나타나기 시작했음을
감안하면, 우리의 과업이 불가능해 보이지는 않는다.
우리의 관심대상은 진보하는 사회에 국한되거니와,
그것들의 숫자가 무척 적다는 점이 무엇보다 두드러진다.
압도적인 증거에도 불구하고, 서유럽 시민의 한 사람으로서
그를 둘러싸고 있는 문명이 세계 역사에서 희귀한 예외에 불과하다는 사실을
완전히 체감하기란 결코 쉬운 일이 아니다.
전체 인류에 대한 진보적 민족의 관계를 또렷이 직시한다면
우리가 공유하는 사상의 풍조, 우리들의 모든 희망, 두려움, 생각이
크게 바뀔 수 있을 것이다.
의심할 여지 없이, 인류의 대부분은
문명제도들을 항구적 기록으로 구체화하여 외면적 완성을 이룩한 순간 이후로
그 문명제도들을 개선하려는 일말의 욕구조차
보여준 적이 없었다.
때로 어떤 관행이 폭력적으로 전복되어 다른 관행에 자리는 내주는 경우는 있었다.
곳에 따라 원시 법전은,
초자연적 기원을 내세우며 대폭 확대되기도 했고,
종교적 주석가들의 왜곡을 거치며 놀랄 만한 형태로 뒤틀려지기도 했다.
하지만 이 세상의 아주 작은 한 지역을 제외하면
법체계의 지속적^^b7점진적 개량 같은 것은 찾아볼 수 없었다.
물질문명은 있었지만, 문명이 법을 확장시키기보다는
법이 문명발달의 족쇄로 작용했다.
인류의 원시적 상태를 연구함으로써 우리는
어떤 문명이 그 발달을 멈춘 지점에 관하여 단서를 얻을 수 있을 것이다.
브라만 지배의 인도는 모든 인간사회가 경험한 단계, 즉
법규칙과 종교규칙이 아직 구별되지 않던 단계를 넘어서지 못했음을 알 수 있다.
그런 사회의 구성원들은 종교적 명령의 위반을 세속적 형벌로 처벌해야 한다고
믿었고, 세속적 의무의 위반을 신의 교정\hanja{矯正}에 맡겨야 한다고 믿었다.
중국은 이 지점을 넘어서긴 했으나,
진보는 거기서 정체되었으니,
민사법이 중국인들의 관념의 한계에 갇혀있었기 때문이다.
하지만 정체된 사회와 진보하는 사회의 차이는
커다란 비밀에 싸여있고 우리는 그 비밀을 여전히 탐구해야 한다.
지난 장의 끝부분에서 나는 이 비밀에 대한 부분적 설명을 시도한 바 있다.
덧붙여 유의할 점은 인류에게 있어 정체된 사회가 일반원칙이고
진보하는 사회가 예외임을 정확히 인식하지 않으면 우리의 탐구는
성공할 수 없으리라는 것이다.
그리고 또 하나의 성공조건은 모든 주요 단계마다 로마법에 대한 정확한 지식이
불가결 요구된다는 것이다.
우리가 알고 있는 모든 인간제도들 중에서 로마법은 가장 긴 역사를 가지고 있다.
로마법이 경험한 모든 변화의 성격을 우리는 비교적 잘 알고 있다.
시작부터 종말에 이르기까지 그것은 보다 나은 방향으로,
혹은 변화의 설계자들이 보기에 더 낫다고 여겨졌던 방향으로,
변화하며 진보했다.
로마법이 개선되어 나가는 동안,
인류의 나머지 부분들은 사상과 행위의 진전이 눈에 띄게 느려졌고,
정체상태로 빠질 위험에 끊임없이 노출되었다.

\para{진보적인 법}
이하에서 나는 진보하는 사회에 국한하여 논의를 전개하겠다.
이들 사회에서는 사회의 필요와 사회의 여론이 대체로 법에 선행한다고
말할 수 있다.
그것들 간의 간격이 끊임없이 메워지는 경향을 보이지만,
그 간격은 항상 또 다시 되살아난다.
법은 안정을 추구하지만, 지금 우리가 말하고 있는 사회는 진보하는 사회인 것이다.
인민의 행복은 이 틈새가 얼마나 신속하게 좁혀지는가에 달려있다.

법이 사회와 조화되도록 하는 장치에 관하여 유용한 명제 하나를
개진하고자 한다.
이러한 장치에는 세 가지가 있거니와,
\wi{법적의제}\latin{legal fictions}, \wi{형평법}\latin{equity},
그리고 \wi{입법}\latin{legislation}이 그것이다.
이들 간의 역사적 순서는 내가 제시한 대로이다.
때로는 이들 중 두 가지가 동시에 작용하기도 하고, 또
이들 중 하나의 영향을 받지 않은 법체계도 존재한다.
그러나 내가 아는 한 이들의 등장 순서가 뒤바뀐 사례는 존재하지 않는다.
이들 중 하나인 형평법은 그 초기 역사가 어디서나 모호했고,
따라서 어떤 이는 시민법을 개혁하는 단발적인 법률들이 형평에 의한 재판보다
더 오래됐다고 생각할지도 모른다.
나는 형평법에 의한 구제가 입법에 의한 구제보다 어디서나 더 먼저였다고 믿는다.
그러나 만약 이것이 완전히 진리가 아니라면,
그들의 순서에 관한 명제를,
초창기 법의 변화에 그들이 지속적이고 실질적인
영향력을 행사한 기간에
국한해야 할 필요성이 있을 수는 있다.

\para{의제의 용도}
나는 ``의제''라는 단어를 영국 법률가들이 익히 사용하고 있는 것보다
훨씬 더 넓은 의미에서 사용한다. 또한 로마인들이 ``의제''\latin{fictiones}라는
말에 부여했던 것보다 훨씬 더 포괄적인 의미로 사용한다.
고대 로마법에서 의제\latin{fictio}는 기실 소송변론상의 용어였으니,
원고 측의 거짓 진술로서 피고가 이를 부인하는 것이 허용되지 않는 것을 뜻한다.
가령 원고가 사실은 외인\hanja{外人}이면서 로마시민이라고 진술하는 것이 그 예다.
이러한 로마법상 ``의제''의 목적은 말할 것도 없이 재판권을 부여하기 위한
것이었다.
\hypertarget{commonlawfiction}{따라서} 이는 영국의 왕좌법원\hanjalatin{王座法院}{Queen's Bench}이나
재무법원\latin{Exchequer}의 영장\latin{writ}에 담긴
진술---피고가 국왕의 감옥에 구금되어 있다는 진술, 혹은
원고가 왕의 채무자인데 피고의 채무불이행으로 인해
자신의 채무를 이행할 수 없다는
진술---과 무척 흡사하거니와,
이로써 이들 법원은 민소법원\hanjalatin{民訴法院}{Common Pleas}의 재판권을 빼앗아올 수 있는
것이다.
그러나 내가 사용하는 ``\wi{법적의제}''라는 표현은
법규칙의 문언은 그대로인 채 그 실제적 작용이 바뀐 변화의 사실을
숨기거나 숨기는 데 영향을 주는 일체의 가정\hanja{假定}을 총칭한다.
따라서, 앞서 인용한 로마법과 영국법의 의제 사례들뿐만 아니라
그 이외의 것도 여기에 포함된다. 나는 영국의 판례법과 로마의
\wi{법학자의 해답}\latin{responsa prudentium}도 의제에 기초한 것으로 보기 때문이다.
이들 두 가지에 대해서는 조금 있다 설명할 것이다.
이들 두 경우, \hemph{사실}로는 법이 완전히 변화했으나,
\hemph{의제}적으로는 법이 예전 그대로 동일한 것이다.
모든 형태의 의제가 왜 사회의 유년기와 특히 친화성이 있는지는 어렵지 않게
이해할 수 있다.
의제는 가끔 등장하는 개선 욕구를 충족시키면서도
변화에 대한 상존하는 미신적 거부감을 거스르지 않기 때문이다.
사회진화의 특정 단계에서 의제는 법의 엄격함을 극복하는 유용한 수단이 된다.
실로 그중 하나인, 인위적인 가족관계 형성을 가능케 하는
\wi{입양}\hanja{入養}이라는 의제가 없었다면, 어떻게 사회가 그 요람기를 벗어나
문명을 향한 첫걸음을 뗄 수 있었을지 상상하기 어렵다.
그러므로 우리는 \wi{벤담}이 법적의제에 대해 퍼부은 조롱과 비난에 마음상할
필요가 없다.
의제를 속임수에 불과하다고 욕하는 것은 법의 역사적 발달에서
의제가 수행한 특수한 역할에 대한 무지를 드러낼 뿐이다.
하지만 동시에, 의제의 유용성을 인식하면서 우리 법체계에 의제가
확고하게 뿌리내려야 한다고 주장하는 일부 논자들에게 동조하는 것도 똑같이
어리석은 일이 될 것이다.
영국 법률가들의 관념에 심각한 충격을 주지 않고
그들의 언어에 중대한 변화를 초래하지 않는 한
내다버릴 수 없는 몇몇 의제들이 여전히 강력한 영향력을 영국법에 행사하고 있다.
하지만 법적의제와 같은 거친 장치로써 어떤 유익한 결과를 도모하는 것이
우리에게는 어울리지 않는다는 것도 틀림없이 일반적 진리일 것이다.
법을 더 이해하기 어렵게 만들거나 조화로운 질서의 형성을 더 어렵게 만드는
어떠한 변칙도 무고하지 않다고 나는 생각한다.
그런데 여러 장애 중에서도 법적의제야말로 체계적인 분류에 가장 큰
장애가 된다.
저 법규칙은 여전히 법체계에 들러붙어 있으나,
그것은 껍질에 불과하다.
저 규칙은 이미 오래 전에 쇠퇴했고, 새로운 규칙이 껍질 아래 몸을 숨기고 있다.
그리하여 실제 작동하는 규칙을 그 진정한 장소에 분류해야 할지,
아니면 그 외관상의 장소에 분류해야 할지 알기 어려운 상황이 발생하거니와,
어느 선택지를 택할 지를 두고 여러 부류의 학자들 간에 의견이
갈라지는 일이 발생할 것이다.
영국법이 질서있는 분류를 채택하려 한다면,
최근의 몇몇 입법적 개선에도 불구하고 여전히 영국법에 널리 퍼져있는
법적의제들을 뿌리뽑지 않으면 안 될 것이다.

\para{형평법}
사회적 필요에 법이 적응하는 또 다른 수단은 내가 형평법이라 부르는 것이다.
여기서 \wi{형평법}이란 초창기 시민법에 병존하는 법체계로서
독자적인 원리에 기초하고 있고 그 원리에 내재한 우월한 신성함에 기대어
시민법을 넘어선다고 주장되는 것을 말한다.
로마 \wi{법무관}\latin{praetor}들의 형평법이든,
영국 챈슬러\latin{chancellor}들의 형평법이든,
형평법은
개방적이고 공공연하게 기존 법에 간섭한다는 점에서
각각의 경우 그것에 선행했던 의제들과 차이가 있다.
한편, 형평법은 법 개선의 동인으로 나중에 등장하는 입법과도 다르다.
형평법의 권위는
법 바깥의 어떤 사람이나 집단의 대권\hanja{大權}이 아니라,
법을 천명하는 정무관의 대권이 아니라,
모든 법이 따라야 한다고 여겨지는 법원리의 특별한 성격에
근거하고 있다는 점에서 그 차이가 있는 것이다.
초창기 법보다 더 높은 신성함을 가지고 있고
외부 기관의 승인과 무관하게 효력을 주장하는
일련의 원리들이라는 이러한 관념은
법적의제가 처음 등장했던 사고 단계보다 더 발달된 단계에 속한다.

\para{입법}
전제군주의 형태로든, 의회의 형태로든,
전체 사회를 대표한다고 간주되는 입법기관의 법제정인 \wi{입법}은
법 개선 수단 중에서 마지막 것이다.
입법과 법적의제의 차이는 형평법과 법적의제의 차이와 동일하다.
입법은
그 권위가 외부의 기구나 사람에게서 나온다는 점에서
형평법과도 구별된다.
입법의 구속력은 그것의 법원리와 무관하다.
현실적으로는 여론에 의한 제약이 있다 하더라도,
이론적으로 입법기관은 스스로가 원하는 바를 공동체 구성원들에게
의무로 부과할 권한을 가진다.
입법기관이 자의적 변덕에서 하는 입법을 막을 것은 아무 것도 없다.
만약 형평이 어떤 선악의 기준을 뜻하는 말로 사용되고
법제정이 어쩌다 이러한 기준에 맞추어 행해진다면,
그러한 입법은 형평에 의해 지시된 것이라 할 수 있을 것이다.
하지만 이런 경우에도 법제정의 구속력은 입법기관의 권위에 빚지고 있는 것이지,
입법기관의 행위 근거가 된 원리의 권위에 빚지고 있는 것이 아니다.
그리하여 입법이 기술적 의미의 용어인 형평법 규칙과 다른 점은,
후자는 최고의 신성함을 내세우며 군주나 의회의 협찬이 없더라도
즉각 법원에 받아들여질 것을 요청한다는 데 있다.
이런 차이에 주목해야 할 더 큰 이유는,
어떤 벤담 학도는 법적의제, 형평법, 제정법을 뭉뚱그려
이 모두를 입법이라는 단일 범주로 포괄하려 할 것이기 때문이다.
이 모두가 \hemph{법창조}\latin{lawmaking}에 관한 것으로,
그것들 간 차이는 단지 새 법이 만들어지는 장치의 차이일 뿐이라고 그는
말할 것이다.
이것은 분명 진실이고 우리는 이것을 망각해서는 안 된다.
하지만 그렇다고 해서 입법과 같은 무척이나 편리한 용어를
특수한 의미로 사용해서는 안 될 이유가 되지는 못한다.
입법과 형평법은 대중의 정신에서, 그리고 대부분의 법률가들의 정신에서,
서로 분리되어 있다.
특히, 중요한 실제적 결과의 차이가 뒤따른다면,
아무리 인습적이라 해도 양자의 차이를 무시하는 것은 결코 정당화될 수 없다.

\para{법적의제}
거의 모든 발달된 법체계에서 \hemph{법적의제}의 사례들을 선별하기란
쉬운 일일 것이며, 그것들은 즉시 법적의제의 진정한 성질을 현대의 관찰자들에게
드러낼 것이다.
하지만 이제부터 내가 다루려는 두 가지 경우에는
거기에 사용된 수단의 본질이 그리 쉽게 드러나지 않는다.
이들 의제의 최초 창시자들은 아마 혁신을 의도하지 않았을 것이며,
혁신의 의심을 사기는 더더욱 바라지 않았을 것이다.
게다가 그러한 혁신의 과정에 의제가 들어있음을 부인하는
사람들이 늘 있고 또 있어왔거니와,
전래의 인습적 언어가 그들의 부인\hanja{否認}을 실증한다.
그러므로 \wi{법적의제}의 광범위한 확산을 보여주는,
그리고 법체계를 변화시키면서도 그 변화를 감추는 이중적 역할의
효율적 수행을 보여주는 사례로서 이보다 더 나은 것들은 없을 것이다.

\para{사법적 입법}
이론적으로는 조금도 기존 법을 바꿀 힘이 없는 장치가
법을 확대하고 수정하고 개선해나가는 것에
우리 영국인들은 아주 익숙하다.
이러한 사실상의 입법이 작동하는 과정은 감지될 수 없는 것이 아니라
인정되지 않을 뿐이다.
판례들에 담겨있고 판결집들에 기록돼있는 우리 법체계의 방대한 부분에 대해
우리는 습관적으로 이중적 언어를 사용하고 이중의 모순적인 관념들을 구사한다.
어떤 사실관계가 영국 법원에 제소되면
판사와 변호사들 간의 모든 논쟁은
옛 법원리 외의 어떤 법원리도,
오래된 개념구분 외의 어떤 개념구분도
적용될 필요가 없고 적용될 수도 없다는
가정 하에 진행된다.
계쟁 분쟁의 사실관계를 포섭하는 기존의 법규칙이 어딘가에 존재하며,
설령 그러한 규칙이 발견되지 않더라도 인내, 지식, 통찰력을 발휘하면
얼마든지 그것을 찾아낼 수 있다는 믿음을 극히 당연한 것으로 받아들인다.
그러나 일단 판결이 내려지고 기록되고 나면, 우리는 무의식적으로 혹은 은밀하게
새로운 언어, 새로운 사고 맥락으로 넘어간다.
이제 우리는 새로운 판결로 법이 수정\hemph{되었다}고 인정한다.
적용가능한 법규칙이, 흔히 쓰이는 부정확한 표현을 사용하자면,
보다 유연해졌다고 믿는다.
실제로 법규칙은 변경되었다.
선례에 새로운 것이 첨가되었고, 선례들을 비교하여 얻어지는 법원리는
일련의 판례들에서 하나의 사례를 제외했을 때 얻어지는 것과는 다른 것이 되었다.
옛 규칙이 폐지되고 새로운 것으로 대체되었다는 사실을 우리는 받아들이기 어려운데,
선례에서 얻어지는 법적 공식을 정확한 언어로 표현하는 습관을 갖고 있지 못하여,
변화의 광채가 강렬하고 눈부신 것이 아닌 한 쉽게 포착하지 못하기 때문이다.
진기한 변종 판결 앞에서 영국 법률가들이 침묵으로 일관하는
이유를 여기서 장황하게 늘어놓을 생각은 없다.
아마도, 구름 속이든\latin{in nubibis} 혹은
판사의 마음 속이든\latin{in gremio magistratuum} 어딘가에
완전하고 일관되고 체계적인 영국법이 존재한다는, 그리하여 상상할 수 있는 어떤 상황에도
적용할 수 있는 풍부한 법원리의 체계가 존재한다는 것이 전래의 교리였다는 것은
발견할 수 있을 것이다.
처음에는 이 이론이 지금보다 훨씬 더 철저히 신봉되었으며,
실제로 그럴 만한 근거가 더 충분했다.
13세기 판사들은 변호사나 일반 대중에게는 알려지지 않은
법의 보고\hanja{寶庫}를 이용할 수 있었으니,
그들은 은밀히 당대의 로마법과 \wi{교회법} 집성들로부터, 항상 현명하게는 아닐지라도,
자유롭게 빌려왔다고 믿을 만한 이유가 있다.
하지만 웨스트민스터 홀의 법원들이 판결을 양산하여 실체법 체계의 토대가 마련되자,
이 저장고는 폐쇄되었다.
그리하여 수 세기 동안 영국의 법률가들은, 형평법과 제정법이 아닌 한 아무 것도
이미 형성된 이 토대에 첨가된 것이 없다는 역설적인 명제를 전승시켜온 것이다.
우리는 우리 법원들이 입법을 한다는 것을 인정하지 않는다.
우리는 우리 법원들이 결코 입법을 한 적이 없다고 생각한다.
그럼에도 불구하고 우리는 영국의 보통법 규칙들이, 형평법법원\latin{Court of Chancery}과
의회로부터 약간의 도움을 받아, 현대사회의 복잡한 이해관계에 충분히 대처할 수 있다고 주장한다.

\para{법학자의 해답}
방금 언급한 특징에 있어서 우리 판례법과 무척 가깝고 교훈적인 유사성을 가진 법체계가
로마에서는 ``법에 식견 있는 자의 답변''이란 뜻의 \wi{법학자의 해답}\latin{responsa prudentium}이었다.
이들 해답은 로마법의 발달 시기에 따라 상당히 다른 형태를 띠었지만,
전 시기에 걸쳐 어떤 권위 있는 성문의 문헌들을 해설하는 주석임에는 변함이 없었고,
처음에는 오로지 12표법에 대한 해석 의견의 모음이었다.
우리와 마찬가지로, 모든 법적 언어는 이 옛 법전의 텍스트가 불변이라는 가정에 기초했다.
거기에 명시적인 규칙이 있었다.
그것은 어떤 주석이나 주해보다 위에 있었고, 어떤 해석도, 설령 위대한 해석자의 것이라 해도,
거룩한 텍스트에 호소하여 수정될 수 있음을 누구도 공공연히 부인할 수 없었다.
하지만 사실 저명한 법학자의 이름을 달고 있는 해답집은
적어도 우리의 판결집에 버금가는 권위를 누렸고,
12표법의 규정을 지속적으로 수정하고 확장하고 제한하고 사실상 뒤엎었다.
새로운 법학의 형성기 동안 법학의 저술가들은 법전의 문구에 꼼꼼한 충실함을 내세웠다.
단지 그것을 설명하고 독해하고 그 의미를 온전히 드러낼 뿐이었다.
그러다 결국 그들은, 텍스트를 이어붙이고,
실제로 발생한 사실관계에 법을 적응시키고,
일어날 법한 사실관계에 법의 적용가능성을 탐구하고,
다른 성문 문헌에서 도출한 해석 원리를 가져오는 등에 의해
12표법의 편찬자들은 꿈도 꾸지 못했던, 실로 12표법에서는 거의 혹은 전혀 찾아볼 수 없는
사뭇 다양한 법원리들을 이끌어냈다.
법학자들의 저술은 모두 법전과 일치한다는 근거에서 존중받을 자격을 주장했으나,
그것의 상대적 권위는 저술을 발표한 특정 법학자의 명성에 크게 좌우되었다.
널리 알려진 위대한 학자의 이름은 입법기관의 법제정에 버금가는 구속력을 해답집에 부여했다.
그리고 그러한 저서가 이번에는 한층 더 나아간 법학 발달의 새로운 토대로 작용했다.
하지만 초기 법학자들의 해답은 오늘날처럼 저자에 의해 출간된 것이 아니었다.
그것은 그의 학생들이 기록하고 편집한 것이어서,
대개는 어떤 체계적인 분류법에 따라 배열된 것이 아니었다.
이렇게 출간에 있어 학생들이 한 역할은 특히 주목할 필요가 있거니와,
그들이 스승에게 행한 봉사는 학생 교육에 대한 스승의 충실성에 의해 보상받는 것이 일반적이기 때문이다.
후대에 가서 이 의무의 결실로 인정받게 되는
\wi{법학제요}\hanjalatin{法學提要}{Institutes}, 즉
주해서\latin{Commentary}라 불리는 교육용 저술들은
로마법 체계의 사뭇 중요한 특징을 이루는 것이다.
법학자들이 대중들에게 개념의 분류와 전문용어의 개선을 제안한 것은
이러한 법학제요 형태의 작품에서였지, 훈련된 법률가들을 겨냥한 저서에서가 아니었다.

로마의 \wi{법학자의 해답}과 그것의 영국적 대응물을 비교할 때,
로마법학의 이 부분이 가지는 권위는 \hemph{판사직}\latin{bench}이 아니라
\hemph{변호사직}\latin{bar}에서 유래한다는 점에 주의해야 한다.
로마에서 법원의 결정은, 개별 사건을 종결짓는 것이었지만,
어떤 장래를 향한 권위도 가지지 못하였고, 다만
해당 사건을 잠시 담당하게 된 정무관의 전문직업적 명성에 의해 주어지는
권위만 누릴 뿐이었다.
사실 공화정기 동안 로마는 영국의 왕좌법원\latin{Bench}이나
신성로마제국의 제실법원\hanjalatin{帝室法院}{Chamber},
프랑스왕국의 파를르망\latin{Parliament} 비슷한 제도를 전혀 알지 못했다.
각자 맡은 분야의 사법적 기능을 그때그때 담당하는 정무관\latin{magistrate}들은 있었지만,
정무관의 임기는 1년에 불과했기에, 그것은 상설 법관이라기보다는
정상급 변호사들이 돌아가면서 잠깐씩 맡는 순환 공직에 가까웠다.\footnote{%
  기실
  법무관(praetor) 등 정무관은 변호사---오늘날의 전문직 법률가로서의
  변호사가 아니라 웅변가(orator)였음---나 법학자일 수도 있지만
  그냥 정치가인 경우도 많았다.}
우리 눈에는 무척 이상하게 보이는 이러한 제도의 기원에 대해 다양한 견해가 있을 수 있지만,
사실 그것은 현대 우리의 제도보다 고대사회의 정신에,
개별 신분집단들로 분열되지만 아무리 배타적이더라도
그들 간에 전문직업적 상하관계는 허용하지 않는 정신에, 더 잘 부합했다.

이 체제는 그로부터 기대할 법한 효과를 가져오지 못했다는 점에 주목할 필요가 있다.
가령 그것은 로마법을 \hemph{대중화}하지 못했다.
비록 법학의 확산과 권위 있는 해설에 인위적인 장벽을 두지는 않았으나,
몇몇 그리스 공화국에서처럼 법학을 습득하는 데 필요한 지적 노력을 완화해주지 못했다.
오히려, 어떤 다른 원인들이 작동하지 않았더라면, 후대의 지배적 법체계들처럼 로마의 법학도
사소한 데 치중하고 기술적이고 배우기 어려운 학문이 되었을 확률이 상당히 컸다.
또한, 훨씬 더 마땅히 발견될 법한 어떤 결과도 전혀 나타나지 않은 듯하다.
로마 공화정이 무너지기 전까지 법학자들은 명확하게 정의되지 않은 집단을 형성하고 있었던 것이다.
또한 그 숫자도 틀림없이 큰 폭으로 오르내렸을 것이다.
그럼에도 불구하고, 주어진 사례에 대해서 어떤 사람의 의견이 그들 세대에서 결정적인 권위를
누렸는지는 의심의 여지가 거의 없었던 것 같다.
여러 라틴어 문헌에 전해지는, 정상급 법학자들의 일상 업무에 관한
생생한 묘사---이른 아침부터
시골에서 올라온 고객들이 그의 대기실에 몰려들고,
공책을 든 학생들은 그의 주변에 둘러서서 위대한
법률가의 답변을 기록한다---는 일정 기간에 국한해 본다면
한 두 명의 저명한 이름을 거의 혹은 전혀 벗어나지 않는다.
또한 고객들과 변호사의 직접적인 접촉 덕분에,
로마 사람들은 전문가들의 명성의 오르내림을 즉각적으로 알고 있었던 듯하다.
저 유명한 키케로의 <<무레나를 위한 변론>>\latin{Pro Muraena}을 비롯한 풍부한 증거가
있거니와, 법정에서의 성공에 대한 일반인들의 존경은 과도하면 과도했지 부족하지 않았다.

의심할 여지 없이,
로마법의 발달을 추동한 수단에 관한 전술한 특징은
그것의 우수성, 즉 일찍부터 법원리가 풍부했던 것의 원천이었다.
법원리의 성장과 풍부함은 부분적으로는 법해설자들 간의 경쟁에 의해
촉진되었으니, 국왕이나 국가에 의해 부여된 사법권의 담지자인
법관직이 존재하는 곳에서는 이러한 것이 작동할 수 없는 것이다.
하지만 주된 동력은 말할 것도 없이 사법판결의 대상이 되는 사례의
무제한적 증가에 있었다.
시골 고객들을 당혹케했던 사실관계들이 법학자의 해답이나 사법판결의
토대가 되었을 뿐만 아니라, 똑똑한 학생들이 제기한 가상의 사례들도
그에 못지 않았다.
실제 사례든 가상의 사례든, 모든 사실관계는 자격에 있어 차이가 없었다.
법학자들로서는 그의 고객의 사건을 재판하는 정무관이 그의 의견을
퇴짜놓는다고 해도 아무 문제가 되지 않았다.
오히려 정무관이 법지식에 있어서나 전문직업적 평판에 있어서 자신보다
위에 있는 것이 문제였다.
그렇다고 해서 법학자들이 고객의 이익에 무심했다는 말은 아니다.
고객들은, 초기에는 저명한 법률가들의 선거인단이었고
후기에는 돈을 벌게 해주는 사람들이었기 때문이다.
그러나 야망을 충족시키는 주된 길은 동료 집단의 평판을 통해서였던 것이다.
전술한 이러한 체제 하에서 평판을 확보하는 좋은 방법은 각 사례를,
법정에서 승리하기 위한 고립된 사건으로 접근하는 것이 아니라,
어떤 포괄적 법원리나 법규칙의 예시의 하나로 바라보는 것이다.
있을 수 있는 사례를 제시하거나 발명해내는 데 아무런 제약이 없었던 것도
큰 영향력을 발휘했을 것임에 틀림없다.
데이터를 마음껏 증가시킬 수 있는 곳에서는
일반적 규칙을 진화시키는 능력이 대폭 증대된다.
우리의 사법체계에서는 판사들이 자기 앞에 놓인,
혹은 그의 전임자들 앞에 놓였던 사실관계를 벗어날 수가 없다.
따라서 재판의 대상이 된 각 사실관계는,
프랑스 식으로 표현하면, 일종의 성별\hanja{聖別}이 이루어진다.
실제 사건이든 가상의 사례든, 다른 모든 사건들과 구별되는 성질을 가지는 것이다.
하지만 로마에서는, 전술한 바에서 짐작할 수 있듯이
판사집단\latin{bench}이나 판사들의 법원\latin{chamber}
같은 것이 전혀 없었고,\footnote{%
  여기서 판사(judge)는
  법무관이 방식서(formula)에서 지시한 바에 따라
  사실관계를 심리하고 판결을 내리는
  심판인(iudex)을 뜻하는 듯하다.
  로마법 발달의 주역은 심판인들이 아니라 법학자들과 법무관들이었다. }
따라서 어떤 사실관계도 다른 사실관계보다 더 특별한 가치를 지니지 않았다.
어떤 어려운 사안이 법학자의 의견을 요청하는 경우,
뛰어나 유추 감각을 지닌 이는 거리낌없이 그것과 어떤 특징을 공유하는
모든 상상할 수 있는 사례들을 즉시 인용하고 고려할 수 있었다.
고객에게 주어진 실무적 조언이 무엇이든 간에,
학생들의 공책에 씌어진 해답\latin{responsum}은 분명
숭고한 법원리로 규율되는, 또는 포괄적인 법규칙에 포섭되는,
그러한 사실관계를 고려하였을 것이다.
우리에게는 이러한 것이 한 번도 가능한 적이 없었다.
그리고 영국법에 가해진 수많은 비판 속에서
영국법이 선언되는 양식에 대한 비판은 잊혀져버린 것 같다고 인정하지 않을 수 없다.
우리 법원이 법원리를 선언하는 데 인색한 것은
우리 판사들의 기질 탓보다는
우리에게 선례가,
다른 법체계들을 알지 못하는 이들에게는 많아 보일지 모르나,
상대적으로 부족한 데 더 큰 원인이 있는 듯하다.
법원리의 풍부함에 있어 여러 근대 유럽대륙의 국가들에 비해
우리가 대단히 빈약한 것이 사실이다.
하지만 그들은 민사법 제도의 기초로 로마법을 채택했음을 기억해야 한다.
그들은 로마법의 파편들을 가지고 그들의 성채를 건설했다.
그러나 그밖의 재료나 솜씨에 있어서는 영국 법원이 건설한 구조보다
우월할 것이 별로 많지 않다.

\para{이후의 로마법}
로마 공화정기는 로마법학에 그 특징이 각인된 시기였다.
로마법학의 초기 동안 법학자의 해답이 법발달의 주역이었다.
그러나 공화정의 몰락이 다가오면서 해답들은 더 이상의 확장을 저해하는
형태를 띠기 시작한 것으로 보인다.
그것들은 이제 체계화되어갔고 단순한 모음집이 되어갔다.
신관\hanjalatin{神官}{pontifex}이었던
무키우스 스카이볼라\latin{Q. Mucius Scaevola}는
시민법 전체의 매뉴얼을 출간했다고 한다.
키케로의 저술들은
능동적인 법 혁신 수단들에 대비되는 낡아빠진 방법들에 대한 염증이
커지고 있었음을 보여준다.\footnote{%
  가령 키케로, <<법률론>>, 1.14. }
사실 이때쯤이면 다른 요인들도 법에 영향을 미치게 된다.
\wi{법무관}이 매년 선포하는 \index{고시|see{법무관}}고시\hanjalatin{告示}{edict}는
이제 법개혁의 주된 동력으로 인정받고 있었다.
코르넬리우스 술라\latin{L. Cornelius Sylla}는
코르넬리우스 법\latin{Leges Corneliae}이라 불리는 일련의 위대한 법률들을
제정함으로써 직접적 \wi{입법}에 의해 얼마나 빨리 개선이 이루어질 수 있는지
잘 보여주었다.
\wi{법학자의 해답}에 최종 일격을 가한 것은 아우구스투스였다.
제출된 사안에 대해 구속력 있는 해답을 줄 수 있는 권리를
몇몇 정상급 법학자들에게만 부여한 것이다.
이 변화는, 비록 근대적 관념에 가까이 다가가는 것이기는 하나,
확실히 법전문직의 성격 및 그것이 로마법에 미친 영향의 성질을
근본적으로 바꾸어놓았다.
법학의 영원하고 위대한 등불이 되는
또 다른 일군의 법학자들이 후대에 등장하지만,
울피아누스, 파울루스, \wi{가이우스}, 파피니아누스는 해답의 저자들이 아니었다.
그들의 저술은 법의 특정한 분야, 특히 법무관의 고시에 대해 쓴
본격적인 전문법학서적이었다.

\para{로마의 제정법}
로마의 형평법 및 이것을 로마법에 만들어넣은 법무관 고시에 대해서는
다음 장에서 살펴볼 것이다.
제정법에 대해서는, 공화정 시기에는 수가 많지 않았으나
제정기에는 양산되었다는 점만 말해두고자 한다.
국가의 청년기나 유년기에는 사법\hanja{私法}의 일반적 개혁에
입법기관이 동원되는 경우가 드물다.
민중의 요구사항은 법을 변화시키는
것---이것은 실제 가치보다 높게 평가받는 경향이 있다---이 아니라
재판이 깨끗하고 완전하고 수월하게 진행되는 데 있었다.
입법기관에 대한 호소는 대체로 어떤 큰 권력남용을 제거해달라든가,
해결하기 어려운 신분 간의 혹은 권문세족 간의 다툼에 대해
결정을 내려달라는 정도에 불과했다.
로마인들은 대규모의 법률 제정과
큰 내란 뒤의 사회 안정 사이에
어떤 연관성이 있다고 생각했던 듯하다.
술라는 코르넬리우스 법들로써 공화국의 재건을 알렸다.
율리우스 카이사르는 방대한 양의 제정법을 추가하려는 계획을 가지고 있었다.
아우구스투스는 율리우스 법\latin{Leges Juliae}이라 불리는
매우 중요한 일군의 법률을 통과시켰다.
후대의 황제들 가운데 가장 적극적으로 \wi{칙법}\latin{constitution}을 공포한 이는,
콘스탄티누스처럼, 세상을 재조정하는 데 관심을 가졌던 황제들이었다.
로마에서 제정법의 진정한 시대는 제정기에 비로소 시작된다.
황제들의 법제정은,
처음에는 민중의 지지에 의해 제정되는 척 치장했으나
나중에는 황제의 대권에서 유래하는 것이라고 공공연히 인정되는데,
아우구스투스의 권력이 공고해진 이후 유스티니아누스 법전의 공표에 이르기까지
점점 더 그 양이 증가하였다.
이미 제2대 황제 치세 때에 오늘날 우리 모두에게 친숙한 법상태 및
법집행 양태와 상당히 비슷해졌다고 할 수 있다.
제정법이 등장하고 한정된 인원의 법해설자단\hanja{團}이 등장했다.
얼마 후에는 상설 상소 법원과 공인된 주해을 모은 주해집이 여기에 추가될 것이다.
그리하여 우리는 오늘날의 관념에 가까이 다가가게 된다.


\chapter{자연법과 형평법}

내재적인 탁월함을 가진
일군의 법원리가
낡은 법을 대체한다는 이론은
로마에서도 영국에서도 아주 일찍부터 통용되었다.
어떤 법체계에서도 발견되는
이러한 원리들을 앞 장에서 우리는 형평법\latin{equity}이라고 불렀다.
곧 살펴보겠지만, 이 용어는
로마 법학자들이 이러한 법변화 작인\hanja{作因}을 지칭하는
여러 명칭 가운데 하나, 오직 하나에 불과했다.
영국에서는 형평법법원\latin{Court of Chancery}의 법이
형평법이라는 이름으로 불리고 있거니와,
이것은 별도의 논저를 통해서만 제대로 논의될 수 있을 것이다.
그것의 구조는 대단히 복잡하고, 여러 다양한 원천에 기원을 두고 있다.
초기 챈슬러들은 성직자들이었기에 그들은 교회법으로부터
형평법의 바탕이 되는 법원리들을 이끌어냈다.
후대의 챈슬러들은
세속 사건에 적용할 수 있는 법규칙이 교회법보다 더 풍부한 로마법을
자주 원용했다.
그들의 판결문 중에는, 비록 출처는 밝히고 있지 않으나
로마법대전에서 따온 텍스트 전체가 토씨 하나 바뀌지 않고
들어가 있는 경우가 적지 않다.
더 최근에는, 특히 18세기 후반에는,
네덜란드 공법학자들의 법학 및 윤리학의 혼합체계가 영국 법률가들에 의해
널리 연구되었거니와,
이들 연구는
탈보 경\latin{Lord Talbot}에서 엘던 경\latin{Lord Eldon}에 이르는
챈슬러들의 형평법법원 판결에 큰 영향을 끼쳤다.
이렇게 다양한 기원의 요소들로 구성된 형평법 체계는,
보통법의 유추적용과 정합성을 가져야 한다는 요청으로 인해
그 성장이 크게 제한되었으나,
상대적으로 새로운 법원리들을 기술하는 일에 언제나 응답해왔다.
그 법원리들은 내재적인 윤리적 탁월함에 있어 영국의 옛 법을 능가한다고
주장되었다.

\para{로마의 형평법}
로마의 형평법은 구조가 훨씬 단순했고, 발달과정도 보다 쉽게 추적할 수 있다.
그것의 성질과 역사는 주의깊게 살펴볼 가치가 있다.
그것은 인간의 사고에 심대한 영향을 끼친 개념들을 창조하였고,
인간의 사고를 통해 인류의 운명에도 심대한 영향을 주었다.

로마인들은 그들의 법체계가 두 부분으로 구성된다고 보았다.
유스티니아누스 황제의 명에 의해 편찬된 법학제요는 이렇게 말한다.
``법과 관습에 의해 규율되는 모든 민족들은, 부분적으로는 그들 자신의
고유한 법에 의해, 부분적으로는 모든 인류에 공통되는 법에 의해,
통치된다. 당해 인민이 제정한 법은 그 민족의 시민법\latin{civil law}이라
불리고, 자연이성\latin{natural reason}이 모든 인류에게 지시한 법은
모든 민족이 사용하기 때문에
만민법\hanjalatin{萬民法}{Law of Nations}이라 불린다.''
여기서 ``자연이성이 모든 인류에게 지시한 법''은 법무관의 고시가
로마법에 엮어넣은 요소를 의미했다.
다른 곳에서는 이것을 단순히 자연법\latin{ius naturale}이라고 불렀는데,
자연법은 자연이성뿐 아니라 자연적 형평\latin{naturalis aequitas}에 의해서도
명령된다고 여겨졌다.
나는 여기서 만민법, 자연법, 형평법이라는 유명한 표현들의 기원을 탐구할 것이고,
또 이들이 지시하는 관념들의 상호 관련성을 탐구할 것이다.

\para{만민법}
로마의 역사를 조금만 살펴보아도,
여러 다른 이름으로 불리우며
로마의 영토 안에
살고 있는 외인\hanja{外人}들의 존재에 의해 공화국의 운명이
좌우되었음을 알고 놀라게 된다.
이러한 이주의 원인은 후대에 분명히 드러나게 되거니와,
왜 모든 민족의 사람들이 세상의 주인인 도시로 몰려드는지를
짐작하기란 그다지 어렵지 않다.
그러나 외인들과 준\hanja{準}시민\latin{denizen}들의 대규모 존재는
로마의 초기 역사에서도 그 기록이 발견된다.\footnote{여기서 `준시민'은
`라틴인'을 뜻하는 듯하다.}
말할 것도 없이,
다수의 약탈적 부족들로 구성된 고대 이탈리아의 사회적 불안정성은
사람들로 하여금, 공동체와 그 구성원들을 외적으로부터 보호할 수 있는
강력한 힘을 가진 공동체의 영토에 몰려가 살도록 하는 유인을 제공했다.
그 보호가 과중한 세금, 선거권 박탈, 사회적 신분 저하의 대가로
주어지는 것이라 할지라도 말이다.
하지만 이러한 설명은 불완전하며,
활발한 상거래 관계를 고려에 넣어야만 완전해질 수 있을 것이다.
이러한 상거래 관계는 공화국의 군사적 전승\hanja{傳承}에는
별로 반영되어 있지 않지만,
분명 로마는 카르타고와, 그리고 이탈리아 안에서,
선사시대부터 상거래 관계를 유지해온 것으로 보인다.
그 원인이야 무엇이든 간에,
국가 내에 외인들의 존재는
로마의 전체 역사 과정을 결정했으며,
그 모든 역사단계는 완고한 국수주의와 이방인 인구 간의 갈등의 이야기를
크게 벗어나지 않는다.
이와 같은 것이 현대에는 발견되지 않거니와,
우선 현대 유럽 국가들은 다수 국민들이 너무 많다고 여길 정도의
외국이민을 거의 혹은 전혀 받아들이지 않아왔기 때문이며,
또한
국왕이나 주권기구에 대한 충성으로 결합되는 현대국가들은
상당한 규모의 이민자 집단도
신속하게 흡수하기 때문이다.
고대 세계는 이러한 신속한 흡수를 알지 못했다.
고대사회에서 국가의 최초 시민들은 언제나 스스로를 혈연의 친족관계에 의해
결합되어 있다고 생각했고,
특권의 평등을 주장하는 것은 그들의 생래적 권리를 찬탈하려는 것이라
여기며 분개했다.
로마 공화정 초기에는
공법 영역은 물론이고 사법 영역에서도
외인들의 철저한 배제가 만연했다.
외인이나 준\hanja{準}시민은
국가 영역에 해당하는 어떠한 제도에도 참여할 수 없었다.
그들은 로마시민법\latin{Quiritarian Law}의 혜택도 누릴 수 없었다.
그들은 초창기 로마인들의 물권이전방식이자 계약방식이었던
구속행위\hanjalatin{拘束行爲}{nexum}의 당사자가 될 수 없었다.
그들은 문명의 유년기로 그 기원이 거슬러올라가는 소송방식인
신성도금소송\hanjalatin{神聖賭金訴訟}{sacramental action}도 제기할 수 없었다.
그럼에도 불구하고 로마의 이익도 로마의 안전도 그들이 법적 보호를 박탈당하는
상태를 허용하지 않았다.
어떤 고대 공동체도 약간의 평화교란으로도 전복될 수 있는 위험을 안고 있었다.
그리하여 단순한 자기보존의 본능에서 로마인들은
외인들의 권리와 의무를 조정하는 방법을 고안해냈거니와,
그렇지 않았다면---그리고 이것은 고대 세계에서는
진짜로 중대한 위험요인이었는데---외인들의 무장봉기가 일어났을 것이기 때문이다.
더욱이 로마 역사의 어느 시기에도 외인들의 상거래가 완전히 무시된 적은
한 번도 없었다.
따라서,
당사자 모두가 외인인 분쟁이나 시민과 외인 간의 분쟁에 대해
재판권을 처음 인정한 것은
반쯤은 치안을 위한 조치였을 것이고, 반쯤은 상거래의 지속을 위해서였을 것이다.
이러한 재판권의 인정은, 재판의 대상이 된 문제들을 해결할 어떤 법원리들을
발견해야할 필요성을 즉시 불러왔다. 그리고
로마 법률가들이 이들 대상에 적용한 법원리들은 그 시대의
두드러진 성격을 반영한 것이었다.
전술했듯이 그들은 이들 새로운 사건에 로마 시민법을 적용하기를 거부했다.
그들이 거부한 이유는 분명, 외인인 당사자의 출신 모국의 법을 적용하는 것은
일종의 체면손상이라고 여겼기 때문일 것이다.
그들이 채택한 방법은 로마를 비롯하여
그 이주민들이 태어난 다른 이탈리아 공동체들에
공통되는 법규칙을 찾아내 적용하는 것이었다.
다시 말해서, 그들은
모든 민족들에 공통되는 법, 즉 만민법\latin{ius gentium}의 원시적이고
문자적인 의미에 합치하는 법체계를 만들어냈다.
실로 만법법은 옛 이탈리아 부족들의 관습 가운데 공통된 요소의 총합이었다.
이들 부족이 로마인들이 관찰할 수 있었던 \hemph{모든 민족들}이었고,
로마의 영역에 지속적으로 이주민 무리를 보낸 민족들이었던 것이다.
어떤 특정 관행이 개별 민족들의 대다수에서 공통적으로 발견되면,
모든 민족들에 공통되는 법, 즉 만민법으로 선언되었다.
그리하여,
비록 물건의 양도는 로마 인근의 여러 다른 국가들에서 각기 다른 형식으로
수행되었으나, 그 실제적 이전인 인도\hanjalatin{引渡}{tradition}, 즉
양도할 목적물을 교부하는 것은 그들 모두에서 의례행위의 일부를 구성했다.
예컨대 인도는 로마 특유의 양도방식인
악취행위\hanjalatin{握取行爲}{mancipation}의 일부분을, 비록
부차적인 부분에 불과했지만, 구성했다.
따라서,
법학자들이 관찰할 수 있었던 양도행위 방식들의 유일한
공통요소였을 인도는
만민법, 즉 모든 민족들에 공통되는 법의 규칙으로 선언되었다.
다른 수많은 관찰들도 마찬가지 방법으로 심사대상이 되었다.
공통의 대상을 가진 관찰들 모두에서
어떤 공통의 성질이 발견되면,
이러한 성질은 만민법에 속하는 것으로 분류되었던 것이다.
따라서 만민법은,
여러 이탈리아 부족들의 지배적 제도들에 공통적이라고
관찰로써 확인된,
그러한 법규칙과 법원리의 총체였다.

만민법의 기원에 관한 이러한 서술은
로마 법률가들이 만민법을 특별히 존중했을 것이라는 오해에 대한
좋은 방패막이 될 것이다.
만민법은 부분적으로는 일체의 외국법에 대한 경멸의 결과였으며,
부분적으로는 그들 고유의 시민법의 혜택을 외인들에게 주기를 꺼려하는
마음의 결과였다.
물론, 로마 법학자들이 수행했던 역할을 오늘날의 우리가 수행한다면,
우리는 만민법에 대해 사뭇 다르게 접근했을 것이다.
우리라면 그렇게 다양한 관행들을 관통하는 배경적 요소로 판별된 것에 대해
어떤 탁월성이나 우선성을 부여할 것이다.
우리라면 그렇게 보편적인 법규칙과 법원리에 어떤 존중심을 가질 것이다.
우리라면 그 공통의 요소를 당해 거래의 본질이라고 말할 것이다.
그리고 공동체마다 서로 다른, 나머지 의례적 장치들은
우연적이고 부수적인 것으로 폄하할 것이다.
혹은, 우리가 비교하고 있는 민족들이 한때 어떤 위대한 공통의 제도를
따랐고 만민법은 그것의 재현\hanja{再現}이라고 추론할 것이다.
그리고 개별 국가들의 복잡다기한 관행들은 한때 원시국가를 규율했던
보다 단순한 법제의 타락이고 퇴폐일 뿐이라고 추론할 것이다.
하지만 근대적 관념이 이끌어낸 이러한 결과들은
초기 로마인들이 본능적으로 감지하고 있던 것들과
거의 항상 정반대이다.
우리가 존중하고 칭송하는 것을 그들을 싫어하고 질시하고 두려워한다.
그들의 법 중에 그들이 애정했던 부분은
오늘날의 학자라면 모두 우연적이고 일시적인 것으로 무시할 것들뿐이다.
악취행위의 장엄한 몸짓, 언어계약\latin{verbal contract}의 정연한 질문과 답변,
변론과 소송절차의 한없는 형식주의 등등.
만민법은 단지 정치적 필요 때문에 어쩔 수 없이 용인한 법체계에 불과했다.
그들은 외인들을 사랑하지 않았듯이 만민법도 사랑하지 않았다.
만민법은 외인들의 법제도에서 추출한 것이고 외인들의 이익을 위한 것에 불과했던
것이다.
만민법이 그들의 존중을 받기 위해서는 근본적인 혁명이 필요했다.
그 일이 실제 발생했을 때 그것은 너무나 근본적이었으니,
만민법에 대한 현대적 평가가 방금 언급한 그들의 것과 다른 진정한 이유는
현대의 법학과 현대의 철학이
이 주제에 관한 후대 법학자들의 성숙한 관념을
물려받았기 때문이다.
시민법에 붙은 비천한 부속물에서
만민법은 일약 모든 법이 따라야할 위대한, 그러나 아직은 발달 중에 있는,
전범\hanja{典範}으로 간주되는 시대가 도래했다.
그 결정적 전기는
로마인들이
모든 민족들에 공통인 법의 실무적 집행에
그리스의 자연법이론을
적용하기 시작하면서 도래했다.

\para{자연법}
자연법\latin{ius naturale}은 만민법을 특정한 이론의 관점에서 바라본 것에
지나지 않는다.
법률가의 특징인 분류 성향에 따라
법학자 울피아누스가
이 둘을 구분하려는 애처로운 시도를 했지만,\footnote{%
  만민법과 자연법이 다른 것은 쉽게 알 수 있거니와,
  자연법은 모든 동물에 공통적인 법이지만 만민법은 인간들 사이에서만
  공통적인 법이다. Dig.\,1.1.1.4.}
훨씬 높이 평가되는 가이우스의 말에 따르면, 그리고
앞서 인용한 법학제요의 문구에 따르면,
이들 표현은 의심의 여지 없이 사실상 서로 바꾸어 쓸 수 있는
것들이었다.\footnote{다만 노예제도에 관한 한 자연법과 만민법은
서로 분기했다. 노예제도는 고대 모든 민족들에서 발견할 수 있었으나,
자연법상으로는 모든 인간이 자유롭게 태어났다고 여겨졌다. Inst.\,1.2.2.}
그들 간의 차이는 순전히 역사적인 것이었으며
본질적인 구별은 성립될 수 없었다.
만민법\latin{ius gentium}, 즉 모든 민족에 공통인 법과
\hemph{국제법}\latin{international law} 간의 혼동은 전적으로 근대적인 것임은
부연할 필요조차 없다.
국제법의 고전적 표현은
선전강화법\hanjalatin{宣戰講和法}{jus feciale},
즉 협상과 외교에 관한 법이었다.
하지만 만민법의 의미에 관한 모호한 인상은
독립 국가들 간의 관계가 자연법의 지배를 받는다는 오늘날의 이론을
낳는 데 크게 기여했을 것임에 틀림없다.

여기서
자연과 자연법에 관한 그리스인들의 관념을 살펴볼 필요가 생긴다.
퓌시스\latin{physis}는 라틴어로 나투라\latin{natura}, 우리말로는
자연\latin{nature}이라 번역되는데,
확실히 원래는 물질적인 우주를 뜻하는 말이었다.
그러나 그것은 현대적 언어로 표현하기 힘든---고대와 현대의 지적인 거리가
그만큼 멀다---어떤 관점에서 사고된
물질적 우주였다.
자연은 어떤 근원적인 요소 또는 근원적인 법칙의 결과로서의
물리적 세계를 의미했다.
초기 그리스 철학자들은
창조의 구성을 어떤 단일한 원리의 발현으로 설명하곤 했거니와,
그 원리를 그들은 운동, 힘, 불, 습기, 생성 등으로 다양하게 주장했다.
가장 단순한고 가장 고대적인 의미의 자연은 다름 아니라
이렇게 어떤 원리의 발현으로
간주된 물리적 우주였다.
후대에 이르러 그리스인들은, 위대한 그리스 지식인들이 벗어났던 길을 되돌려,
자연 개념의 \hemph{물리적} 세계에 \hemph{정신적} 세계를 추가했다.
자연이라는 말이 확장되어 가시적인 피조물뿐만 아니라 인간의 사상, 관찰, 소망까지
포괄하게 된 것이다.
그럼에도 불구하고 여전히, \hemph{자연}이라는 단어로 그들이 이해한 것은
그저 인간사회의 정신적 현상만이 아니라,
이러한 현상이 어떤 일반적이고 단순한 법칙으로 환원된다는 것까지 포함했다.

\para{스토아 철학}
초기 그리스 이론가들은 물리적 우주가 우연의 장난으로 단순한 원시적 형태에서
오늘날의 이질적인 복잡한 상태로 변화했다고 생각했다.
마찬가지로 이제 그들의 지적인 후손들도 만약 불행한 사고가 없었다면
인류는 보다 단순한 행위규칙과 보다 고난이 덜한 삶에 만족하며
살았을 것이라고 상상했다.
\hemph{자연}에 따라 사는 것이 인간이 창조된 목적이자
탁월한 인간이 달성해야할 목적으로 간주되기 시작했다.
\hemph{자연}에 따라 사는 것은 난잡한 습관과 저속한 것에의 탐닉을 넘어서는
고차원적인 행위법칙으로 고양되었고, 자제와 극기만이
이것을 따를 수 있게 해 준다고 생각되었다.
이 명제---자연에 따라 사는 것---가 저 스토아 철학의 핵심 신조였던 것은
너무도 유명하다.
그리스의 정복과 더불어 이 철학은 즉시 로마 사회로 흘러들어갔다.
이 철학에는 로마의 엘리트 계급을 사로잡는 매력이 있었다. 그들은, 적어도 이론적으로는,
고대 이탈리아 민족의 단순한 습관을 고수했고
외국풍의 혁신에 굴복하기를 경멸했던 것이다.
이런 사람들은 자연에 따른 삶이라는 스토아의 명제에 즉각 매료되었다.
세상을 약탈하고 가장 사치스런 민족의 대명사가 된 저 제국의 수도에 만연했던
무절제한 방종에 비추어볼 때, 참으로 감사한 매료요, 생각건대 참으로 고귀한 매료였다.
새로운 그리스 철학의 사도 무리의 맨 앞 열은,
역사적으로 증명할 수는 없을지라도, 로마 법률가들이 차지하고 있었음이 거의 확실하다.
여러 증거로 추정컨대,
로마 공화국에는 사실상 두 종류의 전문직만 있었거니와,
군인들은 일반적으로 변혁을 추진하는 당파에 속했고,
법률가들은 일반적으로 변혁에 저항하는 당파의 선두에 서 있었다.

\para{법무관의 고시}
법률가들과 스토아 철학의 결합은 수 세기에 걸쳐 지속되었다.
저명한 법학자들의 몇몇 초기 이름들은 스토아주의와 관련되어 있다.
나중에는 안토니누스 황조\latin{Antonine Caesars} 시대로 널리 합의되어 있는
로마법학의 황금기가 도래하거니와,
이 시기 황제들은 저 철학을 생활의 규칙으로 삼았던 유명한 사도들이었다.
특정 전문직 구성원들 사이에 이 신조가 장기간 확산됨에 따라
그들이 실무에 활용하고 영향을 끼쳤던 학문도 영향을 받지 않을 수 없었다.
저 스토아적 신조를 열쇠말로 사용하지 않으면
로마 법학자들이 남긴 유산에 속하는 몇몇 견해들은 거의 이해가 불가능해진다.
그러나 그렇다고 해서,
스토아주의가 로마법에 끼친 영향을,
스토아 교리에서 기원했다고 생각되는 법규칙의 숫자를 세어 측정하는 것은,
매우 흔하지만 심각한 오류에 해당한다.
스토아주의의 강점은,
때로 거부감을 불러일으키는 어처구니없는 행위준칙들에 있는 것이 아니라,
정념에의 저항을 가르치는 모호하지만 위대한 원리에 들어있다고 널리 인정되어왔다.
마찬가지로, 스토아주의로 대표되는 그리스 철학이 법학에 끼친 영향도
그것이 로마법에 기여한 여러 특정 견해들의 숫자가 아니라
그것이 가져다준 단일한 근본적인 전제에서 찾아야 한다.
자연이라는 단어가 로마인들이 일상적으로 사용하는 말이 되면서,
로마 법률가들 사이에서는
옛 만민법이 사실은 잃어버린 자연의 법전이라는 믿음이
점차 확산되어갔다.
또한 만민법 원리에 기초하여 고시법\hanja{告示法}을 형성함으로써
쇠퇴하기 시작한 법을 법무관들이 점차 다시 되살리고 있다는 믿음도 확산되어갔다.
이러한 믿음으로부터,
고시를 통해 가능한 한 많이 시민법을 대체하는 것이,
원시상태의 인간에게 자연이 가르쳐준 제도들을 가능한 한 많이 되살리는 것이,
법무관의 의무라는 생각이 즉각 추론되어 나온다.
물론 이러한 방법으로 법을 개선하는 데는 많은 장애가 따른다.
법전문직 내에서도 극복해야할 편견들이 있었고,
로마인들의 습관도 꽤나 끈질겨서 단순한 철학 이론에 당장 굴복하지는 않았다.
법무관들이 고시를 가지고 몇몇 법기술적 변칙들과 싸운 간접적인 방법들을 통해
우리는 그들이 신중하게 준수해야만 했던 것들을 엿볼 수 있다.
또한 유스티니아누스 시대에 이르기까지도 고법\hanja{古法}의 일부는
이러한 영향력에 완고하게 저항했던 것이다.
하지만 법 개선에 있어 로마인들의 진보는
자연법이론의 자극이 주어지자마자 신속하게 전개되었다.
단순화와 일반화의 관념이 자연의 개념에 밀접히 연관되어 있었다.
그리하여 단순성, 조화성, 명료성이 좋은 법체계의 특징으로 간주되었고,
복잡한 언어, 복잡한 의례, 무의미한 장애물들은
모두 사라져갔다.
로마법을 기존의 모습으로 되살리는 데는
유스티니아누스의 강력한 의지와 흔치않은 기회가 필요했지만,
로마법의 기초 계획도는 그의 제국 개혁이 착수되기 오래 전에
이미 수립되어 있었던 것이다.

\para{형평법의 기원}
옛 만민법과 자연법이 만나는 접점은 무엇인가?
나는 본래적 의미의 형평\latin{aequitas}을 통해
이 둘이 만나고 결합된다고 생각한다.
여기서 우리는 형평법\latin{equity}이라는 유명한 용어가
법학에 처음 등장함을 보게 된다.
이처럼 그 기원이 멀고 역사가 오래된 표현을 탐구할 때에는,
가능한 한,
일견 어렴풋이 개념의 그림자만 보여주는
단순한 은유나 상징을 파고드는 것이
언제나 가장 안전할 것이다.
흔히들 라틴어의 `형평'이 그리스어 `이소테스'\latin{isotes}와 동의어라고
하는데, 후자는 평등한 또는 비례적인 분배의 원리를 뜻한다.
숫자나 물리적 양을 평등하게 나누는 것은 분명 우리의 정의\hanja{正義} 관념과
밀접히 연관되어 있다.
인간의 정신에서 이처럼 강고하게 결합되어 있는 관념 연관을 찾기란 쉽지 않으며
가장 깊이있는 사상가들의 고된 작업을 통해서도 이것을 분리하기가 쉽지 않다.
하지만 이들의 연관을 역사적으로 추적해보면,
아주 초기의 사상에서는 이것이 나타나지 않거니와,
오히려 상대적으로 후대의 철학의 산물인 것으로 보인다.
또한 주목할 점은, 그리스 민주주의가 자랑하는
법의 ``평등''\latin{equality}---칼리스트라토스\latin{Callistratus}의
아름다운 권주가에 따르면
하르모디오스\latin{Harmodius}와 아리스토게이톤\latin{Aristogiton}이
아테네인들에게 주었다고 전해지는 그 평등---이
로마인들의 ``형평''\latin{equity}과 거의 공통점이 없다는 것이다.
전자는 시민들 사이에, 그 시민들의 계급이 비록 낮다고 할지라도,
시민법의 집행이 평등해야 한다는 의미이다.
후자는 시민이 아닌 자가 포함된 계급에게는 시민법이 아닌 법이
적용될 수 있다는 의미이다.
전자는 폭군을 배제한다는 뜻이고, 후자는 외인을, 경우에 따라서는 노예를,
포함한다는 뜻이다.
전반적으로 보아, 여기서 방향을 약간 틀어 로마인들의 ``형평''이란 단어의
기원을 살펴볼 필요가 있겠다.
라틴어 ``아이쿠스''\latin{aequus}는 그리스어 ``이소스''\latin{isos}보다
\hemph{평평하게 하기}\latin{levelling}라는 의미를 더 명백하게 가진다.\footnote{%
  `aequus'는 `평평한' `고른'이란 뜻으로, 라틴어 `aequitas'(형평)의 어원이 된다.}
이러한 평평하게 하는 경향은 정확히 만민법의 성격이었다.
초기 로마인들에게 만민법은 상당히 충격적이었을 것이다.
순수한 로마시민법은 사람과 물건에 대해 여러 가지 자의적인 분류를 두고 있었지만,
여러 민족들의 관습에서 일반화된 만민법은 로마시민법상의 구분을
알지 못했다.
예컨대 옛 로마법은 ``종족''\hanjalatin{宗族}{agnatic}인 친족과
``혈족''\hanjalatin{血族}{cognatic}인 친족을 근본적으로 구분했다.
전자는 공통의 가부장권\hanja{家父長權}에 복속하는 가족관계를
지칭하고,\footnote{정확히 말하면 `종족'은
나와 상대방(이들은 여자라도 상관없다)을 이어주는 가계도상의 연결점들이
모두 남자인 경우의 혈족관계를 의미한다. 남계혈족(男系血族)과 같은 뜻이다.}
후자는 \paren{오늘날의 관념에 일치하는 것으로} 단순히 공동의 혈통으로
결합된 가족관계를 지칭한다.
이러한 구분은 ``모든 민족들에 공통인 법''에서는 존재하지 않았다.
또한 ``악취물''\hanjalatin{握取物}{things \textit{mancipi}}과
``비악취물''\hanjalatin{非握取物}{things \textit{nec mancipi}}이라는
물건 분류의 옛 방식도 존재하지 않았다.
따라서 구분과 경계의 부재는 형평\latin{aequitas}으로 묘사되는 만민법의
특징이라 할 수 있다.
나는 이 형평이라는 단어가 처음에는 단지
이러한 끊임없는 \hemph{평평하게 하기}, 즉
울퉁불퉁함의 제거를 뜻했다고 생각한다.
이것은 외인 당사자가 개재된 사건에 법무관법이 적용될 때면 지속적으로 일어났다.
처음에는 이 표현에 어떤 윤리적 의미도 들어있지 않았을 것이다.
또한 초기 로마인들은 이러한 과정을 무척 싫어했을 것이라고
추정하지 않을 이유도 전혀 없다.

\para{형평과 평등}
한편, 형평이란 말로써 로마인들이 이해한 만민법의 특징은
최초로 생생하게 감지된 가상의 자연상태의 성격과 완전히 일치했다.
자연은 처음에는 물리적 세계의, 나중에는 정신적 세계의, 균형잡힌 질서였고,
질서에 대한 최초의 관념은 분명 직선, 평면, 측정된 거리 같은 것과
관련되어 있었다.
인간 정신의 눈이 가상의 자연상태의 윤곽을 그려내려 할 때든,
혹은 ``모든 민족들에 공통인 법''의 실제 집행을 바라보고 받아들일 때든,
그 정신의 눈 앞에는 이러한 종류의 그림 혹은 상징이 무의식적으로 그려졌을 것이다.
그리고 원시적 사고에 대한 우리의 모든 지식으로 판단하건대,
이러한 관념적 유사성은 이들 두 개념 간의 동일성에 대한 믿음을
불러일으켰을 것이다.
그런데 당시,
만민법은 로마에서 예전에 거의 혹은 전혀 권위를 인정받지 못하던 것이었으나,
자연법이론은 철학적 권위의 위신을 두른 채 들어왔을 뿐 아니라,
그것도 역사가 더 깊고 더 축복받은 민족의 것이라는 매력까지 품고 있었다.
이러한 관점의 차이가
옛 법원리의 작동과 새로운 이론의 결과를 동시에 기술하는 저 용어의 위엄에
어떤 영향을 주었을까는 쉽게 이해할 수 있다.
어떤 과정을 ``평평하게 하기''라고 묘사하는 것과
``변칙적인 것의 교정''이라고 부르는 것 사이에는
현대인들이 듣기에도 큰 차이가 있으나, 그러나
그 은유는 완전히 똑같은 것이다.
또한 형평이 저 그리스 이론을 암시하는 것으로 이해되자,
이제 `이소테스'라는 그리스적 관념이 형평 개념을 둘러싸기 시작했음에 틀림없다.
키케로의 언어가 이런 일이 실제 일어났음을 보여주고 있거니와,
이는 형평 개념의 변용의 첫 번째 단계였던 것이다.
그리고 그때 이후 등장한 거의 모든 윤리체계는 이 형평 개념을 전승해왔다.

\para{영구고시록}
처음에는 모든 민족들에 공통인 법과 관련되고 나중에는 자연법과 관련되는
법원리와 법개념들이 차츰 로마법에 흡수되어간 형식적 도구에 대해
몇 마디 말해둘 것이 있다.
타르퀴니우스 왕조 축출 사건으로 대변되는 로마 역사상 최초의 위기 시에
많은 고대국가의 초기 연대기에 나타나는 것과 유사한 변화가 일어났지만,
이는 오늘날 우리가 혁명이라고 부르는 정치적 변화와는 거의 공통점이 없는
것이었다.
왕정이 계속 유지되었다고 하는 것이 보다 정확한 기술일 것이다.
지금까지 한 사람의 수중에 집중되었던 권력이
다수의 선출직 관리들 사이에 분할되었으나,
왕이라는 명칭은 나중에
제사왕\hanjalatin{祭祀王}{rex sacrorum; rex sacrificulus}이라고
불리게 되는 사람에게 주어져 그대로 유지되었다.
변화의 일부로서 최고 사법관직의 기존 임무는 당시 국가의 최고 관리였던
법무관\latin{praetor}에게 부여되었다. 또한
이러한 임무와 더불어
법과 입법에 관한 불명확한 대권\latin{大權}도 그에게 이전되었거니와,
이는 고대의 통치자들이라면 누구나 가졌던 것이지만
한때 그들이 누렸던 가부장적이고 영웅적인 권위와의 희미한 연관성은
사라지고 없었다.
로마의 상황으로 인해 이렇게 이전된 기능 중에 보다 불명확한 부분이
더 큰 중요성을 가졌는데,
법기술상 본래의 로마인으로 분류할 수 없지만
그러나 로마의 법역 안에 상주하고 있는 사람들을 다루는 어려운 문제를
안겨준 재판들이
공화국의 수립 이후
지속적으로 제기되기 시작했기 때문이다.
이러한 사람들 간의 쟁송 및 이러한 사람들과 생래적 시민들 간의 쟁송은
법무관이 이러한 재판업무를 떠맡지 않았다면
로마법상 구제수단이 전혀 주어질 수 없는 것들이었다.
또한 곧이어 상거래가 확산되면서 로마 시민들과 자칭 외인들 사이에 발생한
보다 중대한 분쟁들에 대해서도 법무관이 대처하지 않으면 안 되었다.
제1차 포에니 전쟁을 전후하여 로마 법원에 이러한 소송이 대폭 증가하자,
후에 외인법무관\latin{praetor peregrinus}이라 불리게 되는,
이런 종류의 사건만 전담하는 특별한 법무관이 임명되기에 이른다.
한편, 압제의 부활에 대한 로마 인민들의 두려움으로 인해,
업무영역이 확장되는 경향을 가진 모든 정무관은
매년 임기 초에
자신이 맡은 업무를 앞으로 어떻게 수행할지를 선언하는
고시\hanjalatin{告示}{edict}를 공표할 의무가 부과되었다.
다른 정무관들과 함께 법무관도 이 규칙의 적용대상이었다.
그런데 해마다 따로 다수의 법원칙들을 고안해내는 것은 사실상 불가능하므로
법무관은 전임자의 고시를 거의 답습하여 재공표하고,
다만 그때그때의 상황에 따라 혹은 자신의 법적 견해에 따라
약간의 추가와 변경을 가하는 데 그쳤던 듯하다.
그리하여 장기간 매년 반복되는 법무관의 선포는
영구고시\latin{edictum perpetuum}라는 이름을 얻게 되었으니,
이는 \hemph{지속적인} 또는 \hemph{중단없는} 고시라는 뜻이다.
이것이 너무나 오랫동안 계속되자,
그리고 아마도 그 무질서해질 수밖에 없는 구조에 대한 염증 때문에,
하드리아누스 황제 재위기에 정무관직에 있었던
살비우스 율리아누스\latin{Salvius Julianus}의 임기 때에 이르러
더 이상의 확장이 중단되게 된다.
그리하여 이 법무관의 고시는 형평법의 총체였거니와,
아마도 새롭고 체계적인 질서를 갖추었을 것이다.
이후 로마법에서 이 영구고시록은 단순히
율리아누스 고시\latin{Edict of Julianus}로 흔히 인용되곤 했다.

고시의 특수한 메커니즘을 고찰하는 영국인의 머리에 떠오르는
첫 번째 의문은 이런 것이리라: 법무관의 이러한 포괄적 권한을 통제하는
제약\hanja{制約}은 무엇이었을까? 어떻게 그렇게나 불명확한 권한이
기존의 사회상황 및 법상태와 조화될 수 있었을까?
이에 대한 답변은 우리의 영국법이 운용되는 상황을 면밀히 관찰함으로써만
주어질 수 있을 것이다.
법무관은 그 자신이 법학자이거나, 아니면 법학자인 조언자들의 수중에 있는
사람임을 상기할 필요가 있다.
또한 로마 법률가라면 누구나 저 위대한 사법정무관직에 취임하거나 아니면
그 직을 통제할 날을 손꼽아 기다렸을 것이다.
그 사이 기간동안 그의 취향, 감정, 편견, 그리고 계몽의 정도는
불가피 그의 동료집단의 그것이었으며,
또한 후에 공직에 취임하거나 통제하게 될 때의 그의 자질은
그가 전문직으로서 실무와 연구에서 얻었던 것이었다.
영국의 챈슬러도 정확히 동일한 훈련을 거치며, 또한 동일한 종류의 자질을 가지고
챈슬러직을 수행한다.
그가 공직에 취임할 때는, 공직을 떠나기까지 어느 정도는
그가 법을 변경하리라는 것이 확실하다.
하지만 공직을 물러나고 그가 내린 판결들이 판례집에 수록되기
전에는, 그가 전임자에게서 물려받은 법원리를 얼마나 더 분명히 밝히고
또 새로운 것을 추가했는지 우리는 알 수 없다.
로마법에 대한 법무관의 영향도 단지 그 영향의 정도가 확인되는 시기에 있어서만
차이가 날 따름이었다.
전술했듯이 법무관의 임기는 1년에 불과했다.
또한 임기 동안 그가 내린 결정은, 물론 소송당사자들에게는 불가역적인 것이었으나,
장래에 대해 구속력을 갖지 않았다.
따라서 그가 계획하는 변화를 선포하는 순간은 당연히
법무관직에 취임하는 순간일 수밖에 없었다.
그리하여 임기 시작 시에 그는,
후에 영국의 챈슬러가 부지불식간에 그리고 때로는 몰래 행하는 것을
공개적이고 명시적으로 수행했다.
그의 외관상의 재량에 대한 통제는 영국 판사들에 대한 통제와
하등 다를 것이 없었다.
이론상으로는 양자의 권한에 거의 아무런 제한이 없는 것처럼 보이나,
실제적으로는 로마의 법무관도 영국의 챈슬러도
초기 훈련 과정에서 습득한 선이해에 의해, 그리고
전문직 그룹의 여론이라는 강력한 제약에 의해
엄격하게 한계지워진다.
이러한 제약의 엄격함은 직접 경험한 사람들만이 실감할 수 있는 것이다.
부연하건대, 움직임이 허락된 공간의 경계선, 넘어서는 안 되는 그 경계선은
영국만큼이나 로마에서도 분명히 그어져 있었다.
영국의 판사들은 고립된 사실관계에 대한 공표된 판결들의 유사성을 따라야 한다.
로마에서는, 법무관의 개입이 처음에는 국가의 안전이라는 단순한 고려에 의해
지배되었기에 아주 초기에는 제거하고자 하는 문제의 곤란함 정도에 비례하여
개입이 이루어졌을 것이다.
후에, 법학자의 해답에 의해 법원리에 대한 애호가 확산되자,
법무관은 그러한 근본원리들을 더 폭넓게
적용하기 위한 수단으로 그의 고시를 이용했을 것이 틀림없다.
이때 그와 나머지 실무 법학자들, 그의 동시대인들은
법의 저변에 놓여있는 그 원리들을 발견했다고 믿었다.
더 시간이 흐른 후에,
법무관은 이제 전적으로 그리스 철학이론의 영향력 하에서
행동했거니와, 이 이론은 특정한 진화의 방향으로 그를 이끄는 동시에
그 방향으로 가도록 그를 한계지웠다.

\para{그후 로마 형평법}
살비우스 율리아누스의 조치는 그 성격이 큰 논쟁의 대상이 되었다.
그 성격이 어떠하든 간에, 그것이 고시에 미친 효과는 사뭇 명백했다.
고시는 이제 해마다 확장되기를 그쳤고, 이후로
로마의 형평법은 하드리아누스 황제 치세와 알렉산데르 세베루스 황제 치세 사이에
활발한게 저술활동을 펼친 일련의 위대한 법학자들에 의해 발달하게 된다.
그들이 이룩한 경탄스런 체계의 일부가
유스티니아누스의 학설휘찬\latin{Pandects}에
남아있거니와, 이를 통해 우리는 그들의 작품이 로마법의 모든 영역에 관한
논저의 형태를,
그러나 주로 고시에 대한 주해서의 형태를, 띠고 있었음을 알 수 있다.
실로 이 시대의 어떤 법학자가 어떤 주제 하나를 다루었다 할지라도
언제나 그는 형평법의 해설자로 불릴 만하다.
고시에 담긴 법원리들은 이 시대가 끝나기 전에 로마법학의 모든 영역에
침투해 들어갔다.
로마의 형평법은, 비록 시민법과 완전히 동떨어진 경우에도,
언제나 동일한 법원에 의해 재판되었다는 점을 잊지 말아야 한다.
법무관은 형평법 수석판사인 동시에 보통법 수석판사이기도 했다.
그리하여 고시가 어떤 형평법규칙을 발달시키면,
법무관의 법원은 바로 옛 시민법규칙을 대체하여 혹은 그것과 병행하여
그 형평법규칙을 적용하기 시작했다. 이것은
입법기관의 명시적 법제정 없이 시민법이 직^^b7간접적으로 폐지되는 결과를 낳았다.
물론 이것은 시민법과 형평법의 완전한 통합에는 전혀 이르지 못하는 것이었다.
이 통합은 후에 유스티니아누스의 개혁에 의해 비로소 이루어진다.
두 영역의 법이 법기술상 분리되어 있다는 사실은
일말의 혼동와 일말의 불편함을 낳았다. 또한
시민법 법리 가운데 아주 완고한 것들은 고시의 선포자들도 그 해설자들도
감히 건드리지 못하는 것들이 있었다.
하지만 법학 분야 중에
형평법의 영향이 다소간 휩쓸고 지나가지 않은 구석은 하나도 없었다.
그것은 법학자들에게 일반화를 위한 자료를,
해석의 방법을, 근본원리들의 해명을 제공했다. 또한
입법자에 의한 개입이 거의 없는,
오히려 입법의 적용에 중대한 통제를 가하는,
다량의 요건규칙들도 제공했다.

법학자의 시대는 알렉산데르 세베루스 황제와 더불어 종말을 고한다.
하드리아누스로부터 이 황제에 이르기까지 법의 발달은,
오늘날 대부분의 대륙법계 국가들에서와 마찬가지로,
부분적으로는 공인된 주해에 의해,
부분적으로는 직접적인 입법에 의해 이루어졌다.
하지만 알렉산데르 세베루스의 치세에 로마 형평법의 성장력은 소진되었고,
법학자들의 잇따른 등장도 마감되었다.
로마법의 나머지 역사는 황제의 칙법의 역사이고,
종국에는 오늘날 로마법의 거창한 집적물로 남겨진 것을 편찬하려는
시도의 역사이다.
이런 종류의 실험 중에 최후의 그리고 가장 칭송받는 것으로
유스티니아누스 황제의 로마법대전\latin{Corpus Juris}이 우리에게 전해지고 있다.

\para{영국과 로마의 형평법}
영국과 로마의 형평법을 집요하게 비교하고 대비시키는 것이 지루하게 느껴질 수도
있겠다. 하지만 그들의 공통점 두 가지는 언급해둘 가치가 있다.
첫째는 이렇게 말할 수 있을 것이다:
그 둘은 모두, 모든 이러한 체계가 그러하듯이,
형평법이 처음 개입했을 때의 옛 보통법의 상태와 정확히 같은 상태에
이르는 경향이 있었다.
최초에 도입된 도덕적 원리들이 모든 정당한 결과들을 낳으며 역할을 다한 후,
그들에 기초한 체계가 굳어지고, 더 이상 확장이 안 되고,
보통법이라 부르는 아주 엄격한 규칙체계와 마찬가지로
도덕적 진보에 뒤처지기 시작하는
시기가 반드시 도래한다.
로마에서는 그 시기가 알렉산데르 세베루스 재위기에 도래했다.
그후, 전체 로마 세계가 정신적 혁명에 휩싸였지만, 로마의 형평법은
더 이상 확장되지 못했다.
영국의 법제사에서는 동일한 시기가 엘던 경\latin{Lord Eldon}이
챈슬러직에 있을 때 도달했다.
간접적인 입법에 의해 형평법을 확장시키는 대신, 그는
형평법을 설명하고 조화시키는 데만 평생을 바친 최초의 형평법 판사였다.
법제사의 교훈이 영국에서 좀 더 잘 이해되었더라다면,
엘던 경의 업적은
당대 법률가들 사이의 평판보다
한편으로는 덜 과장되었을 것이고,
다른 한편으로는 더 나은 평가를 받았을 것이다.
실무적 결과에 영향을 주는 또 다른 오해도 또한 불식되어야 한다.
영국의 형평법이 도덕 규칙들에 기초한 체계임을
영국 법률가라면 누구나 쉽게 이해한다.
하지만 이 규칙들이---현재가 아니라---수 세기 전 과거의 도덕임은 잊고 있다.
그동안 너무 많이 적용되어 능력이 거의 소진될 지경에 이르렀음은 잊고 있다.
그것들이 물론 오늘날의 도덕적 신조와 크게 다르지는 않다 할지라도
오늘날의 그것을 따라잡지 못하는 것일 수 있음은 잊고 있다.
이 주제에 관한 불완전한, 그러나 널리 받아들여지고 있는, 이론들이
서로 상반되는 종류의 오류를 생산해왔다.
형평법에 관한 논저의 저자들 다수는
현재 상태의 체계의 완전성에 매료되어
명시적^^b7묵시적으로 역설적인 주장을 펼치고 있거니와,
형평법의 창시자들이 처음 그 기초를 다졌을 때 이미 현재와 같은 고정된 형태를
만들어냈다는 주장이 그것이다.
또한 다른 이들은---법정 변론에서 자주 들리는 불평인데---형평법법원이
강제하는 도덕 규칙들이 오늘날의 윤리 기준에 미치지 못한다고 불평하고 있거니와,
그들은 영국 형평법의 창시자들이 옛 보통법에 대해
행하던 역할과 정확히 동일한 역할을
지금의 챈슬러들이
수행해주기를 형평법에 대해 요구하고 있는 것이다.
하지만 이것은 법의 발달이 진행되는 순서를 거꾸로 뒤집는 것이다.
형평법에는 자신만의 장소와 시간이 있다.
나는 다른 수단이 있음을, 에너지만 주어진다면 그 수단이 성공할 것임을,
앞서 지적한 바 있다.

영국과 로마의 형평법의 또 하나의 주목할만한 성격은
형법법이 보통법이나 시민법보다 우월하다는 주장이 처음
개진될 때, 이 주장이 근거했던 가정들이 모두 허위였다는 점이다.
개인이든 집단이든 인간에게 있어 도덕적 진보를 실제적 현실로
받아들이는 것만큼 싫은 것이 없다.
이 거부감이 개인에게서는 일관성이라는 의심스런 덕목을 과장되이 존중하는
모습으로 통상 나타난다.
전체 사회의 수준에서도 집단적 여론의 움직임은 너무나 명백해서 무시할 수 없고
대체로 너무나 뚜렷이 더 좋은 것을 향하기에 대놓고 비난할 수 없으나
그것을 주요한 현상으로 인정하기를 꺼리는 경향이 강하게 존재하거니와,
보통은 잃어버린 완전성의 회복---인류가 타락하기 이전 상태로의 점진적
회귀---으로 설명된다.
이렇게 도덕적 진보의 목표를 앞을 바라보는 데서가 아니라
뒤를 돌아보는 데서 찾는 경향은, 전술했듯이,
고대 로마법에 가장 심각하고 영속적인 영향을 주었다.
로마 법학자들은, 법무관에 의한 법 발달을 설명하기 위해서,
실정법에 의해 통치되는 국가들이 조직되기 이전에 존재한
인간의 자연상태---자연적 사회---의 이론을 그리스로부터 빌려왔다.
한편, 영국에서는, 당시 영국인들의 취향에 특별히 부합하는 관념으로
보통법에 대한 형평법의 우월성 주장을 설명했거니와,
국왕이 갖는 가부장적 권위의 당연한 결과로서 국왕에게는
사법\hanja{司法}을 감독할 일반적 권리가 있다는 것이 그것이었다.
동일한 견해가
``형평법은 국왕의 양심에서 유래한다''는
옛 법리에서
보다 고풍스런 형태로
등장했으니, 이는
실제로는 공동체의 도덕 기준에 진보가 일어난 것을
주권자의 내면적 도덕 감각의 상승으로 치환시키고 있는 것이다.
영국 헌정의 발달로 후에 이러한 이론은 부적합한 것이 되었지만,
실은 형평법법원의 재판권이 확고하게 자리잡음에 따라 공식적인 대체물을
고안할 가치가 없어졌던 것이다.
오늘날 형평법 교재들에서 발견되는 이론들은 참으로 다양하지만,
하나같이 유지될 수 없는 이론들뿐이다.
그 대부분은 자연법에 기초한 로마법 이론의 변용이거니와,
이는
자연적 정의와 시민적 정의의 구분으로써
형평법법원의 재판권에 대한 논의를 시작하는
저술가들에 의해
실제로 그대로
채용되고 있다.


\chapter{자연법의 근대사}

지금까지의 논의로부터, 로마법의 변화를 가져온 저 이론은 어떤 철학적 엄밀성을
주장한 것이 아니었음을 알 수 있을 것이다.
그것은 사실 일종의 ``혼합적 사고양식''이었거니와,
이러한 사고양식은 오늘날 최고의 정신을 제외한 모든 인간 정신의 유아기적 사고의
특징으로 인식되고 있으며 또한 현 시대의 정신에서도 어렵지 않게
발견할 수 있는 것이다.
자연법 이론은 과거와 현재를 혼동했다.
논리적으로는, 그것은 한때 자연법에 의해 통치되었던 자연상태를 상정한다.
하지만 로마의 법학자들은 그러한 자연상태의 존재를 분명하게 그리고
자신있게 말하지 않았다. 사실 황금시대를 상상하는 시적인 표현을
제외하면 고대인들은 그러한 상태에 대해 거의 언급하지 않았다.
실무적 목적에서는, 자연법은 현재에 속하는 어떤 것이고,
기존의 제도와 얽혀있는 어떤 것이며, 유능한 관찰자에 의해
기존 제도와 구분될 수 있는 어떤 것이다.
자연의 명령을 이와 함께 섞여있는 조잡한 요소들로부터 분리하는 기준은
단순성과 조화성의 감각이었다.
하지만 이들 더 세련된 요소가 애초 존중받은 것은
단순성과 조화성 때문이 아니라,
자연의 원초적 지배의 후예라는 데에 있었다.
이러한 혼동은 근대의 법학자들에 의해서도 성공적으로 해명되지 못했다.
실로
로마 법률가들이 받아야 할 비난보다 오히려
오늘날의 자연법사상이 인식의 불명료성을 훨씬 더 많이 노정하고 있으며
언어의 절망적인 모호성에 의해 더 많이 오염되어 있다.
이 주제에 관한 저자들 몇몇은, 자연법법전은 미래에 존재하는 것이고
모든 시민법들이 지향해야 할 목표라고 주장함으로써,
이러한 근본적인 난제를 피해가려고 시도하나,
이는 옛 이론이 근거하고 있던 가정을 순서만 뒤집는 것이거나,
아니면 서로 양립할 수 없는 두 이론을 뒤섞는 것에 불과할 것이다.
과거가 아니라 미래에서 완전성을 찾는 경향은 기독교에 의해
이 세상에 도입된 것이다.
사회의 진보가 더 나쁜 것에서 더 좋은 것으로 필연적으로 진행된다는 믿음은
고대 문헌에서는 거의 혹은 전혀 발견되지 않는다.

하지만 그 철학적 결함에 비해 이 이론이 인류에게 미친 영향은 훨씬 더 심대했다.
만약 자연법의 믿음이 고대세계에 보편적으로 퍼지지 않았다면
어떤 사상사적 전환이, 또 그에 따른 인류사적 전환이, 일어났을까는
실로 말하기가 쉽지 않다.

\para{자연법}
법, 그리고 법에 의해 결합되는 사회는 그 유아기에
두 가지 위험에 특히 취약하다.
하나는 법이 너무 빨리 발달할 수 있다는 것이다.
진보적인 그리스 공동체들에서 이런 일이 발생했거니와,
이들 공동체는 놀라운 능력으로 불편한 소송절차와 불필요한 법률용어의
질곡을 벗어던졌고, 곧이어 엄격한 규칙과 법규정들에 미신적 가치를 부여하는 일을
그만두었다.
이것으로 그 공동체의 시민들이 누린 직접적 혜택은 상당히 컸지만,
그것은 인류의 궁극적 이익에 기여하지는 못했다.
민족성의 드문 자질 중 하나는,
보다 높은 이상에 법을 일치시키려는 희망을 잃지 않으면서도,
법 자체의 적용과 운용에 있어
추상적 사법\hanja{司法}을 구현하는 데는 지속적으로 실패하는 능력이다.
유연성과 탄력성에 뛰어난
그리스의 지식인들은
엄격한 법형식의 틀 속에 스스로를 가둘 수가 없었던 것이다.
우리가 비교적 소상히 알고 있는 아테네의 인민법원을 두고 판단하건대,
그리스의 법원은 법률문제와 사실문제를 혼동하는 경향을 강하게 나타냈다.
아리스토텔레스의 <<수사학>>\latin{Treatise on Rhetoric}에 남아있는
웅변가\latin{orator}들과 법정 표현들의 흔적을 보건대,
순수한 법률문제의 변론은
판사들에게 영향을 줄 수 있는 모든 것을 끊임없이 고려하면서 이루어졌다.
이런 방식으로는 지속가능한 법학체계가 만들어질 수 없다.
특정 사건의 사실관계에 대한 완벽한 이상적인 결정에 성문법 규칙이 방해되는
경우라면 언제나 그 성문법 규칙을 완화하는 데 거리낌이 없었던 공동체는,
설령 후대에 어떤 법원리들을 물려준다 하더라도
오직 당대에 지배적이었던 옳고 그름의 관념에 기초한 것들만 물려줄 수 있을 뿐이다.
이러한 법은 후대의 보다 발달된 관념에 어울릴만한 틀을 전혀 제공할 수 없다.
기껏해야 그 법을 둘러싼 문명의 볼완전성을 드러내는 철학이 될 수 있을 뿐이다.

국가 사회 중에 그들의 법이
이러한 때이른 성숙과 때아닌 해체의 위험에 의해 위협받은 곳은 많지 않다.
로마인들이 이러한 위협에 심각하게 노출된 적이 있었는지는 모르겠으나,
어쨌든 그들의 자연법 이론에는 적절한 보호장치가 들어있었다.
분명 법학자들은 시민법을 점진적으로 흡수하는 체계로 자연법을 관념했으며,
시민법이 폐지되지 않는 한 자연법이 시민법을 대체할 수는 없다고 생각했다.
특정 소송사건을 감독하는 판사들이 자연법의 호소에 압도당할 정도로
그렇게 자연법이 신성하다는 인상은 유포되지 않았다.
이러한 관념의 가치와 유용성은 완벽한 유형의 법이 인간 정신의 눈 앞에
펼치지지 못하게 한 것이었고, 그러한 법에 무한히 가까이 다가갈 수 있다는
희망을 품지 못하게 한 것이었으며, 또한 아직 자연법에 조응하지 못한
기존 법이 부과한 의무를 실무가나 시민들이 거부하지 못하게 한 것이었다.
무엇보다 중요한 점은 이 모범적인 체계가,
후대에 인간의 희망을 꺾어놓았던 다른 많은 체계들과 달리,
결코 상상의 산물이 아니었다는 것이다.
그것은 결코 허황된 원리에 기초한 것으로 관념되지 않았다.
그것은 기존 법의 저변에 존재하며 기존 법을 통하여 추구되어야 한다고 생각되었다.
한마디로 그것의 기능은 구제수단을 제공하는 데 있었지, 혁명적이거나
무정부적인 것이 아니었다.
그리고 정확히 이 점에 있어 불행하게도 근대 자연법 사상은 고대의 그것을
닮지 않은 경우가 많다.

\para{벤담주의}
유아기의 사회에 나타나는 또 하나의 취약성은 훨씬 더 많은 민족들의 진보를
방해하고 가로막았다.
원시법의 엄격성은 대개 일찍이 종교와의 관련성 및 동일시에 의해 등장했거니와,
대부분의 민족들은 그 관행이 처음 체계적인 형태로 굳어질 당시 그들이
가지고 있던 인생관과 행위관에 얽매여왔다.
놀라운 운명으로 이러한 재난을 벗어난 민족이 한 둘 있거니와,
이들 나무줄기에 접목하여 몇몇 근대사회가 기름진 곳이 될 수 있었다.
하지만 여전히 세계의 더 많은 곳에서는 최초 입법자가 그려놓은 기본계획을
추종하는 것이 법의 완성이라고 여겨지고 있다.
만약 그런 곳에서 지식인이 법을 공부했다면, 한결같이 그들은
고대 텍스트에서 그 문자적 의미를 눈에 띄게 벗어나지 않고 이끌어낸 결론의
미묘한 고집스러움을 자랑스러워했을 것이다.
만약 자연법 이론이 비범한 탁월함을 로마법에 주지 않았더라면
로마인들의 법이 인도인들의 법에 비해 우월하다고 할 것이
무엇이 있는지 나는 모르겠다.
이 유일한 예외적인 사례에서, 다른 여러 이유들로 인류에 막대한 영향을 끼치게 될
한 사회의 눈 앞에서, 단순성과 조화성이 이상적이고 가장 완전한 법의 성질로
나타났던 것이다.
진보를 추구함에 있어 어떤 뚜렷한 목표를 가진다는 것이
한 민족이나 전문직업군에게 가지는 중요성은 아무리 강조해도 지나치지 않다.
지난 30년 동안 영국에서 벤담이 가졌던 막대한 영향력의 비밀은
이 나라 앞에 그러한 목표를 성공적으로 제시한 데 있다.
그는 우리에게 개혁의 뚜렷한 원칙을 제시했다.
지난 세기의 영국 법률가들은 아마도 명민했기에
영국법이 인간 이성의 완성이라는 흔해빠진 역설적 표현에 눈멀지는 않았겠지만,
일을 추진해나갈 다른 원리가 없었기 때문에 영국법을 믿는 척 행동했다.
벤담은 다른 모든 목표 위에 공동체의 선\hanja{善}을 두었고,
그리하여 오랫동안 밖으로 빠져나갈 길만 찾고 있던 흐름에 나갈 길을 열어주었다.

우리가 기술해온 관념을 벤담주의의 고대적 대응물이라고 부른다면
그것은 그리 훌륭한 비교가 못 될 것이다.
로마인의 이론은 저 영국인의 이론과 마찬가지 방향으로 인간의 노력을 이끌었다.
그것의 실천적 결과도 공동체의 일반적 선\hanja{善}을 꾸준히 추구해온
일군의 법개혁가들이 달성한 것과 크게 다르지 않았다.
하지만 그것이 벤담의 원리를
의식적으로 예견한 것이었다고 보는 것은 잘못일 것이다.
로마인들의 대중문헌이나 법학문헌에서, 분명
구제수단을 제공하는 입법의 목표로
때로 인류의 행복이 제시되곤 했지만,
자연법이라는 돋보이는 주장에 주어진 끊임없는 칭송에 비하면
벤담의 원리를 보여주는 증언은 거의 없거나 희미하다는 점을
주목해야 한다.
로마 법학자들이 기꺼이 수용한 것은 인간에 대한 사랑 같은 것이 아니라
단순성과 조화성---그들이 ``전아''\hanjalatin{典雅}{elegance}하다고 부르며
강조했던 것---의 감각이었다.
그들의 노력이 보다 정밀한 철학의 조언을 받아들인 자들의 노력과
우연히 일치했다는 것은 인류에게는 행운이었다.

\para{프랑스 법률가들}
자연법의 근대사로 전환하면, 우리는 그것의 영향력이 막대하다는 것은
말하기 쉬워도 그 영향이 좋은 것인지 나쁜 것인지를 자신있게
말하기는 어렵다는 것을 알고 있다.
근대 자연법 이론에서 나왔다고 할 수 있는 신조와 제도들은
우리 시대의 가장 첨예한 논쟁의 대상이거니와,
지난 백년 동안 프랑스가 서구 세계에 확산시킨 법, 정치, 사회에 관한
특수한 이념들 대부분이 자연법 이론에 그 원천을 두고 있다고 말할 때
이것이 잘 드러난다.
프랑스 역사에서 법률가들의 역할은,
그리고 프랑스 사상에서 법사상의 비중은,
언제나 대단히 컸다.
근대 유럽의 법학이 발흥한 곳은 사실 프랑스가 아니라 이탈리아였지만,
이탈리아로 유학하고 돌아와 전 유럽대륙에 건설된,
그리고 {\small(허사로 돌아갔지만)} 우리 영국에 건설이 시도된,
학자군 중에서 프랑스에 건설된 학자군이 그 나라의 운명에 가장 큰
영향을 끼쳤다.
프랑스의 법률가들은 즉각 카페 왕조 및 발루아 왕조의 왕들과
강력한 동맹관계를 형성했다.
프랑스의 왕권이 여러 소국과 속령들의 집합체에서 떨어져나와
결국 그 위에 성장하게 된 것은 무력에 의한 것인 동시에
법률가들이 왕의 대권을 옹호해주고 봉건적 세습규칙을 해석해 준 데에도 기인했다.
왕과 법률가들의 동맹으로 프랑스 왕들이
강력한 봉건제후들, 귀족들, 교회와의 투쟁과정에서 누린 이점은
중세를 거슬러 올라가 당시 유럽을 지배하던 이념을 고려하지 않으면
제대로 평가할 수 없다.
우선 일반화를 향한 강한 열정이 있었고 모든 일반적 명제를 향한 찬양이 높았다.
그리하여 법의 분야에서는, 여러 지방에서 관습적으로 사용되던 고립된
다수의 법규칙들을 하나로
포괄하고 요약하는 모든 일반적 공식\latin{formula}에 대한 무의식적 존중이 있었다.
이러한 공식은 로마법대전이나 표준주석\latin{the Glosses}\footnote{%
  주석학파를 집대성한
  아쿠르시우스(Accursius)의 표준주석(glossa ordinaria)을 말하는 듯.}에
익숙한 실무가라면 물론 얼마든지 제공할 수 있었다.
하지만 법률가들의 권력을 사뭇 증대시킨 또 다른 원인도 있었다.
우리가 말하는 이 시대에는 쓰여진 법 텍스트가 가진는 권위의 정도와 성질에 관한
관념이 보편적으로 퍼져있었다.
대체로 ``이것이 쓰여진 법이다''\latin{Ita scriptum est}라는 우선권을 가진
주장은 모든 항변을 침묵시키기에 충분했다.
우리 시대의 학자라면 인용된 공식을 조바심내며 조사하고,
그 출처를 따져묻고, {\small(필요하다면)} 인용이 들어있는 법령집이
지방관습을 대체할 만한 권위를 가지고 있지 않다고 부인하겠지만,
그 시대의 법학자들은 규칙의 적용가능성을 의문시하거나
기껏해야 학설휘찬이나 교회법에서 반대명제를 인용하는 것 외에
다른 것은 거의 시도하지 않았다.
법논쟁의 이러한 중요한 측면에 대해 사람들이 주저하는 관념을 가졌다는 것을
염두에 두는 것은 무척 중요하거니와,
그것은 법률가들이 국왕에게 힘을 실어준 것을 설명하는 데 도움이 될 뿐만 아니라,
몇몇 흥미로운 역사적 문제를 해명하는 데도 도움을 주기 때문이다.
위조된 교령\hanja{敎令}들\latin{forged decretals}\footnote{%
  `콘스탄티누스의 기증'을 포함한 `이시도르 위서'%
  \hyphlatin{(Pseudo-Isidore)}를 말한다.}을
만든 저자의 동기와 그의 특별한 성공은 이런 맥락에서
더 잘 이해될 수 있는 것이다.
보다 덜 흥미로운 예를 들자면, 브랙턴\latin{Bracton}의 표절을 이해하는 데도,
비록 부분적이지만, 도움을 준다.
헨리3세 시대의 저 영국 법률가는
순수한 영국법의 집성을 당대 영국인들에게 내놓을 수 있었는데,
이는 편제의 전부와 내용의 1/3을 로마법대전에서 직접 빌려온 논저였다.
로마법의 체계적인 연구가 공식적으로는 금지된 나라에서
이러한 작업을 감행했을 것이니, 이는 법학의 역사에서 영원히 풀리지 않는
수수께끼의 하나일 것이다.
그러나, 텍스트의 출처에 대한 고려는 차치하고라도,
쓰여진 텍스트가 가지는 구속력에 관한 당대의 여론 상황만 감안해도
우리의 놀라움은 다소간 완화된다.

프랑스의 왕들이 주권 확립을 위한 긴 투쟁을 성공적으로 종결지은 때,
즉 대체로 발루아^^b7앙굴렘 왕조의 재위기에 이르러,
프랑스 법률가들의 상황은 사뭇 특수한 것이었고 이 상태는
프랑스혁명 발발 시까지 지속된다.
한편으로, 그들은 프랑스에서 가장 식자층에 속했고
대단히 강력한 권세를 누리는 계급을 형성했다.
그들은 봉건귀족들과 나란히 특권계급의 신분을 가졌으며,
프랑스 전역에 걸쳐 분포한 기구를 통해 그들의 영향력을 행사했거니와,
이 기구는 국왕의 특허로 각지에 설립되어
폭넓은 명시적 권한과 더 폭넓은 묵시적 권리를 행사했다.
변호사, 판사, 입법자의 권한을 모두 가진 그들은 유럽 전역의 다른 동료집단들을
훨씬 능가하는 권력을 누렸다.
그들의 재판 기술, 표현의 능란함, 유추와 조화에 대한 세련된 감각,
그리고 {\small(가장 뛰어난 인물들로 판단하건대)}
그들의 정의관에 대한 열정적 헌신
따위는 그들 중에 특출난 재능을 보였던 다양한 인물들만큼이나 두드러졌다.
이들 인물들의 다양성은
퀴자\latin{Cujas}와 몽테스키외, 다그소\latin{D'Aguesseau}와
뒤물랭\latin{Dumoulin}처럼 서로 대척점에 위치한 이들 사이의
전 영역을 아우르는 것이었다.
하지만, 다른 한편으로, 그들이 집행해야 했던 법체계는
그들이 훈련받은 법학의 정신과는 현저히 다른 것이었다.
상당 부분 그들의 노력으로 만들어진 당대의 프랑스는
유럽의 다른 어떤 나라 이상으로 법의 변칙성과 불일치라는 저주에 휩싸여 있었다.
프랑스를 가로지르는 큰 구획선이 그 나라를
성문법지역\latin{Pays de Droit Écrit}과
관습법지역\latin{Pays de Droit Coutumier}으로 갈라놓고 있었으니,
전자는 쓰여진 로마법을 그들 법의 토대로 받아들이고 있었고,
후자는 지방적 관습과 조화되는 한도 내에서
단지 표현의 일반적 양식으로만, 그리고
법적 추론의 수단으로만, 로마법을 인정하고 있었다.
이러한 분열 아래에는 계속적으로 하위 분열이 존재했다.
관습법지역에서는 지방\latin{province}마다,
군현\latin{county}마다,
성읍\latin{municipality}마다 그 관습의 성질이 달랐다.
성문법지역에서는
로마법 층위 위에 봉건규칙들의 층위가 대단히 잡다한 양상으로
펼쳐져 있었다.
이러한 혼란상은 영국에는 존재한 적이 없었다.
독일에서는 존재했지만, 그것은 이 나라의 정치적^^b7종교적 분열상에
어울리는 것이었기에 한탄의 대상도, 심지어 감지의 대상도 되지 못했다.
국왕의 중앙권위가 부단히 강화되고 있었음에도,
행정의 통일성을 달성하기 위한 노력이 빠르게 진행되고 있었음에도,
인민들 사이에의 뜨거운 민족정신이 발달하고 있었음에도,
법의 비상한 다양성이 특별한 변화없이 지속된 곳은 프랑스가 유일했다.
이러한 현저한 대비\hanja{對比}는 여러 가지 심각한 결과를 낳았는데,
그중에서 첫째로 꼽아야 할 것은 프랑스 법률가들의 정신에 미친 영향일 것이다.
그들의 사변적 의견과 지성적 성향은 그들의 이해관계나 직업적 관행과
크게 상반되는 것이었다.
단순성과 통일성에 기초한 완전한 법에 대해 민감하게 느끼고 이를 완전히
수용하고 있음에도 불구하고,
그들은 프랑스법을 감싸고 있는 악덕을 근절 불가능하다고 믿었거나
혹은 그렇게 믿는 듯이 보였다.
실제로 그들은 덜 계몽된 프랑스인들 사이에서는 볼 수 없는 고집스러움으로
악습의 개혁에 저항하곤 했다.
그러나 이러한 자기모순을 조화시키는 길이 있었다.
그들은 열렬한 자연법론자들이었다.
그 자연법은 지방 간의 경계를, 성읍 간의 경계를 뛰어넘는 것이었고
귀족과 도시민\latin{burgess}의 구별을, 도시민과 농민의 구별을
알지 못하는 것이었으며
명료성, 단순성, 체계성에 최고의 지위를 부여하는 것이었으나,
그 신봉자들에게 어떤 특정한 진보도 의무지우지 않는 것이었고
존경받고 돈벌이되는 법기술을 직접 위협하지도 않는 것이었다.
자연법은 프랑스의 보통법이 되었다고 말할 수 있을 것이다.
아니면, 여하튼 자연법의 존엄과 가치는
모든 프랑스 법실무가들이 한결같이 승인하는 유일한 신조였던 것이다.
혁명 이전의 법률가들의 언어에서 자연법의 찬미는 자못 무조건적이었다.
특히,
순수한 로마법을 폄하하는 것을 의무로 여기곤 했던
관습법 연구자들이,
학설휘찬과 칙법휘찬만 존중하는 당대 로마법 학자들보다
훨씬 더 자연과 자연법을 열성적으로 이야기했던 것이다.
옛 프랑스 관습법의 최고 권위자였던 뒤물랭은
자연법에 대해서 몇몇 과도한 진술들을 하였거니와,\footnote{메인의 `고대법'에
대한 폴록의 주석에 따르면,
뒤물랭의 저술에서 이러한 자연법의 찬사는 찾을 수 없었다고 한다.}
그의 찬사가 담고 있는 특유의 수사학적 전환은
그것이 로마시대의 법학자들의 조심성과는 사뭇 거리가 먼 것이었음을 알려준다.
자연법의 가설은 이제 법실무를 이끄는 이론이 아니라
사변적 믿음의 규약이 되었다.
그리하여, 곧 언급하겠지만, 자연법의 최근의 변화에서는
지지자들에게 가장 약하게 존중받던 부분이 가장 강하게 존중받는 지위로
올라섰던 것이다.

\para{루소와 그의 이론}
18세기가 절반이 지났을 때 자연법의 역사에서 가장 중대한 시기가 도래한다.
그 이론과 결과에 대한 논의가 법전문가들 사이에서만 계속되었다면,
자연법이 누리던 신망은 감소되어갈 가능성이 있었다.
바로 이때 <<법의 정신>>\latin{Esprit des Lois}이 등장했던 것이다.
아무런 심사없이 무사통과되던 가정\hanja{假定}들에 극히 격렬하게 반발하는
특징을 조금은 과장되게 보여주는,
그러면서도 기존의 편견과 타협하려는 욕망의 흔적을
조금은 모호하게 보여주는,
몽테스키외의 이 책은,
그 모든 결함에도 불구하고,
자연법이 한 순간도 발붙인 적이 없었던 저 역사적 방법\latin{historical method}에
기초하여 논의를 전개한다.
이 책의 인기만큼이나 그 사상적 영향력도 마땅히 컸어야 했으나,
실은 꽃피울 시간이 허락되지 않았으니,
이 책이 파괴하고자 했던 반대가설이 갑자기 법정에서 거리로 뛰쳐나가
과거 법정이나 대학을 뒤흔들었던 때보다 훨씬 더 격렬한 논쟁의
한복판에 서게 된 것이다.
논쟁을 새롭게 촉발시킨 저 비범한 인물은
배운 것도 없고, 그리 유덕하지도 않으며, 강한 성격의 소유자도 아니었으나,
그럼에도 불구하고 생동하는 상상력과
그의 단점 대부분을 용서하고도 남을
인간에 대한 진정어린 불타는 사랑의 힘에 의해
역사에 지울 수 없는 각인을 남겼다.
1749년부터 1762년 사이에 루소가 출간한 문헌들만큼
인간의 정신에, 지성적 사고의 모든 형태와 색조에,
막대한 영향력을 행사한 예는
우리 시대에는 전혀 볼 수 없으며, 실로 세계 전체의 역사를 통틀어 한 두 번
있을까 말까 한 것이다.
그것은 벨\latin{Pierre Bayle}과
부분적으로 영국의 로크에 의해
시작되고 볼테르에 의해 완성된 저 순수한 우상파괴적인 노력 이래
처음으로 인간의 믿음의 건축물을 새로 건설하려는 시도였다.
그것은, 단지 파괴만 일삼는 노력에 비해 모든 건설적인 노력이
언제나 갖는 우월성 외에도,
사변적인 문제에 관하여 과거의 모든 지식의 건전성을 의심하는
거의 보편적인 회의주의의 한 복판에서 출현했다는 매우 큰 장점을 지닌다.
루소의 사상 가운데 핵심적인 상징은,
사회계약의 체약자라는 영국적 옷을 입은 존재이든,
일체의 역사적 특성에서 벗어난 벌거벗은 존재이든,
가상의 자연상태에서 살고 있는 인간인 점에는 변함이 없다.
이러한 이상적 상황 하의 상상적 존재에 어울리지 않는
모든 법과 제도는 원초적 완전성으로부터 타락한 것으로 낙인찍힌다.
저 자연의 피조물이 지배했던 세상의 모습에 가까이 다가가는 모든 사회적 변화는
칭송받을 만하고 어떤 대가를 지불하더라도 달성할 가치가 있다.
이 이론은 여전히 로마 법률가들의 그것과 일치하거니와,
자연상태를 채우고 있는 일련의 환영들 중에서
로마 법학자들을 매료시킨 단순성과 조화성을 제외한 일체의
속성과 특징들은 받아들여지지 않기 때문이다.
하지만 이 이론은 말하자면 아래위가 뒤집힌 이론이다.
이제 `자연법'이 아니라 `자연상태'가 숙고의 제일가는 주제이기 때문이다.
로마인들은 기존 제도들을 주의깊게 관찰하면 그중 일부는,
그들이 어렴풋이 실감했던 저 자연의 지배의 흔적을 이미 보여주고 있는 것으로,
혹은 사법\hanja{司法}적 정화를 거쳐서 보여줄 수 있는 것으로,
골라낼 수 있다고 생각했다.
루소의 믿음은 완전한 사회질서는 오직 자연상태를 고려함으로써만
도출될 수 있다는 것이었으니, 그 사회질서는
현실세계의 상태와 전적으로 무관한 것이고 그것과 전혀 닮지 않은 것이었다.
두 견해 간의 큰 차이는, 하나는
이상적인 과거와 닮지 않았다는 이유로 현재를 몹시 그리고 대체로 부정하는 반면,
다른 하나는 과거 못지 않게 현재도 필요한 것으로 보고 현재를 무시하거나
비난하려 하지 않는다는 데 있다.
자연상태에 기초하여 건설된 저 정치철학, 예술철학, 교육철학,
윤리학, 사회철학을 여기서 일일이 분석할 필요는 없을 것이다.
이 철학은 지금도 여러 나라의 저급한 사상가들을 매혹하는 힘을 가지고 있으며,
또한 의심할 바 없이 역사적 방법에 기초한 탐구를 방해하는
대부분의 선입견들을 낳은 다소 먼 조상이기도 하다.
하지만 오늘날 수준 높은 지성인들 사이에는 이 철학에 대한 불신이 무척 깊어,
사변적 오류의 비상한 생명력에 익숙한 이들도 놀라게 할 정도이다.
아마 오늘날 가장 빈번하게 제기되는 질문은 이 철학의 가치가 무엇이냐의 문제가
아니라, 백 년 전에 이 철학이 지배적 영향력을 가졌던 원인이
무엇이었냐의 문제일 것이다.
생각건대, 그에 대한 답은 간단하다.
고대법에만 관심을 두었을 때 초래될 법한 오해를 교정하는 데에
가장 적합했을 지난 세기의 연구분야는 종교에 관한 연구였다.
그러나 그리스 종교는, 당시에 이해된 바로는, 가공의 신화에 불과한 것으로
타기시되었다.
동양의 종교는, 설령 관심을 받았을지라도, 허황된 우주생성론에
빠져있는 것으로 여겨졌다.
연구의 가치가 있는 원시적 기록은 단 하나밖에 없었다.
바로 유대인들의 초기 역사였다.
하지만 이것에 관한 연구는 당대의 편견으로 인해 저지당했다.
루소 학파와 볼테르 학파의 공통점이 하나 있다면 그것은
일체의 고대종교에 대한, 무엇보다 히브리 민족의 종교에 대한,
철저한 경멸이라 할 것이다.
주지하듯이 당대의 지식인들에게는, 모세에게서 유래했다는 제도들이
실은 신이 명령한 것도 아니고, 그렇다고 모세 이후에 성문화된 것도 아니며,
오히려 그 제도들과 모세오경 전체가
바빌론 유수에서 귀환한 후에 위조된 것에 불과하다고
주장하는 것이 일종의 명예로운 일이었다.
그리하여 사변적 망상을 방지하는 담보장치 하나를 이용할 수 없게 된
프랑스 철학자들은, 성직자들의 미신이라 여겨진 것에서 탈출하려는 열망에서,
법률가들의 미신에로 앞뒤 가리지 않고 뛰어들었던 것이다.

\para{프랑스의 자연법}
자연상태 가설에 기초한 철학에 대한 존중이 일반적으로 감소했다고는 하나,
보다 거칠고 보다 쉽게 만져지는 측면에 관한 한,
뒷마당에서는 여전히 그것의 설득력과 인기와 권력이 사라지지 않고 있다고
해야할 것이다.
전술했듯이 여전히 그것은 역사적 방법의 적대자이다.
역사적 탐구 방법에 {\small(종교적 반대는 제쳐놓고)} 저항하거나
이를 비난하려 하는 자들은
대체로 사회나 개인의 비역사적^^b7자연적 상태에 대한 의식적^^b7무의식적
믿음에 기인하는 선입견이나 악의에 찬 편견에 의해 영향받고 있음을 알 수 있다.
그러나 자연의 교리와 자연법의 교리가 에너지를 잃지 않고 있는 것은
주로 정치적^^b7사회적 경향과의 동맹관계 덕분이다.
저 교리들은 이러한 경향의 일부를 자극했고, 다른 일부는 실제로 만들었으며,
대다수의 경향에게는 표현과 형식도 제공했다.
그것들은 프랑스로부터 다른 문명세계로 지속적으로 퍼져나가는 뚜렷한 관념이
되었고, 그리하여 문명을 변화시키는 일반적 사상체계의 일부가 되었다.
물론 이것이 인류의 운명에 행사하는 영향력의 가치는 우리 시대에
뜨거운 논쟁의 대상이 되고 있고, 또한 본 논저가 논의대상으로 삼는
목적이기도 하다.
하지만, 자연상태 이론이 최대의 정치적 중요성을 가졌던 때를 돌이켜볼 때,
제1차 프랑스혁명이 무수히 낳았던 엄청난 실망감의 원인을 제공하는 데
그것이 크게
기여했음을 부인할 이는 거의 없을 것이다.
그것은 당대에 거의 보편적이었던 악한 정신적 습관, 즉
실정법에 대한 경멸, 경험에 대한 조급증,
다른 모든 추론에 앞선 선험\latin{à priori}의 우선성 등을 낳거나
강하게 자극했다.
또한 이 철학이 생각이 짧고 관찰이 부족한 사람들의 마음을
사로잡아가는 것에 비례하여, 그것은 확실히 무정부주의적으로 되는 경향을 보였다.
뒤몽\latin{Dumont}이 벤담을 위해 출간하였으며,
특별히 프랑스적인 오류를 폭로한 벤담의 문건을 담고 있는
<<무정부적 궤변>>\latin{Sophismes Anachiques}\footnote{%
이 책은 프랑스어로 먼저 출간되었으며,
벤담 사후에 ``무정부적 오류''(Anarchical Fallacies)라는 제목으로
영어판이 나왔다. 부제는
``프랑스 제헌의회가 선포한 인간과 시민의 권리 선언에 대한 검토''다.}의
얼마나 많은 부분이
프랑스어로 번안된 로마인들의 가설에서 유래한 것인지, 그리고
그 가설을 참조하지 않고는 이해되지 않는 것인지를 알면 놀라지 않을 수 없다.
또한 이 점에 관하여 혁명의 절정기에 발간된 모니퇴르\latin{Moniteur}지를
찾아보면 흥미로울 것이다.
자연법과 자연상태에 대한 호소는 시대가 어두워질수록 점점 짙어졌다.
제헌의회 시절에는 비교적 드문 편이었으나,
입법의회 시절에는 훨씬 빈번하였으며,
음모와 전쟁에 관한 논쟁으로 시끄러웠던 국민공회 시절에는
항시적으로 나타났다.

\para{인간의 평등}
자연법 이론이
근대사회에 끼친 영향을 여실히 보여주는,
그리고 이러한 영향이 얼마나 소진되기 어려운가를 보여주는
한 가지 예가 있다.
생각건대,
인간의 근본적 평등의 원리가
자연법이라는 가정\hanja{假定}에
빚지고 있음에는 의문의 여지가 없다.
바로 이 ``모든 인간은 평등하다''야말로
시간의 흐르면서 법적 명제가 정치적 명제가 된
대표적 예인 것이다.
안토니누스 황조 시대의 로마의 법학자들은
``모든 인간은 자연적으로 평등하다''\latin{omnes homines naturâ
aequales sunt}라고 단언했지만,
그들의 눈에 이것은 어디까지나 법적 공리\hanja{公理}였다.
그들이 의도한 것은
가설적인 자연법 아래에서는, 그리고 실정법이 그것에 근접하는 한에서는,
로마 시민법이 가지고 있던 사람의 신분 간의 자의적 구별이
법적으로 사라진다는 것이었다.
이 규칙은 로마의 실무가들에게 엄청나게 중요했거니와,
로마법이 자연의 법전을 따른다고 생각되는 때면 언제나
시민과 외인\hanja{外人}, 자유인과 노예,
종족\hanja{宗族}과 혈족\hanja{血族} 간의 구별이
로마의 법정에서 사라졌다.
이와 같이 명확히 자기 의견을 밝힌 로마의 법학자들은
시민법이 사변적 법유형에 비해 모자란다고 해서 사회제도를
결코 비난하지 않았고, 자연의 질서에 완전히 일치하는
인간사회가 이 세상에 존재할 수 있으리라고 믿지도 않았다.
하지만 인간의 평등 원리가 근대의 옷을 입고 등장했을 때
그 옷은 확실히 전혀 새로운 색조의 의미를 띠고 있었다.
로마 법학자들이 ``평등하다''\latin{aequales sunt}라고 썼을 때는
그 의미가 글자 그대로였지만,
근대 로마법 학자가 ``모든 인간은 평등하다''라고 썼을 때
그 의미는 ``모든 인간은 평등해야 한다\latin{ought to be equal}''였던 것이다.
자연법은 시민법과 공존하는 것이고 차츰 시민법을 흡수하는 것이라는
로마 특유의 자연법 관념은 이제 확실히 망각되었거나
적어도 이해할 수 없는 것이 되었다.
기껏해야 인간 제도의 기원, 구성, 발달에 관한 이론을 말하던 단어들이
인류가 겪고 있는 기존의 커다란 해악을 지칭하는 표현이 되기 시작했다.
일찍이 14세기 초에, 인간의 출생시 상태에 관해 말하는 당시의 언어는,
분명 울피아누스나 그의 동료들의 언어를 그대로 따라하려 했으나,
실은 완전히 다른 형태와 의미를 띠게 되었다.
왕실의 농노들을 해방시킨
완고왕\hanja{頑固王} 루이의 유명한 왕령의 서문은
로마인들의 귀에는 생경하게 들렸을 것이다.
``자연법에 따르면 모든 사람은 자유롭게
태어나야 한다\latin{ought to be born free}.
그런데 아주 오래 전에 우리 왕국에 도입되어
지금까지 이어지고 있는 관행과 관습으로 인하여,
그리고 어쩌면 그들 선조들이 저지른 범죄행위로 인하여,
우리 평민들 가운데 다수가 예속상태에 떨어져 있다. 그리하여 우리는 \ldots'' 등.
이것은 법규칙이 아니라 정치적 도그마의 선언이었다.
그리고 이때부터 인간의 평등을 프랑스 법률가들은
그것이 마치 그들 학문의 저장고에 보관되어온 정치적 진리인 양 말해왔다.
자연법의 가설에서 연역되어 나온 모든 다른 것과 마찬가지로,
그리고 자연법 그 자체에 대한 믿음과 마찬가지로,
그것은 맥없이 승인되었고 여론이나 실무에 거의 영향을 끼치지 못했다.
그러나 그것이
법률가들의 점유에서 벗어나 18세기 문필가들과
그들에게 감화된 대중들의 점유로 넘어가면서,
이들의 신념을 표현하는 가장 두드러진 교리가 되었고
나아가 모든 신념을 요약하는 교리로 간주되었다.
하지만 그것이 1789년 사건 이후 마침내 권력을 획득하게 된 것은
프랑스 안에서의 인기에만 기인한 것이 아니었을 것이다.
18세기 중엽에 그것은 미국으로 건너갔던 것이다.
당시 미국 법률가들, 특히 버지니아 주의 법률가들은
당대 영국인들의 것과는 사뭇 다른 지식 계통을 가지고 있었던 것으로 보인다.
대륙 유럽의 법문헌들에서 유래한 것일 수밖에 없는 것들이
다수 포함되어 있었던 것이다.
제퍼슨의 저술을 조금만 들여다보더라도
프랑스에서 유행하던 반쯤은 법적이고 반쯤은 대중적인 견해들로부터
그가 강하게 영향받고 있었음을 알 수 있을 것이다.
의심할 여지 없이, 미국에서의 일련의 사건들을 이끌었던
그와 기타 식민지 법률가들은
프랑스 법률가들의 특유한 관념에 공감하였고,
``모든 인간은 평등하게 태어난다''라는 특히 프랑스적인 가정\hanja{假定}을
영국인들에게 보다 친숙한 ``모든 인간은 자유롭게 태어난다''는 가정과
결합하였으니,
이는 독립선언문의 첫 몇 줄에 잘 나타나있다.
독립선언문의 이 문장은 우리가 다루는 교리의 역사에서
가장 중요한 문장 중의 하나이다.
미국의 법률가들은 이렇게 인간의 근본적 평등을 무엇보다 강하게 긍인함으로써
그들 조국의 정치적 운동에 추동력을 부여했다.
영국에서는 영향력이 덜 하였으나, 영국은 아직
그 힘을 다 써버린 상태가 아니다.
그런데 그밖에도 그들은 저 교리를 수용한 본국인 프랑스에 그것을 되돌려주어
훨씬 더 큰 에너지를 만들어냈고 일반적 수용과 존중이 훨씬 더 강하게
주장될 수 있도록 했다.
제1차 제헌의회의 보다 신중한 정치가들조차
저 울피아누스의 명제를,
마치 그것이 인류의 직관과 본능에 동시에 기초하고 있다는 듯이,
반복하여 외쳤다.
``1789년의 원리들'' 가운데 그것은 가장 덜 공격받은 것이고,
근대의 여론에 가장 큰 영향을 끼친 것이며,
여러 사회의 헌정과 여러 국가의 정치에 가장 근본적인 변화를
약속하고 있는 것이다.

\para{국제법}
자연법의 가장 큰 기여는 근대 국제법과 근대 전쟁법을 탄생시키는 데서 수행되었다.
하지만 여기서는 이 영역에 대한 자연법의 영향을
그 중요성에 비해 훨씬 소략하게 고려하는 것으로 만족할 수밖에 없다.

국제법의 기초를 이루는 공준\hanja{公準} 중에,
또는 국제법의 초기 건설자들에서 유래한 상징을 많이 담고 있는 공준 중에,
무척 중요한 것들이 두 세 가지 있다.
그 중 첫 번째는 결정가능한 자연법이 존재한다는 입장이다.
그로티우스와 그 후계자들은 로마인들에게서 직접 이 가정을 가져왔으나,
결정의 양상에 관해서는
로마 법학자들과 차이가 크고 또한 그들 상호간에도 차이가 크다.
문예부흥 이후 대거 등장한 공법학자\latin{publicist}들의 대다수는
자연과 자연법에 대한 정의\hanja{定義}를 다루기 쉽게 새로 제공하려는
야심을 가지고 있었다.
공법학자들의 긴 행렬이 이어지면서 저 개념에는 첨가물이 대거 덧붙여졌거니와,
이는 윤리학의 거의 모든 이론들에서 따온 관념의 조각들로 이루어진 것이었으니,
이제 윤리학이 공법의 학파들을 장악하기에 이르렀음에 틀림없다.
자연상태의 필수적 성격으로부터 자연의 법전을 도출해내려는
그 모든 노력에도 불구하고, 그 결과물이란 것이
로마 법률가들의 진술을 묻지도 따지지도 말고 그대로 수용했더라면
얻을 수 있었을 결과물과 별반 다르지 않다는 점은
저 개념이 본디 역사적 성격의 개념이란 것에 대한 뚜렷한 증거인 것이다.
조약에 관한 국제법을 제쳐둔다면,\footnote{`약정은 지켜져야 한다'는
그로티우스의 계약법 이론은 로마법보다는 오히려 교회법의 영향을
더 강하게 받았다.}
국제법 체계의 얼마나 많은 부분이 순수한 로마법으로 만들어져 있는지
놀라울 지경이다.
로마 법학자들의 법리가 만민법\latin{ius gentium}과 조화된다고 생각되면
어디서나,
그것이 아무리 순수히 로마적 기원을 가진 것이라 할지라도,
공법학자들은 그것을 빌려올 구실을 발견했던 것이다.
또한 우리는
이렇게 파생된 이론들이
원래의 관념이 가지고 있던 약점을
그대로 안고 있다는 점도 관찰할 수 있다.
대부분의 공법학자들의 사고양식은 여전히 ``혼합적''인 것이었다.
이들의 저술을 연구함에 있어서 항상 부딪치는 큰 어려움은
그들이 논하는 것이 법인지 아니면 도덕인지,
그들이 기술하는 국제관계의 상태가 현실의 것인지 아니면 이상적인 것인지,
그들이 진술하는 것이 존재에 관한 것인지 아니면
그들이 생각하는 바람직한 당위에 관한 것이지를 판가름하는 일이다.

자연법이 국가들 사이에서\latin{inter se} 구속력을 가진다는 가정이
국제법의 근저에 놓여있는 두 번째 공준이다.
이 원리에 대한 일련의 주장과 수용은 근대 법학의 유아기로 거슬러올라가며
추적할 수 있을 것인데,
일견 그것은 로마인들의 가르침에서 직접 추론한 것으로 보인다.
사회의 국가적 상태와 자연적 상태의 차이는
전자에는 입법자가 뚜렷이 존재하지만 후자에는 없다는 것이므로,
만약 다수의 \hemph{단위들}\latin{units}이
어떤 공통의 주권자나 정치적 상급자에게도
복종하지 않는다고 인정되면 그들은 자연법의 지배 상태로 되돌아간다고 볼 수 있다.
국가들이 바로 그러한 단위들이다.
국가의 독립성 가설은 공통의 입법자 관념을 배제하거니와,
몇몇 이론에 의하면,
따라서 국가들은 자연의 원시적 질서에 복종한다는 관념이 도출되는 것이다.
그에 대한 대안은 독립된 공동체들 간에는 어떠한 법도 존재하지 않는다는 것이나,
이러한 무법\hanja{無法} 상태야말로
로마 법학자들의 성정이 끔찍히도 싫어했던 진공상태인 것이다.
물론, 로마 법률가들은 시민법이 추방당한 어떤 영역에 맞딱뜨리면
즉시 그 빈 공간을 자연의 명령으로 채워넣었을 것이라고 추정할 만한
외견상의 이유는 존재한다.
하지만 어떤 결론이 우리의 눈에 아무리 확실하고 자명해 보일지라도,
역사의 어느 순간에
실제로 그러한 결론이 도출되었을 것이라고 가정하는 것은 위험한 일이다.
현존하는 로마법 텍스트 가운데,
로마 법학자들이 자연법을 독립된 국가들 간에 구속력을 갖는 것으로
믿었다는 증거는, 내가 알기로는, 전혀 발견된 바 없다.
로마 제국의 시민들은,
자기들 국가의 통치영역이 문명의 영역과 경계를 같이 한다고 생각했기에,
국가들이 모두 동등하게 자연법에 복종한다는 것은,
설령 그런 생각을 해봤다 할지라도,
기껏해야 유별난 사변의 극단적 결과 쯤으로 치부했을 것이 분명하다.
사실 근대 국제법은,
로마법의 후손임에는 틀림없으나,
비정상적인 계통을 거쳐 로마법에 연결될 뿐이라고 해야할 것이다.
로마 법학의 근대 초기 해석자들은
만민법\latin{ius gentium}의 뜻을 잘못 이해하여,
로마인들이 국제 거래를 규율하는 법체계를 그들에게 물려주었다고
서슴없이 믿었다.
이 ``만민법''\latin{law of nations}은 처음에는 강력한 경쟁상대들과
권위를 두고 싸워야 했고,
유럽의 상황은 오랫동안 그것의 보편적 수용을 방해했다.
그러나 차츰 서구 세계는 로마법 학자\latin{civilian}들의 저 이론에
보다 우호적인 형태로 재편되어갔고,
상황의 변화와 더불어 경쟁적 이론들의 신용은 땅에 떨어졌다.
마침내, 특별한 행운이 겹쳐,
아얄라\latin{Balthazar Ayala}와 그로티우스는 그것에 대한 유럽의 열광적인 동의를
얻어낼 수 있었으니, 다양한 유형의 장엄한 계약이 체결될 때마다 이 동의는
계속해서 갱신되어갔다.
승리의 주역이라 할 수 있는 저 위인들이
그것을 완전히 새로운 기초 위에 놓으려 시도했음은 말할 필요도 없거니와,
이러한 재배치 과정에서 그 구조를
많이 바꾸었음---그러나 일반적으로 알려진 것보다는 훨씬 덜 바꾸었다---도
의문의 여지가 없다.
안토니누스 황조 시대 법학자들이 만민법과 자연법이 동일하다고 말한 것에
착안하여
그로티우스는 그의 직접적 선학들 및 후학들과 더불어
자연법에 특별한 권위를 부여하였으니,
그 권위는 만약 ``만민법''이 당시 모호한 의미를 갖지 않았다면
아마 결코 주장될 수 없었을 것이다.
그들은 자연법이 국가들의 법전임을 스스럼없이 주장했다.
그리하여
오로지 자연 개념에 대한 숙고로부터 도출된 것으로 여겨진 규칙들을
국제법 체계에 접목시키는 과정이 시작되었고,
이 과정은 거의 우리 시대까지 지속되고 있다.
또한 이는 인류에게 대단히 중요한 현실적 결과 하나를 낳았으니,
그것은 근대 초기 유럽의 역사에 전혀 알려지지 않은 것은 아니나
그로티우스 학파의 법리가 지배적 위치를 차지하면서 비로소
명백하게 그리고 보편적으로 인식된 것이다.
만약 국가들의 사회가 자연법의 지배를 받는다면,
그 사회를 구성하는 원자들은 절대적으로 평등해야 한다.
자연의 홀\hanja{笏} 아래서 모든 인간이 평등하듯이,
국가들 간의 상태가 일종의 자연상태라면 국가들도 평등하다.
크기와 힘이 서로 다르더라도 독립된 공동체들은
국제법의 관점에서 모두 평등하다는 이 명제는,
시대마다의 정치적 경향에 의해 위협받아온 것도 사실이지만,
대체로 인류의 행복에 기여해왔다.
문예부흥 이후 공법학자들이
자연의 존엄하신 주장으로부터 국제법을 도출하지 않았더라면,
저 법리는 결코 굳건한 반석 위에 설 수 없었을 것이다.

전체적으로 볼 때, 전술했듯이,
단순히 로마 만민법이라는 고대 지층에서 가져온 요소들에 비해
그로티우스 시대 이래 국제법에 추가된 것이
얼마나 작은 비율인지 놀라울 정도이다.
영토의 취득은 언제나 국가의 야심을 자극해왔거니와,
이러한 취득을 규율하는 규칙들은,
그 야심이 너무나 자주 불러오는 전쟁을 억제하는 규칙들과 더불어,
만민법상의\latin{jure gentium} 물건의 취득 방식에 관한 로마법을
단순히 옮겨적은 것에 지나지 않는다.
앞서 설명했듯이,
옛날 법학자들은
로마 인근의 여러 부족들을 관찰하여 그들 사이에 지배적인 관행에서
공통의 요소를 추출함으로써
이러한 취득 방식을
획득했다.
그 기원에 따라
``모든 민족들에 공통적인 법''으로 분류된 이 방식들을
후대의 법률가들은
그 단순성에 착목하여 자연법이라는 보다 최근의 개념과 어울린다고 생각했다.
그리하여 그들은 근대 만민법\latin{law of nations}으로 이어지는
길을 열었으니, 결과적으로
\hemph{영토}\latin{dominion}와 그것의 성격, 한계, 취득방식 및
안전하게 지키는 방식에 관한 국제법 분야는
순수한 로마 물권법---즉,
안토니누스 황조 시대 법학자들이 자연상태와의 모종의 일치를 보인다고
생각했던 바로 그 로마 물권법---인 것이다.
국제법의 이 분야가 적용될 수 있으려면,
주권자들 사이의 관계가
로마의 소유권자 집단의 성원들처럼 될 필요가
있었다.\footnote{로마에서는
원칙적으로 가부장(pater familias)들만이 소유권자가 될 수 있였다.}
이것이 국제법전의 초입에 놓여있는 또 하나의 공준인 것이다.
또한 이것은 근대 유럽 역사의 첫 몇 세기 동안은 지지받지 못한 공준이었다.
이것은 두 개의 명제로 분해될 수 있거니와,
``주권은 영토적이다'' 즉,
지구 표면의 한정된 부분에 대한 소유권을 갖는다는 명제와,
``주권자들 사이에서는 주권자가 당해 국가의 영토의,
\hemph{최고}\latin{paramount} 소유자가 아니라,\footnote{봉건제 하의
중층소유권 이론을 부정한다는 의미이다.}
\hemph{절대적}\latin{absolute} 소유자로 간주된다''는 명제가
그것이다.

오늘날 국제법 학자들은
형평과 상식에 기초한
그들의 국제법 법리들이
근대 문명의 모든 단계에서 쉽게 추론되어 나올 수 있다고
암묵적으로 전제한다.
이 전제는, 국제법 이론의 몇몇 현실적 결함을 감추고 있기는 하지만,
근대 역사의 대부분의 시기에 있어 결코 주장될 수 없는 전제이다.
국가들의 문제에 관하여 만민법\latin{ius gentium}의 권위가
전혀 도전받지 않았다는 것은 사실이 아니다.
오히려 그것은 오랫동안 몇몇 경쟁적인 이론들과 투쟁해야 했다.
또한 주권의 영토적 성격이 항상 인정되어왔다는 것도 사실이 아니다.
로마의 영토가 해체된 이후 오랫동안 인간의 정신은 그러한 이론과
조화될 수 없는 관념에 의해 지배되었던 것이다.
사물의 옛 질서와 그것에 기초한 견해가 쇠퇴하고 나서야,
새로운 유럽과 그것에 부합하는 새로운 관념이 등장하고 나서야,
국제법의 저 두 가지 공준이 보편적으로 받아들여질 수 있었다.

\para{영토주권}
근대사라고 불리는 것의 대부분의 기간 동안
``영토주권''\latin{territorial sovereignty}이라는 관념이
존재하지 않았다는 점을 명심할 필요가 있다.
주권은 지구의 일부분 또는 보다 세분된 영역에 대한 영유권과
아무런 관련이 없었던 것이다.
이 세계는 너무나 오랜 기간 동안 제국 로마의 그림자 아래 존재해왔기 때문에,
제국의 영토로 편입된 광대한 지역이 한때는,
외부의 간섭으로부터 면제되고 국가 간에 평등한 권리를 요구하는
다수의 독립국가들로 나뉘어져 있었다는 사실을 망각해버렸다.
만족\hanja{蠻族}들의 침입이 진정된 이후,
주권의 개념에는 다음과 같이 양면성이 있었던 것으로 보인다.
우선 그것은 ``부족주권''\hyphlatin{tribe-sovereignty}이라고 부를 수 있는
형태를 띠고 있었다.
물론
프랑크 족, 부르군드 족, 반달 족, 롬바르드 족, 서고트 족은
그들이 차지한 영토의 주인이었거니와,
이는 몇몇 지역의 지리적 명칭으로도 남아있다.
하지만 그들은 영토적 점유에 기초한 어떤 권리도 주장하지 않았으니,
사실 영토적 점유를 중요하게 여기지도 않았다.
그들은 삼림과 초원에서 가져온 전통을 계속 유지했던 것으로 보이며,
여전히 가부장적 사회의 유목 무리로서
단지 생계 수단을 제공하는 토지 위에 잠시
캠프를 치고 있을 뿐이라는 견해를 가지고 있었던 듯하다.
알프스 너머 갈리아 지방의 일부와 게르마니아 지방의 일부는
이제 프랑크 족이 사실상 지배하는 나라---오늘날의 프랑스---가 되었다.
하지만 클로비스의 후손인 메로빙거 왕조의 군장\hanja{君長}들은
프랑스의 왕이 아니었다. 그들은 프랑크 족의 왕이었던 것이다.
영토적 권리를 뜻하는 용어가 알려져 있지 않았던 것은 아니나,
처음에는 단지 부족이 점유한 땅의 \hemph{일부}를
통치하는 통치자를 지칭하는 편리한 수단의 하나로만 사용되었던 듯하다.
부족 \hemph{전체}의 왕은 그의 백성들의 왕이었지,
그의 백성들이 살고 있는 여러 토지의 왕은 아니었다.
이러한 특수한 주권 관념에 대한
대안으로---이 논점은 매우 중요한데---보편적 지배의 관념이
존재했던 것으로 보인다.
군주가 부족원들의 군장이라는 관계를 청산하고
자기자신을 위해 새로운 주권 형태를 만들어내고자 원했을 때,
받아들일 만한 선례로서 그에게 주어진 것은 로마 황제들의 지배형태였다.
흔히 쓰이는 인용구를 차용하자면, 그는
``황제가 아니면 아무 것도 아닌''\latin{aut Caesar aut nullus} 것이
된 것이다.\footnote{`전부 아니면 전무'(all or nothing)의 뜻으로 종종 쓰인다.}
비잔틴 황제의 완전한 대권\hanja{大權}을 주장하거나, 아니면 아무런 정치적 지위를
갖지 않는다는 것이다.
우리 시대에는 새로운 왕조가 폐위된 왕조의 기존의 권리를 지워버리고자 할 때,
\hemph{영토}가 아닌 \hemph{인민}을 지칭하는 용어를 사용한다.
그리하여 오늘날에는 프랑스인의 황제들과 왕들이 존재하고,
벨기에인의 왕이 존재한다.
그러나 우리가 다루는 저 시대에는 다른 대안이 사용되었다.
더 이상 부족의 왕으로 불리고 싶지 않은 군장은 세계의 황제를 자처해야 했다.
따라서, 세습 궁재\hanja{宮宰}들은 그들이 이미 오래 전부터 사실상
무력화시켰던 국왕들과 더 이상 타협하고 싶지 않았을 때,
스스로를 단순히 프랑크 족의 왕이라고 부르길 원하지 않았다.
이 호칭은 폐위된 메로빙거 왕조에 속하던 것이기 때문이다.
그렇다고 프랑스의 왕이라는 호칭도 쓸 수 없었다.
이 호칭은,
비록 알려져 있지 않은 것은 아니었으나, 존엄성을 갖지 못하던 것이기 때문이다.
그리하여 그들은 보편 제국을 지향하는 호칭을 사용했다.
그들의 동기는 크게 오해의 대상이 되었다.
최근의 프랑스 학자들은 샤를마뉴를 시대를 앞서간 인물로 그려내는 것을
당연시하거니와,
계획을 추진하는 에너지에 있어서는 물론이고
그의 계획의 성격 또한 그러하다는 것이다.
어떤 사람이 그의 시대를 앞서갈 수 있는지의 여부는 차치하고라도,
한 가지 분명한 것은, 무한한 영토를 추구했던 샤를마뉴는
그 시대의 특유한 관념이 그에게 허락한 유일한 길을 따랐을 뿐이라는 점이다.
지성을 중시하는 그의 탁월한 능력에는 이론\hanja{異論}이 없지만,
이는 그의 행위 때문에 그러한 것이지, 그의 이론 때문에 그러한 것이 아니다.

이러한 독특한 견해는 샤를마뉴의 세 명의 손자들 사이에서
상속재산이 분할되었을 때에도 그대로 유지되었다.
대머리 샤를, 루이, 그리고 로타르는 이론적으로는 여전히
로마 제국의---이 용어를 사용하는 것이 적절하다면---황제들이었다.
동로마황제와 서로마황제가 각각 법적으로는 세계 전체의 황제이지만
사실은 그 절반씩을 통치했던 것처럼,
저 세 명의 카롤링거 황제들도 권력은 제한되어 있지만
법적 타이틀은 무제한적이라 여겼던 것으로 보인다.
비만왕\hanja{肥滿王} 샤를의 죽음으로 또다시 분할이 이루어진 이후에도
이러한 주권의 보편성 관념은
오랫동안 황제의 지위와 관련되어 있었고,
실로 게르만 제국이 존속하는 한 그것과 완전히 단절될 수 없었다.
영토주권---주권을 지구 표면의 한정된 부분의 점유와 관련짓는 견해---은
명백히 봉건제도\latin{feudalism}의 자손, 그것도 뒤늦은 자손이었다.
이는 선험적으로 예상할 수 있거니와,
봉건제도는 역사상 최초로 인적\hanja{人的} 의무를, 그리고 결과적으로
인적 권리를, 토지 소유와 연결지었던 것이다.
그것의 기원과 법적 성격에 관한 적절한 견해가 무엇이건 간에,
봉건 구조를 생생하게 묘사하는 가장 좋은 방법은
그 밑바닥부터 시작하는 것이다.
우선 봉신\hanja{封臣}의 역무를 설정하고 제한하는 한 조각 토지에 대한
봉신의 관계를 고려하고,
그 다음 차츰 상위의 수봉\hanja{受封}관계로 올라가면서 원의 반경을 좁혀나가,
마침내 체제의 정점에 이르는 방법인 것이다.
암흑시대 후기 동안 그 정점이 정확히 어디에 위치했는지는 확인하기가 쉽지 않다.
아마도, 부족주권의 관념이 실제 쇠퇴한 곳이라면 어디서나,
그 정점은 서구 세계의 황제로 여겨지던 자들에게 언제나 주어졌을 것이다.
그러나 머지않아 제국의 권위가 먹혀드는 영역이 대폭 축소되자,
그리고 황제들이 얼마 남지 않은 그들 권력을
독일 지역과 북이탈리아 지역에 집중시키자,
과거 카롤링거 제국의 나머지 모든 지역에서 최고 봉건 수장들은 사실상
상급자가 없는 상태가 되었다.
차츰 그들은 새로운 상황에 적응해갔고,
불입\hanjalatin{不入}{immunity}의 사실상태는
마침내 종속\hanja{從屬}의 법이론을 가려버렸다.
그러나 이러한 변화가 쉽게 일어나기 어려웠을 것을 알려주는 여러 징후가 존재한다.
사실, 사물의 본성상 어딘가에 최고 권력이 반드시 존재해야 한다는 관념 탓에,
세속적 최고성을 로마 교황청에 부여하는 경향이 점점 커지고 있었던 것이다.
관념 혁명의 최초의 단계는
프랑스의 카페 왕조에 의해 완성된다.
그 이전까지는,
이제 카롤링거 제국에서 갈라져나온 몇몇 대\hanja{大}영지의 보유자들이
스스로를 공작이나 백작이 아닌 왕으로 자처하기 시작하고 있었다.
그런데 파리와 그 인근에 한정된 영토를 가진 저 봉건군주가
옛 왕가로부터 \hemph{프랑스인의 왕}이라는 타이틀을 찬탈하면서
중요한 변화가 일어나기 시작했다.
위그 카페와 그 후계자들은 전혀 새로운 의미의 왕들이었으니,
남작의 그의 영지에 대한 관계, 봉신의 그의 자유보유지에 대한 관계와
동일한 관계에서 프랑스 토지에 대한 주권자였던 것이다.
비록 오랫동안 저 옛 부족적 호칭이 통치왕가의 공식 라틴어 호칭으로 남아있었으나,
고유어 호칭에서는 빠르게 \hemph{프랑스의 왕}으로 변모되어갔다.
프랑스에서의 국왕 지위의 형식은 다른 곳에서 동일한 방향으로 일어나고 있던
변화를 뚜렷이 촉진시키는 결과를 가져왔다.
앵글로색슨 왕가들의 왕은 부족적 군장과 영토적 주권자 사이의
중간지대에 머물러 있었는데,
노르만 왕조 군주들의 권력은 프랑스 왕의 그것을 본받아 명백히
영토적 주권자의 모습을 띠었다.
이후 건설되거나 공고화된 모든 영토는 이러한 후대의 모델에 입각하여 형성되었다.
스페인, 나폴리, 그리고 이탈리아 자유도시들의 폐허 위에 건설된
군주국들은 모두 영토적 주권을 가진 통치자들의 지배에 놓였다.
부연컨대, 베네치아가 이 견해에서 저 견해로 옮겨가면서 점진적으로
타락해간 것만큼 이상한 일도 별로 없을 것이다.
해외정복을 시작할 때의 베네치아 공화국은
다수의 피지배 속주들을 통치하는
로마 국가 체제의 예시의 하나로
스스로를
간주했었다.
그로부터 한 세기가 지난 후의 베네치아는
이탈리아와 에게해의 점유지들에 대해
봉건영주의 권리를 주장하는 주권체로 보여지기를
바라게 되었던 것이다.

\para{국제법}
주권이라는 주제에 관한 대중의 관념이 이러한 근본적인 변화를 겪고 있던 동안,
우리가 오늘날 국제법이라고 부르는 것을 대신하던 체계는
오늘날의 그것과 형식에 있어서도 달랐고 원리에 있어서도 불일치했다.
유럽 중에서도 로마^^b7게르만 제국에 속하는 드넓은 지역에서의
국가들 간의 연합관계는 제국칙령이라는 복잡하고 불완전한 메커니즘에 의해
규율되었다.
우리에게는 놀랍게 보일지 몰라도, 당시 독일지역 법률가들 사이에서는
국가들 간의 관계가 제국 내부에서든 바깥에서든
만민법\latin{ius gentium}에 의해서가 아니라,
황제를 중심으로 하는 순수한 로마법에 의해
규율되어야 한다는 생각이 널리 선호되었다.
이 법리는, 우리의 예상과는 달리,
제국 바깥의 나라들에서도
그다지 확신에 찬 거부의 대상이 되지 못했다.
그러나 실제적으로는, 유럽의 나머지 지역에서는
봉건제의 지배복종관계가 공법의 대체물을 제공하고 있었다.
그리고 봉건제가 쇠퇴하거나 모호해지자 그 배후에서는,
적어도 이론적으로는 최고 규율권력이 교회 수장의 권위에 속한다는 이론이
모습을 드러냈다.
하지만 봉건권력도 교회권력도 15세기에 이르면, 아니 이미 14세기부터,
빠르게 쇠퇴하고 있었음이 확실하다.
또한
오늘날의 전쟁의 구실이나 동맹의 동기로 공언된 것들을 살펴보면,
조금씩 옛 원리들이 추방되고 있었고,
그 대신 나중에 아얄라와 그로티우스에 의해 조화되고 공고화될 견해들이
비록 조용하고 느리지만
괄목할만한 진전을 이루고 있었음을 알 수 있다.
저 모든 권위가 융합되었다면 모종의 국제관계 체계가 진화되어 나올 수 있었을까,
그리고 그 체계는 그로티우스의 체계와 중대한 차이를 갖는 것이었을까 따위는
오늘날 우리로서는 알 수 없거니와, 실로 종교개혁이 한 가지를 제외하고는
모든 잠재적 가능성을 파괴해버렸기 때문이다.
독일지역에서 시작된 종교개혁으로 제국의 제후들은 건널 수 없는 깊은 골을
사이에 두고 분열되었고, 최고 권력의 황제라도
이 골을 메울 수가 없었다.
비록 황제가 중립적이었다 할지라도 그러했을진대,
하물며 황제는 종교개혁에 반대하는 교회의 입장에 동조해야 했다.
교황도 동일한 곤경에 처해 있었음은 말할 나위도 없다.
그리하여 분쟁당사자들 사이에서 중재의 역할을 담당해야 할 저 두 권위는
그들 스스로가 국가들 간의 분열에서 한 쪽 당파를 대표하는 수장들이 되어버렸다.
이미 허약해진 봉건제도는 공법관계의 원리로서의 신망을 잃어버려,
종교적 당파성에 대항할만한 어떠한 안정적 결합도 제공할 수 없었다.
거의 카오스에 가까운 이러한 공법의 상황 하에서,
로마 법학자들이 지지했을 법한 그러한 국가체체에 관한 견해만이
유일하게 남게 된 것이다.
그로티우스가 보여준 저 견해의 겉모습과 조화성과 탁월성은
사실 당대의 지식인이라면 누구나 알고 있었다.
그러나 <<전쟁과 평화의 법>>\latin{De Jure Belli et Pacis}의 경이로움은
그것이 신속하고도 완전하게 그리고 보편적으로 성공을 거두었다는 데 있다.
30년전쟁이 가져온 전율, 고삐풀린 군인들의 방종이 불러온 무한한 공포와 연민,
이런 것들도 분명 그것의 성공을 어느 정도 설명할 수 있겠지만,
이것만으로 충분한 설명이 되지는 못한다.
만약 그로티우스의 저 위대한 저서에서 스케치된 국제관계의 건축물의 설계도가
이론적으로 완벽한 모습을 띠지 않았다면,
저 저서는 법률가들에 의해 버림받고
정치인들과 군인들에 의해 무시당했을 것이라는 점은
당대의 관념을 깊이 천착해들어가지 않더라도 쉽게 알 수 있다.

\para{그로티우스의 체계}
말할 것도 없이,
그로티우스의 체계가 갖는 사변적 완벽함은 우리가 논의해온
영토주권의 관념과 밀접하게 관련되어 있다.
국제법의 이론은 국가들이 그들 상호간의 관계에서는 자연상태에 있다고 가정한다.
그러나
그 근본전제에 의하면,
자연적 사회를 구성하는 원자들은
상호간에 고립되어 있어야 하고
독립되어 있어야 한다.
만약
약하게라도 그리고 가끔씩이라도
그들을 연결시켜주는 상위 권력이 존재하여
공통의 주권자임을 주장한다면,
바로 그 공통의 주권자 개념으로부터 실정법 관념이 도입될 것이고
자연법 관념은 배제될 것이다.
따라서 제국 수장의 보편적 영주권\hanja{領主權}이
순 이론적으로라도 받아들여졌다면, 그로티우스의 노력은 헛수고가 되었을 것이다.
근대 공법이론과 지금까지 그 발달과정을 서술해온 주권개념 간의 접점이
이것만 있는 것은 아니다.
전술한대로, 국제법의 분야 중에는
그 전부를 로마 물권법에서 가져온 분야들이 있다.
이로써 무엇을 추론할 수 있는가?
주권의 관념에 내가 서술했던 변화가 일어나지 않았다면---주권이
지구의 한정된 부분에 대한 소유권과 관련하여 관념되지 않았다면,
다시 말해, 영토주권이 되지 않았다면---그로티우스 이론의 세 부분은
적용 불가능한 것이 되었으리라는 것이다.\footnote{그로티우스의
<<전쟁과 평화의 법>>은 모두 세 권으로 구성된다.}


\chapter{원시사회와 고대법}

법이라는 주제를 과학적으로 다룰 필요성은 현 시대 들어
완전히 망각된 적이 없거니와,
다양한 재능을 가진 인재들이
이러한 필요성의 인식 하에 논문들을 제출해왔다.
그러나, 생각건대,
지금까지 과학의 자리를 대신 차지하고 있던 것은
대체로 일군의 추측이었다는 것에는 의심의 여지가 별로 없다.
앞의 두 장에서 살펴보았던 로마 법률가들의 추측이 바로 그런 것들이다.
이렇게 추정적 자연상태 이론과
그것에 어울리는 법원리의 체계를 인정하고 수용하는 일련의 명시적 진술들이
이것들을 발명한 시대로부터 오늘날에 이르기까지 거의 중단없이 지속되어왔다.
근대 법학을 기초놓은 주석학파\latin{Glossators}의 주석에서도,
이들을 계승한 스콜라주의 법학자들의 저술에서도 그것들은 등장한다.
교회법학자들의 법리에서도 쉽게 눈에 띈다.
문예부흥기에 쏟아져나온 놀라울 정도로 박학다식한
로마법 학자\latin{civilian}들에서는
그것들이 전면에 부상한다.
\wi{그로티우스}와 그 후계자들은 그것들에 명료함과 그럴듯함뿐만 아니라
실무적 중요성도 부여했다.
그것들은 \wi{블랙스톤}의 저서의 서론 장들에서도 읽을 수 있거니와,
이는 뷔를라마키\latin{Jean-Jacques Burlamaqui}의 저서에서
글자 그대로 옮겨적은 것이다.
오늘날 법학도와 실무가들을 위해 출간된 교재들의 첫머리를 장식하고 있는
법의 제1원리에 관한 논의는 언제나 저 로마인들의 가설을
재진술하고 있는 것에 불과하다.
그러나 이들 추측의 고유한 형식뿐 아니라
때로는 스스로를 감추고 있는 위장술에서도
우리는 그것들이 얼마나 교묘하게 인간 정신에 섞여드는지 잘 파악할 수 있다.
로크의 사회계약론에서 법의 기원에 관한 이론은
그 로마적 유래를 거의 숨기지 않거니와,
실로 고대의 견해가 근대인들에게 매력적으로 보이려면
어떤 모습을 갖추어야 하는지를 알려준다.
한편, 동일한 주제에 대한 홉스의 이론은
로마인들과 그 후예들이 생각했던 자연법의 현실성을
부인하기 위해 의도적으로 고안된 것이다.
그러나 영국의 정치인들을 오랫동안 적대적 진영으로 양분했던
이들 두 이론은 양자 모두 인류의 비역사적이고 검증불가능한 상태를
근본적 전제로 삼는다는 점에서 서로 닮아있다.
물론 로크와 홉스는 사회 이전 상태의 성격에 대해서, 그리고
그 상태로부터 우리가 알고 있는 사회 상태로 이월하는 계기가 되는
비상한 행위가 어떤 것이냐에 대해서, 서로 의견을 달리한다.
하지만 원시상태의 사람과 사회상태의 사람 사이에
이들을 갈라놓는 커다란 틈이 있다는 생각에는 일치하거니와,
이 관념이 의식적으로든 무의식적으로든
로마인들에게서 빌려온 것이라는 점에는 의문의 여지가 없다.
사실 법현상을 이들 이론가들이 생각한 방식대로---즉, 하나의 거대한
복합체로---파악한다면, \paren{그럴듯하게 해석되면}
모든 것을 조화시킬 수 있는 영리한 추측에 의지하여
우리가 우리 스스로 설정한 과업을 자주 회피하게 되더라도,
아니면 절망에 빠져 체계화의 노력을 때로 포기하게 되더라도,
그것은 놀라운 일이 아닐 것이다.

\para{몽테스키외}
로마인들의 법리와 동일한 사변적 기초를 가지는 법이론으로부터
두 명의 유명인사는 제외함이 마땅하다.
그 첫 번째는 \wi{몽테스키외}라는 위대한 이름과 관련된 인물이다.
<<법의 정신>>의 첫 부분에는 다소 모호한 표현들이 나오는데,
저자가 당대의 지배적 견해에 공개적으로 도전장을 제출하기를 꺼려했기
때문이라고 여겨진다.
하지만 저 책의 일반적 흐름은 확실히 그 주제에 관한 이전의 어떤 관념과도
결별하는 모습을 보여준다.
흔히들 지적된대로,
방대한 조사를 통해 가상의 법체계들로부터 끌어모은 다양한 사례들 속에는,
상스럽고 생경하고 외설스런 습속과 제도들을 특별히 강조함으로써
문명사회의 독자들을 놀라게 하려는 갈망이 뚜렷이 엿보인다.
그것의 일관된 주장은 법이 기후, 지리적 위치, 우연, 기망 따위의
산물---용인할만한 항구성을 가지고 작용하는 것을 제외한 모든 원인의
결실---이라는 것이다.
실로 몽테스키외는 인간의 본성을 전적으로 유연한 것으로,
외부의 영향을 수동적으로 재생산하고 외부에서 주어진 충동에 묵묵히 복종하는
존재로 보는 듯하다.
바로 여기에 그의 체계가 체계로서 실패할 수 밖에 없는 오류가 있다.
그는 인간 본성의 안정성을 지나치게 평가절하한다.
그는 인류가 상속받은 자질을,
각 세대가 윗 세대에게서 물려받고 약간의 변경을 주어 다음 세대에
전달하는 자질을,
거의 혹은 완전히 무시한다.
물론, <<법의 정신>>에서 지적된 저 변경 원인들에 대한 적절한 고려가
없는 한, 사회현상도, 그리고 결과적으로 법현상도, 제대로 설명할 수 없다는
것은 틀림없는 진실이다.
그러나 몽테스키외는 그 원인들의 숫자와 힘을 지나치게 과대평가한 듯하다.
그가 나열하고 있는 비정상적 현상들은 거짓된 보고서나
잘못된 해석에 기초한 것이었음이 그후 밝혀졌다.
또한 나머지 것들 중에서도 상당수는 인간 본성의 가변성이 아니라
항구성을 증명하는 것들이니,
그것들은 인류의 이전 단계의 유산이며,
그렇지 않았다면 받았을 영향력을 끈질기게 거부해온
결과이기 때문이다.
진실은 인간의 정신, 도덕, 신체의 구조에서 안정적 부분이 대부분을
차지한다는 것이다.
그것이 변화에 저항하는 힘은 충분히 커서,
비록 세계의 일부 지역에서 인간 사회의 다양한 변이는 분명 존재하지만,
변화는 그것의 양, 성격, 일반적 방향성을 확인할 수 없을 정도로
그렇게 빠르게 일어나지도 광범위하게 일어나지도 않는다.
우리는
현재 우리가 가지고 있는 지식만을 이용하여 진리에 접근할 수밖에 없지만,
그렇다고
진리가 너무 멀리 있으므로, 혹은 \paren{같은 말이지만}
장래에 너무 많은 수정이 필요하게 될 것이므로,
그것이 쓸모없고 배울 바가 없다고 생각할 필요는 없는 것이다.

\para{벤담}
주목의 대상이 되어온 또 하나의 이론은 벤담의 역사이론이다.
\wi{벤담}의 저술 여기저기서 모호하게 \paren{그리고 어쩌면 소심하게}
전개된 이 이론은 <<정부론 단편>>에서 시작되어 최근 존 \wi{오스틴}에 의해 완성된
법개념 분석론과는 사뭇 차별성을 보인다.
법을 특수한 상황에서 부과된 특정한 성격의 명령으로 분석하는 것은
언어의 문제---물론 이것도 자못 무서운 것이지만---로부터 우리를 보호해주는 것
이상을 하지 못한다.
그러한 명령을 부과하는 사회적 동기가 무엇인지,
그러한 명령들 사이의 관계는 어떠한지,
종래의 명령을 대체한 새로운 명령이 종래의 것에 대해
어떤 의존성을 갖는지 등에 대해서는
아무 것도 답해주지 않는다.
벤담이 제공하는 답변은,
일반적 공리\hanja{功利}에 관한 사회의 견해가 변함에 따라
사회는 자신의 법을 변경해왔고 또 변경하고 있다는 것이다.
이 명제가 거짓이라고 말하기는 어렵겠지만,
확실히 별로 실속은 없는 명제로 보인다.
법규칙을 변경할 때 한 사회에게, 더 정확히는 그 사회의 통치 부분에게,
공리로 여겨지는 것은 변경을 만들어낼 때 그것이 가지는 어떠한 목표와도
정확히 같은 것이기 때문이다.
공리나 최대의 선\hanja{善}이란 결국 변경을 추동하는 힘의 다른 이름에 불과하다.
우리가 법이나 여론의 변화 규칙으로 공리를 주장할 때,
이 명제로부터 우리가 얻는 것이라고는
변화가 일어나고 있다고 말할 때 거기에 암시되어 있는 용어를
명시적인 용어로 대체하는 것말고는 아무 것도 없다.

\para{적절한 탐구방법}
이렇게 기존의 법이론에 대한 불만이 널리 퍼져 있기에,
또한 그들이 해결한다고 내세운 문제가 실제로는 전혀 해결되지 않고 있다는
확신이 너무나 일반적이기에,
완전한 결과를 얻기 위해 필요한 어떤 탐구방법을 저 이론가들이
불완전하게 따랐거나 아니면 전적으로 무시한 것이 아닌가 하는 의심이
뒤따를 수밖에 없는 것이다.
실로, 아마도 \wi{몽테스키외}의 것을 제외한 저 모든 사변적 이론들은
한 가지를 철저하게 무시했다.
저 이론들이 등장한 특정 시대로부터 멀리 떨어진 시대의 법이 실제로
어떠했는가에 대해 그들은 전혀 고려하지 않는다.
저 이론의 창시자들은 그들 자신의 시대와 문명의 제도에 대해서는,
그리고 그들이 어느 정도 지적으로 공감하는 다른 시대와 문명의 제도에 대해서는
주의 깊게 관찰했지만,
그들 자신의 사회와 많은 외관상의 차이를 가진 초기사회의 상태에
관심을 돌릴 때면 누구도 예외없이 관찰하기를 중단하고 추측하기를 시작했다.
따라서 그들이 저지른 잘못은 물질적 우주의 법칙을 탐구하려는 자가
가장 단순한 구성요소인 입자로부터 시작하는 대신에
기존의 물질세계 전체를 명상하는 것으로부터 시작하는 오류에 비견될 만하다.
다른 사고 영역보다 법학의 영역이라고 해서 이런 과학적 오류가 더 많이
용서받을 수 있다고 생각하는 사람은 분명 아무도 없을 것이다.
먼저, 가능한 한 원초적 상태에 가까운 가장 단순한 사회형태로부터
출발해야 할 것이다.
다시 말해, 이러한 탐구의 통상적인 과정을 따르고자 한다면,
우리는 원시사회의 역사를 가능한 한 멀리 거슬러올라가야 한다.
초기 사회가 보여주는 현상은 처음에는 이해하기 쉽지 않겠지만,
이러한 이해의 어려움은 우리를 당혹케하는 현대 사회 구조의
난해한 복잡함에 비하면 아무 것도 아니다.
그것은 생경함과 상스러움에 기인하는 어려움일 뿐,
초기사회의 숫자나 복잡성에 기인하는 것이 아니다.
현대적 관점에서 바라볼 때 만나는 놀라움을 극복하기가
쉽지 않은 것일 뿐, 이것을 극복하고 나면 그것들은 충분히 그 수가 적고
또한 충분히 단순하다.
하지만, 비록 그것들이 생각보다 많은 어려움을 준다 할지라도,
오늘날 우리의 행위를 규율하고 있고 우리의 행동을 형성하고 있는
모든 형태의 도덕적 제한이 전개되어나올 맹아를 확인하기 위해 들이는
우리의 고통은 결코 낭비라 할 수 없을 것이다.

\para{타키투스의 게르마니아}
우리가 알고 있는 원초적 사회 상태는 세 가지 전거\hanja{典據}에 기초한다:
\hypertarget{contemporary}{당대의}
관찰자들이 그들보다 문명의 진보 수준이 낮은 사회를 기술\hanja{記述}한 것,
특정 민족이 자신들의 초기 역사에 관해 기록한 것,
그리고 고대법이 그것이다.
첫 번째 종류는 우리가 기대할 수 있는 가장 좋은 것이다.
사회들은 동시에 진보하는 것이 아니라 진보의 정도가 서로 다르기 때문에,
체계적인 관찰의 습관을 가진 사람들이 인류의 유년기에 놓여있는 사람들을
관찰하고 기술할 수 있는 입장에 설 수가 있다.
\wi{타키투스}가 바로 그런 기회를 잘 활용했다.
하지만 <<게르마니아>>\latin{Germany}는 다른 많은 고전 저술과는 달리
저 저자의 모범적인 전례\hanja{前例}를 따르는 다른 사람들을 갖지 못했으며,
따라서 이런 종류의 전거로 우리에게 전해지는 것은 극히 적다.
문명화된 민족이 이웃의 미개인들에 대해 가지는 오만한 경멸심은
그들을 관찰함에 있어 뚜렷한 과실\hanja{過失}을 낳았거니와,
때로는 두려움으로 인해, 때로는 종교적 편견으로 인해,
때로는 바로 그 용어---사람들에게 그저 정도의 차이가 아니라 질적인 차이가
있다는 인상을 주는 `문명'\latin{civilisation}과 `미개'\latin{barbarism}라는
용어---의 사용에 의해서도
이러한 부주의는 가중되었다.
몇몇 비평가들에 의해
<<게르마니아>>조차도
날카로운 대비와 선명한 이야기를 위해 정확성을 희생시켰다는 의심을 받고 있다.
나아가, 자신들의 유년기를 말하고 있는 민족들의 고문헌 가운데
우리에게 전해지고 있는 역사 서술 또한,
민족의 자긍심에 의해 혹은 새 시대의 종교적 감정에 의해
적잖이 왜곡되었다고 평가되어왔다.
그런데, 근거있는 의심이든 근거없는 의심이든 이러한 의심이
대부분의 초기법에는 주어지지 않는다는 데 주목할 필요가 있다.
우리에게 전해지는 옛 법의 상당수는
단순히 옛 법이라는 이유로 보존되었다.
이 법을 적용하고 준수했던 이들은 그것을 이해한다고 내세우지 않았으며,
경우에 따라서는 비웃고 멸시하기까지 했다.
그들은 조상으로부터 전래되었다는 것을 제외하고는 그것을 설명하려 하지 않았다.
따라서, 가공되지 않았을 것이라고 합리적으로 믿을 만한 옛 법의 단편들에
주의를 집중하면, 우리는 그것들이 속했던 사회의 몇몇 중요한 성격에 대해
명료한 관념을 획득할 수 있게 된다.
한 걸음 나아가, 이렇게 얻은 지식을 \wi{마누법전}처럼
전체적으로
진정성이
의심되는 법체계에 활용할 수도 있을 것이다.
우리가 얻은 열쇠를 사용하여 진정으로 원시적인 부분과
편견, 이해관계, 무지 등에 의해 영향받은 부분을 분리해낼 수 있는 것이다.
이러한 탐구를 위한 자료가 충분하다면,
그리고 비교가 정확히 이루어진다면,
우리가 따르는 방법은 놀라운 결과를 이끌어낸 비교문헌학의 방법만큼이나
반대할 거리가 적을 것이라고 인정해도 좋을 것이다.

\para{가부장제 이론}
비교법 연구에서 나온 증거로부터
\wi{가부장제} 이론\latin{patriarchal theory}이라는
인류의 원시상태에 관한 이론이 수립되었다.
물론 이 이론은 원래 하\hanja{下}아시아\latin{Lower Asia} 지역%
\footnote{유프라테스강 이남의 아시아 지역을 말한다.}
히브리 민족의 가부장들에 관해 성서에 기록된 역사에 기초한 것이 틀림없다.
그러나 전술했듯이 성서와의 연계성은 그것이 완전한 이론으로
받아들여지는 데 방해가 되었다.
최근까지 사회현상들의 결합관계를 열성적으로 연구한 연구자들의 다수가
히브리 고대를 타기시하는 아주 강한 편견에 사로잡혀 있었거나
그들의 이론 체계를
종교 기록의 도움 없이
구성하고자 하는
아주 강한 욕망에 사로잡혀 있었기 때문이다.
지금도 성서의 기록을 폄하하는 경향,
보다 정확히는 그것을 일반화해서
셈족의 전통의 일부를 구성하는 것으로 인정하기를 거부하는
경향이 남아있는 듯하다.
하지만 중요한 점은 \wi{가부장제}의 법적 증거가
로마인, 인도인, 슬라브인이 대다수를 점하는
인도^^b7유럽 계통에 속하는 사회들의 제도에서 거의 전적으로 발견된다는 것이다.
실로 탐구의 현 수준에서 문제되는 것은 오히려 어디에서 그칠 것인가를 아는 것,
애초에 사회가 가부장제 모델에 입각하여 조직되었다고 할 수 \hemph{없는}
민족은 어떤 민족들인가를 확인하는 것에 있다고 할 것이다.
창세기 앞 부분 몇 개의 장에서 수집되는 가부장제 사회의 주요 특징들을
여기서 상세히 묘사할 생각은 없다.
우리 대부분은 어렸을 때부터 그것을 익히 들어왔기 때문이고, 또한
로크와 필머 간의 논쟁에서 그 명칭이 유래한 가부장권 논쟁에 대한
한때의 관심으로 인해
영국의 문헌들에서는
그것에 관해
별 쓸모도 없으면서
장\hanja{章} 하나를 통째로 할애하고 있기 때문이다.
저 역사의 표면에 드러난 요점은 이런 것들이다:
가장 나이 많은 남자 어른---가장 연장인 선조---이 그의 가\hanja{家}를
절대적으로 지배한다.
\wi{생사여탈권}을 포함하는 그의 지배권은 그의 노예들뿐만 아니라
그의 자식들과 가족들에 대해서도 무제한적이다.
사실 아들의 지위와 노예의 지위는 거의 차이가 없으며,
다만 아들은 언젠가는 가의 수장이 될 가능성이 더 크다는 점에서 더 높은 대우를
받을 뿐이다.
자식들의 양떼와 소떼는 아버지의 양떼와 소떼이며,
아버지의 재산---소유권자라기보다는 가의 대표자로서 점유하는 것인데---은
그의 사망시에 일촌\hanja{一寸} 자손들에게 균등 분배된다.
때로 먼저 태어났다는 이유로 장자가 두 배의 몫을 차지하기도 하지만,
일반적으로는 약간의 명목상의 우선권 외에는 장자가 상속에서 더 유리한
지위를 갖지 않는다.
조금은 덜 분명한 어떤 추론이 성서의 기록으로부터 나올 수 있겠는데,
가부장의 제국에서 이탈한 최초의 흔적을 엿볼 수 있을 듯하다.
야곱의 가족과 에서의 가족은 서로 분리되어 두 개의 민족을 형성하지만,
야곱 자식들의 가족들은 하나로 뭉쳐 하나의 인민이 되는 것이다.
이것은 국가 상태로 나아가는,
가족관계에 기초한 자격보다 권리에 기초한 자격이 우선하는 상태로 나아가는
때이른 맹아가 아닌가 한다.

\para{가족집단}
법학자로서의 특수한 목적을 위해서
역사의 여명기에 인류가 처해있던 상황의 특징을 간략하게 표현하고자 한다면,
나는 \index{호메로스}\hypertarget{cyclops}{호메로스}의 <<\wi{오디세이아}>>에서 몇 구절을 인용하고 싶다.
\begin{center}
  \greekfont\latinmarks
  \begin{tabular}{l}
    τοῖσιν δ᾽ οὔτ᾽ ἀγοραὶ βουληφόροι οὔτε θέμιστες\rlap{.}\\
    \hfill$\cdot$\hfill$\cdot$\hfill$\cdot$\hfill θεμιστεύει δὲ ἕκαστος\\
    παίδων ἠδ᾽ ἀλόχων, οὐδ᾽ ἀλλήλων ἀλέγουσιν.
  \end{tabular}
\end{center}
``그들에게는 회의하는 집회도 없었고 \wi{테미스테스}도 없었다. \ldots{}
하지만 그들 각자는 부인들과 자식들에 대해 재판권을 행사했거니와,
이에 관해 서로 아무런 간섭도 하지 않았다.''\footnote{\latin{Hom. Od. 9.112--115}}
이 인용문은 퀴클롭스에 해당하는 대목인데,
호메로스가
문명의 진보 수준이 낮은 외국인의 전형\hanja{典型}으로서
퀴클롭스를
묘사했다고 보아도 완전히 비현실적인 생각은 아닐 것이다.
원시공동체는 자신과 전혀 다른 풍속을 가진 사람들에 대해 느끼는
거의 본능적인 혐오감으로 인해 그들을 대개 거인 따위의 괴물로,
혹은 \paren{동양의 신화에서 거의 항상 등장하는} 악령으로
묘사하는 것이다.
어쨌거나, 저 싯귀에는 우리가 고대법에서 얻을 수 있는 힌트의 요지가 담겨있다.
처음에는 사람들이 완전히 고립된 집단들로 분산되어 있었고,
각 집단은 가부장에 대한 복종으로 결합되어 있었다.
가부장의 말이 곧 법이지만, 그것은 본서 제1장에서 분석한
테미스테스의 단계는 아직 아니다.
이 초기 법관념이 형성되기 시작하는 사회상태로 한 걸음 더 나아가면,
여전히
법관념은 전제\hanja{專制}적 가부장의 명령을 특징짓던 신비로움과 자연스러움을
어느 정도 가지고 있지만,
명령이 한 명의 주권자에서 비롯되는 것인 한,
그것은 가족집단들이 좀 더 넓은 범위의 조직으로 결합하는 것을
전제\hanja{前提}한다.
이제 문제는 이 결합의 성질이 무엇이냐, 그리고
결합에 따른 친밀성의 정도는 어떠하냐, 라는 것이다.
바로 여기서 고법\hanja{古法}이 우리에게 큰 기여를 하게 되거니와,
그렇지 않다면 단지 추측으로밖에 연결할 수 없는 틈새를 메워주고 있는 것이다.
어느 지역에서나 그것은 원시시대의 사회가 오늘날 우리가 상정하는
\hemph{개인들}의 집합이 아니었음을 보여주는 명백한 증거들로 가득하다.
사실, 그리고 그 구성원들의 견해에 의하더라도,
그것은 \hemph{가}\hanja{家}\hemph{들의 집합체}였다.
이 대비는
고대사회의 \hemph{단위}는 가족이었고
현대사회의 그것은 개인이라는 말에 의해서
자못 강력하게 표현된다.
우리는 이 차이가 가져오는 모든 결과를 고대법에서 기꺼이 발견할 수 있어야 한다.
고대법은 작은 독립된 단체들의 체계에 부응하도록 짜여져 있었다.
그리하여 고대법은 드문드문 규율할 뿐이니,
가부장들의 전제적 명령에 의해 보충될 것이기 때문이다.
고대법은 의례\hanja{儀禮}에 관한 것이니,
그것이 관심을 두는 관계는
개인들 간의 신속한 교섭보다는
국제 관계를 닮은 것이기 때문이다.
무엇보다, 고대법은 오늘날에는 볼 수 없는 사뭇 중요한 특성 하나를 가진다.
그것의 \hemph{생명}에 관한 견해는
발달된 법체계에서 보이는 것과는 완전히 다른 것이다.
단체는 \hemph{죽지 않는다}.
따라서 원시법은 그것이 다루는 대상, 즉 가부장적^^b7가족적 집단들을
영구적이고 소멸불가능한 것으로 취급한다.
이 견해는 먼 고대의 도덕적 태도에서 보이는 특정 측면과 밀접하게 관련되어 있다.
개인의 도덕적 상승과 도덕적 하락은 그 개인이 속한 집단의 공과\hanja{功過}와
동일시되거나 혹은 그보다 후순위로 밀렸다.
공동체가 죄\latin{sin}를 범하면
그것은 그 구성원들이 저지른 범죄의 총합보다 더 큰 죄가 된다.
범죄는 단체의 행위이고,
범죄의 결과는 직접 실행에 가담한 자들을 넘어 더 큰 범위에까지 미치는 것이다.
한편, 명백히 개인이 범죄를 저질렀다 할지라도,
그의 자식이나 친족이나 부족이나 동료시민들이 그와 함께,
때로는 그를 대신해서, 벌을 받는다.
그리하여 도덕적 책임과 보복의 관념은
더 진보된 시기보다는 먼 고대에 훨씬 더 확실하게 실현되었거니와,
가족집단은 불멸이었고 그것의 형벌 책임은 무제한이었기에,
원시적 정신은 개인이 집단으로부터 완전히 분리되면서 나타나는 곤란한 문제들에
구애받지 않았기 때문이다.
이러한 고대의 단순한 견해로부터 후대의 신학적^^b7형이상학적 설명의 방향으로
한 걸음 더 나아가면,
`\wi{저주의 상속}'\latin{inherited curse}이라는 초기 그리스의 관념을 만나게 된다.
최초의 범죄자로부터 그의 후손들이 상속받는 것은
형벌을 감수할 책임이 아니라,
합당한 보복을 초래할
새로운 범죄를 저지른 데 대한 책임이었다.
그리하여 가족의 책임은
범죄를 실행한 개인에게 범죄의 결과를 국한시키는 새로운 사고의 국면과
양립하게 되었다.

\para{친족이라는 의제}
전술한 성서의 사례가 제공하는 힌트만으로
일반적 결론을 얻는다면,
그리하여
가부장의 사망 후 가족이 분리되는 대신 하나로 단결하는 경우 어디서나
공동체가 존재하기 시작한다고 생각한다면,
그것은 사회의 기원에 관한 지나치게 단순한 설명이 될 것이다.
그리스의 대다수 국가들과 로마에서는
일련의 점증하는 집단들이 모여서 최초의 국가를 구성한
흔적이 오랫동안 남아있었다.
그것의 전형적인 예로서
로마의 가족\latin{family}, \wi{씨족}\latin{house}, 부족\latin{tribe}을
들 수 있거니와,
이들은 동일한 중심으로부터 점점 확장해가는 동심원들의 체계로
이해하지 않을 수 없다.
가장 기본되는 집단은 가족으로, 이는
제일 높은 남자 어른에게 함께 복종하는 관계로써 결합된다.
가족들의 집합이 씨족\latin{gens; house}을 형성한다.
씨족들의 집합은 부족을 만든다.
그리고 부족들의 집합이 국가를 구성한다.
이러한 예시를 따라 우리는 국가를
최초의 가족이라는 조상에서 유래한 공통의 후손들이 결합한
사람들의 집합체라고 이해해도 좋은 것일까?
이에 관하여 적어도 확실한 점은
모든 고대사회는 스스로를 공통의 계통에서 유래한 것으로 간주했을 뿐만 아니라,
이런 이유 외에는 정치적 결합체를 묶어주는
다른 어떤 이유도 생각할 수 없다고 믿고 있었다는 것이다.
사실, 정치적 관념의 역사는
피를 나눈 친족이야말로
정치적 기능에 있어 공동체의 유일한 근거라는 가정에서 출발한다.
어떤 다른 원리---가령 \hemph{지리적 근접성}\latin{local contiguity}
같은 것---가 역사상 최초로 공동의 정치행위의 근거로 등장함으로써
일어난 변화만큼 그렇게 놀랍고 그렇게 완전한 감정의 전복\hanja{顚覆}은,
강하게 표현하자면 혁명은, 이 세상에 존재하지 않는다.
따라서 초기 국가의 시민들은
그들이 소속된 모든 집단을 공통의 혈통에 기초한 것으로 간주했다고
인정해도 좋을 것이다.
가족에게 타당한 것은 우선 씨족에도, 다음으로 부족에도, 끝으로 국가에도 타당했다.
그런데
각 공동체는
이러한 믿음---혹은, 용어가 허용된다면, 이러한 이론---과 더불어,
이 근본가정이 틀렸음을 명확히 보여주는 기록이나 전승\hanja{傳承}들도
보존해왔다는 것을 우리는 알고 있다.
그리스 국가들을 보더라도, 로마를 보더라도,
니부르\latin{Barthold Georg Niebuhr}\footnote{팔림프세스트되었던
  가이우스의 <<법학제요>>를 발견해냈던 바로 그 사람. 본문의 내용은
  그의 주저 <<로마사>>에서 로마 씨족을 설명하는 과정에 원용된 사례들로 보인다.}%
에게 다양한 귀중한 사례들을 제공했던
디트마르쉔\latin{Ditmarsh} 지방의 튜턴족 귀족들을 보더라도,
켈트족의 \wi{씨족}연합을 보더라도,
최근에 관심의 대상이 된
슬라브족 러시아인과 폴란드인의 유별난 사회조직을 보더라도,
어디서나 우리는
이방인 혈통 사람들이 원주민 사회에 받아들여지고 서로 혼합되는
역사 기록의 흔적을 발견한다.
다시 로마로 눈을 돌리면,
\wi{입양}의 관행에 의해
기본 집단인 가족에 지속적으로
불순물이 섞이고 있었을 뿐만 아니라,
원래의 부족들 중 하나가 극적으로 축출당한 이야기라든가,%
\footnote{에트루리아 혈통의 왕과 그 일족이 축출되고 공화정이 수립된 사건을
말하는 듯하다.}
초기 왕들 중 하나에 의해 대규모로 씨족들의 편입이 이루어진 이야기\footnotemark\
따위가 항상 사람들의 입에 오르내리고 있었다.
자연적이라고만 여겨졌던 국가의 구성이 실은 다분히 인공적이었던 것이다.
\footnotetext{툴루스 호스틸리우스가 알바 롱가(Alba Longa)를 파괴하고
그 주민들을 로마로 이주시킨 사건을 말하는 듯하다.}%
믿음 혹은 이론과 명백한 사실 간의 이러한 불일치는 언뜻 보면
우리를 무척 당혹케 한다.
하지만 이것이 실제로 보여주는 바는
사회의 유년기에 있어 법적의제\latin{legal fiction}가 대단히 효율적으로 작동했다는 것이다.
가장 먼저 그리고 가장 널리 사용된 \wi{법적의제}는
가족관계를 인공적으로 만들어내는 것이었으니,
생각건대 이보다 더 강하게 인류가 빚지고 있는 것은 없다고 할 것이다.
이것이 존재하지 않았다면
어떤 원시집단이, 그 집단의 성격이 어떠하든,
어떻게 다른 집단을 흡수할 수 있었을 것이며,
또한 두 집단이,
한쪽이 절대적 우위를 점하고 다른 쪽이 절대적 종속에 들어가는 것을 제외하면,
어떻게 하나로 결합할 수 있었을 것인지
도무지 알지 못하겠다.
물론,
근대적 관점에서 두 공동체의 결합에 관해 생각할 때,
우리는 그것을 만들어낼 수 있는 수없이 많은 방법들을 제안할 수 있을 것이다.
가장 간단한 것으로는 통합하려는 집단의 구성원들이 투표를 하거나
아니면 지리적 근접성에 기초하여 함께 행동하는 것을 생각할 수 있다.
그러나,
단지 동일한 지리적 영역 안에 살고 있다는 이유만으로
다수의 사람들이 공동으로 정치적 권리를 행사한다는
관념은 원시적 고대에는 전적으로 생경하고 전적으로 기괴한 것이었다.
당시에 널리 애호된 방법은
편입되는 집단이 자신들을 편입하는 집단과 동일한 계통의 후손이라고
\hemph{가장하는} 것이었다.
이러한 의제를 당사자들이 얼마나 믿었는지,
이러한 의제는 그것이 모방하려는 현실과 얼마나 가까웠는지,
지금의 우리로서는 알 수 없다.
하지만 반드시 잊지 말아야 할 점은,
다양한 정치집단을 형성한 사람들은
그들의 연맹을 인정하고 축성\hanja{祝聖}하기 위해
주기적으로 함께 모여
공동의 제의\hanja{祭儀}를 개최하는 습속을
분명 가지고 있었다는 것이다.
원주민 사회에 혼합되어 들어간 이방인들도 이러한 공동제의에 참가했을
것임은 물론이거니와,
이것이 일단 행해지고 나면 그들을 공통의 혈통으로 인정하는 것이
쉬우면 더 쉬웠지 더 어렵게 되지는 않았을 것이다.
따라서 증거를 통해 얻을 수 있는 결론은,
모든 초기 사회가 동일한 조상에서 유래한 후손들로 구성되었다는 것이 아니라,
지속성과 공고함을 조금이라도 가진 모든 초기 사회는 그러한 후손들 또는
그러한 후손이라고 의제된 사람들로 구성되었다는 것이다.
무한한 수의 원인들이 원시집단들을 흩어지게 만들었을 것이나,
그들이 재결합할 때면 그것은 언제나 친족관계의 모델 혹은 원리에
기초하였다.
사실이야 어떻든 간에, 사상과 언어와 법은 무엇이든 이 가정에 적응하였다.
그러나,
우리에게 기록을 남겨준 공동체들에 관하여
이 모든 것이
타당하다 하더라도, 그들의 나머지 역사는
아무리 강력한 \wi{법적의제}라도 일시적이고 기한부의 영향력만 가진다는
전술한 약점을 안고 있는 역사였다.
역사의 어느 시기에 이르면---아마도 그들이 외부의 압력에 저항할 수 있을 만큼
강력해졌다고 믿기 시작하면서---이들 모든 국가는
혈족관계를 의제적으로 확장함으로써 인구를 충원하는 일을 그만두었다.
그리하여,
이제 공통의 기원이 아닌 다른 이유로 새로운 인구를 끌어들이게 된 곳이라면
어디서나 필연적으로 귀족정이 시작되었다.
실제적인 것이든 가상의 것이든
피로 연결된 관계가 아니고는 정치적 권리를 획득할 수 없다는
핵심원리를 완고하게 유지하던 그들이 이제
훨씬 더 생명력이 강한 것으로 드러난
어떤 다른 원리를
그들의 피지배자들에게
가르치기 시작했다.
그것은 바로 \hemph{지리적 근접성}의 원리였거니와,
이는 오늘날 어디서나 정치적 공동체의 필수조건으로 인정되고 있다.
동시에 새로운 정치관념이 들어서게 되었으니,
이 관념은,
우리 영국인들의 것이자 우리와 동시대 사람들의 것이며
다분히 우리 조상들의 것이기도 한 까닭에,
그것이 정복하고 폐위시킨 옛 이론에 대한 우리의 인식을 다소간
방해하고 있다.

\para{가부장권}
그리하여 가족은 초기사회의 있을 수 있는 모든 변이에도 불구하고
그것의 전형\hanja{典型}이었던 것이다.
하지만 여기서 말하는 가족은 현대의 우리가 이해하는 가족과 똑같은 것이 아니다.
고대적 관념에 접근하기 위해서는 현대의 관념을 한편으로는 확장하고
한편으로는 축소해야 한다.
고대의 가족은
외부인을 경계선 안으로 흡수함으로써
지속적으로 확장되는 것이었다.
\wi{입양}의 의제는 현실의 친족관계를 그대로 모방하는 것이었기에,
현실적 관계와 입양에 의한 관계 간에는
법적으로든 여론상으로든
거의 차이가 없었다.
한편, 이론적으로 가족의 일원으로 편입된 자는
기존 가족 성원들과 함께 그들의 가장 높은 살아있는 조상---아버지든,
할아버지든, 증조부든---에게
공통의 복종을 바치는 것으로써 사실상 하나로 결합된다.
가족집단이라는 관념에서 가장\hanja{家長}의 \wi{가부장권}은
그의 슬하에 태어났다는 사실\paren{또는 의제된 사실}만큼이나
필수불가결한 요소였다.
따라서, 아무리 진정한 핏줄로써 가족이 되었다 할지라도
가부장의 제국에서 사실상 퇴출된 사람은,
초기의 법에서는,
결코 가족의 일원으로 간주되지 않았다.
원시법의 초입에서 우리가 만나는 것은
이와 같이 가부장을 중심으로 한 집합체---현대의 가족보다
한편으로는 축소된 것이자 다른 한편으로는 확장된 것---인 것이다.
아마도 국가보다, 부족보다, 씨족보다도 더 오래된 이것은
씨족이나 부족이 잊혀진 이후에도 오랫동안,
혈연이 국가의 구성과 무관해진 이후로도 오랫동안,
사법\hanja{私法}에 그 흔적을 남기게 된다.
그것은 법의 모든 분야에 자신의 각인을 남겼으며,
생각건대 이들 법 분야의 가장 중요하고도 가장 지속력있는 성격의 다수가
그것에서 흘러나왔던 것이다.
초기의 대부분의 고대국가에서 보이는 법의 특징들로부터 불가피 도출되는 결론은,
오늘날 유럽 전역을 지배하고 있는 권리의무의 체계가 개인을 취급하고 있는 것과
정확히 동일한 관점에서 그때의 법은 가족집단을 취급했다는 것이다.
우리가 관찰할 수 있는 당시의 사회 가운데는
이러한 원시적 조건으로부터 생겨났다고 보아서는
설명하기 어려운 법과 관행을 가진 사회가 없지는 않다.
그러나 보다 운 좋은 상황에 놓여있던 공동체들에서는
법체계가 점점 분화되어 갔는데,
이러한 분화과정을 면밀히 살펴보면
우리는
그것이 법체계 중 가족이라는 원시적 관념에 의해 강하게 영향받았던 부분에서
주로 일어났음을 알게 된다.
무엇보다 중요한 사례인 로마법의 경우,
이러한 변화가 무척 천천히 일어났기에,
우리는 시대별로 이 변화의 경로와 방향을 관찰할 수 있을 뿐만 아니라,
그것이 결국 도달하게 될 결과까지도 어느 정도 짐작할 수 있을
정도이다.
방금 말한 이 탐구를 수행함에 있어 우리는 고대사회와 근대사회를
가르고 있는 가상의 장벽때문에 탐구를 중단할 필요가 없다.
로마의 세련된 법과 만족\hanja{蠻族}들의 원시적 관행이 혼합되어
나타난 결과의 하나, 즉 우리에게 \wi{봉건제}도라는 기만적인 이름으로 알려진 것은
로마 세계에서는 사라졌던 고법\hanja{古法}의 많은 특징들을 되살려낸 것이었으니,
이미 끝난 줄 알았던 분화과정이 다시 시작된 것이며 어느 정도는 지금도
진행되고 있는 것이기 때문이다.

\para{로마의 가부장권}
초기 사회의 가족조직은
후손들과 그들의 재산에 대해
아버지 등의 조상이 평생동안 행사하는 권력에 관해
몇몇 법체계에 뚜렷하고 폭넓은 자국을 남겼다.
이 권력을 나중에 로마인들이 이름붙인
`\wi{가부장권}'\latin{patria potestas}이라는 용어로 편의상 부르기로 하자.
원초적 인간관계의 특징 중에 이보다 더 많은 증거를 보여주는 것도 없지만,
\wi{진보하는 사회}의 관행으로부터 이보다 더 널리 그리고 더 빠르게
소멸해버린 것도 없을 것이다.
안토니누스 황조 시대에 활동했던 \wi{가이우스}는
가부장권이 로마에 특유한 제도라고 말한다.
물론, 만약 그가 라인강과 다뉴브강 건너편으로 눈을 돌려
당대 몇몇 사람들의 호기심을 자아내던 미개 부족들을 바라보았다면,
거기서 그는 조야한 형태의 가부장권의 사례들을 발견할 수 있었을 것이다.
멀리 동쪽에서는, 로마인이 갈라져나온 민족 계통과 동일한 줄기에서
뻗어나온 가지 하나가 로마의 가부장권을 몇몇 법기술적 측면에서
그대로 반복하고 있었다.
그러나
로마 제국 안에 포함된다고 생각되던 민족들 중에서
로마의 ``\wi{가부장권}''과 유사한 제도를 가진 민족으로
가이우스가
발견한 것은
아시아의 갈라티아인\latin{Galatae}을 제외하고는
전혀 없었다.\footnote{\latin{Gai.\,1.55.}}
사실, 생각건대,
왜 조상의 직접적 귄력이
다수의 진보하는 사회에서는
얼마 안 가서 초기 상태보다 미약해졌을까 하는 것에는
그럴 법한 이유가 있다.
버릇없는 자식이 아버지에게 맹목적으로 복종하는 것을
자식이 자신의 이해관계를 계산했기 때문이라고 설명해버리고 마는 것은
물론 부조리하기 짝이 없는 일이지만,
동시에, 자식이 아버지에게 복종하는 것이 자연스러운 일이라면,
자식이 아버지에게서 뛰어난 힘이나 뛰어난 지혜를 기대하는 것도 똑같이
자연스러운 일이다.
그리하여,
육체적 힘이나 정신적 능력에 특별한 가치를 부여하는 상황에 놓인 사회에서는,
그 보유자가 능력있고 힘있는 경우에만 한정하여
가부장권을
인정하려는 경향이 나타난다.
조직화된 그리스 사회를 처음 일별하여 얻은 우리의 인상에 의하면
아버지의 육체적 힘이 쇠약해졌더라도 아버지의 뛰어난 지혜가
그의 권력을 계속 유지시켜주는 듯하지만,
<<\wi{오디세이아}>>에 나오는
오디세우스\latin{Ulysses}와 라에르테스\latin{Laertes}의 관계처럼,%
\footnote{라에르테스는 아들 오디세우스가 트로이 전쟁에 참전하기 위해
이타케를 떠나기 전에 이미 통치권을 아들에게 물려주고 낙향했다.}
아들에게서 특별한 용기와 지혜를 모두 발견할 수 있다면
노쇠한 연령의 아버지는 가부장의 자리에서 물러났던 것이다.
후기 그리스 법에서는
호메로스의 문학에 암시된 관행에 더욱 진전이 이루어졌거니와,
비록 엄격한 가족적 의무의 흔적이 여전히 많이 남아있었지만,
이제 아버지의 직접적 권력은, 오늘날 유럽의 법전들처럼,
자식이 미성숙 또는 미성년인 경우에만, 다시 말해 그들의 정신적^^b7육체적
열등함이 추정되는 기간에만 국한되었던 것이다.
하지만 로마법은,
국가적 필요가 있는 때에 한해서만 고대 관행을 혁신하는 특이한 경향 덕택에,
원초적 제도와 내가 말한 그것의 자연적 한계, 양자 모두를 보존했다.
집합적 공동체에 관련된 생활관계에서는,
자문을 구하기 위해서든 전쟁을 치르기 위해서든
시민의 지혜와 힘을 이용하게 되는 그 어떤 경우에 있어서도,
가부장권에 복속해있는 아들\latin{filius familias}은 그의 아버지 못지않게
자유로웠다.
``\wi{가부장권}은 공법\latin{jus publicum}에는 미치지
않는다''는 것이 로마법의 법언이었다.
아버지와 아들은 시민으로서 함께 투표했고, 전장에서는 나란히 싸웠다.
사실, 장군인 아들이 아버지에게 명령을 내리는 것도 가능했고,
정무관인 아들이 아버지의 계약을 재판하거나
아버지의 범죄를 벌하는 일도 있을 수 있었다.
하지만 사법\hanja{私法}상의 모든 관계에서는
아들은 아버지의 전제권력 아래 살았으니,
끝까지 계속된 가부장권의 가혹함과
이 제도가 유지된 장구한 기간을 생각할 때,
이는 법제사에서 가장 이해하기 힘든 문제의 하나에 해당한다.

우리에게 원시 가부장권의 전형\hanja{典型}일 수밖에 없는
로마인들의 가부장권은
문명사회의 제도로서도
그 신분법적 측면에서나 그 재산법적 측면에서나
이해하기 어려운 제도이다.
역사의 이 틈새를 보다 완전히 메울 수 없음이 한스러울 따름이다.
\wi{신분법}적 측면에 관한 한,
우리의 정보 중 가장 빠른 것에 의할 때,\footnote{12표법을 말한다.}
아버지는 자식들에 대해 \wi{생사여탈권}\latin{jus vitae ncisque}을 가졌고,
게다가 무제한적인 체벌의 권한을 가졌다.
그는 자식들의 신분법적 지위를 마음대로 바꿀 수 있었다.
그는 아들에게 배우자를 정해줄 수 있었고, 딸을 \wi{혼인}시킬 수 있었다.
그는 아들이든 딸이든 자식을 이혼시킬 수 있었다.
그는 그들을 다른 가\hanja{家}에 \wi{입양}보낼 수 있었고, 그들을 팔아버릴 수 있었다.
나중에 제정기에 이르면 이 모든 권한은, 여전히 그 흔적이 남아있기는 했으나,
사뭇 좁은 범위로 축소된다.
가내 체벌에 관한 무제한적 권리는 이제
정무관에게 가내 범죄를 고발하는 권리가 되었다.
혼인을 명령할 특권은 조건적인 거부권 정도로 약화되었다.
자식을 팔아버릴 자유는 사실상 폐지되었다.
입양의 경우, 이제 양부에게 입적되는 아들의 동의가 없이는
무효가 되었으며, 후에 유스티니아누스의 개혁에 의해
입양의 고대적 성질은 거의 전부 사라져버린다.
요컨대, 바야흐로 근대 세계를 지배하고 있는 관념의 언저리에
가까이 다가가게 된 것이다.
하지만 이렇게 서로 멀리 떨어진 시기 사이에는 우리가 잘 모르는
중간 시기가 널리 펼쳐져 있으니,
보다 관용적인 모습을 띠어가면서도
로마의 \wi{가부장권}이
그렇게 오래도록 지속될 수 있었던
이유에 대해서는 추측만 할 수 있을 따름이다.
아들이 국가를 위해 부담하는 의무 중에서 가장 중요한 것들을
실제로 이행하는 것은 가부장권을 폐지까지는 아니지만 약화시키는 데
크게 기여하였을 것이 틀림없다.
고위 관직을 수행하고 있는 나이 많은 사람에 대해
아버지의 전제 권력이 아무런 스캔들 없이 행사될 수 있으리라고는
생각하기 어렵다.
그러나
초기 역사에서
사실상의 \wi{부권면제}\hanjalatin{父權免除}{emancipation}의 이러한 유형은
로마 공화국이 지속적인 전쟁상태에 놓임으로써 나타난 유형에 비해
상대적으로 드물었을 것이다.
초기 전쟁에서 장교들과 사병들은 일년의 4분의 3을 전장에 나가 있었고,
후대에 전직\hanja{前職}집정관\latin{proconsul}은 속주를 책임지고 있었고
군단병들은 속주를 점령하고 있었으므로,
그들은 스스로를 전제적 가부장의 노예로 생각할 이유가 사실상 없었다.
가부장권을 벗어나기 위한 이러한 통로들은 계속해서 늘어갔다.
승리는 정복을 낳고, 정복은 점령을 낳거니와,
식민도시의 건설을 통해 점령하던 방식이
상비군을 속주에 주둔시켜 점령하는 방식으로
변화되어갔다.
진보의 각 단계마다,
더 많은 로마 시민들이 변방으로 이주해야 했고,
격 낮은 라틴인의 피를 새로 수혈받아야 했다.
생각건대,
제정\hanja{帝政}의 확립과 더불어 세계의 평화가 시작되면서
가부장권의 약화를 지지하는 강력한 감정이 정착된 것이 아닐까 한다.
이 고대 제도에 대한 첫 번째 타격은 초기 황제들에 의해 가해졌거니와,
트라야누스와 하드리아누스의 단발적인 개입들은,\footnote{트라야누스는
아들을 가혹하게 다룬 아버지에게 그 아들을 \wi{부권면제}시키라고 명했다.
하드리아누스는 계모와 간통했다는 이유로 아들을 사냥 중에 살해한 아버지를
국외로 추방했다.}
비록 그 연대는 정확히 알 수 없으나
한편으로는 가부장권을 제한하고
다른 한편으로는 그것의 자발적 양도 가능성을 확대했다고 알려진
일련의 명시적 입법을 낳는 단초를 마련했던 것이다.
덧붙이건대, 아들을 세 번 매도하여 \wi{가부장권}을 벗어나게 하는 옛 방식은
가부장권의 불필요한 연장\hanja{延長}에 대한 반감을 보여주는
초기의 증거라 하겠다.
아버지에 의해 세 번 매각된 아들은 자유롭게 된다는 이 법규칙은
원래는
초기 로마인들의 불완전한 도덕관으로도
용납하기 어려운 관행에 대해 처벌적 결과를 부과한다는 의미를 가졌을 것이다.
그러나 12표법이 공표되기 이전에도,
법학자들은 창의력을 발휘하여,\footnote{여기서 `법학자들'은
공화정 후기에 등장하는 진정한 의미의 법학자라기보다는
신관단(collegium pontificum)으로 이해해야 할 것이다.}
아버지가 원하는 경우라면 가부장권을 소멸시킬 수 있는 수단으로
이 규칙을 전용\hanja{轉用}했다.

물론,
자식의 신분에 대한 가부장의 권력을 약화시키는 데 기여한 많은 원인들은
역사의 전면에 드러나지 않는다.
법이 수여한 권력을 여론이 얼마나 무력화시켰는지,
자연적 애정이 얼마나 그것을 지속시켰는지,
우리는 알지 못한다.
그러나, 비록 \hemph{신분}\latin{person}에 대한 가부장의 권력은 나중에 가서
명목적인 데 그치게 되지만,
현존하는 로마법 전체의 취지에 비추어 볼 때
아들의 \hemph{재산}\latin{property}에 대한 가부장의 권리는 언제나
법이 인정한 최대 범위까지 거리낌없이 행사되었다.
이 권리가 처음 등장했을 때 그것의 광범위함에는 놀라울 것이 전혀 없다.
로마 고법\hanja{古法}은 \wi{가부장권}에 복속해있는 자식이 가부장과 별도로 재산을
가질 수 없도록 금지했다. 아니, \paren{보다 정확히 말하자면}
자식들이 별개의 소유권을 가진다는 것을 상상조차 하지 못했다.
아버지는 자식이 취득하는 모든 것을 취득할 권리가 있었고,
자식이 맺는 계약으로부터 이익을 누릴 권리도 있으나
그에 상응하는 책임에는 얽혀들지 않았다.
초기 로마 사회의 구조를 감안할 때
이 정도는 우리가 예상할 수 있는 것들이다.
구성원들의 모든 수입이 공동 재산에 편입되지만
개인의 경솔한 계약이 공동 재산을 구속할 수는 없다는 것을
전제하지 않으면 우리는 원시 가족집단의 관념을 거의 이해할 수 없기 때문이다.
가부장권의 진정한 수수께기는 여기에 있는 것이 아니라,
가부장의 이러한 \wi{재산법}적 특권이 축소되는 데 그렇게나 오래 걸렸다는 것이며,
중대한 축소가 있기 전에 이미 문명 세계의 전부가 그 지배영역 안에
놓이게 되었다는 사실에 있는 것이다.
제정 시대 초창기 들어 비로소 어떤 혁신이 시도되기 시작하였으니,
군인들이 복무 중 취득한 재산이 가부장권의 통제를 벗어나게 된 것이다.
물론 이것은 공화정을 무너뜨린 군대에 대해 보상으로 주어진 것이라는 데서
일말의 이유를 찾을 수 있을 것이다.
그로부터 3세기가 지나, 동일한 면책특권은 국가 공직의 수행 중에 취득한 재산에도
확대되었다.
이 두 가지 변화는 명백히 가부장권의 적용범위를 제한하는 것이었으나,
되도록이면 가부장권의 원리를 훼손시키지 않으려는 법기술적 외관을 띠고 있었다.
일찍부터 로마법에는 어떤 제한적이고 종속적인 소유권이 인정되고 있었거니와,
가부장권 아래 있는 노예나 아들의 수입이나 비용절감으로서
가족재산에 포함시키지 않아도 되는 것들이 있었다. 이렇게 허용된 재산을
`\wi{특유재산}'\hanjalatin{特有財産}{peculium}이라는 특별한 이름으로 부르고 있었으니,
이 이름을 \wi{가부장권}에서 새롭게 면제된 재산에도 준용하여,
군인의 경우에는 `군영특유재산'\hanjalatin{軍營特有財産}{castrense peculium}이라
불렀고, 공직자의 경우에는
`준\hanja{準}군영특유재산'\latin{quasi-castrense peculium}이라
불렀던 것이다.
뒤이어 취해진 가부장의 특권을 변경시키는 다른 조치들은
고대 원리에 대한 존중이라는 외관을 훨씬 덜 갖춘 것들이었다.
준군영특유재산의 도입 직후,
콘스탄티누스 대제는 자식이 어머니로부터 상속받은 재산에 대한
가부장의 절대적 지배권을 박탈하여
일종의 \wi{용익권}\hanjalatin{用益權}{usufruct},
즉 생애 동안만 가지는 사용수익권으로 축소시켰다.
그후 서로마제국에서는 몇몇 대수롭지 않은 변화만이 이어졌으나,
동로마제국에서는 유스티니아누스 치세에 마지막 일격이 가해졌으니,
아버지로부터 유래한 재산이 아닌 한
자식의 재산에 대한 아버지의 권리는
아버지의 생애 동안 사용수익하는 권리에 불과한 것으로 축소되었던 것이다.%
\footnote{\latin{Inst.\,2.9.1.}}
로마 가부장권의 최종적 축소판이었던 이것조차도
현대 세계의 상응하는 제도에 비하면 훨씬 폭넓은 것이었고
훨씬 엄격한 것이었다.
근대 초기의 법학자들은 로마 제국을 정복한 민족 가운데
아주 사납고 미개한 민족만이, 특히 슬라브적 기원을 가진 민족만이,
학설휘찬\latin{Pandects}이나 칙법전\latin{Code}의 서술과 유사한
가부장권을 가지고 있었다고 말한다.
모든 게르만 이주민들은 가부장의 권력이라는 뜻의 `문트'\latin{mund}라고 불린
가족 단위 집합체를 인정하고 있었던 것으로 보인다.
그러나 그의 권력은 쇠퇴한 로마 가부장권의 유물에 지나지 않았고,
로마 가부장이 누렸던 권력에는 한참 못 미치는 것이었음에 틀림없다.
특히 프랑크족은 저 로마적 제도를 갖지 않았던 것으로 알려져 있거니와,
따라서 옛 프랑스 법률가들은,
만족\hanja{蠻族}의 관습의 빈틈을 로마법 규칙으로 열심히 메우고 있던 시절에도,
가부장권의 침입으로부터 자신들을 보호하기 위해
``가부장권은 프랑스에는 존재하지 않는다''\latin{Puyssance de père
en France n'a lieu}라는 법언을 내세우지 않으면 안 되었던 것이다.
고대적 상황의 유물인 이 제도를 유지함에 있어 로마인들의 보여준 고집스러움은
그 자체로 독특하지만, 이보다 더 독특한 것은
일단 사라져버린 가부장권이 문명세계 전체로 다시 확산되어 나간 사실이다.
군영\wi{특유재산}이 아직 가부장의 재산권에 대한 유일한 예외였던 시절,
자식들의 신분에 대한 가부장의 권한이 여전히 포괄적이었던 시절에,
로마 시민권이 가부장권과 더불어 제국의 구석구석으로 퍼져나간 것이다.
아프리카인이건, 에스파니아인이건, 갈리아인이든, 브리타니아인이든, 유대인이든,
누구라도 증여나 매수나 상속 등을 통해 시민권을 취득한 이는
로마의 \wi{신분법}\latin{law of persons}을 받아들였다.
또한, 남아있는 전거에 따르면, 시민권 취득 이전에 출생한 자식들은
그들의 동의 없이는 가부장권에 복속하지 않았지만,
시민권 취득 후에 출생한 자식들과 그 후손들은 모두 로마의
`\wi{가부장권}에 복속하는 아들'\latin{filius familias}이라는 통상적인 지위에 놓였다.
후기 로마 사회의 메커니즘을 규명하는 것은 본 저서의 영역을 벗어나는 일이지만,
한 가지 확실히 해두고 싶은 것은
안토니누스 카라칼라 황제가 제국 내의 모든 신민들에게 로마 시민권을 부여한
칙법이 별 중요성을 갖지 않는다는 의견은 근거가 박약하다는 점이다.
어떻게 해석하든 간에
그것은 가부장권의 적용대상을 대폭 확장하였으며,
생각건대 그것이 가져온 가족관계의 결속은
세상을 변혁시킨 도덕 혁명을 설명함에 있어
과거 우리가 생각했던 것보다 훨씬 더 중요한 인자\hanja{因子}였던 것이다.

이 주제에 관한 논의를 끝마치기 전에,
가부장은 자신의 권력 하에 있는 아들의 불법행위\latin{delict}에 대해
책임을 졌다는 점을 분명히 해둘 필요가 있다.
가부장은 노예의 불법행위에 대해서도 마찬가지로 책임을 졌다.
양자 모두의 경우에
원래
가부장은
불법행위를 범한 자를
손해배상을 대체하여
피해자에게 넘겨줄 수 있는 독특한 권리를 가지고 있었다.
이렇게 아들을 대신하여 지는 책임은,
아버지와 그 권력 하에 있는 자식 간에는
서로 소송을 제기할 수 없다는 점과 더불어,
이를 두고 몇몇 법학자들이
아버지와 그 권력 하에 있는 아들 간의
``인격의 통합''이라는 가설로써 잘 설명할 수 있다는
주장을 펼치도록 했을 것이다.
상속법에 관한 장에서 나는 이러한 ``통합''이 어떤 의미에서
그리고 어느 정도까지 현실로 받아들여졌는지 살펴볼 것이다.
지금 내가 말할 수 있는 것은
이러한 가부장의 책임이, 그리고 앞으로 다룰 다른 법현상들이,
원시 가부장의 \hemph{권리}에 상응하여 어떤 \hemph{의무} 또한
지시하고 있는 것으로 보인다는 점이다.
생각건대,
만약 가부장이 그의 가족들의 신분과 재산에 대한 절대권을 가지고 있었다면,
가족을 대표하는 그의 소유권은 모든 구성원을 위해 공동재산에서 책임을
부담하는 것과 궤를 같이하였을 것이다.
문제는 우리의 습관적 관념에서 벗어나야만 가부장의 이러한 의무의 성격을
제대로 이해할 수 있다는 것이다.
그것은 법적 의무가 아니었다. 법은 가족이라는 성역\hanja{聖域}에 아직
들어갈 수 없었기 때문이다.
그것을 \hemph{도덕적}이라고 부르는 것은 정신발달의 나중 단계에 속하는 것을
당겨쓰는 것이라 주저된다.
하지만 우리의 목적에는 ``도덕적 의무''라는 용어가 적합할 것이니,
명확한 제재가 아니라 본능과 습관에 의해 반\hanja{半}의식적으로
준수되고 강제되는 의무로 그것을 이해해야 할 것이다.

\para{종족과 혈족}
통상적인 형태의 가부장권은 일반적으로 오래 지속될 수 있는 제도가 아니었거니와,
내가 보기에도 그런 것 같다.
따라서 초기 \wi{가부장권}의 보편성의 증거는 그것 자체만 고려해서는 불완전하다.
하지만 고대법의 다른 분야들을 살펴봄으로써 입증은 계속될 수 있거니와,
그것은 궁극적으로는 가부장권에 의존하는 분야들이지만,
그 연관성이 모든 부분에 있어서 모든 사람에서 뚜렷이 보이는 것은 아니다.
가령 친족법으로,
즉 고법\hanja{古法}에서 사람 간의 관계의 근친성이 계산되는 척도에 관한 분야로
눈을 돌려보자.
여기서도 편의상 로마법상의 \wi{종족}\hanjalatin{宗族}{agnatic}관계와
\wi{혈족}\hanjalatin{血族}{cognatic}관계라는 용어를 사용하기로 하자.
\hemph{혈족}관계는 근대적 관념에 가까운 친족 관념이다.
그것은 동일한 한 쌍의 부부로부터 이어지는 공통의 후손들을 지칭하는데,
남자로 이어지든 여자로 이어지든 상관하지 않는다.
\hemph{종족}관계는 이것과 사뭇 다르다.
그것은 오늘날 우리가 친족이라 간주하는 다수의 사람들을 제외하며,
오늘날 우리가 친족에 포함시키지 않는 더 많은 사람들을 포함한다.
실로 그것은 아주 초기 고대에 가족 구성원들 간에 존재하는 관계였다.
그 관계의 범위는 근대적 관계의 범위와 사뭇 달랐다.

\wi{혈족}은 단일한 남자조상과 여자조상으로 혈연을 거슬러올라갈 수 있는
모든 사람들을 뜻한다.
혹은 로마법의 엄격한 법기술적 의미를 가져다쓴다면,
합법적으로 \wi{혼인}한 공통의 한 쌍의 조상으로
혈연을 추적할 수 있는 모든 사람을 뜻한다.
그리하여 ``혈족''은 상대적인 개념이 되거니와,
그것이 지칭하는 혈연관계의 범위는 계산의 출발점으로 선택된
특정한 혼인관계에 의존한다.
아버지와 어머니의 혼인에서 출발한다면, 혈족은 오직
형제자매의 관계만 표현하게 될 것이다.
할머니와 할아버지의 혼인에서 출발한다면, 아저씨와 아주머니와 그들의 후손들도
혈족 개념에 포함될 것이다.
이와 같이 출발점을 계속 윗대로 올라가며 선택하면 더 많은 수의 혈족들이
계속해서 포함될 것이다.
이 모든 것을 현대인들은 쉽게 이해한다.
하지만 종족은 어떠한가?
한마디로, 오로지 남자로만 이어지는 혈족이 \wi{종족}이다.
혈족의 가계도는 물론 각 혈통의 조상을 차례로 모두 추적하고
그 후손들을 남녀 불문하고 모두 포함하여 그려지거니와,
이러한 가계도의 여러 가지\latin{branch}들을 추적함에 있어
여자의 이름을 만날 때마다 가지의 추적을 중단함으로써
그 여자의 모든 후손들을 제외시켜버리면,
남는 사람들이 바로 종족이며 그들간의 연결이 바로 종족관계이다.
내가 혈족에서 종족을 분리하는 과정을 좀 자세히 말한 이유는 이것이
``여자는 가족의 종단\hanja{終端}이다''\latin{Mulier est finis familiae}라는
기억할 만한 법언을 잘 설명해주기 때문이다.
여자의 이름은 가계도의 가지를 마감한다.
여자의 후손은 누구도 원시적 가족관계의 개념에 포함되지 않는다.

우리가 고찰하는 고법\hanja{古法}이 입양을 인정하는 법체계라면,
가족 경계선의 이러한 인공적 확장에 의해 받아들여진 사람도,
남자든 여자든 상관없이, 모두 종족에 포함시켜야 한다.
하지만 그들의 후손은 우리가 좀 전에 기술한 조건을 충족하는 경우에만
종족이 될 수 있다.

\para{종족}
그렇다면 이러한 자의적인 포함과 배제의 이유는 무엇인가?
어째서 친족 관념은 \wi{입양}으로 가족에 받아들여진 이방인은 포함하면서
여자 구성원의 후손은 배제하는 탄력성을 보여주는가?
이 질문에 답하기 위해서는 가부장권 개념을 소환해야 한다.
\wi{종족}의 기초는 아버지와 어머니의 혼인관계가 아니라 아버지의 권력이다.
동일한 \wi{가부장권}에 복속하고 있는 사람들,
그 가부장권에 복속했었던 사람들, 또는
혈통상의 조상이 가부장권을 행사할 만큼 오래 살았다면 그 가부장권에
복속했을 사람들, 이들은 모두 종족인 것이다.
원시적 관념에 의하면 실로 친족관계는 바로 가부장권에 의해 정해진다.
가부장권이 시작하는 곳에서 친족관계가 시작한다.
따라서 입양에 의한 관계도 친족관계인 것이다.
가부장권이 끝나는 곳에서 친족관계도 끝난다.
따라서 아버지에 의해 \wi{부권면제}된 아들은 종족으로서의 권리를 상실한다.
왜 여자의 후손들은 원시적 친족의 경계선 바깥에 있었는지도
여기서 그 이유를 찾을 수 있다.
여자가 혼인하지 않고 죽으면, 그녀는 합법적 후손을 가질 수 없다.
그녀가 혼인하면, 그녀의 자식들은 그녀의 아버지가 아니라 남편의 가부장권에
복속하므로 그녀 자신의 가족에 속하지 않는 것이다.
만약 사람들이 어머니의 친족들까지 자신의 친족이라 불렀다면,
원시사회의 구조는 매우 혼란스러웠을 것임에 틀림없다.
이는 한 사람이 두 개의 가부장권에 복속하는 결과를 가져왔을 것이며,
두 개의 가부장권은 두 개의 재판권을 의미하거니와,
둘 모두에 순종하는 자는 동시에 두 개의 서로 다른 체제 하에서 살아가는 결과가
되기 때문이다.
통치권 안의 통치권\latin{imperium in imperio}이자
국가 안의 공동체인 가족이
가부장을 원천으로 하는 자신만의 제도로 통치되는 것이라면,
친족관계를 종족으로 국한하는 것은
가내법정\hanja{家內法廷}에서 법의 충돌\latin{conflict of laws}을 방지하는
필수적 안전장치였던 것이다.

진정한 의미의 가부장권은 가부장 지위의 죽음으로 소멸한다.
하지만 종족은 \wi{가부장권}이 사라진 후에도 그것의 각인을 담고있는
일종의 주형\hanja{鑄型} 같은 것이다.
따라서 법제사 연구자에게 종족은 자못 흥미를 불러일으키는 주제이다.
가부장권 자체는 비교적 소수의 기념비적인 고대법체계에서만 발견되지만,
과거에 가부장권이 존재했었음을 암시하는 종족관계는
거의 어디서나 발견할 수 있는 것이다.
인도^^b7유럽 계통의 공동체에 속하는 법체계로서
그들의 사회구조의 아주 고대적인 부분에서
종족이라고 부를 만한 특징을 보여주지 않는 법체계는 거의 없다.
가령
원시적인 가족적 위계관계 관념으로 가득차있는
힌두법의 친족관계는 전적으로 종족적인 것이며,
내가 알기로 인도인의 가계도에서 일반적으로 여자의 이름은 아예 제외된다.
로마 제국을 침략했던 민족들의 법에서도
친족관계에 관한 동일한 견해가 다수 발견되고 있어
실제로
그들의 원시 관행의 일부를 이루었을 것으로 짐작되거니와,
만약 후기 로마법이 근대법에 끼친 막대한 영향력이 없었더라면
그들의 법은 지금 우리가 보는 것보다 훨씬 더 많이 존속했을 것이다.
일찍이 \wi{법무관}들은 \wi{혈족}을 \hemph{자연적} 친족관계로 파악하여
로마법체계를 옛 관념으로부터 정화시키는 데 수고를 아끼지 않았다.
그들의 관념이 오늘날 우리에게 전해지지만,
종족의 흔적도 다수의 근대 상속법에서 여전히 발견되고 있다.
흔히 프랑크족의 일파인 살리족\latin{Salian Franks}의 관행에서
기인한다고 여겨지는,
여자와 그 후손들을 통치기능에서 제외하는 상속법은
확실히 \wi{종족}적 기원을 가진 것으로,
\wi{자유소유지}\latin{allodial property}의 상속에 관한
고대 게르만법에서 유래한 것이다.\footnote{%
  `자유소유지'란 봉건적 부담을 지지 않는 부동산을 말한다.
  그런데 본문의 설명은 부정확하다.
  렉스 살리카(Lex Salica) 제59장에 의하면
  가족재산으로서 자유소유지는 여자도 상속할 수 있었다.
  여성의 상속이 배제되는 것은 자유소유지가 아니라
  `테라 살리카'(terra Salica)라 불린 은대지(恩貸地 benefice) 따위의 토지였다.
  물론 메로빙거 왕위의 상속은 테라 살리카의 규칙에 따랐다.
  %\latin{Katherine Fischer Drew,
  %\textit{The Laws of the Salian Franks},
  %Phliladelphia: University of Pennsylvania Press, 1991, pp.\,43f.}
  }
최근에야 폐지된,
%\footnote{Inheritance Act 1833. }
한쪽 부모만 같은\latin{half-blood} 형제 간의 토지 상속을 금지한
특수한 영국법 규칙도 \wi{종족}에 기초하여 설명할 수 있을 것이다.
노르만의 관습에 의하면
이 규칙은 \hemph{어머니만 같은}\latin{uterine} 형제,
즉 아버지가 다른 경우에만 적용되고
아버지가 같은 형제 간에는 적용되지 않는다.
이렇게 본다면 이 규칙은 종족 개념에서 연역된 것이 틀림없으니,
어머니만 같은 형제는 서로 간에 종족이 아니기 때문이다.
이 규칙이 영국에 이식되었을 때,
그 배경 원리를 이해하지 못한 영국의 판사들이
한쪽 부모만 같은 형제 간의 상속이 일체 금지된다고 해석함으로써
\hemph{아버지만 같은}\latin{consanguineous} 형제,
즉 아버지는 같지만 어머니는 다른 아들들에까지 확대적용했던 것이다.
자칭 법철학이란 것을 담고 있는 문헌들 가운데,
\wi{블랙스톤}의 저서 중
한쪽 부모만 같은 형제 간의 상속 금지를 설명하고 정당화하려 한,
정교한 궤변들로 가득찬 페이지들보다 더 이상한 것은 없을 것이다.

\para{여성 후견}
가부장에 의해 통합되어 있는 가족이야말로
신분법 전체가 발달되어 나온 모태였다고 생각한다.
\wi{신분법}의 여러 장\hanja{章}들 중에
가장 중요한 것은 여성의 지위에 관한 것이다.
좀 전에 말한 것처럼,
원시법에 따르면,
여자는 자신의 후손에게 \wi{종족}의 권리를 전해줄 수는 없지만,
그래도 그녀 자신은 종족관계에 포함된다.
사실, 어떤 여성이 자신의 태어난 가족과 맺는 관계는
그녀의 남자친족들 간의 결합 관계보다
훨씬 더 엄격하고 친밀하고 또 지속적이다.
누차 말했듯이,
초기법은 가족 외에는 알지 못한다.
이는 곧 초기법은 \wi{가부장권}을 행사하는 사람 외에는 알지 못한다는 말과 같다.
따라서 가부장의 사망으로 아들이나 손자가 해방되는 유일한 이유는
그 아들이나 손자에게 내재해있는,
스스로 새로운 가족의 수장이 되고
장차 새로운 가부장들의 뿌리가 될 수 있는
능력을 고려해서인 것이다.
그러나 여자는 이런 종류의 능력을 갖지 못함은 물론이요,
그런 능력에 수반되는 해방될 수 있는 자격도 갖지 못한다.
그리하여 고법\hanja{古法}에서는
그녀를 평생동안 가족의 속박 아래 두기 위한 특유한 장치가 있었으니,
초창기 로마법에서
`여자의 영구적 \wi{후견}'으로 불리던 제도가 그것이다.
이에 따르면 여성은,
비록 아버지의 사망으로 그의 가부장권에서는 벗어나지만,
최근친 남자 친족 또는 아버지가 지명한 자의 후견에 의해
평생동안
복속이
계속된다.
영구적 후견 제도는
다른 경우라면 해소되었을
가부장권의 인위적 연장, 그 이상도 이하도 아니다.
인도에서는 이 제도가 완벽하게 살아남아
아주 엄격하게 작동하는지라, 인도의 어머니는
자신의 아들의 후견을 받는 경우가 흔하다.
유럽에서도,
스칸디나비아 민족들의
여성에 관한
법은 최근까지도 이 제도를 유지해왔다.
서로마제국을 침공했던 민족들의 토착 관행에서도 이것은 보편적으로
발견되거니와, 실로 후견에 관한 그들의 관념은, 그 모든 형태에도 불구하고,
서구 세계에 도입된 모든 관념 중에서 가장 퇴행적인 것이었다.
그러나 성숙한 로마법에서는 이 제도가 완전히 사라졌다.
만약 유스티니아누스 법전만을 참조한다면
우리는 그것을 전혀 알아채지 못할 것이다.
그러나 \wi{가이우스}의 <<\wi{법학제요}>> 필사본이 발견됨으로써
이 제도의 가장 흥미로운 시기,
즉 완전히 불신되어 사라지기 일보직전 시기의
모습이 우리 앞에 나타났다.
저 위대한 법학자는,
이 제도를 옹호하는 통속적인 변명인
여성의 지려박약\hanja{智慮薄弱}을 말하고 있기는 하지만,\footnote{%
  \latin{Gai.\,1.190.}}
여자들이
고대 규칙을 깨뜨릴 수 있도록
로마 법률가들이 고안해낸, 때로는 비상한 독창성을 보여주는,
수많은 장치들을
저서의 상당 부분을 할애하여
서술하고 있다.
당시의 법학자들은,
자연법 이론에 기초하여,
그들의 형평법 법전의 원리의 하나로
양성평등을
받아들인 것이 확실하다.
그들이 공격한 지점이 재산 처분에 대한 제한,
즉 여전히 형식적으로는 여자의 \wi{후견}인의 동의가 요구되던
제한이었다는 점은 주목할 만하다.
여성의 신분\latin{person}에 대한 통제는 이미 옛날 일이었던 것이다.

\para{고대 로마의 혼인, 여성의 지위}
고대법은 여자를 그녀의 혈연 친족에게 종속시키지만,
근대법의 주요 특징의 하나는 그녀를 남편에게 종속시키는 것이다.
이 변화의 역사는 주목할 가치가 있다.
그것은 로마 연대기의 저 멀리까지 거슬러올라가 출발한다.
아주 옛날의 로마 관행에 따르면
\wi{혼인}이 체결되는
방식에는
세 가지가
있었다.
하나는 종교적 엄숙함에 기초한 것이고,
나머지 둘은 어떤 세속적 방식을 준수하는 데 기초했다.
종교적 혼인인 콘파레아티오\latin{confarreation},
세속적 혼인의 고상한 형태인 코엠프티오\latin{coemption},
세속적 혼인의 통속적 형태인 우수스\latin{usus},
이것들에 의해 남편은 아내의 신분과 재산에 대한 다수의 권리를 취득했거니와,
이는 대체로 그 어떤 근대법에 의해 남편에게 주어지는 것보다도 훨씬 큰 것이었다.
그런데 남편은 어떤 자격에서 이런 권리를 취득하는 것이었을까?
그것은 \hemph{남편}으로서가 아니라 \hemph{아버지}로서였다.
콘파레아티오, 코엠프티오, 그리고 우수스에 의해
여자는 남편의 수권\hanja{手權}에\latin{in manum viri} 넘겨졌으니,
법적으로는 남편의 \hemph{딸}이 되는 것이다.
그녀는 남편의 \wi{가부장권}에 복속했다.
남편의 가부장권이 존속하는 동안은 그로부터 생겨나는 모든 책임을 부담했고,
그것이 종료된 후에는 그것이 남겨놓은 모든 책임을 부담했다.
그녀의 모든 재산은 전적으로 남편의 것이 되었으며,
남편이 죽고 나서도 그가 유언으로 지명한 후견인의 \wi{후견}에 놓였다.
하지만
저 세 가지 고대적 혼인 형태는 점차 안 쓰이게 되었거니와,
그리하여 로마가 가장 융성했던 시절에는
세속적 혼인의 통속적 형태가 변형되어 생겨난 어떤 혼인 방식---분명
오래된 것이나, 그때까지는 그다지 존중되고 있지는 않던 것---에 의해
거의 완전히 대체되었다.
새로이 널리 대중화된 제도의 법기술적 메커니즘에 대해서는 자세히 다루지
않겠으나, 다만 그것이 법적으로는
여자의 가족이 여자를 일시적으로 맡기는 것 정도에
지나지 않았음을 언급해두고 싶다.
여자 가족의 권리는 그대로 유지되었고,
부인은 그녀의 부모가 지명한 후견인의 후견에 계속 놓여있었으며,
이 후견인의 통제권은
여러 주요 측면에 있어
남편의 권력보다 훨씬 큰 것이었다.
결과적으로 로마 여성의 지위는
혼인 여부를 불문하고
\wi{신분법}적으로나 \wi{재산법}적으로나 사뭇 독립적인 것이 되었다.
앞서 언질을 준 것처럼, 이후의 법발달에 의해
후견권은 거의 없는 것이나 진배없는 수준으로 축소되었고,
그에 비해 저 대중적 혼인 방식은 남편에게 그에 상응하여
권력을 더 부여하지 않았기 때문이다.
그런데 기독교는 거의 처음부터 이러한 비범한 자유를 축소시키는 경향이 있었다.
처음에는 쇠퇴해가는 이교도 세상의 방탕한 풍속에 대한 그럴 만한 혐오에서,
나중에는 금욕주의의 열광에서 나온 조급함에서,
새로운 신앙의 추종자들은 서구 세계가 보여준 것들 중
사실상 가장 느슨한 \wi{혼인}관계를 자못 불쾌한 눈으로 바라보았다.
기독교도 황제들의 칙법에 의해 수정된 최후의 로마법은
안토니누스 황조 시대의 위대한 법학자들의 자유주의적 법리에 대한
반동을 다분히 담고 있었다.
또한 종교적 감정의 고양된 상태는
만족\hanja{蠻族}들의 정복의 용광로에서 벼려지고
가부장제 관행의 로마법이 융합되어 형성된
근대법이
어째서
저 원초적 규칙들 중에서도
여성의 지위에 관해서는
덜 발달된 문명의 특징에 속하는 규칙들을 기대 이상으로 많이 흡수했는지를
설명해줄 수 있을 것이다.
근대 역사가 시작되던 혼란기에,
게르만족과 슬라브족 이주민들의 법이
그들의 지배를 받는 로마인들의 법 위에 또 하나의 층으로 따로 존재하던 시기에,
지배층 민족의 여자들은 어디서나
다양한 형태의 원시적 \wi{후견} 아래 놓여있었고,
다른 가족으로부터 아내를 취하는 남편은
후견권을 사오는 대가로
그녀의 친족들에게
신붓값을 지불했다.
시간이 흘러 중세의 법전에서 두 법체계의 융화가 이루어졌을 때,
여자에 관한 법은 저 두 가지 기원의 각인\hanja{刻印}을 모두 담고 있었다.
혼인하지 않은 여성에 관해서는
로마법의 원리가
지배적이어서,
일반적으로 \paren{지역에 따라 예외는 있으나}
그들은 가족의 속박으로부터 벗어나 있었다.
그러나
혼인한 여성에 관해서는
만족\hanja{蠻族}들의 옛 원리가 지배하였고,
종래 아내의 남자 친족들에게 속했던 권력을 남편이 차지하게 되었거니와,
다만 이제는 대가를 지불하고 권리를 사오는 일이 없어진 점이 다를 뿐이다.
그리하여 이 시점의 남유럽과 서유럽의 근대법은 그 주요 특징 중 하나를
뚜렷이 나타내기 시작하니,
미혼의 여자나 과부에게는 상대적으로 자유가 허용된 반면,
아내들에게는 무능력이 무겁게 부과되었던 것이다.
혼인한 여성에게 부과된 종속이 눈에 띄게 감소하는 것은 오랜 시간이 흐른 뒤이다.
유럽에서 부활한 미개함을 부드럽게 만든 강력한 주된 용매는
언제나 유스티니아누스 법전이었으니,
이것이 불러일으킨 열성적 연구가 행해진 곳에서는 어디서나 그러했다.
그것은
단지 옛 관습을 해석할 뿐이라고 내걸었지만 실제로는
은밀하게 그러나 효과적으로 저 관습을 약화시켰다.
하지만 혼인한 여성에 관한 법은 대체로 로마법보다는 \wi{교회법}의 조명을 받았다.
\wi{혼인}으로 형성된 관계에 대한 견해만큼
교회법이
세속법의 정신에서
멀리 떨어진 것은 없었다.
이것은 어느 정도 불가피한 일이었으니,
기독교 제도의 색채를 보유한 사회 중에
혼인한 여자에게 중기 로마법이 부여했던 신분상의 자유를 회복시켜 줄 사회는
없었던 것이다.
그러나 혼인한 여성의 \wi{재산법}상 무능력은 \wi{신분법}상 무능력과는 전혀 다른
기초 위에 서 있었는데,
전자를 계속 유지하고 강화하는 법리를 통해
교회법 해설자들은 문명 발달에 깊은 상처를 주었다.
세속법과 교회법 간의 갈등의 흔적은 여러 곳에서 발견되지만,
거의 어디서나 교회법이 승리를 거두었다.
프랑스의 일부 지방에서
귀족 아래 계급의 혼인한 여자들은
로마법이 인정했던 재산에 관한 모든 귄한을 가지게 되었으니,
이 지방법은 나폴레옹 법전에 대체로 수용되었다.
하지만 스코틀랜드법의 상태는
로마 법학자들의 법리에 대한 우직한 존경이
반드시 아내들의 무능력을 완화하는 데 기여하는 것은 아님을 잘
보여준다.
\wi{혼인}한 여성에게 가장 덜 관대한 법체계는 모두
교회법을 배타적으로 추종했던 법체계들이거나,
아니면
유럽 문명과의 접촉이 늦어 자신들의 고대적 유물을 솎아낼 기회가 없었던
법체계들이었다.
수 세기 동안 모든 여성들을 가혹하게 대해왔던 덴마크법과 스웨덴법은
여전히 아내들에게는
일반적인 유럽 대륙의 법전들보다
훨씬 덜 우호적이다.
그러나 \wi{재산법}적 무능력에 있어 대륙법보다 더 엄격한 것이
영국 보통법이거니와,
그것은 \wi{교회법}으로부터 근본원리를 대량으로 빌려왔던 것이다.
실로 혼인한 여성의 법적 지위에 관한 보통법은
본 장의 주제인 저 중차대한 제도에 대한 명료한 관념을
영국인들에게 심어줄 수 있을 법하다.
일체의 권리와 의무와 구제수단을 통하여,
형평법이나 제정법의 손을 타지 않은
순수한 영국 보통법이
남편에게 수여한 대권\hanja{大權}을 돌아보는 것만큼,
아내의 완전한 법적 종속을 관철시킨 엄격한 일관성을 회상하는 것만큼,
고대 가부장권의 성질과 작동을 생생하게 보여주는 것이 또 있을지 의문이다.
보통법법원과 형평법법원이 아내들에게 각각 적용하는 규칙들 간의 차이는
가부장권에 복속하는 자식에 관한 가장 이른 로마법과 가장 늦은 로마법 간의
거리만큼이나 멀다고 해도 좋을 것이다.

\para{후견제도}
두 가지 형태의 후견제도의 진정한 기원을 무시한다면,
그리고 두 가지 주제에 대해 동일한 언어를 사용한다면,
고법\hanja{古法}체계들에서
여자의 후견\latin{tutelage}은
권리의 정지라는 의제\hanja{擬制}를 지나치게 길게 밀고 나간 사례인 반면,
`아버지 없는\latin{orphan} 남성의 \wi{후견}'이라 불리는 것은
정확히 그 반대방향의 과오를 보여주는 사례라고 말하는
우리 자신을 발견하게 될 것이다.
이들 법체계에서 남성 후견은 상당히 이른 시기에 종료된다.
그것의 전형이라 할 수 있는 고대 로마법에서는
아버지나 할아버지의 사망으로 \wi{가부장권}에서 벗어난 아들은
일반적인 경우 15세에 도달할 때까지 후견에 놓이게 되고, 일단
그 연령에 도달하면 \wi{신분법}적으로나 \wi{재산법}적으로나 완전히 독립한다.
따라서 미성년의 기간은 지나치게 짧고
여성의 무능력 기간은 지나치게 긴 것으로 보인다.
하지만, 실은,
두 종류의 후견에 최초의 형태를 부여한 상황을 감안하면
지나치게 길거나 지나치게 짧은 요소는 전혀 없다.
어느 쪽에든 공적 또는 사적 편의성의 고려는 조금도 들어있지 않다.
여자의 후견이 여성의 취약함을 보호하기 위해 도입된 것처럼,
본디 아버지 없는 남성의 후견도
결정권을 행사하는 시기에 도달할 때까지 그들을 보호하기 위해 고안된 것이다.
아버지의 사망으로 아들이 가족의 속박에서 벗어나는 이유는
새로운 가족의 수장이 되고 또 다른 새로운 가부장권의 기초자가 되는
아들의 능력에 있는 것이다.
이에 비해 여자는 이러한 능력을 가지지 않으며, 따라서
\hemph{결코} 해방되지 않는다.
그리하여 아버지 없는 남성의 후견은
자식이 스스로 아버지의 능력을 행사할 수 있을 때까지
아버지의 가\hanja{家}에 복속되는 모양새를 유지하는
장치였다.
그것은 신체적 남성성이 발현되기까지의 일종의 가부장권의 연장이었다.
성숙기\latin{puberty}가 도래하면 \wi{후견}은 끝나거니와,
이러한 이론이 그것을 엄중하게 요청하기 때문이다.
하지만 이는 아버지 없는 남성 피후견인을
지적으로 성숙하거나 거래에 적합한 나이에
도달하기까지 보호한다는 말은 아니기 때문에
일반적 편의성의 목적에는 전혀 미치지 못한다.
로마인들은 사회진보의 무척 이른 시기에 이 문제를 인지했던 것 같다.
아주 오래된\footnote{기원전 3세기 말 또는 2세기 초.}
기념비적인 로마 입법으로 라이토리우스 법\latin{Lex Laetoria}
혹은 플라이토리우스 법\latin{Lex Plaetoria}이라 불리는 것이 있다.
이로써 나이가 차고 완전한 권리를 가진 모든 자유인 남성을
\wi{보좌}인\latin{curator}이라 불리는
새로운 종류의 \wi{후견}에 의한
일시적 통제 하에 두게 되었으니,
그들의 단독행위와 계약이 유효하기 위해서는 보좌인의 승인이 요구되었다.
젊은이의 나이가 26세에 이르면 이 제정법상의 감독권은 종료되었다.
그리하여 로마법에서는 ``성년''과 ``미성년''을 가르는 기준으로
25세가 전적으로 사용되었던 것이다.
미성숙\latin{pupilage} 또는 피후견\latin{wardship}의 신분은
근대법에서는
법적안정성을 적절히 고려하면서
미성년자의 신체적 미성숙과 정신적 미성숙을 모두 보호한다는
단순한 원리로 변용되었다.
결정권을 행사할 수 있는 연령이 그것의 자연적 종료시기이다.
그러나 신체적 취약함의 보호와 지적 무능력의 보호를
로마인들은 두 가지 서로 다른 제도에 맡겼으니, 그 둘은
이론에서도 의도에서도 서로 구분되는 것이었다.
양 제도에 따라붙는 관념들이 현대의 후견 관념에서는 통합되어 있다.

\para{노예제도}
우리의 목적을 위해서 원용할 만한 또 하나의 \wi{신분법} 제도가 남아있다.
\hemph{주인}과 \hemph{노예}의 관계에 대해 성숙한 법체계가 가진 법규칙들은
고대 사회에 공통된 원초적 상황의 흔적을 그다지 뚜렷이 보여주지 못한다.
그러나 여기에는 그럴 만한 이유가 있다.
아무리 성찰의 습관이 형성되지 못했다 할지라도,
이무리 도덕적 본능의 함양이 낮은 수준에 머물러 있다 할지라도,
시대를 막론하고 인류에게 충격을 주고 인류를 당혹케하는 어떤 것이
\wi{노예제}도에는 들어있는 듯하다.
고대 공동체들이 거의 무의식적으로 겪었던 양심의 가책이
노예제도를 그럴 듯하게 방어하는, 적어도 정당화하려는,
어떤 가상의 원리를
반드시
채택하도록 만들었던 것 같다.
그들의 역사 초기에 그리스인들은
이 제도를
특정 민족의 지적 열등함에 기초하는 것으로,
따라서 노예상태에의 자연적 적합성에 기초하는 것으로
설명했다.
마찬가지로 독특한 정신의 소유자인 로마인들은
승자와 패자 간의 가상의 합의에서 그것을 이끌어냈으니,
전자는 그의 적의 영구적 서비스를 청약하고
후자는 승낙의 대가로 합법적으로 몰수당한 목숨을 구한다는 것이다.
이들 이론은 건강하지 못한 이론일 뿐만 아니라
그들이 설명하고자 하는 사태에도 전혀 부합하지 않는 것이었다.
그럼에도 불구하고 이들 이론은 여러 모로 강력한 영향력을 행사했다.
그것들은 주인의 양심을 만족시켰다.
그것들은 노예의 낮은 지위를 영속시켰고 어쩌면 더 가중시켰다.
그것들은 노예가 가\hanja{家}의 나머지 부분과 원래 어떤 관계를 가졌는지를
무시하게 만들었다.
이 관계는, 비록 명확히 드러나는 것은 아니나,
고대법의 여러 부분에서 산발적으로 징후를 나타내고 있거니와,
특히 그 전형적인 법체계---고대 로마법---에서 그러하다.

미국에서는
초기 사회에서 노예가 가족의 구성원으로 인정되었는가에 대해
많은 연구와 학습이
수행되었다.
이에 대해 반드시 긍정적 답이 주어져야 한다는 생각이 존재한다.
고대법과 초기 역사의 증거에 의하면
특정 상황에서는 노예가 주인의 상속인, 즉 포괄승계인으로 지정될 수 있었음이
분명하다.
이러한 의미심장한 능력이 뜻하는 바는,
상속법에 관한 장에서 설명하겠지만,
가\hanja{家}의 통치와 대표를 특정 상황에서는 노예에게 맡길 수 있다는 것이다.
하지만 이 주제에 관한 미국의 논의에서 상정되고 있는 것은,
만약 \wi{노예제}도가 원시적 가족제도의 일부였다면,
오늘날의 흑인노예제 역시 도덕적으로 정당화할 수 있다고 인정해야
한다는 것으로 보인다.
그렇다면 노예가 원래 가족의 일원이었다는 것은 무엇을 의미하는가?
인간을 부추기는 저속한 동기 때문에 노예제가 생겼다는 뜻이 아니다.
나의 안락과 쾌락을 도모하는 수단으로 타인의 육체적 힘을 사용하고자 하는
단순한 희망은 물론 노예제의 토대이며,
인간의 본성만큼이나 오래된 것이다.
그러나 노예가 가족의 일원이었다고 말할 때 우리가 뜻하는 바는
그를 데려와서 노예로 삼은 자들의 동기에 관한 것이 아니다.
단지,
노예를 주인에게 묶어주는 관계가
다른 모든 가족구성원을 그 수장에게 결합시키는
관계와 동일한 일반적 성질을 가지는 것으로 간주되었음을 의미할 뿐이다.
사실,
가족 관계를 떠난, 개인들 사이의\latin{inter se} 관계를
관계의 기초로 이해하는 것은
인류의 원시적 관념과
전혀 부합하지 않는다는
전술한 일반적 주장에서 이러한 결론이 나오는 것이다.
가족은 우선 혈연에 의해 가족에 속하게 된 자들과,
다음으로 \wi{입양}에 의해 가족에 편입된 자들로 구성된다.
그런데
단지 수장에게 공동으로 복종한다는 것에 의해 가족에 속하게 되는
세 번째 카테고리의 사람들이 있으니,
이들이 바로 노예인 것이다.
수장에게 출생에 의해 복속하는 자와 입양에 의해 복속하는 자가
노예보다 더 우대받는 것은 통상적으로 사태가 진행된다면
언젠가 그들은 속박에서 벗어나 스스로 \wi{가부장권}을 행사하게 될 것이라는
확실성에 기인한다.
그러나 노예의 지위가 낮다는 것이
그를 가족의 울타리 바깥에 둔다든가
영혼없는 물건의 지위로 격하시킨다는 따위가 아니라는 것은,
생각건대,
최후 수단으로 상속인이 될 수 있는 능력에 관한
고대의 많은 흔적들이 남아있어
명백히 증명되고 있다.
물론,
가부장의 제국에서 확실하게 자기 자리를 차지하고 있다는 것으로써
사회의 초창기에 노예의 운명이 얼마나 개선되었을까 무모하게 추측하는 것은
대단히 안전하지 않은 추측일 것이다.
아마도,
후대에 아들에게 주어질 부드러운 취급을 노예도 받았을 것이라기보다는
오히려 아들이 사실상 노예와 비슷하게 취급받았을 것이라고 보는 쪽이
보다 그럴 법하다.
그러나
다소 자신있게 말할 수 있는 것은,
발달된 성숙한 법체계 중에서
\wi{노예제}도가 인정되는 곳이라면 어디서나,
노예의 열등한 지위에 관한 어떤 다른 이론을 채택한 법체계보다
노예의 초기 상황에 대한 기억을 보존하고 있는 법체계에서
항상 노예가 더 나은 취급을 받는다는 것이다.
법이 노예를 바라보는 관점이 그에게는 무엇보다 중요하다.
로마법에서는
노예를 점점 더 물건의 일종으로 취급하던 경향이
자연법 이론 덕분에
더 이상 확대되지 않고 정지되었다.
그리하여,
로마법에 의해 깊이 영향받은 제도에 기초하여 노예제를 인정하는 곳에서는
노예의 지위가 결코 참을 수 없을 만큼 열악한 것이 아니다.
공포스런 내전의 영향 하에 통과된 최근의 입법으로
헌법에 수정조항이 추가되기 전까지는,
%\footnote{이 구절은 <<고대법>> 초판에는 없던 구절이다.}
미국의 주들 중에
영국 보통법에 토대를 둔 제도를 채택한 주들보다
대단히 로마적인 루이지아나법을 토대로 삼은 주들에서
흑인들의 운명과 전망이 여러 주요 측면에서 더 나았다는
많은 증거가 있다.
영국 보통법은, 최근에 해석된 것처럼,
노예를 위한 자리를 전혀 인정하지 않으며
따라서 노예는 그저 일종의 동산으로 취급될 수 있을 뿐이다.

\para{가족의 해체}
지금까지 본 저서에서 다루어야 할 고대 신분법의 모든 부분들을 살펴보았다.
이러한 탐구의 결과로
법의 유년기에 관한 우리의 관점이 보다 분명하고 정확하게 되었다고
나는 믿는다.
국가법은
가부장적 주권자의 \wi{테미스테스}로
처음 등장했거니와,
이제 우리는
저 테미스테스가
인류의 훨씬 더 초기 상태에
각각 독립된 가\hanja{家}의 수장들이 그의 아내와 자식들과 노예들에게 내리던
책임지지 않는 명령의
발달된 형태에 불과하다는 것을
알 수 있을 것이다.
하지만, 국가가 형성된 뒤에도,
법은 여전히 무척 제한된 적용범위를 가질 따름이었다.
법이 테미스테스라는 원시적 형태를 취하든,
아니면 관습법이나 법전이라는 보다 진보된 형태를 취하든,
그것은 개인이 아니라 가족을 구속하는 것이었다.
고대법은,
다소 무리한 비교를 한다면, \wi{국제법}과 유사하다고 할 수 있다.
말하자면, 사회의 원자들인 저 중차대한 집단들 사이의 틈새만을 메울 뿐인 것이다.
이런 상태의 공동체에서는
입법기관의 입법과 법원의 판결은
가족의 수장들에게만 미칠 뿐,
다른 모든 개인에게는
그의 가부장을 입법자로 하는
가\hanja{家}의 법이 곧 행위규칙인 것이다.
그러나 처음에는 좁았던 국가법의 영역이 꾸준히 그 범위를 넓혀간다.
법 변동의 인자\hanja{因子}들, 즉
법적의제, 형평법, 입법이 차례로 원초적 제도에 영향을 미치고,
진보의 각 단계마다
다수의 신분적 권리와 더 많은 수의 재산적 권리가
가내법정을 떠나 국가법정의 관할로 넘어가게 된다.
정부의 명령은 공적 문제뿐 아니라 사적 문제에도 차츰 간여하고,
더 이상 각 가정의 전제군주의 명령에 의해 무력화되지 않게 된다.
로마법의 연대기에는
고법\hanja{古法}체제가 무너져내린 거의 완전한 역사,
재조합한 재료들로 새로운 제도가 형성되어간
거의 완전한 역사가 담겨있거니와,
그 제도들의 일부는 근대 세계로 고스란히 전해졌으나,
다른 것들은 암흑시대에 만족\hanja{蠻族}들과의 접촉에서 파괴되거나 타락하여
인류에 의해 다시 회복되어야 했다.
유스티니아누스에 의해 마지막으로 재구성된 시대로써
로마법이 그 역사를 마감할 때,
살아있는 가부장에게 여전히 남겨졌던 폭넓은 권한에 관한 항목 하나를 제외하면
그 어떤 부분에서도 우리는 더 이상 고대법의 흔적을 발견할 수가 없다.
이 예외를 제외한 모든 곳에서는
편의성과 조화성과 단순성의 원리---여하튼 새로운 원리들---가
고대의 양심을 만족시켰던 조잡한 고려들의 권위를 전복시켰다.
모든 영역에서
새로운 도덕성이 고대적 관행에 부합했던 행위 기준과 묵인 근거를
내쫓았다. 사실 이들은 고대적 관행의 산물이었기 때문이다.

진보하는 사회들의 운동은 한 가지 면에서는 일치했다.
모든 과정을 통털어, 가족적 위계관계가 해체되고 그 대신 개인적 의무가
성장한 점이 뚜렷하게 나타났다.
국가법의 고려 단위로서
개인은 꾸준히 가족을 대체해갔다.
진보의 속도는 서로 달랐다.
현상의 면밀한 연구를 통해서만
고대 조직의 붕괴를 인지할 수 있는,
정체된 사회에 가까운 사회도 있었다.
하지만,
속도의 차이에도 불구하고,
변화는 역행 없이 계속되었다.
멈칫거리는 듯 보여도 그것은 외부에서 유입된 원시적 관념과 관습 때문인 것으로
판명될 것이다.
가족에서 기원한 권리와 의무의 호혜성\hanja{互惠性} 형식을 점차 대체해간
사람들 간의 관계가 무엇인지 이해하는 것도 어렵지 않다.
계약이 바로 그것이다.
역사의 한쪽 끝에서,
사람들 간의 모든 관계가 가족 관계로 귀결되던 사회 상태에서 출발하여,
사람들 간의 모든 관계가 개인들의 자유로운 합의에서 생겨나는
사회질서의 국면으로
지속적으로 변화해온 것으로 보인다.
이런 방향으로
서유럽에서
이루어진 진보는 엄청난 것이었다.
그리하여 노예의 신분은 사라졌다.
그것은 주인에 대한 하인의 계약관계로 대체되었다.
피후견 여성의 신분도,
후견을 남편 아닌 다른 사람의 \wi{후견}으로 이해한다면,
역시 사라졌다.
성년에 이른 후 혼인할 때까지 그녀가 맺는 모든 관계는 계약관계이다.
가부장권에 복속하는 아들의 신분도 근대 유럽 사회들의 법에서는
더 이상 존재하지 않는다.
아버지와 성년의 자식을 묶어주는 민사법적 의무가 있다면,
그것은 오직 계약에 의해 법적 효력이 주어지는 것일 뿐이다.
예외처럼 보이는 것도 원칙을 보여주기 위한
인영\hanja{印影}으로서의 예외일 뿐이다.
결정권을 행사할 연령에 이르지 못한 자식,
피후견 고아, 심신상실의 선고를 받은 자,
이들은 모두 \wi{신분법}\latin{law of persons}에 의해
그 능력과 무능력이 규율된다.
왜 그런가?
그 이유는 법체계마다 언어 관용\hanja{慣用}이 달라 서로 달리 표현되고 있으나,
본질에 있어서는 모두 동일한 효과를 말하고 있다.
법학자들 대다수는
방금 언급한 부류의 사람들이
타인의 통제를 받는 근거가
오직
스스로의 이익을 위한 판단능력을 갖추지 못했기 때문이라는 원리에
일치하고 있다.
다시 말해, 계약을 체결하는 데 불가결한
첫 번째 요소를 결여하고 있기 때문이라는 것이다.

\para{신분에서 계약으로}
`신분'\latin{status}이라는 용어는
이러한 법의 진보를 표현하는 어떤 공식을 만드는 데
유용하게 쓰일 수 있다.
이 공식은, 그 가치가 어떠하든,
내가 보기에 충분히 확인된 것이다.
신분법이 다루는 신분의 모든 형태는
고대 가족에 기거하던 권력과 특권에서 유래한 것이며,
어느 정도는 지금도 그것의 색채를 띠고 있다.
그리하여 우리가 신분이라는 용어를,
최고의 학자들의 용법에 따라,
이러한 신분\latin{personal} 상태만을 의미하는 데 사용하고,
합의에 의해 직^^b7간접적으로 결과하는 상태를
지칭하는 데 사용하지 않는다면,
우리는 진보하는 사회의 운동이 지금까지
\hemph{신분에서 계약으로}\latin{from status to contract}의 운동이었다고
말할 수 있을 것이다.


\chapter{유언상속법의 초기 역사}

\para{교회의 영향}
역사적 연구방법이
법에 관한
기존에 널리 퍼진 연구방법에 비해 우수하다는 것을
증명하는 시도가
영국에서
행해진다면,
유언법\latin{testament; will}보다 더 좋은 예를 보여주는 법분야는 없을 것이다.
그러한 능력은 유언법의 긴 역사와 오랜 지속성에 빚지고 있다.
역사의 초창기의 유언법에서
우리는
아주 유년기의 사회상태를 발견하거니와,
그것은 어느 정도 노력을 기울여야만
그 고대적 형태를 깨달을 수 있는 개념들로 둘러싸여있다.
반면, 진보의 반대편 극인 지금의 우리는
동일한 개념들이
현대적인 용어와 사고습관에 의해 감추어진 것에
불과한
법관념들 가운데에 서있거니와,
따라서
우리의 일상적 정신에 속하는 관념들을 분석하고 조사할 필요성을
인식하는 또 다른 종류의 어려움에 처한다.
이들 양 극 사이의 유언법의 발달을
우리는
자못 뚜렷하게 추적할 수 있다.
유언법의 역사는
다른 어떤 법분야의 역사보다
봉건제 탄생 시기의 단절을 훨씬 덜 보여준다.
물론
고대사와 근대사의 분리에 의해 촉발된 단절,
즉 로마제국의 붕괴에 의해 촉발된 단절이
모든 법영역에서
지나치게 과장되어온 것은 사실이다.
나태한
많은 학자들은
혼돈의 여섯 세기 동안 착종\hanja{錯綜}되고 희미해진 관련성의 실타래를
찾는 수고를 하려들지 않았고,
인내와 노력이 부족하지 않은 다른 학자들은
자기 나라 법체계에 대한 헛된 자부심에 가득차
로마법에 대한 감사의 뜻을 고백하지 않아 잘못된 길을 갔다.
그러나 이러한 좋지 못한 힘들이 유언법 영역에는 거의 영향력을 발휘하지 못했다.
스스로도 인정하듯이 만족\hanja{蠻族}들은 유언이라는 개념을 알지 못했다.
최고의 학자들이 이구동성으로 하는 말에 따르면,
만족들의 법전 중에
원래 거주지에서 따르던 관습과 그후
로마제국 변방의 정착지에서 따르던 관습을 모은 법전에서는
유언 개념이 흔적조차 발견되지 않는다.
그러나 곧이어 로마속주의 주민들과 혼합되면서, 그들은
로마법으로부터, 처음에는 부분적으로 나중에는 전면적으로,
유언 개념을 수용했다.
이러한 신속한 동화에는 교회의 영향력이 크게 작용했다.
일찍이 교회권력은 몇몇 이교도 사원들이 누리던,
유언을 보관하고 등록하는 특권을 물려받았거니와,
일찍부터 종교단체가 취득하는 세속적 재산은
거의 전적으로 사적인 유증\hanja{遺贈}에 기인한 것이었다.
그리하여 초기의 지방 종교회의들의 결정에는
유언의 신성함을 부인하는 자들에 대한 파문이 지속적으로 등장한다.
여기 영국에서도,
다른 법 영역에는 존재한다고 여겨지기도 하는
역사적 단절이
유언법의 역사에서는, 널리 인정되듯이, 방지되었으니, 그 원인 중에
가장 주된 원인은 교회였음이 확실하다.
유언 사건에 대한 재판권은 교회법원에 주어졌거니와,
교회법원은 유언 사건에, 항상 현명한 것은 아니지만, 로마법 원리를 적용하였다.
보통법법원도 형평법법원도,
교회법원의 결정에 따라야 할 적극적 의무는 없었지만,
교회법원의 법적용 과정에서 이미 형성된 법규칙 체계의 강력한 영향력으로부터
벗어날 수는 없었다.
인적재산\hanjalatin{人的財産}{personalty}%
\footnote{영미법에서 동산과 채권을 포괄적으로 지칭하는
`인적재산'(personal property; personalty)은
봉건적 토지보유권에서 유래하는 `물적재산'(real property; realty)과
대비되는 개념이다.}%
의 유언상속에 관한 영국법은
로마 시민들의 상속을 규율하던 제도의 변형된 형태가 된 것이다.

\para{고대의 유언}
이 주제의 역사적 탐구방법에 의한 결론과
역사의 도움없이 단지
일견\latin{primâ-facie} 우리가 받는 인상만을 분석하는 방법에 의한
결론 사이에
커다란 차이가 있음은 그리 어렵지 않게 지적할 수 있다.
유언이란 것의 대중적 개념에서 출발하여,
아니면 심지어 법적 개념에서 출발하더라도,
거기에 어떤 속성이 반드시 수반된다는 것을 생각해내지 못할 사람은
아무도 없을 것이다.
예컨대, 유언은 반드시 \hemph{사망시에만}\latin{at death only}
효력이 생긴다는 것, 유언은 \hemph{비밀}\latin{secret}이라서
그것에 이해관계를 가지는 사람에게 알려져서는 안 된다는 것,
유언은 \hemph{철회가능}\latin{revocable}해서
언제든 새로운 유언에 의해 번복될 수 있다는 것 따위를 거론할 수 있다.
하지만 나는 이러한 속성들 중 어느 것도 유언의 속성이 아니었던 시대가
있었다는 것을 보여줄 수 있을 것이다.
우리의 유언의 직접적 선조였던 유언은 처음에는
작성 즉시 효력이 발생했고, 비밀도 아니었고, 철회도 불가능했다.
사실,
법제도 중에서
인간의 의도를 담은 문서에 의해 사후\hanja{死後}의 그의 재산 처분이
좌우된다는 것만큼 복잡한 역사적 작용의 산물인 것은 거의 없을 것이다.
유언이
위에 언급한 속성들을 얻게 되는 것은
아주 천천히 그리고 점진적으로 이루어졌다.
그것은
우연이라 할 만한 사건들이,
적어도 법의 역사에 영향을 준 것이 아닌 한
어쨌든 지금 우리의 관심대상은 아닌 사건들이,
원인이 되어 그 압력으로 이루어졌다.

\para{유언이라는 자연권}
법이론이 지금보다 풍부했던 시절,
법이론이 대체로 근거없고 미성숙한 상태에 머물러 있는 것은 사실이지만
그럼에도 불구하고 어떤 일반화도 없이 단지
경험적으로만 법을 추구하던 열등하고 조야한 상태는 벗어난 시절,
유언의 속성에 대해 우리가 쉽게 가지는 표면적인 직관을 설명하기 위해
널리 사용된 방식은
그 속성들이 유언에 자연적이라고 말하는 것,
또는, 이 용어를 끝까지 밀고나가,
자연법에 의해 유언에 주어진다고 말하는 것이었다.
생각건대,
일단 이 모든 속성들이
역사의 기억 속에 그 기원을 가지고 있음을 알게 되면
아무도 이런 법리를 유지하려 하지 않을 것이다.
또한,
우리 모두가 사용하고 있고
그것 없이는 어찌할 바를 잘 모르는 표현 형태 안에는
이 법리를 낳은 이론의 흔적들이
여전히 남아있는 것이다.
나는 17세기 법문헌에 자주 등장하는 명제를 가지고 이를 보여주고자 한다.
당시의 법학자들은 흔히 유언권한을 자연법상의 것이라고,
자연법이 부여한 권리라고 주장했다.
그들의 가르침은,
비록 모든 사람들이 연관성을 한눈에 알아보는 것은 아니지만,
재산의 사후\hanja{死後}처분을 지시하고 통제하는 권한이
소유권 자체의 필연적이고도 자연적인 귀결이라고
인정하는 사람들의 주장으로 실제 이어졌다.
전문기술적인 법을 공부하는 법학도라면 누구나,
학파 간에 서로 다른 언어의 옷을 입고 있더라도,
이와 동일한 견해를 만나보았을 것이다.
그것은, 이 법분야의 논리에 따르면,
유언\latin{ex testamento}상속을
망자\hanja{亡者}의 재산이 일차적으로 따라야 할 이전 방식으로 취급하고,
이어서 무유언\hanjalatin{無遺言}{ab intestato}상속은
사망한 소유권자가 실수로 또는 불운으로
행하지 않은 것만을 처리하기 위한 입법자의 부수적 대응책으로
설명하는 것이다.
이러한 견해는 유언 처분이 자연법상의 제도라는 보다 간결한 법리가 확장된
형태에 불과한 것이다.
물론, 자연과 자연법을 숙고했던 근대 학자들의 관념 연관의 범위를
도그마틱하게 단정짓는 것은 결코 안전한 일이 아닐 것이다.
그러나 유언권이 자연권이라고 인정하는 사람들의 대다수는
그것이 사실상 보편적이라는 것을 의미하거나, 아니면
그것이 인간의 최초의 본능과 충동에 의해 인정되었다는 것을 의미하고 있다고
나는 믿는다.
전자의 입장에 관해서는,
나폴레옹법전\latin{Code Napoléon}에 의해 유언권이 심하게 제약되고 있고
프랑스법전을 모델로 삼은 법체계들이 속속 증가하고 있는 이 시대에,
그것을 명시적으로 주장한다면
이는 진지하게 내세울 수 있는 주장이 못된다는 것이 내 생각이다.
후자의 주장에 관해서는,
그것이 법의 초기 역사에서 충분히 확인된 사실에 반한다고 해야 할 것이다.
모든 토착적 사회에서는 유언이 허용되지 \hemph{않는},
아니, 생각조차 되어보지 못한 법상태가
소유권자의 단순한 의사가 다소간의 제약 하에
그의 혈연 친족들의 요구에 우선하게 되는 후대의 발달된 법상태에
일반적으로 선행하였다고
나는 감히 주장하고 싶다.

\para{유언의 성질}
유언의 개념은 그것 자체만으로 이해될 수 있는 것이 아니다.
그것은 일련의 개념들 중 하나일 뿐이며, 그들 중 첫 번째 것도 아니다.
유언 자체는 유언자의 의사가 선언되는 수단일 뿐이다.
생각건대,
이러한 수단을 논의하기에 앞서,
우선 예비적으로 몇 가지 논점을 밝혀둘 필요가 있다.
가령, 망자의 사망으로 그에게서 이전되는 것은 무엇이며 어떤 종류의
권리 혹은 이익인가? 그것은 누구에게 어떤 형태로 이전되는가?
망자가 자기 재산의 사후\hanja{死後} 처분을 통제할 수 있는 것은 어째서인가?
따위가 그것들이다.
유언이라는 관념에 기여하는 다양한 종속적 개념들은 다음과 같이 법기술적인
용어로 표현될 수 있다.
유언이란 상속재산의 이전\hanja{移轉}을 정하는 수단이다.
상속이란 포괄적 승계\latin{universal succession}의 일종이다.
포괄적 승계란 포괄적 재산\latin{universitas juris}의 승계, 즉
권리와 의무의 총체를 승계하는 것이다.
그리하여 우리는, 역순으로,
무엇이 포괄적 재산인지, 무엇이 포괄적 승계인지,
그리고 상속이라 불리는 포괄적 승계의 형식은 무엇인지를 탐구해야 한다.
나아가, 내가 말한 논점들과 어느 정도 무관한, 그러나
유언이라는 주제를 마치기 전에 해결해야 할, 두 가지 논점이 더 있다.
어떻게 해서 상속재산이 유언자의 의사의 통제대상이 되었을까,
그리고 상속재산을 통제하는 수단의 성질은 무엇인가, 하는 것이 그것들이다.

\para{포괄적 재산}
첫 번째 문제는 ``포괄적 재산'',
즉 권리와 의무의 총체\paren{혹은 묶음}에 관한 것이다.
포괄적 재산이란
특정한 한 시점에 특정한 한 사람에게 속한다는 단일한 상황으로 결합된
권리와 의무의 집합물을 말한다.
말하자면 그것은 특정 개인의 법적인 옷\hanja{[衣服]}인 것이다.
몇몇 권리와 몇몇 의무를 하나로 묶는다고 해서 되는 것이 아니다.
오직 특정인의 모든 권리와 모든 의무를 하나로 묶어 성립할 수 있을 뿐이다.
다수의 소유권, 통행권, 유증에 대한 권리, 특정한 급부 채무, 금전채무,
불법행위 손해배상 채무 따위의 모든 법적 권리와 의무를 하나로 묶어
포괄적 재산을 이루게 하는 힘은
이를 행사하고 이행할 수 있는 어떤 개인에게
이것들이
속해있다는
\hemph{사실}에 있다.
이러한 \hemph{사실}이 없으면 포괄적 재산은 존재할 수 없다.
`포괄적 재산'이라는 용어는 고전기의 것이 아니지만,
그 관념은 오로지 로마법에 빚지고 있다.
그것을 이해하는 것도 결코 어렵지 않다.
우리들 각자가 바깥 세상에 대해 갖는 모든 법률관계의 집합을
하나의 개념 아래에 끌어모으면 된다.
그 성격이나 성분이 어떠하든, 이들이 모여 포괄적 재산을 이루는 것이다.
권리뿐만 아니라 의무도 포함된다는 점만 명심한다면,
개념을 형성하는 데 있어 실수할 위험은 거의 없다.
의무가 권리보다 더 많을 수도 있다.
어떤 이는 적극재산보다 채무가 더 많아서,
그의 법률관계의 집합을 금전적으로 평가하면 소위 지급불능 상태인 것으로
판명될 수도 있다.
그렇다고 그를 둘러싸고 있는 권리와 의무의 총체가
``포괄적 재산''이 아닌 것은 아니다.

\para{포괄적 승계}
다음으로 ``포괄적 승계''가 문제된다.
포괄적 승계란 포괄적 재산을 승계하는 것이다.
어떤 사람이 다른 사람의 법적인 옷을 입어,
그의 모든 책임을 부담하고 그의 모든 권리를 가질 때 이런 일이 일어난다.
포괄적 승계가 진정하고 완전한 것이 되기 위해서는,
이전\hanja{移轉}이,
법학자들의 표현을 빌면, `일거\hanja{一擧}에'\latin{uni ictu}
일어나야 한다.
물론, 어떤 이가 다른 사람의 권리와 의무의 전부를 여러 번에 걸쳐,
가령 순차적 매수를 통해서, 취득할 수 있다.
혹은 서로 다른 자격으로, 가령 일부분은 상속인으로,
다른 일부분은 매수인으로, 나머지는 수유자\hanja{受遺者}로서 취득할 수도 있다.
그러나 이렇게 해서 얻은 권리와 의무의 집합이 사실상 특정인의 법인격 전부라
할지라도 이러한 취득은 포괄적 승계가 아니다.
진정한 포괄적 승계가 되기 위해서는,
권리와 의무의 총체의 이전이 \hemph{동일한} 시점에 이루어져야 하고
수령인이 \hemph{동일한} 법적 자격에서 넘겨받는 것이라야 한다.
``포괄적 재산'' 개념과 마찬가지로 포괄적 승계의 개념도
법학에서 항상 발견되는 개념이다.
다만 영국법에서는 권리를 취득하는 사뭇 다양한 자격으로 인해,
특히 ``물적재산''\latin{realty}과 ``인적재산''\latin{personalty}이라는
영국 재산법의 두 개의 큰 영역 간의 구분으로 인해, 그 개념이 흐려져있을 뿐이다.
하지만 파산관재인이 파산자의 전 재산을 승계하는 것은
일종의 포괄적 승계에 해당하거니와,
다만
파산관재인은 그 재산의 한도 내에서만 채무를 지불하므로
원래 개념의 변형된 형태일 뿐이라 할 것이다.
만약 어떤 이가 어떤 다른 사람의 \hemph{모든} 채무를 지불한다는 조건으로
그의 \hemph{모든} 재산을 양수하는 일이 영국에도 흔한 일이라면,
그러한 양수는 초기 로마법이 알고 있던 포괄적 승계와 정확히 일치할 것이다.
어떤 로마 시민이 아들을
\hemph{자권자입양}\hanjalatin{自權者入養}{adrogate}할 때,
즉 가부장권에 복속하지 않는 남자를 양자로 입양할 때,
그는 입양되는 아들의 재산을 \hemph{포괄적으로} 승계했다.
모든 재산을 취득했을 뿐만 아니라 모든 의무에 대해서도 책임을 졌던 것이다.
초기 로마법에는 포괄적 승계의 몇몇 다른 형태들도 있었으나,
무엇보다 가장 중요하고 가장 지속적인 행태는 지금 우리의 관심대상인 것,
즉 상속\latin{haereditas}이었다.
상속은 사망으로 발생하는 포괄적 승계였다.
이때의 포괄승계인은 `상속인'\latin{Haeres}이라고 불렸다.
그는 망자의 모든 권리와 모든 의무를 동시에 물려받았다.
그는 즉시 망자의 법인격 전체를 옷으로 입었다.
유언으로 지명된 상속인이든, 아니면
무유언 상속인이든, 상속인이라는 지위에는 아무런 차이가 없었다는 점은
부연할 필요가 없을 것이다.
`상속인'이라는 용어는 유언상속인에 대해서도, 무유언상속인에 대해서도
똑같이 사용되었다.
상속인이 되는 것은 그가 갖는 법적 성격이 무엇이냐와
무관하기 때문이다.
유언에 의해 상속인이 되든, 무유언상속의 상속인이 되든,
망자의 포괄승계인은 그의 상속인인 것이다.
그러나 상속인은 반드시 한 사람이어야 할 필요는 없었다.
법적으로 하나의 단위로 간주되는 일군의 사람들이
\hemph{공동상속인}이 되어 승계할 수 있었다.

\para{포괄승계인}
이제 로마인들의 상속의 정의를 인용해보자.
독자들은 각각의 용어를 완전히 이해할 수 있을 것이다.
``상속은 망자가 가졌던 포괄적 법적 지위를 승계하는 것이다.''\latin{Haereditas
est successio in universum jus quod defunctus habuit.}
비록 망자의 육신은 소멸하지만,
그의 법인격은 살아남아 그와의 동일성이
\paren{적어도 법의 관점에서는}
이어지는 상속인 또는 공동상속인들에게
전달된다는 뜻이다.
영국법에서 유언집행인\latin{executor} 또는
유산관리인\latin{administrator}이
인적재산에 관한 한 망자의 대표자로 이해되는 것은
여기서 유래한 이론으로 예시될 수 있을 것이다.\footnote{유언집행인은
유언장에 의해 지명되고, 유산관리인은 유언이 없는 경우 법원에 의해
선임된다. 인적재산은 이들이 일단 취득하여 관리^^b7집행하고,
물적재산은 상속인이나 수유자가 직접 승계한다.}
그러나 예시는 몰라도 설명까지 되는 것은 아니다.
후기 로마법의 견해에 의하더라도
망자와 상속인 간에는 긴밀한 견련\hanja{牽連}관계가 필요했거니와,
이는 영국법상의 대표자에게는 요구되지 않는 것이다.
또한 원시법에서는 모든 것이 승계의 연속성을 지향하고 있었다.
유언자의 권리와 의무를 즉시 넘겨받을 상속인이나 공동상속인들이
유언장에
지시되지 않았다면,
그러한 유언장은 전부 무효였다.

\para{고대의 상속}
후기 로마법과 마찬가지로 근대 유언법에서도
가장 중요한 목적은 유언자의 의사를 집행하는 일이다.
고\hanja{古}로마법에서 그에 상응하는 중요성을 띠는 것은
포괄적 승계를 수여하는 일이었다.
우리 눈에 전자는 상식이 명령한 원리로 보이지만,
후자는 한가한 변덕의 발로로 보이기 십상이다.
하지만 후자가 없었다면 전자도 존재할 수 없었을 것이라는 점은
유사한 다른 명제들만큼이나 여기서도 확실하다.

\para{원시 사회}
이 역설처럼 보이는 것을 풀기 위해서는,
그리고 내가 보이고자 하는 관념의 연쇄를 보다 명백히 보여주기 위해서는,
바로 앞 장에서 행했던 탐구의 결론을 빌어와야 할 것 같다.
거기서 우리는 사회의 유년기에 나타나는 보편적인 특징 하나를 발견했다.
사람들은 개인이 아니라 언제나 어떤 집단의 구성원으로
간주되고 취급되었다.
누구나 우선은 시민이었다.
다음으로 시민으로서의 그는
신분집단---그리스의 귀족\latin{aristocray}이나 평민\latin{democracy},
혹은 로마의 귀족\latin{patrician}이나 평민\latin{plebeian},
혹은 발달과정에서 불행한 운명이 할퀴고 간 타락한 사회에서는
카스트---에 소속되었다.
그 다음으로 그는 씨족\latin{gens; house; clan}의 구성원이었다.
그리고 마지막으로 \hemph{가족}의 구성원이었다.
이 마지막 것은 그가 서있는 가장 좁은 관계이자 가장 친밀한 관계였다.
역설적으로 보일지 몰라도 그는 \hemph{그 자신}으로,
독립적 개인으로, 간주되지 않았다.
그의 개인성은 가족에 의해 흡수되었다.
전술한 원시사회의 정의를 재차 강조하자면,
그것의 단위는 개인이 아니라,
실제의 또는 의제\hanja{擬制}의 혈연 관계에 기초한 사람들의 집단이었다.

\para{가족이라는 단체}
저발전 사회의 이러한 특성에서 우리는 포괄적 승계의 최초의 흔적을 발견한다.
근대국가의 구조와 달리, 원시시대의 국가는
다수의 작은 전제\hanja{專制}적 정부들로 구성되어 있었고,
그 각각은 다른 것들로부터 완전히 독립적이었으며,
각각은 단일한 군주의 대권\hanja{大權}에 의해 절대적으로 지배되고 있었다고
기술\hanja{記述}함이 마땅하다.
하지만, 비록 이 가부장---아직 로마의 가부장\latin{pater-familias}이라고
해서는 안 된다---이 광범위한 권리를 가지고 있었지만,
의심할 여지 없이 그는 수많은 의무도 마찬가지로 부담하고 있었다.
그가 가족을 지배했다면, 그것은 가족을 위한 것이었다.
그가 가족의 물건의 주인이었다면, 그것은 그의 자식들과 친족들을 위한
수탁자\hanja{受託者}로서 보유하는 것이었다.
그가 특권이나 높은 지위를 가졌다면, 그것은
그가 지배하는 작은 국가와의 관계에 의해 그에게 부여되는 것이 전부였다.
가족은 실로 단체였고, 그는 그 단체의 대표자였다.
어쩌면 그 단체의 공직자였다고도 말할 수 있을 것이다.
그는 권리를 향유하고 의무를 부담했으나,
동료시민들이 보기에는, 그리고 법의 관점에서는,
이들 권리와 의무는 그의 것인 동시에 집합체의 것이기도 했다.
이러한 대표자의 사망으로 어떤 일이 발생하는지 잠시 생각해보자.
법의 관점에서는, 국가 정무관의 관점에서는,
가내 권위자의 사망은 전혀 중요하지 않은 일이었다.
가족이라는 집합체를 대표하고 국가법정에서 일차적으로 책임지는 사람이
이제 다른 이름을 가진다는 것, 그것이 전부였다.
사망한 가\hanja{家}의 수장에게 주어졌던 권리와 의무는
연속성이 끊어지지 않은 채 그의 승계인에게 주어진다.
사실 이들 권리와 의무는 가족의 권리와 의무였고,
가족은 단체 특유의 성질---결코 죽지 않는 성질---을 가지기 때문이다.
채권자들은 과거의 수장\hanja{首長}에 대한 것처럼
새로운 수장에 대해서도 동일한 구제수단을 가진다.
가족은 여전히 존속하고, 가족의 책임도 완전히 동일하기 때문이다.
가족이 가지던 모든 권리는 수장의 사망 전과 똑같이 사망 후에도
가족에게 남는다.
다만, 이제 단체는 조금 다른 이름으로---이렇게 정확한 법기술적인 용어를
저 초기 사회에 대해서도 사용할 수 있다면---\hemph{소송}을 제기해야 할
따름이다.

\para{가족과 개인}
법제사의 전 역사를 추적해야만,
가족이 어떻게 해서 그것을 구성하는 원소들로 점차 느리게
해체되어갔는지---어떤 비가시적인 점진적 변화에 의해
개인의 가족에 대한 관계가,
그리고 가족의 가족에 대한 관계가,
개인의 개인에 대한 관계로 대체되어갔는지---를
이해할 수 있다.
지금 우리가 살펴볼 논점은,
혁명이 완수된 후에도,
가부장의 자리를 정무관이 거의 떠맡은 후에도,
국가법정이 가내법정을 대체한 후에도,
사법당국이 다루는 권리와 의무의 체계 전체에는 여전히
낡은 특권의 영향이 남아있었고 그 반향이 구석구석을 물들이고 있었다는 것이다.
로마법에서 유언상속이나 무유언상속의 첫 번째 요건으로 강조되었던
포괄적 재산의 이전은 옛 사회구조의 특징이었고
새로운 발달단계와는 진정한 또는 적합한 결합을 이루지 못함에도 불구하고,
인간의 정신은 새로운 사회형태에서 옛 형태를 떨쳐버릴 수 없었음이
거의 틀림없어 보인다.
어떤 사람의 법적 존재가 그의 상속인이나 공동상속인단\hanja{團}에게
연장된다는 것은
\hemph{가족}의 성격이 의제\hanja{擬制}에 의해 \hemph{개인}에게 부여된다는 것,
그 이상도 이하도 아니다.
단체의 승계는 어디서나 있을 수밖에 없거니와, 가족은 단체였다.
단체는 죽지 않는다.
개인 구성원의 사망은 집합체의 집단적 존재에는 아무런 차이도 가져오지 못하고,
그것의 법적 측면, 즉 그것의 권한과 책임에도 아무런 영향을 주지 못한다.
이제 로마법상의 포괄적 승계 개념에서는
단체의 이 모든 속성들이 개인 시민에게
부여된 것으로 보인다.
그의 물리적 죽음은 그가 가졌던 법적 지위에 아무런 영향도 주지 못했다.
이는 그의 지위를
가족이라는 것의 유추\hanja{類推}에 가능한 한 가까운 것으로 만드는
원리에 기초한 것으로 보인다.
물론 가족은 단체의 성격을 가지고 있어서 물리적으로 소멸하지 않았다.

\para{단독법인}
포괄적 승계를 구성하는 개념들 간의 관계의 본질을 이해함에 있어
적지 않은 수의 대륙의 법학자들이
큰 어려움을 겪고 있는 것으로 보이며,
그들의 법철학의 주제 중에 이것만큼
일반원리로서의 가치를 거의 갖고 있지 못한 것도 없는 것 같다.
하지만 영국의 법학자들은 지금 우리가 다루고 있는 관념을 분석하는 데
실패할 위험이 없다고 할 것이다.
모든 법률가들이 익히 알고 있는 영국법상의 의제\hanja{擬制} 하나를 가지고
그것을 해명할 수 있기 때문이다.
영국의 법률가들은 법인을 집합법인\latin{corporation aggregate}과
단독법인\latin{corporation sole}으로 구분한다.
집합법인은 진정한 단체이다.
하지만 단독법인은 의제를 통해 단체의 속성이
주어지는 개인에 불과하거니와,
이 개인은 연속적으로 등장하는 개인들의 일원이다.
국왕이나 교구목사를
단독법인의 예로 드는 일은 굳이 필요치 않을 것이다.
이 자리가 갖는 권한은 그 자리를 수시로 차지하는 특정인과 분리하여 취급된다.
또한 이 권한은 영구적이므로, 그 자리를 차치하는 일련의 개인들은
단체의 제일가는 속성---영구성---의 옷을 입는다.
옛 로마법 이론에서 개인의 가족에 대한 관계는
영국법의 법리에서 단독법인이 집합법인에 대해 갖는 관계와 정확히 일치한다.
관념들의 파생관계와 연합관계가 완전히 동일하다.
실로, 로마 유언법을 가르칠 목적으로
각 개인 시민은 단독법인이었다고 영국인들에게 말한다면,
영국인들은 상속의 개념을 완전히 이해할 뿐 아니라,
그것이 어떤 생각에서 기원했는지에 대한 실마리도 얻어낼 수 있을 것이다.
국왕은 단독법인이어서 죽지 않는다는 것이 영국인들의 공리\hanja{公理}이다.
국왕의 권한은 즉시 그의 계승자에 의해 채워지고,
통치권의 연속성은 단절되지 않는다.
로마인들에게도
권리와 의무의 이전으로부터 사망의 사실을 제거하는 것은
똑같이 단순하고 자연스러운 과정이었을 것이다.
유언자는 그의 상속인 또는 공동상속인단 속에 여전히 살아있었다.
법적으로 그는 그들과 동일인이었다.
만일 누군가의 유언장이, 그것의 해석에 의해서라도,
그의 현실적 존재와 사후\hanja{死後}적 존재를 결합시키는 원리를
위반하는 것이라면,
법은 그러한 흠결 있는 유언장을 무효로 선언하고
그의 혈연 친족들에게 상속권을 부여했다.
이 경우 혈연 친족들의 상속능력은 법 자체에 의해 주어진 것이지,
잘못 작성되었을 수 있는 유언장에 의한 것이 아니었다.

\para{무유언상속}
로마 시민이 유언 없이 사망하거나 유효한 유언을 남기지 못한 경우,
조금 뒤에 언급할 순위에 따라 그의 자손들이나 친족들이 상속인이 되었다.
상속인 또는 상속인단은 단순히 망자를 \hemph{대표하는} 것이 아니라,
좀 전에 서술한 이론에 따라 그의 시민적 삶, 그의 법적 존재를 계속
\hemph{이어갔다}.
이런 결과는 유언에 의해 상속의 순위가 정해지더라도 똑같이 발생한다.
그러나 망자와 그 상속인 간의 동일성 이론은 분명 그 어떤 유언의 형식보다도,
그 어떤 유언법의 단계보다도, 더 오래된 것이다.
실로, 이 주제를 파고들면 들수록 점점 더 강하게 우리를 압박해오는
의문 하나를 제기할 적절한 때가 된 것 같다.
포괄적 승계에 관련된 저 중차대한 관념이 없었더라면 도대체
\hemph{유언}이라는 것이 등장할 수 있었을 것인가 하는 의문이 그것이다.
오늘날 유언법에 적용되는 원리는
근거는 없지만 그럴 듯해 보이는 다양한 철학적 가설에 기초하여 설명될 수 있다.
그것은 근대사회의 모든 부분과 얽혀있고, 일반적 공리\hanja{功利}라는
사뭇 폭넓은 근거에서 정당화되고 있다.
하지만,
오늘날
기존 제도를 유지하게 하는 저 근거들이
그 제도의 기원을 가져왔던 감정과 반드시 같을 것이라는
인상이야말로
법학에 있어 잘못된 생각의 큰 원천이 되고 있다는
경고는 아무리 반복해도 지나치지 않을 것이다.
확실히,
옛 로마 상속법에서 유언의 관념은
사람이 상속인의 인격 속에서 사후에도 존재한다는 이론과
불가분 혼합되어 있었던, 아니 어쩌면 일체화되어 있었던 것이다.

\para{초기 로마의 유언, 로마와 인도의 사크라}
포괄적 승계의 개념은 법학에 굳건히 뿌리내렸지만
모든 법체계의 기초자들에게 자동적으로 주어졌던 것은 아니다.
오늘날 그것이 발견되는 어디서나 그것은 로마법에서 유래한 것임을
보일 수 있을 것이다.
또한 그것과 더불어 유언과 유증에 관한 일군의 법규칙들이 전해내려왔거니와,
오늘날의 법률가들은 저 원초적 이론과의 관계를 알지 못한 채 이들을 적용하고 있다.
그러나 순수한 로마법에서는 사람이 그의 상속인 속에서
계속 살아남는다---말하자면 사망의 사실이 제거된다---는
원리가 너무도 자명해서, 유언상속법과 무유언상속법 전체의 핵심이
무엇인지 도저히 오인할 수가 없었다.
지배적 이론을 따르도록 강제하는 로마법의 단호한 엄격함은
저 이론이 로마 사회의 초기 구조에서 자라났을 것임을 짐작하게 한다.
하지만 우리는 추정을 넘어 입증까지 나아가야 한다.
로마의 초기 유언 제도에서 기원하는
몇몇 법기술적인 표현들이 우연히도 우리에게 전해졌다.
가이우스의 저서에서는 포괄승계인을 지명하는 방식\latin{formula}이
발견된다.\footnote{고법상의 `구리와 저울에 의한 유언'을 말하고 있다.
  매수인은 ``당신의 가와 재산이 나의 책임과 보관에 넘어오도록 \ldots''
  이라고 언명한다. \latin{Gai.\,2.104.}}
나중에 상속인이라고 불리게 되는 사람을 애초에 지칭했던 옛 이름도
등장한다.\footnote{`구리와 저울에 의한 유언'에서
  상속인에 해당하는 자는 `가(家)의 매수인'(familiae emptor)이라고
  불리고 있다. \latin{Gai.\,2.103.}}
또한 우리는 유언권한을 명시적으로 인정하는
유명한 12표법 조항의 텍스트를 알고 있으며,\footnote{%
  ``가부장이 자신의 가(家)와 재산에 관하여 종의처분(終意處分)한 바가 있으면
  그대로 법으로 한다''(5.3). Cicero.\,De Inventione.\,2.148.}
무유언상속을 규율하는 조항들 역시 전해지고 있다.\footnote{``무유언으로
  사망하는 자에게 가내상속인이 없을 경우에는 가장 가까운 종친이
  가(家)를 상속한다''(5.4).
  ``그러한 종친이 없을 경우에는 씨족원들이 가(家)를 상속한다''(5.5). }
이 모든 고법\hanja{古法}상의 표현들은 두드러진 특징 하나를 가진다.
유언자로부터 상속인에게 넘어가는 것은 다름아닌 \hemph{가}\hanja{家}라는 것,
즉 가부장이 보유하고 그로부터 유래하는 권리와 의무의 총체라는 것이다.
물질적인 재산은 세 가지 경우에는 전혀 언급되지 않고 있으며,
나머지 두 가지 경우에는 가\hanja{家}의 부속물로서 거론되고 있을 뿐이다.
그리하여 원래의 유언은 \hemph{가}\hanja{家}의 이전을 규율하는 문서,
또는 \paren{처음에는 문서로 작성되지 않았을 것이므로} 절차였던 것이다.
그것은 유언자를 승계하여 누가 수장\hanja{首長}이 될 것인가를 선언하는
양식이었다.
유언의 원래의 목적이 이런 것임을 이해할 때,
우리는 그것이 고대종교와 고대법의 가장 진기한 유물의
하나---\hemph{사크라}\latin{sacra}, 즉 가족제의\hanja{祭儀}---와
어떻게 연결되는지 즉시 알 수 있다.
사크라는 원시의 옷을 완전히 벗어버리지 못한 사회라면 어디서나 보이는
제도의 로마적 형태였다.
그것은 가족의 동포애를 기념하는 희생제의였고,
가족의 영구성을 담보하고 증언하는 장치였다.
그 성격이 어떠하든---어떤 신화적인 조상에 대한 숭배이든 아니든---그것은
어디서나 가족관계의 신성함을 증명하기 위해 사용되었다.
따라서 수장의 인격이 바뀜으로써 가족의 연속성이 위협받는 곳이라면
그것은 특별한 의미와 중요성으로 다가왔다.
그리하여 우리는 그것을 가내 주권자의 사망과 관련하여 자주 듣게 되는 것이다.
인도인들 사이에서, 망자의 재산을 상속하는 권리는 그의 장례식을 치르는 의무와
정확히 궤를 같이했다.
제의가 제대로 거행되지 않거나 엉뚱한 사람에 의해 거행된다면,
죽은 자와 산 자 간에는 아무런 관계도 형성되지 않는 것으로 간주된다.
상속법이 적용되지 않고, 누구도 재산을 상속받을 수 없는 것이다.
인도인이 삶에서 겪는 중대한 사건들은 모두 이 장엄한 의례와 관련되고 이를
지향하는 것으로 보인다.
인도인이 혼인을 한다면, 그것은
그의 사후에 제의를 거행할 자식을 가지기 위해서이다.
그에게 자식이 없다면,
그는 다른 가족으로부터 양자를 들여야 한다는 강한 의무감을 갖는다.
인도인 학자에 따르면 ``장례식의 떡과 물과 신성한 제물을 염두에 두고''
그렇게 한다는 것이다.
키케로 시대에 로마의 사크라가 포괄하는 범위도 그에 못지 않았다.
그것은 상속과 입양을 다 포괄했다.
아들을 내어주는 가족의 사크라에 대한 적절한 대비\hanja{對備}가 없다면
입양은 효력을 발생하지 않았다.
공동상속인들 간에 장례식 비용을 엄격히 분배하지 않으면
유언에 의한 상속재산 분할은 일어날 수 없었다.
사크라를 마지막으로 엿볼 수 있는
이 시대의 로마법과 현존하는 힌두법 간의 차이는
시사하는 바가 크다.
힌두법에서는 종교적 요소가 전적으로 우세했다.
가족제의는 친족법 전부와 물권법 대부분의 초석이 되었다.
그것은 심지어 기괴하게 확장되기까지 했으니,
인도인들에 의해 역사시대에 이르기까지 지속된 관행이며
몇몇 인도^^b7유럽 민족의 전승\hanja{傳承}에도 남아있는,
남편의 장례식에서 과부가 스스로를 제물로 바치는 관행이,
인간의 피야말로 최고의 제물이라는, 희생제의에 언제나 동반되는 생각으로 인해
원시적 사크라에 접목되어 들어간 것이다.
반면, 로마인들에게는 법적 의무와 종교적 의무가 분리되기 시작했다.
사크라를 엄숙히 거행해야 한다는 요청은 세속법의 이론에 속하지 않았고,
대신 신관단\hanjalatin{神官團}{college of pontiffs}이 다루는
별도의 법역에 속했다.
물론,
키케로가 아티쿠스에게 보내는,
사크라에 대한 언급으로 가득한
편지들을 보면, 그것이 상속에 참기 힘든 부담이 되고 있었음을 알 수 있다.
하지만 법이 종교로부터 분리되는 발달지점을 지나,
후대의 법에서는 사크라가 완전히 사라진 것을 발견하게 된다.

\para{로마의 상속 관념}
힌두법은 진정한 유언에 해당하는 것을 알지 못한다.
유언의 자리를 대신 차지하고 있는 것은 입양이다.
이제 우리는 유언권한의 입양권능에 대한 관계를,
그리고 어째서 양자의 행사가 사크라의 거행에 대한 염려를 상기시키는지 그 이유를
알 수 있을 것이다.
유언과 입양은 둘 다 가계\hanja{家系}의 통상적인 진행을 왜곡시킬
위험을 안고 있지만, 명백히 이들은
계승할 친족이 존재하지 않을 때 가계가 단절되는 것을 막기 위한
장치들인 것이다.
이들 두 가지 수단 중에,
혈연 관계의 인위적 창설인
입양만이
대다수 고대 사회에서 발견된다.
사실 인도인들은 분명 고대의 관행인 것에서 한 걸음 더 나아가,
남편이 생전에 하지 못했다면 과부가 입양을 할 수 있도록 허용하였다.
또한 벵갈 지방의 관습에서는 유언권한의 희미한 흔적도 보이고 있다.
하지만,
계약 다음으로 인류 사회의 변화에 큰 영향을 끼친 제도인
유언을 발명한 것은 다름아닌 로마인들의 공적에 속한다.
그러나
보다 최근에 유언에 부여된 기능을
유언의 초기 형태에 부여하지 않도록 주의해야 한다.
초기의 유언은 망자의 재산을 분배하는 방법이 아니라,
가\hanja{家}의 대표권을 새로운 수장에게 전하는 여러 방법 중 하나였던 것이다.
물론 재산도 상속인에게 승계되지만,
그것은 이전되는 가\hanja{家}의 통치권에
공동재산을 처분하는 권한까지 포함되기 때문일 뿐이다.
유언의 역사에 있어
아직 우리는
유언이
재산의 유통을 자극하고
소유권에 유연성을 가져옴으로써
사회 변화의
강력한 도구가 되는 그러한 단계로부터 한참 멀리 있다.
사실 최후의 로마 법률가들조차도
유언권한에 그러한 결과가 부여되는 상태를 만들어내지 못한 것으로 보인다.
후술하겠지만,
로마 사회는 유언을
재산과 가족을 남에게 넘겨주는 장치나
잡다한 이해관계를 만들어내는 장치로
결코 생각한 적이 없으며, 오히려
무유언상속의 법규칙에 의한 것보다는 유언이
가의 구성원들을 위해 더 나은 대비\hanja{對備}를 할 수 있는 수단이라고
생각했던 것이다.
실로 로마인들이 유언 관행에 대해 가지는 관념은
오늘날 우리가 친숙하게 여기는 관념과는 완전히 달랐다고 보아도 좋을 것이다.
입양과 유언을 가\hanja{家}의 연속성을 위한 장치로 보는 습관은
주권\hanja{主權}의 상속에 관하여 로마인들이 가졌던 특유의 느슨한 관념과도
무언가 관련이 있음이 분명하다.
초기 로마 황제들 간의 승계가 꽤나 정상적이라고 여겨졌던 점,
그리고 테오도시우스나
유스티니아누스 같은 황제들이 카이사르와 아우구스투스의 전례\hanja{前例}를
따른다는 구실을 내세웠을 때
우여곡절은 있었지만
이를 비정상이라고 여기지 않았던 점 등은
보기 싫어도 보이는 사실들이다.\footnote{%
  379년 테오도시우스 1세는 선임인 서로마 지역 황제 그라티아누스에 의해
  동로마 지역 황제로 지명되었다.
  동로마제국의 유스티니아누스 1세는 전임 황제 유스티누스 1세에 의해
  양자로 입양되었다.}

\para{유언권한의 희소성}
원시사회의 현상들에 비추어볼 때,
17세기 법학자들이 의문시했던 명제, 즉
무유언상속이 유언상속보다 더 오래된 제도라는 명제를
논박하기는 불가능해 보인다.
이 문제가 해결되면 즉시 대단히 흥미로운 질문 하나가 떠오른다:
어떻게, 어떤 조건 하에서
유언의 지시가
가\hanja{家}에 대한 권위의 이전을,
따라서 재산의 사후적 분배를,
처음으로 규제할 수 있게 되었는가 하는 것이다.
이것을 규명하기 어려운 이유는
유언권한을 인정하는 고대 공동체가 드물었다는 데 있다.
로마를 제외하면 진정한 유언권한을 알고 있는 초기 사회가 있었는지 의심스럽다.
유언의 초보적 형태는 여기저기서 발견되지만,
그들 대부분은 로마의 것을 빌려왔다는 의심에서 자유롭지 못하다.
물론 아테네의 유언은 자생적인 것이었으나,
후술하듯이 그것은 미숙한 제도에 지나지 않았다.
로마제국을 정복한 만족\hanja{蠻族}들의 법전의 형태로 우리에게 전해지는
법체계들에서 인정되는 유언을 보면, 이것들은 거의 확실히 로마적이다.
이들 `만족\hanja{蠻族}들의 법'\latin{Leges Barbarorum}에 대한
사뭇 철처한 분석이 최근 독일에서 행해졌거니와,
그 목적은 각 법체계에서 어떤 부분이
그들의 원래 고향에서부터 부족 관습을 구성하던 부분이며
어떤 부분이 로마인들의 법에서 빌려온 부차적인 요소인지를
가려내는 것이었다.
이러한 과정을 거쳐 하나의 결과가 한결같이 도출되었으니,
각 법전의 오래된 핵심부분에는 유언의 흔적이 발견되지 않는다는 것이었다.
유언법이 존재한다면, 어느 것이나 로마법에서 가져온 것이었다.
마찬가지로, \paren{내가 알기로} 율법학자들이 유대법에 첨가한
초보적 유언도 로마인들과의 접촉에서 유래했다는 것이 정설이다.
로마나 그리스 사회에 속하지 않으면서
어떤 이유에서건 자생적이라고 할 만한 유언의 형태로는
벵갈 지방의 관행에서 인정되는 것이 유일하거니와,
인도에 사는 영국인 법률가들의 발명이라고까지 보는 이도 있는
이 벵갈의 유언은
기껏해야 초보적인 유언에 지나지 않는다.

\para{초기 유언의 작동}
그렇지만,
이러한 증거가 가리키는 것으로 보이는 결론은
유언이 처음에는
진정한 혹은 의제적 혈연권\hanja{血緣權}에 의해
상속자격을 갖는 사람이 없을 때에만 허용되었다는 것이다.
그리하여,
솔론의 법에 의해
아테네 시민들에게 처음
유언을 할 수 있는 권한이 주어졌을 때,
남성 직계비속을 상속에서 제외하는 것이 금지되었다.
마찬가지로 벵갈 지방의 유언도
가족의 어떤 우월적 권리와 부합하는 경우에만
상속을 통제할 수 있도록 허용된다.
또한, 유대인들의 초기 제도는 유언자의 특권을 전혀 알지 못했는데,
후대의 율법학자들의 법은
모세법에 누락이 있는 경우\latin{casus omissi}에만 보완한다는 명분으로
모세법상 상속자격 있는 친족들이 전혀 없거나 발견되지 않을 때에만
유언권한을 인정한다.
고대 게르만 법전들이 유언법을 수용하면서도 그것에 울타리를 쳐서 제한한 것도
의미심장하며, 동일한 방향을 지시한다.
우리에게 알려진 형태로만 볼 때 이들 게르만법 대다수는,
자유소유지\latin{allod}, 즉 가\hanja{家}의 소유지 외에,
여러 하위 유형 또는 하위 등급의 재산을 인정하는
특징을 갖고 있거니와,
이들 후자의 각각은 로마법 원리가 튜턴족의 원시적 관습체계에
개별적으로 융합되어 들어간 결과로 보인다.
원시 게르만법상의 재산, 즉 자유소유지는 전적으로 친족들에게만 주어진다.
그것은 유언 처분의 대상이 될 수 없을 뿐만 아니라,
살아있는 사람들 간\latin{inter vivos}에도 양도가 거의 불가능하다.
힌두법과 마찬가지로, 고대 게르만법에서도
아들들은 아버지와 함께 공동소유권자가 되며,
가족의 토지는 모든 구성원의 동의 없이는 처분할 수 없다.
그러나 보다 후대에 기원하며 자유소유지보다 등급이 낮았던
다른 종류의 재산은 훨씬 더 쉽게 양도될 수 있으며,
훨씬 더 느슨한 상속규칙에 따른다.
여자들과 그녀의 후손들도 그것을 상속할 수 있거니와, 이는
그것이 종족\hanja{宗族}관계의 성역\hanja{聖域} 바깥에 있다는
원리에 의한 것이 분명하다.
로마로부터 차용한 유언법이 처음 작동하도록 허용된 것은
바로 이 후자의 성격의 재산에 대해서였고, 오직 그것에 국한되었다.

\para{귀족의 유언}
이들 몇몇 사례가 주는 암시는
로마 유언법의 초기 역사에서 확인되는 사실을 사뭇 그럴 듯하게 설명하는 것으로
보이는 것에 추가적인 설득력을 제공할 수 있을 것이다.
풍부한 전거에 따르면,
로마 국가의 초기 역사 동안 유언은
코미티아 칼라타\latin{comitia calata}\footnote{굳이 번역하자면
`소집된 민회'라는 뜻이다.}에서 행하여졌다.
코미티아 칼라타는 로마 귀족 시민들의 입법기구인
쿠리아 민회\latin{comitia curiata}가 사적인 안건을 위해 소집된 것이다.
이러한 유언 방식은 대륙법학자들 사이에서 세대를 거쳐 전해져온 주장,
즉 로마 역사에서 한때 모든 유언은 장엄한 입법행위였다는 주장의
근거가 되었다.
그러나 고대 민회의 절차에 지나친 정확성을 부여하는 결함이 있는
설명에 굳이 의존해야 할 필요가 있는지 의문이다.
코미티아 칼라타에서의 유언에 관한 이야기를 풀이할 적절한 열쇠는
\hemph{무유언}상속에 관한 옛 로마법에서 찾아야 할 것이 분명하다.
친족 간의 상속을 규율하는 원시 로마법의 규칙은,
법무관의 고시법\hanja{告示法}에 의해 변경되기 전까지는,
다음과 같은 순서를 따랐다:
첫째, 가내상속인\hanjalatin{家內相續人}{sui}, 즉 부권면제되지 않은
직계비속이 상속한다.
가내상속인이 없으면 최근친 종족\latin{nearest agnate},
즉 망자와 동일한 가부장 아래 있었거나 있을 수 있었던 친족 중에
가장 가까운 사람 또는 사람들이
그 자리를 차지한다.
세 번째이자 마지막 순위로 상속재산은 씨족원들\latin{gentiles}, 즉
망자가 속한 씨족\latin{gens; house}의 공동의 구성원들에게 넘어간다.
전술했듯이 씨족은 가족의 의제\hanja{擬制}적 확장이었으니, 그것은
동일한 이름을 가진,
그리고 동일한 이름을 가졌기에 공통의 조상의 후손들이라고 믿어진,
모든 로마 귀족 시민들로 구성되었다.
쿠리아 민회라고 불린 귀족들의 집회는 그야말로 씨족들만을 대표하는 입법기구였다.
국가를 구성하는 단위가 씨족이라는 전제 위에 구성된,
로마 인민을 대표하는 집회였던 것이다.
사정이 이러하다면 다음과 같은 추론이 불가피해진다.
저 민회에 의한 유언의 확인은 씨족원들의 권리와 관련된다는 것,
그리고 그들의 최종적 상속권을 보호하려는 의도에서 이루어졌다는 것이다.
씨족원들을 발견할 수 없을 때에만,
혹은 씨족원들이 권리를 포기했을 때에만,
유언이 행해질 수 있었다고 가정한다면,
그리고 로마 씨족들의 총회에 제출된 모든 유언은
그 유언 처분에 의해 손해를 입을 사람들이 원한다면 거부권을 행사하거나
아니면 유언을 통과시킴으로써 그들의 상속권을 포기한 것으로 간주하기 위해서
제출되었다고 가정한다면,
일견 이상하게 보이던 모든 것들이 말끔히 해소된다.
12표법의 제정 직전에는 이러한 거부권이 대단히 축소되었거나
아니면 간헐적으로만 그리고 예외적으로만
행사되었을 수 있다.
하지만 코미티아 칼라타에 주어진 권한이 어떻게 점차 발달했는지 혹은
점차 쇠퇴했는지 추적하는 일보다 그것의 본래의 의미와 기원을 밝히는 일이
훨씬 더 수월하다.

하지만,
근대 유언법의 계보를 거슬러올라갈 때 만나는 유언은
코미티아 칼라타에서 행한 유언이 아니라,
그것과 경쟁하여 마침내 그것을 대체한 또 다른 유언이다.
로마 유언법의 초기 역사가 대단히 중요하고
또 그것을 통해 많은 고대 관념들을 해명할 수 있기에
다소 장황한 서술이 이어지더라도 양해하시길 바란다.

\para{평민의 유언}
유언권한이 법의 역사에 처음 등장했을 때,
로마의 거의 모든 위대한 제도들과 마찬가지로,
이것도 귀족들과 평민들 간의 투쟁의 대상이었다.
``평민은 씨족을 갖지 않는다''\latin{Plebs gentem non habet},
즉 평민은 씨족의 구성원이 될 수 없다는 정치적 격률이 뜻하는 바는
평민은 쿠리아 민회로부터 전적으로 배제되었다는 것이다.
그리하여 일부 학자들은, 평민은 귀족들의 집회에서 유언을
낭독하거나 구술할 수 없었고
따라서 유언의 특권을 완전히 박탈당했다고 주장했다.
다른 학자들은, 유언자가 대표되지 못하는 비우호적인 집회에
유언 안건을 제출해야 하는 고초를 지적하는 데 만족했다.
어느 견해가 진실이든,
불쾌한 어떤 의무를 회피할 의도에서 고안되었다고 볼 수밖에 없는
유언 형식 하나가 널리 사용되기에 이르렀다.
문제의 유언은 살아있는 사람 간의\latin{inter vivos} 양도로서,
유언자의 가\hanja{家}와 재산을 그가 상속인으로 점찍은 사람에게 양도하는
완전하고 철회불가능한 행위였다.
로마의 엄격법에 따라 이러한 양도행위는 언제나 허용되었으나,
그것이 사후\hanja{死後}적 효과를 의도하는 경우에는
귀족들의 입법기구의 공식적 승인을 받지 않고도 유효한 유언이 될 수 있는지
논란이 될 수 있었다.
이 점에 관하여 로마의 두 신분집단 간에 의견대립이 있었다 할지라도,
다른 많은 시기\hanja{猜忌}의 원인들과 함께
이것도 십인위원회\hanja{十人委員會}의\latin{decemviral}
타협에 의해 사라졌다.
``가부장이 금원과 그의 물건의 후견에 관하여 종의처분\hanja{終意處分}한 바
있으면, 그대로 법으로 한다''%
\latin{Pater familias uti de pecuniâ tutelâve rei suae legâssit, ita jus esto}는
12표법상의 텍스트가 우리에게 전해지거니와,
이 조항은
평민의 유언을 합법화하는 것 이외의 목적을 가졌다고 보기 어렵다.

귀족들의 집회가 로마 국가의 입법기구임을 그치고 수 세기가 지나서도
여전히 사적인 안건을 처리하기 위해 그것이 공식적으로 개최되었다는 것은
학자들 사이에 잘 알려져 있다.
결과적으로, 12표법이 공표되고 나서도 오랫동안
유언의 확인을 위해 코미티아 칼라타가 소집되었다고 믿을 이유가 있는 것이다.
아마도 그것의 기능은
그것이 유언등기소\latin{court of registration}였다는
말로써 가장 잘 표현될 수 있을 것이다.
하지만 이 말은 제출된 유언이 \hemph{대장에 기록}되었다는\latin{enrolled}
뜻이 아니라,
단지 참가자들에게 구술하여 그들이 그 취지를 이해하고 기억하도록 하는 데
그쳤음에 유의해야 한다.
이러한 형식의 유언은 문서로 작성되지 않았을 것이 거의 확실하며,
설령 유언이 애초 문서화되었다 할지라도, 민회의 임무는 분명
그것을 큰 소리로 낭독하는 것을 듣는 것에 그쳤을 것이니,
그후 그 문서는 유언자가 보관하거나 아니면
어느 신전\hanja{神殿}의 보호에 맡겨졌을 것이다.
코미티아 칼라타에서의 유언의 한 측면인
이러한 공개성은 대중들이 그것을 꺼리는 원인이 되었다.
제정 초기에도 저 민회는 개최되었으나,
단순히 형식적인 것으로 전락하였던 듯하고, 아마도
정기집회에 제출되는 유언은 거의 또는 전혀 없었을 것이다.

\para{악취행위}
근대 세계의 문명을 크게 바꾼 장기적 영향력을 가진 것은
고대 평민의 유언---방금 기술한 유언의 대체물---이었다.
그것은 코미티아 칼라타에 제출되는 유언이 상실한 인기를 고스란히 획득했다.
그것의 성격을 이해하는 열쇠는
고대 로마의 양도방식인
악취행위\hanjalatin{握取行爲}{mancipium}에서
그것이
유래했다는 데 있다.
악취행위는 근대사회에서는 하나로 연결지어 생각하기 힘든
두 가지 위대한 제도, 즉 계약과 유언의 모태라고 서슴없이 말할 수 있는 절차이다.
후대 라틴어에서 만키파티오\latin{mancipation}라고 불리게 되는
악취행위의 제반 측면들은 우리를 국가사회의 유년기로 이끌고 간다.
문자의 발명까지는 아니더라도 어쨌든 문자의 대중화 이전으로
그 기원을 소급하기에,
몸짓과 상징적 행위와 장엄한 어구\hanja{語句}가 문서의 형식을 대신했다.
길고 복잡한 의식\hanja{儀式}은 거래의 중요성에 대한
당사자들의 주의를 환기시키는 동시에 증인들의 기억에 각인을 남기려는 것이었다.
또한, 문서화된 증거에 비해 불완전할 수밖에 없는 구두\hanja{口頭}절차였기에,
후대 사람들이 적절하다고 생각하는 또는 한계라고 생각하는 선 이상으로 많은
증인들과 보조자들이 필요했다.

로마의 악취행위에는 우선 당사자 모두, 즉 매도인과 매수인이,
혹은 오늘날의 법률용어로는 양도인과 양수인이라고 불러야 할 사람들이
참가해야 했다.
또한 적어도 \hemph{다섯 명}의 증인들과 더불어 좀 특이한 인물인
저울소지자\latin{libripens}가 필요했다.
저울소지자는 고대 로마의 주조되지 않은 구리 화폐의 무게를 다는
천칭을 가지고 왔다.
우리가 다루는 유언---오랫동안
`구리와 저울에 의한\latin{per aes et libram} 유언'이라고
법기술적으로 불리어온 유언---은
통상적인 악취행위와 형식에 있어서 동일했고
언표하는 내용도 거의 다르지 않았다.
유언자가 양도인이 된다.
다섯 명의 증인과 저울소지자도 현장에 나와있다.
양수인의 자리에는 법기술적으로
`가\hanja{家}의 매수인'\latin{familiae emptor}이라고 불리던 사람이 선다.
이제 악취행위의 통상적인 의식이 거행된다.
어떤 형식적인 몸짓들이 행해지고 형식적인 문장들이 선언된다.
가의 매수인이
구리 화폐 조각으로 저울을 쳐서 대금을 지불하는 행위를 흉내낸다.
끝으로 유언자가
거래의 공표에 해당하는 ``언명''\hanjalatin{言明}{nuncupatio}이라
불리는 일련의 형식적인 말로써 지금까지 행하여진 것을 승인한다.
법률가들에게는 상기시킬 필요가 없겠지만, 이 언명은
유언법에서 장구한 역사를 가지고 있다.
특히 `가의 매수인'이라고 불리는 사람의 성격에 주목할 필요가 있다.
처음에는 그가 상속인 자신이었다는 데 의심의 여지가 없다.
유언자는 그에게 ``가\hanja{家}'' 전체, 즉
가에 대해 그리고 가를 통해 유언자가 향유하는 일체의 권리를 완전히 양도했다.
그의 재산, 그의 노예, 선조에게 물려받은 그의 모든 특권을,
다른 한편으로 그의 모든 의무 및 채무와 더불어, 함께 양도했던 것이다.

\para{양도로서의 유언}
이러한 자료를 앞에 두고,
우리는---이렇게 부를 수 있다면---`악취행위에 의한 유언'이
그 원시적 형태에 있어서 근대의 유언과 어떻게 다른지 몇 가지 주목할 점을
지적할 수 있다.
그것은 유언자의 가산을 아주 양도해버리는 것이므로
\hemph{철회가능}하지 않았다.
이미 소진해버린 권한을 새로이 행사할 수는 없었던 것이다.

또한 그것은 비밀성이 없었다.
가의 매수인은 자신이 상속인이면서도 그의 권리가 무엇인지 정확히 알았고,
상속재산에 대한 권원을 불가역적으로 가지게 되었음을 알았다.
가장 질서잡힌 고대사회라 하더라도 없을 수 없는 폭력이 이 지식을 대단히
위험한 것으로 만들었다.
그러나 아마도 양도에 대한 유언의 이러한 관계가 가져오는 가장 놀라운 결과는
상속인에게 상속재산이 즉시 주어진다는 점일 것이다.
적지 않은 대륙법학자들에게 이것은 너무도 믿을 수 없는 일이었기에,
그들은 유언자의 가산이 유언자의 사망을 조건으로 하여 주어졌다거나,
불특정 시점부터, 즉 양도인의 사망시부터 주어졌다고 말해왔다.
하지만 로마법의 마지막에 이를 때까지,
조건에 의해 직접 변경될 수 없는, 혹은
어떤 시점까지 또는 어떤 시점부터라는 제한이 있을 수 없는,
거래의 유형이 존재했다.
법기술적 용어로는 조건\latin{conditio}이나
기한\latin{dies}이 붙을 수 없는 거래들이 있었던 것이다.
악취행위가 바로 그런 거래의 하나였다.
따라서, 이상하게 보일지 몰라도, 우리는 초기 로마의 유언은,
비록 유언자가 자신의 유언행위 이후에 오래 산다고 하더라도,
즉시 효력을 발생했다고 결론짓지 않을 수 없다.
어쩌면 사실 로마 시민들은 원래 사망에 임박해서만 유언을 했을 것이고,
한창 나이의 남자가 가의 연속성을 위한 대비를 할 때는
유언이 아니라 입양의 형식을 취했을 것이라고 추정할 수 있다.
그렇지만, 만약 유언자가 건강을 회복하였다 해도,
그는 상속인의 묵인 하에서만 그의 가를 계속 지배할 수 있었을 것이다.

\para{고대 유언의 비서면성}
어떻게 해서 이러한 불편함이 치유되었는지,
어떻게 해서 유언이 오늘날 널리 부여되는 성격을 가지게 되었는지
설명하기 전에
두 세 가지 먼저 말해둘 것이 있다.
유언은 문서화될 필요가 없었다:
처음에는 언제나 구두\hanja{口頭}였던 것으로 보이며,
나중에도 유증을 선언하는 문서는 유언에 부수적인 요소였을 뿐,
본질적 구성요소를 이루지는 않았다.
그것의 유언에 대한 관계는
옛 영국법에서
종국화해\hanjalatin{終局和解}{fine}나
공모회수소송\hanjalatin{共謀回收訴訟}{recovery}의
이용을 이끄는 날인증서\latin{deed leading the uses}가
종국화해와
공모회수소송에
대해 가지는 관계,\footnote{%
  종국화해(final concord; fine)와 공모회수소송(common recovery; recovery)은
  확실하고 완전한 소유권을 양도하기 위한
  공모소송(collusive action)의 방식들로서,
  전자는 재판상화해의, 후자는 판결의 형식을 취한다.
  1833년 `종국화해 및 공모회수소송에 관한 법률'(Fines and Recoveries Act)에
  의해 모두 폐지되었다.
  한편, 이들 공모소송과 관련하여 몇몇 날인증서가 작성되었는데,
  `종국화해의 이용을 이끄는 날인증서'(deed to lead the uses of a fine)와
  `공모회수소송의 이용을 이끄는
  날인증서'(deed to lead the uses of a common recovery)가 대표적이다.
  이들 증서에는 왜 이러한 공모소송을 이용하려는지 그 목적이 제시된다.
}
또는
토지보유권양도날인증서\latin{charter of feoffment}가
토지보유권양도 자체에 대해 가지는 관계\footnote{%
  원래 토지보유권은 어떤 상징적 행동과 언명에 의해 양도되었는데,
  이를 확인하는 날인증서의 작성 또한 관행상 널리 행하여졌다.
  하지만 1677년 사기방지법(Statute of Frauds) 제정 이전에는
  날인증서의 작성이 필수요건은 아니었다.
}와
정확히 일치한다.
실로 12표법 이전에는 문서가 조금도 이용되지 않았을 것이니,
유언자에게는 유증\latin{legacy}할 권리가 없었고,
유언으로 이익보는 자는 상속인 또는 공동상속인들에 국한되었기 때문이다.
하지만 12표법 조문의 극단적 일반성으로 인해
곧이어
상속인은 유언자의 지시를 이행할 부담을 안고서,
다시 말해 유증의 부담을 안고서,
상속재산을 취득해야 한다는 법리가
형성되었다.
따라서 서면으로 작성된 유언장은
수유자\hanjalatin{受遺者}{legatee}의 권리를 침해하는 상속인의 기망행위를
방지한다는 새로운 가치를 띠게 되었다.
그러나
증인들의 증언에만 의존할 것인가, 즉
가의 매수인이 지불해야 할 유증의 선언을 말로써 할 것인가 여부는
마지막까지도 유언자의 재량에 맡겨져 있었다.

\para{가의 매수인}
`가의 매수인'\latin{emptor familiae}이라는 용어는 특별히 주목을 요한다.
``매수인''\latin{emptor}은 유언이 글자 그대로 매매였음을 의미한다.
``가\hanja{家}''라는 단어는,
12표법의 유언 관련 조문의 표현에 비추어볼 때,
시사하는 바가 적지 않다.
고전 라틴어에서 ``가\hanja{家}''는 항상 어떤 사람의 노예를 뜻한다.
하지만 여기서는, 그리고 고대 로마법의 일반적 용법에서는,
그것은 그의 가부장권에 복속하는 모든 사람을 포함하는 의미였으며,
유언자의 물질적 재산은 그의 가\hanja{家}에 부수하는 부속물로서
이전된다고 이해된다.
다시 2표법으로 돌아가면, ``그의 물건의 후견''\latin{tutela rei suae}이라는
표현이 등장하거니와, 이는 방금 설명한 용어를 정확히 거꾸로 뒤집은 표현형태이다.
따라서,
비교적 늦은 시기인
십인위원회의 타협의 시기에도,
``가''를 지칭하는 용어와 ``재산''을 지칭하는 용어가 당대의 용법에서
서로 혼재되어 쓰였다는
결론을 피해가기는 불가능해보인다.
만약 어떤 사람의 가\hanja{家}가 그의 재산이라고 말하여진다면,
이 표현은 가부장권의 범위를 가리키는 말로 이해할 수 있을 것이다.
그러나, 거꾸로 바꾸어 쓸 수도 있는 것이기에,
저 표현형태는
재산은 가족에 의해 소유되고 가족은 시민에 의해 지배되는,
그리하여 공동체의 구성원은 재산\hemph{과} 가족을 소유하는 것이 아니라,
가족을 \hemph{통하여} 재산을 소유하는,
그러한 원시적 시기를 우리에게 시사한다고
인정하지 않을 수 없는 것이다.

\para{법무관법의 유언}
정확히 언제부터인지는 알 수 없지만,
로마의 법무관들은
엄격한 형식요건을 요하는 유언을
법의 문언보다는 법의 정신에 더 부합하게
취급하기 시작했다.
그때그때의 처리가 어느새 확립된 관행이 되어갔고,
마침내 완전히 새로운 유언의 형태가 자라나
꾸준히 고시법\hanja{告示法}에 접목되어 들어갔다.
이 새로운, \hemph{법무관법의} 유언\latin{praetorian testament}은 그 견실함을
오직 명예관법\hanjalatin{名譽官法}{jus honorarium}, 즉 로마의 형평법에
빚지고 있었다.
어느 해, 신임 법무관은 취임시 선포되는 자신의 고시에
이러저러한 형식요건들을 갖춘 유언은 모두 인정하겠노라는 뜻을 담은
조항 하나를 삽입했을 것이다.
이 개혁조치가 유익한 것으로 판명되자, 관련 조항은
차기 법무관에 의해 재차 도입되었을 것이며,
후임자들에 의해서도 반복되어, 마침내
이러한 연속적 포함 덕분에 영구고시록\latin{Perpetual Edict}이라고 불리게 되는
법체계의 공인된 일부를 형성하게 되었다.
법무관법 유언이 유효하기 위한 요건을 조사해보면,
그것은 악취행위에 의한 유언의 요건에 기초하고 있었음이 명백히 드러날 것이다.
저 혁신가 법무관은 옛 형식요건들 가운데 진정성을 담보할 수 있거나
기망행위를 방지할 수 있는 것들만 보존하기로 결정하였을 것이 분명하다.
악취행위에 의한 유언에서는 유언자 외에 7명의 사람들이 현장에 나와야 했다.
따라서 법무관법 유언에도 7명의 증인이 요구되었다.
그중 두 명은 원래 저울소지자\latin{libripens}와
가의 매수인\latin{familiae emptor}이었으나 이제 이들의 상징적 성격은
제거되고 단지 증인으로서의 역할만 담당하게 되었다.
각종 상징적 의식절차도 사라졌다.
유언이 구술\hanja{口述}될 뿐이었다. 그렇지만 아마도
\paren{전적으로 확실한 것은 아니지만}
유언자의 처분에 관한 증거를 영구화하기 위해 서면이 필요했을 것이다.
어쨌거나, 서면이 유언자의 마지막 의사로서 읽혀지거나 제시된 경우,
7명의 증인들이 각자 그 겉봉에 자신의 인장을 날인하지 않았다면
법무관의 법정이
특별히 개입하여
그 효력을 인정하지 않았을 것임을 우리는 잘 알고 있다.
이것은
\hemph{날인}\latin{sealing}이
법의 역사에서
인증\hanja{認證}의 수단으로
처음 등장하는 사례이다.
하지만
단순히 잠금장치로서 날인이 사용된 것은 물론 훨씬 더 오래 전의 일이며,
히브리인들에게도 알려져있었던 듯하다.
로마인들에게 유언장 또는 다른 중요한 문서의 날인은
날인한 자의 참석과 동의의 지표로서 기능했을 뿐만 아니라,
또한
나중에 서면을 조사하기 전에 깨뜨려야 할, 말그대로 잠금장치이기도 했음을
알 수 있다.

\para{유산점유}
그리하여,
악취행위의 형식을 통해 거행되지 않고
단지 7명의 증인의 날인에 의해 입증되는 경우,
그러한 유언자의 처분은 고시법이 강제할 수 있게 되었다.
그러나
로마인의 재산의 주요 속성들은
시민법과 그 기원을 함께 한다고 여겨진 절차를 통하지 아니하고는
양도될 수 없었다는 것이 일반 명제로 제시될 수 있을 것이다.
따라서 법무관은 그 누구에게도 \hemph{상속재산}을 수여할 수는 없었다.
유언자가 자신의 권리와 의무에 대해 가졌던 관계와 동일한 관계를
상속인이나 공동상속인들에게 줄 수는 없었던 것이다.
법무관이 할 수 있는 것이라고는
물려받은 재산의 사실상의 향유권을
상속인으로 지명된 자에게
주는 것과
유언자의 채무에 대한 그의 변제에 법적인 효력을 인정하는 것이
전부였다.
이러한 목적을 위해 법무관이 권한을 행사했을 때,
법기술적으로는
`유산점유'\latin{bonorum possessio}를 수여했다고
표현되었다.
이 경우 상속인에 해당하는 자, 즉 유산점유자\latin{bonorum possessor}는
시민법상의 상속인이 누리는 모든 재산법상의 특권을 가지고 있었다.
과실\hanja{果實} 수취도 할 수 있었고, 양도도 할 수 있었다.
하지만 피해를 구제받기 위해서는 법무관 법정의,
이런 표현을 쓸 수 있다면, 보통법적 측면이 아니라
형평법적 측면에 호소해야 했다.
그를 상속재산에 대한 \hemph{형평법상의}\latin{equitable} 소유자라고
부르더라도 크게 잘못된 표현은 아닐 것이다.
그렇지만,
이러한 유추가 불러올 수 있는 오해를 불식시키기 위해
한 가지 반드시 유념해야 하는 점은,
1년이 지나면
유산점유가
로마법상 사용취득\latin{usucapion}이라 불린 원리의 적용을 받는다는 것이다.
그리하여
점유자는 상속재산에 속하는 모든 재산에 대해
로마시민법상의\latin{quiritarian} 소유권자가 되었다.

\para{옛 유언의 진화}
옛 민사소송법에 대해 우리가 가진 지식이 얕은 수준에 머물러 있기에,
법무관 법정이 제공한 구제수단의 다양한 유형들 간의 장점과 단점을
균형있게 파악하기가 쉽지 않은 것이 현실이다.
하지만 한 가지 확실한 점은,
포괄적 재산을 한꺼번에 고스란히 이전하는 악취행위에 의한 유언은
그 모든 결함에도 불구하고
새로운 유언에 의해 결코 완전히 대체되지 않았다는 것이다.
옛 방식에 대한 집착이 완화된 이후에도,
옛 방식의 의미가 어느 정도 생동감을 잃은 이후에도,
법학자들의 재능은 보다 유서 깊은 유언 수단을 개량하는 데
집중되었던 것으로 보인다.
가이우스의 시대, 즉 안토니누스 황조 시대에 이르면
악취행위에 의한 유언을 둘러싼 주요 결함들이 사라지게 된다.
전술했듯이, 원래 이 방식의 본질적 성격은
상속인 자신이 가의 매수인이 될 것을 요구했고,
따라서 상속인은
유언자의 재산에 속한 기존 권리의무를 즉시 취득했을 뿐만 아니라,
자신의 권리가 무엇인지도 공식적으로 알 수 있었다.
하지만 가이우스의 시대에는
이해관계 없는 사람이 가의 매수인의 역할을 할 수 있었다.
그리하여 실제 상속인에게는 그가 받게 될 상속에 대해 굳이 알려줄 필요가 없었으니,
이후로는 유언이 \hemph{비밀성}을 획득하게 되었다.
이렇게 실제 상속인 대신에 국외자가 ``가의 매수인'' 기능을 담당함으로써
먼 훗날 또 다른 결과도 생겨났다.
이것이 합법화되자 로마의 유언은 두 부분 또는 두 단계로
구성되는 것이 되었다.
하나는 순수한 형식이었던 양도\latin{conveyance}이고,
다른 하나는 언명\latin{nuncupatio}, 즉 공표이다.
후자의 절차단계에서 유언자는
자신의 사후에 무엇이 행해져야 할지 의사를 보조자들에게 구두로 선언하거나,
아니면 자신의 의사가 담겨있는 문서를 제출했다.
거래의 핵심 부분에 주어지는 관심이
가상의 양도로부터 멀어지고
언명에 집중되자,
이제 유언은 \hemph{철회가능한} 것이 될 수 있었다.

지금까지 법사\hanja{法史}를 따라 내려오면서 유언의 계보를 살펴보았다.
그것의 뿌리는 악취행위, 즉 양도에 기초한 ``구리와 저울에 의한'' 유언이다.
하지만 이 고대의 유언은 다수의 결함을 가지고 있었고,
그것은 법무관법에 의해 간접적으로만 교정될 수 있었을 뿐이다.
그러는 동안, 재능있는 법학자들은
법무관들이 형평법을 통해 동시대에 수행해온 것과 같은 개량을
보통법적 유언, 즉 악취행위에 의한 유언에 대해 수행했다.
하지만 이러한 개량은 단지 법적인 재간에 의존한 것이었기에,
가이우스나 울피아누스의 시대의 유언법은 과도기적인 것에 불과했다.
그후의 변화과정에 대해서 우리는 잘 알지 못한다.
그러나 마침내 유스티니아누스에 의한 법학의 재건이 있기 직전
동로마제국의 백성들이 사용하고 있던 유언의 형태는
그 계보를 한편으로는 법무관법의 유언에,
다른 한편으로는 ``구리와 저울에 의한'' 유언에 소급할 수 있는
것이었다.
법무관법의 유언처럼, 그것은 악취행위를 요구하지 않았고,
7명의 증인들의 날인이 없으면 무효였다.
악취행위에 의한 유언처럼, 그것은 단순히 유산점유가 아니라
상속재산을 이전하는 것이었다.
하지만 그것의 주요 특징의 일부는 실정적 입법에 의해 추가된 것이었다.
이렇게 법무관의 고시, 시민법, 그리고 황제의 칙법이라는
세 가지 기원을 가진다는 의미에서
유스티니아누스는 당시의 유언법을
`삼중의 법'(jus tripertium)이라고 불렀던 것이다.\footnote{Inst.\,2.10.3.}
방금 언급한 이 새로운 유언이
로마인의 유언이라고
일반적으로
알려져 있는 것이다.
그러나 그것은 동로마제국에 국한된 것이었다.
사비니의 연구가 밝혀놓은 것처럼,
서유럽에서는 옛 악취행위에 의한 유언이,
양도, 구리, 저울 등 그것의 모든 장치들과 함께,
중세에 들어서도 한동안 계속 사용되었다.


\chapter{고대와 근대의 유언 및 상속에 관한 관념}

근대 유럽의 상속법에는
역사 초기에 사람들 사이에 준행된 유언 처분의 규칙과
밀접한 관련된 것들이 많이 있지만,
유언과 상속이라는 주제에 대한 고대와 근대의 관념들 간에는
몇 가지 중요한 차이점들이 있다.
이번 장에서는 그러한 몇 가지 차이점들을 설명해보고자 한다.

\para{자녀의 상속제외}
12표법이 제정되고 몇 세기가 흐른 시점에 이르면,
자녀를 상속에서 제외하는 것을 제한하는
여러 법규칙들이
로마 시민법에
접목되어 들어가 있었다.
이러한 관심에서 법무관의 재판권이 적극적으로 행사되고 있었으며,
또한
아버지의 유언에 의해 상속에서 부당하게 제외된 자녀에게
상속재산을 회복시켜주는,
``배륜유언\hanja{背倫遺言}의
소\hanja{訴}''\latin{querela inofficiosi testamenti}라고 불리는,
성질이 특이하고 기원이 모호한 새로운 구제수단이 제공되고 있었다.\footnote{%
  이유 없이 상속제외되거나
  법정상속분의 1/4에 미달하는 재산을 물려받은 경우
  지정상속인들을 상대로 이 소송을 제기하여 자신의 법정상속분을
  회복할 수 있었다. }
이러한 법상태를
최대한의 유언 자유를 문언상 인정하는 12표법의 텍스트와 비교하면서,
몇몇 학자들은 로마 유언법에 어떤 커다란 극적인 변화가 있었다고
추정하기도 한다.
그들에 따르면,
가족의 수장들은 무제한인 상속제외의 권한을 즉각 남용하기 시작했고,
이 새로운 관행이 대중의 도덕감정을 침해하여 스캔들을 일으켰으며,
가부장의 악행을 저지하는 법무관들의 용기에 모든 선량한 사람들이
환호를 보냈을 것이라고 한다.
이러한 이야기는,
관련 주요 사실에 비추어 전혀 근거가 없지는 않으나,
법제사의 기본 원리에 대한 자못 심각한 오해를 노정한다고
일반적으로 평가된다.
\wi{12표법}은 제정될 당시의 시대적 성격에 비추어 설명되어야 한다.
12표법은 후대에 반작용을 불러올 어떤 경향을 허용하려고 만들어진 것이 아니다.
그것은 그러한 경향이 존재하지 않는다는 가정 하에,
아니 어쩌면 그러한 경향의 존재 가능성조차 인식하지 못한 채,
만들어진 것이다.
로마 시민들이 상속제외의 권한을 즉시 자유롭게 이용하기 시작했다는 것은
개연성이 없다.
가족이라는 굴레의 멍에가
여전히 끈질기게 추종되고 있고
가혹하게 강제되고 있는 곳에서,
오늘날에도 환영받지 못할 일로써
그 멍에를 벗어던진다는 것은
역사의 모든 합리성과 건전한 판단에 역행하는 것이다.
12표법은 유언이 실행될 수 있을 경우에만,
즉 자식들과 근친들이 없는 경우에만,
유언이
실행되도록 허용한 것이다.
직계비속의 상속제외를 금지하지 않은 것은
당대의 로마 입법자가 생각할 수 없었던 일에 대비한 입법을
하지 않은 것에 불과하다.
물론, 가족애의 의무가 점차 개인의 일차적 의무인 측면을 상실하면서,
자식의 상속제외도 이따금 행해졌을 것이다.
하지만 법무관의 개입은,
남용이 보편적이었기 때문에 촉발된 것이 아니라,
애초 그러한 부자연스런 변덕이 소수의 예외적인 사례였고
당대의 도덕감정에 반하였기 때문에 촉발되어졌음에 틀림없다.

\para{로마의 무유언상속}
로마 유언법의 이 부분이 시사하는 것은 전혀 다른 종류의 것이다.
로마인들은 유언을 가족을 상속제외하는 수단이나
재산을 불균등 분배하는 수단으로 보지 않았을 것이라는 데 주목해야 한다.
유언법이 전개되어감에 따라
그러한 목적에 유언이 사용되는 것을 방지하는 법규칙들이
증가하고 강화되었던 것이다.
이러한 법규칙들은,
개인의 일시적 감정 변화와는 구분되는,
로마 사회의 불변의 감정에 분명 부합하는 것이었다.
유언 권한의 주된 가치는 가족을 위해 \hemph{대비를 하는} 데에,
무유언상속법이 나누어주는 것보다 더 균등하고 공정하게
상속재산을 나누어주는 데에,
기여하는 것이었다고 생각된다.
대중의 일반적 감정을 이렇게 읽는 것이 옳다면,
이것은 로마인들이 줄곧 가졌던 무유언상속에 대한 특유의 두려움을
어느 정도 설명할 수 있을 것이다.
유언의 특권을 박탈당하는 것보다 더 큰 불행은 없다고 생각되었던 것이다.
적에게 퍼부을 수 있는 저주 가운데
유언 없이 죽으라는 말보다 더 심한 저주는 없다고 생각되었던 것이다.
오늘날의 여론에는 이것에 해당하는 감정이 아예 없거나 찾아보기가 어렵다.
물론 어느 시대 어느 누구든
자신의 임무를 법이 대신 수행해주는 것보다
스스로 자신의 재산의 운명을 설계하는 것을
더 선호할 것이다.
그러나 유언을 향한 로마인들의 열정은 그 강도\hanja{强度}에 있어
단순히 자의로 처분하고자 하는 욕구와는 차원을 달리하는 것이었다.
또한 그것은
\wi{봉건제}의 산물에 불과한,
재산의 한 종류를 단일한 대표자의 수중에 모아두어
가족의 위신을 지키려는 것과도
물론 아무런 공통점이 없었다.
추론컨대,
무유언상속법에 들어있는 무언가가
법에 의한 재산분배보다
유언에 의한 재산분배를
이렇게 강렬하게
선호하게 만들었던 것 같다.
하지만 문제는,
근대 입법자들이 거의 보편적으로 받아들인
유스티니아누스의 상속법 정비 이전으로 수 세기를
거슬러올라간 시대의
무유언 상속에 관한 로마법을 일별해 보더라도,
눈에 띄게 불합리하다거나 불공정한 부분을 발견하기가 어렵다는 것이다.
오히려 법이 예정한 분배 규칙이 사뭇 공정하고 합리적이었고
근대사회가 널리 만족하고 있는 것과 별반 다르지 않았음에도,
무엇보다
보살필 자식을 둔 사람의 유언 권한이
법에 의해 이미 좁게 제한되고 있었음에도,
왜 그렇게 그것이 특별히 꺼리는 대상이 되어야 했는지 이해하기 어려운 것이다.
차라리, 오늘날의 프랑스처럼,
일반적으로 가족의 수장은 수고스럽게 유언장을 작성하느니
법이 예정한대로 재산이 분배되도록 내버려두는 현상까지도
기대할 수 있을 법한 지경이었다.
하지만 생각건대,
유스티니아누스 이전 시대의 무유언상속의 척도를 좀 더 면밀히 들여다보면
이 미스테리를 해결할 열쇠를 발견할 수 있을 것이다.
당시의 법은 두 가지 독립된 부분으로 직조되어 있었다.
하나는 로마의 보통법이라 할 수 있는 \wi{시민법}\latin{ius civile}에서
유래한 법규칙들이고, 다른 하나는 \wi{법무관}의 고시에서 유래한 것들이었다.
앞서 다른 목적에서 언급했듯이,
시민법은 오직 세 단계의 상속순위를 차례로 인정할 뿐이다.
\wi{부권면제}\hanja{父權免除}되지 않은 자녀, 최근친 \wi{종족}\hanja{宗族},
그리고 \wi{씨족}원들\latin{gentiles}이 그것이다.
법무관은
이들 세 가지 상속인 집단들 사이에
시민법에서 전혀 고려되지 않는
여러 친족 집단들을 끼워넣는다.
나중에
이러한 고시법과 이러한 시민법이 결합하여
근대 법전들에 널리 전해진 것과 사실상 다르지 않은 상속순위를
형성하게 되는 것이다.

\para{유언과 자연적 감정}
한때 무유언상속이 시민법으로만 규율되고
고시법은 아직 존재하지 않던, 또는 일관되게 적용되지 않던
시기가 있었다는 것을 기억해야 한다.
유년기의 법무관법은 강력한 저항에 부딪쳐 고전하였을 것이 틀림없다.
또한
대중의 여론과 법감정이 법무관법을 용인한 이후로도 오랫동안,
그것이 정기적으로 도입하는 변경은
확고한 원리에 의해 지배되기보다는
개별 법무관들의 서로 다른 견해에 따라 동요하였을 것이 거의 확실하다.
이 시기의 로마인들이 적용한 무유언상속법은, 생각건대,
로마 사회가 그렇게 오랫동안 유지했던 무유언상속에 대한 강렬한 혐오를
설명---아니, 설명 이상의 것을---할 수 있을 것이다.
상속의 순위는 다음과 같았다:
로마 시민이 유언 없이 또는 무효인 유언만 남기고 사망하면,
\wi{부권면제}되지 않은 자녀가 그의 상속인이 된다.
\hemph{부권면제된} 아들은 상속분이 전혀 없는 것이다.
사망시 살아있는 직계비속을 남기지 않았다면,
최근친 \wi{종족}\hanja{宗族}이 상속한다.
망자와 \paren{아무리 가깝더라도} 여계\hanja{女系}로 이어지는 친족은
한푼도 받지 못하는 것이다.
종족 이외의 다른 모든 가계도상의 가지\hanja{[枝]}들은 배제되고,
결국 상속재산은 최종적으로 씨족원들\latin{gentiles}, 즉
망자와 동일한 이름을 가진 로마 시민들의 집단 전체에게 귀속된다.
그리하여, 유효한 유언이 행해지지 않았다면,
지금 고찰대상인 시대의 로마인은 부권면제된 자식에게 생계를 위해
아무 것도 남겨주지 못하게 된다.
또한 만약 자식 없이 사망한다면,
그의 재산은 가족을 완전히 떠나, 단지
동일 \wi{씨족}\latin{gens}의 모든 구성원을 공통 조상의 후손으로 만드는
제사\hanja{祭祀}라는 의제\hanja{擬制}에 의해서만 연결될 뿐인
다수의 사람들에게 넘어가게 될 위험이 즉각 발생한다.
바로 이 문제점만으로도 저 대중적 감정에 대한 거의 충분한 설명이
될 수 있을 것이다.
그러나 실은,
독립된 가\hanja{家}들로 구성된 원시적 구조로부터
로마 사회가 막 벗어나기 시작한 바로 그 단계에
내가 서술해온 사회 상태가
놓여있었을 것이라는 점을
생각하지 못한다면
아직 우리는 절반만 이해한 것에 불과하다.
사실,
부권면제가 합법적 관행으로 인정받으면서
가부장의 제국에
최초의 일격이 가해졌다.
그러나 가부장을 여전히 가족관계의 뿌리로 간주했던 법은
\wi{부권면제}된 자식을 친족적 권리에 있어 남으로,
혈연에 있어 이방인으로 취급하고 있었다.\footnote{부권면제된 자식은
  종족(宗族)에 속하지 않고 단지 혈족(血族)에 속할 뿐이었다.}
하지만, 가족범위의 한계가 법기술적으로 부과되었다고 해서
부모의 자연적 감정에도 동일한 한계가 있었다고는 조금도 생각할 수 없는 것이다.
\wi{가부장제} 아래에는
가족애가 거의 무한한 신성함과 강력함을 유지한 채
흐르고 있었다.
부권면제 행위에 의해 가족애까지 소멸된다는 것은
상상하기 어려우며, 오히려 완전히 그 반대였을 확률이 높다.
부권으로부터의 해방이 친애\hanja{親愛}의 단절이기는커녕
표현---가장 사랑하고 가장 우대하는 자식에 대한 은혜와 호의의 표시---이라고
주저없이 인정할 수 있는 것이다.
이렇게 다른 자식들보다 존중받는 아들이 무유언상속에 의해
유산에서 전적으로 배제된다면,
이것을 꺼리게 되는 것은 더 이상의 설명을 요하지 아니한다.
지금까지
우리는
무유언상속법이 초래하는 어떤 도덕적 부정의로 인해
유언상속에 대한 열정이 생겨났다는 추론을 진행해왔다.
그리고 무유언상속법이 바로 저 초기사회를 묶어주었던 본능과
배치되고 있었음을 알게 되었다.
지금까지 주장한 것을 어떤 간명한 행태로 요약할 수 있을 것이다.
원시 로마인들을 지배한 모든 감정은
가족관계와 밀접하게 엮여있었다.
그런데 어떤 가족을 말하는가?
법은 이 가족을, 자연적 감정은 저 가족을 말하고 있었다.
양자 간의 갈등에서 우리의 분석대상인 열정이 자라난 것이다.
그것은
친애가 지시하는 바대로 목적 재산의 운명을 정할 수 있게 한 제도에 대한 열정의
형태로 나타났다.

따라서 나는 무유언상속에 대한 로마인들의 두려움을
가족에 관한 고대법과 서서히 변화할 뿐인 고대적 감정 간의 옛 갈등의 유물이라고
이해한다.
몇몇 로마 제정법들이, 그중에서도 특히
여자의 상속권을 제한하는 법률 하나가,\footnote{%
  기원전 169년의 보코니우스 법(Lex Voconia)을 말하는 듯하다.
  이는 10만 아스 이상의 재산을 가진 자는 여자를 상속인으로 삼을 수
  없도록 한 법률이다. \latin{Gai.\,2.274.}}
통과됨으로써
저 열정이 계속 유지되는 데 기여하였을 것이다.
그리고 이러한 제정법들이 부과한 금지를 회피하기 위해
신탁유증\hanjalatin{信託遺贈}{fideicommissa}%
\footnote{시민법상의 유증(legatum)이 갖는 엄격한 제한을 회피하기 위해
  널리 이용된 것으로, 상속재산의 일부 또는 전부를 제3자에게 이전하도록
  상속인에게 지시했다.
  이로써 가령 시민이 아닌 자에게 유증하거나,
  상속인의 상속인을 지정할 수 있었다.
  원래는 법적 강제력이 없었으나,
  아우구스투스 시절부터는 비상심리절차를 이용해 강제할 수 있었다.
  나중에는 유언을 대체하는 비공식적 유언으로 기능하기도 했다.
  유스티니아누스에 의해 유증과 신탁유증은 하나로 통합된다. }%
이라는 제도가 고안되었다는 것이 정설이다.
하지만 저 열정의 특별한 강렬함에 비추어볼 때,
그것은 법과 여론 사이의 어떤 더 깊은 대립관계를 지시하고 있는 듯하다.
그러니 법무관법의 발달에 의해 저 열정이 완전히 소멸하지 않은 것도
놀라운 일이 아니다.
여론의 철학에 능통한 사람이라면
어떤 감정을 만들어낸 상황이 사라졌다고 해서 반드시
그 감정까지 소멸하는 것은 아님을 잘 알 것이다.
감정은 더 오래 살아남을 수 있다.
아니, 상황이 아직 지속될 때는 볼 수 없었던 최고도의 강렬함이
상황이 사라진 후에 나타날 수 있다.

\para{중세의 유언, 프랑스의 유언법}
재산을 가족 바깥으로 유출하는 권한,
혹은
유언자의 임의대로
불균등하게 재산을 분배하는 권한으로서의 유언 관념은
\wi{봉건제}가 완전히 공고해진 중세 후반기에 이르러 비로소 등장한다.
근대법이 처음 그 거친 모습을 드러내기 시작했을 때,
유언법은 망자의 재산 처분의 절대적 자유를 거의 인정하지 않았다.
이 시기, 유언에 따른 재산의 승계가 인정되는 지역 어디서나---유럽
대부분 지역에서 동산 또는 인적재산\hanja{人的財産}은 유언 처분의 대상이 될 수
있었다---이전되는 재산 중 과부에게 주어지는 일정한 몫의 권리와
자식들에게 주어지는 일정한 비율의 권리를
침해하여 유언 권한을 행사하는 것은 거의 허용되지 않았다.
자식들의 몫은, 그 양에서 알 수 있듯이,
로마법의 선례를 따른 것이었다.\footnote{%
  여기서 자식들의 몫은 1/3을 말하는 듯하다.
  게르만법에서 유류분은 대체로
  과부 몫으로 1/3, 자식들 몫으로 1/3이었고 나머지가
  망자의 처분 대상이었다.
  로마법에서 유류분 비율은 원래 무유언상속분의 1/4이었다가
  유스티니아누스 신칙법에 의해 (네 자녀까지는) 1/3이 되었다.
}
과부에게 주어지는 몫은 교회의 노력에 기인한 것이다.
교회는 과부의 이익을 위한 배려를 멈추지 않았고
마침내 가장 힘겨운 승리 중의 하나를 쟁취했다.
혼인식에 임한
남편에게
아내에게 나누어줄 것을 명시적으로 약속하도록
2, 3백년에 걸쳐
압박한 결과,
서유럽 전역의 \wi{관습법}에 과부산\hanjalatin{寡婦産}{dower}의 원리를
집어넣는 데 성공하였던 것이다.
놀랍게도, 부동산의 과부산이
그에 대응하는 더 오래된,
인적재산에 대한 과부 및 자식들의 유류분\hanja{遺留分}보다
더 안정적인 제도였음이 드러났다.
프랑스 일부 지역의 관습법은 \wi{프랑스혁명}기까지 이 권리를 유지했고,
영국에서도 유사한 관행의 흔적이 남아있는 것이다.
그러나
동산은 유언에 의해 자유롭게 처분할 수 있다는 법리가
전반적으로 지배하게 되었고,
과부의 권리가 계속 존중되고 있었던 때에도
자식들의 권리는 법에서 사라져갔다.
이러한 변화의 원인을 장자상속제\latin{primogeniture}의 영향에서
찾는 데 주저할 필요는 없을 것이다.
봉건 토지법이 한 명을 제외한 다른 자식들을 상속에서 사실상 제외함에 따라,
종전에 균등하게 분배되었던 종류의 재산조차도
더 이상 그것의 균등한 분배를 의무로 여겨지 않게 되었다.
유언은 불균등 분배를 만들어내는 주요한 수단이었고,
이러한 상황 속에서
고대와 근대의 유언 관념 간에 미묘한 차이가 나타났다.
유언을 통해 행사되는 유증의 자유는 이렇게 \wi{봉건제}의 우연한 산물이었지만,
그러나
유언에 의해 자유롭게 처분되는 재산의 체계와
봉건 토지법에서처럼
정해진 계통을 따라 강제적으로 상속되는 재산의 체계
간에 존재하는 차이만큼 큰 차이도 아마 없을 것이다.
프랑스 법전의 입안자들은 이 진리를 보지 못한 것 같다.
그들은
가족승계적 재산설정\latin{family settlement}%
\footnote{재산이 가족 바깥으로 유출되지 못하도록 하는 부동산권 설정.
  한번의 설정에서 여러 개의 연속적 토지보유권을 창설함으로써 만들어진다.
  그러나 아래 한정승계부동산권도 넓은 의미의 가족승계적 재산설정으로 볼 수 있다.
}%
에 주로 기초한
장자상속제를
파괴해야 할 사회 구조의 일부라고
보았을 뿐만 아니라,
유언도
가장 엄격한 한정승계부동산권\latin{entail}%
\footnote{직계비속 또는 직계비속남성(여성)에게만 상속될 수 있도록
  설정된 부동산권.
  \latin{`estate in fee tail'} 또는 간단히 \latin{`fee tail'}이라고도 부른다.
  ``to A and the heirs (male) of his body'' 따위의 말로써 수여된다.
  }
하에서 장자에게 주어진 것과
동일한 우선권을
장자에게
주기 위해 자주 사용되는 것이라고 보았다.
따라서,
원하는 바를 이루기 위해 그들은
부부재산계약에서 다른 자식들에 비해 장자를 우대하는 것을 금지했을 뿐만 아니라,
상속재산이 자식들 사이에 균등하게 분배되도록 한 원칙을 회피하는 데
유언이
사용되지 않도록
유언상속을 법에서 거의 추방해버렸다.
결과적으로 그들은
유증의 자유가 완전히 인정되는 체계보다는
봉건제 하의 유럽의 체계에 무한히 가까운
일종의 작은 영구적 한정승계부동산권 체계를 만들어냈다.
물론
``\wi{봉건제}의 헤르쿨라네움''인 영국의 부동산법은
대륙 국가들의 그것보다 중세 부동산법에 훨씬 가깝다.\footnote{%
  헤르쿨라네움은 베수비우스 화산의 폭발로 폼페이와 함께 매몰된 고대 도시.}
또한 영국에서는
부동산에 관한 부부재산계약에 거의 보편적으로 등장하는
장자 및 그의 계통이 갖는 우선권을 조장하거나 흉내내는 데
유언이
자주 사용된다.
그럼에도 불구하고 영국인들의 법감정과 여론은
자유로운 유언 처분의 관행에 의해 크게 영향을 받았다.
내가 보기에,
가족 내에 재산을 보존하는 문제에 관한
대부분의 프랑스 사회의 법감정의 상태는
현재 영국인들의 여론보다는 2, 3백년 전 유럽을 지배했던 것에
훨씬 더 가까운 것 같다.

\para{장자상속제}
장자상속제를 언급하였거니와,
이것은 법제사의 가장 어려운 문제의 하나를 제기한다.
자세히 설명하지는 않았지만,
로마 상속법과 관련하여
단일한 상속인과 나란히
다수의 ``공동상속인''을
수차
언급했던 것을 기억하실 것이다.
사실, 로마법 역사 전체에 걸쳐
공동상속인 집단이 상속인, 즉 포괄승계인의 지위를
가질 수 없었던 때는 한 번도 없었다.
이 집단은 하나의 단위로서 상속했고,
그후 상속재산은 별도의 법적 절차를 통해 그들 사이에 분할되었다.
무유언상속의 경우,
이 집단이 망자의 자식들로 이루어진다면
그들은 균등한 몫으로 재산을 나누어가졌다.
남성들이 여성들에 비해 약간의 우대를 받는 때도 있었지만,
장자상속제의 흔적은 조금도 발견되지 않는다.
이러한 분배방식은 초기법을 통털어 일관된다.
실로, 국가사회가 시작되고
여러 세대의 가족이 하나로 모여 살기를 그친 무렵
인간의 자연스러운 관념은
재산을 각 세대의 구성원들 간에 균등하게 분할하여
장자나 그의 계통에 어떠한 특권도 부여하지 않는 것이었다고 보여진다.
이러한 현상과
원시적 사고 간의
밀접한 관계에 관하여
특별히 중요한 힌트 몇 가지를
로마보다 더 오래된 법체계들에서 발견할 수 있다.
인도인들 사이에서는 아들이 태어나면
그는 아버지의 재산에 대해 확정적 권리를 가지지만,
공동소유자들의 승인 없이는 이것을 팔아버릴 수 없다.
아들이 성년에 이르면,
아버지의 반대에도 불구하고
그는 가산의 분할을 때로 강제할 수 있고,
아버지가 묵인한다면
한 아들은 다른 아들들의 반대에도 불구하고
항상 분할을 강제할 수 있다.
그러한 분할이 일어날 때,
아버지는
자식들 몫의 두 배를 가져가는 것 외에는
자식들보다 더 우대받지 않는다.
게르만의 고대 부족법도 대단히 유사하다.
\wi{자유소유지}\latin{allod}, 즉 가\hanja{家}의 소유지는
아버지와 아들들의 공동소유였다.
하지만 이것은 아버지의 사망시에도 쉬이 분할되지 않았던 것으로 보인다.
마찬가지로 인도인의 토지도,
이론적으로는 분할가능하지만,
실제로는 좀처럼 분할되지 않아서
수 세대 동안 분할 없이 상속되는 일이 흔하다.
그리하여
인도의 가족은 끊임없이 촌락공동체\latin{village community}로
확장되는 경향을 보이거니와,
어떤 조건 하에서 그러한지는
추후 설명할 것이다.
이 모든 것들은
아버지의 사망시 아들들 간의 철저한 재산 균등 분배가
가족종속성이 해체되기 시작할 무렵의 사회에 나타나는
일반적 관행이었음을
사뭇 명료하게 지시하고 있는 것이다.
여기서 장자상속제라는 법제사적 난제가 등장한다.
봉건제가 형성되고 있을 무렵,
한편으로 로마 속주들의 법과 다른 한편으로 만족\hanja{蠻族}들의 옛 관습 외에는
세상 어디에도 \wi{봉건제}를 이루는 요소들의 원천이 될 만한 것이 없었음을
분명히 인식하면 할수록,
우리는 일견 더욱더 당혹스러움에 빠져들지 않을 수 없거니와,
로마인들도 만족\hanja{蠻族}들도 재산상속에서 장자나 그의 계통에
아무런 우선권도 주지 않고 있었다는 사실을 잘 알게 되었기 때문이다.

\para{은대지와 봉토, 자유소유지와 봉토}
만족\hanja{蠻族}들이 로마제국 내에 처음 정착했을 때
그들의 관습은 장자상속제가 아니었다.
그것의 기원은 제국을 침공한 수장들이 나누어준
은대지\hanjalatin{恩貸地}{benefice}에 있다는 것이 정설이다.
초기 이주민 국왕들에 의해 이따금 주어진,
그러나 샤를마뉴에 의해 대규모로 주어진,
이 은대지는 수혜자의 군사적 봉사를 대가로
로마 속주의 토지를 나누어준 것이었다.
\wi{자유소유지} 소유자들은 그들 군주의 멀고 험난한 원정에
잘 따라나서려 하지 않았던 것으로 보이며,
프랑크의 수장들이나 샤를마뉴의 대규모 원정은
모두
왕실에 복속되어있는 군인들이나
토지 보유의 대가로 봉사에 나서야했던 군인들로
군대를 구성하여 수행되었다.
하지만 은대지는 처음에는 결코 세습적인 것이 아니었다.
수여자가 원하면 언제든 되돌려주어야 하는 것이거나,
기껏해야 수혜자의 생애 동안만 보유할 수 있는 것이었다.
그러나 처음부터 수혜자들은 토지보유를 늘리는 데, 그리고
사망 후에도 가족들이 그 토지를 계속 보유토록 하는 데
모든 노력을 아끼지 않았다.
샤를마뉴 이후 허약한 후계자들이 속출하자,
그들의 노력은 예외 없이 성공을 거두었고
은대지는 점차 세습봉토\latin{fief}로 변모해갔다.
그러나 세습적이었다고는 해도 그것이 반드시
장자에게 상속되었다는 말은 아니다.
상속의 규칙은 전적으로 수여자와 수혜자 간에 맺어진,
혹은 그들 중 강자가 약자에게 강요한,
약정에 의해 결정되었다.
따라서 애초에는 토지보유권이 무척이나 다양했다.
물론 지금까지 서술한 로마의 상속방식과 만족\hanja{蠻族}들의 상속방식이
결합한 것이기에 완전히 무작위적인 것은 아니었으나,
그래도 대단히 잡다한 양상이었다.
어떤 곳에서는 분명 장자와 그의 계통이 우선하여 봉토를 상속했으나,
그러한 상속은 보편적이기는커녕 일반적이지도 않았던 것으로 보인다.
정확히 동일한 현상이
보다 후대에 일어난 유럽 사회의 변화,
즉 \paren{로마의} 완전소유지\latin{domainia}와
\paren{게르만의} \wi{자유소유지}\latin{allodial}가
봉건적 토지보유로 대체되는 과정에서도
반복되었다.
자유소유지는 완전히 봉토로 흡수되어갔다.
대규모 자유소유지 소유주들은
그들 땅의 일부를 종자\hanja{從者}들에게 조건적으로 양도함으로써
봉건영주가 되어갔다.
소규모 자유소유지 소유주들은
그들의 땅을 어떤 힘있는 수장에게 양도하고
전쟁시 복무한다는 조건으로 다시 되돌려받음으로써
험악한 시절의 압박으로부터 벗어나고자 했다.
그러는 동안,
서유럽 인구의 대다수를 차지하는
예속적 또는 반\hanja{半}예속적 신분의 사람들---로마와 게르만의 노예들,
그리고 로마의 콜루누스\latin{coloni}와 게르만의 리디\latin{lidi}---도
동시에
봉건 조직에 흡수되어 갔으니,
그들 중 일부는 영주의 하인이 되었으나
대부분은 당시 굴욕적이라 여겨진 조건으로 땅을 하사받았다.
이렇게 보편적 봉건화가 진행되는 동안 형성된 토지보유권은
토지보유자가 그들의 새로운 수장들과 맺은, 또는
맻도록 강요당한 조건에 따라 사뭇 다양했다.
은대지의 경우처럼,
전부가 아닌 일부 토지보유권만이 장자상속의 규칙을 따랐다.
하지만 \wi{봉건제}가 서유럽 전역을 지배하게 되자,
다른 상속방식보다 장자상속제가 큰 장점을 갖는 것임이 명백히 드러났다.
그것은 놀라운 속도로 전 유럽에 퍼져나갔거니와,
확산의 주요 수단은 영국의 가족승계적 재산설정\latin{family settlement},
프랑스의 팍트 드 파미유\latin{pacte de famille},
독일의 하우스게제츠\latin{Hausgesetz} 따위였으니,
이들은 모두 기사\hanja{騎士} 봉사의 대가로 보유한 토지를
장자가 상속하도록 설정한 것이었다.
마침내 이 규칙은 만성적인 관행으로 굳어져,
서서히 형성되어온 모든 \wi{관습법} 체계에서
장자와 그의 계통이
자유토지보유권\footnote{%
  단순토지보유(estate in fee simple),
  한정승계토지보유(estate in fee tail),
  생애토지보유(estate for life)를 통칭하는 말.
}과
군역\hanja{軍役}토지보유권의 상속에서 우선권을 가지게 되었다.
예속적 토지보유의 경우 \paren{처음에는 보유자가 금전을 지불하거나
노역을 제공할 의무가 있는 모든 보유지가 예속적이었다}
관습법상의 상속체계는 나라에 따라 그리고 지방에 따라 사뭇 달랐다.
그나마 일반적이라 할 만한 규칙은 이런 성격의 토지는 자식들 간에
균분상속하는 것이었으나, 그래도 어떤 곳에서는
장자\hanja{長子}가 우대되었고 어떤 곳에서는 말자\hanja{末子}가 우대되었다.
그러나,
영국의 농역\hanja{農役}토지보유권\latin{socage}\footnote{%
  일정한 지대의 정기적 제공을 조건으로 보유하는 토지보유양태.
}처럼,
비교적 나중에 등장하였고 완전히 자유롭지도 완전히 예속적이지도 않은
유형의 토지보유권의 상속은 가장 중요한 몇몇 측면에서
통상적으로 장자상속제에 의해 규율되었다.

\para{정치적 장자상속제}
장자상속제의 확산을 설명하기 위해 일반적으로 거론되는 것은
이른바 `\wi{봉건제}적' 근거이다.
봉건관계의 주군\hanja{主君}의 입장에서는
최후 보유자의 사망으로
봉토가
여러 명에게 분산되는 것보다
한 명에게 상속되는 것이 군사적 봉사를 안정적으로 확보하는 데
더 낫다는 것이다.
이런 이유가 장자상속제의 점진적 확산에 대한 부분적인 설명이 될 있음은
부인할 수 없지만,
장자상속제가 유럽 전역의 관습이 된 데에는
주군이 누리는 이익보다 토지보유자들 사이에서의 인기가 더 크게
작용했다는 점을
지적하지 않을 수 없을 것이다.
더욱이, 전술한 이유는 장자상속제의 기원을 전혀 설명하지 못한다.
편의성에 대한 감각만으로는 어떤 법제도도 생겨나지 않는다.
반드시 미리 어떤 관념들이 존재하고, 여기에 편의성의 감각이 작용하여
어떤 새로운 결합이 형성될 수 있을 따름인 것이다.
이 관념들이 무엇인지 찾는 것이 지금 우리에게 주어진 과제이다.

이에 관한 암시가 풍부한 어떤 지역에서 유용한 힌트를 얻을 수 있겠다.
인도에서는
아버지의 사망시, 또는 아버지가 살아있을 때에도,
그의 재산이 아들들 간에 균분으로 분할될 수 있거니와,
이러한 \hemph{재산}의 균분 원칙은 힌두법의 모든 법제에
두루 퍼져있다.
하지만 최후 보유자의 사망으로 \hemph{공직} 또는 \hemph{정치권력}이
이양되는 경우, 상속은 거의 보편적으로 장자상속의 규칙을 따른다.
그리하여 통치권은 장자에게 세습된다.
또한 인도 사회를 구성하는 단위체인 촌락공동체의 업무가
단일한 관리자에게 맡겨져있는 경우,
아버지의 사망시
일반적으로 장자가
관리업무를 이어받는다.
실로, 인도의 공직은 세습되는 경향이 있으며, 또한
성질상 허용된다면
가장 손윗 계통의 가장 연장인 구성원에게 주어지는 경향을 보인다.
이러한 인도의 상속제도를
유럽에 거의 오늘날까지 남아있는 몇몇 보다 미개한 사회조직과
비교해보면,
\wi{가부장권}이 \hemph{가족내부적}인 것을 넘어 \hemph{정치적}인 것일 때
그것은 아버지의 사후 모든 자식들에게 고루 분배되는 것이 아니라
장자의 생득권\hanja{生得權}이 된다는 결론이 자연스레 도출된다.
가령 스코틀랜드 산악지대의 \wi{씨족}장의 자리는
장자상속제를 따른다.
거기서는, 사실, 조직된 국가사회의 초기 기록에서 발견되는 것보다
더 오래된 가족종속성의 한 형태가 남아있는 듯하다.
옛 로마법에서 보이는 저 종족\hanja{宗族}연합은,
그리고 다른 많은 유사한 징후들도,
가계도의 이리저리 뻗어있는 모든 가지들이 한때
하나의 단일한 유기체로 결합되어 있던 때가 있었음을 암시한다.
그리고 이렇게 결합된 친족단체가 그 자체로 하나의 독립된 사회였을 때
그것은 가장 손윗 계통의 가장 연장인 남자에 의해 통치되었다고 보더라도
지나친 억측은 아닐 것이다.
물론 우리는 그러한 사회에 대한 실제적 증거는 갖고 있지 않다.
우리가 알고 있는 가장 초보적인 사회에서도
가족 조직은 기껏해야 `통치권 안의 통치권'\latin{imperia in imperio}인 것이다.
그러나 그들 중 일부, 특히 켈트족 씨족의 상태는
역사 시대에 속하면서도 독립적 상태에 사뭇 가까워서
그것이 한때 독립된 `통치권'\latin{imperia}이었다는 확신을,
그리고 장자상속제가 그 씨족장의 상속을 규율했다는 확신을 우리에게 심어준다.
하지만 현대적 법률용어가 불러오는 인상은 주의할 필요가 있다.
지금 우리는 인도 사회나 고대 로마법을 통해 알고 있는 그 어떤 것보다
훨씬 친밀하고 훨씬 엄격한 가족결합의 형태를 말하고 있는 것이다.
로마의 가부장이 가족재산의 가시적 관리인이었다면,
인도의 아버지가 아들들과의 공동소유자에 불과하다면,
저 순수한 가부장은 더더욱 공동재산의 관리인에 불과했을 것이 분명하다.

\para{카롤링거 제국의 몰락}
따라서,
은대지에서 보이는 장자상속의 예는
로마제국을 침공한 민족들이 가지고 있던, 그러나 보편적이었다고 할 수는 없는,
가족 통치권을 흉내낸 것이라 할 수 있다.
몇몇 보다 미개한 부족들은 여전히 그것을 행하고 있었기에,
혹은, 보다 그럴듯하기로는, 보다 원시적 상태를 거의 벗어나지 못하고 있었기에,
새로운 형태의 재산에 관한 상속규칙을 정해야 했을 때
사람들의 정신은 자동적으로 그것을 떠올렸을 터이다.
그러나 한 가지 문제가 아직 남아있다.
어째서 장자상속제가 점차 다른 상속원리들을 대체해갔는가? 하는 것이다.
생각건대, 그에 대한 답은
카롤링거 제국의 해체가 진행되는 동안 유럽 사회가 결정적으로 퇴보했다는 데 있다.
과거 만족\hanja{蠻族}들의 왕국 시절의
몹시도 저급한 수준보다도 한 두 가지 점에서는 더 퇴보했던 것이다.
저 시대의 큰 특징은 왕의 권위가, 따라서 국가의 권위가,
허약했다는 것, 아니 차라리 부재했다는 것이다.
그리하여, 국가사회가 더 이상 결속하지 못하는 상황에서,
사람들은
국가공동체의 등장보다 더 오래된 사회조직에
너도나도 뛰어들었던 것으로 보인다.
9세기와 10세기 무렵, 봉신\hanja{封臣}들은 둔 주군은,
초기 사회에서와 같은 입양이 아니라
이제 수봉\hanja{授封}에 의해 사람들을 충원하는,
일종의 가부장적 가\hanja{家}라고 보아도 좋을 것이다.
장자상속제는
그러한 연합체제에
힘과 지속성의 원천이 되었다.
조직 전체가 의존하고 있는 토지가 하나로 결합되어 있는 한,
그것은 방어와 공격에서 큰 힘을 발휘했다.
땅을 분할하는 것은 안 그래도 작은 사회를 또 분할하는 것이고,
폭력이 만연한 시대에 자동적으로 공격을 불러오는 일이었다.
이러한 장자상속제의 선호에는
한 명을 위해 나머지 자식들을 모두 상속배제한다는 관념이
들어있지 않았다고
전적으로 확신할 수 있다.
봉토가 분할되면 모두가 고통받을 것이었다.
봉토의 결합으로 모두가 이득을 누렸다.
권력을 한 사람에게 집중시킴으로써 가족은 더 강해졌다.
그러하기에, 상속재산을 차지한 주군이
사용과 수익과 처분에 있어
그의 형제들과 친족들에 비해
더 큰 권한을 누렸다고 볼 수가 없는 것이다.
봉토의 상속인이 상속하는 특권을
영국의 엄격한 가족승계적 재산설정 하에서 장자가 누리는 것과
동일시하는 것은 완전히 시대착오적인 발상이라 하겠다.

\para{가부장이 소유권자가 되다}
나는 초기의 봉건 연합체를 원시적 가족 형태에서 유래한 것으로,
그리고
그것과 강한 유사성을 갖는 것으로 본다고 말했다.
그렇지만 고대 세계에서는,
그리고 봉건제의 시련을 겪지 않은 사회에서는,
장자상속제가 지배적이었더라도 그것은 나중에 유럽 봉건제에서 보이는
장자상속제로 결코 전환되지 않았다.
친족집단이 각 세대마다 한 명의 세습 수장에 의해 통치되던 시대가 마감되자,
모두를 위해 관리되던 토지가 이제 모두에게 균등하게 분할되기
시작한 것으로 보인다.
왜 봉건제의 세상에서는 이런 일이 일어나지 않았을까?
봉건제 초기의 혼란 속에서는 장자가
전체 가족을 대신해서 토지를 보유했다 하더라도,
유럽에서 봉건제가 공고해진 이후에는,
정상적인 공동체가 다시 수립된 이후에는,
왜 가족들이
로마도 게르만도 다 가지고 있던 균분상속의 권리를
회복하지 못했을까?
봉건제의 계보를 추척해온 학자들은
이 난제를 해결할 열쇠를 거의 찾지 못했다.
그들은 봉건제를 구성하는 재료들은 발견했지만
그것들을 연결시키는 데 실패했다.
봉건제의 형성에 기여한 관념들과 사회형태들은
분명 미개하고 원시적인 것이었음에도,
법원과 법률가들이 그것을 해석하고 정의하는 일에 소환되자
그들이 적용한 해석원리는 최후의 로마법의 그것이었고
따라서 자못 세련되고 성숙한 것이었다.
가부장에 의해 통치되는 사회에서
장자는 종족\hanja{宗族}집단의 통치권을,
그리고 집단의 재산에 대한 절대적 권한을 상속할 수도 있을 것이다.
그러나 그렇다고 해서 그가 진정한 소유권자인 것은 아니다.
소유권 개념에는 들어있지 않은,
사뭇 불확정적이고 사뭇 정의 불가능한,
상응하는 의무도 그는 부담한다.
하지만 후기 로마법은, 오늘날의 법과 마찬가지로,
재산에 대한 무제약적 권한을 소유권과 등치시켰기에,
흔히들 법이라고 부르는 것이 등장하기 이전 시기에 속하는
그러한 책임 개념을 알지 못할 뿐만 아니라 사실 알 수도 없는 것이다.
세련된 관념과 미개한 관념의 접촉은 불가피
장자를 상속재산의 법적 소유권자로 전환시키는 결과를 낳았다.
교회법학자들과 세속법학자들은 장자의 지위를 처음부터 그렇게 정의했다.
그러나 부지불식간에 그의 연하 형제는
친족의 모든 위험한 일과 즐거운 일에 동등하게 참여하던 것에서
사제로, 용병으로, 영주 저택의 식객으로 그 지위가 조금씩 떨어져갔다.
이와 동일한 성격의 법적 혁명이
최근에
보다 작은 규모로
스코틀랜드 산악지역 대부분에 걸쳐 일어났다.
\wi{씨족}의 생계를 책임지던 토지에 대한 씨족장의 법적 권한을
결정하도록 소환되었을 때, 스코틀랜드법은
토지지배권의 완전성이
씨족원들의 권리에 의해 막연하게나마 제한된다는 것을
알아차릴 수 있는 시대를 한참 지나있었고,
따라서 다수의 재산은 일인\hanja{一人}의 재산으로 불가피 전환되었던 것이다.

\para{장자상속의 형태들}
설명을 단순화하기 위해,
나는
어떤 가\hanja{家}나 단체의 권위를
한 명의 아들 또는 자손이
상속할 때
이것을 장자상속제라고 불러왔다.
하지만,
이런 상속유형의 고대 사례를 보여주는 소수의 남아있는 기록들 중에는
반드시 우리가 생각하는 그러한 장자가 대표자의 자리를 차지하는 것은
아닌 경우도 있음을 유의해야 한다.
서유럽에 널리 퍼진 장자상속의 형태가 인도인들 사이에서도 준행되어왔거니와,
그것이 통상적인 형태라는 것은 의심의 여지가 없다.
여기서는 장자뿐만 아니라 가장 손윗 계통이 항상 우대된다.
만약 장자가 먼저 죽고 없다면, 그의 장자가 다른 형제들뿐만 아니라
삼촌들에 대해서도 우선하는 것이다.
그러나 단순히 \hemph{민사적} 권력이 아니라 \hemph{정치적} 권력의 상속이
걸려있을 때는 곤란한 문제가 대두될 수 있거니와,
이 문제는 사회의 결속력이 불완전할수록 더 커질 것으로 짐작된다.
마지막으로 권력을 가졌던 수장이 그의 장자보다 오래 살았고,
일차적 상속권을 갖는
손자는 아직 너무 어리고 미성숙해서
공동체를 실제로 이끌고 관리할 수 있는 상태가 아닐 수 있는 것이다.
이런 경우, 어느 정도 안정된 사회에서 널리 사용되는 방식은
어린 상속인이 통치에 적합한 나이가 될 때까지 그를 \wi{후견} 아래 두는 것이다.
일반적으로 \wi{종족}\hanja{宗族} 남성이 후견인이 되지만,
다른 대안도 있을 수 있음에 유의해야 한다.
드물지만 어떤 고대 사회는 여자가 권력을 행사하는 데 동의했거니와,
이는 분명 어머니의 권리가 우선한다는 인식에서 나온 것이다.
인도에서는 통치자의 과부가 어린 아들의 이름으로 대신 통치한다.
또한 프랑스의 왕위계승에 관한 관습---그 기원이 무엇이건 이는
아주 오래된 고대의 유산임이 분명하다---에서는,
여성이 왕위에 오르는 것은 엄격히 배제하면서도,
다른 어느 누구의 섭정보다도
대비\hanja{大妃}의 섭정이
선호되었다는 것을
떠올리지 않을 수 없다.
하지만, 어린 상속인에게 통치권이 이전될 때 발생하는 문제점을 회피하는
또 다른 방법이 있거니와, 이는 미개한 구조를 가진 공동체에서
자연스럽게 일어나는 일이라고 보면 틀림없을 것이다.
어린 상속인을 완전히 제치고,
윗 세대의 가장 연장인 살아있는 남성이
수장 자리를 차지하는 것이 그것이다.
켈트족의 \wi{씨족}연합은
국가사회 또는 정치사회가 아직 초보적으로 분화되지도 못한 시대의
현상들을 다수 보존해왔거니와,
그 중에서도 특히
이러한 상속규칙을 역사 시대에 이르기까지 보존해왔다.
장자가 이미 죽고 없다면,
통치권 이양시 손자들의 나이를 전혀 불문한 채
손자들에 우선하여,
그 장자의 바로 밑 남동생이
상속한다는 것이
실정규범의 형태로 그들에게 존재해온 것으로 보인다.
혹자는
마지막 족장을 일종의 뿌리 또는 계통으로 파악하여
그에게 가장 가까운 후손에게 상속시키는 것이 켈트족의 관습이었다고
가정함으로써 이 원리를
설명한다.
삼촌이
손자보다
공통의 뿌리에 더 가까우므로 삼촌이 우선한다는 것이다.
이런 진술은 단지 상속규칙을 기술\hanja{記述}하는 목적만 가진다면
아무런 이의도 제기할 수 없을 것이다.
그러나
저 규칙을 처음 채용한 사람이
법률가들 사이에서
봉건제적 상속규칙이
논쟁의 대상이 되기 시작할 시대에
기원한 것이
분명한 추론과정을
적용했다고 본다면 이는
심각한 오류가 아닐 수 없다.
손자보다 삼촌이 선호되는 것의 진정한 기원은
어린 아이보다는 어른 족장이 다스리는 것이 더 낫다는,
그리고 장자의 자식들보다는
차자\hanja{次子}가 성숙기에 도달했을 확률이 더 높다는,
미개한 사회의 미개한 사람들의 단순한 계산이었을 것이 틀림없다.
뿐만 아니라,
우리가 잘 알고 있는 장자상속의 형태가 일차적 형태라는 것을 보여주는
증거도 있으니,
전승에 의하면
어린 상속인을 무시하고 그의 삼촌이 우대될 경우
씨족원들의 동의가 필요했던 것이다.
스코틀랜드의 맥도널드\latin{Macdonald} 씨족의 연대기에는
이러한 의식\hanja{儀式}의 사례가 전해지거니와,
그 진정성은 상당히 믿을 만하다.
또한 최근 연구자들의 해석에 따르면
아일랜드 켈트족의 유물들도 비슷한 관행의 흔적을 다수 보여준다고 한다.
인도의 촌락공동체에서도
선거를 통해
``보다 훌륭한'' 종친\hanja{宗親}이
손윗 종친을
대체하는 일이 없지 않다.

\para{이슬람의 법}
아라비아 관습을 보존해온 것으로 보이는
이슬람 법에 의하면,
상속재산은 아들들 간에 균등하게 분할되고,
딸들은 아들 몫의 절반을 갖는다.
그런데 상속재산 분할 전에 자식들 중 누군가가 자녀를 남기고 사망했다면,
이들 손자녀는 상속에서 완전히 배제되고 삼촌들과 고모들이 재산을 독차지한다.
이 원리가 동일하게 적용되어,
통치권이 이양될 때의 상속도
켈트족 사회에서 행해져온 장자상속의 형태와
같은 형태를 따른다.
서양의 무슬림 가문 중 가장 큰 두 가문의 상속규칙은,
조카가 장자의 아들이라 할지라도, 조카에 우선하여 삼촌이 왕위를
계승하는 것이라고 믿어지고 있다.
이 규칙이 이집트와 투르크 양자 모두에서
최근까지 준수되어왔지만,
그러나
투르크의 통치권 이양에 관해서는 이 규칙은
늘 어떤 의문의 대상이 되어왔다고 한다.
술탄들의 정책으로 말미암아 그것의 적용이 사실상 널리 방해받았다는 것이다.
물론 연하의 남동생들을 대량 학살함으로써
왕좌를 둘러싼 위험한 경쟁자들을 제거할 수 있을 뿐만 아니라
자기 자식들의 이익도 확보할 수 있었을 것이다.
하지만 한 가지 확실한 것은
일부다처제 사회에서
장자상속제는 반드시 다양한 형태를 띨 수밖에 없다는 것이다.
상속권 주장에는
가령 어머니의 순위라든가, 혹은 아버지의 총애를 얼마나 받는가 따위의
다양한 사항이 고려대상이 될 수 있다.
따라서 인도의 몇몇 무슬림 통치자들은,
유언 권한은 전혀 내세우지 않은 채,
자신을 계승할 아들을 지명할 권리가 자신에게 있다고 주장한다.
이삭과 그의 아들들 이야기와 관련하여 성서에 언급된
\hemph{축복}\latin{blessing}을 일종의 유언이라고 보는 이도 있지만,
그보다는 장자를 지명하는 방법의 하나였던 것으로 보인다.


\chapter{물권법의 초기 역사}

로마의 \wi{법학제요} 저서들\footnote{%
  가이우스의 법학제요와 유스티니아누스의 법학제요를 일컫는다.
}은
소유권의 여러 형태들과 변종들을 정의한 후,
자연법상의 물건취득 방식들에 대하여 논한다.
법제사를 잘 모르는 이들은
취득의 이러한 ``자연법상의 방식들''이
일견
사변적으로나 실무적으로나 큰 관심의 대상이 아닐 것이라고
생각하기 쉽다.
야생동물을 덫으로 잡거나 사냥해서 죽이는 것,
토양이 강물에 의해 충적되어 부지불식간에
내 땅에 부합\hanja{附合}하는 것,
나무가 내 땅에 뿌리를 내리는 것 따위를
로마 법률가들은 모두 \hemph{자연적으로} 취득한다고 말했다.
옛 법학자들은
그들 주위의 여러 작은 사회의 관행에서
이들 취득이
보편적으로 인정되는 것을 분명 관찰했을 것이다.
후대의 법률가들은
이들이 옛 \wi{만민법}\hanjalatin{萬民法}{jus gentium}에 분류되어 있고
단순명쾌하게 기술\hanja{記述}되어 있는 것을 보았고, 그리하여
이것들에
자연법의 자리를
내주었을 것이다.
이것들에 부여된 존엄성은 근대에 이르러 점점 커져,
이제는 원래 가졌던 중요성을 훨씬 능가하는 것이 되었다.
자연법 이론은 이것들을 가장 즐기는 음식으로 삼았고,
실무에 사뭇 심각한 영향력을 행사할 수 있도록 만들었다.

\para{선점}
이러한 ``자연법적 취득방식들'' 가운데
한 가지만은 반드시 짚고 넘어갈 필요가 있거니와,
\wi{선점}\hanjalatin{先占}{occupatio}이 그것이다.
선점은
취득 당시 누구의 물건도 아닌 것을
\paren{법기술적 정의\hanja{定義}가 이어진다}
당신의 물건으로 삼고자 하는 의사로써
점유하는 것을 말한다.
로마 법률가들이 \wi{무주물}\hanjalatin{無主物}{res nullius}---소유주가
없거나 있어본 적이 없는 물건---이라 불렀던
것이 무엇인지는 열거함으로써만 알 수 있을 뿐이다.
소유주가 \hemph{있어본 적이 없는} 물건에는
야생 동물, 물고기, 야생 조류\hanja{鳥類}, 최초로 캐낸 보석,
새로 발견했거나 경작된 적 없는 토지 따위가 속한다.
소유주가 \hemph{없는} 물건에는
포기된 동산, 버려진 토지,
\paren{특이한 그러나 가공스러운 항목인데}
적\hanja{敵}이 소유한 물건 따위가 속한다.
이 모든 것들은
자기 것으로 삼으려는 의사---일정한 경우 이 의사는
특정한 행위에 의해 명시적으로 드러나야 한다---를 가지고서 처음 점유한
\hemph{선점자}가 완전한 소유권\latin{dominion}을 취득한다.
생각건대,
선점 관행의 보편성으로 인해
한 세대의 로마 법률가들이 그것을 모든 민족에 공통인 법으로
자리매김한 것,
그리고 그 단순성으로 인해
다음 세대의 로마 법률가들이 그것을 자연법에 귀속시킨 것은
그리 어렵지 않게 이해할 수 있다.
그러나 근대 법사\hanja{法史}에서 그것이 누린 행운은
선험적인 고찰로는 얼른 이해되지가 않는다.
로마법의 \wi{선점} 원리, 그리고 이를 둘러싸고 로마 법학자들이 전개한 법규칙들은
근대 국제법 중에서도
전쟁시 포획에 관한 법과
새로 발견한 땅에 대한 주권 획득에 관한 법의
원천이 되었다.
또한 소유권의 기원에 관한 어떤 이론의 근거가 되었거니와,
이 이론은 대중적으로 인기있는 이론인 동시에,
다수의 위대한 사변적 법학자들이
이러저러한 형태로
널리 수긍하고 있는 이론이다.

\para{적의 소유물, 발견의 법리}
방금 나는 로마법의 선점 원리가
전쟁시 포획에 관한 \wi{국제법}의 흐름을 결정했다고 말했다.
전쟁시 포획법의 법규칙들은,
적대관계의 발발에 의해 국가들은 일종의 자연상태로 환원되고
이렇게 만들어진 인위적인 자연상태 하에서
교전국 간에는
사적 소유권 제도가
중지된다는
가정\hanja{假定}에 기초한다.
후기의 국제법 학자들은
그들이 설명하는 법체계에서도
사적 소유권이 어떤 의미에서는 인정된다는 주장을
유지하려고 했기 때문에,
적의 재산이 \wi{무주물}이라는 가설은 그들에게
정도\hanja{正道}를 벗어난 충격적인 것으로 여겨졌고,
따라서 그들은 이 가설을 단지 법적인 의제\hanja{擬制}에 불과하다고
내세우는 신중함을 보였다.
그러나 자연법이 만민법에 그 기원을 두고 있음을 잘 아는 우리는
어떻게 적의 재산이 무주물로 취급되고 그리하여
최초의 점유자에 의해 취득될 수 있었는지 금방 이해할 수 있다.
고대적 형태의 전쟁을 수행하는 사람들은
승전으로 정복군의 군대가 해산되고
해산된 군인들이 무차별적인 약탈을 자행했을 때
저 관념을 자동적으로 떠올렸을 것이다.
하지만
이때 포획자가 취득하도록 허용된 재산은
원래는 동산에 국한되었을 것으로 보인다.
우리는
고대 이탈리아에서
피정복 국가의 토지에 대한 소유권의 취득에 관해서는
전혀 다른 규칙이 지배했음을 별도의 전거를 통해 알고 있다.\footnote{%
  로마 국가의 국유지(ager publicus)가 되었다는 의미일 것이다. }
따라서 토지에 대해 선점의 원리가 적용되기 시작한 것은
\paren{항상 어려운 문제이지만}
만민법이 자연법으로 전환되는 시기였을 것으로,
그리하여 황금시대의 법학자들이 행한 일반화의 결과였을 것으로
짐작할 수 있다.
이에 관한 법리는 \wi{유스티니아누스}의 \wi{학설휘찬}\latin{Pandects}에 보존되어 있거니와,
그것은 모든 종류의 적의 재산은 교전 상대방에게 \wi{무주물}이라는,
그리고 포획자가 그것을 자기 것으로 만드는 \wi{선점}은 자연법상의 제도라는,
무제한적 주장으로 나아간다.\footnote{%
  \latin{Dig.\,41.1.5.7; Dig.\,41.2.1.1.} }
이러한 명제로부터 국제법이 이끌어낸 규칙들은
때로 군인들의 만행과 탐욕을 필요 이상으로 부추긴다고
비판받았지만,
생각건대 이 비판은
전쟁의 역사를 잘 모르는 사람들에 의해,
그리하여 어떤 종류의 규칙이든 규칙에 대한 복종을 명하는 것이
얼마나 위대한 업적인지 잘 모르는 사람들에 의해 가해진 비판이다.
선점에 관한 로마법 원리가 전쟁시 포획에 관한 근대법에 수용되어 들어왔을 때,
그 남용을 제한하고 정밀함을 부여하는
많은 부수적인 법규칙들도 함께 들어왔으니,
만약 \wi{그로티우스}의 저서가 권위를 획득한 후에 수행된 전쟁들을
그 이전의 전쟁들과 비교해본다면,
로마법의 규칙들이 수용되자마자 이제 전쟁은 그나마 어느 정도
인내할 만한 성질의 것이 되었음을 알 수 있을 것이다.
선점에 관한 로마법이 근대 \wi{만민법}\latin{law of nations}에
어떤 해로운 영향을 끼쳤다고 비난받아야 한다면,
해로운 영향을 입었다고 자신 있게 말할 수 있는
분야는
근대 만민법의 다른 영역에 존재한다.
보석의 발견에 로마인들이 적용한 원리를 새로운 땅의 발견에도 적용함으로써,
공법학자\latin{publicist}들은
원래 기대되는 용도와 전혀 맞지 않는 곳에다 억지로
저 법리를
가져다 썼다.
15, 16세기의 위대한 항해자들의 발견으로 극히 중요한 것으로 부상한
저 법리는
문제를 해결하기보다는 오히려 야기시켰다.
확실성이 무엇보다 요청되는 두 가지 사항에 관하여
커다란 불확실성이 존재한다는 것이 당장 드러났거니와,
하나는 발견자가 주권자를 위해 취득한 영토의 범위에 관한 것이고,
다른 하나는 `집지'\hanjalatin{執持}{adprehensio},
즉 주권적 점유의 확보에
필요한 행위가 무엇이냐에
관한 것이다.
더욱이,
약간의 행운의 결과치고 엄청난 이득을 가져다주는 저 원리는
유럽의 가장 모험적인 몇몇 국가들, 즉 네덜란드, 영국, 그리고 포르투갈에 의해
본능적으로 거부되었던 것이다.
우리 영국인들은,
저 \wi{국제법} 규칙을 대놓고 부인하지는 않았지만,
실제로는
멕시코만 이남의 아메리카 대륙을 전부 독점한다는 스페인의 주장을
결코 받아들이지 않았다.
오하이오강 유역과 미시시피강 유역을 독점한다는 프랑스 왕의 주장도 마찬가지였다.
엘리자베스 1세의 등극부터 찰스 2세의 등극에 이르기까지
아메리카의 수역\hanja{水域}에는 완전한 평화가 깃든 날이
하루도 없었다고 할 수 있고,
프랑스 왕의 영토에 대한
뉴잉글랜드 식민지인들의
잠식은
그로부터도 한 세기 이상 계속되었다.
저 법리의 적용을 둘러싼 혼란상에 충격을 받은 \wi{벤담}은
아조레스 제도 서쪽 100리그\latin{league} 지점에 그은 선을 기준으로
스페인과 포르투갈 간에
이 세상의 미발견된 땅을
나누어갖도록 한
저 유명한 알렉산데르 6세 교황의 칙서를 짐짓 칭송하기까지 했다.\footnote{%
  1493년 알렉산데르 교황의 칙서는
  아조레스 제도 서쪽 100리그 지점 자오선의 서쪽을
  아라곤^^b7카스티야 왕국에 주었다.
  이에 불만을 품은 포르투갈은 스페인과의 협상 끝에
  다음 해인 1494년 저 유명한 `토르데시야스 조약'을 맺어
  교황의 자오선을 조금 더 서쪽으로 옮겼다.
  }
그의 칭송이 일견 생뚱맞아 보이기는 하지만,
손으로 잡을 수 있는 귀중품의 취득 요건으로
로마 법학자들이 내건 조건을
어떤 군주의 신민이
수행했다고 해서 그 군주에게
대륙의 절반을 내주는 공법학자들의 법규칙보다
과연
저 알렉산데르 교황의 조치가
원칙적으로 더 불합리한 것인지는 의문의 대상일 수 있다.

\para{소유권의 기원}
본 저서의 주제를 연구하는 모든 사람에게
선점은
그것이
사변적 법학에
사적 소유권의 기원에 관한 가상의 설명을
제공하고 있다는 점에서
특히 관심의 대상이다.
애초 공유의 대상이었던 대지와 그 열매가
개인적 소유권의 대상으로 허용되는 과정이
\wi{선점}이 이루어지는 과정과 동일하다고 한때 널리 믿어졌다.
자연법에 관한 고대적 관념과 근대적 관념 간의 미묘한 차이를 포착한다면,
이러한 가정\hanja{假定}을 이끌어내는 사고방식을
그리 어렵지 않게 이해할 수 있다.
로마 법률가들은 선점을 자연법적 물건취득의 한 방식이라고 주장했고,
만약 인류가 자연의 제도 하에 살고 있다면
선점도 인류의 관행의 일부일 것이라고 그들은 분명 믿었을 것이다.
인류가 실제로 그러한 상태에서 살았던 적이 있다고 그들이 과연 믿었는지는,
전술한 바처럼, 남아있는 자료로는 확인하기가 어렵다.
그러나 확실히 그들은
소유권 제도가 인류의 존재만큼 오래된 것은 아니라고 생각했던 것으로 보이며,
이런 생각은 시대를 막론하고 상당한 설득력을 가지는 것이다.
그들의 모든 도그마를 유보 없이 수용한 근대법학은
가상의 자연상태를 강조하는 열성에 있어서만큼은
그들보다 훨씬 멀리 나아갔다.
그리하여 근대법학은
대지와 그 열매가 한때 \wi{무주물}이었다는 명제를
수용했을 뿐만 아니라,
자연에 대한 특유한 견해로 인해
국가사회가 형성되기 오래 전부터
인류가 무주물의 선점을 실제로 관행했었다고
서슴없이
가정하기에 이르렀다.
그리고 이로부터
원시 시대의 ``누구의 것도 아닌 물건''\latin{no man's goods}이
역사 시대의 개인의 사적 소유권으로 되는 과정이
바로 \wi{선점}이었다는 추론이
즉시
도출되었다.
이런 이론을
이런저런 형태로
지지하는 법학자들을 일일이 열거하는 것은
지루한 일이 될 터이고,
그다지 필요하지도 않을 것이다.
언제나
당대의 평균적 의견을 충실하게 드러내주는 인물인
\wi{블랙스톤}이 그의 저서 제2권 제1장에서
그것을 잘 요약해놓았기 때문이다.

\para{블랙스톤의 이론}
그는 이렇게 쓰고 있다.
``대지와 대지 위의 모든 것은 창조주의 직접적 증여로서
인류 공동의 재산이었다.
물론
최초의 시기에도
물건의 공유성은
물건의 본질에만 적용될 수 있을 뿐이었고,
그것의 사용에까지 확장될 수 없었다.
왜냐하면, 자연법과 이성법에 따르면,
물건을 처음 사용하기 시작한 사람은
일종의 일시적 소유권을 취득하고
그것을 계속 사용하고 있는 동안은 그 일시적 소유권도 계속되기 때문이다.
보다 정확히 말하자면,
점유 행위가 지속되는 동안은 점유권도 지속되는 것이다.
그리하여 토지는 공유였고,
토지의 그 어떤 일부도 특정인의 영구적 소유권의 대상일 수 없었으나,
누군가가
휴식을 위해, 그늘을 위해, 또는 다른 이유로
특정 장소를 선점\latin{occupation}하면,
그는 당분간 일종의 소유권을 취득하고,
그에게서 강제로 그 소유권을 빼앗는 것은 부정의하고
자연법에 반하는 일이 될 것이다. 하지만
그가 사용이나 점유\latin{occupation}를 그치는 순간,
다른 사람이 그 장소를 차지하는 것은 아무런 부정의가 아니다.''
그리고 이렇게 주장을 이어간다.
``인류의 인구가 증가하면서,
보다 영구적인 소유권 관념이 필요하게 되었고,
개인에게
일시적인 사용을 넘어
물건의 본질을 사용할 수 있도록
허용할 필요가 생겨났다.''\footnote{%
  \latinmarks
  William Blackstone,
  \textit{Commentaries on the Laws of England: In Four Books},
  Philadelphia: George W. Childs, 1866,
  Book 2, p.\,2. }

위 인용문에 나타난 몇몇 모호한 표현을 볼 때,
\wi{블랙스톤}은
그가 참조한 전거들에 나오는,
\hemph{선점자}가 자연법에 의하여 지구 표면에 대한 소유권을 취득한다는
명제를 제대로 이해하지 못한 것이 아닌가 한다.
그러나
의도적이든 오해에 의해서든
저 이론에 이처럼 제한을 가함으로써
그는
흔히들 상정되어온 형태를 그대로 따르고 있다.
언어의 정확한 구사에 있어
블랙스톤보다 더
유명한 많은 학자들이
태초에는
우선
\wi{선점}에 의해
배타적이지만 일시적인 향유권이 대세\hanja{對世}적으로
부여되었고,
그후 이 권리가 배타성은 유지한 채 영구적인 것이 되었다고
주장해왔다.
그들이 이렇게 이론을 전개한 목적은
자연상태에서는
\wi{무주물}이
선점에 의해
소유권의 대상이 된다는 법리와
가부장들이
양떼와 소떼에게 풀을 먹이던 토지를
처음에는
영구적으로 차지하지 않았다는
성서의 이야기에서
추론한 결과\footnote{%
  창세기 \latin{13:5--9}와 관련있는 듯하다.
}를
조화시키려는 것이었다.

\wi{블랙스톤}의 이론에 직접 적용될 수 있는 한 가지 비판은
그가 묘사하는 원시사회의 상태가
똑같이 쉽게 상상할 수 있는 다른 상태들보다 과연 더 설득력이 있는 것이냐
하는 데 있다.
이 문제를 탐구하기 위해
우리는
토지의 특정 장소를 휴식이나 그늘을 위해
\hemph{선점한}\latin{occupied}
\paren{블랙스톤은 이 단어를
일상적인 의미로 사용하고 있음이 분명하다}
사람이 아무런 방해 없이 그것을 보유할 수 있겠는가를
질문해보는 것이 좋겠다.
그의 점유권은 그것을 지킬 수 있는 힘에 정확히 비례할 것이고,
똑같이 그 장소를 갈망하고 있고
점유자를
충분히
힘으로
내쫓을 수 있다고 생각하는
경쟁자들에 의해 끊임없이 방해를 받을 것임에 틀림없다.
그러나 사실,
이 모든 트집잡기는 저 이론 자체의 근거없음에 비하면 한가한 이야기에 불과하다.
원시 상태의 사람들이 무엇을 했는가 탐구하는 것은
전혀 희망없는 일은 아닐 수 있지만,
그들 행위의 동기를 안다는 것은 도저히 불가능한 일이다.
태초의 사람들에 대한 저 이론의 묘사는,
오늘날 우리가 처해있는 상태와는 사뭇 다른 상태에
그들이 놓여있었다고 우선 가정함으로써,
그리고는
이런 가상의 상황에서도
지금 우리가 가지고 있는 감정과 선입견을
그들도
똑같이 가지고 있었다고 상정함으로써 이루어진다.
그 감정이 실은 가설상의 그들의 상태와는 전혀 다른 상태에서
만들어진 감정일 수 있음에도 말이다.

\para{사비니의 금언}
소유권의 기원에 관하여
블랙스톤이 요약한 것과 비슷한 견해를 지지하는 것으로
때로 여겨져온 \wi{사비니}의 금언\hanja{金言}이 있다.
저 위대한 독일 법학자는
모든 소유권의 기초가
\wi{취득시효}\hanjalatin{取得時效}{prescription}에 의해 완성되는
\wi{적대적 점유}\latin{adverse possession}에 있다고 주장했다.
사비니의 이러한 진술은 오직 로마법에만 근거한 것이고,
진술에 사용된 표현들을 설명하고 정의하는 데 충분한 노력을 들여야만
완전히 이해될 수 있는 것이다.
하지만
로마인들이 채용한 소유권 관념을 아무리 깊게 탐구하더라도,
법의 유년기에 이르기까지 그 관념을 아무리 멀리 추적하더라도,
저 금언에 포함된 세 가지 요소---\wi{점유}, 적대적 점유, 그리고 취득시효---로
구성된 것 이상의
다른 소유권 개념은
얻을 수는 없다는 주장으로
그의 주장을
이해한다면,
우리는 그가 말하고자 한 바를 충분히 정확하게 이해했다고 할 수 있다.
여기서
\wi{적대적 점유}란 허락받은 점유나 종속적인 점유가 아닌,
온 세상을 상대로 배타성을 주장하는 점유를 말한다.
\wi{취득시효}란 적대적 점유가 평온하게 지속되어온 시간의 경과를 말한다.
이 금언은 사비니가 의도한 것을 넘어 더 일반적으로 적용될 수도 있다고 믿는다.
그리하여 어떠한 법체계를 조사해보더라도
이들 세 가지가 결합된 소유권 개념 이상의
건전하고 안정적인 결론을 찾아내기란 불가능하리라고 믿는다.
동시에,
소유권의 기원에 관한 대중적인 이론을 지지하기는커녕,
\wi{사비니}의 금언은 그것의 가장 약한 고리를 드러내는 특별한 가치를 가지고 있다.
블랙스톤 및 그가 추종하는 사람들의 견해에서는,
배타적 점유를 획득하는 방식이
인류의 조상들의 정신에 어떤 신비로운 영향을 주었었다.
그러나 사비니의 금언에는 이러한 신비로움이 없다.
\wi{적대적 점유}에서 소유권이 시작한다는 데는 놀라울 것이 전혀 없다.
최초의 소유자는 자신의 재산을 안전하게 지켜내는,
무장\hanja{武裝}한 힘있는 사람이었을
것이라는 데는 놀라울 것이 전혀 없다.
하지만 어째서 시간의 경과가 그의 점유를 존중하는 감정---이것이야말로
장기간 사실상\latin{de facto} 존재해온 것에 대한 인류 보편의 존중심의
원천이다---을 만들어내는지는
진정 깊이 연구해볼 가치가 있는 문제이지만,
이는 지금 우리의 탐구 범위를 훨씬 넘어선다.

\para{소유권의 추정}
드물고 불확실한 정보에 불과하지만
소유권의 초기 역사에 관한 약간의 정보를 얻을 수 있을 법한 지역을 다루기에 앞서,
우선 나는
문명의 초기 단계에서 \wi{선점}이 수행한 역할에 주목하는 대중적 견해가
실은 진실을 거꾸로 뒤집은 것이라는 점을 감히 지적하고자 한다.
선점은 의사\hanja{意思}에 의한 물리적 점유의 획득이다.
이런 유\hanja{類}의 행위가 ``무주물''에 권리를 수여한다는 관념은,
초기 사회의 특징이기는커녕,
세련된 법학과 확립된 법상태의 결과물일 공산이 농후하다.
소유권이
오랜 관행을 통해
그 불가침성을
인정받은 후에야,
대부분의 향유의 객체가 사적 소유권의 대상이 되고 난 연후에야,
이전에 소유권이 주장된 바 없는 물건에 대한 소유권을
최초의 점유자가
단순한 점유에 의해
수여받도록
비로소
허용되는 것이다.
이러한 원리를 만들어낸 감정은
문명의 시초를 특징짓는
저 희소하고 불확실한 소유권과는 전혀 조화되지 않는 것이다.
그 감정의 진정한 기초는
소유권 제도를 지향하는 본능적인 선입견이 아니라,
소유권 제도의 오랜 지속으로 등장한,
\hemph{모든 것은 주인이 있어야 한다}는 추정\hanja{推定}인 것이다.
``\wi{무주물}'', 즉
소유권의 대상이 \hemph{아닌}
또는 소유권의 대상인 적이 \hemph{없는}
객체가 점유될 때,
점유자가 소유권자로 허용되는 것은
모든 가치있는 물건은 당연히 배타적 향유의 대상이라는,
그리고
당해 사례에서는
\wi{선점}자 외에 소유권을 수여받을 사람이 없다는 감정에
기인한다.
요컨대,
모든 물건은 누군가의 소유물이어야 하기에,
그리고
특정 물건의 소유권자로 선점자보다 더 나은 권리를 갖는
사람을 찾을 수 없기에,
선점자가 소유권자가 되는 것이다.

\para{대중적 이론의 반박}
우리가 논의해온 자연상태 사람들에 대한 기술\hanja{記述}에 대해
다른 반박이 없다 할지라도,
적어도 한 가지 점에 있어서만은
그것은 우리가 가진 믿을 만한 증거에 결정적으로 배치되고 있다.
저 이론이 상정하는 행위와 동기는 개인의 행위와 동기라는 점을 주목하자.
사회계약 체결의 당사자는 각 개인들이다.
홉스의 이론에 의하면
개인이라는 모래알로 구성된 어떤 움직이는 모래더미가
완전한 강제력에 의해 사회적 바위로 굳어지는 것이다.
\wi{블랙스톤}의 묘사에서
``휴식을 위해, 그늘을 위해, 또는 다른 이유로 특정 장소를 선점하는''
것도 개인이다.
로마인들의 자연법에서 유래한 모든 이론이
이 결함으로부터 자유로울 수 없거니와,
로마인들의 자연법은 개인을 취급함에 있어 그들의 시민법과 근본적으로 달랐고,
초기사회의 권력으로부터 개인을 해방시킴으로써
문명의 진보에 크게 기여했다.
그러나 고대사회는, 반복하여 말하지만,
개인을 거의 알지 못한다.
그것의 관심은 개인이 아니라 가족에 있었고,
단독의 인간이 아니라 집단에 있었다.
국가법이 친족집단이라는 작은 원\hanja{圓}들을
원래는 전혀 뚫지 못하다가
마침내 뚫고 들어갔을 때도,
그것이 바라보는 개인은 훗날 성숙한 단계의 법이 바라보는 개인과
사뭇 달랐다.
각 시민의 생애는
출생과 사망에 의해 한계지워지지 않았다.
그는 그의 선조들의 존재의 계속이었을 뿐이고,
또한 그의 후손들의 존재 속에 계속 살아갈 것이었다.

\para{인법과 물법}
편리하기는 하지만 전적으로 인위적인,
\wi{신분법}\latin{law of persons}과 \wi{재산법}\latin{law of things} 간의
로마인들의 구별은
지금 우리 앞에 놓인 주제에 대한 탐구를 올바른 경로에서
벗어나게 하는 데 분명 크게 기여했다.
\index{인법|see{신분법}}인법\hanjalatin{人法}{jus personarum}에서 배운 지식은
\index{물법|see{재산법}}물법\hanjalatin{物法}{jus rerum}에 이르러서는 완전히 망각되었다.\footnote{%
  인법과 물법은 각각 신분법과 (물권법을 포함하는) 재산법을 지칭하는
  로마인들의 용어.
  }
그리하여
인법의 원초적 상태에 관해 알게 된 사실로부터
물권법, 계약법, 불법행위법의 기원에 관한 힌트를 전혀 얻을 수가 없는 것처럼
지금까지 생각되어왔다.
이런 사고방식이 잘못되었다는 것은
순수한 고법\hanja{古法}체계 하나를 가져다 놓고
그것에 로마법의 분류를 적용하는 실험을 해볼 수 있다면
분명해질 것이다.
법의 유년기에는
재산법으로부터 신분법을 분리하는 것이 무의미하다는 것을,
두 영역에 속하는 법규칙들이 불가분 서로 엉켜있다는 것을,
후대 법학자들의 구분은 후대의 법에만 적합하다는 것을
알게 될 것이다.
본 저서의 앞 부분에서 말한 것들을 종합하면,
우리가 개인의 소유권에만 관심을 국한하면
초기 소유권의 역사에 대한 어떠한 단서도 얻을 수 없을 것이라는
강한 선험적 개연성을 감지할 수 있을 것이다.
개인적 소유가 아니라 공동소유가 초기법의 진정한 모습이라는 것은,
우리에게 시사점을 주는 소유 형태는 가족의 권리, 친족집단의 권리와
관련된다는 것은 그저 그럴 수도 있겠다는 정도를 넘어선다.
여기서 로마법은 우리를 깨우치는 데 별 도움을 주지 못하거니와,
자연법 이론에 의해 변형되어
우리에게 전달된 바로 그 로마법이
개인적 소유가 소유권의 정상적 상태라는 인상을,
인간집단이 공유하는 소유권은 원칙에 대한 예외에 불과하다는 인상을,
우리에게 심어주기 때문이다.
하지만 원초적 사회의 잃어버린 제도를 탐구하는 연구자라면
반드시 주의깊게 살펴봐야 할 공동체가 하나 있다.
오래 전부터 인도에 정착해 살아온,
인도^^b7유럽 계통의 한 갈래인 사람들 사이에서
그 제도가
어떤 변천을 겪어왔든 간에,
그것은
자신을 배태한 껍질을 완전히 벗어버리지 못했다는 것을 알 수 있을 것이다.
소유의 원초적 형태에 관하여
우리의 신분법 연구로부터
얻을 수 있는
아이디어의
형태에 정확히 들어맞는다는 점에서
즉시
우리의 눈길을 끄는 그러한 소유 형태가
인도인들 사이에서
발견되는 것이다.
인도의 \wi{촌락공동체}\latin{village community}는 조직화된 \wi{가부장제} 사회이자
공동소유자들의 연합체이다.
그것을 구성하는 사람들 간의 인적 관계는
그들의 소유권과 불가분 결합되어 있어서,
영국인 관리들이 이 둘을 분리하려고 하면 이는
영국의 인도 통치에 있어 가장 치명적인 실책이 될 것이다.
이 촌락공동체는 무한히 오래된 것으로 알려져있다.
인도 역사를 어느 방향에서 접근하든 간에,
일반 역사든 지방 역사든 간에,
이 공동체가 진보의 초기부터 존재했음이 항상 발견되어왔다.
대부분 그것의 성격과 기원에 대한 특별한 이론을 갖지 않는
무수한 지식인들과 관찰자들이
이구동성으로 말하기를,
이 사회가
어떤 혁신에도 좀처럼 굴하지 않고 지켜온 관행 중에서도
그것이야말로 가장 파괴되기 어려운 관행이라고 한다.
정복과 혁명이 수차례 휩쓸고 지나갔지만
그것을 어지럽히거나 없애지 못했으며,
인도에서 가장 유익한 통치체제는 언제나
그것을 행정의 기초로 인정하는 통치체제였던 것이다.

\para{공동체와 분할}
성숙한 로마법과 그 자취를 따른 근대법은
공동소유를 소유권의 예외적이고 일시적인 상황으로 바라본다.
이런 견해는
``누구도 자신의 의사에 반하여 공동소유에 묶이지 않는다''%
\latin{Nemo in communione potest invitus detineri}는,
서유럽에서 보편적으로 받아들여지는 법언에 명료하게 드러나 있다.
그러나 인도에서는
관념의 순서가 거꾸로이며,
개별 소유권은 언제나 공동소유권으로 되돌아가는 경향이 있다고 할 수 있다.
그 과정에 대해서는 이미 언급한 바 있다.
아들이 태어나자마자
그는 아버지의 재산에 대해 확정적 권리를 취득한다.
결정권을 행사할 수 있는 나이가 되면,
일정한 경우
가족 재산의 분할을 요구할 권리가
법문\hanja{法文}에 의해 주어진다.
하지만 사실
아버지의 사망시에도 분할은 잘 일어나지 않는다.
재산은 수 세대에 걸쳐 분할되지 않은 채 계속 유지되거니와,
다만
각 세대의 각 구성원들이 미분할된 지분에 대해 법적 권리를 가질 뿐이다.
이러한 공동소유의 토지는 때로는 선출된 관리자에 의해 관리되지만,
일반적으로는, 그리고 일부 지역에서는 언제나,
가장 나이 많은 종친\hanja{宗親}, 다시 말해
가장 손윗 계통의 가장 나이 많은 대표자에 의해 관리된다.
이러한 공동소유자들의 연합체, 즉
토지를 공동소유하는 친족집단은
인도 \wi{촌락공동체}의 가장 단순한 형태이다.\footnote{%
  하지만 <<고대법>>에 대한 폴록의 주석에 따르면
  이러한 공동소유적 촌락공동체보다
  각 가(家)의 개별 소유권을 인정하는
  공동체 유형(특히 인도 중부와 남부에서 지배적이다)이
  더 오래된 유형임이 이후의 인도 연구에서 밝혀졌다고 한다. }
그러나
이 공동체는 친족으로 구성된 동족집단 그 이상이고
조합원들로 구성된 조합 그 이상이다.
그것은 하나의 조직화된 사회이다.
공동재산을 관리하는 것 외에도,
거의 항상 그것은 다수의 스태프들을 통하여
내치\hanja{內治}를,
치안을,
사법\hanja{司法}을,
그리고 조세와 부역\hanja{賦役}의 할당을
수행한다.

\para{인도의 촌락}
내가 기술한 촌락공동체의 형성과정은 전형적인 것으로 간주해도 좋다.
하지만
인도의 모든 촌락공동체가
그러한 단순한 방식으로 결합되어 있다고 생각해서는 안 된다.
인도 북부에서는,
기록에 의하면,
공동체는
거의 항상
혈연관계에 기초한 단일한 연합체의 모습이라지만,
같은 기록은
때로 외부인이 접목되어 들어가는 일이 늘 있어왔다는 것도
알려준다.
일정한 조건 하에서는
단순히 지분을 매수한 것에 불과한 자가 동족집단에 받아들여지는 것이다.
인도 반도 남부에서는
하나의 가족이 아니라 둘 이상의 가족에서 유래한 것으로 보이는
공동체도 다수 존재하거니와,
어떤 공동체는 그 구성이 전적으로 인위적인 것으로 알려져있다.
사실,
서로 다른 카스트에 속하는 사람들이
동일한 사회로
결합하는 것은
공통 조상의 후손이라는 가설에 전혀 부합하지 않는 것이다.
그럼에도 불구하고 이 모든 동족집단에는
최초의 공통 조상에 관한 전승\hanja{傳承}이 내려오거나
혹은 그러한 가정\hanja{假定}이 만들어져있다.
남부의 \wi{촌락공동체}를 집중 연구한
마운트스튜어트 엘핀스톤\latin{Mountstuart Elphinstone}은
이렇게 말한다\paren{<<인도사>>\latin{History of India}, 71쪽. 1905년판}:
``대중적 견해에 따르면,
마을의 지주들은 모두가 그 마을에 정착한 하나 이상의 개인들의 후손이다.
유일한 예외는 원주민 혈통의 구성원에게서 매수 등을 통해
권리를 취득한 사람들뿐이다.
이런 견해는
오늘날
작은 마을에는 지주 가족이 하나만 존재하고
큰 마을에도 몇 안 되는 가족만 존재한다는
사실에서도 확인된다.
그러나 각 마을은 수많은 구성원들로 분기\hanja{分岐}되어왔기에,
농사에 필요한 노동은
소작인이나 노무자의 도움 없이
전적으로 지주들에 의해 행해지는 경우가 적지 않다.
지주들의 권리는 그들에게 집단적으로 속한다.
그들은 거의 항상 그 권리에 대한 다소간의 완전한 지분을 갖지만,
완전한 분리가 일어나는 일은 결코 없다.
가령 어떤 지주가 그의 권리를 매각하거나 저당잡힐 수는 있다.
그러나 그러려면 그는 우선 마을의 동의를 얻어야 한다.
또한 매수인은 정확히 매도인의 지위를 대신하고 그의 모든 의무도 넘겨받는다.
만약 어떤 가족이 소멸하게 되면, 그 지분은 공동재산으로 되돌려진다.''

\para{공동체의 유형}
본서 제5장에서 살펴본 고찰이 엘핀스톤의 인용문을 이해하는 데
독자들에게 도움을 주리라 믿는다.
원시 세계의 어떤 제도도
생동력 있는 법적의제를 통해
원래의 성질에는 없는 유연함을
얻지 못했다면
오늘날까지 전해지지 못했을 것이다.
그리하여 \wi{촌락공동체}는 반드시 혈연관계의 연합체인 것이 아니라,
그러한 연합체\hemph{이거나 아니면} 친족관계의 모델에 기초하여 형성된
공동소유자 집단인 것이다.
이것에 비견되어야 할 유형은 로마의 가족이 아니라,
로마의 \wi{씨족}\latin{gens}임에 틀림없다.
씨족 또한 가족의 모델에 기초한 집단이었다.
그것은 다양한 의제를 통해 확대된 가족이었거니와,
그 의제의 정확한 성질이 무엇인지는 아주 오래 전에 잊혀져버렸다.
역사 시대에 이르렀을 때, 그것의 주된 성질은
촌락공동체에 관한 엘핀스톤의 언급에 나타난 바로 그 두 가지였다.
공통의 기원에 관한 가정\hanja{假定}이 항상 있었거니와,
다만 그 가정은 때로는 실제 사실과 노골적으로 배치되기도 한다.
그리하여, 저 역사학자의 말을 반복하자면,
``만약 어떤 가족이 소멸하게 되면, 그 지분은 공동재산으로 되돌려진다.''
옛 로마법에서도 상속인 없는 상속재산은
씨족원들\latin{gentiles}에게 복귀했던 것이다.
나아가, 로마사 연구자라면 누구나
씨족과 같은 공동체는 외부인을 수용함으로써 수시로 불순물이 혼입되었다고,
그러나 수용의 정확한 방식은 지금으로서는 알 수 없다고 믿고 있다.
이제 인도에서는, 엘핀스톤이 알려주듯이,
동족집단의 동의를 얻어 매수인이 받아들여짐으로써
외부인이 유입되는 것이다.
하지만 수용된 자의 취득의 성질은
\wi{포괄적 승계}\latin{universal succession}에 해당한다.
매수한 지분뿐 아니라,
그는
전체 집단에 대해 매도인이 부담하고 있던 책임도
함께
승계하는 것이다.
\index{가의 매수인}%
그는 바로 가\hanja{家}의 매수인\latin{emptor familiae}으로서,
그가 대체하게 될 사람의 법적인 옷\hanja{[衣服]}을 물려입는 것이다.
그를 수용하는 데 필요한 전체 동족집단의 동의는,
\wi{쿠리아 민회}\latin{comitia curiata}, 즉
동일한 이름을 가진 친족집단이 모인 보다 큰 동족집단인 고대 로마 국가의
입법기구가
\wi{입양}의 허가나 유언의 확인에 반드시 필요하다고 강하게 주장했던
그 동의를 상기시킨다.

\para{러시아 및 크로아티아의 촌락}
인도 \wi{촌락공동체}의 거의 모든 특징들에서
그것이 대단히 오래된 것임을 알려주는 징후를 발견할 수 있다.
이 법의 유년기에는 공동소유가 지배적이었음을,
신분권과 재산권이 서로 엉켜있었음을,
공적 의무와 사적 의무가 혼재되어 있었음을 알려주는
수많은 근거들이 있어서,
이들 공동소유의 동족집단을 관찰함으로써 여러 중요한 결론들을
이끌어내도 무리가 없다 하겠거니와,
유사한 구조를 가진 사회가 세계 어디서도 발견되지 않는다 해도 그러할 것이다.
하지만,
봉건제에 의한 소유권의 격변을 그다지 겪지 않았고
여러 중요한 면에서 동양과도 서양과도 밀접한 친화성을 갖는
유럽의 한 지역에 존재하는 유사한 구조의 현상들이
최근 많은 진지한 관심의 대상이 되고 있다.
학스타우젠\latin{August von Haxthausen} 씨,
텡고보르스키\latin{Ludwig Tengoborski} 씨 등의 연구자들이
러시아의 촌락은 사람들의 우발적인 결합도 아니고
그렇다고 계약에 기초한 결합도 아님을 보여주었다.\footnote{%
  러시아의 촌락공동체는 흔히 `미르'(mir) 혹은 `옵쉬나'(obshchina)라고 불린다. }
그것은 인도의 공동체와 마찬가지로 자연적으로 조직화된 공동체인 것이다.
물론,
이들 촌락은 이론적으로는 언제나 어떤 귀족의 소유지이며,
역사 시대 내내
농부들은
그 영주의 토지에 예속된 농노\latin{predial serf}로,
혹은 보다 일반적으로는
그에게 신분적으로 예속된 농노\latin{personal serf}로 전락해갔다.
그러나
이러한 상급소유권의 압력도
촌락의 고대적 구조를 파괴시키지 못했다.
농노제를 도입한 것으로 여겨지는 러시아 짜르의 입법도
기실 옛 사회질서를 유지하는 데 불가결한 저 협력관계를
농부들이 버리지 못하도록 막기 위해 만들어진 것이었다.
마을 사람들 간의 종족\hanja{宗族}적 관계를 고려할 때,
\wi{신분법}과 \wi{재산법}의 혼재를 고려할 때,
또한 다양한 자생적 자치규범들을 고려할 때,
러시아의 촌락은 인도 \wi{촌락공동체}의 거의 정확한 반복으로 보인다.
그러나 한 가지 중대한 차이점이 있으니,
크게 관심을 둘 만하다.
인도 촌락의 공동소유자들은,
비록 그들의 소유권이 통합되어 있기는 하나,
각자 자기만의 권리를 가지며,
이러한 권리들이 분할되면 그 분리는 완전하고 또한 무한히 지속된다.
러시아 촌락에서도 이론적으로 권리들의 분할은 완전하지만,
그러나 여기서는 그것이 일시적인 데 그친다.
일정한, 그러나 모든 경우에 다 동일하지는 않은,
시간이 경과하면
분리된 소유권들은 소멸하여,
그 촌락의 토지가 하나의 덩어리로 합쳐지고, 그후
공동체를 구성하는 가족들 간에 식구 수에 따라 재분배된다.
이러한 재분할이 행해지면,
가족들의 권리와 개인들의 권리는 다시
다수의 계통으로 가지를 칠 수 있거니와,
이러한 가지치기는 또 다른 분할의 시기가 도래할 때까지 계속된다.
이러한 소유권 유형의 변종인 더욱 특이한 형태가
오랫동안
투르크 제국과 오스트리아 왕가의 영토 사이에 분쟁지역이 되어온
몇몇 나라들에서 발견된다.
세르비아, 크로아티아, 오스트리아령 슬라보니아에서도
촌락들은 공동소유자이자 친족관계인 사람들로 구성된 동족집단이다.\footnote{%
  이들 남슬라브 지역의 씨족공동체는 흔히 `자드루가'(zadruga)라고 불린다. }
그러나 여기서는 공동체의 내부 구조가
앞서 살펴본 두 사례와 상이하다.
여기서는 공동소유의 재산이 실제로도 분할되지 않고
이론적으로도 분할될 수 없다고 간주된다.
토지 전부가 마을 사람 모두의 공동의 노동으로 경작되고,
수확물은 매년 가구별로 분배되거니와,
때로는 각 가구의 필요에 따라서,
때로는 특정인에게 \wi{용익권}\latin{usufruct}의 일정 몫을 주는 규칙에 따라서,
분배된다.
동유럽의 법학자들은 이 모든 관행이
초기 슬라보니아법에 기초한 원리에서 나왔다고 주장하거니와,
그것은 가족의 재산은 영원토록 분할될 수 없다는 원리인 것이다.

\para{공동체의 다양성, 소유권 기원에 관한 문제}
우리의 탐구에서 이러한 현상들이 큰 관심의 대상이 되는 이유는
원래 공동으로 재산을 소유하던 집단 \hemph{속에서}
어떻게 해서 개별적 소유권이 발달했는가에 대해
그 현상들이 실마리를 던져주기 때문이다.
개인도 아니고,
독립된 가족도 아니고,
가부장적 모델에 기초한 더 큰 사회단위에
재산이 한때 속해 있었다고 믿을 만한 강력한 근거가 있다.
그러나
고대로부터 근대로의 소유권의 변화 양상은,
그 자체로도 모호하지만,
서로 다른 여러 형태의 \wi{촌락공동체}들이 발견되어 조사되지 않았다면
무한히 더 모호해졌을 것이다.
인도^^b7유럽 혈통의 민족들 사이에서
현재 관찰되는, 혹은 최근까지 관찰되었던,
가부장적 집단들의 내부 구조의 다양성은 충분히 주목할 가치가 있다.
스코틀랜드 산악지대의 미개한 \wi{씨족}들의 씨족장들은
그들이 관할하는 가\hanja{家}의 수장들에게
아주 짧은 간격으로, 때로는 매일,
식량을 분배해주었다고 한다.
오스트리아와 투르크의 변방 지역의 슬라보니아 촌락에서도
촌장들에 의한 주기적 분배가 행해지고 있거니와, 다만
이 경우에는 일년에 한번씩 수확물 전체를 분배하는 것이다.
하지만 러시아의 촌락에서는
재산은 불가분이라는 관념이 존재하지 않아서
개별 소유권이 성장할 수 있도록 기꺼이 허용되지만,
그러다가 일정 시간이 경과하면 소유권 분리의 진행이 단호히 중단된다.
인도에서는 공유재산의 불가분성이 부재할 뿐만 아니라,
거기서 분리된 소유권은 영구히 존속할 수 있고
무한히 많은 파생적 소유권으로 가지치기를 할 수 있지만,
재산의 분할은
뿌리깊은 관행에 의해,
그리고 동족집단의 동의 없는 이방인의 유입을 막는 규칙에 의해,
사실상\latin{de facto}
제한되고 있다.
물론,
이러한 다양한 형태의 촌락공동체들이
어디서나 똑같은 방식으로 이루어지는 진화 과정의
각 단계들을 대표한다고 주장하려는 것은 아니다.
이렇게까지 주장하기에는 증거가 부족하지만,
그러나
이들 증거에 의해,
공동체의 공동의 권리로부터
개인의 분리된 권리가 점차 해방되어나옴으로써
우리에게 익숙한 형태의 사적 소유권이
주로 형성되었을 것이라는 추측이
그나마 덜 뻔뻔한 주장이 되는 것은 사실이다.
신분법에 관한 우리의 연구로부터,
가족이 \wi{종족}\hanja{宗族}집단으로 확장되고
그후 종족집단이 개별 가\hanja{家}들로 해체된다는 것을,
그리고 종국에는
그 가\hanja{家}가 개인에 의해 대체되는 것을
알 수 있었을 것이다.
그런데
이제 우리는 이 변화의 각 단계에 상응하여
소유권의 성질도 함께 변화한다고 시사받고 있는 것이다.
이런 생각에 일말의 진실이 들어있다면,
소유권의 기원에 관한 이론가들이 널리 제기했던 문제에
그것이
중대한 영향을 줄 수 있다는 것을 알아챌 수가 있다.
그들이 주로 불러일으킨 문제---아마도 해결불가능한 것일텐데---는
사람들이 처음 서로의 점유를 존중하도록 만든 동기는 무엇이었는가? 라는 것이다.
이를 달리 표현하면---그런다고 답을 발견할 희망이 그리 많이 커지는 것은
아니지만---하나의 복합집단이 다른 복합집단의 소유물에 대해
무심해지게 되는 이유는 무엇인가라는 질문 형태로 바꿀 수 있을 것이다.
그러나,
사적 소유권의 역사에서 가장 중요한 과정이
친족의 공동소유로부터 사적 소유권이 점차 분리되어 나오는 과정이라고 한다면,
저 중대한 질문은 역사 시대의 모든 법의 초입에 놓여있는 문제, 즉
애초에 사람들을 가족의 결합으로 묶어주었던 동기는 무엇이었는가? 라는 질문과
동일한 것이 된다.
다른 학문의 도움 없이 법학만으로는 이러한 질문에 대한 답을 찾을 수 없다.
사실만 알 수 있을 뿐이다.

\para{고대의 양도 곤란성}
고대사회의 미분할된 재산 상태는
집단의 재산으로부터 하나의 지분이 완전히 분리되자마자 나타나는
특유의 날카로운 분할과 모순되지 않는다.
물론 분할이라는 현상은
그 재산이 어떤 새로운 집단의 소유물이 된다고 상정되는 상황에서
생겨나는 것이므로,
분리된 상태에서의 그것의 거래는
두 개의 대단히 복합적인 집단 간의 거래가 된다.
앞서 나는 고대법을 근대 \wi{국제법}에 비유한 바 있거니와,
그것이 다루는 권리와 의무의 주체인 단체의 크기와 복합성에 비추어 그리 했다.
고대 세계에 알려진 계약과 양도는
개인들이 당사자가 되는 것이 아니라
사람들로 조직된 단체들이 당사자인 계약과 양도이기에,
그것은 사뭇 의례\hanja{儀禮}적일 수밖에 없다.
거기에는
참석자 모두의 기억에 거래를 각인시키기 위한
다양한 상징적 행동과 단어들이 요구되고,
또한 지나치게 많아 보이는 증인들의 참석이 요구된다.
이러한 특징들 및 기타 부수적인 특징들로부터
재산의 고대적 형태에 보편적으로 나타나는 경직성이 생겨난다.
슬라보니아의 경우처럼
때로 가족의 재산은 전혀 양도불가능하다.
좀 더 흔하기로는,
대부분의 게르만 부족법에서처럼
양도가 완전히 불법은 아니지만
수많은 사람들의 동의가 요구되어
사실상 양도가 거의 불가능한 경우도 있다.
이러한 장애물이 없는
또는 극복될 수 있는 경우에도,
미세한 잘못 하나조차 허용하지 않는 철두철미한 의례성이
양도 행위 자체에 널리 부담으로 작용한다.
한결같이
고대법에서는
아무리 이상하게 보이는 몸짓 하나라도,
아무리 그 의미가 망각된 음절 하나라도,
아무리 쓸모없어 보이는 증인 하나라도
빠뜨려서는 안 된다.
엄숙한 의례 하나하나가
그것을 수행할 법적 권리를 가진 사람들에 의해
정확하게 수행되어야 하고,
그렇지 못하면
양도가 무효가 되어,
매도인은 그가 헛되이 내주려 했던 그 권리를 그대로 가지게 된다.

\para{물건의 분류}
이용과 향유의 객체인 물건의 자유로운 유통에 대한 이러한 다양한 장애는
사회가 조금이라도 활기를 얻게 되면 즉시
고통으로 느껴지기 시작한다.
진보적 공동체가 이를 극복하기 위해 애써 강구한 수단들은
물권법의 역사의 주요 주제를 이룬다.
그러한 수단 중에
그 고대성과 보편성에 있어서 다른 것들을 능가하는 한 가지가 있다.
대다수 초기 사회에서 자생적으로 생겨난 것으로 보이는 이 관념은
바로 물건을 종류에 따라 분류하는 것이다.
어떤 종류의 물건은 다른 종류의 물건보다 낮은 가치의 지위에 놓이지만,
동시에 옛 법이 부과한 족쇄로부터 면제된다.
그후,
낮은 등급의 물건을 규율하는 양도 및 상속 규칙의 편리함이
널리 인식되고,
점진적인 혁신을 통해 낮은 가치의 물건 유형이 갖는 유연성이
전통적으로 높은 지위에 있던 물건 유형에도 전파되어간다.
로마 물권법의 역사는 \wi{악취물}\hanjalatin{握取物}{res mancipi}이
비악취물\hanjalatin{非握取物}{res nec mancipi}에 동화되어가는
역사이다.
대륙 유럽의 물권법의 역사는
봉건적 토지법이 로마법을 이어받은 동산\hanja{動産}법에 의해
대체되어가는 역사이다.
영국의 소유권의 역사는 아직 완성되지 않았지만,
\wi{인적재산}\hanjalatin{人的財産}{personalty}법이
\wi{물적재산}\hanjalatin{物的財産}{realty}법을
흡수하고 폐기시킬 공산이
농후하다.

\para{고대의 분류들, 상급재산과 하급재산}
향유의 객체인 물건의 유일한 \hemph{자연법적} 분류는,
물건의 본질적 차이에 따른 유일한 분류는,
동산\latin{movables}과 부동산\latin{immovables}의 구분뿐이다.
법학에서 이 분류는 익숙한 것이지만,
이것은 로마법에 의해 사뭇 느리게 발달했거니와,
결국 로마법의 마지막 단계에 가서야 그것에 수용된 것을
우리가 물려받은 것이다.
고대법의 분류들은 때로 이 분류와 피상적인 유사성을 가질 뿐이다.
고대법의 물건의 종류 중에는 부동산을 포함하는 것이 있지만,
부동산과 아무 관련 없는 다수의 물건을
부동산과 함께
묶어 분류하거나,
아니면
부동산과 무척 가까운 권리들을 부동산과 따로 떼어 분류하는 경우가 흔하다.
그리하여
로마법의 \wi{악취물}은 토지뿐만 아니라 노예, 말, 소를 포함한다.
스코틀랜드법은 토지를 몇몇 다른 담보권들과 함께 분류한다.
힌두법은 토지를 노예와 함께 묶는다.
\index{정기부동산임차권}%
한편, 영국법은 정기\hanja{定期}부동산임차권\latin{lease of land for years}을
토지에 대한 다른 권리들과 분리하여
`\wi{부동산에 관한 인적재산}'\latin{chattel real}이라 이름 하에
\wi{인적재산}에 포함시킨다.
더욱이 고대법의 분류는 상급과 하급의 우열을 나누는 분류이다.
동산과 부동산의 구분은,
적어도 로마법에 관한 한,
그러한 가치의 차이를 상정하지 않는 것이었다.
그러나
악취물은 비악취물에 비해서 처음에는 확실히 우월한 지위를 누렸다.
스코틀랜드의 \wi{세습재산}\latin{heritable property}과
영국의 \wi{물적재산}도 이것들에 대비되는 인적재산에 비해 그러했다.
모든 법체계에서 법률가들은
이러한 분류를 어떤 합리적인 원리로 설명해보려는
노력을 아끼지 않았다.
그러나 구별의 근거를 법철학적으로 찾으려는 노력은 허사로 끝날 수밖에
없거니와,
그것은 철학이 아니라 역사에 속하는 문제이기 때문이다.
대부분의 경우를 포괄할 수 있을 만한 설명은 이러하다:
다른 것들보다 우대받는 향유의 객체는
각 공동체의 초창기에 가장 먼저 알려져있던 유형의 물건들이었고,
따라서 \hemph{재산}\latin{property}이라는 이름으로 강조되어
불리면서 존종받았다.
한편, 우대받는 객체에 포함되지 못하는 물건들은
상급재산의 목록이 정착되고 나서 나중에야
그 가치가 알려졌기 때문에 낮은 지위에 자리매김되었다.
이들은 처음에는 알려지지 않았거나, 드물었거나, 용도가 제한적이었거나,
아니면 우대받는 객체에 딸린 부속물로만 취급되었다.
그리하여 로마법의 악취물은 가치가 큰 여러 동산들을 포함하지만,
아주 값나가는 보석들은
초기 로마인들에게 알려져있지 않았기 때문에
악취물로 분류되지 못했던 것이다.
마찬가지로 영국법의 `\wi{부동산에 관한 인적재산}'은
봉건 토지법 시대에는 그러한 부동산권\latin{estate}이 흔하지도 않았고
별 가치도 없었기 때문에 \wi{인적재산}의 지위로 떨어졌던 것이다.
그러나
무엇보다 주목할 점은
중요성이 커지고 숫자가 늘어난 뒤에도
이들 물건과 권리들이 계속해서 낮은 지위에 머물렀다는 것이다.
왜 우대받는 향유의 객체에 계속 포함되지 못했을까?
고대법의 분류가 갖는 완고함에서
한 가지 이유를
찾을 수 있다.
무지한 사람들과 초기 사회들의 공통된 특징은
관행을 통해 익숙해진 것의 특정한 적용에 매몰되어
일반 원리를 거의 발견하지 못한다는 것이다.
그들은 일상 경험에서 만나는 특수한 사례들로부터
일반적 공준\hanja{公準}을 분리하지 못한다.
그리하여,
잘 알려진 유형의 물건에 대한 명칭을
그것과 정확하게 닮은 향유의 객체이자 권리의 대상인 물건에
붙이기를 거부하는 것이다.
그러나
법의 힘만큼이나 안정적인 힘을 대상에 가하는
이러한 영향력 외에도,
그후
계몽주의와 일반적 공리\hanja{功利} 개념의 진보에 유사한
다른 영향력이 더해졌다.
법원과 법률가들은
우대받는 물건의 양도, 회수, 상속에 필요한 성가신 형식요건들의
불편함에 마침내 눈을 뜨게 되어,
새로운 유형의 물건들에
유년기 법의 특징인 법기술적 속박을 씌우기를 점점 꺼리게 된다.
그리하여
법학 체계에서
이 후자의 것들을 계속 낮은 등급에 머물러둠으로써,
그것의 양도가
신의성실에 걸림돌이 되고 기망행위에 디딤돌이 되는
옛 양도방식보다 간편한 과정으로 이루어질 수 있도록
하려는 경향이 일어난다.
우리에게는 고대 양도방식의 불편함을 과소평가하는 위험이 있는 것 같다.
우리의 양도수단은 서면에 의한 것이기에,
전문가에 의해 신중하게 작성되면 그 문언에는 흠결이 별로 없다.
그러나 고대의 양도는 서면이 아니라 \hemph{행위}에 의한 것이었다.
몸짓과 말이 서면의 법기술적 문언을 대신했고,
공식\hanja{公式}을 조금이라도 잘못 발음하면,
상징적 행위를 하나라도 빠뜨리면,
그 절차는 치명적인 흠결이 있는 것으로 간주되었다.
마치 2백년 전\footnote{%
  1677년 사기방지법(Statute of Frauds) 제정 이전을 말하는 듯하다.
  이 법률로 부동산권의 양도나 임대차, 일정 유형의 계약이나 유언 등은
  서면으로 행하지 않으면 효력을 인정받을 수 없게 되었다.
  }
영국에서 \wi{유스}\latin{use}\footnote{%
  `신탁'의 전신인 유스---이익(benefit)의 뜻으로,
  가령 A에게 보통법상의 부동산권을 양도하면서
  이를 B의 이익을 위해 보유하라고 지시하는 설정을 말한다---는 원래 보통법에서는
  효력이 인정되지 않았다. 그러나 형평법법원이 이를 보호하기 시작하자,
  이를 이용한 각종 탈법을 막기 위해
  유스금지법(Statute of Uses, 1536)이 제정되었다. 이로써
  ``to A, to the use of B'' 같은 부동산권 설정에서 B는 동법에 의해
  완전한 보통법상의 권리를 얻게 된다.
  그러나 동법의 헛점을 파고든, 가령
  ``to A, to the use of B, to the use of C'' 같은 경우의 C에게
  이후 형평법법원에 의한 보호가 다시 주어지고
  C는 형평법상의 소유자로 불리게 된다.
  그런데 이러한 설정은 흔히 `신탁'(信託\,trust)이라 부른다.
  그렇다면 본문은 이것이 아니라 유스금지법에 의해 보통법적 효력이
  인정되는 유스, 가령
  ``to A and his heirs, to the use of B for his life'' 따위를
  말하는 것이 아닐까 한다.
}의 진술이나
\wi{잔여권}\hanjalatin{殘餘權}{remainder}\footnote{%
  가령 ``to A for life, and then to B and his heirs''라며
  부동산을 양도할 때,
  B가 갖는 장래의 권리를 `잔여권'이라 부른다.
}의 설정에
중대한 실수가 있으면 양도행위가 무효로 되었던 것처럼 말이다.
사실, 이것으로는 원시적 의례적 절차가 갖는 문제점을 절반만 말한 것에 불과하다.
서면이든 행위든 복잡한 양도요건이 \hemph{토지}의 양도에만 요구되는 한,
급하게 거래할 일이 적은 유형의 재산의 양도인지라
실수할 가능성은 그리 크지 않다.
그러나 고대 세계의 상급재산의 범주에는
토지뿐만 아니라 몇몇 아주 평범한, 그리고 몇몇 아주 값나가는 동산들이
들어있었다.
사회의 수레바퀴가 빠르게 굴러가기 시작하자,
말이나 소의 양도에, 혹은
고대 세계에서 가장 값나가는 동산---노예---의
양도에 대단히 복잡한 방식을 요구하는 것은
큰 불편을 초래했을 것이 틀림없다.
분명
이러한 물건들을 불완전한 방식으로 양도하는 일이,
따라서 불완전한 권리가 보유되는 일이
지속적으로, 심지어 일상적으로 벌어졌을 것이다.

\para{악취물과 비악취물}
옛 로마법에서 \wi{악취물}은 토지---역사 시대에는 이탈리아의 토지---와
노예와 짐을 끄는 가축, 예컨대 말이나 소를 의미했다.
의심할 여지 없이 이러한 유형의 객체는 농사를 짓기 위한 주요 수단이었고,
원시시대의 사람들에게는 무엇보다 중요한 물건이었을 것이다.
처음에는 이들 물건이 `재물' 또는 `재산'이라며 강조하여 불렸을 테고,
이들을 양도하는 방식이 바로 \wi{악취행위}\latin{mancipium; mancipation}였으나
이들을 ``악취행위가 요구되는 물건''이란 뜻의
`악취물'이라는 명칭으로 부르게 된 것은 나중에 가서였을 것이다.
그런데 이들 외에도,
악취행위의 복잡한 의례를 전부 거칠 필요는 없다고 생각되는
유형의 객체들이 존재했거나 발달하게 되었을 것이다.
이 후자의 물건들의 소유권 양도에는
통상적으로 요구되는 형식적 요건 중 일부,
즉 현실적인 교부, 물리적인 이전만 행해지면 충분하다고 생각되었다.
이것이 바로 \hemph{\wi{인도}}\hanjalatin{引渡}{tradition}이거니와,
이는 소유권 변동의 가장 명백한 지표인 것이다.
이런 물건들을 고법\hanja{古法}에서는
``악취행위가 필요치 않은 물건''이란 뜻의
`비악취물'이라고 불렀으니,
처음에는 그다지 값어치 없는 것들이었고
한 집단에서 다른 집단으로 양도될 일도 별로 없는 것들이었을 것이다.
하지만, 악취물의 목록은 전적으로 폐쇄적이었으나,
비악취물의 목록은 개방적이었고 무한히 확장되어갔다.
그리하여 인간이 물질적 자연을 하나씩 정복해감에 따라
비악취물은 항목이 하나씩 늘어나거나
기존의 항목에 개선이 이루어졌다.
결과적으로 부지불식간에
이들이 악취물과 동등한 가치를 갖는다고 여겨지고, 따라서
본질적으로 낮은 등급의 물건이라는 인상이 사라져가면서,
사람들은 복잡하고 장엄한 의례절차보다
이들의 양도에 수반되는 간편한 요건이 여러 모로 장점을 갖는다는 것을
알아채기 시작했다.
로마 법률가들은
법 개선의 두 가지 장치,
즉 \wi{법적의제}\latin{fiction}와 \wi{형평법}\latin{equity}을
열심히 활용하여
\wi{인도}에 사실상 악취행위와 동일한 효과를 부여하려했다.
비록 로마의 입법자들은
단순한 교부에 의해
악취물의 소유권이
즉시 이전한다는 입법에는
오랫동안 몸을 사려왔으나,
마침내 \wi{유스티니아누스}에 의해 이러한 조치가 단행되어
\wi{악취물}과 비악취물의 차이가 사라졌고,
인도는 법이 인정하는 유일한 양도방식이 되었다.
로마 법률가들이 일찍부터
인도를 높이 평가했기 때문에
근대 법학자들은 그것의 진짜 역사를 잘 모르는 경향이 있다.
\wi{인도}는 ``\wi{자연법}적'' 취득방식으로 분류되었거니와,
그것이 이탈리아 부족들 사이에서 널리 행해지는 방식이었을 뿐만 아니라,
또한 물건을 취득하는 가장 단순한 방식이었기 때문이다.
로마 법학자들의 표현들을 미루어 짐작하건대,
그들은 자연법에 속하는 인도가 \wi{시민법}상의 제도인 악취행위보다
더 오래된 것이라고 생각했을 것이 분명하다.
이것은, 말할 것도 없이, 진실과 정반대인 것이다.

\para{세습재산과 취득재산, 부동산과 동산}
악취물과 비악취물의 구분은 문명 세계가 크게 빚지고 있는 구분 유형이거니와,
모든 물건을 그중 일부는 그 자체로 가치있는 것으로 자리매김하고
다른 것들은 낮은 등급의 범주에 집어넣는 구분 방식이다.
멸시받고 무시당한
낮은 등급의 물건들은
원시법이 즐겨 채용한 복잡한 의례절차에서 처음 면제된 것들이었으나,
그후 지성의 상태가 진보하자
단순한 방법의 이전 및 회복 방식이 널리 사용되었고 이는
그 편리성과 단순성으로 인해
고대로부터 내려온 성가신 의례절차를 폐기하는 데 모범으로 작용했다.
그러나 몇몇 사회에서는
`재산'을 얽어매는 질곡이 무척 복잡하고 강고해서,
그것이 그리 쉽게 완화되지 못하는 경우가 있다.
인도에서 남자 아이가 출생하면,
전술했듯이,
인도의 법은 그에게 재산권을 완전히 부여하고,
재산의 양도에 그의 동의가 필수요건이 된다.
마찬가지 정신에서,
게르만 민족들의 일반적 관습---앵글로색슨의 관습이 예외였던 것은
주목할 만하다---은
아들들의 동의 없는 양도를 금지했다.
또한 슬라보니아의 원시법은 양도 자체를 아예 금지했다.
분명, 이러한 장애가
모든 종류의 물건에 확대적용되는 것인 한, 이는
물건의 종류를 구분함으로써 극복될 수 있는 장애가 아니었다.
따라서, 일단 진보의 물결이 일어나자,
고대법은 또 다른 성격의 구분을 행함으로써 장애에 대처했거니와,
이는 물건의 성질이 아니라 물건의 기원에 따라 구분하는 분류인 것이다.
두 가지 분류체계의 흔적을 다 가지고 있는
인도에서는, 지금 우리가 다루는 분류를
\wi{세습재산}\latin{inheritances}과
\wi{취득재산}\latin{acquisitions} 간의 힌두법상의 분류에서 볼 수 있다.
아버지의 세습재산은 자식들이 태어나자마자 그들과 공유된다.
그러나 대부분 지방의 관습에 따르면
아버지가 그의 생애 동안 획득한 취득재산은 전적으로 그의 것이며
그는 마음대로 그것을 양도할 수 있다.
유사한 구분이 로마법에서도 없지 않거니와,
가부장권에 대한 최초의 혁신은
아들이 군복무 중 취득하는 재산은 아들 자신의 것으로 삼도록
허용하는 형태를 취했던 것이다.
그러나 이러한 방식의 분류를 가장 광범위하게 사용한 것은
게르만인들이었을 것이다.
누차 언급했듯이,
\wi{자유소유지}\latin{allod}는 양도불가능은 아니었지만
대체로 양도가 무척 어려웠다.
더욱이 종친\hanja{宗親}들만이 그것을 상속받을 수 있었다.
그리하여 여러 특별한 분류방식이 인정되기에 이르렀으니,
그 모두가 자유소유지와 불가분 결합된 불편함을 줄이려는 것이었다.
예컨대,
게르만법의 큰 부분을 차지하는
\wi{속죄금}\hanjalatin{贖罪金}{wehrgeld}, 즉
친족이 살해되어 받은 배상금은
가족재산의 일부를 구성하지 않았기에
전혀 다른 상속규칙에 따라 귀속되었다.\footnote{%
  렉스 살리카(Lex Salica) 제59장에 따르면 자유소유지는
  직계비속--부모--형제자매--고모--이모--최근친 부계혈족(6촌까지) 순으로 상속한다.
  하지만 제62장에서는
  살인에 기한 속죄금은 망자의 자식들이 그 절반을 가져가고,
  나머지 절반은 부계와 모계의 최근친 혈족들이 나누어가진다고 규정한다. }
마찬가지로,
과부가 재혼할 때 부과되는 벌금인
\wi{레이푸스}\latin{reipus}도
지불받는 자의 자유소유재산에 속하지 않았고,
따라서 그것의 귀속에서는 종친의 특권이 무시되었다.\footnote{%
  렉스 살리카 제44장에 의하면, 과부와 혼인하고자 하는 남자는
  소집된 법정에서 3솔리두스를 지불해야 한다. 특이하게도
  이 벌금은 죽은 남편의 여계혈족에게 귀속되었다.
  1순위는 전남편의 누이의 장자, 즉 조카(생질)이고,
  2순위는 조카딸의 장자,
  3순위는 이모의 아들,
  4순위는 모계 사촌의 아들,
  5순위는 전남편의 어머니의 남자형제, 즉 외삼촌이며,
  그 다음으로 비로소 남계혈족에게 넘어간다. }
또한, 인도인들과 마찬가지로,
게르만법도
가\hanja{家}의 수장의 \wi{취득재산}을 그의 \wi{세습재산}과 구별하여,
취득재산은 그가 보다 자유롭게 처분할 수 있었다.
다른 유형의 분류들도 인정되었거니와,
가장 친숙한 것은 부동산과 동산의 구분일 것이다.
그러나 동산은 몇 가지 하위범주로 다시 세분되었고
각각에는 서로 다른 규칙이 적용되었다.
이렇게 분류가 많은 것은,
로마제국을 정복한 게르만인들의 미개한 특성이라고 느껴질 수도 있겠으나,
실은
로마의 국경 부근에서 오랫동안 체류하는 동안
로마법적 요소가 그들 법에 상당히 많이 유입되어 들어갔다는 것으로
설명할 수 있을 것이다.
\wi{자유소유지}를 제외한 물건의 양도와 상속을 규율하는 법규칙의 대부분은
로마법에서 유래한 것으로 그 기원을 어렵지 않게 추적할 수 있거니와,
그것들은 아마도 장기간에 걸쳐
조금씩 로마법에서 빌려왔을 것이다.
재산의 자유로운 유통에 대한 장애가
이러한 방법들로
얼마는 극복되었을 것인가는 우리로서는 추측조차 할 수 없으니,
근대사에서는 내가 말한 그러한 구분들이 존재하지 않기 때문이다.
전술했듯이,
자유소유 형태의 재산은 \wi{봉건제}의 와중에서 망각되었고,
봉건제가 완전히 공고화된 이후에는
서구 세계에 알려졌던 모든 구분 중에 오직 한 가지 구분만이
사실상 남게 되었다.
부동산과 동산의 구분이 그것이다.
표면적으로 이 구분은 로마법이 마침내 채택한 그것과 동일한 것이었으나,
중세의 법은 부동산을 동산보다 훨씬 높게 평가했다는 점에서
로마법과 달랐다.
하지만 이 한 가지 예만 가지고도
이를 포함하는 분류 장치들의 중요성을 보여주는 데 충분하다.
\index{나폴레옹 법전}%
프랑스 법전들에 기초한 법체계를 가진 모든 나라들에서는,
다시 말해 대륙 유럽의 대부분의 지역에서는,
언제나 로마법적이었던 동산법이 봉건 토지법을 대체하고 무효화시켰다.
주요국 중에
이러한 변화가 어느 정도 진행되었으나 아직 완성되지 못한
유일한 국가가 바로 영국이다.
게다가 주요 유럽 국가 중에 영국은
자연법적으로 용인되는 유일한
분류\footnote{%
  동산과 부동산의 구분을 말한다.
}로부터
고대법의 분류를 이탈하게 만들었던
바로 그 영향력에 의해
동산과 부동산의 구분이
상당 정도 방해받은
유일한 국가인 것이다.
영국법의 구분도 대체로 부동산과 동산에 일치하지만,
어떤 종류의 동산은 \wi{세습동산}\latin{heirloom}으로서 부동산과 함께 상속되고,
어떤 종류의 부동산권은 역사적 이유에서 \wi{인적재산}으로 분류되어왔다.
영국법이
법발달의 주류와 동떨어져
고법\hanja{古法}의 현상을 재현한 예는 이것만이 아니다.

\para{시효의 이론들}
소유권에 대한 고대적 질곡을 비교적 성공적으로 완화시킨 장치들 가운데
한 두 가지를 더 다루고자 한다.
그러나 본 저서의 구도상 아주 오래된 것들만 언급하는 것을
양해해 주시길 바란다.
그 중 하나는 조금 자세히 들여다볼 필요가 있거니와,
초기법의 역사를 잘 모르는 사람들에게는
근대법이 아주 천천히 그리고 대단히 어렵게 겨우 인정하게된
어떤 원리가 유년기의 법학에는 실로 친숙한 원리였다는 것이
쉽게 믿기지 않을 것이기 때문이다.
법의 원리들 중에
근대인들이 그 유용성에도 불구하고
수용하기를 꺼리고 그 합당한 결론들을 관철시키기를 꺼린
원리로
로마인들이 `\wi{사용취득}'\hanjalatin{使用取得}{usucapion}이라고
불렀고
근대법에서는
이것을 물려받아
`\wi{취득시효}'\latin{prescription}라고 부르는 제도에
비할 만한 것이 또 있을까 싶다.
가장 오래된 로마법 규칙으로
\wi{12표법}보다도 더 오래된
이 실정규칙은
일정 기간 동안 중단 없이 점유상태가 지속된 물건은
그 점유자의 소유로 된다는 규칙이었다.
점유의 기간은 대단히 짧았으며---물건의 성질에 따라
1년 또는 2년\footnote{%
  부동산은 2년, 동산은 1년.
}---역사
시대에는 특정한 방식으로 점유가 개시된 경우에만 작동이 허용되었다.\footnote{%
  정당한 원인(iusta causa, 가령 매매)에 의해, 그리고
  선의로(bona fide, 가령 매도인이 무권리자임을 모르는 상태)
  점유가 개시되어야 한다.
  단, 도품(盜品)이나 강탈된 물건은 대상에서 제외된다. }
그러나 생각건대 그전에는
지금 우리가 전거들에서 보는 것보다 훨씬 덜 엄격한 요건으로
점유가 소유권으로 전환되었던 것 같다.
전술했듯이,
나는 사실상의 점유에 대한 사람들의 존중이
법학 그 자체만으로 설명될 수 있는 현상이라고 주장하지 않는다.
단지 \wi{사용취득}의 원리를 채택함에 있어 원시사회는
근대인들 사이에 그것의 수용을 방해했던
어떤 사변적인 의심이나 망설임을 전혀 갖지 않았음을 말하고 싶을 뿐이다.
근대 법률가들 사이에서 \wi{취득시효}는
처음에는 반감의 대상이었고
나중에는 어쩔 수 없이 수용하는 것으로 여겨졌다.
영국을 포함한 몇몇 나라에서는
과거 특정 시점---일반적으로 몇몇 선임 국왕들의 치세 원년---이전에 입은
손해에 기해서 제소가
이루어지는 것을 막는 것 이상으로
입법이 나아가지를 못했다.
중세가 마침내 마감하고
제임스 1세가 영국왕에 오른 이후에
비로소
사뭇 불완전한 것이었으나 진정한
\wi{출소기한법}\hanjalatin{出訴期限法}{statute of limitation}이
제정되었다.
대다수 유럽 법률가들이 계속 읽어왔음이 분명한
로마법의 가장 유명한 분야 하나를 근대 세계에 재현하는 데
이렇게 오래 걸린 것은
무엇보다 \wi{교회법}의 영향 탓이다.
교회법의 기원이 된 교회의 관습은
성스러운 권리 또는 그에 준하는 권리로 여겨지는 것을 취급하므로,
교회가 인정한 특권은
아무리 오랫동안 불사용\hanja{不使用}되더라도
상실될 수 없다고
자연스레
간주되었다.
이런 견해에 따라,
후대의 안정된 교회법도 \wi{취득시효}를 배척하는 특징을 나타냈다.
교회법학자들에 의해 세속 입법이 따를 본보기로 치켜세워지자
교회법이 세속 입법의
핵심 원리들에 특유의 영향을 미친 것은 당연한 일이었다.
유럽 전역에 걸쳐 형성되어가던 \wi{관습법} 체계들에게 교회법은
비록 로마법보다 적은 수의 명시적 법규칙들만 가져다 주었지만,
놀랄 정도로 많은 근본적 문제에 관하여
전문가들에게 어떤 선입견을 심어준 것으로 보이며,
이렇게 해서 형성된 경향은 각 법체계가 발달하면서 점점 강화되었다.
그렇게 만들어진 성향 중 하나가 취득시효에 대한 혐오였다.
그러나 만약
세속 세계의
스콜라주의 법학자들\footnote{%
  주해학파(Commentators)를 말하는 듯하다.
}의
법리와 일치하지 않았다면
그러한 편견이
과연 그렇게 강력하게 작용했을 것인지는 의문이다.
이들의 가르침에 따르면,
실제 입법이 아무리 반복되더라도,
\hemph{권리}는 아무리 오래 방치되어도
사실상 파괴될 수 없는 것이었다.\footnote{%
  여기서 `입법'은 제국 아래 영방이 주권을 사실상 행사하는 것을,
  `권리'는 황제의 주권을 뜻한다고 읽으면 혹시 이해에 도움이 될지도 모르겠다.
  }
이런 상황의 유산은 오늘날까지도 남아있다.
법철학이 진지하게 논의되는 곳이라면 어디서나
취득시효의 사변적 기초에 관한 문제는 항상 열띤 논쟁의 대상인 것이다.
여전히 프랑스와 독일에서는,
수년간 계속해서 점유하지 않은 자가
오랫동안 방치한 벌로
소유권을 박탈당할 수 있느냐,
또는 소송의 종료\latin{finis litium}를 바라는 법의 개입만으로
소유권을 상실할 수 있느냐가
큰 관심의 대상이 되고 있다.
그러나 초기 로마 사회의 사람들은 이러한 망설임으로 구애되지 않았다.
그들의 고대 관행은
일정 조건 하에 1년이나 2년 동안 점유를 상실한 자로부터
바로 소유권을 빼앗았다.
초기 형태의 사용취득 규칙이 정확히 어떤 취지에서 만들어진 것인지는
알기 어렵다.
그러나 전거들에 나타난 사용취득의 조건들을 살펴보건대,
그것은 지나치게 복잡한 양도방식의 해악에 대한 사뭇 유용한 안전장치였음이
드러난다.
\wi{사용취득}이 인정되려면
\wi{적대적 점유}가 선의\hanjalatin{善意}{good faith}로,
즉 점유자가 그 물건을 합법적으로 취득한다고 믿으면서
시작되어야 한다.
또한 그 물건이
당해 사안의 권리양도에 필요한 완전한 양도방식은 아니더라도
적어도 법이 인정하는
어떤 양도방식을 통하여 그에게 이전되었어야 한다.
따라서 \wi{악취행위}가 요구되는 사안에서
아무리 절차가 날림으로 수행되었더라도
적어도 \wi{인도}\hanja{引渡}가 행해졌다면,
길어야 2년이면 사용취득에 의해 권리의 하자가 치유되는 것이다.
로마인들의 관행 중에 사용취득의 관행보다
그들의 법적 천재성을 강하게 입증하는 것은 없다고 나는 생각한다.
로마인들을 괴롭힌 문제는 영국 법률가들을 괴롭혔고
지금도 괴롭히고 있는 문제와 거의 동일한 것이었다.
그들이 재건축할 용기도 힘도 아직 갖고 있지 않은
영국법의 복잡성 탓에,
실질적 권리와 법기술적 권리가,
\wi{형평법}적 소유권과 보통법적 소유권이,
계속 분리된 상태로 남아있다.
그러나 로마 법학자들이 솜씨있게 요리했던 사용취득은
소유권의 흠결이 계속해서 치유되어가는,
소유권들 간에 일시적 분리가 생기더라도
최소한의 지체 후에는 다시 재빨리 결합되어가는,
일종의 자동기계 같은 것이었다.
사용취득은 \wi{유스티니아누스}의 개혁 전까지
그 장점을 잃지 않았다.
그러나 시민법과 형평법이 완전히 통합되자마자,
악취행위가 로마법의 양도방식이기를 그치자마자,
고대적 장치는 더 이상 필요하지 않게 되었다.
기나긴 생애를 마감한
\wi{사용취득}은
이제
\wi{취득시효}가 되었거니와,
이것이 마침내 거의 모든 근대 법체계에 수용된 것이다.

\para{법정양여}
방금 살펴본 것과 동일한 목적을 갖는 또 다른 수단 하나를
간단히 언급하고자 한다.
그것은 영국법사에서는 초기부터 등장한 것이 아니었으나,
로마법에서는 먼 옛날부터 있었던 오래된 제도이다.%
\footnote{%
  로마법에서는 법정양여에 관한 언급이 12표법에도 나온다(6.6.b).
  영국법의 공모회수소송(common recovery)과 종국화해(final concord)는
  15세기 후반부터 비로소 널리 이용되었다.
  공모회수소송과 종국화해에 대해서는
  본서 제6장 `고대 유언의 비서면성' 부분(\pageref{finerecovery}쪽)과
  관련 각주 참조.
  }
영국법을 유추하여 이 제도를 조명할 능력이 부족한
독일의 몇몇 로마법 학자들은 심지어 악취행위보다도 오래되었다고
생각할 정도로 오래된 제도이다.
내가 말하려는 것은 바로
양도하려는 물건을
\wi{법정양여}\hanjalatin{法廷讓與}{cessio in jure}하는 것으로,
일종의 공모회수소송\latin{collusive recovery}이다.
원고는 통상적인 소송 방식을 통해 객체에 대한 권리를 주장하고,
피고는 불출석한다. 그러면 당연히 판결을 통해 그 물건의 권리는
원고에게 주어진다.
영국 법률가들에게는
이 수단이 그들 선조들에게 어떤 의미를 가졌는지 굳이 설명할 필요가 없을 것이다.
봉건 토지법의 가혹한 질곡에서 벗어나기 위해
영국인들은
저 유명한
\wi{종국화해}\latin{fine}와 \wi{공모회수소송}\latin{recovery}을 만들어냈던
것이다.
이들 로마법의 장치와 영국법의 장치는 공통점이 많고
사뭇 유익한 사례를 서로에게 제공한다.
그러나 그것들 사이에는 차이점이 있으니,
영국법의 장치는
이미 취득한 권리에 붙어있는 골칫거리를 제거하는 데 목적이 있었던
반면,\footnote{%
  간단한 예를 들자면,
  공모회수소송은 한정승계부동산권(fee tail)에서 직계비속의 승계권을
  끊어내는 데 이용되었고,
  종국화해는 한정승계부동산권에 그래도 붙어있는
  잔여권(remainder)과 복귀권(reversion,
  원래 주인에게 부동산권이 되돌아가는 것)을 끊어내는 데 이용되었다.
  이로써 한정승계부동산권은 자유롭게 양도할 수 있는
  단순부동산권(fee simple)으로 전환될 수 있었다. }
로마법의 장치는
자칫 잘못 수행하기 쉬운 양도방식을 대신하여
일종의 탄핵불가능한 양도방식을 제공함으로써 골칫거리를 미리 막는 데
목적이 있었다.
사실 이 장치는 법원이 안정적으로 작동하는 단계에 이르면
언제든 등장할 수 있지만,
그래도 역시 원시적 관념의 제국에 속하는 것이다.
진보된 상태의 법에서는 법원이
공모소송을 소권\hanja{訴權}의 남용이라고 간주한다.
그러나 방식만 정확히 갖추어진다면
그 이상은 따지지 않는 그런 시절이 반드시 있었던 것이다.

\para{소유권과 점유권}
법원과 소송절차가 물권법에 끼친 영향은 광범위한 것이었지만,
이 주제는 너무나 방대해서 본서에서 다 다룰 수 없을 뿐만 아니라
본서의 기획보다 훨씬 더 많이 법사\hanja{法史}를 거슬러내려가야 한다.
하지만 소유권과 \wi{점유}권이라는 중요한 구분이 이 영향에서
유래하는 것임은 언급해둘 필요가 있겠다.
사실, 구분 그 자체---이는
\paren{어느 저명한 영국의 로마법 학자의 말에 의하면}
물건에 대하여 결정할 법적 권리와
그렇게 할 사실적\latin{physical} 권리 간의 구분과
같은 것이라 한다---가 아니라,
이 구분이 법철학에서 가지는 특별한 중요성에 대해 말하려는 것이다.
교양있는 사람이라면 법문헌을 별로 읽어보지 않았더라도,
점유라는 주제에 관한
로마 법학자들의 언어가 오랫동안 큰 혼란을 만들어냈다는 것,
그리고 \wi{사비니}의 천재성이 이 수수께끼를 해결한 데서
주로 입증되었다는 것을
들어본 적이 있을 것이다.\footnote{%
  사비니의 1803년 저서 <<점유권론>>(Das Recht des Besitzes)을 말한다. }
실로 로마 법률가들은
쉽게 설명할 수 없는 다양한 의미로
점유라는 말을
사용했던 것으로 보인다.
\wi{점유}라는 말의 어원을 따져보면, 이 말은 본디
물리적 접촉 또는
원한다면 언제든 회복할 수 있는 물리적 접촉을 뜻했음이 거의 확실하다.
그러나 실제로 사용될 때는,
수식어가 따로 붙지 않는 한,
그것은 단순한 물리적 소지\hanjalatin{所持}{detention}가 아니라,
물리적 소지에 덧붙여
물건을 자기 것으로\latin{as one's own} 보유하려는 의사가 결합된 것을
의미한다.\footnote{%
  사실적 소지로 점유를 파악하는 우리 민법의 점유 개념과 다름에 유의할 것.
  }
니부르의 견해를 수용한 \wi{사비니}는
이러한 특이한 개념이 오로지 역사의 산물이라고 보았다.
사비니에 따르면,
명목상의 차임\hanja{借賃}만 내면서
국유지의 상당 부분을
보유하게 된
로마의 귀족시민들은
옛 로마법에 의하면 단순한 점유자에
불과했지만, 그러나
그들의 점유는
모든 도전자들을 상대로 자기 땅을 지키고자 하는 의사를 가진 점유였다.
실로 그들의 주장은 최근 영국에서 교회토지의 임차인들이 내세운 주장과
거의 동일한 것이었다.
이론적으로 보면 그들은 국가의
\index{임의부동산임차권}%
임의\hanja{任意}부동산임차인\latin{tenants-at-will}\footnote{%
  기한의 보장이 없이
  당사자 일방의 의사에 의해 언제든지 종료될 수 있는 부동산임대차의 임차인.
}에 불과했으나,
그들은 시간의 경과와 평온한 향유로
자신들의 보유가 일종의 소유권으로 성숙했으며
토지의 재분배를 위해 자신들을 퇴거시키는 것은 부당하다고 주장했다.
이러한 주장은 귀족들이 토지를 보유한다는 점과 결합하여
``점유'' 개념에 항구적인 영향을 끼쳤다.
이 경우
토지보유자들이
퇴거당하거나 방해의 위협을 받을 때
이용할 수 있는 법적 구제수단은
점유보호특시\hanja{特示}명령\latin{possessory interdicts}이 전부였다.
로마법의 약식절차였던 점유보호\wi{특시명령}은
\wi{법무관}이 그들을 보호하기 위해 특별히 고안해낸 수단이었거나,
또는 다른 이론에 의하면
법적 권리를 다투는 동안 임시로 점유를 유지하도록 옛날부터 사용되어온
수단이었다.
그리하여 물건을 \hemph{자기 것으로}\latin{as his own}
\wi{점유}하는 모든 이들이 이 특시명령을 신청할 자격이 있다고
여겨지게 되었고,
사뭇 특별한 변론 절차에 기초하여
이 특시명령 절차는 점유를 둘러싼 분쟁을 재판하는 데 적합한 형태를
갖추어나갔다.
그러자, 존 \wi{오스틴}\latin{John Austin} 씨가 지적한 대로,
영국법에서도 똑같이 반복된 어떤 흐름이 시작되었다.
소유권자들\latin{domini}이
지루하고 복잡한 대물소송\hanjalatin{對物訴訟}{real action} 대신에
간편하고 신속한 \wi{특시명령} 절차를 선호하기 시작한 것이다.
점유권적 구제수단을 이용하려는 목적에서
그들은 소유권에 내포되어 있는 점유를 원용하기 시작한 것이다.
협의의 점유권자가 아닌 소유권자인 사람들에게
점유권적 구제수단에 의한 권리 주장을 허용한 것은,
처음에는 은혜로운 일이었을지 몰라도,
결국에는 영국법과 로마법 모두에 심각한 퇴행을 낳았다.
로마법은
이로 인해 생긴
점유 개념을 둘러싼 온갖 복잡미묘한 주장들로
불신을 초래했으며,
영국법은
\wi{물적재산}\latin{real property}의 회수를 위한 소송들이
절망적 혼란상태에 빠져들자
결국 영웅적인 결단으로
저 혼란한 덩어리 전체를 잘라내버렸던 것이다.
누구도 거의 30년 전에 단행된
영국 물적소송\hanjalatin{物的訴訟}{real action}의 사실상의 폐지\footnote{%
  1833년 물적재산출소기한법(Real Property Limitation Act).
  물적소송의 출소기한(이 법률에서는 20년, 그후 12년으로 단축된다)만
  정한 것이 아니라, 본문에서 언급하고 있듯이
  각종 부동산권 소송을 부동산점유회복소송(ejectment)으로 거의 단일화했다.
}가 공공에 이익이었다는 점을 의심할 수 없을 것이다.
그러나 여전히
법학의 조화를 중시하는 사람들은
물적소송을
정비하고 개선하고 단순화하는 대신에
물적소송 전부를 \wi{부동산점유회복소송}\latin{ejectment}\footnote{%
  부동산에 관한 것이지만 법적으로는 인적소송(personal action)에
  해당한다. 인적재산인 정기부동산임차권(term of years)의 임차인이
  퇴거당했을 때 성립하는 소송이었기 때문이다.
  지금은 널리 부동산 일반의 점유 회복을 위한 소송이 되었다.
}에 갖다 바친 것은
부동산 회수 소송의 체계 전체를 \wi{법적의제}에 기초하게 만든 것이라며
한탄할 것이다.

\para{형평법상의 소유권}
또한 법원은
재판권 간의 최초의 구별일 수밖에 없는
보통법과 \wi{형평법} 간의 구별을 통해
소유권 개념을 형성하고 수정하는 데
크게 기여했다.
영국의 형평법상의\latin{equitable} 소유권은 단지
형평법법원\latin{Court of Chancery}의 재판권에 의해 인정된 소유권일 뿐이다.
로마에서는
\wi{법무관}의 \wi{고시}\latin{edict}에 의해,
일정한 경우 일정한 소송이나 신청을 허용하겠다는 약속의 형태로
새로운 원리들이 도입되었다.
따라서 `\wi{법무관법상의 소유권}'\latin{property \textit{in bonis}},
다시 말해 로마법의 \wi{형평법상의 소유권}은
고시에서 기원한 구제수단들에 의해서만 보호되었다.
형평법상의 권리가 보통법상의 소유권에 의해
무효화되지 않게 된 방법에는 양 법체계가 다소 차이가 있다.
영국에서 그것의 독립성은 형평법법원의
\wi{금지명령}\latin{injunction}에 의해 보장되었다.
하지만 로마법에서는
시민법과 형평법이
아직 완전히 통합된 것은 아니나
동일한 법원에 의해 다루어졌기에,
금지명령 같은 것이 필요치 않았고
정무관은 보다 간단한 방법을 사용할 수 있었다.
그 방법은 형평법상 타인에게 속하는 물건을 회수하려는
\wi{시민법}상의 소유자에게 정무관이 소송이나 신청을 허가해주지 않는 것이었다.
그러나 양 법체계의 실제 작동은 거의 동일했다.
양자 모두 절차의 구별을 통해,
나중에 법 전체에 의해 인정될 때까지,
일종의 잠정적인 존재로 새로운 소유권 형태를
보호할 수 있었다.
그리하여
로마 법무관은
단순한 \wi{인도}로 \wi{악취물}을 취득한 자에게
\wi{사용취득} 기간이 완성되기를 기다리지 않고
즉각 소유권을 부여했다.\footnote{%
  이른바 푸블리키우스 소권(actio Publiciana).
  사용취득을 의제하여 소권을 부여했다. }
마찬가지로 때로 그는
애초 ``보관인''\latin{bailee} 또는
수치인\hanjalatin{受置人}{depositary}에 불과했던
질권자\hanjalatin{質權者}{mortgagee}에게,\footnote{%
  이른바 세르비우스 소권(actio Serviana).
  농지 임대인이 차임의 담보물에 대해 갖는 대물소권이다.
  \latin{Inst.\,4.6.7.}}
그리고 정액지료\hanja{定額地料}를 정기적으로 내는 영구적 토지임차인인
\wi{영차권}자\hanjalatin{永借權者}{emphyteuta}에게
일종의 소유권을 인정했던 것이다.
유사한 진보 과정을 거쳐,
영국의 형평법법원도
\wi{양도저당}권\hanja{讓渡抵當權}설정자\latin{mortgagor}에게,\footnote{%
  우리의 저당권 제도와 달리
  1925년 재산권법(Law of Property Act) 이전의 양도저당(mortgage)에서는
  채권자인 저당권자에게 담보물의 보통법상 부동산권이 귀속했다.
  보통법상으로는 사소한 채무불이행으로도
  채무자는 그 즉시 담보물을 완전히 상실하지만,
  형평법법원은
  `형평법상 상환권'(equity of redemption) 법리를 발달시켜
  설령 기한 내에 채무를 변제하지 못하더라도
  상당한 기간 내에 이자와 비용을 붙여 갚으면
  담보물을 되찾아올 수 있도록 했다.
  이 상환권은 양도저당권자가 반역을 저질러 재산을 몰수당하더라도
  보호될 정도로 확고한 물권적 권리이다.
  }
\wi{신탁}\hanja{信託}의 수익자\latin{cestui que trust}에게,
특정 유형의 재산설정의 이익을 갖는 기혼여성에게,\footnote{%
  혼인한 여자의 재산은 보통법상으로 남편의 것이었으나,
  형평법의 신탁(trust) 법리에 의해
  차츰 아내에게 일정한 재산권을 부여하는 설정이 가능하게 되었다.
  그런데 강박 등으로 아내가 남편에게 자신의 재산을 양여하거나
  남편의 채무를 자신의 재산으로 변제하는 등의 처분이 문제되자,
  혼인 중에 그러한 처분을 하지 못하도록 하는
  조항(`기한전 처분 금지'restraint on anticipation)이
  부부재산계약에 포함되는 일이 많아졌고, 다시
  형평법법원이 나서서 그러한 조항의 효력을 인정한 것이다.
  }
완전한 보통법적 소유권을 취득하지 못한 매수인에게,\footnote{%
  매매계약이 체결되면
  아직 양도(날인증서의 작성 및 교부, 오늘날은 등기)를 경료하지 않았더라도
  매수인이 형평법상의 소유권자가 된다. }
특수한 소유권을 부여했다.
이 모두는 분명 새로운 형태의 소유권을 인정하고 보호한 사례들이다.
그러나
간접적으로는
영국의 물권법도 로마의 물권법도
형평법에 의해 수천 가지 방법으로 영향받았다.
법학자들이 구사하는 강력한 수단이
법학의 어떤 분야에든
밀고 들어가면,
법학자들은
반드시
물권법을 만나고, 건드리고, 어느 정도 변경하게 되어있다.
지난 몇 페이지에 걸쳐
내가
이런저런 고대법적 구별과 수단들이
소유권의 역사에 큰 영향력을 행사했다고
말했다면,
그것은
법학자들이
시대정신에
불어넣은 개선의 힌트와 제안들이
형평법 체계의 담당자들에 의해 호흡되어
그 영향력의 대부분이 생겨났다는
의미였다고 이해해주시길 바란다.

\para{로마법과 봉건법, 만족들의 법전}
그러나 소유권에 대한 형평법의 영향력을 기술하는 것은
현대에 이르기까지의 역사를 기술하는 것이 될 터이다.
앞서 내가 이런 말을 잠깐 언급한 이유는,
오늘날의 몇몇 저명한 학자들에 따르면,
로마제국의 법과 중세의 법이 소유권의 개념에서 차이를 보이는 것의 단서를
로마인들이 \wi{형평법상의 소유권}과 \wi{시민법}상의 소유권을 분리한 데서
찾을 수 있다고 하기 때문이다.
\wi{봉건제}적 소유권 개념의 주요 특징은
이중\hanja{二重} 소유권\latin{double proprietorship},
즉 봉토 주군의 상급소유권과
토지보유자의 하급소유권 또는 보유권의
병존을 인정하는 것이다.
이제 이러한 소유권의 이중성이
\hemph{시민법}상의\latin{quiritarian} 또는 보통법상의 소유권과
\paren{나중에 생겨난 용어를 쓰면}
\index{법무관법상의 소유권}%
\hemph{법무관법}상의\latin{bonitarian} 또는 \wi{형평법}상의 소유권 간의
로마인들의
구별을
일반화한 형태에 대단히 유사하다는 것이다.
\wi{가이우스}는
\hemph{소유권}\latin{dominion}이 두 부분으로 분리되는 것을
로마법에 특유한 현상이라고 보았고,
이를 다른 민족들의 관습인
단일한 또는 완전한\latin{allodial} 소유권과 뚜렷이
대비시키고 있다.\footnote{%
  \latin{Gai.\,2.40.} }
물론
\wi{유스티니아누스}가 소유권을
하나로 재통합했지만,\footnote{%
  \latin{Cod.\,7.25.1.}}
만족\hanja{蠻族}들이 수 세기 동안 접촉했던 것은
서로마제국의 부분적으로 개량된 법체계였지
유스티니아누스의 법이 아니었다.
로마제국의 경계선 근처에 자리잡으면서,
그들은
나중에 중요한 결실을 낳은
저 구분을 알게 되었을 가능성이 농후하다.
이 이론에 유리하게도,
만족들의 관습을 모은 법전들에
로마법적 요소가 얼마나 들어있는지가
제대로 조사되지 않았음을
어쨌든 인정하지 않을 수 없는 것이다.
봉건제도를 설명하는 잘못된 또는 불충분한 이론들은
봉건제도라는 직조물에 들어있는
이 특별한 요소를 무시하는 경향을 보인다는 공통점이 있다.
과거에 연구자들은,
그 추종자들이 주로 영국에 많았거니와,
봉건제가 형성되던 시기의 혼란스런 상황에만 중점을 두었다.
이러한 오류에 후에 새로운 오류가 더해졌으니,
독일 학자들은
민족적 자부심에서
그들 선조들이 로마 세계에 등장하기 전부터
가지고 있던 사회 조직의 완전성을 강조했다.
한 두 명의 영국 학자는
봉건제의 토대에 관해
올바른 방향으로 연구를
시도했으나
만족할 만한 결과를 얻는 데 실패하고 말았거니와,
\wi{유스티니아누스} 법전에서 유사한 것을 찾는 데만 너무 몰두했거나,
혹은 현존하는 몇몇 만족\hanja{蠻族}들의 법전에 추가된
로마법 집성\hanja{集成}들에 관심을 한정했기 때문이었다.\footnote{%
  가령 서고트 왕국은 게르만인을 위한 에우릭법전(Codex Euricianus)과
  로마인들을 위한 알라릭약전(Breviarium Alaricianum, 506년에 공포)을 편찬했다.
  후에 서고트법전(Lex Visigothorum)으로 통합된다. }
그러나,
로마법이 만족들의 사회에 어떤 영향을 끼쳤다면,
그것은
대부분
유스티니아누스의 입법 이전의 일이었을 테고,
또한 저 집성들을 준비하기 이전의 일이었을 것이다.
생각건대
만족\hanja{蠻族}들의 관습이라는 골격에 살과 근육을 붙인 것은
유스티니아누스의 개혁되고 정화된 법이 아니라,
동로마제국의 로마법대전이 결코 완전히 대체하지 못한,
서로마제국에서 지배적이었던
정돈되지 못한 법이었다.
게르만 부족들이
정복자로서
로마 영토의 일부라도 본격적으로 차지하기 전에,
따라서 게르만 왕들이 로마인 백성들을 위해
로마법 약전\hanjalatin{略典}{breviary}들을 편찬하도록 명하기 한참 전에,
이미 변화가 일어났던 것으로 보아야 한다.
원시적인 법과 발달된 법 사이의 차이를 잘 아는 사람이라면
이 가설을 지지하지 않을 수 없을 것이다.
\index{만족들의 법전}%
만족\hanja{蠻族}들의 법전\latin{leges barbarorum}은
비록 미개한 모습으로 우리에게 전해지지만,
오직 만족들의 법에서만 기원한다는 이론을 충족시킬 만큼
그렇게 미개하지는 않다.
또한
성문\hanja{成文}의 기록으로 우리에게 남겨진 규칙의 전부가
정복 부족의 구성원들 사이에서 관행되던 규칙이라고
믿을 만한 근거도 없다.
비속\hanja{卑俗}로마법의 상당수가
이미 만족들의 법체계에 들어있었다고 확신할 수 있다면,
우리는 커다란 난제 하나를 제거할 수 있다.
정복자들의 게르만법과
그들의 백성의 로마법이
세련된 법과 야만인의 관습 간에 통상 존재하는 정도를 넘어
서로 친화성을 갖지 않았다면
이들은
결합할 수 없었을 것이기 때문이다.
\wi{만족들의 법전}은,
비록 원시적으로 보일지라도,
진정한 원시적 관행과 반쯤 이해된 로마법의 복합체일 뿐이며,
그것이
서로마제국 아래서의 비교적 세련된 형태로부터 이미 다소 후퇴한
로마법과 융합할 수 있었던 것은
이러한 외부적 요소 덕분이었을
가능성이 대단히 크다.

\para{영차권, 콜로누스, 봉건적 봉사}
그러나,
이 모든 것을 인정하더라도,
봉건적 소유 형태가 로마법의 이중 소유권에서 직접
유래했다고 보기 어렵게 만드는 몇 가지 고려사항이 있다.
시민법상의 소유권과 \wi{형평법상의 소유권}의 구분은
만족\hanja{蠻族}들이 이해하기 힘들 정도로 복잡미묘한 것으로 보인다.
더욱이 그것은 법원이 정상적으로 작동하는 곳에서가 아니면 이해되기 어려운 것이다.
그러나 이 이론을 반박할 수 있는 무엇보다 강력한 근거는
로마법에 존재하는 어떤 재산권 형태 하나---분명 \wi{형평법}의 산물이다---가
하나의 관념 체계로부터 다른 것으로의 전이\hanja{轉移}를 훨씬 더
간단하게 설명할 수 있게 한다는 것이다.
\wi{영차권}\hanjalatin{永借權}{emphyteusis}이 그것이니,
봉건적 소유권을 탄생시키는 데
그것이 기여한 지분이 얼마인지 그다지 정확한 지식 없이
중세의 봉토권이 바로 여기서 유래했다고
종종
주장되어왔다.
실로
영차권은,
어쩌면 아직 저 그리스식 명칭으로 불리기도 전에,
나중에 봉건제를 만들어낸 관념의 흐름에 중대한 획을 그었다.
로마 역사에서
가부장이 자기 가\hanja{家}의 아들들과 노예들을 데리고
농사를 지을 수 없을 만큼의 넓은 소유지에 대한 언급은
로마 귀족들의 토지에서 최초로 만나게 된다.
이들 대\hanja{大}소유주들은
자유차지인\hanja{借地人}들을 이용해 농사짓는 체제를 알지 못했던 듯하다.
그들의 라티푼디움은 어디서나
노예집단을 부려 경작되었고,
이들을 감독하는 관리자도 노예이거나 \wi{해방노예}\latin{freedman}였다.
유일하게 시도된 조직형태로,
하급 노예들을 작은 집단으로 나누고
이들 집단을 상급의 보다 믿을 만한 노예에게
\wi{특유재산}\hanjalatin{特有財産}{peculium}으로 맡기는
형태가 있었던 것으로 보이며,
이로써 특유재산을 가진 노예는 노동의 효율성에 대해
일종의 이해관계를 갖게 된다.
하지만 이 체제는 토지소유자의 한 부류, 즉
지방정부\latin{municipality}에게는
특히 불리했다.
이탈리아의 관리들은 교체가 사뭇 빈번하여,
로마의 행정 중에서도 우리를 자주 놀라게 하는 점이다.
그러므로 이탈리아의 지방정부가 소유한 넓은 토지를
감독하는 일은 대단히 불완전했을 것이 틀림없다.
그리하여 지방정부들은
\index{전세징수지}%
`전세\hanja{田稅}징수지'\latin{agri vectigales}를 세놓는,
다시 말해 특정한 조건 하에
정액지료를 받으며
자유차지인에게
영구적으로
토지를
임대하는
관행을 형성하기 시작했다고 한다.
이 체제는 그후 개인 소유자들도 대거 모방했다.
토지보유자들은
원래는 계약에 따라
소유주와의 관계가
정해졌으나, 후에
\wi{법무관}에 의해 제한적 소유권을 갖는 것으로 인정받았고,
시간이 흐르면서 이것이 `\wi{영차권}'이라는 명칭으로 불리게 된 것이다.
이때부터
\wi{토지보유권}의 역사는 두 갈래로 갈라진다.
로마 제국의 남아있는 역사기록이 사뭇 불완전한 긴 기간 동안,
로마의 대토지 소유주의 노예집단들은
\wi{콜로누스}\latin{coloni}로 전환되어갔다.
콜로누스의 기원과 지위에 관한 문제는
역사학 전체에서 가장 모호한 문제의 하나이다.
부분적으로는 노예에서 신분이 상승하고,
부분적으로는 자유농에서 신분이 하강하여
형성되었을 것이라고 짐작해볼 수 있다.
그들은 로마의 부유한 계층에게
경작자들이 토지의 산물에 대해 이해관계를 가질 때
부동산의 생산성이 증가한다는 사실을
깨닫게 해주었다.
그들이 토지에 예속되어 있었다는 점,
완전한 노예의 속성 중 다수가 그들에게는 없다는 점,
매년 수확물의 일정 부분을 지주에게 바침으로써 그들의 봉사의무를
다했다는 점 등을
우리는 알고 있다.
또한 우리는 고대로부터 근대에 이르는 온갖 변화의 소용돌이 속에서도
그들이 살아남았다는 것을 안다.
봉건구조의 하층에 편입되어서도,
그들은 로마의 소유주\latin{dominus}에게 지불하던 소작료와 정확히
똑같은 것을 지주들에게 내면서 여러 나라에서 그 존재를 이어갔다.
그들 중 일부 계층인
분익\hanja{分益}\wi{콜로누스}\latin{coloni medietarii}는
수확물의 절반을 소유주에게 지불했거니와,
이는
오늘날 남유럽의 대부분의 토지를 경작하고 있는
\wi{분익농}\hanjalatin{分益農}{metayer} \wi{토지보유권}으로 이어지고 있다.
다른 한편,
로마법대전에 언급된 것으로부터 이해하건대,
영차권은
소유권의 변종 중에서도
가장 선호되는 유익한 변종이었을 것이다.
자유농이 존재하는 곳이라면 어디서나,
그들의 부동산권을 규율하는 것은 바로 이 토지보유권이었을 것으로 추정할 수 있다.
전술했듯이, 법무관은 \wi{영차권}자\latin{emphyteuta}를
진정한 소유권자의 하나로 취급했다.
퇴거당하면, 그는
소유권의 독특한 표장\hanja{標章}인
대물소송\hanjalatin{對物訴訟}{real action}을 통해 토지를 회복할 수 있었다.
`정조'\hanjalatin{定租}{canon},
즉 정액지료\latin{quit-rent}만
제때 납부하면 임대인의 방해로부터도 보호되었다.
그렇지만 임대인의 소유권이
소멸했다거나 휴면 중이라고 생각해서는 안 된다.
그것은 여전히 살아서,
지료 지불 해태의 경우 점유회복권,
매매의 경우 선매권\hanjalatin{先買權}{pre-emption},
경작방식에 대한 일정한 통제권 등을 가지고 있었다.
그리하여 우리는 영차권에서
봉건 소유권의 특징이었던
이중 소유권의 현저한 예를 보게 된 것이다.
더욱이 이는
시민법상의 권리와 형평법상의 권리의 병렬보다
훨씬 간단하고 훨씬 모방하기 쉬운 것이다.
하지만 로마 보유권의 역사는 여기서 끝나지 않는다.
라인강과 다뉴브강 줄기를 따라
배치되어 있으면서
오랫동안
인접 만족\hanja{蠻族}들을 상대로
국경선을 지켜온
큰 요새들 사이로
`\wi{국경지}'\hanjalatin{國境地}{agri limitrophi}라고 불리는
기다란 땅뙈기들이
연속적으로 펼쳐져 있어서
로마 군대의 퇴역 군인들이 이를 \wi{영차권}에 기하여
점유하고 있었다는
명백한 증거가 있다.
역시 이중 소유권이 있었다.
지주는 로마 국가였으나,
국경 상황이 요구할 때면 군역\hanja{軍役}에 소환될 채비가 갖추어져 있는 한
아무런 방해도 받지 않고
군인들이
경작하고 있었던 것이다.
사실,
오스트리아^^b7투르크 국경지대의 군사식민지 체제와 사뭇 유사한
이런 종류의 주둔지 군역은
통상적인 영차권의 대가였던 정액지료를 대체하는 것이었다.
의심할 여지 없이,
이것이야말로
봉건제의 기초를 놓은
만족\hanja{蠻族}의 군주들이 모방한 선례였을 것이다.
그들은 수백년 동안 그것을 지켜보았을 것이고,
또한 국경을 수비하는 퇴역 군인들 중에는
게르만어를 구사할 줄 아는 만족 출신 군인도 다수 있었다는 점을
기억해야 한다.
이렇게 쉽게 따라할 수 있는 모델이 근처에 있었다는 것은,
프랑크와 롬바르드 군주들이
국유지를 나누어주고 종자\hanja{從者}들의 군사적 봉사를 확보한다는
아이디어를 어디서 얻었겠는가를
설명해줄 수 있다.
뿐만 아니라,
\wi{은대지}\hanjalatin{恩貸地}{benefice}가
곧장 세습화되어간
경향을
설명할 수도 있을 것이니,
비록 원래의 계약조건에 따라 달라질 수 있다 해도
일반적으로 \wi{영차권}은 수혜자의 상속인에게 상속되는 것이었기 때문이다.
사실,
은대지의 보유자는,
그리고 나중에 은대지에서 전환된 봉토의 영주는,
군사식민지 주민들은 제공하지 않았던 것 같은,
그리고 영차권자도 제공하지 않았음이 분명한,
제반 봉사를
제공할 의무가 있었던 것으로 보인다.
봉건 상급자에게 존경과 감사를 바칠 의무,
그의 딸의 혼인지참금과 그의 아들의 군장비 조달에 조력할 의무,
미성년자인 경우 그의 \wi{후견}을 받아야할 의무,
기타 토지보유에 수반되는 여러 부담들은
로마법상의 보호자\latin{patron}와 \wi{해방노예}\latin{freedman}의 관계,
즉 전\hanja{前}주인과 전\hanja{前}노예 간의 관계를
그대로 빌려온 것임에 틀림없다.
그런데 초기의 은대지 수혜자는
주군의 인간적 동료였다고 하나,
이 지위는,
보기에는 화려해도,
처음에는 일말의 예속상태로의 강등을 수반하는 것이었음을
부인할 수 없다.
주군의 궁정의 가신\hanja{家臣}이 된 사람들은
\wi{자유소유지} 소유자의 자랑스런 특권이었던
절대적인 신분상의 자유를 일부 포기했던 것이다.


\chapter{계약법의 초기 역사}

우리가 속한 시대에 관한 명제로,
오늘날의 사회가
지난 시대의 사회와 차이나는 주요 특징은
계약법이 차지하는 영역이 대폭 증가했다는 데 있다는
주장만큼
일견 쉽게 수긍할 수 있을 법한 것도 없을 것이다.
이 명제가 근거하고 있는 현상들 중 일부는
대단히 빈번하게 선택되어 관심과 논평과 칭송의 대상이 되고 있다.
%우리들 중에서
옛 법이 사람의 출생에 따라 그의 사회적 지위를
불가역적으로 고정시켰던 수많은 사안들에서
근대법은 합의에 의해 그 스스로 자신의 지위를 만들어갈 수 있도록
허용하고 있음을
알아차리지 못할 정도로
무감한 사람은
별로 없을 것이다.
실로 이 원칙에 대한 예외로 남아있는 소수의 몇몇 것들은
열정적 분노에 찬 비난을 지속적으로 받고 있다.
가령 흑인\wi{노예제}를 둘러싸고 여전히 진행 중인 열띤 논쟁에서
실로 다투어지고 있는 논점은
노예제가 지난 시대의 제도가 아니냐는 것,
그리고
근대적 도덕성에 부합하는
고용주와 노동자 간의 관계는
오직 계약에 의해 정해지는 관계뿐이지 않겠느냐는 것이다.
과거와 현재 간의 이러한 차이의 인정은
현대의 가장 유명한 사변적 논의의 핵심으로 우리를 끌고 들어간다.
확실히,
명령법\latin{imperative law}이
한때 장악하고 있던 영역의 많은 부분을
포기하지 않았다면,
그리고
최근까지 허용되지 않던 자유를 누리며
사람들이 스스로의 행위규칙을 정하도록 허용하지 않았다면,
도덕에 관한 연구 분야 중에
우리 시대에 비약적인 진보를 보인 유일한 분야인
정치경제학은
생활 현실에 부응하지 못하고 실패할 것이다.
정치경제학의 훈련을 받은 사람들의 대다수가
실로 가지고 있는 선입견은
그들 학문이 의지하고 있는 일반적 진리가
보편적인 것이 될 권리가 있다고 보는 것이다.
그리하여 그들이 그것을 학문으로 적용할 때면,
그들의 노력은 대개 계약법의 영역을 확장하고
명령법의 영역은 축소하는 방향을 지향하거니와,
단지 계약의 이행을 강제하는 데 필요한 한에서만
명령법을 용인하는 것이다.
이러한 관념의 영향을 받은 사상가들이 불러일으킨 충격은
바야흐로 서구 세계에서 사뭇 강력하게 느껴지기 시작하고 있다.
입법은
발견과 발명과 축적된 부\hanja{富}의 사용에 관한 사람들의 행동을
따라잡을 능력이 없음을
거의 자백했다.
가장 덜 진보된 공동체의 법조차
점점 단지 껍데기에 불과한 것이 되어가고 있거니와,
그 아래에는
지속적으로 변화하는 계약적 규칙들의 연합이 존재하여,
여기에 법이 개입하는 경우는
약간의 근본원리들의 준수를 강제하거나
신의\hanja{信義} 위반을 벌하기 위해 소환되는 경우 외에는
거의 없는 실정이다.

\para{계약의 강제}
법현상을 고려해야 하는 것인 한
사회 탐구는 그 상황이 매우 낙후되어 있는지라,
사회의 진보에 관하여 널리 통용되는
통속적인 견해에서 저 진리가
발견되지 않더라도 놀라울 것이 없다.
이들 통속적인 견해는
우리의 신념보다는 우리의 편견에 더 잘 부응한다.
도덕의 진보를 인정하기를 꺼리는 강한 경향성은
계약의 기초가 되는 미덕을 의문시할 때
특히 더 강력해지는 듯하다.
우리들 중 다수는
신의와 성실이 옛날보다 오늘날에 더 널리 퍼져있음을,
또는 적어도 고대 세계의 충실성에 비견할 만한 풍속이 오늘날에도 존재함을,
인정하는 것에 대한
거의 본능적인 거부감을 가지고 있다.
때로 이러한 선입견은
예전에는 들려오지 않던
사기행각이 만연함을 보면서,
그리고 이들 범죄가 가져오는 커다란 혼란과 충격을 보면서
더욱 강화된다.
그러나 바로 이러한 사기행위의 범죄성으로부터 우리는,
그것을 범죄로 취급할 수 있기 위해서는
우선
그것이 위반하는 도덕적 의무가 더 크게 성장해야 한다는 것을
뚜렷이 알 수 있다.
다수가 믿고 따르는 신뢰가 있어야만
소수의 신뢰 위반도 생길 수 있는 법이므로,
아주 큰 부정직의 사례들이 발생한다면 이는
다수의 평균적 거래에서는 성실한 정직이 지배적이어서
예외적인 경우 범죄자들에게 기회가 주어졌다고
결론짓지 않을 수 없는 것이다.
계약법에서 형법으로 눈을 돌려
법에 반영된 도덕의 역사를 읽어야 한다면,
우리는 그것을 오독\hanja{誤讀}하지 않도록 주의해야 한다.
로마 고법\hanja{古法}에서 부정직한 행위로 취급된 형태는
\wi{절도}가 유일했다.
이 글을 쓰는 순간,
영국 형법에 추가된 최신 영역은
수탁자\hanjalatin{受託者}{trustee}의 사기행위를 처벌대상으로 삼으려는 것이다.
이러한 대비에서 얻을 수 있는 올바른 추론은
원시 로마인들이 우리보다 더 높은 도덕성을 지녔다는 것이 아니다.
오히려 그들 시대에서 우리 시대로 시간이 흐르면서
사뭇 미개한 도덕성으로부터 대단히 세련된 도덕성으로
도덕성 관념이 진보했다는 것을 알 수 있는 것이다.
소유권만을 신성한 것으로 여기던 관념에서
단지 일방적 신뢰의 수여만으로 발생하는 권리까지도
형법에 의해 보호되는 권리로 보는 관념으로 진보가 이루어진 것이다.

\para{사회계약}
이 점에 관하여 법학자들의 정연한 이론이라고 해서 대중들의 의견보다 더
진리에 가까운 것도 아니다.
로마 법률가들의 견해부터 말하자면,
그것은 도덕과 법의 진보에 관한 참된 역사와 일치하지 않았다.
계약 당사자들이 약속한 신의가 유일하게 중요한 요소인
계약의 한 유형을 그들은 \wi{만민법}상의\latin{juris gentium} 계약이라고
지칭했거니와,\footnote{%
  `낙성계약'(contractus consensu)을 말하고 있다.
  }
이 유형의 계약은 로마법에 나중에야 편입되어 들어간 것이 확실함에도
불구하고,
그들이 사용한 표현으로부터 어떤 확정적 의미를 추출해보면
그들은 그것을 로마법이 인정하는 다른 유형의 계약, 즉
법기술적 방식요건이 하나만 잘못되어도 오늘날의 착오나 사기만큼이나
계약의무의 성립에 치명적이었던 다른 유형의 계약들보다
더 오래된 것으로 보았음을 알 수 있다.
하지만 그들이 말하는 옛 것은 모호하고 희미한 것이었고
현재를 통해서만 이해될 수 있는 것이었다.
그리하여 ``만민법\latin{law of nations}상의 계약''을
자연상태의 사람들 사이의 계약으로 간주하게 된 것은
로마 법률가들의 언어가
그러한 사고양식에 진입하는 열쇠를 이미 상실해버린 시대의 언어로 된
이후의 일이었다.
\wi{루소}는 법률가들의 오류와 대중들의 오류를 모두 이어받았다.
관심을 끈 첫 작품이자
그를 한 분야의 선구자로 만든 의견이 사뭇 기탄없이 개진된 논문인
예술과 학문이 도덕에 끼친 영향을 논하는 논문에서,\footnote{%
  <<학문예술론>>(Discours sur les sciences et les arts)을 말한다. }
그는 고대 페르시아인들이 지녔던 정직함과 신의성실이야말로
문명의 등장과 더불어 점차 망각되어간 원시적 순수성의 특징이라고
누차 지적하고 있다.
그리고 나중에 그는
그의 모든 사변\hanja{思辨}의 토대를
원초적 사회계약의 교리에서 발견하게 된다.
<<사회계약론>>은 우리가 논하고 있는 오류를 지닌 것 가운데 가장 체계적인 형태이다.
비록 정치적 열정에 의해 그 중요성이 고양되었지만
이 이론은 법률가들의 사변으로부터
모든 수액\hanja{樹液}을 채취한 이론이다.
처음 이 이론에 감화된 영국의 저명인사들은
주로 정치적 유용성의 면에서 그것의 가치를 높이 평가한 것이 사실이지만,
뒤에서 설명하겠으나
만약 정치가들이 법적인 용어로 논쟁을 해오지 않았더라면
영국인들은 결코 이 이론에 다가서지 못했을 것이다.
그리하여 이 이론을 주창한 영국인 학자들도
그들로부터 그것을 물려받은 프랑스인들에게 강한 호소력을 가졌던
저 사변적 깊이를 모르지 않았다.
그들의 저서는 이 이론이 정치적 현상뿐만 아니라
사회적 현상까지 모두 설명할 수 있다고 그들이 인식했음을 보여준다.
사람들이 준수하는 실정규칙 가운데
계약\latin{contract}으로 만들어진 것이 점점 많아지고
명령법\latin{imperative law}으로 만들어진 것이 점점 줄어들고 있는 현상,
그들 시대에도 이미 현저하게 나타나고 있던 이 현상을
그들은 관찰을 통해 알고 있었다.
그러나 법학의 저 두 구성부분의 역사적 관계에 대해서는
그들은 무지했거나 주의를 게을리했다.
그리하여
그들은 모든 법은 계약에서 기원한다는 이론을 창안하였거니와, 이는
모든 법의 기원을 단일한 원천에 둠으로써 그들의 사변적 취향을
만족시키기 위한 것이었으며,
또한
명령법은 신에게서 기원한다는 교리\footnote{%
  필머(Robert Filmer)로 대표되는 왕권신수설을 말하는 듯하다.
}를 피하려는 견해에서 나온 것이었다.
한 단계 더 사고가 진보한다면,
그들은 기꺼이
그들의 이론을
어떤 기발한 가설이나 편리한 언어 공식\hanja{公式}에 불과했다고
치부했을 것이다.
그러나 당시는 법적 미신\hanja{迷信}이 지배하던 시대였다.
자연상태에 관한 논의는 그것이 역설적이 아니라고 여겨지는 한 계속되었고,
따라서
사회계약을 역사적 사실로 주장함으로써
법의 계약적 기원이라는 거짓 현실과 확신을
쉽게 심어줄 수 있었던 것으로 보인다.

\para{몽테스키외의 혈거인}
우리 세대는 이러한 잘못된 법이론을 떨쳐버렸다.
그것은 부분적으로는 저 이론이 속했던 지적 상태를 벗어났기 때문이고,
또 부분적으로는 그러한 주제를 이론화하는 일을 거의 그만두었기 때문이다.
오늘날 적극적으로 연구를 수행하는 학자들이 선호하는 작업은,
그리고 사회의 기원에 관한 우리 선조들의 사변에 대해 답할 수 있는 작업은,
사회의 존재를 있는 그대로, 사회의 운동을 운동하는 그대로 분석하는 것이다.
그러나 역사의 도움을 받지 못하면,
이런 분석은 단순한 호기심의 충족으로 전락하기 일쑤이거니와,
특히
연구자가 익숙해있는 사회상태와는 자못 다른 사회상태를 이해하는 데
장애물로 작용할 공산이 크다.
우리 시대의 도덕성을 가지고 다른 시대의 사람들을 판단하는 잘못은
현대사회라는 기계장치의 바퀴 하나, 볼트 하나까지
원초적 사회에 그 대응물이 있을 것이라고 가정하는 잘못에 견줄 만하다.
이러한 인상\hanja{印象}은
근대적 양식으로 쓰여진 역사학 저술들에서
사뭇 다양하게 가지를 치고 있으며
사뭇 미묘하게 모습을 숨기고 있다.
그러나
나는
\wi{몽테스키외}의
<<페르시아인의 편지>>에 삽입된
혈거인\hanjalatin{穴居人}{Troglodytes}의 우화\footnote{%
  \latinmarks
  Montesquieu, \textit{Persian Letters}, 11--14.
}에 대해
흔히 주어지는 찬사에서
법학 영역에서의 그러한 인상의 흔적을 발견한다.
혈거인들은 계약을 항상 위반하는 사람들이었으며, 그래서 완전히 멸망해버렸다.
만약 이 이야기에 저자가 의도한 도덕이 담겨있고,
그것이
금세기와 지난 세기를 위협해온 반사회적 이단\hanja{異端}을
폭로하기 위해 사용되었다면,
그것은 전혀 나무랄 데가 없는 것이다.
그러나
성숙한 문명이 보여주는 것과 같은 정도로
약속과 합의에 신성함을 부여하지 않는 한
어떤 사회도 결속을 유지할 수 없다는
주장이
저 이야기로부터
추론되어 나온다면,
그것은 법사\hanja{法史}의 어떤 건전한 이해와도 상반되는 치명적인 오류가
될 것이다.
사실,
혈거인들은 계약적 의무를 아주 조금 준수함으로써
번성할 수 있었고 강력한 국가를 건설할 수 있었던 것이다.
원시사회의 헌정\hanja{憲政}에 관하여
무엇보다 먼저 이해해야 할 것은
개인은 자신을 위해 권리나 의무를 거의 혹은 전혀 만들지 못한다는 점이다.
개인이 지켜야 할 규칙은 우선은 출생에 따르는 지위에서 나오고,
다음으로는 그가 속하는 가\hanja{家}의 수장이 그에게 부과하는
명령에서 나온다.
이러한 체제는 계약을 위한 여지를 거의 남겨두지 않는다.
동일한 가\hanja{家}의 구성원들은
\paren{증거로부터 해석하건대}
서로 간에 전혀 계약을 체결할 수 없으며,
가\hanja{家}는 그 구성원이 가를 구속시키려고 맺은 계약을
무시할 수 있는 권리를 가진다.
물론 가와 가 사이, 가부장과 가부장 사이의 계약은 있을 수 있지만,
그 거래는 물건의 양도와 마찬가지 성격을 지니므로
수많은 방식요건들이 부과되어
실행에 있어
사소한 실수라도 계약의무의 성립에 치명적인 것이 된다.
타인의 말을 신뢰하는 것에서 생겨나는 적극적 의무는
진보된 문명이 아주 나중에야 성취하게 되는 것이다.

\para{초기 로마의 계약들}
어떤 고대법도, 다른 어떤 전거도,
계약의 개념을 전혀 알지 못하는 사회가 있음을 보여주지 못한다.
그러나 이 개념이 처음 나타났을 때
그것은 분명 아주 원시적이었을 것이다.
어떤 믿을 만한 원시 기록에서도
약속을 유효하게 만드는 인간의 정신이 아직 미숙하였음을,
그리고
노골적인 배신행위가 비난 없이, 때로는 칭송의 대상으로, 언급되고 있음을
읽을  수 있다.
가령 \wi{호메로스}의 문헌에서
오뒷세우스의 기망적인 교활함은
네스토르의 현려\hanja{賢慮}, 헥토르의 지조,
아킬레우스의 용기와 동급의 미덕으로 나타난다.
고대법은 계약의 원시적 형태가 그것의 성숙한 형태로부터
멀리 떨어져있었음을 훨씬 더 분명히 보여준다.
처음에는 단순히 약속의 이행을 강제하기 위해
법이 개입하지는 않았던 것으로 보인다.
법이 제재로써 강제하는 것은 단순한 약속이 아니라,
엄숙한 의례\hanja{儀禮}를 수반하는 약속이었다.
요식성\hanja{要式性}은 약속과 마찬가지로 중요했을 뿐만 아니라,
어쩌면 약속 이상으로 훨씬 더 중요했다.
성숙한 법학이 구두\hanja{口頭}의 승인\hanja{承認}이 행해진 상황에 적용하는
섬세한 분석이
고대법에서는
그것의 실행에 수반되는 말과 몸짓에 전가된 듯하다.
사소한 방식\latin{form} 하나라도 빠뜨리거나 잘못 실행되면 어떠한 서약도
강제될 수 없었다.
한편, 방식이 정확히 준수되었음이 입증된다면,
사기나 강박으로 약속하였다는 항변은 아무 소용이 없었다.
법제사에서는
이러한 고대적 관념으로부터 우리에게 친숙한 계약 관념으로의 이행이
명백히 드러난다.
처음에는 의례의 한 두 단계가 건너뛸 수 있는 것이 되고,
그후 일정 조건 하에서 다른 것들도 단순화되거나 생략이 허용되며,
마침내 몇몇 특수한 계약들이 다른 것들로부터 분리되어
방식의 구애를 받지 않고 체결할 수 있게 되거니와,
이들 특수한 계약은
사회적 거래의 활동성과 에너지가
이에
의존하는 계약인 것이다.
서서히, 그러나 사뭇 명백하게,
법기술적 요소들로부터 심적\hanja{心的}인 요소가 분리되어 나오고,
차츰 법학자들의 관심을 한몸에 받는 유일한 요소가 된다.
외부적 행위를 통해 표현되는
이러한 심적 요소를 로마인들은
`\wi{약정}'\hanjalatin{約定}{pact; convention}이라 불렀다.
그리고 약정이 계약의 핵심으로 인정되자,
곧이어
방식과 의례의 껍질을 부수어버리는 것이
진보하는 법의 경향성이 된다.
그후 방식들은 진정성을 보증하는 한에서만,
그리고 주의와 숙고를 담보하는 한에서만
보존될 뿐인 것으로 된다.
이로써 계약의 관념은 완전한 발달을 보이게 되거니와,
로마법의 용어를 사용하자면,
계약은 약정에 흡수되어버리는 것이다.

\para{양도와 계약}
로마법이 보여주는 이러한 변화 과정의 역사는 자못 시사적이다.
로마법의 여명기에
계약에 해당되는 말로 쓰인 용어는
고대 라틴어를 연구하는 학자들에게는 무척 익숙한 용어이다.
그것은 바로 넥숨\latin{nexum}, 즉 \wi{구속행위}\hanja{拘束行爲}로서,
이 계약의 당사자들은 `피구속자'\hanja{被拘束者}들\latin{nexi}이라 불렸다.
이 표현들은 그 근저에 놓인 은유의 이례적인 지속성으로 인해
특히 주목할 필요가 있다.
계약관계에 놓인 사람들이 강력한 \hemph{속박}\latin{bond}
또는 \hemph{사슬}\latin{chain}로 연결되어 있다는
관념은 마지막까지 계속해서 로마계약법에 영향을 주었고,
거기서 흘러나와 근대적 관념에도 섞여들어갔다.
그렇다면 이 구속행위 혹은 속박이란 무엇을 의미하는 것이었을까?
라틴어에 관한 고문헌을 통해 우리에게 전해진 바에 따르면
구속행위는 ``구리와 저울로써 행해지는
모든 것''\latin{omne quod geritur per aes et libram}이라고
정의되어 있거니와,\footnote{%
  \latinmarks
  Varro, \textit{De Lingua Latina}, 7.105.
  }
이 단어들은 상당히 큰 혼란을 불러있으켰다.
구리와 저울은
\wi{악취행위}에 수반되는 것들로 잘 알려져있다.
악취행위는
앞 장에서 서술한 고법\hanja{古法}상의 엄숙한 행위로서,
로마 물권법에서 높은 등급의 물건의 소유권이
한 사람에게서 다른 사람에게 양도되는 방식이었다.
이렇게 악취행위는 \hemph{양도}\latin{conveyance}의 방식이기에
어려운 문제가 부상하게 된다.
위에 인용한 저 정의는
계약과 양도를 혼동하고 있거니와,
법철학에서는 이 두 가지가 단지 구분될 뿐만 아니라
사실상 서로 대립하는 것이기 때문이다.
성숙한 법학의 분석가들은
물\hanja{物}에 대한 직접적 권리\latin{jus in re},
대세적\hanja{對世的} 권리\latin{right \textit{in rem}},
``온 세상에 대하여 주장할 수 있는'' 권리,
즉 물권\hanjalatin{物權}{proprietary right}과
물\hanja{物}에 대한 간접적 권리\latin{jus ad rem},
대인적\hanja{對人的} 권리\latin{right \textit{in personam}},
``특정인이나 특정집단에 대하여 주장할 수 있는'' 권리,
즉 채권\hanjalatin{債權}{obligation}을
날카롭게 구별한다.
그런데 양도는 물권을 이전하고, 계약은 채권을 창설한다.
어떻게 이 두 가지가 동일한 이름 아래, 동일한 일반개념 아래
포섭될 수 있다는 말인가?
다른 유사한 난제들과 마찬가지로 이 문제도
미발달된 사회의 정신적 상태에
진보된 지적 단계에 특별히 속하는 능력을,
현실에서는 혼재되어 있는 것을 사변적 관념들로 구별하는 능력을,
끼워맞추려는
오류 탓에 발생한 것이다.
여기서
우리는
양도와 계약이 현실적으로 혼재되어 있는 사회상태에 관하여
오인하지 말아야 한다는 시사를 받는다.
계약과 양도에 관하여 독자적인 실무관행이 채택되기 전까지는
저 개념들 간의 차이는 인식될 수 없었던 것이다.

\para{구속행위}
로마 고법\hanja{古法}에 관한 우리의 지식으로부터
법의 여명기에 법적 개념과 법적 용어가 어떻게 변해갔는지
그 변화의 양상에 대한 약간의 관념을 얻을 수 있을 것이다.
이 변화는 일반적인 것에서 특수적인 것으로의 변화라고 할 수 있다.
다시 말해 고법상의 개념과 고법상의 용어는 점진적 특수화의 과정을
겪었던 것이다.
고법상의 개념은 하나가 아니라 다수의 근대적 개념에 대응된다.
고법상의 법기술적 표현은 근대법이 여러 개의 이름으로 나누어놓은
다수의 것들을 지칭한다.
하지만 법사\hanja{法史}의 다음 단계에 이르면,
하위 개념들이 점차 서로 분리되어,
예전의 일반적 이름은 특수적 명칭들로 바뀌어가는 것이다.
그렇다고 옛 개념이 사라지는 것은 아니고,
단지 원래 포섭하던 관념의 일부만 포섭하게 된다.
그리하여 예전의 법기술적 이름은 여전히 존재하지만,
한때 수행했던 기능들 중에 하나만 수행할 뿐이다.
이러한 현상의 예로는 여러 가지를 들 수 있겠다.
가령 여러 종류의 \wi{가부장권}은 한때
그 성격이 모두 동일했고,
따라서 하나의 이름으로 불렸을 것이 틀림없다.
존속친\hanja{尊屬親}에 의해 행사되던 가부장권은
가족에 대해 행사되든 물질적 재산에 대해 행사되든---양떼나 소떼, 노예,
자식, 아내를 불문하고---모두 동일했다.
그것의 옛 로마식 명칭에 대해 완전히 확신할 수는 없지만,
가부장\hemph{권}\latin{power}을 지칭하는 여러 명칭들에
\hemph{마누스}\latin{manus}라는 단어가 들어가 있는 것으로 볼 때,
옛 일반적 명칭은 `마누스'였을 것으로 믿을 만한 근거는 충분해보인다.\footnote{%
  `마누스'는 흔히 `수권'(手權)으로 번역되나 여기서는 본문의 의미맥락상
  원어를 살렸다.
  이하 관련 단어들도 마찬가지다.
}
그러나 로마법이 좀 더 진보하면서,
저 이름도 저 관념도 특수화되었다.
\wi{가부장권}은
그것이 행사되는 대상에 따라
단어에서도 개념에서도 분화되어갔다.
물건이나 노예에 대해 행사될 때는
`도미니움'\latin{dominium},
자식들에 대해서는 `포테스타스'\latin{potestas},
존속친에 의해 다른 사람의 권력에 제공된 자유인에 대해서는
`만키피움'\latin{mancipium}이 되었고,
아내에 대해서는 여전히 `마누스'로 남았다.\footnote{%
  \latin{Gai.\,1.49} 참조. }
여기서 알 수 있듯이,
원래의 단어가 전혀 쓰이지 않게 된 것이 아니라,
과거에 지칭하던 권력행사 중 특수한 한 가지 권력행사에 국한하게 된 것이다.
이 사례를 모범삼아 계약과 양도 간의 역사적 결합관계의 성질에 대해서도
이해를 도모할 수 있을 것이다.
처음에는 모든 엄숙한 거래에 오직 하나의 엄숙한 의례\hanja{儀禮}만
존재했을 것이니,
로마에서는 그것의 명칭이 `\wi{구속행위}'\latin{nexum}였던 것으로 보인다.
물건의 양도에 사용되던 바로 그 방식이
계약의 체결에도 사용되었던 것으로 보인다.
그러나 양도 관념으로부터 계약 관념이 분리되어 나오는 데는
그다지 긴 시간이 필요치 않았다.
그리하여 이중\hanja{二重}의 변화가 일어났다.
``구리와 저울에 의한'' 거래가
물건의 이전을 의도하는 경우에는
`\wi{악취행위}'\latin{mancipation}라는 새롭고 특수한 이름으로 불리게 된다.
옛 이름인 `구속행위'는 여전히 동일한 의례절차를 지칭하지만,
이제
오직 계약을 엄숙하게 체결하는 특수한 목적에만 쓰이게 된다.

\para{변화}
두 세 가지 법개념이 고대에는 하나로 혼재되어있었다고 해서,
거기에 포함된 관념 중 하나가 다른 것들보다 더 오래된 것이 아니라는
말은 아니다.
혹은 그 하나가, 다른 것들이 형성된 후,
이것들보다 크게 우세하거나 우선하지 않는다는 말도 아니다.
하나의 법개념이 오래 계속해서 여러 법개념들을 포괄할 수 있는 이유는,
그리고 하나의 법기술적 용어가 여러 용어들을 대신할 수 있는 이유는,
원시사회의 법에 실무관행의 변화가 일어나더라도
오랫동안 사람들은 그것에 주목하거나 이름붙일
필요를 느끼지 못하기 때문일 것이 분명하다.
비록, 전술했듯이,
처음에는 가부장권에 행사대상에 따른 구별이 없었다 할지라도,
자식들에 대한 권력이 옛 \wi{가부장권}의 근본이었다고
나는 믿어 의심치 않는다.
또한
`\wi{구속행위}'라는 말의 최초의 사용은,
그리고 그것을 사용했던 사람들이 주로 염두에 둔 것은,
물건의 양도에 엄숙한 형식을 부여하려는 것이었음을
나는 의심치 않는다.
구속행위가 원래의 기능으로부터 아주 조금 벗어나기 시작했을 때
그것은 바로 계약의 체결에 사용되었을 것이나,
아주 조금의 변화였기에 그 변화는 오랫동안 인정되거나 감지되지 못했다.
새로운 것을 원한다는 것을 사람들이 자각하지 못했기 때문에
옛 이름은 그대로 남았다.
아무도 수고스럽게 새로운 것을 검토해볼 필요를 느끼지 못했기 때문에
옛 관념은 그대로 남았다.
우리는 이러한 과정의 사례를 유언법의 역사에서 명료하게 살펴본 바 있다.
유언은 처음에는 단순히 재산의 양도였다.
점차 이러한 특수한 양도와 다른 모든 양도 간에 커다란 실무상의 차이가
나타나고 나서야 비로소
이들이
서로 다른 것으로 간주되기 시작했고,
그러고도 수 세기가 흐른 뒤에야
법의 개량에 나선 사람들이
허울뿐인 악취행위에 붙어있던 복잡한 절차를
청소했고 마침내
유언에 있어
유언자의 명시적 의사 외에는 다른 어떤 것도 중요하지 않다는
합의가 이루어졌던 것이다.
유언법의 초기 역사만큼의
절대적 확신을 가지고
계약법의 초기 역사를 추적할 수가 없다는 것은 유감스런 일이지만,
구속행위가 새로운 사용에 놓여짐으로써
계약이 처음 등장했고
이어서
이 실험의 중차대한 실무적 결과로써
계약이 독자적 거래형태로 승인되었다는 것을 암시하는
힌트마저 얻을 수 없는 것은 아니다.
다음과 같은 과정을 대체로 따랐을 것이라는 추측이,
그러나 그리 억지스럽지만은 않은 추측이 가능하다.
구속행위의 통상적인 방식에 의해
일정한 대금을 받고 매매가 행해진다고 가정하자.
매도인은 처분하고자 하는 목적물---가령 노예 한 명---을 가지고 온다.
매수인은 매매대금을 해당하는 구리 덩어리를 가지고 참석한다.
필수적 보조인인 저울소지자\latin{libripens}도 저울을 들고 나와있다.
노예는 정해진 요식절차에 따라 매수인에게 건네진다.
저울소지자는 구리 조각을 저울에 달고는 매도인에게 넘겨준다.
이러한 거래행위가 지속되는 한 그것은 `\wi{구속행위}'이고,
당사자들은 `피구속자'들\latin{nexi}이다.
그러나 그것이 완료되자마자,
구속행위는 끝나고,
매도인과 매수인도 그들의 일시적 관계에서 유래하는 이름으로 불리기를 그친다.
이제 여기서
거래의 역사를 한 걸음 진척시켜보자.
노예는 양도되었으나,
대금은 지불되지 않았다고 가정해보자.
\hemph{이} 경우,
매도인에 관한 한 구속행위는 종료된다.
이미 자기 물건을 넘겨주었으므로 그는 더 이상 `피구속자'\latin{nexus}가 아니다.
그러나 매수인에 관해서는 구속행위가 계속된다.
매수인 쪽에서는 거래가 끝나지 않았고 그는 여전히 `피구속자'로 남는다.
따라서 동일한 용어가 물권을 이전하는 양도를 기술\hanja{記述}함과 동시에
미지불된 매매대금에 관한 채무자의 채무도 기술하고 있음을 알 수 있다.
다시 한 걸음 더 나아가,
완전히 형식적인 거래, 즉 아무 것도 건네지지 \hemph{않고}
아무 것도 지불되지 \hemph{않는} 절차를 상상해보자.
우리는 사뭇 발달된 상거래 행위의 하나, 바로
\hemph{미이행}\hanjalatin{未履行}{executory} \hemph{매매계약}에
도달하게 되는 것이다.

\para{양도와 계약}
대중적 견해에서나 전문가적 견해에서나
\hemph{계약}이라는 것이
오랫동안 \hemph{미완의 양도}\latin{incomplete conveyance}라고
간주된 것이 사실이라면,
이 사실은 여러 모로 의미심장하다.
자연상태의 인류에 관한 지난 세기의 사변적 이론을
``원시사회에서는 물권은 아무 것도 아니었고 채권이 모든 것이었다''는
교리로 요약하는 것이 그다지 부당하지는 않을 것이다.
그러나 이제 우리는
저 명제를 거꾸로 뒤집으면
그것이 오히려 진실에 가깝다는 것을 알게 되었다.
다른 한편,
역사적으로 보면,
양도와 계약의 원시적 결합은
학자들과 법률가들에게 특별히 수수께끼로 여겨지곤 했던 어떤 것을
설명할 수 있을 것이다.
초기 고대법은 어디서나 \hemph{채무자들}을 무척 가혹하게 처우했으며,
\hemph{채권자들}에게는 막강한 권한을 주었다는 수수께끼 말이다.
구속행위가 채무자에게는 인위적으로 긴 시간 동안 지속되었음을 알고 나면,
대중들과 법이 바라보는 그가 지위가 어떤 것이었을지
더 잘 이해할 수 있는 것이다.
그의 채무상태는 틀림없이 비정상적인 것으로 여겨졌을 것이고,
지불의 해태\hanja{懈怠}는 일반적으로
간교한 책략이자 엄격법의 왜곡으로 비춰졌을 것이다.
반대로,
거래에서의 자신의 의무를 성실하게 완수한 사람은
특별한 호의로써 대우받았을 것이니,
엄격법에 따르면 연장되거나 지체되어서는 안 될
어떤 절차를 강제로 완성시킬 권한을 그에게 주는 것보다
더 당연한 일은 없어보인다.

\para{로마법의 합의 분석}
따라서 \wi{구속행위}는 원래 재산의 양도를 의미했지만,
부지불식간에 계약도 의미하게 되었고,
구속행위 개념과 계약 관념 간의 결합이 오랫동안 지속되었기에
마침내
\wi{악취행위}\latin{mancipatio}라는 특별한 용어가
진정한 구속행위, 즉 실제로 재산이 양도되는 거래를 지칭하는 데
사용되게 되었다.
그리하여 계약은 이제 양도와 분리되었고
이로써 계약법 역사의 첫 단계가 마무리되었으나,
계약당사자의 약속이 이를 둘러싼 요식성보다 더 신성\hanja{神聖}하게 평가되는
단계에 이르기까지는 아직 한참 멀리 떨어져있었다.
그 사이 기간 동안 진행된 변화의 성격을 알아보려면,
지금 우리가 다루고 있는 주제의 범위를 살짝 넘어갈 필요가 있거니와,
바로 로마 법학자들이 합의\latin{agreement}를 어떻게 분석했는지
살펴보는 것이다.
그들의 재능이 만들어낸 가장 아름다운 기념비인
이 분석에 관하여, 나는
그것이 \wi{채권채무관계}\latin{obligation}를 \wi{약정}\latin{pact}으로부터
이론적으로 분리하는 데 기초하고 있다는 것 이상을 말할
필요를 느끼지 않는다.
\wi{벤담}과 \wi{오스틴} 씨는 이렇게 주장했다.
``계약의 주요 성질은 다음 두 가지이다:
첫째,
하기로 약속하는 작위를 하겠다는,
또는
하지 않기로 약속하는 부작위를 하지 않겠다는
\wi{낙약자}\hanja{諾約者}의
\hemph{의사}의 표시.
둘째,
이 주어진 약속을 낙약자가 이행할 것이라는 데 대한
\wi{요약자}\hanja{要約者}의
\hemph{기대}의 표시.''
이것은
로마 법률가들의 법리와 거의 동일하지만,
그러나 로마 법률가들은
이러한 ``표시''의 결과를 `계약'이 아니라
`\wi{약정}'으로 보았다.
약정은 개인들 간의 합의로 맺어지는 약속의 최종 산물이지만,
그렇다고 그것이 바로 계약인 것은 아니다.
약정이 계약이 되는가 여부는
법이 그것에 채권채무관계를 덧붙이느냐 여부에 달려있다.
계약이란 `약정' \hemph{더하기} `채권채무관계'인 것이다.
약정이 채권채무관계의 옷을 입고 있지 않는 한,
\index{나약정}%
그것은 \hemph{나}약정\hanja{裸約定}, 즉 \hemph{벌거벗은} 약정이라 불렸다.

\para{로마법의 채권채무관계}
\wi{채권채무관계}\latin{obligation}란 무엇인가?
로마 법률가들의 정의에 따르면
``누군가에게 급부\hanja{給付}를 할 것이 필연적으로 강제되는
법의 사슬''\latin{juris vinculum, quo
necessitate adstringimur alicujus solvendae rei}이다.\footnote{%
  \latin{Inst.\,3.13.pr.} }
이 정의는
채권채무관계를 \wi{구속행위}와 연결짓거니와,
이들의 배경에 놓인 공통의 은유를 통해서 그러하다.
또한 이 정의는
특정 개념의 계보를 사뭇 명료하게 보여준다.
채권채무관계는 ``속박'' 또는 ``사슬''이거니와,
이로써
사람들 혹은 사람들의 집단들은
어떤 자발적 행위의 결과로서
법에 의해 하나로 결속되는 것이다.
채권채무관계를 이끌어내는 행위들은 주로
합의와 위법행위, 즉
계약과 불법행위라는 표제 아래 분류되지만,
정확하게 분류되기 힘든 여러 다른 행위들도 유사한 결과를 낳는다.
하지만 유의할 점은
어떤 도덕적 필요성도 \wi{약정}을 바로 채권채무관계로 만들지는 못한다는 것이다.
약정에 채권채무관계의 힘을 완전히 부여하는 것은 법이다.
이 점 더욱 유의할 필요가 있거니와,
도덕적 또는 형이상학적 이론을 지지하는 근대 대륙법학자들에 의해
때로 이와 다른 법리가 주창되어왔기 때문이다.
`법의 사슬'\latin{vinculum juris}이라는 이미지는
로마 계약법과 불법행위법의 모든 부분을 물들이고 있고 지배하고 있다.
법은 당사자들을 속박하거니와,
이 \hemph{사슬}은 `변제'\hanjalatin{辨濟}{solutio}라고 불리는 과정을
통해서만 풀 수 있다.
`변제'라는 표현도 은유적이거니와,
``지불''\latin{payment}이라는 일상용어는 가끔씩 그리고 우연히
여기에 들어맞을 뿐이다.
이들 은유적 이미지의 일관성은
이것이 없었다면 혼란을 초래했을
로마법 용어의 특별한 의미를 이해할 수 있게 해준다.
즉, ``\wi{채권채무관계}''\latin{obligation}는 의무뿐만 아니라
권리도 의미하는 것이다.
이를테면 빌린 돈을 지불할 의무뿐만 아니라
빌려준 돈을 지불받을 권리도 의미한다.
실로 로마인들의 눈 앞에는 ``법의 사슬''의 전체 그림이
펼쳐져있었으며,
사슬의 한쪽 끝을 다른 쪽 끝보다 더 많이 혹은 더 적게
바라보지 않았다.

\para{약정과 계약}
발달된 로마법에서는
\wi{약정}이 체결되자마자 거의 모든 경우
즉시
채권채무관계의 관\hanja{冠}이 씌워지고, 따라서 계약이 된다.
이것은 분명 계약법이 지향하는 결과이다.
그러나 우리의 탐구의 목적을 위해서는
그 중간 단계, 즉 채권채무관계가 되려면 완전한 합의 이상의 어떤 것이
요구되는 단계에 주목해야 한다.
이 단계는
네 종류의 계약---\wi{언어계약}, \wi{문서계약}, \wi{요물계약}, \wi{낙성계약}---을 구분한,
저 유명한 로마법상의 분류가 사용되던 시기와 일치한다.
이 시기 동안 법에 의해 강제된 약속은 저 네 가지에 국한되었다.
채권채무관계를 약정과 분리하는 이론을 알고 있다면
저 네 가지 항목의 의미는
쉽게 이해될 수 있다.
사실,
계약들의 각 항목은
계약당사자들의 단순한 합의 이외에 어떤 요식성이 필요한가에 따라
이름붙여진 것이다.
언어\latin{verbal}계약에서는 약정이 체결되는 순간
일정한 방식의 말들이 발설되어야 ``법의 사슬''이 부여된다.
문서\latin{literal}계약에서는
원장\hanja{元帳}, 즉 회계장부에 기입되어야
\wi{약정}에 \wi{채권채무관계}의 옷이 입혀진다.
요물\latin{real}계약의 경우
예비적 약속의 목적물인
물건의 \wi{인도}\hanja{引渡}가 있어야 동일한 결과가 뒤따른다.
요컨대,
이 모든 경우
계약당사자들 간에는 의사합치가 있어야 하지만,
거기에만 그친다면 그들은 서로에게 \hemph{채권채무}를 갖지 못하고,
따라서 이행을 강제할 수도,
신의\hanja{信義} 위반을 이유로 배상을 청구할 수도 없다.
그러나 그들이 어떤 정해진 요식성을 충족시킨다면,
계약은 바로 체결되고,
그 계약의 이름은 그들이 채택한 특정한 방식에 따라 붙여지는 것이다.
이러한 관행에 대한 예외는 조금 있다 살펴보겠다.

\para{언어계약}
나는 네 가지 계약들을 역사적 순서에 따라 열거했으나,
로마의 \wi{법학제요} 저자들이 이 순서를 반드시 따른 것은 아니다.
\wi{언어계약}이 네 가지 중에 가장 오래된 것임에는 의심의 여지가 없다.
그것은 원시적 \wi{구속행위}의 후손으로 알려진 것 중에 가장 먼저 나타났다.\footnote{%
  <<고대법>>에 대한 폴록의 주석에 따르면,
  문답계약의 기원은 구속행위(nexum)가 아니라
  선서(oath)와 같은 초기의 종교적인 채권채무관계에서 찾는 것이
  오늘날
  일반적이라 한다.
  }
언어계약에 속하는 몇몇 종\hanja{種}이 옛날에는 사용되었으나,
가장 중요한 것이자 우리의 전거들이 다루었던 유일한 것은
질문과 답변으로 이루어진 `\wi{문답계약}'\latin{stipulation}이다.
\wi{요약자}가 질문을 하고 \wi{낙약자}가 답변을 한다.
이러한 질문과 답변이야말로, 전술했듯이,
원시적 관념이
당사자들 간의 단순한 합의를 넘어 추가적으로 요청하는 요소인 것이다.
이들은 채권채무관계가 부여되기 위한 매개체이다.
옛 구속행위는
보다 성숙한 법학에게 무엇보다
계약당사자들을 결속시키는 사슬의 개념을 물려주었으니,
이것이 이제 \wi{채권채무관계}가 되었다.
그것은 또한 약속에 수반하면서 약속을 성별\hanja{聖別}하는
의례행위의 개념도 물려주었으니,
이 의례행위가 이제 질문과 답변으로 변형된 것이다.
초기 구속행위의 특징이었던 엄숙한 양도행위가
어떻게 단순한 질문과 답변으로 전환되었을까 하는 것은
이와 유사한 로마 유언법의 역사가 우리에게 가르쳐준 바가 없었다면
더욱 미스테리로 남았을 것이다.
유언법의 역사를 돌아보면,
실질적 관심 대상에 직접 관련되는 절차 부분\footnote{%
  `양도'와 대비되는 `언명'(nuncupatio)을 말하는 듯하다.
}으로부터
어떻게
형식적 양도가
처음 분리되었는지를,
그리고 어떻게 그후 이것을 완전히 생략하게 되었는지를
이해할 수 있다.
그렇다면
\wi{문답계약}의 질문과 답변은 단순화된 형태의 \wi{구속행위}였을 것이 분명하고,
따라서
그것은 오랫동안 법기술적 형태의 성질을 가졌을 것이라고 쉽게 추정할 수 있다.
옛 로마 법률가들이 문답계약을 옹호했던 이유를
합의에 임한 당사자들에게 숙고하고 성찰할 기회를 제공하는
유용성때문이라고 본다면 이는 잘못일 것이다.
물론 이런 유\hanja{類}의 가치가 있었고 점점 중요하게
인식되었다는 것을 부인할 수는 없다.
그러나, 우리의 전거들에 나타난 증거에 비추어볼 때,
계약에 관련된 그것의 기능은 처음에는 형식적이고 의례적인 것이었다.
문답계약을 구성할 수 있는 오래된 질문과 답변은
특정한 경우에 적합한 법기술적 용어들을 사용한
질문과 답변에만 국한되었고,
아무 질문이나 답변이라고 해서 다 되는 것이 아니었다.

\para{문답계약}
그러나,
비록 \wi{문답계약}이 유용한 안전장치로 인식되기 이전에
엄숙한 형식으로 인식되었다고 이해하는 것이
계약법의 역사를 올바르게 평가하는 데 필수적이라 할지라도,
다른 한편
그것의 현실적 유용성에 눈을 감아버리는 것도
잘못된 일일 것이다.
언어계약은, 비록 고법\hanja{古法}상에서 누리던 중요성을
상당 부분 상실해갔지만, 로마법의 마지막 시기까지 계속 살아남았다.
당연한 말이겠지만,
로마법의 어떤 제도도
어떤 현실적 유용성이 없었다면
그렇게 오래 유지될 수 없었을 것이다.
놀랍게도
어떤 영국 학자의 말에 따르면,
로마인들은
초창기부터도
숙고없이 서둘러 계약을 맺는 것에 대한 방비가 거의 없어도
괘념치 않았다고 한다.
그러나 문답계약을 면밀히 조사해보면,
그리고 서면증거를 만들기 쉽지 않았던 당시의 사회상태를 감안하면,
\wi{문답계약}의 질문과 답변은,
만약 그것이 실제 기여한 목적을 위해 의도적으로 고안되었다면,
그야말로 천재적인 방책이었다고 평가해도 좋다고 생각한다.
\hemph{\wi{요약자}}\latin{promisee}가
계약의 모든 조항을 질문의 형태로 만들어 질문하면,
\hemph{\wi{낙약자}}\latin{promisor}가 답변을 한다.
``당신은 이러이러한 노예를 이러이러한 장소에서 이러이러한 날짜에
나에게 인도할 것을 약속하는가?''
``약속하노라.''
잠깐만 생각해보면,
이렇게 질문 형태로 구성되는 \wi{채권채무관계}는
당사자들의 자연스런 입장을 거꾸로 뒤집고,
대화의 통상적인 흐름을 깨뜨리는 효과를 가져와,
위험한 약속에 빠지지 않도록 주의를 환기시키는 기능을 함을 알 수 있다.
우리 영국인들이 행하는 구두의 약속은
오직 약속자\latin{promisor}의 말로 구성되는 것이 일반적이다.\footnote{%
  원어로는 똑같이 `promisor'이지만,
  로마법의 맥락에서는 `낙약자'(promissor)로,
  영국법의 맥락에서는 `약속자'로 번역하고 있음을 유의할 것.
  마찬가지로 `promisee'는 로마법의 맥락에서는 `요약자'(stipulator)로,
  영국법의 맥락에서는 `수약자'로 번역된다.
  사실 `요약자'니 `낙약자'니 하는 우리의 법률용어는
  바로 로마법의 문답계약에서 유래하는 말이다.
  }
옛 로마법에서는 또 하나의 단계가 반드시 요구된다.
합의가 이루어지고 나면
\wi{요약자}가 엄숙한 질문의 형태로 이 합의의 모든 조항들을 요약해야 하는 것이다.
재판에서 증거로 제출되는 것은,
구속력 없는 약속 자체가 \hemph{아니라},
바로 이 질문과 그에 대한 답변인 것이다.
일견 사소해보이는 이 차이가
계약법 용어에 얼마나 큰 차이를 만들어내는지는
로마법 입문자들이 입문 후 금세 깨닫게 되는 것이니,
그들은 첫 번째 걸림돌을 거의 언제나 여기서 만나고 있다.
우리가 영어로 어떤 계약에 관해 말하면서
그것을 편의상 한쪽 당사자와 결부시키는 경우---가령
어떤 계약의 당사자에 대해 일반적으로 말하려는 경우---우리의
말이 지시하는 것은 언제나
\hemph{약속자}\latin{promis\textit{or}} 쪽이다.
그러나 로마법의 일반적 언어는 방향이 반대이다.
로마법은 계약을 언제나, 이런 용어를 쓸 수 있다면,
\hemph{수약자}\hanjalatin{受約者}{promis\textit{ee}} 쪽에서 바라본다.
계약의 당사자 중에서
주로 언급되는 대상은 언제나 \wi{요약자}\latin{stipulator}, 즉
질문을 하는 사람이다.
하지만
\wi{문답계약}의 유용성이 자못 생생하게 드러나는 예를
라틴 희극작가들의 희곡의 몇몇 페이지에서도 찾아볼 수 있다.
이 대목들이 나오는 장면 전체를 읽어보면
\paren{예컨대, Plautus, \textit{Pseudolus}, 1막 1장;
4막 6장; \textit{Trinummus}, 5막 2장},
질문하는 것이 계약에 임한 사람의 주의를 얼마나 많이 이끌어내는지,
그리고
즉흥적인 합의에 이르지 않게 할 가능성이 얼마나 커지는지
알 수 있을 것이다.

\para{문서계약}
\wi{문서계약}에서 \wi{약정}에 \wi{채권채무관계}가 입혀지기 위해 필요한
요식행위는,
채무액이 확정될 수 있는 경우,
채무의 총액을
원장\hanja{元帳}의 차변\hanja{借邊}에 기입하는 것이었다.
이 계약은 로마의 가\hanja{家}의 관행에 의해 설명될 수 있거니와,
고대에는 그것이 체계적인 성격을 띠었고 무척 규칙적으로 장부작성이
이루어졌던 것이다.
로마 고법\hanja{古法}과 관련하여
가령 노예의 \wi{특유재산}\latin{peculium}의 성격 같은
몇몇 작은 문제들이 있거니와,
이는
로마의 가\hanja{家}가 가부장에게 엄격히 책임지는 다수의 사람들로
구성되었고,
가의 수입과 지출의 모든 항목은,
일단 일지\hanja{日誌}에 기재된 후,
정해진 시기에
가의 총\hanja{總}원장에 이기\hanja{移記}되었음을 상기할 때
비로소 해소될 수 있다.
하지만 문서계약에 관해 남아있는 기술\hanja{記述}에는
몇 가지 모호한 점이 있거니와,
사실
나중에는
장부작성의 습관이
보편적이지 않게 되었고,
``문서계약''이라는 표현은 이제 원래 가졌던 의미와 완전히
다른 형태의 계약을 지칭하게 되었던 것이다.
따라서 우리는
초기의 \wi{문서계약}에서
채권채무관계가 단순히 채권자 측의 기입만으로 성립했는지,
아니면
그것이 법적 효력을 가지려면
채무자의 동일한 행위, 즉 채무자 측 장부의 상응하는 기입도 필요했는지
말할 수 있는 입장에 있지 않다.
하지만
이 계약의 경우
한 가지 조건만 충족되면 다른 모든 요식성은 필요치 않다는
핵심적 성격만은 확실히 알려져있다.
이것은 계약법의 역사에서 또 한 걸음의 진전이었던 것이다.

\para{요물계약}
역사적 순서에 따라 다음에 등장하는 계약인 \wi{요물계약}은
윤리 개념의 큰 진전을 보여준다.
어떤 합의가 물건의 \wi{인도}를 목적으로 한다면---이는
대다수의 단순한 계약을 포괄한다---그 인도가 실제로 행해지는 즉시
\wi{채권채무관계}가 성립하는 것이다.
이러한 결과는 초기 계약 관념에 큰 혁신을 의미했다.
의심의 여지 없이 초창기에는,
계약의 당사자가 자신의 합의에 문답계약의 옷을 입히지 못했다면,
계약의 이행으로써 무엇을 했던지 간에
법은 아무 것도 인정해주지 않을 것이기 때문이다.
공식적으로 \hemph{\wi{문답계약}}을 체결하지 않았다면
돈을 빌려주었더라도 빌린 돈을 갚으라고 소송할 수 없었던 것이다.
그러나 요물계약에서는
일방의 이행이 상대방에게 법적 의무를---분명 윤리적 근거에서---부과한다.
그리하여 도덕적 고려가 계약법의 요소로 처음으로 등장한 것이다.
요물계약이
앞서 살펴본 두 가지와 다른 점은,
법기술적 방식이나 로마의 가\hanja{家}의 습관에 대한 존중이 아니라,
도덕적 고려에 기초한다는 데 있다.

\para{낙성계약}
이제 네 번째 유형, 즉 \wi{낙성계약}에 이르렀거니와,
이것은 가장 흥미롭고 가장 중요한 유형이다.
여기에는 네 가지 계약들이 속했고, 그 이름은 다음과 같다:
위임\latin{mandatum}, 조합\latin{societas},
매매\latin{emtio-venditio}, 임약\hanjalatin{賃約}{locatio-conductio}.\footnote{%
  `임약'은 우리 민법의 `임대차' `고용' `도급'을 포괄하는 개념이다.
  }
몇 페이지 앞에서
계약이란 약정에 채권채무관계가 덧붙여진 것이라고 말한 후,
나는
약정이 채권채무관계로 되기 위해 법이 요구하는
어떤 행위나 요식성에 관해 이야기했다.
나는 일반적 표현의 장점을 활용해 이 말을 하였으나,
적극적인 것 외에 소극적인 것까지 포괄하는 것으로 이해하지 않으면
전적으로 옳은 말이 되지는 못한다.
기실,
낙성계약의 특이성은 \wi{약정} 외에 그 어떤 요식성도
\hemph{전혀} 요구되지 않는다는 것이기 때문이다.
낙성계약에 관하여 많은 옹호될 수 없는 것들이,
더 많은 모호한 것들이 주장되어왔거니와,
심지어 낙성계약에서는 당사자들의 \hemph{동의}\latin{consent}가
다른 어떤 합의 유형들보다 더 강하게 주어진다는 주장까지 있었다.
그러나 저 `낙성'\hanjalatin{諾成}{consensual}이라는 용어는
여기서는
단지 \hemph{합의}\latin{consensus}만 있으면 바로 \wi{채권채무관계}가
부가된다는 것을 의미할 뿐이다.
합의, 즉 당사자들의 상호 동의는
약정에 있어 최종의 그리고 최고의 요소이다.
당사자들의 동의가 이 요소를 제공하자마자
\hemph{즉시} 계약이 성립하는 것은
매매, 조합, 위임, 임약의 네 가지 표제 중 하나에 속하는
합의의 특유한 성질이다.
이 합의는 바로 채권채무관계를 끌고 들어오기에,
특정 종류의 거래에서 그것이 행하는 기능은
다른 종류의 계약에서 물건이, 문답의 언어가,
문서, 즉 장부에의 기입이 행하는 기능과 정확히 일치한다.
따라서 `낙성'은
조금도 이상할 것이 없는 용어이며,
`요물' `언어' `문서'에 정확히 대응한다.

일상생활에서
가장 흔하고 가장 중요한 계약은, 말할 것도 없이,
네 가지 이름의 계약을 포괄하는 \wi{낙성계약}이다.
어떤 공동체든 집단생활의 대단히 큰 부분이
사고 파는 거래, 세\hanja{貰}놓고 세드는 거래,
공동사업을 위해 사람들이 결합하는 거래,
업무처리를 타인에게 맡기는 거래로 구성된다.
바로 그 때문에
다른 많은 사회들과 마찬가지로 로마도
이들 거래에서 법기술적 장애물을 제거하여,
사회적 운동의 효율적 동력이
가능한 한
방해받지 않도록 했을 것이다.
물론 이러한 동기는 로마에만 국한된 것이 아니었다.
로마인들과 이웃 민족들 간의 거래는
우리가 말한 계약들이 어디서나 \hemph{낙성계약},
즉 상호 동의의 의사표시만으로 구속력이 부여되는 계약이
되어가는 것을 관찰할 수 있는
풍부한 기회를
로마인들에게
제공했을 것이다.
그리하여 그들의 통상적인 관행에 따라
로마인들은 이들 계약을
\wi{만민법}상의\latin{juris gentium} 계약으로 분류했다.
하지만 나는
아주 초기부터 이렇게 불리지는 않았다고 생각한다.
만민법\latin{jus gentium}이라는 관념은
외인\hanja{外人}담당법무관\latin{praetor peregrinus}이 임명되기 오래 전부터
로마 법률가들의 정신에 이미 들어있었다.
그러나 다른 이탈리아 공동체들의 계약 관행에 로마인들이 익숙해진 것은
수많은 거래가 일상적으로 이루어지면서부터일 것이고,
이러한 거래는 이탈리아가 완전히 평정되어
로마의 패권이 확고해지고 나서
비로소 대규모로 이루어질 수 있었을 것이다.
하지만, 비록
낙성계약이 가장 늦게 도입된 것일 확률이 무척 크다고 할지라도,
그리고
`만민법상의'\latin{juris gentium}라는 수식어가 그것의 뒤늦은 도입을
나타내는 것이라 하더라도,
낙성계약을 ``만민법''\latin{law of nations}에 귀속시키는
바로 이 표현이 근대에 들어서는
그것이 아득한 옛날의 것이라는 관념을 만들어냈다.
``만민법''\latin{law of nations}이
``자연법''\latin{law of nature}으로
전환되자,
저 표현은
낙성계약이 자연상태에 가장 부합하는 합의 유형임을
의미한다고 여겨졌던 것이다.
그리하여 문명의 나이가 어릴수록
계약의 형태는 더 단순할 것이라는 이상한 믿음이 형성되었다.

\para{자연법적 채무와 시민법적 채무}
전술했듯이 낙성계약에 속하는 계약들은 그 수가 대단히 제한적이었다.
그러나
낙성계약으로 대표되는 계약법 발달의 단계가
계약에 관한 모든 근대적 관념의 출발점이었음은 의심할 여지가 없다.
이제 합의를 구성하는 의사\latin{will} 작용은
다른 것들과 완전히 분리되어 독립적 고찰의 대상이 되었다.
계약 관념에서 방식은 완전히 제거되었고,
외적 행위는 오직 내적 의사의 징표로만 간주되었다.
더욱이 \wi{낙성계약}은 \wi{만민법}\latin{jus gentium}으로 분류되었으니,
이러한 분류는
얼마 안 가
낙성계약이야말로
자연에 의해 승인되고 자연의 법전에 포함된
계약을 대표하는 합의 유형이라는 추론을 형성시켰다.
이 지점에 이르러, 우리는
로마 법률가들의 몇몇 유명한 법리와 구분들을 만나게 된다.
그중 하나가 자연법상의 채무\latin{natural obligation}와
\wi{시민법}상의 채무\latin{civil obligation}의 구분이다.
지적으로 완전히 성숙한 어떤 사람이
자신의 의사에 기해
어떤 약속을 맺었다면,
비록 필요한 방식을 다 갖추지 못했더라도,
또는 어떤 법기술적 장애로 인해
유효한 계약을 체결할 법적 능력이 결여되어 있었다 할지라도,
그는 \hemph{\wi{자연채무}}\latin{natural obligation}를 진다.
법은
\paren{바로 이것이 저 구분이 의미하는 바이다}
이런 채무를 강제하지 않는다.
그러나 법이 이런 채무를 전혀 인정하지 않은 것은 아니다.
\hemph{자연채무}는
단순히 무효인 채무와는 여러 모로 달랐거니와,
특히 계약 체결 능력을 사후적으로 취득한다면
시민법적으로도 인정될 수 있었던 것이다.\footnote{%
  가령 노예가 진 빚은 자연채무였다. 따라서 노예가 해방된 뒤
  스스로 빚을 갚으면 반환받을 수 없었다.
  \latin{D.\,12.6.13.pr.}
  마찬가지로 후견인의 조성(助成) 없이 미성숙자가 돈을 빌려 이득을 본 경우,
  그가 성숙기에 달한 후 갚으면 반환받을 수 없다.
  \latin{D.\,12.6.13.1.}
  }
법학자들의 또 하나의 특유의 법리는 약정이
계약의 법기술적 요소로부터 분리된 후에 비로소 등장할 수 있었다.
그들에 따르면,
계약만이 \hemph{소권}\hanjalatin{訴權}{action}을 근거지울 수 있었지만,
단순한 \wi{약정}은 \hemph{\wi{항변권}}\hanjalatin{抗辯權}{plea}의 기초가 될 수 있었다.
그리하여,
누구도
계약을 성립시키는 데 필요한
방식을 갖추지 못한
합의에 기초하여 소송을 제기할 수 없었지만,
유효한 계약에 기초한 주장이라도
단순한 약정 상태에 불과한 다른 합의가 있었음을 입증함으로써
이를 물리칠 수 있었다.
가령 금전채무의 회수를 구하는 소송은
채무 면제나 유예의
단순한 비공식적 합의를 입증하여 이에 대항할 수 있었다.

\para{계약법의 변화}
방금 언급한 법리는 법무관들이 궁극적 혁신을 이루어내는 데
방해물로 작용했을 것이다.
그들의 자연법 이론은
낙성계약과 이를 포함하는 약정 일반에 대해
특별한 호의를 갖도록 그들을 이끌었을 것이 틀림없다.
그러나 그들이 즉시
모든 약정에 낙성계약의 효력을 확대적용하는 모험을 감행한 것은 아니었다.
그들은 로마법 초기부터 그들에게 주어졌던
소송절차에 대한 감독권한을 이용하였으니,
방식을 갖추지 못한 계약에 근거한 소송은 여전히 허가하지 않았지만,
합의에 관한 새로운 이론을 적극 활용하여
이후의 발달 단계로 향하는 길을 텄다.\footnote{%
  이른바 `법무관법상의 약정'(pacta praetoria)을 말하는 듯하다.
  특정 기일에 자기 또는 타인의 기존 채무를 변제하겠다는 약정,
  선박주인^^b7여관주인^^b7마구간주인 등의 고객 물건에 대한 인수(引受)약정
  따위가 여기에 속한다.
  }
그러나
여기까지 나아가자
계속 더 나아가는 것이 불가피해졌다.
고대 계약법의 혁명이 달성된 것은,
어느 해인가 \wi{법무관}의 고시\hanja{告示}가
계약의 옷을 입지 못한 \wi{약정}이라도
그 약정이 \wi{대가관계}\latin{consideration}\paren{원인\latin{causa}}에
기초하는 것인 한
\wi{형평법}상의 소송을 허가하겠노라고 공표했을 때였다.\footnote{%
  이른바 `무명요물계약'(無名要物契約)을 말하는 듯하다.
  쌍무성(synallagma) 있는 계약의 당사자라면
  자신의 급부를 이행한 경우---따라서 일종의 요물계약이었다---상대방의
  이행을 소구할 수 있었다.
  혹은 자신의 급부를 돌려달라는 부당이득반환청구소송을 제기할 수 있었다.
  }
이런 종류의 약정은 발달된 로마법에서는 항상 강제되었다.
이 원리는
\wi{낙성계약}의 원리가 그 합당한 결과에 도달한 것에 불과했다.
사실, 로마인들의 법기술적 용어가 그들의 법이론만큼이나 유연했다면,
법무관에 의해 강제된 이들 약정은
새로운 계약, 새로운 낙성계약이라고 불렸을지도 모른다.
하지만 법용어는 가장 바뀌기 어려운 법이어서,
형평법적으로 강제된 저 약정들은
여전히 `\wi{법무관법상의 약정}'\latin{praetorian pacts}이라고만 지칭되었다.
만약 약정에 \wi{대가관계}가 없다면,
\index{나약정}%
새로운 법에서도 계속 \hemph{나}약정\hanja{裸約定}이었음을
유의해야 한다.
이것에 법적 효력을 부여하려면,
문답계약을 통해 언어계약으로 전환시켜야 했다.

\para{계약법의 진화}
수많은 오해에 대한 방패막이로서 큰 중요성을 가지기에
계약법의 역사를
이렇게
길게 논하고 있는 것도 이해해주시리라 믿는다.
그것은 하나의 중대한 법관념으로부터
다른 중대한 법관념으로의 관념들의 행진을 완전히 설명해준다.
우리는 구속행위로부터 시작하였으니,
여기서는 계약과 양도가 혼재되어 있고,
합의에 수반되는 요식성이 합의 자체보다 훨씬 중요하다.
구속행위 다음에 오는 \wi{문답계약}은 옛 의례행위의 단순화된 방식이다.
그 다음의 \wi{문서계약}에서는
로마의 가\hanja{家}의 엄격한 관행에 의해
합의가 입증되기만 하면 다른 모든 요식성은 포기된다.
\wi{요물계약}에서는 도덕적 의무가 처음으로 인정되니,
계약의 일부 이행을 수령하거나 묵인한 자는
방식의 흠결을 이유로 계약을 부인하는 것이 금지된다.
끝으로 \wi{낙성계약}이 등장함으로써,
계약 당사자들의 내심의 의사만 고려대상이 되고
외적 격식은 내적 의사의 증거로서만 의미를 갖는다.
조야한 개념에서 세련된 개념으로 나아가는 로마법의 이러한 관념의 진보가
계약에 관한 인간 사고의 필연적 진보과정을 얼마나 예시하고 있는지는
물론 확인할 수 없다.
로마를 제외한 다른 모든 고대사회는
계약법이 너무 부족하여 정보를 얻을 수 없거나,
아니면 계약법 자체가 아예 없다.
또한 근대법은 철저히 로마법의 관념을 이어받은 것이어서
가르침을 구할 만한 비교대상이 되지 못한다.
하지만 우리가 살펴본 고대 로마계약법의 역사에는
억지스럽거나 놀랍거나 불가해한 것이 전혀 없기에,
어느 정도는 그것이
다른 고대사회의 계약법 개념의 역사에도 통용된다고
보아도 불합리하지 않을 것이다.
그러나 로마법의 진보가 다른 법체계의 진보를 대표하더라도,
그것은 어느 정도까지만 그렇다는 것이지 그 이상은 아니다.
자연법 이론은 로마법에만 있었다.
`법의 사슬'이라는 관념도, 내가 알고 있는 한,
로마법에만 있었다.
로마의 성숙한 계약법과 불법행위법의 많은 특징들은
이 두 가지  관념이 따로 혹은 함께 작용한 결과이거니와,
따라서 특정한 사회 하나만의 산물인 것이다.
이 후대의 법관념이 갖는 중요성은,
어떤 상황에서도 진보적 사고의 필연적 결과를 대표한다는 데
있는 것이 아니라,
근대 세계의 지적 기질에 엄청나게 큰 영향력을 행사했다는 데 있다.

\para{로마제국의 법적 사고, 동방과 서방의 관념}
로마법, 특히 로마계약법이
다양한 학문의
사고양식, 추론과정, 전문용어에 기여한 것보다
더 대단한 일이 또 있는지 모르겠다.
자연과학을 제외하고,
근대인의 지적 욕구를 자극한 대상 중에
로마법이라는 여과장치를 통과하지 않은 것은 거의 없다.
순수한 형이상학은 물론 로마보다는 그리스의 후예이지만,
정치학, 도덕철학, 심지어 신학까지,
모두가 로마법에서 표현수단을 발견했을 뿐만 아니라,
거기서 학문 발달을 위한 깊이있는 탐구가 배양되는 거점도 발견했다.\footnote{%
  이 단락부터 이 챕터의 거의 끝까지는
  정치학, 도덕철학(윤리학), 신학에 끼친
  로마법의 영향, 특히 로마계약법의 영향에 대한 논의가 이어진다.
  }
이런 현상을 설명하기 위해,
말과 관념 간의 불가사의한 관계를 논할 필요는 전혀 없을 것이며,
또한
적절한 언어의 저장고와 적절한 추론의 장치가 미리 주어지지 않으면
인간 정신은 어떠한 사고 주제도 다룰 수 없었다는 것을 설명할
필요도 전혀 없을 것이다.
동방과 서방의 철학적 관심이 분리되었을 때,
서방 사상의 기초자들은 라틴어로 말하고 라틴어로 사고하는 사회에
속해있었다는 것만 말해도 충분할 것이다.
그런데
서방에서는
철학적 목적을 충족하는 정확성을 갖는 유일한 언어가
로마법의 언어였거니와,
비속\hanja{卑俗}라틴어가 불길한 만족\hanja{蠻族}들의 방언으로 전락해가는 동안,
로마법의 언어는 특별한 행운으로
아우구스투스 시대의 순수함을 거의 그대로 보존할 수 있었다.
로마법이 언어의 정확성을 위한 유일한 수단이었다면,
사고의 정확성과 명석함과 깊이를 위한 유일한 수단은 더더욱 로마법이었다.
서방에서는
적어도 3백녁 동안 철학과 학문이 자리잡지 못하고 있었다.
비록 형이상학과 형이상학적 신학이 로마인 백성들의 정신적 에너지를
독점하고 있었지만,
이러한 열렬한 탐구에 사용된 언어는 오직 그리스어였고,
그러한 탐구의 무대는 동로마제국이었다.
사실,
때로 동로마의 논쟁들의 결과는 사뭇 중요해져서
그것에 찬성하고 반대하는 모든 이들의 의견이 기록되어졌다.
그후 이러한 동방의 논쟁의 결과가 서방에 소개되었으니,
대체로 그것은
감흥없이 그리고 저항없이 받아들여졌다.
그러는 동안,
가장 근면한 자에게도 어렵고,
가장 명석한 자에게도 멀고,
가장 치밀한 자에게도 까다로운
학문 분야 하나가 서방의 식자층 사이에서
매력을 잃지 않고 있었다.
아프리카, 에스파니아, 갈리아, 북이탈리아의 교양있는 시민들에게
그것은 법학, 오직 법학이었으니,
그것은 시와 역사, 철학과 학문을 대신하는 것이었다.
서방 사상의 초기 성과의 명백히 법적인 양상에는
불가사의한 것이 거의 없었으므로,
그것이 다른 의미로 다가왔다면 오히려 놀라운 일이었을 것이다.
나로서는
어떤 새로운 요소의 존재로 인해 생겨난
서방과 동방의 관념의 차이가,
서방의 신학과 동방의 신학의 차이가,
그동안 거의 주목을 받지 못했다는 것이
놀라울 따름이다.
이렇게 법학의 영향이 강력해지기 시작하기 때문에,
콘스탄티노플의 건설과 이후 서로마제국과 동로마제국의 분리가
철학의 역사에서 획을 긋는 사건이 되는 것이다.
그러나 대륙의 사상가들은
분명 이 중차대한 국면의 중요성을 인식하기 어려운 위치에 있으니,
그들은 로마법에서 유래한 관념들에 친숙하고
그것이 일상적 관념들에 섞여들어가 있기 때문이다.
반면, 영국인들은 놀라울 정도로 그것을 알지 못하니,
근대 지식의 가장 풍부한 원천,
로마 문명의 유일한 지적 성과로부터 스스로를 유폐한 것이다.
동시에,
고전기 로마법에 친숙해지는 데 많은 노력을 들이는
영국인이라면,
지금까지 영국인들이 이 분야에 무심했다는 바로 그 사실 덕분에,
내가 감히 내놓는 주장의 가치에 관해
프랑스인이나 독일인보다
더 나은 판관이 될 수 있을 것이다.
로마인들이 실제 관용한 로마법이 무엇인지 아는 사람은,
그리고 초기 서방의 신학과 철학이 그 이전의 사상 국면과 어떻게
달랐는지 아는 사람은,
사변\hanja{思辨}을 지배하기 시작한 새로운 요소가 무엇이었는지
선언할 수 있는 위치에 있다고 할 것이다.

\para{준계약}
로마법의 여러 영역 중에 다른 학분 분야에 가장 큰 영향을 끼친 것은
채권법, 혹은, 거의 같은 말이지만, 계약법과 불법행위법이었다.
로마인들도
이 법영역에 속하는 용어가 감당하게 될 역할을 모르지 않았거니와,
그것은 그들이 특유의 \hemph{준}\hanjalatin{準}{quasi}이라는
수식어를 `\wi{준계약}'\latin{quasi-contract}과
`\wi{준불법행위}'\latin{quasi-delict} 같은 표현에 사용한 것을 보면 알 수 있다.
여기서 ``준''이라는 말은 분류를 위한 용어일 뿐이다.
흔히 영국의 학자들은 준계약을 \hemph{묵시적}\latin{implied} 계약과 동일시해왔으나,
이는 잘못이다.
묵시적 계약은 진짜로 계약이지만 준계약은 그렇지 않기 때문이다.
명시적 계약에서 말로써 상징되는 것과 동일한 요소가
묵시적 계약에서는 행위와 상황에 의해 상징되거니와,
어떤 이가
어느 쪽 상징집합을 사용하든
합의의 이론에 관한 한
아무런 차이가 없다.
그러나 준계약은 계약이 아니다.
이 유형에 해당하는 가장 흔한 사례는
한 사람이 다른 사람에게 착오로 돈을 지불한 경우 두 사람 간의 관계에서 발견된다.
법은, 도덕의 관점에서,
반환할 채무를 수령자에게 지운다.
그러나 그 성질은 계약에 기초하는 것이 아니니,
계약의 본질적 요소인 \wi{약정}\latin{convention}이 결여되어 있기 때문이다.
로마법의 어떤 용어에 붙는
``준''이라는 말은 그것이 지시하는 개념이
비교대상이 되는 개념과 강한 외관상의 유비\hanja{類比} 혹은 유사성으로
연결되어 있다는 것을 의미한다.
두 개념이 동일하다거나, 혹은 동일한 유\hanja{類}에 속한다는 말이 아니다.
오히려
그것들 간의 동일성을 부정하는 의미가 들어있다.
그러나 그것들은 충분히 유사해서
하나가 다른 하나의 속편\hanja{續篇}으로
분류되고,
하나의 법영역의 용어를 다른 법영역에도 쓸 수 있어서,
그렇지 않으면 불완전하게 표현될 수밖에 없는 법규칙의 진술에
과도한 왜곡 없이 사용할 수 있다는 뜻이다.

\para{준계약과 사회계약}
진짜 계약인 묵시적 계약과 계약이 아닌 \wi{준계약} 간의 혼동이
정치적 권리와 의무의 원천을
통치자와 피치자 간의 원초적 계약에서
찾는
저 유명한 오류와 공통점이 많다는 예리한 지적이 있다.
이 이론이 확립되기 오래 전부터도
주권자와 백성들 간에 존재한다고 여겨지는
권리와 의무의 상호성을 기술하는 데
로마계약법의 용어가 자주 사용되어왔다.
무조건적인 복종을 요구할 수 있는 왕의 권리를 적극적으로 내세우는
격률들---신약성서에서 기원한다고 주장되었으나 실은
황제들의 전제정에 대한 기억이 지속된 데서 유래한 격률들---은
세상에 가득했지만,
피치자들이 갖는 상응하는 권리의 인식은,
만약
아직 제대로 발달하지 못한 관념을 암시하는 언어를
로마채권법이
제공해주지 않았다면,
그것을 표현할 수단이 전혀 없었을 것이다.
왕의 특권과 백성들에 대한 그의 의무 간의 대립은
서양 역사가 시작된 이래 한번도 잊혀진 적이 없다고 믿지만,
봉건제도가 굳건히 존속하는 동안은
아주 예외적인 극소수의 사상가를 제외하고는
이 문제에 관심을 두지 않았다.
\wi{봉건제}의 명백한 관습에 의해
유럽의 대부분 주권자들의 터무니없는 이론적 주장이
효과적으로 통제되었기 때문이다.
하지만,
주지하듯이,
봉건체제의 붕괴로 중세의 헌정질서가 혼란에 빠지자,
그리고
종교개혁으로 교황의 권위가 추락하자,
\wi{왕권신수설}\hanja{王權神授說}은
과거에 한번도 누려보지 못한
중요한 이론의 지위로
급부상했다.
이 이론이 얻은 인기는
로마법 용어에 상시 의존하는 경향을 심화시켰고,
원래 신학적 옷을 입고 있던 논쟁은
점점
법적인 논쟁의 분위기를 띠어갔다.
그리하여 여론의 역사에서 반복적으로 나타나던 현상 하나가 등장했다.
군주의 권력을 옹호하는 주장이 필머\latin{Filmer}의 교리로
확립되자,
피치자의 권리를 방어하는 데 사용되었던,
계약법에서 빌려온
용어가
왕과 신민 간의 원초적 계약이 실재한다는 이론으로
구체화되었던 것이다.
이 이론은 처음에는 영국인들의 손에서,
나중에는 특히 프랑스인들의 손에서,
모든 사회현상과 법현상을 포괄적으로 설명하는 이론으로 확장되었다.
그러나
정치학과 법학의 진정한 결합은
후자가 전자에게
특유의 유연한 용어를 제공한 것이 전부였다.
로마계약법은,
주권자와 백성의 관계에 대해서도,
보다 소박한 영역에서
``\wi{준계약}''의 \wi{채권채무관계}로 묶인 사람들의 관계에 대해 수행하던 것과
정확히 똑같은 기능을 수행했다.
그것은
정치조직이라는 주제에 관하여 수시로 형성되고 있던 관념들에
사뭇 잘 들어맞는
일군의 용어와 문장들을 제공했다.
원초적 계약의 교리는,
비록 부당한 것이지만,
휴얼\latin{William Whewell} 박사의
찬사보다 더 높은 찬사를 받을 수는 없을 것이다.
``그것은 도덕적 진리를 표현하는 \hemph{유용한} 형식일 것이다.''\footnote{%
  \latinmarks
  William Whewell,
  \textit{The Elements of Morality Including Polity},
  Vol.\,2,
  London: John W. Parker, 1848,
  p.\,113.
  }

\para{윤리학과 로마법}
우선
정치적 주제에 관한 법적 용어가
원초적 계약의 발명에
광범위하게 사용되어 들어간 것, 그리고
이후 이 가정\hanja{假定}이 강력한 영향력을 행사한 것은
정치학에는
용어와 개념이
왜 그렇게 풍부한가를 넉넉히 설명할 수 있거니와,
그것은 오로지 로마법의 산물이었다.
\wi{도덕철학}\latin{moral philosophy}에서 용어와 개념이 풍부한 것에는
조금 다른 설명이 주어져야 한다.
정치적 사변\hanja{思辨}에 비해
윤리학 저술들에서는 로마법의 기여가
훨씬 더 직접적이었으며,
윤리학 저자들도
그 은혜의 크기를 훨씬 더 잘 알고 있었다.
내가 도덕철학이 로마법에 크게 빚졌다고 말하는 것은
칸트에 의해 도덕철학의 역사에 단절이 일어나기 이전의
도덕철학을 대상으로 하는 것임을 알아야 한다.
그것은 인간의 행위를 규율하는 규칙들과
그 규칙들의 적절한 해석,
그리고 그 규칙들의 한계를
다루는 학문이었다.
비판철학이 등장한 이후
도덕철학은 옛 의미를 완전히 상실했거니와,
로마 가톨릭 신학자들이 여전히 가꾸고 있는
\wi{결의론}\hanjalatin{決疑論}{casuistry}에서
저급한 형태로 보존되어 있는 것을 제외하면,
도덕철학은 거의 예외 없이 존재론의 한 분야로 간주되고 있는 듯하다.
형이상학에 흡수되기 이전의 도덕철학,
규칙 자체보다 규칙의 근본원리가 더 중요하게 고려되기 이전의 도덕철학을
이해하는
현대 영국 학자는
내가 알기로,
휴얼 박사를 제외하면,
한 사람도 없다.
하지만,
오랫동안
윤리학은
실천적 행위준칙을 다루어왔기에,
그것은 어느 정도 로마법에 물들어 있었다.
근대 사상의 다른 모든 주요 분야들과 마찬가지로,
원래
그것은
신학과 한 몸이었다.
처음에는 `\wi{도덕신학}'\latin{moral theology}이라 불렸고
지금도 로마 가톨릭 신학자들 사이에서는 이렇게 불리고 있는
이 학문은 분명, 그 저자들도 잘 알고 있었듯이,
행위의 원리를
교회체계로부터
가져오는 것, 그리고
이를 표현하고 전개하는 데 법학의 언어와 방법을 사용하는 것으로
구성되어 있었다.
이런 과정이 지속되면서,
사고\hanja{思考}의 운송수단에 불과했어야 할 법학이
사고 그 자체에도 자신의 색깔을 전달하는 일이 불가피하게 일어났다.
법개념들과의 접촉에서 얻은 이러한 색조는
근대 세계의 초창기 윤리학 문헌에서 쉽게 감지할 수 있거니와,
생각건대
만약 계약법이 없었더라면
도덕적 의무를
신국\hanja{神國}의 시민의 공적 의무로만
바라보려는
저자들의
경향을,
철저한 상호성과 권리^^b7의무의 확고한 결속에 기초하는 계약법이
건강한 방향으로 교정했음에
틀림없다.
그러나 \wi{도덕신학}에서 로마법이 차지하는 비중은
스페인의 도덕론자들이 이 학문을 키우면서부터
눈에 띄게 줄어들게 된다.\footnote{%
  이른바 살라망카 학파를 말하고 있는 듯하다.
  }
박사들에 의해 주석에 주석이 달리는 법학적 방법으로 발달되던
도덕신학이 자신만의 용어를 스스로 만들어냈다.
또한
학파들의 도덕 논쟁에서 상당 부분 흡수한 것이 분명한
아리스토텔레스적 추론과 표현의 특색들이
로마법에 정통한 사람이라면 결코 틀릴 수 없는
사고와 언어의 특수한 문체를 대신하게 된다.
스페인 학파의 도덕신학자들이 계속해서 신망을 유지했다면
윤리학에서 법학적 요소는 하찮은 수준으로 쪼그라들었을 것이다.
그러나
그들의 영향력은
다음 세대 로마 가톨릭 저술가들이
이 학문 영역에서
그들의 성과를 이용한 방식에 의해
거의 전적으로 파괴되어버렸다.
\wi{결의론}\hanja{決疑論}으로 전락한
도덕신학은
유럽의 사변\hanja{思辨}을 선도\hanja{先導}하는 자들의
관심을 상실했으며,
전적으로 프로테스탄트의 손에 들어간
새로운 \wi{도덕철학}은
\wi{도덕신학}자들이 추종하던 길을 크게 벗어났다.
결과적으로 윤리학에 대한 로마법의 영향은 대폭 증가했다.

\para{그로티우스 학파, 결의론의 쇠퇴}
``종교개혁 이후,\origfootnote{%
  이 인용문은
  1856년 <<케임브리지 논문집>>(Cambridge Essays)에 기고한
  저자의 논문의 일부를 약간의 가필을 거쳐 가져온 것이다. }
이 학문 영역에서는
사상을 달리하는 두 개의 큰 학파 간의 대립이 나타났다.
둘 중 더 영향력 있는 쪽은 애초
\wi{결의론}자\latin{casuist}로 우리에게 알려진 분파 혹은 학파였거니와,
그들 모두는 로마 가톨릭 교회를 신앙하였고,
그들의 거의 모두는 이런저런 가톨릭 수도회에 소속되어 있었다.
다른 쪽은
<<전쟁과 평화의 법>>의 위대한 저자
후고 \wi{그로티우스}의 지적 후예라는 공통점을 갖는
일군의 학자들이었다.
후자 쪽의 거의 모두는 종교개혁의 추종자들이었으니,
그들이 공식적^^b7공개적으로 결의론자들과 갈등하였다고 할 수는 없을지라도,
그들 체계의 기원과 대상은 결의론자들의 것과 근본적으로 달랐다.
이러한 차이는 주목할 필요가 있거니와,
이들 양 체계의 사상 영역에 끼친 로마법의 영향 문제와 관련되기 때문이다.
그로티우스의 저 저서는,
비록 모든 페이지마다 순수한 윤리학 문제를 다루고 있지만,
또한 비록 수많은 공식적인 윤리학 저서의 직^^b7간접적 선조이지만,
주지하듯이 \wi{도덕철학}에 관한 논저임을 자처하지는 않는다.
그것은 자연의 법\latin{law of nature},
즉 자연법\latin{natural law}을 명확히 하려는 시도이다.
자연법이라는 개념이 로마 법학자들의 배타적 창안이었는지의
문제를 따질 필요 없이,
그로티우스 본인이 스스로 인정한 것에 근거하여,
실정법의 어느 부분을 자연법의 일부로 보아야 하는가에 관한
로마 법학의 언명은,
그것이 오류가 아닌 한,
언제나
사뭇 깊은 존경을 받으며 수용되었다고
볼 수 있다.
그리하여 그로티우스의 체계는
로마법과 근본적으로 얽혀있는 것이다.
이러한 연결로 인해 불가피---저자가 법학으로 교육받았던 것의
결과이기도 하겠지만---단락마다
법기술적 용어가 자유자재로 구사되고 있고,
추론과
정의\hanja{定義}와 예시의 방식도 마찬가지이다.
이들이 어디서 유래했는지 출처를 모르는 독자들은 틀림없이,
때로는 그 논증의 의미를 이해하기 어렵고,
거의 항상은 그 논증의 힘과 설득력을 파악하기 어려울 것이다.
다른 한편,
결의론은 로마법에서 빌려온 것이 거의 없고,
무엇이 도덕적이냐의 견해도 그로티우스의 그것과 공통점이 없다.
\wi{결의론}의 이름 아래 유명해진, 혹은 악명높아진, 옳고 그름에 관한 저 모든 철학은
대죄\hanjalatin{大罪}{mortal sin}와
소죄\hanjalatin{小罪}{venial sin} 간의 구분에 기초한다.
어떤 행위를 대죄로 판정하는 끔찍한 결과를 피하려는 자연스런 염려와,
프로테스탄티즘과 대결하고 있는 로마 가톨릭 교회에게서
부담스런 이론의 짐을 덜어주려는, 역시 이해할 만한, 열망에서,
결의론 철학의 저자들은
비도덕적 행위를 가능하면 대죄의 영역에서 제외하여
소죄의 영역에 편입시키려는 복잡한 행위기준의 체계를 발명하게 되었다.
이러한 실험의 결과는 일반 역사의 영역이다.
주지하듯이 결의론의 저 구분은,
사제들의 영적\hanja{靈的} 통제가 다종다양한 인간성에 부응할 수 있도록 만들어,
실로
군주^^b7정치인^^b7장군들에 대한
사제들의
영향력을
종교개혁 이전에는 들어본 적이 없는 수준으로
키워주었으니, 이는
프로테스탄티즘의 초기 성공을 견제하고 축소시킨
저 반\hanja{反}종교개혁에 크게 기여했던 것이다.
그러나 무언가를 세우려는 것이 아니라 피하려는 시도로,
원리를 발견하는 것이 아니라 공준\hanja{公準}을 피하려는 시도로,
옳고 그름의 본성을 정하는 것이 아니라
특정 본성의 무엇이 그르지 않은지를 정하려는 시도로
출발한 결의론은
교묘한 복잡함을 더해간 결과,
행위의 도덕적 성격을 감소시키고
인간의 도덕적 본능을 배반하는 지경에 이르렀으니,
마침내 그것에 거역하여 인류의 양심이 일거에 들고일어나
그 체계와 그 박사들을 모두 공통의 파멸로 몰아넣었다.
오래도록 유예되었던 일격이 \wi{파스칼}의
<<시골 친구에게 보내는 편지>>\latin{Provincial Letters}에서
가해졌다.
이 주목할 만한 저서가 등장한 이후,
조금이라도 영향력이나 신망이 있는 윤리학자라면
자신의 사변을 공공연히 \wi{결의론}에 기초하여 전개할 수는 없게 되었다.
윤리학의 전 영역은 이제 전적으로
그로티우스의 추종자들의 손에 남겨지게 되었다.
그리하여 지금도 윤리학은,
때로는 \wi{그로티우스} 이론의 흠의 원인으로 평가되기도 했고
때로는 그의 이론에 최고의 상찬을 가져다주기도 했던
로마법과의 연루의 흔적을
비상한 정도로 보여주고 있다.
그로티우스 시대 이래 많은 연구자들이 그의 원리를 수정했고,
비판철학의 등장 이후로는 많은 이들이 그의 원리를 포기했지만,
그의 근본 가정\hanja{假定}으로부터 가장 멀리 떠나온 사람들조차
그의 진술 방법, 그의 사고 순서, 그의 예시 방식의 많은 것을
물려받았다.
그런데 이런 것들은 로마법에 무지한 사람들에게는 거의 혹은 전혀
이해될 수 없는 것들이다.''

\para{형이상학과 로마법}
전술했듯이,
자연과학을 제외하면,
형이상학만큼 로마법의 영향을 적게 받은 학문 영역도 없다.
그 이유는 형이상학적 주제의 논의는 언제나 그리스어로 이루어졌다는 데 있다.
정확히 말하면 처음에는 순수한 그리스어로,
나중에는 그리스어 개념을 표현하기 위해 특별히 만들어진 라틴어 방언으로
이루어졌던 것이다.
현대 언어들은 이 라틴어 방언을 채용함으로써,
혹은 그것의 형성기의 과정을 모방함으로써,
비로소
형이상학적 탐구에 적합한 언어가 될 수 있었다.
근대에 들어 형이상학적 논의에 항상 사용되어온 용어의 출처는
라틴어로 번역된 아리스토텔레스였거니와,
그것이 아랍어판을 번역한 것이든 아니든,
번역자의 의도는
라틴어 문헌에서 유사한 표현을 찾는 것이 아니라,
그리스 철학 관념을 표현하는 일군의 용어들을
라틴어 어근으로부터
새롭게 구성하는 것이었다.
이러한 과정에 로마법 용어가 줄 수 있는 영향은 거의 없었다.
기껏해야 몇몇 라틴어 법률용어가 변형된 형태로
형이상학의 언어에 포함되었을 뿐이다.
동시에 언급하고 싶은 점은,
서유럽을 자못 크게 뒤흔든 형이상학의 문제는 어느 것이든,
그 언어는 몰라도,
그 사상은 법학적 기원을 드러낸다는 것이다.
사변\hanja{思辨}의 역사에서 아마도 가장 인상적인 것은,
그리스어를 말하는 민족은
자유의지\latin{free will}와 필연성\latin{necessity}\footnote{%
  `necessity'는 법률용어인 `긴급피난'으로 번역될 수도 있다.
}이라는 중대한 문제로 심각하게 고민해본 적이 없다는 사실일 것이다.
나는 이 문제를 간략하게라도 감히 설명할 생각이 전혀 없다.
그러나 그리스인들도, 그리고 그리스어로 말하고 생각하는 어떤 사회도,
법철학을 생산할 일말의 능력도 보여준 적이 없다는 사실은
이와 무관하지 않다고 생각된다.
법학은 로마인들의 산물이며,
자유의지의 문제는 형이상학적 개념을 법적인 측면에서 숙고할 때 등장하는
문제이다.
어떻게 해서 이 문제가
불변의 사건 연쇄는 필연적\latin{necessary} 관계와 동일한 것인지 어떤지의
문제로 되었는가?
내가 말할 수 있는 것은,
로마법의 경향은,
시간이 갈수록 강해진 그 경향은,
법적 원인과 법적 효과가
흔들림없는
필연성으로 결합된다고
보았다는 것뿐이다.
앞서 인용한 채권채무관계의 정의가 이러한 경향의 현저한 사례이다:
``누군가에게 급부\hanja{給付}를 할 것이 필연적으로 강제되는
법의 사슬''\latin{juris vinculum quo necessitate adstringimur alicujus
solvendae rei}.

\para{교회에서의 로마법}
그런데 자유의지의 문제는
철학이기 이전에 신학의 문제였으며,
그 용어가 법학의 영향을 받았다면
그것은
법학이 신학에 의해 감지되어 받아들여졌기 때문일 것이다.
여기서 내가 제시하는 주요 논점은 한번도 만족스럽게 해명된 적이 없는 것이다.
우리가 확인해야 할 점은,
법학이
신학적 원리에 접근하는 매개체로
기능했는가,
특유의 언어를, 특유의 추론양식을, 여러 세상사에 대한 특유의 해결책을
제시함으로써 법학은 신학적 사변이 흘러나오고 확장되어가는
새로운 통로를 열었는가 하는 것이다.
답을 구하기 위해서는,
초기에 신학이 흡수한 지적 양식\hanja{糧食}이 무엇이었는가에 관해
최고의 학자들 간에 이미 합의된 것을 상기할 필요가 있다.
기독교 교회의 초창기 언어는 그리스어였으며,
기독교 교회가 초기에 대처한 문제들도
후기 그리스 철학이 그 길을 닦아놓았던 문제였음이
널리 인정되고 있다.
신의 위격\hanjalatin{位格}{persons},
신의 본체\latin{substance},
신의 본성\latin{natures} 같은 심오한 논쟁에 인간 정신이 참여할 수 있도록
해주는 언어와 관념의 유일한 창고는
그리스의 형이상학적 문헌들이었다.
라틴어와 소박한 라틴 철학은 이러한 임무를 감당할 능력이 사뭇 모자랐고,
따라서 서방, 즉 라틴어를 사용하는 유럽 지역은
동방의 성과를 따지지도 검토하지도 않고 그대로 받아들였다.
밀만\latin{Henry Hart Milman} 주임사제에 따르면,
``라틴 기독교는 자신의 협소하고 빈약한 어휘로는 적절하게 표현하기 어려운
저 신조를 받아들였다.
그런데 로마와 서방의 동의는 어디까지나
동방 신학자들의
심오한 신학에 의해 형성된 교리체계를 수동적으로 묵인한 것이었을 뿐,
신학적 난제들을 스스로 열성적으로 그리고 독창적으로 검토한 것이 아니었다.
라틴 교회는 아타나시우스\latin{Athanasius}의 제자였으며
충성스런 지지자였다.''\footnote{%
  \latinmarks
  Henry Hart Milman, \textit{History of Latin Christianity},
  London: John Murray, 1854, p.\,61. }
그러나 동로마와 서로마의 분리가 더욱 확고해지고
라틴어를 쓰는 서로마제국이 스스로의 지적 삶을 살기 시작하면서,
동방에 대한 존경은 갑자기
동방적 사변\hanja{思辨}에는 전적으로 생경한
다수의 문제들에 관한 격론으로 변모했다.
``그리스 신학이
\paren{밀만, <<라틴 기독교>>, 서문, 5쪽}
훨씬 세련된 섬세함으로 삼위일체와 그리스도의 본성을 계속 정의해가는 동안''
``끝없는 논쟁이 여전히 길게 이어지고
허약해진 공동체로부터 이런저런 분파들이 계속 분리되어나가는 동안''\footnote{%
  위의 책, p.\,5. }
서방 교회는
새로운 종류의 논쟁들에 열정적으로 뛰어들었으니,
이는 그때부터 지금까지 라틴 교파에 속하는 사람들이라면 한시도
관심을 놓지 않았던 것들이다.
원죄와 그것의 대물림,
인간의 진 빚과 그것의 대속\hanja{代贖},
속죄\latin{Atonement}의 필연성과 충분성,
특히 자유의지와 신의 섭리 간의 표면적 대립관계,
서방은
이런 것들을
동방이 특정한 신경\hanja{信經}의 조항을 두고 논쟁했던 것 못지않게
가열차게 논쟁하기 시작했다.
그렇다면
그리스어를 쓰는 지역과
라틴어를 쓰는 지역 간에
신학적 문제의 종류가 서로 그렇게 달랐던 까닭은 무엇이었을까?
교회사가\hanja{史家}들은
동방 기독교를 갈라놓았던 문제들보다
새로운 문제들이
더 ``실제적인''\latin{practical},
즉 전적으로 사변적이지는 않은 것이었다고 말하여
어느 정도 정답에 가까이 다가갔으나,
내가 아는 한 어느 누구도 정답에 도달하지는 못했다.
나는
두 신학체계 간의 차이는,
동방에서 서방으로 넘어오면서
신학적 사변의 풍토도
그리스의 형이상학에서 로마법으로 바뀌었다는 사실로
설명된다고
서슴없이 주장하고 싶다.
이들 논쟁이 압도적으로 중요한 논쟁으로 부상하기
수 세기 전부터
서로마인들은 그들의 지적 활동을 전적으로 법학에 쏟아부었다.
그들은
세상사가 조합해낼 수 있는 온갖 상황에
특유의 법원리들을
적용하는 일에 몰두해왔다.
다른 어떤 업무나 취미도
그 일에서 그들의 관심을 멀어지게 할 수 없었으며,
그것을 수행하기 위해 그들은
정확하고 풍부한 어휘,
엄격한 추론방법,
경험에 의해 어느 정도 실증된 일반적 행위 명제의 저장고,
그리고 엄정한 \wi{도덕철학}을
보유하고 있었다.
기독교 기록에 나타난 문제들 중에서
그들에게 친숙한 사고 유형에 가까운 것들을
그들이
발견하지 못한다는 것은 불가능한 일일 것이다.
또한 그것들을 취급하는 방식을
그들의 법학적 습관에서 가져오지 않는다는 것도 불가능한 일일 것이다.
로마법에 대한 지식이 충분해서
로마의 형법체계를,
계약과 불법행위로 성립되는 로마의 채권채무관계 이론을,
채무 및 그것을 부담하고 소멸시키고 이전하는 방식에 관한
로마인의 견해를,
포괄적 승계에 의해 개인의 존재가 계속 이어진다는 로마인의 관념을
이해할 수 있는 사람이라면 거의 누구나,
서방 신학의 저 문제들과 잘 어울리는 것으로 드러난 사고의 틀이 어디서 온 것인지,
이들 문제를 진술하는 용어가 어디서 온 것인지,
그 문제의 해결책에 사용된 추론의 유형이 어디서 온 것인지
자신있게 말할 수 있을 것이다.
다만,
서방 사상에 작용하여 들어간 로마법은
옛 로마시의 고법\hanja{古法}도 아니고,
비잔틴 황제들에 의해 잘려나가 축약된 법도 아니며,
그렇다고 근대의 사변적 교리의 기생\hanja{寄生}적 과대성장 속에 거의 파묻힌,
근대 대륙법이라고 불리는 법규칙의 덩어리도 아니었다는
점만은 유념해야 한다.
나는 바로 안토니누스 황조 시대의 위대한 법학자들이
일구어낸 법철학을 말하고 있거니와,
그것은 유스티니아누스의 학설휘찬\latin{Pandects}을 통해 지금도 부분적으로
재구성할 수 있는 것이다.
그 체계의 흠을 굳이 들라면,
인간의 법이 추구할 수 있을 것으로 보이는 한계를 넘어선
고도의 우아함, 확실함, 정확함을
목표로 했다는 점
정도가 아닐까 한다.

\para{로마에서 법학의 우위}
영국인들이 자진해서 고백하는,
때로 부끄러워하기는커녕 자랑스러워하는,
로마법에 대한 무지로 인해,
다수의 저명하고 신망있는 영국 학자들조차
제정기 로마의 지적 상태에 관해 도저히 지지할 수 없는
역설적 주장을 내놓는 특이한 결과가 생겨났다.
아우구스투스 시대가 마감된 때부터
기독교 신앙에 대한 대중적 관심이 일어나기 전까지
문명 세계의 정신적 에너지가 마비상태에 빠졌다는 명제가,
그 명제의 주장에 아무런 무모함도 없다는 듯이,
서슴없이
주장되어왔다.
하지만
인간 정신이 보유한 모든 힘과 능력을 사용할 수 있도록 하는
사고 영역에는 두 가지---아마도 자연과학을 제외하면
이 두 가지뿐일 것이다---가 있다.
하나는 형이상학으로,
인간 정신이 스스로 즐겨 작동하는 한 한계가 없는 영역이고,
다른 하나는 법학으로,
인간사의 일들과 외연을 같이하는 영역이다.
전술한 바로 그 시기 동안,
그리스어를 말하는 지역에서는 전자가,
라틴어를 말하는 지역에서는 후자가,
몰두의 대상이었다.
알렉산드리아와 동방에서의 사변의 결실에 대해서는 모르겠으나,
로마와 서방은
다른 모든 지적 훈련의 부재를 보상하고도 남을 만한
직업 하나를 수중에 쥐고 있었다고
자신있게 말할 수 있다.
또한 우리가 아는 한,
그것이 이룩한 성취는
그것을 만드는 데 들어간 지속적이고도 배타적인 노력에
충분히
값하는 것이었다.
어쩌면
전문 법률가가 아니라면
법학이 흡수할 수 있는 개인의 지적 능력이 얼마나 큰지
완전히 이해할 수 없을지도 모른다.
그러나 일반인이라도
로마의 집단지성 가운데 이례적으로 큰 몫이
어째서 법학에 의해
독점되었는지
어렵지 않게 이해할 수 있을 것이다.
``장기적으로 볼 때,\origfootnote{%
  앞의 1856년도 <<케임브리지 논문집>>. }
어떤 공동체의
법학적 능숙함은
다른 어떤 학문 분야의 진보와도 동일한 조건에 달려있다.
그중 중요한 것은 한 나라의 지식인 중에 거기에 투입되는 비율과
투입되는 시간의 길이이다.
그런데
학문을 진보시키고 완성시키는 데 기여하는
직^^b7간접적인 원인들이 모두 함께,
12표법부터 두 제국의 분리에 이르는 기간 내내
로마의 법학에 대해 작용하였으니,
그것은 불규칙적이거나 간헐적이 아니라
꾸준히 힘이 증가하고 지속적으로 수가 많아지는 양상이었다.
초창기의 지적 훈련이 법의 연구에 바쳐지고 있는 젊은 나라를 상상해보라.
일반화를 위한 의식적 노력이 행해지면서,
일상생활의 관심은 가장 먼저
그것을 일반적 규칙과 포괄적 공식에 포섭하는 것이 된다.
젊은 공동체의 모든 에너지가 바쳐지고 있는 이 분야의 인기는
처음에는 무제한적이다.
하지만 시간이 흐르면서 그것도 시들해진다.
법학이 인간 정신을 독점하는 상황도 깨져간다.
위대한 로마 법학자의 대기실에 아침부터 몰려들던 고객들도 줄어든다.
영국의 법조원\hanjalatin{法曹院}{inns of court}의 학생 수도
수천명대에서 수백명대로 줄어든다.
예술, 문학, 과학, 정치가 그 나라의 지식인 중의 일정 몫을 가져간다.
법실무는 전문가 그룹 내의 것으로 국한된다.
그러나 쪼그라들거나 하찮은 것이 되지는 않거니와,
보수\hanja{報酬}의 측면에서도 그들의 학문의 고유한 매력의 측면에서도
여전히 사람들을 끌어들인다.
이러한 변화의 과정은 영국보다 로마에서 더 현저하게 나타났다.
공화정 말기에 이르면
군대를 통솔하는 특별한 재능을 제외하면
모든 재능 있는 사람들은
법학을 공부한다.
그러나,
영국의 엘리자베스 1세 시대가 그러했듯이,
아우구스투스 시대와 더불어
지성의 진보는 새로운 단계를 맞이한다.
주지하듯이 시와 산문에 있어 그 시대의 업적은 대단했지만,
장식용에 불과한 문학의 번영 외에도
자연과학을 정복하려는 새로운 경향도 막 등장하려 했음에
유의해야 한다.
하지만 이 시기는 로마 국가의 정신의 역사가
그후 추구되어온 정신 진보의 일반적인 경로와 달라지는 시기였다.
이른바, 그러나 정확한 묘사인, 로마 문학의 짧은 수명은
여러 가지 요인으로 갑자기 종말을 맞았거니와,
여기서 그 요인들을 분석하는 것은,
비록 부분적으로 추적가능하다 할지라도,
적합치 않을 것이다.
고대 지식인들은 급격히
옛 상태로 되돌아갔고,
로마인들이 철학과 시를 유치한 민족의 장난감으로 경멸하던 시절만큼이나
배타적으로 법학이 다시
재능 있는 사람들에게 적합한 영역으로 각광받았다.
제정기 동안,
법학 분야에 적합한
내적 능력을 가진 사람들을 끌어들인
외적 요인을 이해하기 위해서는
그의 앞에 놓인 직업의 선택지를 생각해보는 것이
가장 적절할 것이다.
그는 수사학 교사,
전선의 사령관,
또는 온갖 찬사를 쏟아내는 전문 작가가 될 수 있었다.
하지만 그에게 열려있는 그밖의 활동영역으로는
법실무에 종사하는 것이 유일했다.
\hemph{이것}을 통하여 그는
부, 명예, 관직에 접근할 수 있었고,
황제의 자문단\latin{council chamber}---어쩌면 황제의 자리 자체---에도
오를 수 있었다.''

\para{서방 신학에서의 로마법}
법학이 갖는 장점이 그렇게 컸기에
제국의 모든 지역에, 심지어 형이상학이 번성한 지역에도,
법학교들이 존재했다.
비록 황제의 거처가 비잔티움으로 옮아가
동방에서 법학이 발달할 뚜렷한 계기가 되었음에도,
법학은 거기서 경쟁관계에 있는 다른 학문들을 결코 몰아내지 못했다.
법학의 언어는 라틴어였으니,
제국의 동부에서는 외래 방언이었던 것이다.
오직 서방에서만
법학이 야심과 포부를 가진 사람들의 정신적 양식이었을 뿐만 아니라
지적 활동의 유일한 자양분이기도 했다.
로마의 식자층 사이에서는 그리스 철학이
일시적인 유행 이상의 것이 되지 못했다.
동방에 새로운 수도가 건설되고 그후 제국이 둘로 갈라지자,
서방은
그리스적 사변으로부터
더없이 결정적으로
결별하게 된다.
이제 그리스의 문하생에서 벗어나
독자적으로 신학을 궁구하기 시작하자,
그들의 신학은 법적인 관념에 물들고 법적인 용어로 표현되었다.
확실히 서방 신학에서 이러한 법학적 토대는 대단히 깊은 것이다.
그후 아리스토텔레스 철학이라는 새로운 그리스적 이론이
서방에 유입되었고 서방의 고유한 원리들을 거의 전부 덮어버렸다.
그러나
종교개혁 이후 서방은
그것의 영향력을
부분적으로 떨쳐버렸고,
그 자리에 즉각 법학을 가져다 앉혔다.
칼뱅\latin{Calvin}의 종교체계와
아르미니우스파\latin{Arminians}의 종교체계 중
어느 것이 더 법학적 성격이 강한 지는 판가름하기 어렵다.

\para{계약법과 봉건제도}
로마인들이 생산한 이러한 계약법이
근대 계약법에 끼친 막대한 영향력은
성숙한 법학의 역사에 해당하므로
본 논저의 논의대상을 벗어난다.
그것은
볼로냐 대학이 근대 유럽 법학의 기초를 다지면서
비로소
감지되기 시작했다.
그러나
제국이 몰락하기 전에 이미
로마인들에 의해
계약 개념이
완전히 발달했다는 사실은
그보다 훨씬 이른 시기에
중요한 의미를 갖게 된다.
누차 강조했듯이
\wi{봉건제}도는 옛 만족\hanja{蠻族}들의 관습과 로마법이
결합한 것이었다.
다른 설명은 지지될 수 없거나 심지어 이해조차 불가능하다.
봉건시대 초창기의 사회 형태는
원시 문명의 사람들이 어디서나 보여주는
결합의 형태와 별반 다르지 않았다.
봉건관계는 일종의 유기적으로 완전히 결합된 동료관계로서,
재산적 권리와 신분적 권리가 불가분 혼재되어 있었다.
그것은 인도의 \wi{촌락공동체}와 많은 공통점을 가지며,
스코틀랜드 산악지대의 씨족과도 많은 공통점을 가진다.
그러나 그것은 여러 문명의 초기에 자발적으로 형성된 결합관계와는 다른
특수한 성질도 가진다.
실로 원시적 공동체는 명시적 규칙이 아니라 감정에 의해,
아니 어쩌면 본능에 의해 결합된다.
또한 동료관계에 새로 들어오는 자는
이러한 본능에 부합하게
짐짓
자연적 혈연관계를 공유한다고 내세움으로써 편입되는 것이다.
그러나 초창기의 봉건적 공동체는 단순한 감정에 의해
결합되는 것도 아니었고
의제\latin{fiction}에 의해 충원되는 것도 아니었다.
그들을 결속시키는 것은 계약이었으니,
그들은 계약을 맺음으로써 새로운 성원을 얻었던 것이다.
주군과 가신의 관계는 원래 명시적 계약을 통해 설정되었다.
\hemph{\wi{충성서약}}\latin{commendation}이나
\hemph{\wi{수봉}}\hanjalatin{受封}{infeudation}을 통해
동료관계에 편입되려는 자는
그가 받아들여지는 조건을 분명히 알 수 있었다.
따라서 \wi{봉건제}도가 원시 민족들의 순수한 관행과 다른
주된 차이점은 계약이 차지하는 부분에 의해서인 것이다.
주군은 가부장의 성격을 다분히 가지고 있었으나,
그의 대권\hanja{大權}은
수봉시 합의된 명시적 조건에서 기인하는 다양하게 설정된 관습에 의해
제한되었다.
그리하여 봉건사회를 진정한 원시 공동체로 분류할 수 없는
주요 차이들이 발생하게 된다.
봉건사회는 훨씬 더 지속적이었고 훨씬 더 다양했다.
명시적 규칙은 본능적 습관에 의해 파괴되기 어렵다는 점에서
그것은
훨씬 더 지속적이었다.
봉건사회의 기초인 계약은
세부적인 상황에 따라
그리고
토지를 맡기거나 양여하는 자의 원하는 바에 따라
얼마든지 달라질 수 있다는 점에서
그것은
훨씬 더 다양했다.
이 마지막 점은
근대 사회의 기원에 관한 오늘날 우리의 통속적인 견해가
얼마나 잘못된 것인지를 알려주는 데 도움이 될 수 있다.
근대 문명의 불규칙하고 다양한 모습이
게르만 민족들의 지나치게 많은 변칙적인 풍속 탓이라고 하면서,
이를 지루하리만치 틀에 박힌 로마제국의 그것과 대비시키는 일이 흔히 있다.
그러나 진실은
로마제국이 이 모든 불규칙성의 원인인 저 법개념을
근대 사회에 물려주었다는 데에 있다.
만족\hanja{蠻族}들의 관습과 제도들의 가장 두드러진 특징 하나를 들자면,
그것은
그것들이 무척 단조로웠다는 것이다.


\chapter{불법행위법 및 형법의 초기 역사}

\para{고대 법전에서의 형법}
앵글로색슨 선조들의 법전을 포함한
튜턴족의 법전들은
초기의 법들 간의 비중을 정확히 알 수 있는 상태로
우리에게 전해지는
유일한 원시 세속법 법전들이다.
로마와 그리스 법전들의 현존하는 단편들은
그것들의 일반적 성격을 알려주기에는 충분하지만,
그 부분들 간의 정확한 양이나 비율을 파악하기에는 부족한 상태로 남아있다.
그럼에도 불구하고 전반적으로 보아
우리에게 전해지는 모든 고대법 집성들은
성숙한 법체계와 크게 다른 한 가지 특징을 가지고 있다.
민법 대비 형법의 비율이 대단히 큰 차이를 보이는 것이다.
게르만 법전들에서는 형법에 비해 민법 부분의 비중이 아주 작다.
드라콘의 법전이 규정한 포악한 형벌에 관한 전승\hanja{傳承}을 보더라도
이것 또한 마찬가지 성격이었던 것으로 보인다.
뛰어난 법적 재능과 애초 부드러운 풍속을 가졌던 사회의 작품인
12표법만이 유일하게
근대법과 비슷한 정도로 민법의 우위를 보여주지만,
불법행위의 구제방식이 차지하는 상대적 비중이
아주 크지는 않더라도 상당히 큰 편이다.
생각건대
오래된 법전일수록 형법이 더 풍부하고 더 상세하다고
주장할 수 있겠다.
이 현상은
처음으로 그들의 법을 성문화한 공동체들에 만연했던 폭력때문이라고
흔히
인식되어왔고 설명되어왔거니와,
대체로 정확하다고 할 수 있다.
입법자들은
그들 법전의 부분들의 비율을
미개한 생활에서 발생하는 특정 종류의 사건들의 빈도에 맞추었다는 것이다.
하지만
이런 설명은 불완전하다고 생각한다.
옛 집성들에서 민법이 상대적으로 황무지인 것은
이 논저에서 다룬 고대법의 다른 성격들과 밀접히 관련된다는 점을 기억해야 한다.
문명사회의 민법 부분의 십중팔구는
신분법, 물권 및 상속법, 그리고 계약법으로 이루어져있다.
그러나
이들 법분야는
사회적 결속의 유년기로 거슬러올라갈수록
더욱 좁은 범위로 축소될 수밖에 없음이 명백하다.
\wi{신분법}\latin{law of status}에 다름아닌 인법\hanjalatin{人法}{law of persons}은
모든 형태의 신분이 가부장권에 함께 복속해 들어가있는 한,
아내가 남편에 대해,
아들이 아버지에 대해,
미성숙의 피후견인이 종친\hanja{宗親}인 후견인에 대해
아무런 권리도 갖지 않는 한,
아주 좁은 범위로 축소될 것이다.
마찬가지로
물건과 상속에 관한 법도
부동산과 동산이 가족 내에서 대물림되는 한,
설령 분배되더라도 가족 내에서 분배되는 한,
결코 풍부할 수 없을 것이다.
그러나 고대 민법의 가장 큰 빈틈은
언제나 계약법의 부재에 기인할 것이다.
몇몇 원시 법전은 계약법이 아예 없고,
다른 법전들에서는
선서\hanjalatin{宣誓}{oath}에 관한 복잡한 법이 계약법을 대신하고 있어서
계약 관련
도덕관념의 미성숙을 보여주고 있을 뿐이다.
그런데 이에 상응하는,
형법의 빈곤을 가져올 만한 이유는 없다.
따라서,
설령 제 민족들의 유년기가 언제나 무제약적 폭력의 시기였다고 말하는 것이
위험하다 할지라도,
왜 근대법의 민법 대비 형법의 관계가 고대 법전에서는 역전되는 것인지
우리는 여전히 이해할 수 있는 것이다.

\para{범죄와 불법행위}
나는 후대에 비해 원시법이 형법,
즉 \hemph{범죄}법\latin{\textit{criminal} law}에
큰 비중을 둔다고 말했다.
이 표현은 편의상 사용한 것이고,
실은
고대 법전을 살펴보면
비상한 양을 차지하는 저 법이
진정한 범죄법은 아님이 드러난다.
문명사회의 법은
국가나 공동체에 대한 침해와
개인에 대한 침해를 구분하는 데 일치하고 있다.
이렇게 구분된 두 종류의 침해를 우리는,
법학에서 항상 일관되게 이들 용어가 사용되고 있다고 자신할 수는 없지만,
범죄\latin{crimes; \textit{crimina}}와
불법행위\latin{wrongs; \textit{delicta}}라고
부를 수 있을 것이다.
그렇다면 고대 공동체의 형법은
범죄법이 아니라,
불법행위\paren{영국법 용어로는 토트\latin{torts}}법인 것이다.
피해자는 가해자를 상대로 통상의 민사소송을 제기하고,
승소하면 금전배상의 형태로 전보\hanja{塡補}받는다.
\wi{가이우스}의 주해서 중에
12표법에 기초한 형법을 취급하는 부분을 펼쳐보면,
로마법이 인정하는 민사 불법행위의 첫머리에
\hemph{\wi{절도}}\latin{furtum}가 나오는 것을 볼 수 있다.
우리가 익히 \hemph{범죄}로만 취급된다고 여기는 침해가
\hemph{불법행위}로만 취급되고 있는 것이다.
절도뿐만 아니라
폭행 및 모욕\latin{assault}\footnote{%
  저자는 로마법상의 인격침해(iniuria)를 영어의 `assault'로 옮기고 있는 듯하다.
  로마법의 `인격침해'는 신체적 폭행뿐 아니라
  모욕, 명예훼손 등 인격적 침해도
  포괄하므로 본문에서는 `폭행 및 모욕'으로 번역했다.
}과 강도도
로마 법학자들은 영국법상의 불법침해\latin{trespass},
문서명예훼손\latin{libel}, 구두명예훼손\latin{slander}과
마찬가지로 취급한다.\footnote{%
  이들 세 가지 영미법상의 법개념들은 모두
  불법행위(tort)에 속하는 소송형식들이었다.
  이 가운데 불법침해는 신체와 부동산과 동산에 대한 침해를 포괄한다. }
이들 모두가 채권채무관계, 즉 `법의 사슬'\latin{vinculum juris}을
가져오고, 이들 모두가 금전배상으로 전보되는 것이다.
하지만 이 특징은 게르만 부족들의 법전에서 더욱 뚜렷하게 나타난다.
예외 없이 그것들은
살인에 대한 금전배상의 방대한 체계를 기술하고 있고,
거의 예외 없이
기타 덜 중대한 침해에 대한 방대한 배상체계를 기술하고 있다.
켐블\latin{John Mitchell Kemble} 씨에 따르면
``앵글로색슨법에서는 \paren{<<앵글로색슨>>, 1.177}
모든 자유인의 생명에 그의 신분에 따라 금액이 매겨져있었다.
또한 사람의 신체에게 가해질 수 있는 모든 상해에 대해,
그리고 그의 시민권, 명예, 평온에 대해 가해질 수 있는 거의 모든 침해에 대해
그에 상응하는 금액이 매겨져있었다.
그 금액은 우발적인 상황에 따라 가중된다.''\footnote{%
  \latinmarks
  John Mitchell Kemble,
  \textit{The Saxons in England: A History of the English Commonwealth
  Till the Period of the Norman Conquest},
  Vol.\,1,
  London: Longman, Brown, Green, \& Longmans, 1849,
  pp.\,276f.}
이들 배상금은 중요한 수입원이 되었을 것이 분명하고,
매우 복잡한 규칙이 그것에 대한 권리와 책임을 규율하고 있거니와,
전술했듯이, 귀속되는 사람의 사망으로 그것이 면책되지 않는다면,
어떤 특정한 상속규칙에 따라 상속되는 것이 일반적이다.
따라서,
국가가 아니라 개인을 피해자로 보는 것이
\hemph{불법행위}법의 기준이라면,
법의 유년기에는
형법이 아니라
불법행위법에 의존하여
시민들이
폭력이나 사기로부터 보호받았다고 주장할 수 있는 것이다.

\para{불법행위와 종교적 죄}
그리하여 원시법에서 불법행위는 방대한 양을 차지하고 있다.
또한 종교적 죄\latin{sin}도 원시법에 알려져있었음을 첨언해야 할 것이다.
튜턴족 법전들에 관해서는 이런 주장을 굳이 할 필요도 없거니와,
현존하는 이들 법전은 기독교도 입법자들에 의해
편찬되고 개정되었기 때문이다.
그러나 사실 비\hanja{非}기독교적인 옛 법전들도
일정한 작위 유형과 일정한 부작위 유형에 대해
신의 지시와 명령을 위반했다는 이유로 형벌을 부과한다.
아테네의 \wi{아레오파고스}\latin{Areopagus} 원로회의가 집행한 법은
아마도 어떤 특별한 종교법전이었을 것이다.
로마에서도 일찍이
신관\hanjalatin{神官}{pontifical}법이
간통, 독신\hanja{瀆神}, 그리고 어쩌면 살인도
처벌했던 것이다.
따라서 아테네와 로마 국가에서는 \hemph{종교적 죄}를 벌하는 법이 있었다.
또한 \hemph{불법행위}를 벌하는 법도 있었다.
신에 대한 침해라는 개념이 전자의 법을 만들었고,
이웃에 대한 침해라는 개념이 후자의 법을 만들었다.
그러나 국가 또는 전체 공동체에 대한 침해라는 관념은
초기에는 진정한 의미의 형법을 만들지 못했다.

\para{범죄의 개념}
그렇다고 해서
국가에 대한 침해라는 간단하고도 기초적인 개념이 원시사회에
부재했다고 생각해서는 안 된다.
그보다는,
이 개념이 실현되는 특별한 방식이
초기에 형법의 성장을 가로막는 원인이었다고 보인다.
여하튼
로마 공동체는 자신이 침해당했다고 생각되면
개인이 침해당한 경우를 유추하여
완전히 똑같은 결과를 강제했거니와,
국가는 어떤 특별한 행위로써 개인 침해자를 응징했던 것이다.
즉, 로마 공동체의 유년기에는
국가의 안전과 이익을 중대하게 침해하는 모든 행위는
입법기관의 개별 \wi{입법}에 의해 처벌되었다.
그리고 이것이 범죄\latin{crimen}의 초창기 개념이었거니와,
국가가 사건을 민사법원이나 종교법정에 맡기지 아니하고
침해자에 대한 특별법\latin{privilegium}을 만들어 처벌할 정도로
중대한 문제를 야기하는 행위가
바로 범죄인 것이다.
따라서 모든 기소는
`\wi{처벌법안}'\latin{bill of pains and penalties}의
형태를 띠었고,
\hemph{범죄자}의 재판은
정해진 규칙이나 정해진 절차로부터 전적으로 독립된
전적으로 특별하고 전적으로 비정규적인 절차였다.
결과적으로
재판을 담당하는 법원이 통치자인 국가 자신인 까닭에,
또한
미리 어떤 행위 유형을 지시하거나 금지하는 것이 불가능한 까닭에,
이 시대에는 형\hemph{법}이 존재하지 않았던 것이다.
그 절차는 통상적으로 법률이 통과되는 형태와 동일했다.
동일한 사람들에 의해 발의되었고
동일한 엄숙한 절차에 의해 진행되었다.
나중에 법원 및 사법관의 조직을 갖춘 정규의 형사법이
들어서고 나서도,
옛 절차는,
이론상 모순될 것이 없다는 점에서 짐작할 수 있듯이,
여전히
가용한 상태로
엄연히
남아있었음을 유념해야 한다.
이러한 수단을 사용하는 것이 다분히 경원시되었음에도,
로마 인민은
이 권한을 항상 보유했고 이를 이용해
특별법으로 \wi{대역죄}를 처벌하곤 했다.
고전학자들에게는
정확히 동양\hanja{同樣}으로
아테네의 `\wi{처벌법안}'인
\wi{에이산겔리아}\greek{εἰσαγγελία}가
정규 법원의 설치 이후에도 계속 유지되었음을 굳이 상기시킬 필요가 없을 것이다.
또한
주지하듯이
튜턴족의 자유민들이 입법을 위해 집회했을 때
그들은
특별히 사악한 범죄
또는
높은 신분의 범죄자가 저지른 범죄를 처벌하는 권한도 행사했다.
앵글로색슨의 \wi{위테나게모트}\latin{witenagemot}\footnote{%
  `현자(賢者)들의 모임'이라는 뜻. 앵글로색슨 왕국들에서
  일종의 국왕평의회(curia regis)에 해당하는 기구였다.
}의 형사 재판권도
동일한 성격을 가졌다.

\para{중재로서의 재판}
내가 주장한 고대 형법관과 근대 형법관 사이의 차이가
말로만 존재하는 것이 아닐까 생각될 수도 있을 것이다.
가령 공동체는,
입법을 통해 범죄를 처벌하는 것 외에도,
일찍부터
자신의 법원을 통해
위법행위자에게 피해 배상을
강제해왔으며,
이렇게 할 경우
그의 행위로 공동체가 침해받았다고
어떤 식으로든
가정하지 않으면 안 된다고
말할 수 있는 것이다.
그러나,
이런 추론이
현대인에게
아무리 그럴듯해 보이더라도,
먼 옛날 태곳적 사람들이
실제로
이런 추론을
했을까는 대단히 의문스럽다.
공동체에 대한 침해라는 관념이
\hemph{자신의 법원을 통한}
초기 국가의 개입과
얼마나 무관한지는,
초창기 재판 절차가
사적 영역에서 분쟁하고 있는,
그러나 이후 그 분쟁을 진정시키는 데 동의하는
사람들이 행할 법한
일련의 행위를 거의 그대로 모방하는 것이었다는
흥미로운 사실에서도
알 수 있다.
정무관은 무심코 불려온 사적 중재인의 행위를 꼼꼼히 흉내냈던 것이다.

\para{법률소송}
이 진술이 그저 상상으로 꾸며낸 말이 아님을 보여주기 위해
이제 나는 그 근거가 되는 증거를 제시하고자 한다.
우리에게 알려진 가장 오래된 소송절차의 하나가
바로 로마의
\wi{신성도금법률소송}\hanjalatin{神聖賭金法律訴訟}{legis actio sacramenti}이거니와,
이것으로부터 모든 후대의 로마 소송법이 발달되어 나오는 것을 볼 수 있다.
\wi{가이우스}가 그 의례절차를 꼼꼼히 기술하고 있다.\footnote{%
  \latin{Gai.\,4.16.} }
일견 무의미하고 기이하게 보이지만,
조금만 주의를 기울이면 그것을 해독할 수 있고 해석할 수 있다.

소송의 목적물이 법정에 나와야 한다.
동산이면, 실제로 가지고 나오면 된다.
부동산이면, 그것의 일부분 또는 견본을 대신 가지고 온다.
예컨대, 토지라면 한줌의 흙덩이, 가옥이라면 벽돌 한 장으로 대신한다.
가이우스가 채택한 예에서는 노예가 소송의 목적물이다.
절차가 시작되면 원고가 막대기 하나를 들고 나서는데,
가이우스가 밝혀놓았듯이 막대기는 창\hanja{槍}을 상징한다.
원고는 노예에게 손을 얹고 다음과 같은 말로 권리를 주장한다.
``말했듯이
이 사람은 \wi{시민법}에 의해 정당한 권원에 따라 나의 것임을
주장하노라\latin{Hunc ego hominem ex jure quiritium meum esse dico
secundum suam causam sicut dixi}.''
그러고는 이어서 ``보라! 이 사람에게 나의 창을
두었노라\latin{Ecce tibi vindidam imposui}''고 말하면서
그에게 창을 갖다댄다.
이어서 피고도 동일한 언명과 몸짓을 수행한다.
그러고나면 \wi{법무관}이 개입하여
당사자들에게 손을 뗄 것을 명한다.
``둘 다 그 사람을 놓아주라\latin{Mittite ambo hominem}.''
당사자들은 명을 따른다.
이어 원고는 피고에게 주장의 근거를 요청한다.
``너는 어떤 권원에서 주장하였는지 진술해줄 것을
요청하노라\latin{Postulo anne dicas
quâ ex causâ vindicaveris}.''
이 질문에 대해 피고는 또 다시 자신의 권리를 주장함으로써 답한다.
``나의 창을 두었거니와 나는 나의 권리를 행사하였노라\latin{Jus peregi
sicut vindictam imposui}.''
그러면 원고는 이 사건 재판에 대해
`신성도금'\latin{sacramentum}이라 불리는 일정액의 금전을 걸자고 제안하다.
``너는 불법적으로 주장하였므로,
나는 500아스의 신성도금으로 너에게
도전하노라\latin{Quando tu injuriâ provocasti, D aeris sacramento te provoco}.''
피고는 ``나도 너에게 똑같이 도전하노라\latin{Similiter ego te}''는 말로
내기\latin{wager}를 받아들인다.
이후의 절차는 더 이상 요식적 성격을 띠지 않는다.
그러나 법무관은 신성도금 명목으로 보증금을 받았고,
이 돈은 언제나 국고에 귀속되었다는 점을 명심해야 한다.

\para{호메로스가 묘사한 고대 소송}
고대 로마의 모든 소송은 반드시 이러한 절차로 시작했다.
생각건대
재판의 기원이 드라마 형태로 각색되어 있는 것을
여기서
볼 수 있다는 데
동의하지 않을 수 없을 것이다.
두 명의 무장한 사람들이 분쟁 대상을 두고 다투고 있다.
큰 자비\hanja{慈悲}의 소유자\latin{vir pietate gravis}인 \wi{법무관}이 마침
그곳을 지나가다가 분쟁을 해결하기 위해 개입한다.
분쟁당사자들이 각자 자신의 사정을 말하고,
법무관이 중재해 줄 것에 동의한다.
즉, 패자\hanja{敗者}는 분쟁의 목적물만 잃는 것이 아니라
중재인에게 그의 노고와 시간에 대한 보답으로
일정액의 금전도 지불하기로 합의하는 것이다.
이런 해석은,
가이우스가 법률소송\latin{legis actio}의 필수과정으로 묘사한
의례절차가
헤파이스토스 신이
아킬레우스의 방패의 첫 번째 부분에 새겨넣은 것으로
\wi{호메로스}가 묘사한
두 가지 주제 중 하나\footnote{%
  아킬레우스의 방패에는 두 개의 도시가 새겨졌으니,
  한 도시에는 결혼식 장면과 재판 장면이 그려졌고,
  다른 도시에는 전쟁 장면이 그려졌다고 한다.
  첫 번째 도시에 대한 묘사는
  \latin{Hom. Il. 18.490.}
}와
놀라울 정도로 일치하지 않았다면,
설득력이 반감되었을 것이다.
\index{일리아스}%
호메로스가 묘사한 재판 장면은,
마치 원시사회의 성격을 드러내고자 의도하였다는 듯이,
물건에 대한 소송이 아니라
살인에 대한 \wi{속죄금} 소송이다.
한 사람은 이미 지불했다고 주장하고,
다른 사람은 받은 적이 없다고 주장한다.
하지만 상세하게 묘사된 부분은
옛 로마의 소송관행의 대응물을 그려내고 있거니와,
판관들에게 주어지는 보답 부분이다.
2달란트의 황금이 가운데 놓여있고,
판결의 근거를 청중들에게 가장 만족스럽게 설명하는 자에게
이것이 주어진다.
이 금액이 신성도금의 사소한 금액에 비해 무척 큰 것은
유동적인 관행과 법으로 고정된 관행 간의 차이를 시사하는 것으로 보인다.
저 시인에 의해
영웅시대의 도시생활의 두드러진 특징으로,
그러나 아직은 임시적인 성격의 것으로
소개된 이 장면은
문명의 역사가 열리면서
정규의 통상적인 소송 형식으로 굳어져간다.
따라서
법률소송에 이르면 판관의 보수는
당연히 합리적인 금액으로 줄어들고,
군중의 환호에 따라 여러 중재인 중 한 명에게 주어지는 대신
당연히 법무관으로 대표되는 국가에 귀속되는 것이다.
그러나
호메로스에 의해 생생하게 묘사된 장면과
가이우스에 의해
흔히 법기술적 언어에서 보이는 것 이상으로 날것 그대로
묘사된 장면은 의심할 여지 없이
사실상 동일한 것을 의미한다.
또한
근대 유럽의 초기 재판관행을 연구한 다수의 학자들은
법원이 법위반자에게 부과한 벌금이 원래는 신성도금\latin{sacramenta}이었다고
말하고 있거니와, 이 또한
우리의 견해를 뒷받침해준다.
국가는
피고인에게서
자신에게 가해진 침해에 대한 배상을
받아낸 것이 아니라,
원고에게 주어지는 배상금의 일정 비율을
단지
시간과 노고를 들인 데 대한 정당한 대가로서
요구했던 것이다.
켐블 씨는 앵글로색슨의
`반눔'\latin{bannum} 또는 `프레둠'\latin{fredum}에
명시적으로
이러한 성격을
부여하고 있다.\footnote{Kemble, 앞의 책, p.\,270.}

\para{로마 고법상의 절도}
초기의 재판관들이 사적 분쟁에 연루된 당사자들 간에 있음직한 행위를
모방했다는 것을 보여주는 다른 증거들도 고대법에서 발견된다.
구제수단을 정함에 있어
그들은 당해 사건의 상황 하에서 피해자가 감행했을 법한 복수\hanja{復讐}의 양태를
감안했던 것이다.
이것은 현장에서 또는 범행 직후에 붙잡힌 범죄자와
상당 기간 경과 후에 발각된 범죄자를
고대법이
사뭇 다르게 처벌했던 점을 올바르게 설명해줄 수 있다.
이러한 차이를 보여주는 다소 특이한 예는 로마 고법\hanja{古法}상의 절도에서 볼 수 있다.
\wi{12표법}은 \wi{절도}를
\wi{현행도}\hanjalatin{現行盜}{manifest theft}와
비\hanja{非}현행도\latin{non-manifest theft}로
구분하였으며,
어느 쪽에 속하는가에 따라 완전히 다른 처벌을 부과했다.
현행도는 절도행위를 하던 중 그 집 안에서 붙잡힌 자,
혹은 훔친 물건을 가지고 안전한 장소로 달아나다가 붙잡힌 자를 말한다.
12표법에는 이런 자가 노예인 경우 \wi{사형}에 처하도록 하였고,
자유인인 경우 그 물건의 주인의 노예가 되도록 정해놓았다.\footnote{%
  \latin{XII Tab.\,8.14.}}
비현행도는 그밖의 상황 하에서 발각된 자를 말한다.
저 옛 법전은 이런 종류의 절도는 훔친 물건의 2배액을 배상한다고만
정해놓았다.\footnote{%
  \latin{XII Tab.\,8.16.}}
당연히
\wi{가이우스}의 시대에는
현행도에 대한 12표법의 가혹함이 상당 부분 완화되었다.
그러나
비현행도의 경우 여전히 2배액만 배상하는 데 비해
\wi{현행도}는 훔친 물건의 4배액을 배상하게 함으로써
\index{비현행도|see{현행도}}%
옛 원리를 계속 유지하고 있었다.\footnote{%
  \latin{Gai.\,3.189--190.}}
고대 입법자는
피해자인 소유주가,
스스로 처벌한다면,
격정에 휩싸여 있을 때 처벌하는 것과
시간이 한참 지나 절도범이 발견될 때 만족할 만한 것 간에
큰 차이가 있으리라는 점을 고려했음이 분명하고,
이에 맞추어 처벌 수위를 조정했을 것이다.
이러한 원리는 앵글로색슨 및 기타 게르만 법전들이 따르는 원리와
정확히 일치한다.
\wi{절도}범을 추격하여 도품과 함께 붙잡은 경우 그들은
그 자리에서 그를 교수형 또는 참수형에 처했던 반면,
추격이 중단된 후에는 살인자라도 살인에 대한 완전한 배상금을 받아내는 데
그쳤던 것이다.
옛 법상의 이러한 구분은 원시적 법과 세련된 법 사이에
간격이 얼마나 큰지 실감케 해준다.
근대의 재판관들은
법기술적으로 동일한 종류에 속하는 범죄들 간에
죄의 경중을 구별하는 일이
그들 업무 중 가장 고역이라고 고백하고 있다.
어떤 사람이 살인죄, 절도죄, 또는 중혼\hanja{重婚}죄의 유죄라고
말하기는 쉬워도, 그가 어느 정도로 도덕적 죄질이 나쁜지,
그리하여
어떤 형벌을 부과하는 것이 적절한지
판단하기란 무척 어렵다.
우리가 이런 사항을 정확히 정해놓으려 시도한다면,
동기를 분석하는 결의론\latin{casuistry}에 빠진다 해도
그다지 잘못이라 할 수 없을 것이다.
따라서 오늘날의 법은
이 문제에 관하여 세세한 실정규칙을 정해놓는 일을
가능하면 피하려는 경향을 보인다.
프랑스에서는
유죄로 판명된 범죄에 참작할 만한 정황이 있는지 결정하는 것이
배심원단에게 맡겨져있다.
영국에서는
형벌을 선택함에 있어 거의 무제한적 재량이
판사에게 주어져있다.
또한 모든 국가에서
법의 오판을 교정하는 수단으로서
어디서나 사면권이 최고 통치권자에게 유보되어 있다.
신기하게도
원시시대의 사람들은 이러한 망설임으로 고민하는 일이 거의 없었고,
전적으로
피해자의 충동이 그가 가할 수 있는 복수의 적절한 기준이라고 생각했으며,
또한 형벌의 척도를 정함에 있어
피해자의 격정의 있음직한 등락을 그대로 모방했던 것이다.
그들의 입법의 방법이 이제 완전히 사라졌다고 말할 수 있기를 나는 바란다.
하지만 몇몇 근대법체계에서는,
중대한 침해의 경우,
가해자가 행위 중에 붙잡혔다는 사실이
피해자가 그에게 가한 과잉된 응징을
정당화\latin{justification}하는 항변 사유로 인정되고 있거니와,
겉보기에는 타당해보일지 몰라도,
이는
저급한 도덕성에 기초한 면죄부라고 나는 생각한다.

\para{로마의 사문회}
전술했듯이
고대사회로 하여금 결국 진정한 의미의 형법을 갖도록 이끈
고려사항은 무척 단순한 것이었다.
국가는 스스로를 피해자로 관념했고,
민회는 입법행위에 수반되는 것과 동일한 절차로
가해자를 직접 타격했다.
또한 실로 고대 세계에서---후술할 기회가 있겠지만,
현대 세계에서는 그렇지 않을 수 있다---초창기 형사 법정은
단지 입법기구의 하위 분과, 즉 위원회에 지나지 않았다.
어쨌든 이것이
법사\hanja{法史}에서 두 개의 위대한 고대 국가들에 관하여,
하나는 그럭저럭 명료하게, 다른 하나는 완전히 확실하게,
드러나는 결론이다.
아테네의 원시 형법은 범죄의 처결을
일부는 \wi{아르콘}\latin{archon}들에게 맡겨
\hemph{불법행위}로서 처벌하게 했고,
일부는 \wi{아레오파고스} 원로회의에 맡겨
\hemph{종교적 죄}로서 처벌하게 했다.
양자의 재판권은 결국 \wi{헬리아이아}\latin{Heliaea}, 즉
최고인민법원에 사실상 넘어갔고,
아르콘의 기능과 아레오파고스의 기능은 행정적인 것에 불과하게 되거나
아니면 거의 무의미한 것이 되어버렸다.
그런데 ``헬리아이아''는 애초에 단지 민회를 지칭하는 말이었다.
고전 시대의 헬리아이아는 사법\hanja{司法}적 목적을 위해
모인 민회였을 뿐이고,
아테네의 저 유명한 \wi{디카스테리온}\latin{dikastery}들은
민회의 하위 분과, 즉 배심원단들이었을 뿐이다.
로마에서 일어난 이와 유사한 변화는
훨씬 더 수월하게 해석될 수 있거니와,
로마인들은 배심원단에 관한 법만 수정했고,
아테네인들처럼 민사재판권과 형사재판권을 포괄하는 인민법원을
만들지는 않았기 때문이다.
로마 형법의 역사는
왕이 주재했다고 전해지는
옛 인민재판\latin{judicia populi}에서 시작한다.
이는 입법의 형태로 중범죄자들을 처벌하는 엄숙한 재판이었을 뿐이다.
하지만
일찍부터
민회\latin{comitia}는 자신의 형사재판권을
\wi{사문회}\hanjalatin{査問會}{quaestio}, 즉 위원회에
위임하곤 했었던 것으로 보인다.
사문회의 민회에 대한 관계는
영국 하원의 위원회가 하원에 대해 갖는 관계와 거의 동일하지만,
다만 로마의 사문관들은
민회에 \hemph{보고}\latin{report}만 한 것이 아니라
민회가 행사하던 모든 권한을 행사했으며,
따라서 피고인에게 판결을 선고할 수 있었다.
이런 종류의 사문회는
특정 범죄자를 재판하기 위해서만 임명되었으나,
두 세 개의 사문회들이 동시에 재판하는 일도 불가능하지는 않았다.
또한
공동체를 위협하는 여러 개의 중대한 범죄사건이 동시에 발생하면,
여러 개의 사문회가 동시에 임명되는 일도 있었던 것으로 보인다.
또한 사문회가
\hemph{상임}위원회의 성격을 띠는 경우도 가끔 있었거니와,
이는 중대한 범죄의 발생을 기다리지 아니하고
정기적으로 임명되는 사문회였다.
아주 오래된 업무집행과 관련하여 언급되는
\index{가부장살해사문관}%
옛 `가부장\hanja{家父長}살해사문관'들\latin{quaestores parricidii}은
모든 존속살해 및 일반 살인 사건의
재판\paren{수사와 재판을 모두 담당했다는 의견도 있다}을
위임받았거니와,
매년 정기적으로 임명되었던 것으로 보인다.
두 명으로 구성되어
국가에 대한 폭력적 위해를 재판하던
`\wi{반역이인관}'\hanjalatin{反逆二人官}{duumviri perduellionis}도
정기적으로 임명되었다고 보는 것이 통설이다.
이들 두 경우의 권한 위임은 우리를 한걸음 더 후대로 데려다준다.
국가에 대해 범죄가 \hemph{이미} 행해진 후 임명되는 것이 아니라,
\hemph{장래} 행해질 경우에 대비해
일시적이기는 해도 일반적인 재판권을 가졌던 것이다.
``가부장살해''\latin{parricidium}와
``반역''\latin{perduellio}이라는 일반적 용어도
정규의 형법에 다가가고 있음을 시사하거니와,
범죄의 분류에 유사한 것에 접근하고 있는 것이다.

\para{상설사문회}
하지만 진정한 의미의 형법은 기원전 149년에 비로소 등장한다.
칼푸르니우스 피소\latin{Calpurnius Piso}에 의해
이른바
`부당착취에 관한 칼푸르니우스 법'\latin{Lex Calpurnia de Repetundis}이
만들어졌던 것이다.
이 법률은 부당착취금회수\latin{repetundarum pecuniarum}소송,
즉 속주총독이 부당하게 수탈한 돈을 반환하라고 속주민들이 제기한
소송에 적용되었다.
그러나 이 법률이 항구적으로 큰 중요성을 갖게 된 것은
최초로 \wi{상설사문회}\hanjalatin{常設査問會}{quaestio perpetua}가 설치되었다는 데 있다.
상설사문회는 일종의 \hemph{상임}위원회로서
임시적인 또는 일시적인 위원회와는 그 성격을 달리했다.
그것은 정규의 형사법원으로서,
이를 창설하는 법률이 통과된 때부터 존재하여
폐지하는 법률이 통과될 때까지 존속하였던 것이다.
그것의 구성원들은
옛 \wi{사문회} 구성원처럼 특별히 임명되는 것이 아니라,
그것을 창설하는 법률의 규정에 의해
특정 신분의 사람들 중에서 직무를 맡을 심판인들이 선발되었고
또한 정해진 규칙에 따라 갱신되었다.
그것이 담당할 범죄들의 이름과 정의\hanja{定義}도
이 법률에
명시적으로 규정되어 있었다.
그리하여 이 새로운 사문회는
장래에
이 법률에 규정된 범죄의 정의에 해당하는 행위를 한
모든 사람들을 재판하고 선고를 내릴 권한을 가졌다.
그것은 따라서
진정한 형법을 집행하는
정규의 형사법원이었다.

\para{형법의 역사}
원시적 형법의 역사는 따라서 네 단계로 구분된다.
\hemph{범죄}의 개념을
\hemph{불법행위}와 구분하고
\hemph{종교적 죄}와도 구분하여
국가 혹은 전체 공동체에 대한 침해로 관념할 때,
이 개념에 글자 그대로 부합하게,
우선
국가는
개별적 행위에 의해 스스로 직접 개입하여
자신을 침해한 범죄행위자에게 보복했다.
이것이 출발점이다.
각각의 기소는 일종의 \wi{처벌법안}이고,
범죄자의 이름과 그에 대한 형벌이 개별 특별법에 의해 정해진다.
\hemph{두 번째} 단계는
다수의 범죄에 대해
입법기구가 자신의 권한을 특별한 사문회 또는 위원회에 위임할 때
달성된다.
각 사문회 또는 위원회는
특정한 하나의 기소에 대해 수사하고,
유죄로 입증되면 그 특정 범죄자를 처벌할 권한을 부여받는다.
\hemph{또 하나의} 단계는
입법기구가,
범죄의 실행을 기다려
사문회를 임명하는 것이 아니라,
가부장살해사문관이나 반역이인관처럼
일정 유형의 범죄가 저질러질 가능성과
\hemph{장래} 그것들이 실행될 것이라는 기대 하에
정기적으로 위원들을 임명할 때에 이루어진다.
\hemph{마지막} 단계는
정기적이거나 임시적이던 \wi{사문회}가
상설적인 판사단\latin{bench} 또는 법원\latin{chamber}이 될 때,
위원회를 임명하는 특별법에 의해 그때마다
심판인들을
지명하는 것이 아니라
장래를 향해 영구적으로
특정한 방식으로 특정한 신분 중에서
심판인들을
선발하도록 정해져있을 때,
일정한 행위가 일반적 언어로 기술되고 범죄로 선언되어 있어서
그 행위가 실행되면 각 유형에 적합한 예정된 형벌로 처벌될 때에
도달된다.

\para{사문회의 결과}
상설사문회의 역사가 더 오래 지속되었다면,
그것은 틀림없이 독자적인 제도로 관념되었을 것이고,
민회에 대한 그것의 관계도
영국의 보통법법원들이
모든 사법권의 이론적 원천인
국왕에 대해 갖는 관계 이상으로
가깝게 생각되지 않았을 것이다.
그러나
\wi{상설사문회}는 그 기원이 미처 망각되기 전에
제정기의 전제정치에 의해
파괴되고 말았다.
그리고 이들 상임위원회가 지속되는 동안
로마인들은
그것을 단지 위임된 권한의 담지자로만 생각했다.
그것이 담당하는 범죄는 본디 입법기구의 권한에 속하는 것으로 여겨졌고,
\wi{사문회}를 떠올릴 때면
로마인들의 머리에는
끊임없이 민회가,
자신의 불가양의 기능의 일부를 행사하도록 위임한 저 민회가
연상되었다.
사문회가
상설화된 이후에도 여전히
그것을 민회의 위원회로만---상위의 권위에 종속된 기구로만---간주하는 견해는
로마 형법의 마지막 시기까지 각인을 남긴 어떤 중요한 결과를 낳았다.
한 가지 직접적인 결과는
사문회들이 설치되고 나서도 오랫동안
민회가
\wi{처벌법안}을 통해
계속
형사재판권을 행사했다는 것이다.
입법기구가 편의상 다른 기구에게 권한을 위임했다고 해서
그 권한을 완전히 양도해버렸다고 할 수는 없다.
민회와 사문회들은 서로 나란히 범죄자들을 재판하고 처벌했다.
그리하여
민중의 분노가 끓어오르는 이례적인 사건이 발생하면,
공화정이 끝날 때까지는,
반드시 그 대상에 대한 기소가
\wi{트리부스 민회}\latin{assembly of the tribes}에 제출되었던 것이다.

\para{민회의 재판권, 사형}
민회에 대한 사문회의 종속성에서
로마 공화정의 또 하나의 중요한 특징 하나를 더
끌어낼 수 있다.
로마 공화정의 형벌체계에서 \wi{사형}이 소멸한 것은
지난 세기 학자들이 사뭇 선호하던 논의주제였다.
그들은 끊임없이 이를 이용해
로마인들의 심성에 관한 이론이나
근대적 사회경제에 관한 이론을 내세웠다.
그러나 자신있게 말할 수 있는 저 현상의 원인은
기실 순전히 우연적인 것이었다.
연속해서 나타난 로마의 입법기구들 중의 하나인
\wi{켄투리아 민회}\latin{comitia centuriata}는 주지하듯이
오로지 군사적 목적을 위한 로마 국가의 대표기구였다.
따라서 켄투리아 민회는
군대를 통솔하는 사령관에게 주어질 법한 모든 권한을 가졌고,
그중에서도
군율을 위반한 군인에게 가해지는 처벌과 동일한 처벌을 범죄자들에게
부과하는 권한을 가졌다.
그리하여 켄투리아 민회는 \wi{사형}을 언도할 수가 있었다.
하지만 \wi{쿠리아 민회}\latin{comitia curiata}나
\wi{트리부스 민회}\latin{comitia tributa}는 달랐다.
이 점에 관하여 이들 민회는
로마시 성벽 안의 시민들의 인격에는
종교와 법에 의해
신성함이 부여되어 있다는 관념에 의해 제약받았다.
이들 중 후자인 트리부스 민회에 관해 말하자면,
트리부스 민회는 기껏해야 벌금형만 부과할 수 있다는 것이
확고한 원칙이 되어갔음을 우리는 확실히 알고 있다.
형사재판권이 입법기구에 국한되어 있는 한,
그리고 켄투리아 민회와 트리부스 민회가
동등한 권한을 계속 행사하는 한,
중형을 과할 수 있는 입법기구에
중대한 범죄를
기소하는 것이
선호되었을 것임은 어렵지 않게 짐작할 수 있다.
그러나
보다 민주적인 기구, 즉 \wi{트리부스 민회}가
거의 전적으로 다른 민회들을 대체하여
공화정 후기의 통상적인 입법기구가 되는 일이 발생했다.
또한
공화정의 몰락기는
저 상설사문회들이 창설되던 시기와 정확히 일치한다.
그리하여
\wi{상설사문회}를 창설하는 법률들은 모두
정상적인 집회로써는
범죄자에게 사형을 부과할 수 없는 입법기구에 의해 통과되었다.
결과적으로,
위임받은 권한만 갖는
상설 사법위원회들의
속성과 능력은
위임하는 기구가 갖는 권한의 한계에 의해 제한되었다.
트리부스 민회가 할 수 없는 것은
그것들도 할 수가 없었고,
\wi{트리부스 민회}는 \wi{사형}을 과할 수 없었으므로
사문회들도 똑같이 사형을 과할 수 없었다.
이렇게 해서 생겨난 특별한 상황을 바라보는
고대인들의 관점은
근대인들 사이에 인기있는 어떤 관념과 전혀 다른 것이었다.
실로,
로마인들의 심성이 그것에 더 부합했는지는 모르겠으나,
로마의 헌정에는 확실히 훨씬 더 나쁜 결과를 가져왔다.
역사적으로 인류가 경험했던 다른 모든 제도들과 마찬가지로,
사형도 문명화과정의 일정 단계에서는 사회의 필수요소가 된다.
어떤 시대에는
사형을 없애려는 시도가
형법의 근저에 놓여있는 두 가지 중대한 본능을 거스르게 된다.
사형이 없이는,
공동체는 범죄자에게 충분히 복수했다는 느낌도 갖지 못하고,
그에 대한 처벌의 사례를 가지고
그를 모방하고자 하는 다른 사람들을 억지시킬 수 있다는 느낌도
갖지 못하는 것이다.
\wi{사형}을 언도할 수 없는 로마 법원의 무능력은
명백하게 그리고 직접적으로
`\wi{추방}'\hanjalatin{追放}{proscription} 시대라고 불리는
저 가공스런 혁명적 시간들을 낳았다.\footnote{%
  여기서 `추방'이란 법의 보호를 박탈(outlaw)하는 것을 의미한다.
  추방자 명부에 등재되면 재산도 몰수당하고 생명도 언제 누구에게 빼앗길지
  모를 상태에 놓인다.
  본문은 기원전 82년 독재관 술라에 의한 대규모 숙청과
  기원적 43년 옥타비아누스, 안토니우스, 레피두스의 삼두체제에 의한
  (키케로를 포함하는) 대규모 숙청을 말하고 있는 듯하다. }
당시
복수에 목말라하던
당파적 폭력이
어떤 다른 출구도 찾지 못했다는 단순한 이유로
모든 법이 공식적으로 정지되었던 것이다.
법의 이러한 일시적 정지상태보다 더 강력하게
로마 인민의 정치적 능력을 몰락하게 만든 것은 없었다.
일단 이런 사태를 겪고 나자
로마 공화정의 몰락은 이제 시간 문제에 불과하게 되었다고
우리는 주저없이 주장할 수 있다.
민중의 감정이 분출될 수 있는 적절한 출구를
로마 법원이
제공했다면,
사법과정의 모습은 분명
스튜어트 왕조 후기의 영국이 경험했던
도착\hanja{倒錯}적인 형태를 노골적으로 띠었겠지만,
국민성은 그렇게 철저히 손상받지 않았을 것이며
제도의 안정성도 그렇게 심각하게 허약해지지 않았을 것이다.

\para{사문회의 결과, 범죄의 분류}
재판권에 관한 전술한 이론에 의해 생겨난
로마 형사사법체계의 특징을 두 가지 더 언급하고자 한다.
하나는 로마에서는 형사법원의 숫자가 대단히 많았다는 것이고,
또 하나는
범죄의 분류가 변칙적이고 부조화스러웠다는 것으로, 이는
로마 역사 전체에 걸쳐 형법의 성격을 규정했다.
상설이든 아니든, 모든 \wi{사문회}는
각각 개별적인 법률에 의해 탄생한다.
각 사문회는 그것을 만든 법률에 의해 권위가 부여되었고,
그 법률이 정한 한계를 엄격히 준수해야 했으며,
그 법률이 명시적으로 정의하지 않은 범죄 형태는 결코 취급할 수 없었다.
그런데 다양한 사문회를 만든 법률들은
모두 특정한 위기상황에 대응하는 것이었고,
각각은 실로 당시의 상황에서
특별히 혐오스럽거나 특별히 위험하다고 여겨진
일군의 행위를 처벌하기 위해 통과되었기에,
이들 제정법들은 조금도 서로를 고려하지 않았고,
따라서 공통의 원리에 의해 통합되어 있지 못했다.
20 내지 30개의 서로 다른 형법들이 공존하고 있었거니와,
정확히 같은 숫자의 사문회들이 난립하고 있었다.
또한 공화정 시기에는
이들 개별 사법조직들을 하나로 통합하거나,
이들을 창설하고 이들의 책무를 정한 법률조항들을 조화시키려는
시도가 전혀 행해지지 않았다.
당시 로마의 형사사법의 상태는
영국의 보통법법원들이
서로 간에 다른 법원의 관할을 넘나들 수 있도록 하는,
영장\latin{writ}에 기입하는
의제적 사실 진술\latin{fictitious averment}이
도입되기 이전\footnote{%
  본서 제2장 \hyperlink{commonlawfiction}{의제의 용도} 부분 참조.}
영국의 민사 구제수단들의 사법체계와
비슷한 면을 보이고 있었다.
사문회들처럼,
왕좌법원과 민소법원과 재무법원도 모두
이론적으로
상위의 권위에서 권위가 유래하며,
그 상위의 권위에 의해 재판권이 부여된
특별한 사건군\hanja{群}을 각각 관장하고 있었던 것이다.
그러나 로마의 사문회들은 그 수가 세 개보다 훨씬 많았고,
각 사문회가 관장하는 행위군을 구분하는 것은
웨스트민스터 홀의 세 법원 간에 관할을 구분하는 것보다
훨씬 더 어려웠다.\footnote{%
  엄밀히 말하면, 웨스트민스터 궁의 웨스트민스터 홀에서
  개정했던 법원은 왕좌법원, 민소법원, 그리고 형평법법원이었다.
  재무법원은 웨스트민스터 홀이 아니라 인근의 다른 방에서 개정했다.
  }
서로 다른 사문회 간에 정확하게 경계선 긋기가 어려웠던 점은
여러 로마 법원들에게 단순히 불편함을 주는 것 이상의 문제를 낳았다.
범죄로 여겨지는 누군가의 행위가 어떤 일반적 유형에 해당하는지
즉각 명료하지 않은 경우,
서로 다른 여러 위원회들이
동시에 또는 순차적으로 그를 기소할 수 있고,
그중 어느 하나가
그에게 유죄판결을 내릴지도 모른다는 데서
우리는 놀라지 않을 수 없다.
또한 비록 하나의 사문회가 유죄판결을 내리면
다른 사문회의 재판권이 배척된다 하더라도,
하나의 사문회가 무죄판결을 내려도 그것이
다른 사문회의 기소를 저지하는 항변사유가 될 수 없었다.
이는 로마 민사법의 규칙과는 완전히 다른 것이었다.
또한 로마인들처럼 법의 부조화\paren{혹은 그들이 쓰는
의미심장한 용어에 따르면,
\hemph{전아}\hanja{典雅}\hemph{하지 못함}\latin{inelegancy}}에 대해
그렇게 민감한 민족이
이런 상태를 오래도록 방치하지 않을 것이라는 것도
확신할 수 있다.
다만,
범죄를 교정하는 항구적인 제도로
사문회를
보기보다는,
사문회를 둘러싼 우울한 역사로 인해
이를 파벌들이 장악하는 일시적인 무기로 보고 있었을 뿐이다.
제정기의 황제들은 머지않아 다수의 재판권 간의 충돌을 없애버린다.
그러나 그들은 법원의 난립과 밀접하게 연관된 형법의 또 다른 특징은
제거하지 않았다는 데 유의해야 한다.
유스티니아누스의 로마법대전에 이르러서도
범죄의 분류는 무척 변칙적인 것이었다.
사실
각각의 \wi{사문회}는
법률에 의해 수권\hanja{授權}된 범죄들만 취급할 수 있었다.
하지만 이들 범죄는
그 법률의 제정 당시
우연히
동시에 처벌대상이 되었다는 이유로
함께 묶여있었을 뿐이다.
따라서
그것들 간에는 필연적인 공통성이 없었다.
그러나
그것들이 특정한 사문회의 재판 대상이라는 사실은
자연스레 대중의 머리에 각인되었고,
동일한 법률에 언급된 범죄들 간의 관념적 연관은
고정된 틀로 굳어져,
술라와
황제 아우구스투스에 의해
로마 형법을 통합하려는 공식적 노력이 행해질 때에도,
입법자들은 옛 범죄군을
그대로 유지했다.
술라와 아우구스투스의 법률들은
로마 제국의 형법의 원천이 되었거니와,
그 법률들이 후대에 물려준 범죄의 분류만큼
특이한 것도 없을 것이다.
한 가지 예만 들자면,
\hemph{위증}은 언제나
\hemph{절단상해}\latin{cutting and wounding} 및
\hemph{독살}과 함께 분류되었으니,
이는 분명
술라의 법률 중 하나인
`자살\hanja{刺殺}범과 독살범에 관한
코르넬리우스 법'\latin{Lex Cornelia de Sicariis et Veneficis}이
이 세 가지 범죄 형태 모두를 동일한 \wi{상설사문회}의 관할로 삼았기
때문인 것이다.\footnote{%
  이 코르넬리우스 법에서 말하는 `위증'은 일반적인 위증이 아니라
  이른바 사법살인, 즉 형사재판을 통해 누군가를 죽이려는 의도로
  행해진 위증을 뜻하던 것으로 보인다.
  \latin{D.\,48.8.1.1.} }
나아가
이러한 변칙적인 범죄 분류는 로마인들의 일상언어에도 영향을 끼친 것으로 보인다.
하나의 법률에 나열된 모든 범죄를
자연스레
목록의 첫 번째 이름을 가지고 지칭하는 습관이 생겼던 것으로 보이며,
이들 범죄 모두를 재판하는 법원도 이 이름으로  부르게 되었음이 분명하다.
그리하여
`간통사문회'\latin{quaestio de adulteriis}가 재판하는
모든 범죄가 `간통죄'로 불리게 되었던 것이다.

\para{이후의 형법}
지금까지 로마 사문회의 역사와 성격을 다루었거니와,
다른 곳에서는 형법의 형성과정을 알려주는 예를 찾을 수 없기 때문이다.
아우구스투스 황제에 의해
마지막 \wi{사문회}가 설치되었고,
그때부터 로마는 어느 정도 완비된 형법을 가진다고 말할 수 있게 되었다.\footnote{%
  유스티니아누스 시대에도 적용되던, 상설사문회에서 유래한 범죄---이를
  아래 `비상심리절차 범죄'와 구분하여
  `공적 형사소송'(iudicia publica)이라 불렀는데 피해자뿐 아니라 누구나
  기소를 할 수 있었다---는 다음과 같다:
  대역죄에 관한 율리우스 법(lex Iulia maiestatis),
  간통처벌에 관한 율리우스 법(lex Iulia de adulteriis coercendis),
  자살범(刺殺犯)과 독살범에 관한 코르넬리우스 법(lex Cornelia de sicariis et veneficis),
  근친살해에 관한 폼페이우스 법(lex Pompeia de parricidiis),
  위조에 관한 코르넬리우스 법(lex Cornelia de falsis),
  공적^^b7사적 폭력에 관한 율리우스 법(lex Iulia de vi publica seu privata),
  공물횡령에 관한 율리우스 법(lex Iulia peculatus),
  약취범에 관한 파비우스 법(lex Fabia de plagiariis),
  선거부정에 관한 율리우스 법(lex Iulia de ambitu),
  부당착취이득반환에 관한 율리우스 법(lex Iulia repetundarum),
  곡물부당이득에 관한 율리우스 법(lex Iulia de annona),
  공금유용에 관한 율리우스 법(lex Iulia de residuis) 등.
  \latin{Inst.\,4.18.} }
이러한 형법의 성장에 병행하여
이와 유사한 과정, 즉
내가 불법행위의 범죄로의 전환이라고 불렀던 과정이 진행되었다.
로마의 입법자들은,
비록 보다 중대한 범죄에 대한
민사적 구제수단을 없애지는 않았지만,
피해자들이 선호할 것이 분명한 구제수단을 그들에게 제공했던 것이다.
아우구스투스가 그의 입법을 마무리한 후에도,
근대사회라면 범죄로만 취급하는
몇몇 침해가 여전히 불법행위로 간주되고 있었다.
이들이 범죄로 되는 것은,
언제인지는 확인할 수 없으나, 후대의
법이 \wi{학설휘찬}\latin{Digest}에서
`\wi{비상심리절차에 의하는 범죄}'\latin{crimina extraordinaria}라고
불리는 새로운 유형의 범죄들을 인정하면서부터이다.\footnote{%
  원래 로마의 소송절차는 법무관 등 정무관의 면전에서 법률문제를 합의하는
  `법정절차'(in iure)와 일반인인 심판인의 면전에서 사실문제를 확인하고
  판결이 내려지는 `심판절차'(apud iudicem)의 두 단계로
  진행되었다. 고법상의 `법률소송'도, 공화정 후기부터 이를 대체하기 시작한
  `방식서소송'도 마찬가지였다.  그런데 제정기에 들어 황제가 임명한 재판관이
  법률문제와 사실문제를 모두 결정하는 보다 직권주의적 성격의 소송절차가 조금씩
  사용되기 시작하다가 제정 후기에는 마침내 모든 소송이
  이 `비상심리절차'(cognitio extraordinaria)에 의해 대체되었다.
  오늘날 대륙법의 소송절차는 로마법상의 비상심리절차의 후예라 할 수 있다.
  `비상심리절차에 의하는 범죄'는 종래의 통상적인 민사소송에 그치지 않고
  비상심리절차에 의한 특별한 구제수단이 피해자에게 주어지는
  범죄유형을 의미한다. }
분명
이들은
로마법 이론상 불법행위로만 취급되던 행위군이었으나,
사회의 존엄성에 대한 관념이 성장하면서
이들을 위반한 자가 단지 돈으로만 배상하면 그만인 상태를
더는 용납할 수 없게 되었고,
따라서 피해자가 원한다면
비상심리절차에 의하는 범죄로,
즉 통상적인 소송절차와는 여러모로 구별되는 구제방법으로,
가해자를 소추할 수 있게 허용한 것으로 보인다.
비상심리절차에 의하는 범죄가 인정되면서부터
로마 국가의 범죄목록은 근대 세계의 여느 국가들 못지 않는 수준이
되었음에 틀림없다.

\para{주권은 사법권의 원천}
로마 제국 하에서의 형사사법체계를
상세하게 기술할 필요는 없겠으나,
그것의 이론과 실무가 근대사회에 강력한 영향을 끼쳤음은
지적해두어야 하겠다.
황제들은
\wi{사문회}들을 즉시 폐지하지는 않았으며,
처음에는
원로원에게
폭넓은 형사사법권을
주었다.
아무리
원로원이
사실상 굴종적이었다고 해도,
원로원에서 황제는
명목상
다른 원로원의원들과 대등한 관계였을 뿐이다.
그러나
황제는
처음부터
몇몇 부수적인 형사재판권을 행사했고,
자유롭던 공화정 시절의 기억이 서서히 사라지면서
이것이 옛 법원들을 꾸준히 대체해나가게 된다.
차츰
범죄의 처벌은
황제가 직접 임명한 관리들의 손에 넘어갔고,
원로원의 특권도 황제의 추밀원\hanjalatin{樞密院}{privy council}으로 넘어갔거니와,
결국 이 황제의 추밀원이 형사 최고법원이 되었다.\footnote{%
  여기서 `추밀원'은 `근위장관'(近衛長官\,praefectus praetorio)을
  뜻하는 듯하다. 형사뿐 아니라 민사재판도 담당했다.
  파피니아누스, 울피아누스, 파울루스 등이 이 직을 역임한 대표적 법학자들이다.}
이러한 영향력의 결과,
근대인들에게도 친숙한 원리가 부지불식간에 형성되어갔으니,
주권자가 모든 사법권의 원천이자 모든 은혜의 저장소라는 원리가 그것이다.
이것은 아첨과 굴종이 증가한 결과물이 아니라,
이 시기에 이르러 완성된 제국의 중앙집권화의 결과물이었다.
형법의 이론은 사실상 첫 출발점으로 되돌아갔다고 할 수 있다.
자신이 입은 피해에 대해 국가가 스스로 나서서 보복한다는 데서
형법의 역사가 출발했거니와,
범죄의 처벌이
인민의 대표자이자 수임자\hanja{受任者}인 주권자에게
특별한 방식으로 속한다는 원리에서 끝맺고 있는 것이다.
새로운 견해가 옛 것과 다른 점은
정의의 수호자라는 직무로 인해
장엄하고도 위엄있는 분위기가
주권자의 인격을 감싸게 되었다는 것 정도이다.

\para{교회의 영향}
사법권에 대한 주권자의 관계에 관한 후기 로마의 견해는
내가 사문회의 역사로써 예시한 일련의 변화 과정을
근대사회가
되풀이할 필요성을 덜어주었다.
서유럽에 정착한 거의 모든 민족의 원시법에서는
범죄의 처벌이 전체 자유민의 집회에 속한다는 옛 관념의 흔적이 발견된다.
몇몇 나라---스코틀랜드가 그중 하나라고 한다---의
현존하는 법원은 그 기원을 입법기구의 위원회로 소급할 수 있다.
그러나 형법의 발달은 널리 두 가지 원인에 의해 촉진되었거니와,
로마 제국에 대한 기억이 그 하나요,
교회의 영향이 다른 하나이다.
한편으로,
잠시였지만 샤를마뉴 가\hanja{家}의 재위로 인해 이어진
황제들의 존엄성이라는 전통이 주권자들을 감싸고 있었으니,
그 위신은 단순히 만족\hanja{蠻族}의 수장으로서는 도저히 얻을 수 없는 것이었다.
또한 그 전통은
사회의 수호자이자 국가의 대표자로서의 주권자의 성격을
봉건 위계의 최하위 유력자에게까지
전해주고 있었다.
다른 한편으로,
교회는
참혹한 폭력을 억제하려는 열망에서
중대한 범죄행위를 처벌할 권위를 찾고자 했고,
처벌권한이 세속권력에게 주어져있다는 성경구절에서 그것을 발견했다.
악행을 행하는 자를 두려움에 떨게 하기 위해
세속 통치자가 존재한다는 것을 증명하는 데
신약성서가 원용되었고,\footnote{%
  ``관원들은 악을 행하는 자에게나 두려운 존재이지 \ldots''
  로마서 \latin{13:3.}}
구약성서는 ``다른 사람의 피를 흘리면 그 사람의 피도 흘릴 것이니''라는
주장에 원용되었다.\footnote{%
  창세기 \latin{9:6.}}
생각건대,
의심할 여지 없이
범죄라는 주제에 관한 근대적 관념은
암흑시대에 교회가 내세운 두 가지 가정\hanja{假定}에 근거한다.
하나는
각 위계의 봉건 통치자는 사도 바울이 말한
로마 관원들에 비견될 수 있다는 것이다.
다른 하나는
그가 처벌해야 할 범죄는
모세의 십계명으로 금지된 것들,
정확히 말하면 교회가 자신의 관할로 유보하지 않은 것들이라는 것이다.
이단\hanja{異端}---이는 십계명의 첫 번째와 두 번째 계명에
포함된 것으로 볼 수 있다---과 간통과 위증은
교회재판이 관할하는 범죄였고,
\index{교회법}%
교회는
특별히 사안이 중대한 경우
보다 가혹한 처벌을 부과할 목적에서만
세속권력의 협력을 용인했다.
동시에
교회는
살인과 강도는
그 다양한 변종들과 함께
세속 통치자의 관할에 속한다고 가르쳤다.
다만 그것은 세속 통치자 지위의 우연한 속성이 아니라
신의 명시적 말씀에 의해 그러하다는 것이다.

\para{알프레드 대왕의 형법}
\wi{알프레드 대왕}의 법전 중 한 구절\paren{켐블, 2.209}은
형법의 기원에 관한
당대의 여러 관념들 간의 갈등을 사뭇 명료하게 보여주고 있다.
알프레드는 그것을
일부는 교회의 권위에,
일부는 현자\hanjalatin{賢者}{witan}들의 권위에
귀속시키고 있음을 볼 수 있다.
그러면서도
주군에 대한 반역은,
\wi{대역죄}\latin{majestas}에 관한
로마법이
황제에 대한 반역을 통상적인 법에서 면제시켰듯이,
통상적인 법에서 면제된다고 명시하고 있다.
알프레드는 이렇게 적고 있다.
``이후 많은 나라들이 기독교 신앙을 받아들였고,
땅 위의 모든 곳에서 종교회의가 개최되었다.
영국 민족 사이에서도 기독교 신앙을 받아들인 후
성스러운 주교들의 모임과 고귀한 현자들의 모임이 있었다.
그리하여 그들은 이렇게 선포했다.
그리스도께서 가르치신 자비 덕분에,
세속 주군들은,
그들의 허락 하에,
모든 범죄자들에게서 그들이 정한 \wi{속죄금}\latin{bot}을
죄지음 없이
취할 수 있거니와,
다만 주군에 대한 반역의 경우에는
그들은 어떠한 자비도 베풀지 아니할 것이니,
전능하신 주께서는 주를 능멸한 자에게 판결을 내려주지 않으셨고,
그리스도께서도 그분을 죽음에 팔아넘긴 자들에게 판결을 내려주지 않으셨으며,
또한 말씀하시기를 그분과 같이 주군을 사랑하라 하셨기 때문이노라.''\footnote{%
  \latinmarks
  John Mitchell Kemble,
  \textit{The Saxons in England: A History of the English Commonwealth Till the Period of the Norman Conquest}, Vol.\,2,
  London: Longman, Brown, Green \& Longmans, 1849,
  pp.\,208f.}



\end{document}
