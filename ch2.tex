\chapter{법적의제}

원시법이 법전에 구체화되면서 자생적 발달이라고 할 만한 것은
종말을 맞았다.
이후로는 법 내부에 변화가 일어난다면 그것은 의도적으로 일어난,
그리고 외부로부터 영향받은 변화인 것이다.
어떤 민족이나 부족의 관습이
가부장적 왕에 의해 선언된 이후 마침내
성문화되어 공표되기까지 그 긴 시간---몇몇 경우에는 장구한 기간---동안
전혀 변함 없이 유지된다는 것은 상상할 수 없는 일이다.
또한 그 변화의 어떤 부분도 의도적으로 일어난 부분이 전혀 없다고
단정하는 것도 옳지만은 않을 것이다.
그러나,
이 기간의 법발달에 대해 우리가 아는 바가 별로 없긴 하지만,
변화를 가져옴에 있어 미리 계획된 목적이 차지하는 몫은
극히 작았을 것이라고 가정해도 무리가 없다.
초창기의 관행에 일어난 그러한 혁신은
오늘날 우리의 정신 조건을 가지고는 도저히 이해할 수 없는 감정과 사고양식에 의해
주어졌던 것 같다.
하지만 \index{법전 시대}법전과 더불어 새로운 시대가 시작된다.
법전 시대 이래, 법변동의 경로 어디를 추적하더라도
그것이 의식적인 개선 노력에 기인한다는 것을,
적어도 원시 시대에 목표했던 것과는 다른 목표를 달성하려는 노력에
기인한다는 것을 발견할 수 있다.

\para{진보의 희귀성}
언뜻 보면, 법전 시대 이후의 법의 역사에서 어떤 믿을 만한 명제를
이끌어내는 것은 불가능해보인다.
대상 영역이 너무 넓다.
충분히 많은 수의 현상을 관찰했는가,
또 관찰한 것을 정확하게 이해했는가, 따위에 대해 우리는 확신을 가질 수 없다.
그러나,
\wi{정체된 사회}\latin{stationary society}와
\wi{진보하는 사회}\latin{progressive society} 간의 구별이
법전 시대 이후
나타나기 시작했음을
감안하면 우리의 과업이 불가능해 보이지는 않는다.
우리의 관심대상은 진보하는 사회에 국한되거니와,
그것들의 숫자가 무척 적다는 점이 무엇보다 두드러진다.
압도적인 증거에도 불구하고, 서유럽 시민의 한 사람으로서
그를 둘러싸고 있는 문명이 세계 역사에서 희귀한 예외에 불과하다는 사실을
완전히 체감하기란 결코 쉬운 일이 아니다.
전체 인류에 대한 진보적 민족의 관계를 또렷이 직시한다면
우리가 공유하는 사상의 풍조, 우리들의 모든 희망·두려움·생각이
크게 바뀔 수 있을 것이다.
의심할 여지 없이, 인류의 대부분은
문명제도들을 항구적 기록으로 구체화하여 외면적 완성을 이룩한 순간 이후로
그 문명제도들을 개선하려는 일말의 욕구조차
보여준 적이 없었다.
때로 어떤 관행이 폭력적으로 전복되어 다른 관행에 자리는 내주는 경우는 있었다.
곳에 따라 원시 법전은
초자연적 기원을 내세우며 대폭 확대되기도 했고
종교적 주석가들의 왜곡을 거치며 놀랄 만한 형태로 뒤틀려지기도 했다.
하지만 이 세상의 아주 작은 한 지역을 제외하면
법체계의 지속적·점진적 개량 같은 것은 찾아볼 수 없었다.
물질문명은 있었지만, 문명이 법을 확장시키기보다는
법이 문명발달의 족쇄로 작용했다.
인류의 원시적 상태를 연구함으로써 우리는
어떤 문명이 그 발달을 멈춘 지점에 관하여 단서를 얻을 수 있을 것이다.
브라만 지배의 인도는 모든 인간사회가 경험한 단계, 즉
법규칙과 종교규칙이 아직 구별되지 않던 단계를 넘어서지 못했음을 알 수 있다.
그런 사회의 구성원들은 종교적 명령의 위반을 세속적 형벌로 처벌해야 한다고
믿었고, 세속적 의무의 위반을 신의 교정\hanja{矯正}에 맡겨야 한다고 믿었다.
중국은 이 지점을 넘어서긴 했으나
진보는 거기서 정체되었으니,
민사법이 중국인들의 관념의 한계에 갇혀있었기 때문이다.
하지만 정체된 사회와 진보하는 사회의 차이는
커다란 비밀에 싸여있고 우리는 그 비밀을 여전히 탐구해야 한다.
지난 장의 끝부분에서 나는 이 비밀에 대한 부분적 설명을 시도한 바 있다.
덧붙여 유의할 점은 인류에게 있어 \wi{정체된 사회}가 일반원칙이고
\wi{진보하는 사회}가 예외임을 정확히 인식하지 않으면 우리의 탐구는
성공할 수 없으리라는 것이다.
그리고 또 하나의 성공조건은 모든 주요 단계마다 로마법에 대한 정확한 지식이
불가결 요구된다는 것이다.
우리가 알고 있는 모든 인간제도들 중에서 로마법은 가장 긴 역사를 가지고 있다.
로마법이 경험한 모든 변화의 성격을 우리는 비교적 잘 알고 있다.
시작부터 종말에 이르기까지 그것은 보다 나은 방향으로,
혹은 변화의 설계자들이 보기에 더 낫다고 여겨졌던 방향으로
변화하며 진보했다.
로마법이 개선되어 나가는 동안,
인류의 나머지 부분들은 사상과 행동의 진전이 눈에 띄게 느려졌고,
정체상태로 빠질 위험에 끊임없이 노출되었다.

\para{진보적인 법}
이하에서 나는 진보하는 사회에 국한하여 논의를 전개하겠다.
이들 사회에서는 사회의 필요와 사회의 여론이 대체로 법에 선행한다고
말할 수 있다.
그것들 간의 간격이 끊임없이 메워지는 경향을 보이지만,
그 간격은 항상 또 다시 되살아난다.
법은 안정을 추구하지만, 우리가 말하고 있는 사회는 진보하는 사회이다.
인민이 더 행복하냐 덜 행복하냐는 이 틈새가 얼마나 신속하게 좁혀지는가에
달려있다.

법이 사회와 조화되도록 하는 장치에 관하여 유용한 명제 하나를
개진하고자 한다.
이러한 장치에는 세 가지가 있거니와,
\wi{법적의제}\latin{legal fictions}, \wi{형평법}\latin{equity},
그리고 \wi{입법}\latin{legislation}이 그것이다.
이들 간의 역사적 순서는 내가 제시한 대로이다.
때로는 이들 중 두 가지가 동시에 작용하기도 하고, 또
이들 중 하나의 영향을 받지 않은 법체계도 존재한다.
그러나 내가 아는 한 이들의 등장 순서가 뒤바뀐 사례는 존재하지 않는다.
이들 중 하나인 형평법은 그 초기 역사가 어디서나 모호했고,
따라서 어떤 이는 시민법을 개혁하는 단발적인 법률들이 형평에 의한 재판보다
더 오래됐다고 생각할지도 모른다.
나는 형평법에 의한 구제가 입법에 의한 구제보다 어디서나 더 먼저였다고 믿는다.
그러나 만약 이것이 완전히 진리가 아니라면,
그들의 순서에 관한 명제를,
원래의 법의 변화에 그들이 지속적이고 실질적인
영향력을 행사하는 기간에
국한해야 할 필요성이 있을 수는 있다.

\para{의제의 용도}
나는 ``의제''라는 단어를 영국 법률가들이 익히 사용하고 있는 것보다
훨씬 더 넓은 의미에서 사용한다. 또한 로마인들이 ``의제''\latin{fictiones}라는
말에 부여했던 것보다 훨씬 더 포괄적인 의미로 사용한다.
고대 로마법에서 의제\latin{fictio}는 기실 소송변론상의 용어였으니,
원고 측의 거짓 진술로서 피고가 이를 부인하는 것이 허용되지 않는 것을 뜻한다.
가령 원고가 사실은 외인\hanja{外人}이면서 로마시민이라고 진술하는 것이 그 예다.
이러한 로마법상 ``의제''의 목적은 말할 것도 없이 재판권을 부여하기 위한
것이었다.
\phantomsection\label{commonlawfiction}%
따라서 이는 영국의 \wi{왕좌법원}\hanjalatin{王座法院}{Queen's Bench}이나
\wi{재무법원}\hanjalatin{財務法院}{Exchequer}의 영장\latin{writ}에 담긴
진술---피고가 국왕의 감옥에 구금되어 있다는 진술, 혹은
원고가 왕의 채무자인데 피고의 채무불이행으로 인해
자신의 채무를 이행할 수 없다는
진술---과 무척 흡사하거니와,
이로써 이들 법원은 \wi{민소법원}\hanjalatin{民訴法院}{Common Pleas}의 재판권을 빼앗아올 수 있는
것이다.\footnote{%
  여기서의 `영장'은 왕을 대신하여 챈슬러(Chancellor)가 발부하는,
  그리하여 왕좌법원이나 민소법원에서의 소송절차를 허락하는
  소송개시영장(original writ)을 말하는 것이 아니다.
  원래 왕좌법원은 왕국의 평화를 위협하는 위법행위에 대해 관할권을 가졌다.
  원칙적으로 소송개시영장이 있어야 했으나, 미들섹스 주의 사건 또는
  법원 직원이나 국왕 감옥(Marshalsea) 수감자를 상대로 하는 사건은
  소송개시영장 없이 원고의 소장(bill)만으로 절차가 개시될 수 있었다.
  따라서 16세기에 이르러 다음과 같은 의제가 널리 사용되었다.
  가령, 원래 민소법원 관할인 금전채무소송(debt)을 왕좌법원에 제기하고자 하는
  원고는 우선 피고---물론 미들섹스에는 한발짝도 들여놓은 적이 없다---가
  미들섹스의 토지를 불법침해(trespass)했다는 소장을 제출한다.
  당연히 미들섹스 주의 집행관은 피고를 찾을 수 없다고 보고한다.
  그러면 원고는 피고가 실제 거주하고 있는 주에 그가 지금 숨어있다고 진술하고,
  왕좌법원은 그 주의 집행관에게 숨어있는 피고를 체포하라는
  영장(writ of latitat)을 발부한다.
  피고가 체포되어 일단 감옥에 갇히면, 설령 보석으로 풀려나더라도,
  원고는 진짜 목적인 금전채무 소장을 왕좌법원에 제출할 수 있는 것이다.
  한편, 재무법원의 소송은 챈슬러의 통제를 받지 않았기에
  왕의 채무자에게 소권을 부여하는 쿠오미누스 영장(writ of quominus)은
  재무법원의 영장이었다.
  그런데 왕의 채무자가 아님에도 왕의 채무자라고 주장하면서
  재무법원에 소송을 제기하는 현상이 16세기에 주로
  유언집행인을 상대로 금전채무 이행을 구하는 소송의 형태로
  나타나기 시작했고 17세기에는 이를 넘어 널리 일상적인 현상이 된다.
  이로써 왕의 이해관계와 무관한, 따라서 민소법원의 관할이어야 할 사건들이
  왕좌법원에도 재무법원에도 제기될 수 있게 되었고, 세 보통법법원 간의
  관할권 차이는 거의 사라진 것이나 다름없었다.
}
그러나 내가 사용하는 ``\wi{법적의제}''라는 표현은
법규칙의 문언은 그대로인 채 그 실제적 작용이 바뀐 변화의 사실을
숨기거나 숨기는 데 영향을 주는 일체의 가정\hanja{假定}을 총칭한다.
따라서, 앞서 인용한 로마법과 영국법의 의제 사례들뿐만 아니라
그 이외의 것도 여기에 포함된다. 나는 영국의 판례법과 로마의
\wi{법학자의 해답}\latin{responsa prudentium}도 의제에 기초한 것으로 보기 때문이다.
이들 두 가지에 대해서는 조금 있다 설명할 것이다.
이들 두 경우, \hemph{사실}로는 법이 완전히 변화했으나
\hemph{의제}적으로는 법이 예전 그대로 동일하다.
모든 형태의 의제가 왜 사회의 유년기와 특히 친화성이 있는지는 어렵지 않게
이해할 수 있다.
의제는 가끔 등장하는 개선 욕구를 충족시키면서도
변화에 대한 상존하는 미신적 거부감을 거스르지 않기 때문이다.
사회진보의 특정 단계에서 의제는 법의 엄격함을 극복하는 유용한 수단이 된다.
실로 그 가운데 하나인, 인위적인 가족관계 형성을 가능케 하는
\wi{입양}\hanja{入養}이라는 의제가 없었다면, 어떻게 사회가 그 요람기를 벗어나
문명을 향한 첫걸음을 뗄 수 있었을지 상상하기 어렵다.
그러므로 우리는 \wi{벤담}이 법적의제에 대해 퍼부은 조롱과 비난에 마음 상할
필요가 없다.
의제를 속임수에 불과하다고 욕하는 것은 법의 역사적 발달에서
의제가 수행한 특수한 역할에 대한 무지를 드러낼 뿐이다.
하지만 동시에, 의제의 유용성을 인정하면서 우리 법체계에 의제가
확고하게 뿌리내려야 한다고 주장하는 일부 논자들에게 동조하는 것도 똑같이
어리석은 일이 될 것이다.
영국 법률가들의 관념에 심각한 충격을 주지 않고
그들의 언어에 중대한 변화를 초래하지 않는 한
내다버릴 수 없는 몇몇 의제들이 여전히 강력한 영향력을 영국법에 행사하고 있다.
하지만 법적의제와 같은 거친 장치로써 어떤 유익한 결과를 도모하는 것이
우리에게는 어울리지 않는다는 것도 틀림없이 일반적 진리일 것이다.
법을 더 이해하기 어렵게 만들거나 조화로운 질서의 형성을 더 어렵게 만드는
어떠한 변칙도 무고\hanja{無辜}하지 않다고 나는 생각한다.
그런데 여러 장애 중에서도 법적의제야말로 체계적인 분류에 가장 큰
장애가 된다.
법의 규칙은 여전히 법체계에 들러붙어 있으나,
그것은 껍질에 불과하다.
저 규칙은 이미 오래 전에 쇠퇴했고, 새로운 규칙이 표면 아래 몸을 숨기고 있다.
그리하여 실제 작동하는 규칙을 그 진정한 장소에 분류해야 할지,
아니면 그 외관상의 장소에 분류해야 할지 알기 어려운 상황이 발생하거니와,
어느 선택지를 택할 지를 두고 여러 부류의 학자들 간에 의견이
갈라지게 되는 것이다.
영국법이 질서있는 분류를 채택하려 한다면,
최근의 몇몇 입법적 개선에도 불구하고 여전히 영국법에 널리 퍼져있는
법적의제들을 뿌리뽑지 않으면 안 될 것이다.

\para{형평법}
사회적 필요에 법이 적응하는 또 다른 수단은 내가 형평법이라 부르는 것이다.
여기서 \wi{형평법}이란 초창기 시민법에 병존하는 법체계로서
독자적인 원리에 기초하고 있고 그 원리에 내재한 우월한 신성함에 기대어
시민법을 넘어선다고 주장되는 것을 말한다.
로마 \wi{법무관}\latin{praetor}들의 형평법이든,
영국 \wi{챈슬러}\latin{Chancellor}들의 형평법이든,
형평법은
공개적이고 노골적으로 기존 법에 간섭한다는 점에서
각각 그것에 선행했던 의제들과 차이가 있다.
한편, 형평법은 법 개선의 동인으로 나중에 등장하는 입법과도 다르다.
형평법의 권위는
법 바깥의 어떤 사람이나 집단의 대권\hanja{大權}이 아니라,
법을 천명하는 정무관\hanjalatin{政務官}{magistrate}의 대권이 아니라,
모든 법이 따라야 한다고 여겨지는 법원리의 특별한 성격에
근거하고 있다는 점에서 차이가 있는 것이다.
초창기 법보다 더 높은 신성함을 가지고 있고
외부 기관의 승인과 무관하게 효력을 주장하는
일련의 원리들이라는 이러한 관념은
법적의제가 처음 등장했던 사고 단계보다 더 발달된 단계에 속한다.

\para{입법}
전제군주의 형태로든, 의회의 형태로든,
전체 사회를 대표한다고 간주되는 입법기관의 법제정인 \wi{입법}은
법 개선 수단 중에서 마지막 것이다.
입법과 법적의제의 차이는 형평법과 법적의제의 차이와 동일하다.
입법은
그 권위가 외부의 기구나 사람에게서 나온다는 점에서
형평법과도 구별된다.
입법의 구속력은 그것의 법원리와 무관하다.
현실적으로는 여론에 의한 제약이 있다 하더라도,
이론적으로 입법기관은 스스로가 원하는 바를 공동체 구성원들에게
의무로 부과할 권한을 가진다.
입법기관이 자의적 변덕에서 하는 입법을 막을 것은 아무 것도 없다.
만약 형평이 어떤 선악의 기준을 뜻하는 말로 사용되고
법제정이 어쩌다 이러한 기준에 맞추어 행해진다면,
그러한 입법은 형평에 의해 지시된 것이라 할 수 있을 것이다.
하지만 이런 경우에도 법제정의 구속력은 입법기관의 권위에 빚지고 있는 것이지,
입법기관의 행위 근거가 된 원리의 권위에 빚지고 있는 것이 아니다.
그리하여 입법이 기술적 의미의 용어인 형평법 규칙과 다른 점은,
후자는 최고의 신성함을 내세우며 군주나 의회의 협찬이 없더라도
즉각 법원에 받아들여질 것을 요청한다는 데 있다.
이런 차이에 주목해야 할 더 큰 이유는,
어떤 벤담 학도는 법적의제, 형평법, 제정법을 뭉뚱그려
이 모두를 입법이라는 단일 범주로 포괄하려 할 것이기 때문이다.
이 모두가 \hemph{법창조}\latin{lawmaking}에 관한 것으로,
그것들 간 차이는 단지 새 법이 만들어지는 장치의 차이일 뿐이라고 그는
말할 것이다.
이것은 분명 진실이고 우리는 이것을 망각해서는 안 된다.
하지만 그렇다고 해서 입법과 같은 무척이나 편리한 용어를
특수한 의미로 사용해서는 안 될 이유가 되지는 못한다.
입법과 형평법은 대중의 정신에서, 그리고 대부분의 법률가들의 정신에서,
서로 분리되어 있다.
특히, 중요한 실제적 결과의 차이가 뒤따른다면,
아무리 인습적이라 해도 양자의 차이를 무시하는 것은 결코 정당화될 수 없다.

\para{법적의제}
거의 모든 발달된 법체계에서 \hemph{법적의제}의 사례들을 선별하기란
쉬운 일일 것이며, 그것들은 즉시 법적의제의 진정한 성질을 현대의 관찰자들에게
드러낼 것이다.
하지만 이제부터 내가 다루려는 두 가지는
거기에 사용된 수단의 본질이 그리 쉽게 드러나지 않는다.
이들 의제의 최초 창시자들은 아마 혁신을 의도하지 않았을 것이며,
혁신의 의심을 사기는 더더욱 바라지 않았을 것이다.
게다가 그러한 혁신의 과정에 의제가 들어있음을 부인하는
사람들이 늘 있고 또 있어왔거니와,
전래의 인습적 언어가 그들의 부인\hanja{否認}을 실증한다.
그러므로 \wi{법적의제}의 광범위한 확산을 보여주는,
그리고 법체계를 변화시키면서도 그 변화를 감추는 이중적 역할의
효율적 수행을 보여주는 사례로서 이보다 더 나은 것들은 없을 것이다.

\para{사법적 입법}
이론적으로는 조금도 기존 법을 바꿀 힘이 없는 장치가
법을 확대하고 수정하고 개선해나가는 것에
우리 영국인들은 아주 익숙하다.
이러한 사실상의 입법이 작동하는 과정은 감지될 수 없는 것이 아니라
인정되지 않을 뿐이다.
판례들에 담겨있고 판결집들에 기록돼있는 우리 법체계의 방대한 부분에 대해
우리는 습관적으로 이중적 언어를 사용하고 이중의 모순적인 관념들을 구사한다.
어떤 사실관계가 영국 법원에 제소되면
판사와 변호사들 간의 모든 논쟁은
옛 법원리 외의 어떤 법원리도,
오래된 개념구분 외의 어떤 개념구분도
적용될 필요가 없고 적용될 수도 없다는
가정 하에 진행된다.
계쟁 분쟁의 사실관계를 포섭하는 기존의 법규칙이 어딘가에 존재하며,
설령 그러한 규칙이 발견되지 않더라도 인내·지식·통찰력을 발휘하면
얼마든지 그것을 찾아낼 수 있다는 믿음을 극히 당연한 것으로 받아들인다.
그러나 일단 판결이 내려지고 기록되고 나면, 우리는 무의식적으로 혹은 은밀하게
새로운 언어, 새로운 사고 맥락으로 넘어간다.
이제 우리는 새로운 판결로 법이 수정\hemph{되었다}고 인정한다.
적용가능한 법규칙이, 흔히 쓰이는 부정확한 표현을 사용하자면,
보다 유연해졌다고 믿는다.
실제로 법규칙은 변경되었다.
선례에 새로운 것이 첨가되었고, 선례들을 비교하여 얻어지는 법원리는
일련의 판례들에서 하나의 사례를 제외했을 때 얻어지는 것과는 다른 것이 되었다.
옛 규칙이 폐지되고 새로운 것으로 대체되었다는 사실을 우리는 받아들이기 어려운데,
선례에서 얻어지는 법적 공식을 정확한 언어로 표현하는 습관을 갖고 있지 못하여,
변화의 광채가 강렬하고 눈부신 경우가 아닌 한 쉽게 포착하지 못하기 때문이다.
진기한 변종 판결 앞에서 영국 법률가들이 침묵으로 일관하는
이유를 여기서 장황하게 늘어놓을 생각은 없다.
아마도, 구름 속이든\latin{in nubibis} 혹은
판사의 마음 속이든\latin{in gremio magistratuum} 어딘가에
완전하고 일관되고 체계적인 영국법이 존재한다는, 그리하여 상상할 수 있는 어떤 상황에도
적용할 수 있는 풍부한 법원리의 체계가 존재한다는 것이 전래의 교리였다는 것은
말할 수 있을 것이다.
처음에는 이 이론이 지금보다 훨씬 더 철저히 신봉되었으며,
실제로 그럴 만한 근거가 더 충분했다.
13세기 판사들은 변호사나 일반 대중에게는 알려지지 않은
법의 보고\hanja{寶庫}를 이용할 수 있었으니,
그들은 은밀히 당대의 로마법과 \wi{교회법} 집성들로부터, 항상 현명하게는 아닐지라도,
자유롭게 빌려왔다고 믿을 만한 이유가 있다.
하지만 웨스트민스터 홀의 법원들이 판결을 양산하여 실체법 체계의 토대가 마련되자,
이 저장고는 폐쇄되었다.
그리하여 수 세기 동안 영국의 법률가들은 형평법과 제정법이 아닌 한 아무 것도
이미 형성된 이 토대에 첨가된 것이 없다는 역설적인 명제를 전승시켜왔다.
우리는 우리 법원들이 입법을 한다는 것을 인정하지 않는다.
우리는 우리 법원들이 결코 입법을 한 적이 없다고 생각한다.
그럼에도 불구하고 우리는 영국의 \wi{보통법} 규칙들이, \wi{형평법법원}\latin{Court of Chancery}과
의회로부터 약간의 도움을 받아, 현대사회의 복잡한 이해관계에 충분히 대처할 수 있다고 주장한다.

\para{법학자의 해답}
방금 언급한 특징에 있어서 우리 판례법과 무척 가깝고 시사하는 바가 사뭇 유사한 법체계가
로마에서는 ``법에 식견 있는 자의 답변''이란 뜻의 \wi{법학자의 해답}\latin{responsa prudentium}이었다.
이들 해답은 로마법의 발달 시기에 따라 상당히 다른 형태를 띠었지만,
전 시기에 걸쳐 어떤 권위 있는 성문의 문헌들을 해설하는 주석임에는 변함이 없었고,
처음에는 오로지 12표법에 대한 해석 의견의 모음이었다.
우리와 마찬가지로, 모든 법적 언어는 이 옛 법전의 텍스트가 불변이라는 가정에 기초했다.
거기에 명시적인 규칙이 있었다.
그것은 어떤 주석이나 주해보다 위에 있었고, 어떤 해석도, 설령 위대한 해석자의 것이라 해도,
거룩한 텍스트에 호소하여 수정될 수 있음을 누구도 공공연히 부인할 수 없었다.
하지만 사실 저명한 법학자의 이름을 달고 있는 해답집은
적어도 우리의 판결집에 버금가는 권위를 누렸고,
12표법의 규정을 지속적으로 수정하고 확장하고 제한하고 사실상 뒤엎었다.
새로운 법학의 형성기 동안 법학의 저술가들은 법전의 문구에 꼼꼼한 충실함을 내세웠다.
단지 그것을 설명하고 독해하고 그 의미를 온전히 드러낼 뿐이라고 생각했다.
그러다 결국 그들은 텍스트를 이어붙이고,
실제로 발생한 사실관계에 법을 적응시키고,
일어날 법한 사실관계에 법의 적용가능성을 탐구하고,
다른 성문 문헌에서 도출한 해석 원리를 가져오는 등에 의해
12표법 편찬자들은 꿈도 꾸지 못했던, 실로 12표법에서는 거의 혹은 전혀 찾아볼 수 없는
사뭇 다양한 법원리들을 이끌어냈다.
법학자들의 저술은 모두 법전과 일치한다는 근거에서 존중받을 자격을 주장했으나,
그것의 상대적 권위는 저술을 발표한 특정 법학자의 명성에 크게 좌우되었다.
널리 알려진 위대한 학자의 이름은 입법기관의 법제정에 버금가는 구속력을 해답집에 부여했다.
그리고 그러한 저서가 이번에는 한층 더 나아간 법학 발달의 새로운 토대로 작용했다.
하지만 초기 법학자들의 해답은 오늘날처럼 저자에 의해 출간된 것이 아니었다.
그것은 그의 학생들이 기록하고 편집한 것이어서,
대개는 어떤 체계적인 분류법에 따라 배열된 것이 아니었다.
이렇게 출간에 있어 학생들이 한 역할은 특히 주목할 필요가 있거니와,
그들이 스승에게 행한 봉사는 학생 교육에 대한 스승의 충실성에 의해 보상받는 것이 일반적이기 때문이다.
후대에 가서 이 의무의 결실로 인정받게 되는
\wi{법학제요}\hanjalatin{法學提要}{Institutes}, 즉
주해서\latin{Commentary}라 불리는 교육용 저술들은
로마법 체계의 사뭇 중요한 특징을 이루는 것이다.
법학자들이 대중들에게 개념의 분류와 전문용어의 개선을 제안한 것은
이러한 법학제요 형태의 작품에서였지, 훈련된 법률가들을 겨냥한 저서에서가 아니었다.

로마의 \wi{법학자의 해답}과 그것의 영국적 대응물을 비교할 때,
로마법학의 이 부분이 가지는 권위는 \hemph{판사직}\latin{bench}이 아니라
\hemph{변호사직}\latin{bar}에서 유래한다는 점에 주의해야 한다.
로마에서 법원의 결정은, 개별 사건을 종결짓는 것이었지만,
장래를 향한 어떤 권위도 가지지 못했고, 다만
해당 사건을 잠시 담당하게 된 정무관의 전문직업적 명성에 의해 주어지는
권위만 누릴 뿐이었다.
사실 공화정기 동안 로마는 영국의 왕좌법원\latin{Bench}이나
신성로마제국의 제실법원\hanjalatin{帝室法院}{Chamber},
프랑스왕국의 파를르망\latin{Parliament} 비슷한 제도를 전혀 알지 못했다.
각자 맡은 분야의 사법적 기능을 그때그때 담당하는 정무관들은 있었지만,
정무관의 임기는 1년에 불과했기에, 그것은 상설 법관이라기보다는
정상급 변호사들이 돌아가면서 잠깐씩 맡는 순환 공직에 가까웠다.\footnote{%
  기실
  법무관(praetor) 등 정무관은 변호사---오늘날의 전문직 법률가로서의
  변호사가 아니라 웅변가(orator)였음---나 법학자일 수도 있지만
  그냥 정치가인 경우도 많았다.}
우리 눈에는 무척 이상하게 보이는 이러한 제도의 기원에 대해 다양한 견해가 있을 수 있지만,
사실 그것은 현대 우리의 제도보다 고대사회의 정신에,
서로 배타적인 개별 신분집단들로 나누어지지만
그 외에 전문직 간의 상하관계는 허용하지 않는 정신에 더 잘 부합했다.

이 체제는 그로부터 기대할 법한 효과를 가져오지 못했다는 점에 주목할 필요가 있다.
가령 그것은 로마법을 \hemph{대중화}하지 못했다.
비록 법학의 확산과 권위 있는 해설에 인위적인 장벽을 두지는 않았으나,
몇몇 그리스 공화국에서처럼 법학을 습득하는 데 필요한 지적 노력을 완화해주지 못했다.
오히려, 어떤 다른 원인들이 작동하지 않았더라면, 후대의 지배적 법체계들처럼 로마의 법학도
사소한 데 치중하고 기술적이고 배우기 어려운 학문이 되었을 확률이 상당히 컸다.
또한, 훨씬 더 마땅히 발견될 법한 어떤 결과도 전혀 나타나지 않은 듯하다.
로마 공화정이 무너지기 전까지 법학자들은 명확하게 정의되지 않은 집단을 형성하고 있었던 것이다.
또한 그 숫자도 틀림없이 큰 폭으로 오르내렸을 것이다.
그럼에도 불구하고, 주어진 사례에 대해서 어떤 사람의 의견이 그들 세대에서 결정적인 권위를
누렸는지는 의심의 여지가 거의 없었던 것 같다.
여러 라틴어 문헌에 전해지는, 정상급 법학자들의 일상 업무에 관한
생생한 묘사---이른 아침부터
시골에서 올라온 고객들이 그의 대기실에 몰려들고,
공책을 든 학생들은 그의 주변에 둘러서서 위대한
법률가의 답변을 기록한다---는 일정 기간에 국한해 본다면
한 두 명의 저명한 이름을 거의 혹은 전혀 벗어나지 않는다.
또한 고객들과 변호사의 직접적인 접촉 덕분에,
로마 사람들은 전문가들의 명성의 오르내림을 즉각적으로 알고 있었던 듯하다.
저 유명한 \wi{키케로}의 <<무레나를 위한 변론>>\latin{Pro Muraena}을 비롯한 풍부한 증거가
있거니와, 법정에서의 성공에 대한 일반인들의 존경은 과도하면 과도했지 부족하지 않았다.

의심할 여지 없이,
로마법의 발달을 추동한 수단에 관한 전술한 특징은
그것의 우수성, 즉 일찍부터 법원리가 풍부했던 것의 원천이었다.
법원리의 성장과 풍부함은 부분적으로는 법해설자들 간의 경쟁에 의해
촉진되었거니와, 국왕이나 국가가 부여하는 사법대권\hanja{司法大權}의 담지자인
왕좌법원\latin{Bench} 같은 것이 존재하는 곳에서는 이러한 영향력이 작동할 수가 없다.
하지만 주된 동력은 말할 것도 없이 사법판결의 대상이 되는 사례의
무제한적 증가에 있었다.
시골 고객들을 당혹케했던 사실관계들이 법학자의 해답이나 사법판결의
토대가 되었을 뿐만 아니라, 똑똑한 학생들이 제기한 가상의 사례들도
그에 못지 않았다.
실제 사례든 가상의 사례든, 모든 사실관계는 자격에 있어 차이가 없었다.
법학자들로서는 그의 고객의 사건을 재판하는 정무관이 그의 의견을
퇴짜놓는다고 해도 아무 문제가 되지 않았다.
오히려 정무관이 법지식에 있어서나 전문직업적 평판에 있어서 자신보다
위에 있는 것이 문제였다.
그렇다고 해서 법학자들이 고객의 이익에 무심했다는 말은 아니다.
고객들은 초기에는 저명한 법률가들의 선거인단이었고
후기에는 돈을 벌게 해주는 사람들이었기 때문이다.
그러나 야망을 충족시키는 주된 길은 동료 집단의 평판을 통해서였던 것이다.
전술한 이러한 체제 하에서 평판을 확보하는 좋은 방법은 각 사례를,
법정에서 승리하기 위한 고립된 사건으로 접근하는 것이 아니라,
어떤 포괄적 법원리나 법규칙의 예시의 하나로 바라보는 것이다.
있을 수 있는 사례를 제시하거나 발명해내는 데 아무런 제약이 없었던 것도
큰 영향력을 발휘했을 것임에 틀림없다.
데이터를 마음껏 증가시킬 수 있는 곳에서는
일반적 규칙을 진화시키는 능력이 대폭 증대된다.
우리의 사법체계에서는 판사들이 자기 앞에 놓인,
혹은 그의 전임자들 앞에 놓였던 사실관계를 벗어날 수가 없다.
따라서 재판의 대상이 된 각 사실관계는,
프랑스 식으로 표현하면, 일종의 성별\hanja{聖別}이 이루어진다.
실제 사건이든 가상의 사례든, 다른 모든 사건들과 구별되는 성질을 가지는 것이다.
하지만 로마에서는, 전술한 바에서 짐작할 수 있듯이
판사들의
왕좌법원\latin{Bench}이나 제실법원\latin{Chamber}
같은 것이 전혀 없었고,
따라서 어떤 사실관계도 다른 사실관계보다 더 특별한 가치를 지니지 않았다.
어떤 어려운 사안이 법학자의 의견을 요청하는 경우,
뛰어나 유추 감각을 지닌 이는 거리낌없이 그것과 어떤 특징을 공유하는
모든 상상할 수 있는 사례들을 즉시 인용하고 고려할 수 있었다.
고객에게 주어진 실무적 조언이 무엇이든 간에,
학생들의 공책에 쓰여진 해답\latin{responsum}은 분명
숭고한 법원리로 규율되는, 또는 포괄적인 법규칙에 포섭되는,
그러한 사실관계들을 고려했을 것이다.
우리에게는 이러한 것이 한 번도 가능한 적이 없었다.
그리고 영국법에 가해진 수많은 비판 속에서
영국법이 선언되는 양식에 대한 비판은 잊혀져버린 것 같다고 인정하지 않을 수 없다.
우리 법원이 법원리를 선언하는 데 인색한 것은
우리 판사들의 기질 탓보다는
우리에게 선례가,
다른 법체계들을 알지 못하는 이들에게는 많아 보일지 모르나,
상대적으로 부족한 데 더 큰 원인이 있는 듯하다.
법원리의 풍부함에 있어 여러 근대 유럽대륙의 국가들에 비해
우리가 대단히 빈약한 것이 사실이다.
하지만 그들은 민사법 제도의 기초로 로마법을 채택했음을 기억해야 한다.
그들은 로마법의 파편들을 가지고 그들의 성채를 건설했다.
그러나 그밖의 재료나 솜씨에 있어서는 영국 법원이 건설한 구조보다
우월할 것이 별로 많지 않다.

\para{이후의 로마법}
로마 공화정기는 로마법학에 그 특징이 각인된 시기였다.
로마법학의 초기 동안 법학자의 해답이 법발달의 주역이었다.
그러나 공화정의 몰락이 다가오면서 해답들은 더 이상의 확장을 저해하는
형태를 띠기 시작한 것으로 보인다.
그것들은 이제 체계화되어갔고 단순한 모음집이 되어갔다.
신관\hanjalatin{神官}{pontifex}이었던
무키우스 \wi{스카이볼라}\latin{Q. Mucius Scaevola}는
시민법 전체의 매뉴얼을 출간했다고 한다.
\wi{키케로}의 저술들은
능동적인 법 혁신 수단들에 대비되는 낡아빠진 방법들에 대한 염증이
커지고 있었음을 보여준다.\footnote{%
  가령 키케로, <<법률론>>, 1.14. }
사실 이때쯤이면 다른 요인들도 법에 영향을 미치게 된다.
\wi{법무관}이 매년 선포하는 \wi{고시}\hanjalatin{告示}{edict}는
이제 법개혁의 주된 동력으로 인정받고 있었다.
코르넬리우스 \wi{술라}\latin{L. Cornelius Sylla}는
\wi{코르넬리우스 법}\latin{Leges Corneliae}이라 불리는 일련의 위대한 법률들을
제정함으로써 직접적 \wi{입법}에 의해 얼마나 빨리 개선이 이루어질 수 있는지
잘 보여주었다.
\wi{법학자의 해답}에 최종 일격을 가한 것은 \wi{아우구스투스}였다.
제출된 사안에 대해 구속력 있는 해답을 줄 수 있는 권리를
몇몇 정상급 법학자들에게만 부여한 것이다.
이 변화는, 비록 근대적 관념에 가까이 다가가는 것이기는 하나,
확실히 법전문직의 성격 및 그것이 로마법에 미친 영향의 성질을
근본적으로 바꾸어놓았다.
법학의 영원하고 위대한 등불이 되는
또 다른 일군의 법학자들이 후대에 등장하지만,
\wi{울피아누스}, 파울루스, \wi{가이우스}, 파피니아누스는 해답의 저자들이 아니었다.
그들의 저술은 법의 특정한 분야, 특히 법무관의 \wi{고시}에 대해 쓴
본격적인 전문법학서적이었다.

\para{로마의 제정법}
로마의 형평법 및 이것을 로마법에 만들어넣은 법무관 고시에 대해서는
다음 장에서 살펴볼 것이다.
제정법에 대해서는, 공화정 시기에는 수가 많지 않았으나
제정기에는 양산되었다는 점만 말해두고자 한다.
국가의 청년기나 유년기에는 사법\hanja{私法}의 일반적 개혁에
입법기관이 동원되는 경우가 드물다.
민중의 요구사항은 법을 변화시키는
것---이것은 실제 가치보다 높게 평가받는 경향이 있다---이 아니라
재판이 깨끗하고 완전하고 수월하게 진행되는 데 있었다.
입법기관에 대한 호소는 대체로 어떤 큰 권력남용을 제거해달라든가,
해결하기 어려운 신분 간의 혹은 권문세족 간의 다툼에 대해
결정을 내려달라는 정도에 불과했다.
로마인들은 대규모의 법률 제정과
큰 내란 뒤의 사회 안정 사이에
어떤 연관성이 있다고 생각했던 듯하다.
\wi{술라}는 \wi{코르넬리우스 법}들로써 공화국 재건의 징후를 보여주었다.
율리우스 카이사르는 방대한 양의 제정법을 추가하려는 계획을 가지고 있었다.
\wi{아우구스투스}는 \wi{율리우스 법}\latin{Leges Juliae}이라 불리는
매우 중요한 일군의 법률을 통과시켰다.
후대의 황제들 가운데 가장 적극적으로 \wi{칙법}\latin{constitution}을 공포한 이는,
콘스탄티누스처럼, 세상을 재조정하는 데 관심을 가졌던 황제들이었다.
로마에서 제정법의 진정한 시대는 제정기에 비로소 시작된다.
황제들의 법제정은
처음에는 민중의 지지에 의해 제정되는 척 치장했으나
나중에는 황제의 대권에서 유래한다고 공공연히 주장되었는데,
\wi{아우구스투스}의 권력이 공고해진 이후 \wi{유스티니아누스} 법전의 공표에 이르기까지
점점 더 그 양이 증가해갔다.
이미 제2대 황제 치세 때에 오늘날 우리 모두에게 친숙한 법상태 및
법집행 양태와 상당히 비슷해졌다고 할 수 있다.
제정법이 등장했고 한정된 인원의 법해설자단\hanja{團}이 등장했다.
얼마 후에는 상설 상소 법원과 공인된 주해를 모은 주해집이 여기에 추가된다.
그리하여 오늘날의 관념에 가까이 다가가게 되는 것이다.

