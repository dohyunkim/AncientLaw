\chapter{법적 의제}

원시법이 법전에 구체화되면서, 자생적 발달이라고 할 만한 것은
종말을 맞았다.
이후로는 법 내부에 변화가 일어난다면 그것은 의도적으로 일어난,
그리고 외부로부터 영향받은 변화인 것이다.
가부장적 왕에 의해 선언된 후
성문화되어 공표되기까지 그 긴 시간---몇몇 경우에는 장구한 기간---동안
어떤 민족이나 부족의 관습이
전혀 변함 없이 유지된다는 것은 상상할 수 없는 일이다.
또한 그 변화의 어떤 부분도 의도적으로 일어난 부분이 전혀 없다고
단정하는 것도 옳지만은 않을 것이다.
그러나,
이 기간의 법발달에 대해 우리가 아는 바가 별로 없긴 하지만,
변화를 가져옴에 있어 미리 계획된 목적이 차지하는 몫은
극히 작았을 것이라고 가정해도 무리가 없다.
초창기의 관행에 일어난 그러한 혁신은,
오늘날 우리의 정신 조건을 가지고는 도저히 이해할 수 없는 감정과 사고양식에 의해
주어졌던 것 같다.
하지만 법전과 더불어 새로운 시대가 시작된다.
법전 시대 이래, 법변동의 경로 어디를 추적하더라도
그것이 의식적인 개선 노력에 기인한다는 것을,
적어도 원시 시대에 목표했던 것과는 다른 목표를 달성하려는 노력에
기인한다는 것을 발견할 수 있다.

\para{진보의 희귀성}
언뜻 보면, 법전 시대 이후의 법의 역사에서 어떤 믿을 만한 명제를
이끌어내는 것은 불가능해보인다.
대상 영역이 너무 넓다.
충분히 많은 수의 현상을 관찰했는가,
또 관찰한 것을 정확하게 이해했는가, 따위에 대해 우리는 확신을 가질 수 없다.
그러나,
법전 시대 이후 정체된 사회\latin{stationary society}와
진보하는 사회\latin{progressive society} 간의 구별이 나타나기 시작했음을
감안하면, 우리의 과업이 불가능해 보이지는 않는다.
우리의 관심대상은 진보하는 사회에 국한되거니와,
그것들의 숫자가 무척 적다는 점이 무엇보다 두드러진다.
압도적인 증거에도 불구하고, 서유럽 시민의 한 사람으로서
그를 둘러싸고 있는 문명이 세계 역사에서 희귀한 예외에 불과하다는 사실을
완전히 체감하기란 결코 쉬운 일이 아니다.
전체 인류에 대한 진보적 민족의 관계를 또렷이 직시한다면
우리가 공유하는 사상의 풍조, 우리들의 모든 희망, 두려움, 생각이
크게 바뀔 수 있을 것이다.
의심할 여지 없이,
문명제도들을 항구적 기록으로 구체화하여 외면적 완성을 이룩한 순간 이후로
인류의 대부분은 그 문명제도들을 개선하려는 일말의 욕구조차
보여준 적이 없었다.
때로 어떤 관행이 폭력적으로 전복되어 다른 관행에 자리는 내주는 경우는 있었다.
곳에 따라 원시 법전은,
초자연적 기원을 내세우며 대폭 확대되기도 했고,
종교적 주석가들의 왜곡을 거치며 놀랄 만한 형태로 뒤틀려지기도 했다.
하지만 이 세상의 아주 작은 한 지역을 제외하면
법체계의 지속적^^b7점진적 개량 같은 것은 찾아볼 수 없었다.


