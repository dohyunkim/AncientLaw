\chapter{원시사회와 고대법}

법이라는 주제를 과학적으로 다룰 필요성은 현 시대 들어
완전히 망각된 적이 없거니와,
다양한 재능을 가진 인재들이
이러한 필요성의 인식 하에 논문들을 제출해왔다.
그러나, 생각건대,
지금까지 과학의 자리를 대신 차지하고 있던 것은
대체로 일군의 추측이었다는 것에는 의심의 여지가 별로 없다.
앞의 두 장에서 살펴보았던 로마 법률가들의 추측이 바로 그런 것들이다.
이렇게 추정적 자연상태 이론과
그것에 어울리는 법원리의 체계를 인정하고 수용하는 일련의 명시적 진술들이
이것들을 발명한 시대로부터 오늘날에 이르기까지 거의 중단없이 지속되어왔다.
근대 법학을 기초놓은 주석학파\latin{Glossators}의 주석에서도,
이들을 계승한 스콜라주의 법학자들의 저술에서도 그것들은 등장한다.
교회법학자들의 법리에서도 쉽게 눈에 띈다.
문예부흥기에 쏟아져나온 놀라울 정도로 박학다식한
로마법 학자\latin{civilian}들에서는
그것들이 전면에 부상한다.
그로티우스와 그 후계자들은 그것들에 명료함과 그럴듯함뿐만 아니라
실무적 중요성도 부여했다.
그것들은 블랙스톤의 저서의 서론 장들에서도 읽을 수 있거니와,
이는 뷔를라마키\hyphlatin{Jean-Jacques Burlamaqui}의 저서에서
글자 그대로 옮겨적은 것이다.
오늘날 법학도와 실무가들을 위해 출간된 교재들의 첫 머리를 장식하고 있는
법의 제1원리에 관한 논의는 언제나 저 로마인들의 가설을
재진술하고 있는 것에 불과하다.
그러나 이들 추측의 고유한 형식에서뿐만 아니라
그것들을 감싸고 있는 위장된 겉모습에서도
우리는 그것들이 인간 정신에 뒤섞어넣은 미묘함을 잘 파악할 수 있다.
로크의 사회계약론에서 법의 기원에 관한 이론은
그 로마적 유래를 거의 숨기지 않거니와,
실로 고대의 견해가 근대인들에게 매력적으로 보이려면
어떤 모습을 갖추어야 하는지를 알려준다.
한편, 동일한 주제에 대한 홉스의 이론은
로마인들과 그 후예들이 생각했던 자연법의 현실성을
부인하기 위해 의도적으로 고안된 것이다.
그러나 영국의 정치인들을 오랫동안 적대적 진영으로 양분했던
이들 두 이론은 양자 모두 인류의 비역사적이고 검증불가능한 상태를
근본적 전제로 삼는다는 점에서 서로 닮아있다.
물론 로크와 홉스는 사회 이전 상태의 성격에 대해서, 그리고
그 상태로부터 우리가 알고 있는 사회 상태로 이월하는 계기가 되는
비정상적 행위가 어떤 것이냐에 대해서, 서로 의견을 달리한다.
하지만 원시상태의 사람과 사회상태의 사람 사이에
이들을 갈라놓는 커다란 틈이 있다는 생각에는 일치하거니와,
이 관념이 의식적으로든 무의식적으로든
로마인들에게서 빌려온 것이라는 점에는 의문의 여지가 없다.
사실 법현상을 이들 이론가들이 생각한 방식대로---즉, 하나의 거대한
복합체로---파악한다면, {\small(그럴듯하게 해석되면)}
모든 것을 조화시킬 수 있는 영리한 추측에 의지하여
우리가 우리 스스로 설정한 과업을 자주 회피하게 되더라도,
아니면 절망에 빠져 체계화의 노력을 때로 포기하게 되더라도,
그것은 놀라운 일이 아닐 것이다.

\para{몽테스키외}
로마인들의 법리와 동일한 사변적 기초를 가지는 법이론으로부터
두 명의 유명인사는 제외함이 마땅하다.
그중 첫 번째는 몽테스키외라는 위대한 이름과 관련된 인물이다.
<<법의 정신>>의 첫 부분에는 다소 모호한 표현들이 나오는데,
저자가 당대의 지배적 견해에 공개적으로 도전장을 제출하기를 꺼려했기
때문이라고 여겨진다.
하지만 저 책의 일반적 흐름은 확실히 그 주제에 관한 이전의 어떤 관념과도
결별하는 모습을 보여준다.
흔히들 지적된대로,
방대한 조사를 통해 가상의 법체계들로부터 끌어모은 다양한 사례들 속에는,
상스럽고 생경하고 외설스런 습속과 제도들을 특별히 강조함으로써
문명사회의 독자들을 놀라게 하려는 갈망이 뚜렷이 엿보인다.
그것의 일관된 주장은 법이 기후, 지리적 위치, 우연, 기망 따위의
산물---용인할만한 항구성을 가지고 작용하는 것을 제외한 모든 원인의
결실---이라는 것이다.
실로 몽테스키외는 인간의 본성을 전적으로 유연한 것으로,
외부의 영향을 수동적으로 재생산하고 외부에서 주어진 충동에 묵묵히 복종하는
존재로, 보는 듯하다.
바로 여기에 그의 체계가 체계로서 실패할 수 밖에 없는 오류가 있다.
그는 인간 본성의 안정성을 지나치게 평가절하한다.
그는 인류가 상속받은 자질을,
각 세대가 윗 세대에게서 물려받고 약간의 변경을 주어 다음 세대에
전달하는 자질을,
거의 혹은 완전히 무시한다.
물론, <<법의 정신>>에서 지적된 저 변경 원인들에 대한 적절한 고려가
없는 한, 사회현상도, 그리고 결과적으로 법현상도, 제대로 설명할 수 없다는
것은 틀림없는 진실이다.
그러나 몽테스키외는 그 원인들의 숫자와 힘을 지나치게 과대평가한 듯하다.
그가 나열하고 있는 비정상적 현상들은 거짓된 보고서나
잘못된 해석에 기초한 것이었음이 그후 밝혀졌다.
또한 나머지 것들 중에서도 상당수는 인간 본성의 가변성이 아니라
항구성을 증명하는 것들이니,
그것들은 인류의 이전 단계의 유산이며,
다른 경우라면 받았을 영향력을 끈질기게 거부해온
결과이기 때문이다.
진실은 인간의 정신, 도덕, 신체의 구조에서 안정적 부분이 대부분을
차지한다는 것이다.
그것이 변화에 저항하는 힘은 충분히 커서,
비록 세계의 일부 지역에서 인간 사회의 다양한 변이는 분명 존재하지만,
변화는 그것의 양, 성격, 일반적 방향성을 확인할 수 없을 정도로
그렇게 빠르게 일어나지도 광범위하게 일어나지도 않는다.
우리는
현재 우리가 가지고 있는 지식만을 이용하여 진리에 접근할 수밖에 없지만,
그렇다고
진리가 너무 멀리 있으므로, 혹은 {\small(같은 말이지만)}
장래에 너무 많은 수정이 필요하게 될 것이므로,
그것이 쓸모없고 배울 바가 없다고 생각할 필요는 없는 것이다.

\para{벤담}
주목의 대상이 되어온 또 하나의 이론은 벤담의 역사이론이다.



