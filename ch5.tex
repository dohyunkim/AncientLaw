\chapter{원시사회와 고대법}

법이라는 주제를 과학적으로 다룰 필요성은 현 시대 들어
완전히 망각된 적이 없거니와,
다양한 재능을 가진 인재들이
이러한 필요성의 인식 하에 논문들을 제출해왔다.
그러나, 생각건대,
지금까지 과학의 자리를 대신 차지하고 있던 것은
대체로 일군의 추측이었다는 것에는 의심의 여지가 별로 없다.
앞의 두 장에서 살펴보았던 로마 법률가들의 추측이 바로 그런 것들이다.
이렇게 추정적 자연상태 이론과
그것에 어울리는 법원리의 체계를 인정하고 수용하는 일련의 명시적 진술들이
이것들을 발명한 시대로부터 오늘날에 이르기까지 거의 중단없이 지속되어왔다.
근대 법학을 기초놓은 주석학파\latin{Glossators}의 주석에서도,
이들을 계승한 스콜라주의 법학자들의 저술에서도 그것들은 등장한다.
교회법학자들의 법리에서도 쉽게 눈에 띈다.
문예부흥기에 쏟아져나온 놀라울 정도로 박학다식한
로마법 학자\latin{civilian}들에서는
그것들이 전면에 부상한다.
\wi{그로티우스}와 그 후계자들은 그것들에 명료함과 그럴듯함뿐만 아니라
실무적 중요성도 부여했다.
그것들은 \wi{블랙스톤}의 저서\footnote{%
  <<영국법주해>>(Commentaries on the Laws of England)를 말한다.
}의 서론 장들에서도 읽을 수 있거니와,
이는 뷔를라마키\latin{Jean-Jacques Burlamaqui}의 저서에서
글자 그대로 옮겨적은 것이다.
오늘날 법학도와 실무가들을 위해 출간된 교재들의 첫머리를 장식하고 있는
법의 제1원리에 관한 논의는 언제나 저 로마인들의 가설을
재진술하고 있는 것에 불과하다.
그러나 이들 추측의 고유한 형식뿐 아니라
때로는 스스로를 감추고 있는 위장술에서도
우리는 그것들이 얼마나 교묘하게 인간 정신에 섞여드는지 잘 파악할 수 있다.
로크의 사회계약론에서 법의 기원에 관한 이론은
그 로마적 유래를 거의 숨기지 않거니와,
실로 고대의 견해가 근대인들에게 매력적으로 보이려면
어떤 모습을 갖추어야 하는지를 알려준다.
한편, 동일한 주제에 대한 홉스의 이론은
로마인들과 그 후예들이 생각했던 자연법의 현실성을
부인하기 위해 의도적으로 고안된 것이다.
그러나 영국의 정치인들을 오랫동안 적대적 진영으로 양분했던
이들 두 이론은 양자 모두 인류의 비역사적이고 검증불가능한 상태를
근본적 전제로 삼는다는 점에서 서로 닮아있다.
물론 로크와 홉스는 사회 이전 상태의 성격에 대해서, 그리고
그 상태로부터 우리가 알고 있는 사회 상태로 이월하는 계기가 되는
비상한 행위가 어떤 것이냐에 대해서, 서로 의견을 달리한다.
하지만 원시상태의 사람과 사회상태의 사람 사이에
이들을 갈라놓는 커다란 틈이 있다는 생각에는 일치하거니와,
이 관념이 의식적으로든 무의식적으로든
로마인들에게서 빌려온 것이라는 점에는 의문의 여지가 없다.
사실 법현상을 이들 이론가들이 생각한 방식대로---즉, 하나의 거대한
복합적인 총체로---파악한다면, \paren{그럴듯하게 해석되면}
모든 것을 조화시킬 수 있는 영리한 추측에 의지하여
우리가 우리 스스로 설정한 과업을 자주 회피하게 되더라도,
아니면 절망에 빠져 체계화의 노력을 때로 포기하게 되더라도,
그것은 놀라운 일이 아닐 것이다.

\para{몽테스키외}
로마인들의 법리와 동일한 사변적 기초를 가지는 법이론으로부터
두 명의 유명인사는 제외함이 마땅하다.
그 첫 번째는 \wi{몽테스키외}라는 위대한 이름과 관련된 인물이다.
<<법의 정신>>의 첫 부분에는 다소 모호한 표현들이 나오는데,
저자가 당대의 지배적 견해에 공개적으로 도전장을 제출하기를 꺼려했기
때문이라고 여겨진다.
하지만 저 책의 일반적 흐름은 확실히 그 주제에 관한 이전의 어떤 관념과도
결별하는 모습을 보여준다.
흔히들 지적된 대로,
방대한 조사를 통해 가상의 법체계들로부터 끌어모은 다양한 사례들 속에는,
상스럽고 생경하고 외설스런 습속과 제도들을 특별히 강조함으로써
문명사회의 독자들을 놀라게 하려는 갈망이 뚜렷이 엿보인다.
그것의 일관된 주장은 법이 기후, 지리적 위치, 우연, 기망 따위의
산물---용인할만한 항구성을 가지고 작용하는 것을 제외한 모든 원인의
결실---이라는 것이다.
실로 몽테스키외는 인간의 본성을 전적으로 유연한 것으로,
외부의 영향을 수동적으로 재생산하고 외부에서 주어진 충동에 묵묵히 복종하는
존재로 보는 듯하다.
바로 여기에 그의 체계가 체계로서 실패할 수 밖에 없는 오류가 있다.
그는 인간 본성의 안정성을 지나치게 평가절하한다.
그는 인류가 상속받은 자질을,
각 세대가 윗 세대에게서 물려받고 약간의 변경을 주어 다음 세대에
전달하는 자질을,
거의 혹은 완전히 무시한다.
물론, <<법의 정신>>에서 지적된 저 변경 원인들에 대한 적절한 고려가
없는 한, 사회현상도, 그리고 결과적으로 법현상도, 제대로 설명할 수 없다는
것은 틀림없는 진실이다.
그러나 몽테스키외는 그 원인들의 숫자와 힘을 지나치게 과대평가한 듯하다.
그가 나열하고 있는 비정상적 현상들의 다수는 거짓된 보고서나
잘못된 해석에 기초한 것이었음이 그후 밝혀졌다.
또한 나머지 것들 중에서도 상당수는 인간 본성의 가변성이 아니라
항구성을 증명하는 것들이니,
그것들은 인류의 이전 단계의 유산이며,
다른 경우라면 받았을 영향력을 끈질기게 거부해온
결과이기 때문이다.
진실은 인간의 정신, 도덕, 신체의 구조에서 안정적 부분이 대부분을
차지한다는 것이다.
그것이 변화에 저항하는 힘은 충분히 커서,
비록 세계의 일부 지역에서 인간 사회의 변화는 참으로 명백하지만,
변화는 그것의 양, 성격, 일반적 방향성을 확인할 수 없을 정도로
그렇게 빠르게 일어나지도 광범위하게 일어나지도 않는다.
우리는
현재 우리가 가지고 있는 지식만을 이용하여 진리에 접근할 수밖에 없지만,
그렇다고
진리가 너무 멀리 있으므로, 혹은 \paren{같은 말이지만}
장래에 너무 많은 수정이 필요하게 될 것이므로,
그것이 쓸모없고 배울 바가 없다고 생각할 필요는 없는 것이다.

\para{벤담}
주목의 대상이 되어온 또 하나의 이론은 벤담의 역사이론이다.
\wi{벤담}의 저술 여기저기서 모호하게 \paren{그리고 어쩌면 소심하게}
전개된 이 이론은 <<정부론 단편>>에서 시작되어 최근 존 \wi{오스틴}에 의해 완성된
법개념 분석론과는 사뭇 차별성을 보인다.
법을 특수한 상황에서 부과된 특정한 성격의 명령으로 분석하는 것은
언어의 문제---물론 이것도 자못 무서운 것이지만---로부터 우리를 보호해주는 것
이상을 하지 못한다.
그러한 명령을 부과하는 사회적 동기가 무엇인지,
그러한 명령들 사이의 관계는 어떠한지,
종래의 명령을 대체한 새로운 명령이 종래의 것에 대해
어떤 의존성을 갖는지 등에 대해서는
아무 것도 답해주지 않는다.
벤담이 제공하는 답변은,
일반적 공리\hanja{功利}에 관한 사회의 견해가 변함에 따라
사회는 자신의 법을 변경해왔고 또 변경하고 있다는 것이다.
이 명제가 거짓이라고 말하기는 어렵겠지만,
확실히 별로 실속은 없는 명제로 보인다.
법규칙을 변경할 때 한 사회에게, 더 정확히는 그 사회의 통치 부분에게,
공리로 여겨지는 것은 변경을 만들어낼 때 그것이 가지는 어떠한 목표와도
정확히 같은 것이기 때문이다.
공리나 최대의 선\hanja{善}이란 결국 변경을 추동하는 힘의 다른 이름에 불과하다.
우리가 법이나 여론의 변화 규칙으로 공리를 주장할 때,
이 명제로부터 우리가 얻는 것이라고는
변화가 일어나고 있다고 말할 때 거기에 암시되어 있는 용어를
명시적인 용어로 대체하는 것말고는 아무 것도 없다.

\para{적절한 탐구방법}
이렇게 기존의 법이론에 대한 불만이 널리 퍼져 있기에,
또한 그들이 해결한다고 내세운 문제가 실제로는 전혀 해결되지 않고 있다는
확신이 너무나 일반적이기에,
완전한 결과를 얻기 위해 필요한 어떤 탐구방법을 저 이론가들이
불완전하게 따랐거나 아니면 전적으로 무시한 것이 아닌가 하는 의심이
뒤따를 수밖에 없는 것이다.
실로, 아마도 \wi{몽테스키외}의 것을 제외한 저 모든 사변적 이론들은
한 가지를 철저하게 무시했다.
저 이론들이 등장한 특정 시대로부터 멀리 떨어진 시대의 법이 실제로
어떠했는가에 대해 그들은 전혀 고려하지 않는다.
저 이론의 창시자들은 그들 자신의 시대와 문명의 제도에 대해서는,
그리고 그들이 어느 정도 지적으로 공감하는 다른 시대와 문명의 제도에 대해서는
주의 깊게 관찰했지만,
그들 자신의 사회와 많은 외관상의 차이를 가진 초기사회의 상태에
관심을 돌릴 때면 누구도 예외없이 관찰하기를 중단하고 추측하기를 시작했다.
따라서 그들이 저지른 잘못은 물질적 우주의 법칙을 탐구하려는 자가
가장 단순한 구성요소인 입자로부터 시작하는 대신에
기존의 물질세계 전체를 명상하는 것으로부터 시작하는 오류에 비견될 만하다.
다른 사고 영역보다 법학의 영역이라고 해서 이런 과학적 오류가 더 많이
용서받을 수 있다고 생각하는 사람은 분명 아무도 없을 것이다.
먼저, 가능한 한 원초적 상태에 가까운 가장 단순한 사회형태로부터
출발해야 할 것이다.
다시 말해, 이러한 탐구의 통상적인 과정을 따르고자 한다면,
우리는 원시사회의 역사를 가능한 한 멀리 거슬러올라가야 한다.
초기 사회가 보여주는 현상은 처음에는 이해하기 쉽지 않겠지만,
이러한 이해의 어려움은 우리를 당혹케하는 현대 사회 구조의
난해한 복잡함에 비하면 아무 것도 아니다.
그것은 생경함과 상스러움에 기인하는 어려움일 뿐,
초기사회의 숫자나 복잡성에 기인하는 것이 아니다.
현대적 관점에서 바라볼 때 만나는 놀라움을 극복하기가
쉽지 않은 것일 뿐, 이것을 극복하고 나면 그것들은 충분히 그 수가 적고
또한 충분히 단순하다.
하지만, 비록 그것들이 생각보다 많은 어려움을 준다 할지라도,
오늘날 우리의 행위를 규율하고 있고 우리의 행동을 형성하고 있는
모든 형태의 도덕적 제한이 전개되어나올 맹아를 확인하기 위해 들이는
우리의 고통은 결코 낭비라 할 수 없을 것이다.

\para{타키투스의 게르마니아}
우리가 알고 있는 원초적 사회 상태는 세 가지 전거\hanja{典據}에 기초한다:
\phantomsection\label{contemporary}%
당대의
관찰자들이 그들보다 문명의 진보 수준이 낮은 사회를 기술\hanja{記述}한 것,
특정 민족이 자신들의 초기 역사에 관해 기록한 것,
그리고 고대법이 그것이다.
첫 번째 종류는 우리가 기대할 수 있는 가장 좋은 것이다.
사회들은 동시에 진보하는 것이 아니라 진보의 정도가 서로 다르기 때문에,
체계적인 관찰의 습관을 가진 사람들이 인류의 유년기에 놓여있는 사람들을
관찰하고 기술할 수 있는 입장에 설 수가 있다.
\wi{타키투스}가 바로 그런 기회를 잘 활용했다.
하지만 <<게르마니아>>\latin{Germany}는 다른 많은 고전 저술과는 달리
저 저자의 모범적인 전례\hanja{前例}를 따르는 다른 사람들을 갖지 못했으며,
따라서 이런 종류의 전거로 우리에게 전해지는 것은 극히 적다.
문명화된 민족이 이웃의 미개인들에 대해 가지는 오만한 경멸심은
그들을 관찰함에 있어 뚜렷한 과실\hanja{過失}을 낳았거니와,
때로는 두려움으로 인해, 때로는 종교적 편견으로 인해,
때로는 바로 그 용어---사람들에게 그저 정도의 차이가 아니라 질적인 차이가
있다는 인상을 주는 `문명'\latin{civilisation}과 `미개'\latin{barbarism}라는
용어---의 사용에 의해서도
이러한 부주의는 가중되었다.
몇몇 비평가들에 의해
<<게르마니아>>조차도
날카로운 대비와 선명한 이야기를 위해 정확성을 희생시켰다는 의심을 받고 있다.
나아가, 자신들의 유년기를 말하고 있는 민족들의 고문헌 가운데
우리에게 전해지고 있는 역사 서술 또한,
민족의 자긍심에 의해 혹은 새 시대의 종교적 감정에 의해
적잖이 왜곡되었다고 평가되어왔다.
그런데, 근거있는 의심이든 근거없는 의심이든 이러한 의심이
대부분의 초기법에는 주어지지 않는다는 데 주목할 필요가 있다.
우리에게 전해지는 옛 법의 상당수는
단순히 옛 법이라는 이유로 보존되었다.
이 법을 적용하고 준수했던 이들은 그것을 이해한다고 내세우지 않았으며,
경우에 따라서는 비웃고 멸시하기까지 했다.
그들은 조상으로부터 전래되었다는 것을 제외하고는 그것을 설명하려 하지 않았다.
따라서, 가공되지 않았을 것이라고 합리적으로 믿을 만한 옛 법의 단편들에
주의를 집중하면, 우리는 그것들이 속했던 사회의 몇몇 중요한 성격에 대해
명료한 관념을 획득할 수 있게 된다.
한 걸음 나아가, 이렇게 얻은 지식을 \wi{마누법전}처럼
전체적으로
진정성이
의심되는 법체계에 활용할 수도 있을 것이다.
우리가 얻은 열쇠를 사용하여 진정으로 원시적인 부분과
편견, 이해관계, 무지 등에 의해 영향받은 부분을 분리해낼 수 있는 것이다.
이러한 탐구를 위한 자료가 충분하다면,
그리고 비교가 정확히 이루어진다면,
우리가 따르는 방법은 놀라운 결과를 이끌어낸 비교언어학의 방법만큼이나
반대할 거리가 적을 것이라고 인정해도 좋을 것이다.

\para{가부장제 이론}
비교법\latin{comparative jurisprudence} 연구에서 나온 증거로부터
\wi{가부장제} 이론\latin{patriarchal theory}이라는
인류의 원시상태에 관한 이론이 수립되었다.
물론 이 이론은 원래 하\hanja{下}아시아\latin{Lower Asia} 지역%
\footnote{유프라테스강 이남의 아시아 지역을 말한다.}
히브리 민족의 가부장들에 관해 성서에 기록된 역사에 기초한 것이 틀림없다.
그러나 전술했듯이 성서와의 연계성은 그것이 완전한 이론으로
받아들여지는 데 방해가 되었다.
최근까지 사회현상들의 결합관계를 열성적으로 연구한 연구자들의 다수가
히브리 고대를 타기시하는 아주 강한 편견에 사로잡혀 있었거나
그들의 이론 체계를
종교 기록의 도움 없이
구성하고자 하는
아주 강한 욕망에 사로잡혀 있었기 때문이다.
지금도 성서의 기록을 폄하하는 경향,
보다 정확히는 그것을 일반화해서
셈족의 전통의 일부를 구성하는 것으로 인정하기를 거부하는
경향이 남아있는 듯하다.
하지만 중요한 점은 \wi{가부장제}의 법적 증거가
로마인, 인도인, 슬라브인이 대다수를 점하는
인도^^b7유럽 계통에 속하는 사회들의 제도에서 거의 전적으로 발견된다는 것이다.
실로 탐구의 현 수준에서 문제되는 것은 오히려 어디에서 그칠 것인가를 아는 것,
애초에 사회가 가부장제 모델에 입각하여 조직되었다고 할 수 \hemph{없는}
민족은 어떤 민족들인가를 확인하는 것에 있다고 할 것이다.
창세기 앞 부분 몇 개의 장에서 수집되는 가부장제 사회의 주요 특징들을
여기서 상세히 묘사할 생각은 없다.
우리 대부분은 어렸을 때부터 그것을 익히 들어왔기 때문이고, 또한
로크와 필머 간의 논쟁에서 그 명칭이 유래한 가부장권 논쟁에 대한
한때의 관심으로 인해
영국의 문헌들에서는
그것에 관해
별 쓸모도 없으면서
장\hanja{章} 하나를 통째로 할애하고 있기 때문이다.
저 역사의 표면에 드러난 요점은 이런 것들이다:
가장 나이 많은 남자 어른---가장 연장인 선조---이 그의 가\hanja{家}를
절대적으로 지배한다.
\wi{생사여탈권}을 포함하는 그의 지배권은 그의 노예들뿐만 아니라
그의 자식들과 가족들에 대해서도 무제한적이다.
사실 아들의 지위와 노예의 지위는 거의 차이가 없으며,
다만 아들은 언젠가는 가의 수장이 될 가능성이 더 크다는 점에서 더 높은 대우를
받을 뿐이다.
자식들의 양떼와 소떼는 아버지의 양떼와 소떼이며,
아버지의 재산---소유권자라기보다는 가의 대표자로서 점유하는 것인데---은
그의 사망시에 일촌\hanja{一寸} 자손들에게 균등 분배된다.
때로 먼저 태어났다는 이유로 장자가 두 배의 몫을 차지하기도 하지만,
일반적으로는 약간의 명목상의 우선권 외에는 장자가 상속에서 더 유리한
지위를 갖지 않는다.
조금은 덜 분명한 어떤 추론이 성서의 기록으로부터 나올 수 있겠는데,
가부장의 제국에서 이탈한 최초의 흔적을 엿볼 수 있을 듯하다.
야곱의 가족과 에서의 가족은 서로 분리되어 두 개의 민족을 형성하지만,
야곱 자식들의 가족들은 하나로 뭉쳐 하나의 인민이 되는 것이다.
이것은 국가 상태로 나아가는,
가족관계에 기초한 자격보다 권리에 기초한 자격이 우선하는 상태로 나아가는
때이른 맹아가 아닌가 한다.

\para{가족집단}
법학자로서의 특수한 목적을 위해서
역사의 여명기에 인류가 처해있던 상황의 특징을 간략하게 표현하고자 한다면,
\phantomsection\label{cyclops}%
나는 \wi{호메로스}의 <<\wi{오뒷세이아}>>에서 몇 구절을 인용하고 싶다.
\begin{center}
  \vskip-\smallskipamount
  \greekfont\latinmarks
  \begin{tabular}{l}
    τοῖσιν δ᾽ οὔτ᾽ ἀγοραὶ βουληφόροι οὔτε θέμιστες\rlap{.}\\
    \hfill$\cdot$\hfill$\cdot$\hfill$\cdot$\hfill θεμιστεύει δὲ ἕκαστος\\
    παίδων ἠδ᾽ ἀλόχων, οὐδ᾽ ἀλλήλων ἀλέγουσιν.
  \end{tabular}
\end{center}
``그들에게는 회의하는 집회도 없었고 \wi{테미스테스}도 없었다. \ldots{}
하지만 그들 각자는 부인들과 자식들에 대해 재판권을 행사했거니와,
이에 관해 서로 아무런 간섭도 하지 않았다.''\footnote{\latin{Hom. Od. 9.112--115}}
이 인용문은 퀴클롭스에 해당하는 대목인데,
호메로스가
문명의 진보 수준이 낮은 외국인의 전형\hanja{典型}으로서
퀴클롭스를
묘사했다고 보아도 완전히 비현실적인 생각은 아닐 것이다.
원시공동체는 자신과 전혀 다른 풍속을 가진 사람들에 대해 느끼는
거의 본능적인 혐오감으로 인해 그들을 대개 거인 따위의 괴물로,
혹은 \paren{동양의 신화에서 거의 항상 등장하는} 악령으로
묘사하는 것이다.
어쨌거나, 저 싯귀에는 우리가 고대법에서 얻을 수 있는 힌트의 요지가 담겨있다.
처음에는 사람들이 완전히 고립된 집단들로 분산되어 있었고,
각 집단은 가부장에 대한 복종으로 결합되어 있었다.
가부장의 말이 곧 법이지만, 그것은 본서 제1장에서 분석한
테미스테스의 단계는 아직 아니다.
이 초기 법관념이 형성되기 시작하는 사회상태로 한 걸음 더 나아가면,
여전히
법관념은 전제\hanja{專制}적 가부장의 명령을 특징짓던 신비롭고 자생적인 성격을
어느 정도 가지고 있지만,\footnote{%
  제1장에서 테미스테스는 신적 영감에 기초한 고립된 판결들이었음을 상기할 것. }
명령이 한 명의 주권자에서 비롯되는 것인 한,
그것은 가족집단들이 좀 더 넓은 범위의 조직으로 결합하는 것을
전제\hanja{前提}한다.
이제 문제는 이 결합의 성질이 무엇이냐, 그리고
결합에 따른 친밀성의 정도는 어떠하냐, 라는 것이다.
바로 여기서 고법\hanja{古法}이 우리에게 큰 기여를 하게 되거니와,
그렇지 않다면 단지 추측으로밖에 연결할 수 없는 틈새를 메워주고 있는 것이다.
어느 지역에서나 그것은 원시시대의 사회가 오늘날 우리가 상정하는
\hemph{개인들}의 집합이 아니었음을 보여주는 명백한 증거들로 가득하다.
사실, 그리고 그 구성원들의 견해에 의하더라도,
그것은 \hemph{가}\hanja{家}\hemph{들의 집합체}였다.
이 대비는
고대사회의 \hemph{단위}는 가족이었고
현대사회의 그것은 개인이라는 말에 의해서
자못 강력하게 표현된다.
우리는 이 차이가 가져오는 모든 결과를 고대법에서 기꺼이 발견할 수 있어야 한다.
고대법은 작은 독립된 단체\latin{corporation}들의 체계에 부응하도록 짜여져있었다.
그리하여 고대법은 드문드문 규율할 뿐이니,
가부장들의 전제적 명령에 의해 보충될 것이기 때문이다.
고대법은 의례\hanja{儀禮}에 관한 것이니,
그것이 관심을 두는 관계는
개인들 간의 신속한 교섭보다는
국제 관계를 닮은 것이기 때문이다.
무엇보다, 고대법은 오늘날에는 볼 수 없는 사뭇 중요한 특성 하나를 가진다.
그것의 \hemph{생명}에 관한 견해는
발달된 법체계에서 보이는 것과는 완전히 다른 것이다.
단체는 \hemph{죽지 않는다}.
따라서 원시법은 그것이 다루는 대상, 즉 가부장적^^b7가족적 집단들을
영구적이고 소멸불가능한 것으로 취급한다.
이 견해는 먼 고대의 도덕적 태도에서 보이는 특정 측면과 밀접하게 관련되어 있다.
개인의 도덕적 상승과 도덕적 하락은 그 개인이 속한 집단의 공과\hanja{功過}와
동일시되거나 혹은 그보다 후순위로 밀렸다.
공동체가 죄\latin{sin}를 범하면
그것은 그 구성원들이 저지른 범죄의 총합보다 더 큰 죄가 된다.
그러한 범죄는 단체의 행위이고,
범죄의 결과는 직접 실행에 가담한 자들을 넘어 더 큰 범위에까지 미친다.
한편, 명백히 개인이 범죄를 저질렀다 할지라도,
그의 자식이나 친족이나 부족이나 동료시민들이 그와 함께,
때로는 그를 대신해서, 벌을 받는다.
그리하여 도덕적 책임과 보복의 관념은
더 진보된 시기보다는 먼 고대에 훨씬 더 확실하게 실현되었거니와,
가족집단은 불멸이었고 그것의 형벌 책임은 무제한이었기에,
원시적 정신은 개인이 집단으로부터 완전히 분리되면서 나타나는 곤란한 문제들에
구애받지 않았기 때문이다.
이러한 고대의 단순한 견해로부터 후대의 신학적^^b7형이상학적 설명의 방향으로
한 걸음 더 나아가면,
`\wi{저주의 상속}'\latin{inherited curse}이라는 초기 그리스의 관념을 만나게 된다.
최초의 범죄자로부터 그의 후손들이 상속받는 것은
형벌을 감수할 책임이 아니라,
합당한 보복을 초래할
새로운 범죄를 저지른 데 대한 책임이었다.
그리하여 가족의 책임은
범죄를 실행한 개인에게 범죄의 결과를 국한시키는 새로운 사고의 국면과
양립하게 되었다.

\para{친족이라는 의제}
전술한 성서의 사례가 제공하는 힌트만으로
일반적 결론을 얻는다면,
그리하여
가부장의 사망 후 가족이 분리되는 대신 하나로 단결하는 경우 어디서나
공동체가 존재하기 시작한다고 생각한다면,
그것은 사회의 기원에 관한 지나치게 단순한 설명이 될 것이다.
그리스의 대다수 국가들과 로마에서는
일련의 점증하는 집단들이 모여서 최초의 국가를 구성한
흔적이 오랫동안 남아있었다.
그것의 전형적인 예로서
로마의 가족\latin{family}, \wi{씨족}\latin{house}, 부족\latin{tribe}을
들 수 있거니와,
이들은 동일한 중심으로부터 점점 확장해가는 동심원들의 체계로
이해하지 않을 수 없다.
가장 기본되는 집단은 가족으로, 이는
제일 높은 남자 어른에게 함께 복종하는 관계로써 결합된다.
가족들의 집합이 씨족\latin{gens; house}을 형성한다.
씨족들의 집합은 부족을 만든다.
그리고 부족들의 집합이 국가를 구성한다.
이러한 예시를 따라 우리는 국가를
최초의 가족이라는 조상에서 유래한 공통의 후손들이 결합한
사람들의 집합체라고 이해해도 좋은 것일까?
이에 관하여 적어도 확실한 점은
모든 고대사회는 스스로를 공통의 계통에서 유래한 것으로 간주했을 뿐만 아니라,
이런 이유 외에는 정치적 결합체를 묶어주는
다른 어떤 이유도 생각할 수 없다고 믿고 있었다는 것이다.
사실, 정치적 관념의 역사는
피를 나눈 친족이야말로
정치적 기능에 있어 공동체의 유일한 근거라는 가정에서 출발한다.
어떤 다른 원리---가령 \hemph{지리적 근접성}\latin{local contiguity}
같은 것---가 역사상 최초로 공동의 정치행위의 근거로 등장함으로써
일어난 변화만큼 그렇게 놀랍고 그렇게 완전한 감정의 전복\hanja{顚覆}은,
강하게 표현하자면 혁명은, 이 세상에 존재하지 않는다.
따라서 초기 국가의 시민들은
그들이 소속된 모든 집단을 공통의 혈통에 기초한 것으로 간주했다고
인정해도 좋을 것이다.
가족에게 타당한 것은 우선 씨족에도, 다음으로 부족에도, 끝으로 국가에도 타당했다.
그런데
각 공동체는
이러한 믿음---혹은, 용어가 허용된다면, 이러한 이론---과 더불어,
이 근본가정이 틀렸음을 명확히 보여주는 기록이나 전승\hanja{傳承}들도
보존해왔다는 것을 우리는 알고 있다.
그리스 국가들을 보더라도, 로마를 보더라도,
니부르\latin{Barthold Georg Niebuhr}\footnote{팔림프세스트되었던
  가이우스의 <<법학제요>>를 발견해냈던 바로 그 사람. 본문의 내용은
  그의 주저 <<로마사>>에서 로마 씨족을 설명하는 과정에 원용된 사례들로 보인다.}%
에게 다양한 귀중한 사례들을 제공했던
디트마르쉔\latin{Ditmarsh} 지방의 튜턴족 귀족들을 보더라도,
켈트족의 \wi{씨족}연합을 보더라도,
최근에 관심의 대상이 된
슬라브족 러시아인과 폴란드인의 유별난 사회조직을 보더라도,
어디서나 우리는
이방인 혈통 사람들이 원주민 사회에 받아들여지고 서로 혼합되는
역사 기록의 흔적을 발견한다.
다시 로마로 눈을 돌리면,
\wi{입양}의 관행에 의해
기본 집단인 가족에 지속적으로
불순물이 섞이고 있었을 뿐만 아니라,
원래의 부족들 중 하나가 극적으로 축출당한 이야기라든가,%
\footnote{에트루리아 혈통의 왕과 그 일족이 축출되고 공화정이 수립된 사건을
말하는 듯하다.}
초기 왕들 중 하나에 의해 씨족들에 대규모 편입이 이루어진 이야기\footnote{%
  툴루스 호스틸리우스가 알바 롱가(Alba Longa)를 파괴하고
  그 주민들을 로마로 이주시킨 사건을 말하는 듯하다.}
따위가 항상 사람들의 입에 오르내리고 있었다.
자연적이라고만 여겨졌던 국가의 구성이 실은 다분히 인공적이었던 것이다.
믿음 혹은 이론과 명백한 사실 간의 이러한 불일치는 언뜻 보면
우리를 무척 당혹케 한다.
하지만 이것이 실제로 보여주는 바는
사회의 유년기에는 법적의제\latin{legal fiction}가 대단히 효율적으로 작동했다는 것이다.
가장 먼저 그리고 가장 널리 사용된 \wi{법적의제}는
가족관계를 인공적으로 만들어내는 것이었으니,
생각건대 이보다 더 강하게 인류가 빚지고 있는 것은 없다고 할 것이다.
이것이 존재하지 않았다면
어떤 원시집단이, 그 집단의 성격이 어떠하든,
어떻게 다른 집단을 흡수할 수 있었을 것이며,
또한 두 집단이,
한쪽이 절대적 우위를 점하고 다른 쪽이 절대적 종속에 들어가는 것을 제외하면,
어떻게 하나로 결합할 수 있었을 것인지
도무지 알지 못하겠다.
물론,
근대적 관점에서 두 공동체의 결합에 관해 생각할 때,
우리는 그것을 만들어낼 수 있는 수없이 많은 방법들을 제안할 수 있을 것이다.
가장 간단한 것으로는 통합하려는 집단의 구성원들이 투표를 하거나
아니면 지리적 근접성에 기초하여 함께 행동하는 것을 생각할 수 있다.
그러나,
단지 동일한 지리적 영역 안에 살고 있다는 이유만으로
다수의 사람들이 공동으로 정치적 권리를 행사한다는
관념은 원시적 고대에는 전적으로 생경하고 전적으로 기괴한 것이었다.
당시에 널리 애호된 방법은
편입되는 집단이 자신들을 편입하는 집단과 동일한 계통의 후손이라고
\hemph{가장}\hanja{假裝}\hemph{하는} 것이었다.
이러한 의제를 당사자들이 얼마나 믿었는지,
이러한 의제는 그것이 모방하려는 현실과 얼마나 가까웠는지,
지금의 우리로서는 알 수 없다.
하지만 반드시 잊지 말아야 할 점은,
다양한 정치집단을 형성한 사람들은
그들의 연맹을 인정하고 축성\hanja{祝聖}하기 위해
주기적으로 함께 모여
공동의 제의\hanja{祭儀}를 개최하는 습속을
분명 가지고 있었다는 것이다.
원주민 사회에 혼합되어 들어간 이방인들도 이러한 공동제의에 참가했을
것임은 물론이거니와,
이것이 일단 행해지고 나면 그들을 공통의 혈통으로 인정하는 것이
쉬우면 더 쉬웠지 더 어렵게 되지는 않았을 것이다.
따라서 증거를 통해 얻을 수 있는 결론은,
모든 초기 사회가 동일한 조상에서 유래한 후손들로 구성되었다는 것이 아니라,
지속성과 공고함을 조금이라도 가진 모든 초기 사회는 그러한 후손들 또는
그러한 후손이라고 의제된 사람들로 구성되었다는 것이다.
무한한 수의 원인들이 원시집단들을 흩어지게 만들었을 것이나,
그들이 재결합할 때면 그것은 언제나 친족관계의 모델 혹은 원리에
기초했다.
사실이야 어떻든 간에, 사상과 언어와 법은 무엇이든 이 가정에 적응했다.
그러나,
우리에게 기록을 남겨준 공동체들에 관하여
이 모든 것이
타당하다 하더라도, 그들의 나머지 역사는
아무리 강력한 \wi{법적의제}라도 일시적이고 기한부의 영향력만 가진다는
전술한 약점을 안고 있는 역사였다.
역사의 어느 시기에 이르면---아마도 그들이 외부의 압력에 저항할 수 있을 만큼
강력해졌다고 믿기 시작하면서---이들 모든 국가는
혈족관계를 의제적으로 확장함으로써 인구를 충원하는 일을 그만두었다.
그리하여,
이제 공통의 기원이 아닌 다른 이유로 새로운 인구를 끌어들이게 된 곳이라면
어디서나 필연적으로 귀족정이 시작되었다.
실제적인 것이든 가상의 것이든
피로 연결된 관계가 아니고는 정치적 권리를 획득할 수 없다는
핵심원리를 완고하게 유지하던 그들이 이제
훨씬 더 생명력이 강한 것으로 드러난
어떤 다른 원리를
그들의 피지배자들에게
가르치기 시작했다.
그것은 바로 \hemph{지리적 근접성}의 원리였거니와,
이는 오늘날 어디서나 정치적 공동체의 필수조건으로 인정되고 있다.
동시에 새로운 정치관념이 들어서게 되었으니,
이 관념은,
우리 영국인들의 것이자 우리와 동시대 사람들의 것이며
다분히 우리 조상들의 것이기도 한 까닭에,
그것이 정복하고 폐위시킨 옛 이론에 대한 우리의 인식을 다소간
방해하고 있다.

\para{가부장권}
그리하여 가족은 초기사회의 있을 수 있는 모든 변이에도 불구하고
그것의 전형\hanja{典型}이었던 것이다.
하지만 여기서 말하는 가족은 현대의 우리가 이해하는 가족과 똑같은 것이 아니다.
고대적 관념에 접근하기 위해서는 현대의 관념을 한편으로는 확장하고
한편으로는 축소해야 한다.
고대의 가족은
외부인을 경계선 안으로 흡수함으로써
지속적으로 확장되는 것이었다.
\wi{입양}의 의제는 현실의 친족관계를 그대로 모방하는 것이었기에,
현실적 관계와 입양에 의한 관계 간에는
법적으로든 여론상으로든
거의 차이가 없었다.
한편, 이론적으로 가족의 일원으로 편입된 자는
기존 가족 성원들과 함께 그들의 가장 높은 살아있는 조상---아버지든,
할아버지든, 증조부든---에게
공통의 복종을 바치는 것으로써 사실상 하나로 결합된다.
가족집단이라는 관념에서 가장\hanja{家長}의 \wi{가부장권}은
그의 슬하에 태어났다는 사실\paren{또는 의제된 사실}만큼이나
필수불가결한 요소였다.
따라서, 아무리 진정한 핏줄로써 가족이 되었다 할지라도
가부장의 제국에서 사실상 퇴출된 사람은,
초기의 법에서는,
결코 가족의 일원으로 간주되지 않았다.
원시법의 초입에서 우리가 만나는 것은
이와 같이 가부장을 중심으로 한 집합체---현대의 가족보다
한편으로는 축소된 것이자 다른 한편으로는 확장된 것---인 것이다.
아마도 국가보다, 부족보다, 씨족보다도 더 오래된 이것은
씨족이나 부족이 잊혀진 이후에도 오랫동안,
혈연이 국가의 구성과 무관해진 이후로도 오랫동안,
사법\hanja{私法}에 그 흔적을 남기게 된다.
그것은 법의 모든 분야에 자신의 각인을 남겼으며,
생각건대 이들 법 분야의 가장 중요하고도 가장 지속력있는 성격의 다수가
그것에서 흘러나왔던 것이다.
초기의 대부분의 고대국가에서 보이는 법의 특징들로부터 불가피 도출되는 결론은,
오늘날 유럽 전역을 지배하고 있는 권리의무의 체계가 개인을 취급하고 있는 것과
정확히 동일한 관점에서 그때의 법은 가족집단을 취급했다는 것이다.
우리가 관찰할 수 있는 당시의 사회 가운데는
이러한 원시적 조건으로부터 생겨났다고 보아서는
설명하기 어려운 법과 관행을 가진 사회가 없지는 않다.
그러나 보다 운 좋은 상황에 놓여있던 공동체들에서는
법체계가 점점 분화되어 갔는데,
이러한 분화과정을 면밀히 살펴보면
우리는
그것이 법체계 중 가족이라는 원시적 관념에 의해 강하게 영향받았던 부분에서
주로 일어났음을 알게 된다.
무엇보다 중요한 사례인 로마법의 경우,
이러한 변화가 무척 천천히 일어났기에,
우리는 시대별로 이 변화의 경로와 방향을 관찰할 수 있을 뿐만 아니라,
그것이 결국 도달하게 될 결과까지도 어느 정도 짐작할 수 있을
정도이다.
방금 말한 이 탐구를 수행함에 있어 우리는 고대사회와 근대사회를
가르고 있는 가상의 장벽때문에 탐구를 중단할 필요가 없다.
로마의 세련된 법과 만족\hanja{蠻族}들의 원시적 관행이 혼합되어
나타난 결과의 하나, 즉 우리에게 \wi{봉건제}도라는 기만적인 이름으로 알려진 것은
로마 세계에서 사라졌던 고법\hanja{古法}의 많은 특징들을 되살려낸 것이었으니,
이미 끝난 줄 알았던 분화과정이 다시 시작된 것이며 어느 정도는 지금도
진행되고 있는 것이기 때문이다.

\para{로마의 가부장권}
초기 사회의 가족조직은
후손들과 그들의 재산에 대해
아버지 등의 조상이 평생동안 행사하는 권력에 관해
몇몇 법체계에 뚜렷하고 폭넓은 자국을 남겼다.
이 권력을 나중에 로마인들이 이름붙인
`\wi{가부장권}'\latin{patria potestas}이라는 용어로 편의상 부르기로 하자.
원초적 인간관계의 특징 중에 이보다 더 많은 증거를 보여주는 것도 없지만,
진보적 공동체의 관행으로부터 이보다 더 널리 그리고 더 빠르게
소멸해버린 것도 없을 것이다.
안토니누스 황조 시대에 활동했던 \wi{가이우스}는
가부장권이 로마에 특유한 제도라고 말한다.
물론, 만약 그가 라인강과 다뉴브강 건너편으로 눈을 돌려
당대 몇몇 사람들의 호기심을 자아내던 미개 부족들을 바라보았다면,
거기서 그는 조야한 형태의 가부장권의 사례들을 발견할 수 있었을 것이다.
멀리 동쪽에서는, 로마인이 갈라져나온 민족 계통과 동일한 줄기에서
뻗어나온 가지 하나가 로마의 가부장권을 몇몇 법기술적 측면에서
그대로 반복하고 있었다.\footnote{%
  유럽인들과 함께 인도^^b7유럽어족에 속하는 인도인을 의미한다. }
그러나
로마 제국 안에 포함된다고 생각되던 민족들 중에서
로마의 ``\wi{가부장권}''과 유사한 제도를 가진 민족으로
가이우스가
발견한 것은
아시아의 갈라티아인\latin{Galatae}을 제외하고는
전혀 없었다.\footnote{%
  \latin{Gai.\,1.55.}
  갈리아족의 일파로 멀리 아나톨리아 지방에 이주해 정착한 민족이다. }
사실, 생각건대,
왜 조상의 직접적 권력이
다수의 진보하는 사회에서는
얼마 안 가서 초기 상태보다 미약해졌을까 하는 것에는
그럴 법한 이유가 있다.
버릇없는 자식이 아버지에게 맹목적으로 복종하는 것을
자식이 자신의 이해관계를 계산했기 때문이라고 설명해버리고 마는 것은
물론 부조리하기 짝이 없는 일이지만,
동시에, 자식이 아버지에게 복종하는 것이 자연스러운 일이라면,
자식이 아버지에게서 뛰어난 힘이나 뛰어난 지혜를 기대하는 것도 똑같이
자연스러운 일이다.
그리하여,
육체적 힘이나 정신적 능력에 특별한 가치를 부여하는 사회에서는,
그 보유자가 능력있고 힘있는 경우에만 한정하여
가부장권을
인정하려는 경향이 나타난다.
조직화된 그리스 사회를 일별하여 얻은 우리의 인상에 의하면
아버지의 육체적 힘이 쇠약해졌더라도 아버지의 뛰어난 지혜가
그의 권력을 계속 유지시켜주는 듯하지만,
<<\wi{오뒷세이아}>>에 나오는
오뒷세우스와 라에르테스\latin{Laertes}의 관계처럼,\footnote{%
  라에르테스는 아들 오뒷세우스가 트로이 전쟁에 참전하기 위해
  이타케를 떠나기 전에 이미 통치권을 아들에게 물려주고 낙향했다.}
아들에게서 특별한 용기와 지혜를 모두 발견할 수 있다면
노쇠한 연령의 아버지는 가부장의 자리에서 물러났던 것이다.
후기 그리스 법에서는
호메로스의 문학에 암시된 관행에 더욱 진전이 이루어졌거니와,
비록 엄격한 가족적 의무의 흔적이 여전히 많이 남아있었지만,
이제 아버지의 직접적 권력은, 오늘날 유럽의 법전들처럼,
자식이 미성숙 또는 미성년인 경우에만, 다시 말해 그들의 정신적^^b7육체적
열등함이 추정되는 기간에만 국한되었던 것이다.
하지만 로마법은,
국가적 필요가 있는 때에 한해서만 고대 관행을 혁신하는 특이한 경향 덕택에,
원초적 제도와 내가 말한 그것의 자연적 한계, 양자 모두를 보존했다.
집합적 공동체에 관련된 생활관계에서는,
자문을 구하기 위해서든 전쟁을 치르기 위해서든
시민의 지혜와 힘을 이용하게 되는 그 어떤 경우에 있어서도,
가부장권에 복속해있는 아들\latin{filius familias}은 그의 아버지 못지않게
자유로웠다.
``\wi{가부장권}은 공법\latin{jus publicum}에는 미치지
않는다''는 것이 로마법의 법언이었다.
아버지와 아들은 시민으로서 함께 투표했고, 전장에서는 나란히 싸웠다.
사실, 장군인 아들이 아버지에게 명령을 내리는 것도 가능했고,
정무관인 아들이 아버지의 계약을 재판하거나
아버지의 범죄를 벌하는 일도 있을 수 있었다.
하지만 사법\hanja{私法}상의 모든 관계에서는
아들은 아버지의 전제권력 아래 살았으니,
끝까지 계속된 가부장권의 가혹함과
이 제도가 유지된 장구한 기간을 생각할 때,
이는 법제사에서 가장 이해하기 힘든 문제의 하나에 해당한다.

우리에게 원시 가부장권의 전형\hanja{典型}일 수밖에 없는
로마인들의 가부장권은
문명사회의 제도로서도
그 신분법적 측면에서나 그 재산법적 측면에서나
이해하기 어려운 제도이다.
역사의 이 틈새를 보다 완전히 메울 수 없음이 한스러울 따름이다.
\wi{신분법}적 측면에 관한 한,
우리가 알고 있는 가장 초기부터,
아버지는 자식들에 대해 \wi{생사여탈권}\latin{jus vitae necisque}을 가졌고,
게다가 무제한적인 체벌의 권한을 가졌다.
그는 자식들의 신분법적 지위를 마음대로 바꿀 수 있었다.
그는 아들에게 배우자를 정해줄 수 있었고, 딸을 \wi{혼인}시킬 수 있었다.
그는 아들이든 딸이든 자식을 이혼시킬 수 있었다.
그는 그들을 다른 가\hanja{家}에 \wi{입양}보낼 수 있었고, 그들을 팔아버릴 수 있었다.
나중에 제정기에 이르면 이 모든 권한은, 여전히 그 흔적이 남아있기는 했으나,
사뭇 좁은 범위로 축소된다.
가내 체벌에 관한 무제한적 권리는 이제
정무관에게 가내 범죄를 고발하는 권리가 되었다.
혼인을 명령할 특권은 조건적인 거부권 정도로 약화되었다.
자식을 팔아버릴 자유는 사실상 폐지되었다.
입양의 경우, 이제 양부에게 입적되는 아들의 동의가 없이는
무효가 되었으며, 후에 \wi{유스티니아누스}의 개혁에 의해
입양의 고대적 성질은 거의 전부 사라져버린다.\footnote{%
  \latin{Inst.\,1.11.}
  원칙적으로 가부장권이 이전하지 않는다.
  따라서 양자는 생부의 재산도 상속할 수 있다. }
요컨대, 바야흐로 근대 세계를 지배하고 있는 관념의 언저리에
가까이 다가가게 된 것이다.
하지만 이렇게 서로 멀리 떨어진 시기 사이에는 우리가 잘 모르는
중간 시기가 널리 펼쳐져 있으니,
보다 관용적인 모습을 띠어가면서도
로마의 \wi{가부장권}이
그렇게 오래도록 지속될 수 있었던
이유에 대해서는 추측만 할 수 있을 따름이다.
아들이 국가를 위해 부담하는 의무 중에서 가장 중요한 것들을
실제로 이행하는 것은 가부장권을 폐지까지는 아니지만 약화시키는 데
크게 기여했을 것이 틀림없다.
고위 관직을 수행하고 있는 나이 많은 사람에 대해
아버지의 전제 권력이 아무런 스캔들 없이 행사될 수 있으리라고는
생각하기 어렵다.
그러나
초기 역사에서
사실상의 \wi{부권면제}\hanjalatin{父權免除}{emancipation}의 이러한 유형은
로마 공화국이 지속적인 전쟁상태에 놓임으로써 나타난 유형에 비해
상대적으로 드물었을 것이다.
초기 전쟁에서 장교들과 사병들은 일년의 4분의 3을 전장에 나가 있었고,
후대에 전직\hanja{前職}집정관\latin{proconsul}은 속주를 책임지고 있었고
군단병들은 속주를 점령하고 있었으므로,
그들은 스스로를 전제적 가부장의 노예로 생각할 이유가 사실상 없었다.
가부장권을 벗어나기 위한 이러한 통로들은 계속해서 늘어갔다.
승리는 정복을 낳고, 정복은 점령을 낳거니와,
식민도시의 건설을 통해 점령하던 방식이
상비군을 속주에 주둔시켜 점령하는 방식으로
변화되어갔다.
진보의 각 단계마다,
더 많은 로마 시민들이 변방으로 이주해야 했고,
격 낮은 라틴인의 피를 새로 수혈받아야 했다.
생각건대,
제정\hanja{帝政}의 확립과 더불어 세계의 평화가 시작되면서
가부장권의 약화를 지지하는 강력한 감정이 정착된 것이 아닐까 한다.
이 고대 제도에 대한 첫 번째 타격은 초기 황제들에 의해 가해졌거니와,
트라야누스와 하드리아누스의 단발적인 개입들은,\footnote{트라야누스는
아들을 가혹하게 다룬 아버지에게 그 아들을 \wi{부권면제}시키라고 명했다.
하드리아누스는 계모와 간통했다는 이유로 아들을 사냥 중에 살해한 아버지를
국외로 추방했다.}
비록 그 연대는 정확히 알 수 없으나
한편으로는 가부장권을 제한하고
다른 한편으로는 그것의 자발적 포기 가능성을 확대했다고 알려진
일련의 명시적 입법을 낳는 단초를 마련했던 것이다.
덧붙이건대, 아들을 세 번 매도하여 \wi{가부장권}을 벗어나게 하는 옛 방식은
가부장권의 불필요한 연장\hanja{延長}에 대한 반감을 보여주는
초기의 증거라 하겠다.
아버지에 의해 세 번 매각된 아들은 자유롭게 된다는 이 법규칙은
원래는
초기 로마인들의 불완전한 도덕관으로도
용납하기 어려운 관행에 대해 처벌적 결과를 부과한다는 의미를 가졌을 것이다.
그러나 12표법이 공표되기 이전에도,
법학자들은 창의력을 발휘하여,\footnote{여기서 `법학자들'은
공화정 후기에 등장하는 진정한 의미의 법학자라기보다는
신관단(collegium pontificum)으로 이해해야 할 것이다.}
아버지가 원하는 경우라면 가부장권을 소멸시킬 수 있는 수단으로
이 규칙을 전용\hanja{轉用}했다.

물론,
자식의 신분에 대한 가부장의 권력을 약화시키는 데 기여한 많은 원인들은
역사의 전면에 드러나지 않는다.
법이 수여한 권력을 여론이 얼마나 무력화시켰는지,
자연적 애정이 얼마나 그것을 지속시켰는지,
우리는 알지 못한다.
그러나, 비록 \hemph{신분}\latin{person}에 대한 가부장의 권력은 나중에 가서
명목적인 데 그치게 되지만,
현존하는 로마법 전체의 양상에 비추어 볼 때
아들의 \hemph{재산}\latin{property}에 대한 가부장의 권리는 언제나
법이 인정하는 최대 범위까지 거리낌없이 행사되었다.
이 권리가 처음 등장했을 때 그것의 광범위함에는 놀라울 것이 전혀 없다.
로마 고법\hanja{古法}은 \wi{가부장권}에 복속해있는 자식이 가부장과 별도로 재산을
가질 수 없도록 금지했다. 아니, \paren{보다 정확히 말하자면}
자식들이 독립적 소유권을 가진다는 것을 상상조차 하지 못했다.
아버지는 자식이 취득하는 모든 것을 취득할 권리가 있었고,
자식이 맺는 계약으로부터 이익을 누릴 권리도 있으나
그에 상응하는 책임에는 얽혀들지 않았다.
초기 로마 사회의 구조를 감안할 때
이 정도는 우리가 예상할 수 있는 것들이다.
구성원들의 모든 수입이 공동 재산에 편입되지만
개인의 경솔한 계약이 공동 재산을 구속할 수는 없다는 것을
전제하지 않으면 우리는 원시 가족집단의 관념을 거의 이해할 수 없기 때문이다.
가부장권의 진정한 수수께기는 여기에 있는 것이 아니라,
가부장의 이러한 \wi{재산법}적 특권이 축소되는 데 그렇게나 오래 걸렸다는 것이며,
중대한 축소가 있기 전에 이미 문명 세계의 전부가 그 지배영역 안에
놓이게 되었다는 사실에 있는 것이다.
제정 시대 초창기 들어 비로소 어떤 혁신이 시도되기 시작했으니,
군인들이 복무 중 취득한 재산이 가부장권의 통제를 벗어나게 된 것이다.
물론 이것은 공화정을 무너뜨린 군대에게 보상으로 주어진 것이라는 데서
일말의 이유를 찾을 수 있을 것이다.
그로부터 3세기가 지나, 동일한 면책특권은 국가 공직의 수행 중에 취득한 재산에도
확대되었다.
이 두 가지 변화는 명백히 가부장권의 적용범위를 제한하는 것이었으나,
되도록이면 가부장권의 원리를 훼손시키지 않으려는 법기술적 외관을 띠고 있었다.
일찍부터 로마법에는 어떤 제한적이고 종속적인 소유권이 인정되고 있었거니와,
가부장권 아래 있는 노예나 아들의 수입이나 비용절감으로서
가족재산에 포함시키지 않아도 되는 것들이 있었다. 이렇게 허용된 재산을
`\wi{특유재산}'\hanjalatin{特有財産}{peculium}이라는 특별한 이름으로 부르고 있었으니,
이 이름을 \wi{가부장권}에서 새롭게 면제된 재산에도 준용하여,
군인의 경우에는 `군영특유재산'\hanjalatin{軍營特有財産}{castrense peculium}이라
불렀고, 공직자의 경우에는
`준\hanja{準}군영특유재산'\latin{quasi-castrense peculium}이라
불렀던 것이다.
뒤이어 취해진 가부장의 특권을 변경시키는 다른 조치들은
고대 원리에 대한 존중이라는 외관을 훨씬 덜 갖춘 것들이었다.
준군영특유재산의 도입 직후,
콘스탄티누스 대제는 자식이 어머니로부터 상속받은 재산에 대한
가부장의 절대적 지배권을 박탈하여
일종의 \wi{용익권}\hanjalatin{用益權}{usufruct},
즉 생애 동안만 가지는 사용수익권으로 축소시켰다.
그후 서로마제국에서는 몇몇 대수롭지 않은 변화만이 이어졌으나,
동로마제국에서는 \wi{유스티니아누스} 치세에 마지막 일격이 가해졌으니,
아버지로부터 유래한 재산이 아닌 한
자식의 재산에 대한 아버지의 권리는
아버지의 생애 동안 사용수익하는 권리에 불과한 것으로 축소되었던 것이다.%
\footnote{\latin{Inst.\,2.9.1.}}
로마 가부장권의 최종적 축소판이었던 이것조차도
현대 세계의 상응하는 제도에 비하면 훨씬 폭넓은 것이었고
훨씬 엄격한 것이었다.
근대 초기의 법학자들은 로마 제국을 정복한 민족 가운데
아주 사납고 미개한 민족만이, 특히 슬라브적 기원을 가진 민족만이,
학설휘찬\latin{Pandects}이나 칙법전\latin{Code}의 서술과 유사한
가부장권을 가지고 있었다고 말한다.
모든 게르만 이주민들은 가부장의 권력이라는 뜻의 `\wi{문트}'\latin{mund}라고 불린
가족 단위 집합체를 인정하고 있었던 것으로 보인다.
그러나 그의 권력은 쇠퇴한 로마 가부장권의 유물에 지나지 않았고,
로마 가부장이 누렸던 권력에는 한참 못 미치는 것이었음에 틀림없다.
특히 프랑크족은 저 로마적 제도를 갖지 않았던 것으로 알려져 있거니와,
따라서 옛 프랑스 법률가들은,
만족\hanja{蠻族}의 관습의 빈틈을 로마법 규칙으로 열심히 메우고 있던 시절에도,
가부장권의 침입으로부터 자신들을 보호하기 위해
``가부장권은 프랑스에는 존재하지 않는다''\latin{Puyssance de père
en France n'a lieu}라는 법언을 내세우지 않으면 안 되었던 것이다.
고대적 상황의 유물인 이 제도를 유지함에 있어 로마인들의 보여준 고집스러움은
그 자체로 독특하지만, 이보다 더 독특한 것은
일단 사라져버린 가부장권이 문명세계 전체로 다시 확산되어 나간 사실이다.
군영\wi{특유재산}이 아직 가부장의 재산권에 대한 유일한 예외였던 시절,
자식들의 신분에 대한 가부장의 권한이 여전히 포괄적이었던 시절에,
로마 시민권이 가부장권과 더불어 제국의 구석구석으로 퍼져나간 것이다.
아프리카인이건, 에스파니아인이건, 갈리아인이든, 브리타니아인이든, 유대인이든,
누구라도 증여나 매수나 상속 등을 통해 시민권을 취득한 이는
로마의 \wi{신분법}\latin{law of persons}을 받아들였다.
또한, 남아있는 전거에 따르면, 시민권 취득 이전에 출생한 자식들은
그들의 동의 없이는 가부장권에 복속하지 않았지만,
시민권 취득 후에 출생한 자식들과 그 후손들은 모두 로마의
`\wi{가부장권}에 복속하는 아들'\latin{filius familias}이라는 통상적인 지위에 놓였다.
후기 로마 사회의 메커니즘을 규명하는 것은 본 저서의 영역을 벗어나는 일이지만,
한 가지 확실히 해두고 싶은 것은
안토니누스 카라칼라 황제가 제국 내의 모든 신민들에게 로마 시민권을 부여한
칙법이 별 중요성을 갖지 않는다는 의견은 근거가 박약하다는 점이다.
어떻게 해석하든 간에
그것은 가부장권의 적용대상을 대폭 확장시켰으며,
생각건대 그것이 가져온 가족관계의 결속은
세상을 변혁시킨 도덕 혁명을 설명함에 있어
과거 우리가 생각했던 것보다 훨씬 더 중요한 인자\hanja{因子}였던 것이다.

이 주제에 관한 논의를 끝마치기 전에,
가부장은 자신의 권력 하에 있는 아들의 불법행위\latin{delict}에 대해
책임을 졌다는 점을 분명히 해둘 필요가 있다.
가부장은 노예의 불법행위에 대해서도 마찬가지로 책임을 졌다.
양자 모두의 경우에
원래
가부장은
불법행위를 범한 자를
손해배상을 대체하여
피해자에게 넘겨줄 수 있는 독특한 권리를 가지고 있었다.\footnote{%
  유스티니아누스 시대에 이르면
  자녀와 노예 중 자녀의 가해자 위부(加害者委付 noxae deditio) 제도는
  폐지된다. \latin{Inst.\,4.8.7.} }
이렇게 아들을 대신하여 지는 책임은,
아버지와 그 권력 하에 있는 자식 간에는
서로 소송을 제기할 수 없다는 점과 더불어,
이를 두고 몇몇 법학자들이
아버지와 그 권력 하에 있는 아들 간의
``인격의 통합''이라는 가설로써 잘 설명할 수 있다는
주장을 펼치도록 했을 것이다.
상속법에 관한 장에서 나는 이러한 ``통합''이 어떤 의미에서
그리고 어느 정도까지 현실로 받아들여졌는지 살펴볼 것이다.
지금 내가 말할 수 있는 것은
이러한 가부장의 책임이, 그리고 앞으로 다룰 다른 법현상들이,
원시 가부장의 \hemph{권리}에 상응하여 어떤 \hemph{의무} 또한
지시하고 있는 것으로 보인다는 점이다.
생각건대,
만약 가부장이 그의 가족들의 신분과 재산에 대한 절대권을 행사했다면,
가족을 대표하는 그의 소유권은 모든 구성원을 부양할 책임을 공동재산에서
부담하는 것과 궤를 같이했을 것이다.
문제는 우리의 습관적 관념에서 벗어나야만 가부장의 이러한 의무의 성격을
제대로 이해할 수 있다는 것이다.
그것은 법적 의무가 아니었다. 법은 가족이라는 성역\hanja{聖域}에 아직
들어갈 수 없었기 때문이다.
그것을 \hemph{도덕적}이라고 부르는 것은 정신발달의 나중 단계에 속하는 것을
당겨쓰는 것이라 주저된다.
하지만 우리의 목적에는 ``도덕적 의무''라는 용어가 적합할 것이니,
명확한 제재가 아니라 본능과 습관에 의해 반\hanja{半}의식적으로
준수되고 강제되는 의무로 그것을 이해해야 할 것이다.

\para{종족과 혈족}
통상적인 형태의 가부장권 제도는 대체로 오래 지속되지 않았으며,
내가 보기에도 오래 지속될 수 있는 제도가 아니었다.
따라서 초기 \wi{가부장권}의 보편성의 증거는 그것 자체만 고려해서는 불완전하다.
하지만 고대법의 다른 분야들을 살펴봄으로써 입증은 계속될 수 있거니와,
그것은 궁극적으로는 가부장권에 의존하는 분야들이지만,
그 연관성이 모든 부분에 있어서 모든 사람에서 뚜렷이 보이는 것은 아니다.
가령 친족법으로,
즉 고법\hanja{古法}에서 사람 간의 관계의 근친성이 계산되는 척도에 관한 분야로
눈을 돌려보자.
여기서도 편의상 로마법상의 \wi{종족}\hanjalatin{宗族}{agnatic}관계와
\wi{혈족}\hanjalatin{血族}{cognatic}관계라는 용어를 사용하기로 하자.
\hemph{혈족}관계는 근대적 관념에 가까운 친족 관념이다.
그것은 동일한 한 쌍의 부부로부터 이어지는 공통의 후손들을 지칭하는데,
남자로 이어지든 여자로 이어지든 상관하지 않는다.
\hemph{종족}관계는 이것과 사뭇 다르다.
그것은 오늘날 우리가 친족이라 간주하는 다수의 사람들을 제외하며,
오늘날 우리가 친족에 포함시키지 않는 많은 사람들을 포함한다.
실로 그것은 아주 초기 고대에 가족 구성원들 간에 존재하는 관계였다.
그 관계의 범위는 근대적 관계의 범위와 사뭇 달랐다.

\wi{혈족}은 단일한 남자조상과 여자조상으로 혈연을 거슬러올라갈 수 있는
모든 사람들을 뜻한다.
혹은 로마법의 엄격한 법기술적 의미를 가져다쓴다면,
합법적으로 \wi{혼인}한 공통의 한 쌍의 조상으로
혈연을 추적할 수 있는 모든 사람을 뜻한다.
그리하여 ``혈족''은 상대적인 개념이 되거니와,
그것이 지칭하는 혈연관계의 범위는 계산의 출발점으로 선택된
특정한 혼인관계에 의존한다.
아버지와 어머니의 혼인에서 출발한다면, 혈족은 오직
형제자매의 관계만 표현하게 될 것이다.
할머니와 할아버지의 혼인에서 출발한다면, 아저씨와 아주머니와 그들의 후손들도
혈족 개념에 포함될 것이다.
이와 같이 출발점을 계속 윗대로 올라가며 선택하면 더 많은 수의 혈족들이
계속해서 포함될 것이다.
이 모든 것을 현대인들은 쉽게 이해한다.
하지만 종족은 어떠한가?
한마디로, 오로지 남자로만 이어지는 혈족이 \wi{종족}이다.
물론 혈족의 가계도는 각 혈통의 조상을 차례로 모두 추적하고
그 후손들을 남녀 불문하고 모두 포함하여 그려지거니와,
이러한 가계도의 여러 가지\latin{branch}들을 추적함에 있어
여자의 이름을 만날 때마다 가지의 추적을 중단함으로써
그 여자의 모든 후손들을 제외시켜버리면,
남는 사람들이 바로 종족이며 그들 간의 연결이 바로 종족관계이다.
내가 혈족에서 종족을 분리하는 과정을 좀 자세히 말한 이유는 이것이
``여자는 가족의 종단\hanja{終端}이다''\latin{Mulier est finis familiae}라는
기억할 만한 법언을 잘 설명해주기 때문이다.
여자의 이름은 가계도의 가지를 마감한다.
여자의 후손은 누구도 원시적 가족관계의 개념에 포함되지 않는다.

우리가 고찰하는 고법\hanja{古法}이 입양을 인정하는 법체계라면,
가족 경계선의 이러한 인공적 확장에 의해 받아들여진 사람도,
남자든 여자든 상관없이, 모두 종족에 포함시켜야 한다.
하지만 그들의 후손은 우리가 좀 전에 기술한 조건을 충족하는 경우에만
종족이 될 수 있다.

\para{종족}
그렇다면 이러한 자의적인 포함과 배제의 이유는 무엇인가?
어째서 친족 관념은 \wi{입양}으로 가족에 받아들여진 이방인은 포함하면서
여자 구성원의 후손은 배제하는 탄력성을 보여주는가?
이 질문에 답하기 위해서는 가부장권 개념을 소환해야 한다.
\wi{종족}의 기초는 아버지와 어머니의 혼인관계가 아니라 아버지의 권력이다.
동일한 \wi{가부장권}에 복속하고 있는 사람들,
그 가부장권에 복속했었던 사람들, 또는
혈통상의 조상이 가부장권을 행사할 만큼 오래 살았다면 그 가부장권에
복속했을 사람들, 이들은 모두 종족인 것이다.
원시적 관념에 의하면 실로 친족관계는 바로 가부장권에 의해 정해진다.
가부장권이 시작하는 곳에서 친족관계가 시작한다.
따라서 입양에 의한 관계도 친족관계인 것이다.
가부장권이 끝나는 곳에서 친족관계도 끝난다.
따라서 아버지에 의해 \wi{부권면제}된 아들은 종족으로서의 권리를 상실한다.
왜 여자의 후손들은 원시적 친족의 경계선 바깥에 있었는지도
여기서 그 이유를 찾을 수 있다.
여자가 혼인하지 않고 죽으면, 그녀는 합법적 후손을 가질 수 없다.
그녀가 혼인하면, 그녀의 자식들은 그녀의 아버지가 아니라 남편의 가부장권에
복속하므로 그녀 자신의 가족에 속하지 않는 것이다.
만약 사람들이 어머니의 친족들까지 자신의 친족이라 불렀다면,
원시사회의 구조는 매우 혼란스러웠을 것임에 틀림없다.
이는 한 사람이 두 개의 가부장권에 복속하는 결과를 가져왔을 것이며,
두 개의 가부장권은 두 개의 재판권을 의미하거니와,
둘 모두에 순종하는 자는 동시에 두 개의 서로 다른 체제 하에서 살아가는 결과가
되기 때문이다.
통치권 안의 통치권\latin{imperium in imperio}이자
국가 안의 공동체인 가족이
가부장을 원천으로 하는 자신만의 제도로 통치되는 것이라면,
친족관계를 종족으로 국한하는 것은
가내법정\hanja{家內法廷}에서 법의 충돌\latin{conflict of laws}을 방지하는
필수적 안전장치였던 것이다.

진정한 의미의 가부장권은 가부장 지위의 죽음으로 소멸한다.
하지만 종족은 \wi{가부장권}이 사라진 후에도 그것의 각인을 담고있는
일종의 주형\hanja{鑄型} 같은 것이다.
따라서 법제사 연구자에게 종족은 자못 흥미를 불러일으키는 주제이다.
가부장권 자체는 비교적 소수의 기념비적인 고대법체계에서만 발견되지만,
과거에 가부장권이 존재했었음을 암시하는 종족관계는
거의 어디서나 발견할 수 있는 것이다.
인도^^b7유럽 계통의 공동체에 속하는 법체계로서
그들의 사회구조의 아주 고대적인 부분에서
종족이라고 부를 만한 특징을 보여주지 않는 법체계는 거의 없다.
가령
원시적인 가족적 위계관계 관념으로 가득차있는
힌두법의 친족관계는 전적으로 종족적인 것이며,
내가 알기로 인도인의 가계도에서 일반적으로 여자의 이름은 아예 제외된다.
로마 제국을 침략했던 민족들의 법에서도
친족관계에 관한 동일한 견해가 다수 발견되고 있어
실제로
그들의 원시 관행의 일부를 이루었을 것으로 짐작되거니와,
만약 후기 로마법이 근대법에 끼친 막대한 영향력이 없었더라면
그들의 법은 지금 우리가 보는 것보다 훨씬 더 많이 존속했을 것이다.
일찍이 \wi{법무관}들은 \wi{혈족}을 \hemph{자연법적} 친족관계로 파악하여
로마법체계를 옛 관념으로부터 정화시키는 데 수고를 아끼지 않았다.
그들의 관념이 오늘날 우리에게 전해지지만,
종족의 흔적도 다수의 근대 상속법에서 여전히 발견되고 있다.
흔히 프랑크족의 일파인 살리족\latin{Salian Franks}의 관행에서
기인한다고 여겨지는,
여자와 그 후손들을 통치기능에서 제외하는 상속법은
확실히 \wi{종족}적 기원을 가진 것으로,
\wi{자유소유지}\latin{allodial property}의 상속에 관한
고대 게르만법에서 유래한 것이다.\footnote{%
  `자유소유지'란 봉건적 부담을 지지 않는 부동산을 말한다.
  그런데 본문의 설명은 부정확하다.
  렉스 살리카(Lex Salica) 제59장에 의하면
  가족재산으로서 자유소유지는 여자도 상속할 수 있었다.
  여성의 상속이 배제되는 것은 자유소유지가 아니라
  `테라 살리카'(terra Salica)라 불린 부족재산으로
  은대지(恩貸地 benefice)의 대상이 되는 토지였다.
  물론 메로빙거 왕위의 상속은 테라 살리카의 규칙에 따랐다.
  %\latin{Katherine Fischer Drew,
  %\textit{The Laws of the Salian Franks},
  %Phliladelphia: University of Pennsylvania Press, 1991, pp.\,43f.}
  메인은 여기서의 `살리카'라는 단어를 무의미하게 삽입된 것으로 이해한 듯하다.
  \latin{Main, \textit{Early Law and Custom}, p.\,169.} }
최근에야 폐지된,\footnote{%
  1833년의 물적재산상속법(Inheritance Act). }
한쪽 부모만 같은\latin{half-blood} 형제 간의 토지 상속을 금지한
특수한 영국법 규칙도 \wi{종족}에 기초하여 설명할 수 있을 것이다.
노르만의 관습에 의하면
이 규칙은 \hemph{어머니만 같은}\latin{uterine} 형제,
즉 아버지가 다른 경우에만 적용되고
아버지가 같은 형제 간에는 적용되지 않는다.
이렇게 본다면 이 규칙은 종족 개념에서 연역된 것이 틀림없으니,
어머니만 같은 형제는 서로 간에 종족이 아니기 때문이다.
이 규칙이 영국에 이식되었을 때,
그 배경 원리를 이해하지 못한 영국의 판사들이
한쪽 부모만 같은 형제 간의 상속이 일체 금지된다고 해석함으로써
\hemph{아버지만 같은}\latin{consanguineous} 형제,
즉 아버지는 같지만 어머니는 다른 아들들에까지 확대적용했던 것이다.
자칭 법철학이란 것을 담고 있는 문헌들 가운데,
\wi{블랙스톤}의 저서 중
한쪽 부모만 같은 형제 간의 상속 금지를 설명하고 정당화하려 한,
정교한 궤변들로 가득찬 페이지들보다 더 이상한 것은 없을 것이다.

\para{여성 후견}
가부장에 의해 통합되어 있는 가족이야말로
신분법 전체가 발달되어 나온 모태였다고 생각한다.
\wi{신분법}의 여러 장\hanja{章}들 중에
가장 중요한 것은 여성의 지위에 관한 것이다.
좀 전에 말한 것처럼,
원시법에 따르면,
여자는 자신의 후손에게 \wi{종족}의 권리를 전해줄 수는 없지만,
그래도 그녀 자신은 종족관계에 포함된다.
사실, 어떤 여성이 자신의 태어난 가족과 맺는 관계는
그녀의 남자친족들 간의 결합 관계보다
훨씬 더 엄격하고 친밀하고 또 지속적이다.
누차 말했듯이,
초기법은 가족 외에는 알지 못한다.
이는 곧 초기법은 \wi{가부장권}을 행사하는 사람 외에는 알지 못한다는 말과 같다.
따라서 가부장의 사망으로 아들이나 손자가 해방되는 유일한 이유는
그 아들이나 손자에게 내재해있는,
스스로 새로운 가족의 수장이 되고
장차 새로운 가부장들의 뿌리가 될 수 있는
능력을 고려해서인 것이다.
그러나 여자는 이런 종류의 능력을 갖지 못함은 물론이요,
그런 능력에 수반되는 해방될 수 있는 자격도 갖지 못한다.
그리하여 고법\hanja{古法}에서는
그녀를 평생동안 가족의 속박 아래 두기 위한 특유한 장치가 있었으니,
초창기 로마법에서
`여자의 영구적 \wi{후견}'으로 불리던 제도가 그것이다.
이에 따르면 여성은,
비록 아버지의 사망으로 그의 가부장권에서는 벗어나지만,
최근친 남자 친족 또는 아버지가 지명한 자의 후견에 의해
평생동안
복속이
계속된다.
영구적 후견 제도는
다른 경우라면 해소되었을
가부장권의 인위적 연장, 그 이상도 이하도 아니다.
인도에서는 이 제도가 완벽하게 살아남아
아주 엄격하게 작동하는지라, 인도의 어머니는
자신의 아들의 후견을 받는 경우가 흔하다.
유럽에서도,
스칸디나비아 민족들의
여성에 관한
법은 최근까지도 이 제도를 유지해왔다.
서로마제국을 침공했던 민족들의 토착 관행에서도 이것은 보편적으로
발견되거니와, 실로 후견에 관한 그들의 관념은, 그 모든 형태에도 불구하고,
서구 세계에 도입된 모든 관념 중에서 가장 퇴행적인 것이었다.
그러나 성숙한 로마법에서는 이 제도가 완전히 사라졌다.
만약 \wi{유스티니아누스} 법전만을 참조한다면
우리는 그것을 전혀 알아채지 못할 것이다.
그러나 \wi{가이우스}의 <<\wi{법학제요}>> 필사본이 발견됨으로써
이 제도의 가장 흥미로운 시기,
즉 완전히 불신되어 사라지기 일보직전 시기의
모습이 우리 앞에 나타났다.
저 위대한 법학자는,
이 제도를 옹호하는 통속적인 변명인
여성의 지려박약\hanja{智慮薄弱}을 말하고 있기는 하지만,\footnote{%
  \latin{Gai.\,1.144.}}
여자들이
고대 규칙을 깨뜨릴 수 있도록
로마 법률가들이 고안해낸, 때로는 비상한 독창성을 보여주는,
수많은 장치들을
저서의 상당 부분을 할애하여
서술하고 있다.
당시의 법학자들은,
자연법 이론에 기초하여,
그들의 형평법 법전의 원리의 하나로
양성평등을
받아들인 것이 확실하다.
그들이 공격한 지점이 재산 처분에 대한 제한,
즉 여전히 형식적으로는 여자의 \wi{후견}인의 동의가 요구되던
제한이었다는 점은 주목할 만하다.
여성의 신분\latin{person}에 대한 통제는 이미 옛날 일이었던 것이다.

\para{고대 로마의 혼인, 여성의 지위}
고대법은 여자를 그녀의 혈연 친족에게 종속시키지만,
근대법의 주요 특징의 하나는 그녀를 남편에게 종속시키는 것이다.
이 변화의 역사는 주목할 가치가 있다.
그것은 로마 연대기의 저 멀리까지 거슬러올라가 출발한다.
아주 옛날의 로마 관행에 따르면
\wi{혼인}이 체결되는
방식에는
세 가지가
있었다.
하나는 종교적 엄숙함에 기초한 것이고,
나머지 둘은 어떤 세속적 방식을 준수하는 데 기초했다.
종교적 혼인인 \wi{콘파레아티오}\latin{confarreation},\footnote{%
  유피테르 신에게 스펠트(spelt)밀로 만든 떡(farreum)을 바치는 종교의례를 통해
  성립하는 혼인. }
세속적 혼인의 고상한 형태인 \wi{코엠프티오}\latin{coemption},\footnote{%
  악취행위(mancipatio)를 통해, 즉 매매의 형식을 빌어 성립하는 혼인. }
세속적 혼인의 통속적 형태인 \wi{우수스}\latin{usus},\footnote{%
  1년 동안 계속해서 남편과 같이 살면 혼인관계가 성립했다.
  일종의 사용취득(usucapio)에 해당한다.
  12표법(6.5)은 매년 3일을 남편에게서 떠나있으면 시효가 중단된다고 규정했다. }
이것들에 의해 남편은 아내의 신분과 재산에 대한 다수의 권리를 취득했거니와,
이는 대체로 그 어떤 근대법에 의해 남편에게 주어지는 것보다도 훨씬 큰 것이었다.
그런데 남편은 어떤 자격에서 이런 권리를 취득하는 것이었을까?
그것은 \hemph{남편}으로서가 아니라 \hemph{아버지}로서였다.
콘파레아티오, 코엠프티오, 그리고 우수스에 의해
여자는 남편의 수권\hanja{手權}에\latin{in manum viri} 넘겨졌으니,
법적으로는 남편의 \hemph{딸}이 되는 것이다.
그녀는 남편의 \wi{가부장권}에 복속했다.
남편의 가부장권이 존속하는 동안은 그로부터 생겨나는 모든 책임을 부담했고,
그것이 종료된 후에는 그것이 남겨놓은 모든 책임을 부담했다.
그녀의 모든 재산은 전적으로 남편의 것이 되었으며,
남편이 죽고 나서도 그가 유언으로 지명한 후견인의 \wi{후견}에 놓였다.
하지만
저 세 가지 고대적 혼인 형태는 점차 안 쓰이게 되었거니와,
그리하여 로마가 가장 융성했던 시절에는
세속적 혼인의 통속적 형태가 변형되어 생겨난 어떤 혼인 방식---분명
오래된 것이나, 그때까지는 그다지 존중되고 있지는 않던 것---에 의해
거의 완전히 대체되었다.
새로이 널리 대중화된 제도의 법기술적 메커니즘에 대해서는 자세히 다루지
않겠으나, 다만 그것이 법적으로는
여자의 가족이 여자를 일시적으로 맡기는 것 정도에
지나지 않았음을 언급해두고 싶다.
여자 가족의 권리는 그대로 유지되었고,
부인은 그녀의 부모가 지명한 후견인의 후견에 계속 놓여있었으며,
이 후견인의 통제권은
여러 주요 측면에 있어
남편의 권력보다 훨씬 큰 것이었다.
결과적으로 로마 여성의 지위는
혼인 여부를 불문하고
\wi{신분법}적으로나 \wi{재산법}적으로나 사뭇 독립적인 것이 되었다.
앞서 언질을 준 것처럼, 이후의 법발달에 의해
후견권은 거의 없는 것이나 진배없는 수준으로 축소되었고,
그에 비해 저 대중적 혼인 방식은 남편에게 그에 상응하여
권력을 더 부여하지 않았기 때문이다.
그런데 기독교는 거의 처음부터 이러한 비범한 자유를 축소시키는 경향이 있었다.
처음에는 쇠퇴해가는 이교도 세상의 방탕한 풍속에 대한 그럴 만한 혐오에서,
나중에는 금욕주의의 열광에서 나온 조급함에서,
새로운 신앙의 추종자들은 서구 세계가 보여준 것들 중
사실상 가장 느슨한 \wi{혼인}관계를 자못 불쾌한 눈으로 바라보았다.
기독교도 황제들의 칙법에 의해 수정된 최후의 로마법은
안토니누스 황조 시대의 위대한 법학자들의 자유주의적 법리에 대한
반동을 다분히 담고 있었다.
또한 종교적 감정의 고양된 상태는
만족\hanja{蠻族}들의 정복의 용광로에서 벼려지고
가부장제 관행의 로마법이 융합되어 형성된
근대법이
어째서
저 원초적 규칙들 중에서도
여성의 지위에 관해서는
덜 발달된 문명의 특징에 속하는 규칙들을 기대 이상으로 많이 흡수했는지를
설명해줄 수 있을 것이다.
근대 역사가 시작되던 혼란기에,
게르만족과 슬라브족 이주민들의 법이
그들의 지배를 받는 로마인들의 법 위에 또 하나의 층으로 따로 존재하던 시기에,
지배층 민족의 여자들은 어디서나
다양한 형태의 원시적 \wi{후견} 아래 놓여있었고,
다른 가족으로부터 아내를 취하는 남편은
후견권을 사오는 대가로
그녀의 친족들에게
신붓값을 지불했다.
시간이 흘러 중세의 법전에서 두 법체계의 융화가 이루어졌을 때,
여자에 관한 법은 저 두 가지 기원의 각인\hanja{刻印}을 모두 담고 있었다.
혼인하지 않은 여성에 관해서는
로마법의 원리가
지배적이어서,
일반적으로 \paren{지역에 따라 예외는 있으나}
그들은 가족의 속박으로부터 벗어나 있었다.
그러나
혼인한 여성에 관해서는
만족\hanja{蠻族}들의 옛 원리가 지배했고,
종래 아내의 남자 친족들에게 속했던 권력을 남편이 차지하게 되었거니와,
다만 이제는 대가를 지불하고 권리를 사오는 일이 없어진 점이 다를 뿐이다.
그리하여 이 시점의 남유럽과 서유럽의 근대법은 그 주요 특징 중 하나를
뚜렷이 나타내기 시작하니,
미혼의 여자나 과부에게는 상대적으로 자유가 허용된 반면,
아내들에게는 무능력이 무겁게 부과되었던 것이다.
혼인한 여성에게 부과된 종속이 눈에 띄게 감소하는 것은 오랜 시간이 흐른 뒤이다.
유럽에서 부활한 미개함을 부드럽게 만든 강력한 주된 용매는
언제나 \wi{유스티니아누스} 법전이었으니,
이것이 불러일으킨 열성적 연구가 행해진 곳에서는 어디서나 그러했다.
그것은
단지 옛 관습을 해석할 뿐이라고 내걸었지만 실제로는
은밀하게 그러나 효과적으로 저 관습을 약화시켰다.
하지만 혼인한 여성에 관한 법은 대체로 로마법보다는 \wi{교회법}의 조명을 받았다.
\wi{혼인}으로 형성된 관계에 대한 견해만큼
교회법이
세속법의 정신에서
멀리 떨어진 것은 없었다.
이것은 어느 정도 불가피한 일이었으니,
기독교 제도의 색채를 보유한 사회 중에
혼인한 여자에게 중기 로마법이 부여했던 신분상의 자유를 회복시켜 줄 사회는
없었던 것이다.
그러나 혼인한 여성의 \wi{재산법}상 무능력은 \wi{신분법}상 무능력과는 전혀 다른
기초 위에 서 있었는데,
전자를 계속 유지하고 강화하는 법리를 통해
교회법 해설자들은 문명 발달에 깊은 상처를 주었다.
세속법과 교회법 간의 갈등의 흔적은 여러 곳에서 발견되지만,
거의 어디서나 교회법이 승리를 거두었다.
프랑스의 일부 지방에서
귀족 아래 계급의 혼인한 여자들은
로마법이 인정했던 재산에 관한 모든 권한을 가지게 되었으니,
이 지방법은 \wi{나폴레옹 법전}에 대체로 수용되었다.
하지만 스코틀랜드법의 상태는
로마 법학자들의 법리에 대한 우직한 존경이
반드시 아내들의 무능력을 완화하는 데 기여하는 것은 아님을 잘
보여준다.
\wi{혼인}한 여성에게 가장 덜 관대한 법체계는 모두
교회법을 배타적으로 추종했던 법체계들이거나,
아니면
유럽 문명과의 접촉이 늦어 자신들의 고대적 유물을 솎아낼 기회가 없었던
법체계들이었다.
수 세기 동안 모든 여성들을 가혹하게 대해왔던 덴마크법과 스웨덴법은
여전히 아내들에게는
일반적인 유럽 대륙의 법전들보다
훨씬 덜 우호적이다.
그러나 \wi{재산법}적 무능력에 있어 대륙법보다 더 엄격한 것이
영국 보통법이거니와,
그것은 \wi{교회법}으로부터 근본원리를 대량으로 빌려왔던 것이다.
실로 혼인한 여성의 법적 지위에 관한 보통법은
본 장의 주제인 저 중차대한 제도에 대한 명료한 관념을
영국인들에게 심어줄 수 있을 법하다.
일체의 권리와 의무와 구제수단을 통하여,
형평법이나 제정법의 손을 타지 않은
순수한 영국 보통법이
남편에게 수여한 대권\hanja{大權}을 돌아보는 것만큼,
아내의 완전한 법적 종속을 관철시킨 엄격한 일관성을 회상하는 것만큼,
고대 가부장권의 성질과 작동을 생생하게 보여주는 것이 또 있을지 의문이다.
보통법법원과 형평법법원이 아내들에게 각각 적용하는 규칙들 간의 차이는
가부장권에 복속하는 자식에 관한 가장 이른 로마법과 가장 늦은 로마법 간의
거리만큼이나 멀다고 해도 좋을 것이다.

\para{후견제도}
두 가지 형태의 후견제도의 진정한 기원을 무시한다면,
그리고 두 가지 주제에 대해 동일한 언어를 사용한다면,
고법\hanja{古法}체계들에서
여자의 후견\latin{tutelage}은
권리의 정지라는 의제\hanja{擬制}를 지나치게 길게 밀고 나간 사례인 반면,
`아버지 없는\latin{orphan} 남성의 \wi{후견}'이라 불리는 것은
정확히 그 반대방향의 과오를 보여주는 사례라고 말할 수 있을지도 모른다.
이들 법체계에서 남성 후견은 상당히 이른 시기에 종료된다.
그것의 전형이라 할 수 있는 고대 로마법에서는
아버지나 할아버지의 사망으로 \wi{가부장권}에서 벗어난 아들은
일반적인 경우 15세에 도달할 때까지 후견에 놓이게 되고, 일단
그 연령에 도달하면 \wi{신분법}적으로나 \wi{재산법}적으로나 완전히 독립한다.
따라서 미성년의 기간은 지나치게 짧고
여성의 무능력 기간은 지나치게 긴 것으로 보인다.
하지만, 실은,
두 종류의 후견에 최초의 형태를 부여한 상황을 감안하면
지나치게 길거나 지나치게 짧은 요소는 전혀 없다.
어느 쪽에든 공적 또는 사적 편의성의 고려는 조금도 들어있지 않다.
여자의 후견이 여성의 취약함을 보호하기 위해 도입된 것처럼,
본디 아버지 없는 남성의 후견도
결정권을 행사하는 시기에 도달할 때까지 그들을 보호하기 위해 고안된 것이다.
아버지의 사망으로 아들이 가족의 속박에서 벗어나는 이유는
새로운 가족의 수장이 되고 또 다른 새로운 가부장권의 기초자가 될 수 있는
아들의 능력에 있는 것이다.
이에 비해 여자는 이러한 능력을 가지지 않으며, 따라서
\hemph{결코} 해방되지 않는다.
그리하여 아버지 없는 남성의 후견은
자식이 스스로 아버지의 능력을 행사할 수 있을 때까지
아버지의 가\hanja{家}에 복속되는 모양새를 유지하는
장치였다.
그것은 신체적 남성성이 발현되기까지의 일종의 가부장권의 연장이었다.
성숙기\latin{puberty}가 도래하면 \wi{후견}은 끝나거니와,
이러한 이론이 그것을 엄중하게 요청하기 때문이다.
하지만 이는 아버지 없는 남성 피후견인을
지적으로 성숙하거나 거래에 적합한 나이에
도달하기까지 보호한다는 말은 아니기 때문에
일반적 편의성의 목적에는 전혀 미치지 못한다.
로마인들은 사회진보의 무척 이른 시기에 이 문제를 인지했던 것 같다.
아주 오래된\footnote{기원전 3세기 말 또는 2세기 초.}
기념비적인 로마 입법으로 라이토리우스 법\latin{Lex Laetoria}
혹은 플라이토리우스 법\latin{Lex Plaetoria}이라 불리는 것이 있다.
이로써 나이가 차고 완전한 권리를 가진 모든 자유인 남성을
\wi{보좌}인\latin{curator}이라 불리는
새로운 종류의 \wi{후견}에 의한
일시적 통제 하에 두게 되었으니,
그들의 단독행위와 계약이 유효하기 위해서는 보좌인의 승인이 요구되었다.
젊은이의 나이가 26세에 이르면 이 제정법상의 감독권은 종료되었다.
그리하여 로마법에서는 ``성년''과 ``미성년''을 가르는 기준으로
25세가 전적으로 사용되었던 것이다.
미성숙\latin{pupilage} 또는 피후견\latin{wardship}의 신분은
근대법에서는
법적안정성을 적절히 고려하면서
미성년자의 신체적 미성숙과 정신적 미성숙을 모두 보호한다는
단순한 원리로 변용되었다.
결정권을 행사할 수 있는 연령이 그것의 자연적 종료시기이다.
그러나 신체적 취약함의 보호와 지적 무능력의 보호를
로마인들은 두 가지 서로 다른 제도에 맡겼으니, 그 둘은
이론에서도 의도에서도 서로 구분되는 것이었다.
양 제도에 따라붙는 관념들이 현대의 후견 관념에서는 통합되어 있다.

\para{노예제도}
우리의 목적을 위해서 원용할 만한 또 하나의 \wi{신분법} 제도가 남아있다.
\hemph{주인}과 \hemph{노예}의 관계에 대해 성숙한 법체계가 가진 법규칙들은
고대 사회에 공통된 원초적 상황의 흔적을 그다지 뚜렷이 보여주지 못한다.
그러나 여기에는 그럴 만한 이유가 있다.
아무리 성찰의 습관이 형성되지 못했다 할지라도,
아무리 도덕적 본능의 함양이 낮은 수준에 머물러있다 할지라도,
시대를 막론하고 인류에게 충격을 주고 인류를 당혹케하는 어떤 것이
\wi{노예제}도에는 들어있는 듯하다.
고대 공동체들이 거의 무의식적으로 겪었던 양심의 가책이
노예제도를 그럴 듯하게 방어하는, 적어도 정당화하려는,
어떤 가상의 원리를
반드시
채택하도록 만들었던 것 같다.
그들의 역사 초기에 그리스인들은
이 제도를
특정 민족의 지적 열등함에 기초하는 것으로,
따라서 노예상태에의 자연적 적합성에 기초하는 것으로
설명했다.
마찬가지로 독특한 정신의 소유자인 로마인들은
승자와 패자 간의 가상의 합의에서 그것을 이끌어냈으니,
전자는 그의 적의 영구적 서비스를 청약하고
후자는 승낙의 대가로 합법적으로 몰수당한 목숨을 구한다는 것이다.
이들 이론은 건강하지 못한 이론일 뿐만 아니라
그들이 설명하고자 하는 사태에도 전혀 부합하지 않는 것이었다.
그럼에도 불구하고 이들 이론은 여러 모로 강력한 영향력을 행사했다.
그것들은 주인의 양심을 만족시켰다.
그것들은 노예의 낮은 지위를 영속시켰고 어쩌면 더 가중시켰다.
그것들은 노예가 가\hanja{家}의 나머지 부분과 원래 어떤 관계를 가졌는지를
무시하게 만들었다.
이 관계는, 비록 명확히 드러나는 것은 아니나,
고대법의 여러 부분에서 산발적으로 징후를 나타내고 있거니와,
특히 그 전형적인 법체계---고대 로마법---에서 그러하다.

미국에서는
초기 사회에 노예가 가족의 구성원으로 인정되었는가에 대해
많은 연구와 학습이
수행되었다.
이에 대해 반드시 긍정적 답이 주어져야 한다는 생각이 존재한다.
고대법과 초기 역사의 증거에 의하면
특정 상황에서는 노예가 주인의 상속인, 즉 포괄승계인으로 지정될 수 있었음이
분명하다.
이러한 의미심장한 능력이 뜻하는 바는,
상속법에 관한 장에서 설명하겠지만,
가\hanja{家}의 통치와 대표를 특정 상황에서는 노예에게 맡길 수 있다는 것이다.
하지만 이 주제에 관한 미국의 논의에서 상정되고 있는 것은,
만약 \wi{노예제}도가 원시적 가족제도의 일부였다면,
오늘날의 흑인노예제 역시 도덕적으로 정당화될 수 있다고 인정해야
한다는 것으로 보인다.
그렇다면 노예가 원래 가족의 일원이었다는 것은 무엇을 의미하는가?
인간을 부추기는 저속한 동기 때문에 노예제가 생겼다는 뜻이 아니다.
나의 안락과 쾌락을 도모하는 수단으로 타인의 육체적 힘을 사용하고자 하는
단순한 희망은 물론 노예제의 토대이며,
인간의 본성만큼이나 오래된 것이다.
그러나 노예가 가족의 일원이었다고 말할 때 우리가 뜻하는 바는
그를 데려와서 노예로 삼은 자들의 동기에 관한 것이 아니다.
단지,
노예를 주인에게 묶어주는 관계가
다른 모든 가족구성원을 그 수장에게 결합시키는
관계와 동일한 일반적 성질을 가지는 것으로 간주되었음을 의미할 뿐이다.
사실,
가족 관계를 떠난, 개인들 사이의\latin{inter se} 관계를
관계의 기초로 이해하는 것은
인류의 원시적 관념과
전혀 부합하지 않는다는
전술한 일반적 주장에서 이러한 결론이 나오는 것이다.
가족은 우선 혈연에 의해 가족에 속하게 된 자들과,
다음으로 \wi{입양}에 의해 가족에 편입된 자들로 구성된다.
그런데
단지 수장에게 공동으로 복종한다는 것에 의해 가족에 속하게 되는
세 번째 카테고리의 사람들이 있으니,
이들이 바로 노예인 것이다.
수장에게 출생에 의해 복속하는 자와 입양에 의해 복속하는 자가
노예보다 더 우대받는 것은 통상적으로 사태가 진행된다면
언젠가 그들은 속박에서 벗어나 스스로 \wi{가부장권}을 행사하게 될 것이라는
확실성에 기인한다.
그러나 노예의 지위가 낮다는 것이
그를 가족의 울타리 바깥에 둔다든가
영혼없는 물건의 지위로 격하시킨다는 따위가 아니라는 것은,
생각건대,
최후 수단으로 상속인이 될 수 있는 능력에 관한
고대의 많은 흔적들이 남아있어
명백히 증명되고 있다.
물론,
가부장의 제국에서 확실하게 자기 자리를 차지하고 있다는 것으로써
사회의 초창기에 노예의 운명이 얼마나 개선되었을까 무모하게 추측하는 것은
대단히 안전하지 않은 추측일 것이다.
아마도,
후대에 아들에게 주어질 부드러운 취급을 노예도 받았을 것이라기보다는
오히려 아들이 사실상 노예와 비슷하게 취급받았을 것이라고 보는 쪽이
보다 그럴 법하다.
그러나
다소 자신있게 말할 수 있는 것은,
발달된 성숙한 법체계 중에서
\wi{노예제}도가 인정되는 곳이라면 어디서나,
노예의 열등한 지위에 관한 어떤 다른 이론을 채택한 법체계보다
노예의 초기 상황에 대한 기억을 보존하고 있는 법체계에서
항상 노예가 더 나은 취급을 받는다는 것이다.
법이 노예를 바라보는 관점이 그에게는 무엇보다 중요하다.
로마법에서는
노예를 점점 더 물건의 일종으로 취급하던 경향이
자연법 이론 덕분에
더 이상 확대되지 않고 정지되었다.
그리하여,
로마법에 의해 깊이 영향받은 제도에 기초하여 노예제를 인정하는 곳에서는
노예의 지위가 결코 참을 수 없을 만큼 열악한 것이 아니다.
공포스런 내전의 영향 하에 통과된 최근의 입법으로
헌법에 수정조항들이 추가되기 전까지,\footnote{%
  미국 연방헌법 수정조항 제13조(노예제 폐지), 제14조(평등권 조항),
  제15조(투표권 확대)를 말하고 있다.
  물론 이 구절은 <<고대법>> 초판에는 없었다가 후에 추가된 것이다.}
미국의 주들 중에
영국 보통법에 토대를 둔 제도를 채택한 주들보다는
대단히 로마적인 루이지아나법을 토대로 삼은 주들에서
흑인들의 운명과 전망이 여러 주요 측면에서 더 나았다는
많은 증거가 있다.
영국 보통법은, 최근에 해석된 것처럼,
노예를 위한 자리를 전혀 인정하지 않으며
따라서 노예는 그저 일종의 동산으로 취급될 수 있을 뿐이다.

\para{가족의 해체}
지금까지 본 저서에서 다루어야 할 고대 신분법의 모든 부분들을 살펴보았다.
이러한 탐구의 결과로
법의 유년기에 관한 우리의 관점이 보다 분명하고 정확하게 되었다고
나는 믿는다.
국가법은
가부장적 주권자의 \wi{테미스테스}로
처음 등장했거니와,
이제 우리는
저 테미스테스가
인류의 훨씬 더 초기 상태에
각각 독립된 가\hanja{家}의 수장들이 그의 아내와 자식들과 노예들에게 내리던
책임지지 않는 명령의
발달된 형태에 불과하다는 것을
알 수 있을 것이다.
하지만, 국가가 형성된 뒤에도,
법은 여전히 무척 제한된 적용범위를 가질 따름이었다.
법이 테미스테스라는 원시적 형태를 취하든,
아니면 관습법이나 법전이라는 보다 진보된 형태를 취하든,
그것은 개인이 아니라 가\hanja{家}를 구속하는 것이었다.
고대법은,
다소 무리한 비유인지 모르겠으나, \wi{국제법}과 유사하다고 할 수 있다.
말하자면, 사회의 원자들인 저 중차대한 집단들 사이의 틈새만을 메울 뿐인 것이다.
이런 상태의 공동체에서는
입법기관의 입법과 법원의 판결은
가족의 수장들에게만 미칠 뿐,
다른 모든 개인에게는
그의 가부장을 입법자로 하는
가\hanja{家}의 법이 곧 행위규칙인 것이다.
그러나 처음에는 좁았던 국가법의 영역이 꾸준히 그 범위를 넓혀간다.
법 변동의 인자\hanja{因子}들, 즉
법적의제, 형평법, 입법이 차례로 원초적 제도에 영향을 미치고,
진보의 각 단계마다
다수의 신분적 권리와 더 많은 수의 재산적 권리가
가내법정을 떠나 국가법정의 관할로 넘어가게 된다.
정부의 명령은 공적 문제뿐 아니라 사적 문제에도 차츰 간여하고,
더 이상 각 가정의 전제군주의 명령에 의해 무력화되지 않게 된다.
로마법의 연대기에는
고법\hanja{古法}체제가 무너져내린 거의 완전한 역사,
재조합한 재료들로 새로운 제도가 형성되어간
거의 완전한 역사가 담겨있거니와,
그 제도들의 일부는 근대 세계로 고스란히 전해졌으나,
다른 것들은 암흑시대에 만족\hanja{蠻族}들과의 접촉에서 파괴되거나 타락하여
인류에 의해 다시 회복되어야 했다.
\wi{유스티니아누스}에 의해 마지막으로 재구성됨으로써
로마법이 그 역사를 마감할 때,
살아있는 가부장에게 여전히 남겨졌던 폭넓은 권한에 관한 항목 하나를 제외하면
그 어떤 부분에서도 우리는 더 이상 고대법의 흔적을 발견할 수가 없다.
이 예외를 제외한 모든 곳에서는
편의성과 조화성과 단순성의 원리---여하튼 새로운 원리들---가
고대의 양심을 만족시켰던 조잡한 고려들의 권위를 전복시켰다.
모든 영역에서
새로운 도덕성이 고대적 관행에 부합했던 행위 기준과 묵인 근거를
내쫓았다. 사실 이들은 고대적 관행의 산물이었기 때문이다.

\wi{진보하는 사회}들의 운동은 한 가지 면에서는 일치했다.
모든 과정을 통털어, 가족적 위계관계가 해체되고 그 대신 개인적 의무가
성장한 점이 뚜렷하게 나타났다.
국가법의 고려 단위로서
개인은 꾸준히 가족을 대체해갔다.
진보의 속도는 서로 달랐다.
현상의 면밀한 연구를 통해서만
고대 조직의 붕괴를 인지할 수 있는,
정체된 사회에 가까운 사회도 있었다.
하지만,
속도의 차이에도 불구하고,
변화는 역행 없이 계속되었다.
멈칫거리는 듯 보여도 그것은 외부에서 유입된 원시적 관념과 관습 때문인 것으로
판명될 것이다.
가족에서 기원한 권리와 의무의 호혜성\hanja{互惠性} 형식을 점차 대체해간
사람들 간의 관계가 무엇인지 이해하는 것도 어렵지 않다.
계약이 바로 그것이다.
역사의 한쪽 끝에서,
사람들 간의 모든 관계가 가족 관계로 귀결되던 사회 상태에서 출발하여,
사람들 간의 모든 관계가 개인들의 자유로운 합의에서 생겨나는
사회질서의 국면으로
지속적으로 변화해온 것으로 보인다.
이런 방향으로
서유럽에서
이루어진 진보는 엄청난 것이었다.
그리하여 노예의 신분은 사라졌다.
그것은 주인에 대한 하인의 계약관계로 대체되었다.
피후견 여성의 신분도,
후견을 남편 아닌 다른 사람의 \wi{후견}으로 이해한다면,
역시 사라졌다.
성년에 이른 후 혼인할 때까지 그녀가 맺는 모든 관계는 계약관계이다.
가부장권에 복속하는 아들의 신분도 근대 유럽 사회들의 법에서는
더 이상 존재하지 않는다.
아버지와 성년의 자식을 묶어주는 민사법적 의무가 있다면,
그것은 오직 계약에 의해 법적 효력이 주어지는 것일 뿐이다.
예외처럼 보이는 것도 원칙을 보여주기 위한
인영\hanja{印影}으로서의 예외일 뿐이다.
결정권을 행사할 연령에 이르지 못한 자식,
피후견 고아, 심신상실의 선고를 받은 자,
이들은 모두 \wi{신분법}\latin{law of persons}에 의해
그 능력과 무능력이 규율된다.
왜 그런가?
그 이유는 법체계마다 언어 관용\hanja{慣用}이 달라 서로 달리 표현되고 있으나,
본질에 있어서는 모두 동일한 효과를 말하고 있다.
법학자들 대다수는
방금 언급한 부류의 사람들이
타인의 통제를 받는 근거가
오직
스스로의 이익을 위한 판단능력을 갖추지 못했기 때문이라는 원리에
일치하고 있다.
다시 말해, 계약을 체결하는 데 불가결한
첫 번째 요소를 결여하고 있기 때문이라는 것이다.

\para{신분에서 계약으로}
`신분'\latin{status}이라는 용어는
이러한 법의 진보를 표현하는 어떤 공식을 만드는 데
유용하게 쓰일 수 있다.
이 공식은, 그 가치가 어떠하든,
내가 보기에 충분히 확인된 것이다.
신분법이 다루는 신분의 모든 형태는
고대 가족에 기거하던 권력과 특권에서 유래한 것이며,
어느 정도는 지금도 그것의 색채를 띠고 있다.
그리하여 우리가 신분이라는 용어를,
최고의 학자들의 용법에 따라,
이러한 신분\latin{personal} 상태만을 의미하는 데 사용하고,
합의에 의해 직^^b7간접적으로 결과하는 상태를
지칭하는 데 사용하지 않는다면,
우리는 \wi{진보하는 사회}의 운동이 지금까지
\hemph{신분에서 계약으로}\latin{from status to contract}의 운동이었다고
말할 수 있을 것이다.

