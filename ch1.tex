
\chapter{고대 법전}

우리가 알고 있는 가장 유명한 법체계는 법전과 함께 시작해서
법전과 함께 끝난다.\footnote{시작은 12표법, 끝은 로마법대전을 뜻한다.}
로마법의 해설자들은 그들의 법체계가 \wi{12표법}\latin{Twelve Decemviral Tables}에 기초하고 있다는,
따라서 성문법에 기초하고 있다는 취지의 말을 그들의 역사 내내 시종일관 해왔다.
한 가지 예외를 제외하면,\footnote{%
%  사용취득(usucapio)에 관한 시민법을 말하는 것일 수 있다.
%  본서 제8장의 사용취득에 관한 설명 참조.
%  또는 어쩌면 문답계약(stipulatio)를 지칭하는 것일 수도 있다.
%  \latin{Gai.\,4.17.}
  12표법에 언급이 없으면서 그후에도 장기간 인정된 제도를 지칭한다고 이해한다면,
  본서 제6장에서 다루고 있는
  코미티아 칼라타(comitia calata)에서의 유언(\pageref{comitiacalata}쪽)을
  뜻하는 것이 아닐까 한다. }
12표법 이전으로 거슬러 올라가는 제도로 로마에서 인정된 것은 없었다.
로마법이 법전의 후예라는 이론, 영국법은 기억할 수 없는
옛 불문\hanja{不文}의 전통에 기원한다는 이론은
로마법의 발달이 왜 영국법의 발달과 달랐는지를 설명하는 주요 이론들이다.
두 이론 다 사실과 정확히 들어맞지는 않지만, 각각 대단히 중요한 결과들을 낳았다.

\para{원초적 법관념}
12표법의 공표가 법의 역사를 다루는 출발점이 될 수 없음은 말할 것도 없다.
문명화된 민족이면 거의 다 고대 로마의 법전과 비슷한 것을 가지고 있었다.
또한 로마와 헬레니즘 세계에 관한 한, 비교적 서로 가까운 시대에
그러한 법전이 두 세계에 널리 확산되었었다.
그것들은 무척 유사한 상황에서 등장했고, 우리가 아는 한
무척 유사한 원인으로 만들어졌다.
많은 법현상들이 이들 법전에 시기적으로 앞서거나 뒤따랐음은 말할 것도 없다.
적지 않은 문헌기록들이 남아있어 법의 초기 현상들을 우리에게 알려준다.
하지만 언어학이 산스크리트 문헌을 완전히 분석해내기 전까지는
우리에게 주어진 가장 좋은 인식 원천은 그리스의 호메로스의 시임에 틀림없다.
물론 이들은 실제 사건들을 기록한 역사로서가 아니라,
\wi{호메로스}가 알고 있던 사회의 상태를 기술한 것으로,
그러나 완전히 이상화시키지 않고 기술한 것으로 읽어야 할 것이다.
영웅시대의 어떤 특징이나 전사들의 용기, 신들의 힘 따위가
시적 상상력에 의해 과장됐을 수 있지만, 도덕적 또는 형이상학적 관념에 의해
그의 시가 오염되었다고 믿을 이유는 없다.
도덕이나 형이상학은 아직 의식적인 고찰의 주제가 아니었기 때문이다.
이런 점에서, 비슷한 초기 시대를 다룬다고 하면서
철학적 또는 신학적 영향 하에서 만들어진 후대의 문헌들보다
호메로스의 시가 훨씬 더 신뢰할 만하다.
법개념의 초기 형태를 발견하려는 우리에게 그것들은 더없이 소중하다.
법학자에게 이들 원초적 관념이 갖는 중요성은
지질학자에게 초기 지구의 지각이 갖는 중요성에 비할 만하다.
거기에는 후대의 법에 의해 발현될 모든 형태들이 다 담겨있을 수 있다.
조급함이나 편견으로 인해 기껏해야 피상적인 조사만 하고는 더는
아무 것도 하지 않은 탓에 오늘날 우리의 법학은 불만족스런 상태에 머물러있다.
법학자들의 탐구는 실로 물리학이나 생리학에서 관찰이 억측을 대체하기 이전
상태와 비슷한 상태에 머물러있다.
그럴듯하고 포괄적이지만 전혀 증명되지 않은 이론들, 가령
자연법이나 사회계약론 따위의 이론이 널리 인기를 구가하여
사회와 법의 원초적 역사에 대한 냉철한 연구를 압도하고 있다.
저 이론들은 진리를 가리고 있거니와,
진리가 발견될 수 있는 유일한 영역으로부터 관심이 멀어지게 할 뿐만 아니라,
일단 길들여지고 믿게 되면 후대의 법학에 참으로 엄청난 영향력을
행사할 수 있다.

\para{테미스테스}
법이나 생활규칙이라는, 이제는 무척 발달한 관념에 관련된 최초의 인식은
\wi{호메로스}가 사용한 용어 ``\wi{테미스}''\latin{Themis}와
``\wi{테미스테스}''\latin{Themistes}에 담겨있다.
주지하듯이 후대 그리스의 신들 중에서 테미스는 정의의 여신으로 나타난다.
하지만 이것은 근대적인, 무척 발달된 관념의 산물이다.
<<\wi{일리아스}>>에서 제우스의 판결보조자로 묘사된 테미스는 전혀 다른 의미를 가졌다.
오늘날 원시사회의 믿을 만한 관찰자들이 밝혀놓았듯이,
인류의 유년기에 인간은
지속적인 혹은 반복적인 사건을
인격의 작용을 가정함으로써만
설명할 수 있었다.
그리하여 바람이 부는 것도 인격이었고 물론 신적 인격이었다.
태양이 뜨고 정점에 이르고 지는 것도 인격이었고 신적 인격이었다.
대지가 수확물을 내주는 것도 인격이었고 신이었다.
물리적 세계가 그러하듯이 도덕적 세계도 마찬가지였다.
왕이 분쟁에서 판결을 내릴 때, 판결은 신적 영감의 결과로 이해되었다.
왕들에게, 또는 왕 중의 왕인 신들에게, 판결을 제안하는 신이
바로 \hemph{\wi{테미스}}였다.
이 관념의 특이함은 복수형 표현에서 나타난다.
테미스의 복수형 \hemph{\wi{테미스테스}}는 신이 판사에게 지시한
판결들 자체를 뜻했다.
왕들은 바로 꺼내 쓸 수 있는 테미스테스의 저장고를 갖고 있다고 생각되었다.
그러나 이것은 법률이 아니라
판결---게르만인들이 ``\wi{둠}''\latin{doom}이라고 부르는 것에 정확히 일치한다---이었다는
점을 유의해야 한다.
그로트\latin{George Grote} 씨의 <<그리스 역사>>\latin{History of Greece}에 따르면,
``제우스나 지상의 인간 왕은 입법자가 아니라 판사였다.''
그에게는 테미스테스가 주어져 있으나,
위로부터 주어진 것이라는 믿음에 부합하게,
판결들은 어떤 일관된 원칙으로 연결되어 있다고 관념되지 못했다.
그것은 따로따로 분리된 개별적인 판결들이었다.

호메로스의 시에서도 이러한 관념은 잠시 동안의 것이었음을 알 수 있다.
단순한 구조의 고대사회에서 상황의 유사성은 오늘날보다 흔한 일이었을 테고,
유사한 소송이 연달아 제기됨에 따라 판결들도 비슷해지는 경향이 나타났을 것이다.
여기서 우리는 관습의 기원 혹은 초기 형태를 발견할 수 있거니와,
이것은 테미스테스, 즉 판결보다 나중에 등장하는 관념인 것이다.\footnote{%
  그러나 왕의 테미스테스도, 이론적으로는 신적 영감에 의한 것이라 해도,
  실제로는 당시의 관습이나 관행에 기초했을 것이 틀림없다.
  \latin{Maine, \textit{Early Law and Custom}, 1883, p.\,163.} }
근대적 사고방식 탓에 우리는 관습의 관념이 사법적 판결에 선행하고
판결은 관습을 확인하거나 그 위반을 벌하는 것이라고 미리 단정짓는
경향이 강하지만, 관념의 역사적 발달은 내가 제시한 순서대로였음이
틀림없어 보인다.
맹아적 관습을 지칭하는 \wi{호메로스}의 용어는 때로 단수형 ``테미스''였고,
종종 ``\wi{디케}''\latin{Dike}였거니와, 그 뜻은 ``판결''과 ``관습'' 또는
``관행''을 넘나드는 것이었다.
노모스\greek{Νόμος}, 즉 `법'은 후대 그리스 사회의 정치용어로서 대단히 중요하고
유명한 것이지만, 호메로스의 시에는 등장하지 않는다.

신의 작용이라는 이러한 관념,
테미스테스를 제안하고 테미스에 인격화되어있는 신의 작용이라는 관념은
피상적인 연구로는 혼동하기 쉬운
다른 원시적 관념들과 엄격히 구분되어야 한다.
힌두의 \wi{마누법전}에 나타나는, 신이 완성된 법전을 명령한다는 관념은
훨씬 최근의 진보된 관념의 계열에 속하는 것으로 보인다.
``테미스''와 ``테미스테스''는
오랫동안 끈질기게 인간의 정신을 지배했던 관념,
신적 영향력이 모든 생활관계와 모든 사회제도를 지탱하고 지지한다는 관념과
훨씬 더 가깝다.
초기 법에서, 그리고 초기의 정치사상에서,
이러한 믿음의 징후는 모든 면에서 나타난다.
초자연적인 통치권자가 당시의 모든 주요 제도들---국가, \wi{씨족}, 가족---을
성별\hanja{聖別}하고 통합하는 것으로 관념된다.
이러한 제도들 속에서 다양한 관계로 집단을 형성하는 인간은
주기적으로 공동의 제의를 수행하고 공동의 희생물을 바칠 의무를 진다.
때로 이러한 의무는
그들이 수행하는 정화의식과 속죄의식에서
더욱 강하게 인식되거니와,
이는 의도치 않게 또는 부주의로 저지른 불경한 짓에 대해 죄를
사하여 달라는 의미를 띠는 것이었다.
고전문헌에 익숙한 독자라면 누구나,
초기 로마의 입양법과 유언법에 중대한 영향을 미쳤던
\wi{씨족제사}\hanjalatin{氏族祭祀}{sacra gentilicia}에 대해 알고 있을 것이다.
무척 진기한 원시사회의 특징들이 고정되어 남아있는
힌두 관습법에서는 지금도 거의 모든 신분법과 상속법 규칙들이
망자의 장례식에서,
즉 가\hanja{家}의 연속성에 단절이 생기는 때에,
의례를 엄정하게 거행하는 것에 달려있다.

\para{벤담의 분석}
이 단계의 법을 떠나기 전에,
특히 영국의 학자들이 유의해야 점을 지적하고자 한다.
\wi{벤담}은 <<정부론 단편>>\latin{Fragment on Government}에서,
\wi{오스틴}은 <<법학의 영역 확정>>\latin{Province of Jurisprudence Determined}에서,
법을 입법자의 \hemph{명령}으로,
그리하여 시민들에게 부과된 \hemph{의무}로,
그리고 불복종에 대해 주어지는 \hemph{제재}의 위협으로 선언한다.
나아가 법의 첫째 요소인 \hemph{명령}은 하나의 행위가 아니라
일련의 또는 다수의 동종 행위들을 지시해야 한다고 단언한다.
이렇게 여러 요소로 분리한 것은 성숙한 단계의 법학에 정확히 부합하는 것이고,
개념을 좀 무리하게 잡아늘이면 모든 시대 모든 종류의 법과
형식적으로 부합하도록 만들 수도 있을 것이다.
하지만, 오늘날에도 일반인들이 가지는 법관념이
이러한 분석과 완전히 일치한다고 주장할 수는 없다.
또한 원시적 사고\hanja{思考}의 역사를 파고들면 들수록, 이상하게도 우리는
벤담이 말한 요소들의 결합을 닮은 법관념으로부터 점점 멀어짐을 발견하게 된다.
확실히 인류의 유년기에는 어떠한 입법도, 어떤 뚜렷한 입법자도 생각될 수 없었다.
법은 관습의 언저리에도 도달하기 어려웠다.
법은 오히려 습관이었다.
프랑스식 표현으로 법은 ``대기 중에 퍼져 있었다''\latin{in the air}.
옳고 그름의 유일한 권위적 진술은 사건이 일어난 뒤에 내려지는 판결이었다.
위반된 법을 전제하여 내려지는 판결이 아니라,
재판의 순간에 저 위의 권력이 판사의 마음에
처음 영감을 불어넣어 내려지는 판결이었다.
물론 우리는 시간적^^b7관념적으로 우리와 멀리 떨어진 사고방식을
이해하기가 무척 어렵다.
그러나 고대사회의 헌정을 더 장기간 천착하고 나면 그것은 더 설득력있게
다가올 것이다.
고대사회에서는 모든 사람이
생애 대부분을 가부장의 전제\hanja{專制} 아래서 살았으므로
모든 행위는 사실상 법이 아닌 변덕에 의해 통제되었던 것이다.
생각건대 다른 나라 사람보다 영국인은
``테미스테스''가
어떤 다른 법 관념보다
선행한다는 역사적 사실을 더 쉽게 이해할 수 있을 것이다.
왜냐하면 영국법의 성격에 관한 여러 부조리한 이론들 중에서도
가장 널리 퍼져있는, 적어도 실무에 가장 영향력 있는 이론은 분명
판결과 선례가 규칙이나 원리나 개념구분에 선행한다는 이론이기 때문이다.
주목할 점은,
벤담 및 오스틴의 견해에서
법이 단일한 또는 단순한 명령과 구별되었듯이,
``\wi{테미스테스}''에서도 양자가 구별된다는 것이다.
진정한 법은 유사한 종류의 행위를 모든 시민에게 똑같이 명한다.
이것이야말로 대중들의 마음에 깊이 각인된 법의 성질이며,
``법''\latin{law}이라는 말이 단순히 불변성, 연속성, 유사성에도 사용되고 있는
이유이다.\footnote{%
  이런 맥락의 `법'을 우리말로는 보통 `법칙'이라고 부른다. 중력의 법칙 등. }
이에 비해 \hemph{명령}은 하나의 행위만 지시하며,
따라서 ``테미스테스''는 법보다는 명령에 더 가깝다.
그것은 따로 떨어진 하나의 사실관계에 대한 재판일 뿐이며,
전후의 판결들 간에 규칙적인 연계가 반드시 존재하지는 않는다.

\para{귀족정 시기}
영웅시대의 문헌은 ``테미스테스''와 이보다 좀 더 발달된 ``디케''라는 말로써
맹아기의 법을 우리에게 드러내보인다.
법의 역사의 다음 단계는 무척 흥미로운 시기이다.
그로트 씨의 <<역사>> 제2부 제9장은 호메로스가 묘사했던 것과는
사뭇 다른 성격의 사회가 등장하는 과정을 잘 기술하고 있다.\footnote{%
  \latinmarks
  George Grote,
  \textit{History of Greece},
  Vol.\,3,
  Boston: John P. Jewett and Company,
  1852. }
영웅시대의 왕의 권위는 부분적으로는 신에게서 부여받은 대권에,
또 부분적으로는 탁월한 힘과 용기와 지혜를 가진 데 의존했다.
점차, 왕의 신성함에 대한 관념이 약해지고 또
일련의 세습 과정에서 허약한 왕들이 배출됨에 따라
왕의 권력은 쇠퇴했고, 마침내는 귀족정으로 대체되었다.
혁명에 관한 정확한 용어를 사용할 수 있다면,
\wi{호메로스}가 여러 번 언급했던 족장들의 위원회\latin{council of chiefs}에 의해
왕의 자리가 찬탈당했다고 말할 수 있을 것이다.
여하튼 이제 유럽 각지에서 왕정 시대가 가고 과두정의 시대가 도래했다.
왕이라는 직함이 완전히 없어지지 않은 곳에서도 왕의 권위는 그저
이름에 불과했다.
라케다이몬에서처럼 그저 세습장군이거나,
아테네의 \wi{아르콘} 왕처럼 그저 관리이거나,
로마의 \wi{제사왕}\hanjalatin{祭祀王}{rex sacrificulus}처럼
그저 사제\hanja{司祭}에 불과했다.
그리스, 이탈리아, 소아시아에서
지배집단은 어디서나
가상의 혈연관계로 결합된 다수의 가\hanja{家}로
구성되었다.
애초에 그들은 모두 일종의 신성성을 주장했으나,
그들이 힘이
자칭의 신성성에 기반했던 것 같지는 않다.
민중파에 의해 일찍이 전복되어버린 경우가 있었거니와,
그렇지 않은 경우 결국 그들 모두는
오늘날 우리가 정치적 귀족이라고 부르는 것에 아주 근접해갔다.
이탈리아와 그리스 세계의 이러한 혁명에 비해,
더 먼 아시아 지역의 공동체에서의 사회 변화는
물론 시간적으로 훨씬 더 전에 일어났다.
하지만 문명화과정에서 이들 변화의 상대적 위치는 동일했고,
변화의 일반적 성격도 대단히 유사했던 것 같다.
나중에 페르시아 군주정 아래 통합되는 제 민족들이,
그리고 인도 반도 곳곳에 살았던 제 민족들이,
모두 영웅시대와 귀족정시대를 거쳤다는 여러 증거가 있다.
하지만 여기서는 군사적 귀족과 종교적 귀족이 각각 따로 성장했고,
왕의 권위도 대체로 폐기되지 않았다.
또한 서구의 역사 전개와 달리, 동양에서는
종교적 요소가 군사적^^b7정치적 요소를 압도하는 경향이 있었다.
왕과 사제집단의 틈바구니에서 군사적^^b7세속적 귀족은 보잘 것 없이
절멸당하고 파괴당하여 사라진다.
그리하여 도달한 최종 결과는
왕이 커다란 권력을, 그러나 사제계급의 특권에 의해 제한되는 권력을,
누리게 되는 것이다.
동양의 종교적 귀족과 서양의 세속적^^b7정치적 귀족이라는
이러한 차이에도 불구하고,
영웅적 왕의 시대에 이어 귀족정 시대가 도래한다는 역사적 명제는
참이라 간주해도 좋을 것이다.
전 인류에 타당할지는 모르겠으나, 적어도 인도^^b7유럽 계통 민족들에게는
두루 타당한 것이다.

\para{관습법}
법학자들이 주목할 점은
어디서나 이들 귀족이 법의 저장소이고 법의 집행자였다는 것이다.
그들은 이제 왕의 대권을 계승한 것으로 보인다.
그런데 중요한 차이가 있거니와,
그들은 매번 판결마다 직접 신의 영감을 받는다고 내세우지 않았다.
가부장적 족장들의 판결이 초인간적 지시에 연결된다는 관념은
법규칙의 전부 또는 일부가 신에게서 기원한다는 주장을 통해 여기저기서 여전히
나타나고 있지만,
사고의 발달로 이제 더는 구체적 분쟁의 해결을
인간 외부의 힘의 개입으로 설명할 수 없게 되었다.
법적 과두정이 주장하는 바는 이제 법\hemph{지식}의 독점, 즉
분쟁을 해결하기 위한 법원칙을 그들만이 가진다는 것이다.
실로 우리는 \wi{관습법}\latin{customary law}의 시대에 들어선 것이다.
이제 관습이나 관례는 실체적인 집합체로 존재하고,
귀족 집단 혹은 귀족 카스트가 그것을 정확히 알고 있다고 간주된다.
옛 전거들에 따르면 과두정에 주어진 이러한 신뢰가
때로 남용되기도 했음이 분명하지만,
이를 단순한 찬탈이나 폭정의 장치로만 보아서는 안 될 것이다.
문자 발명 이전에는, 그리고 기술이 유년기에 머물던 시절에는,
법적 특권을 가진 귀족들이야말로 민족 혹은 부족의 관습을
거의 정확하게 보존하는 유일한 현실적 방법을 구성했다.
공동체의 일부 구성원의 기억에 관습을 맡김으로써
관습의 진정성은 최대한 담보될 수 있었다.

관습법의 시대, 그리고 특권 계급에 의한 관습법의 보존은
자못 흥미를 불러 일으킨다.
당시의 법 상태는 오늘날의 법률용어나 일상용어에도 그 흔적을 남기고 있다.
그리하여
카스트이든, 귀족이든, 사제 지파든, 신관단\latin{sacerdotal college}이든,
특권을 가진 소수만이 알고 있는 법은 진정한 불문법이다.
이것을 제외하면 세상에는 불문법이 존재하지 않는다.
영국 판례법이 흔히 불문법이라 불리고 있고, 또 어떤 영국 학자들은
영국법을 법전으로 편찬하면 불문법이 성문법으로
대체---그들이 비판적인 취지에서 그러나 사뭇 진지하게 사용하는 용어로는,
개종---될 것이라고 주장한다.
물론 영국 보통법을 마땅히 불문법이라고 칭해도 좋을 시기가 한때 있었음이
분명하다.
영국의 옛 판사들은 변호사나 일반인은 온전히 알 수 없는
규칙, 원리, 개념구분 등을 알고 있다고 내세웠다.
그들이 독점한다고 주장한 법의 전부가 진정 불문법이었는지는 무척 의문스럽다.
하지만, 어쨌든 판사들에게만 알려진 민사 및 형사 규칙들이 한때 상당히 있었다고
가정하더라도, 오늘날에는 그것은 더 이상 불문법이 아니다.
웨스트민스터 홀의 법원들이
연감\latin{yearbook} 등에 기록된 선례에 따라 판결을 내리기 시작하면서,
그들의 법은 성문법이 되었다.\footnote{메인의 이러한 성문법 개념은
  오늘날 통용되는 개념과 다르다는 데 주의할 것. 우리는 판례법, 관습법,
  조리법 등을 모두 불문법으로 분류한다.
  메인이 연감에 기록된 옛 보통법 판례의 성문법성을 주장하는 것은
  이를 일종의 `고대법전'으로 간주하기 위해서인 듯하다.}
오늘날 영국의 법규칙은 우선 인쇄된 선례의 사실관계로부터 분리되고,
특정 판사의 성향, 꼼꼼함, 지식에 따라 어떤 언어의 형식으로 만들어진 후,
해당 사건의 사실관계에 적용되는 것이다.
그러나 이 과정의 어느 단계에서도 성문법과 구별되는 성질은 나타나지 않는다.
그것은 성문의 판례법인 것이다.
법전법과 다른 점은 단지 쓰여진 방식이 다르다는 것뿐이다.

\para{12표법}
관습법의 시대로부터 이제 우리는 법제사에 뚜렷이 획을 긋는 다른
시대로 진입하게 된다.
그것은 \index{법전 시대}법전\latin{code} 시대로,
로마의 \wi{12표법}으로 대표되는 고대 법전의 시대다.
그리스에서, 이탈리아에서, 그리스화된 서아시아 해안 지역에서,
이들 법전은 모두 어디서나 동일한 시기에 등장했다.
여기서 동일한 시기란
시간적으로 동시라는 뜻이 아니라,
각 공동체의 상대적 진화 단계에서 유사한 시기를 점한다는 뜻이다.
내가 언급한 지역 어디서나 법은 판자\latin{tablets}에 새겨져 대중에게 공표되었고,
그리하여 특권 귀족의 기억 속에 저장된 관행들을 대체했다.
오늘날의 법전편찬이라는 것에 가까운 어떤 세련된 숙려가
내가 말한 변화에 조금이라도 들어있었다고 생각해서는 안 된다.
고대 법전은 애초에 문자 기술의 발견과 확산에 의해 도입된 것이 분명하다.
물론 귀족들이 법지식의 독점을 남용했음에 틀림없고,
어쨌든 그들에 의한 배타적 법 전유\hanja{專有}가 서구에서 보편적으로 등장하기 시작한
민중 운동의 성공에 커다란 장애가 되었던 것은 사실이다.
하지만, 비록 민주적 감정이 법전의 확산에 도움을 주었을지라도,
대체로 법전은 문자 발명의 직접적 산물이었음이 확실하다.
일군의 사람들의 기억이
비록 반복적 사용에 의해 강화된다 할지라도,
그러한 기억보다는
글자가 새겨진 판자가 법의 저장소로서 더 훌륭했고,
법의 정확한 보존을 더 잘 담보했다.

로마의 법전은 내가 묘사한 그러한 유형의 법전에 속한다.
그것의 가치는 조화로운 분류라든가 표현의 간결성과 명확성 따위에
있는 것이 아니라, 그 공개성, 즉 무엇을 하고 무엇을 하지 말아야 할 지에 관한
지식을 모든 사람들에게 제공하는 데 있었다.
물론 로마의 \wi{12표법}은 어느 정도 체계성을 보여주긴 하지만,
아마도 이는 후기 그리스의 발달된 입법기술을 갖춘 그리스인들의 도움을 받아
12표법이 기초되었다는 전승\hanja{傳承}에 의해 설명할 수 있을 것이다.
하지만 아테네의 솔론 법전의 남아있는 단편들은
체계가 별로 없었음을 보여주며, 아마도 드라콘의 입법은 더욱 그러했을 것이다.
또한 동^^b7서양을 막론하고 이들 법전의 유물들은
종교적, 시민적, 그리고 단순한 도덕적 명령들이
그 성질의 차이를 고려하지 않은 채 무질서하게 혼재되어 있었음을 보여준다.
이는 법 외의 다른 분야의 초기 사상에 관해 우리가 알고 있는 것과 일치한다.
법과 도덕의 분리, 법과 종교의 분리는 정신의 진화에서
분명히 더 후대의 단계에 속하는 것이다.

\para{마누법전}
그러나, 현대인의 눈에 이들 법전이 아무리 이상하게 보일지라도,
고대사회에서 이 법전들의 중요성은 이루 다 말할 수 없을 정도이다.
문제는---이는 각 공동체의 장래에 큰 영향을 미치게 되는 것인데---도대체
법전이 있어야 하는가 아닌가가 아니었다.
대부분의 고대사회는 어쨌거나 조만간 법전을 가지게 되기 때문이거니와,
봉건제에 의해 만들어진 법제사의 큰 단절이 없었다면
모든 근대법은 이들 원천 중 하나 이상으로
기원을 소급할 수 있었을지도 모른다.
오히려 인류 역사의 전기\hanja{轉機}는
사회 진화의 어느 시기, 어느 단계에서 그들의 법이 성문화되었는가와 관련된다.
서양에서는 각 나라의 평민적^^b7민중적 요소가 과두제의 독점을 성공적으로
공격했고, 국가 역사의 비교적 초기에 거의 보편적으로 법전을 획득했다.
하지만 전술했듯이 동양에서는 군사적^^b7정치적 귀족이 아니라
종교적 귀족이 지배 귀족이 되어 권력을 장악하는 경향이 있었다.
그런데 몇몇 경우 서구에 비해 아시아 나라들은 그 물리적 조건으로 인해
개별 공동체가 더 커지고 인구도 더 많아지는 경향이 있었다.
그리고 어떤 제도가 적용되는 공간이 크면 클수록
그 제도의 완고함과 생명력이 더 커진다는 것은 널리 알려진 사회법칙이다.
원인이야 어찌되었든, 동양사회의 법전은 서구에 비해
상대적으로 훨씬 늦게 획득되고, 그리하여 사뭇 다른 성격을 띠게 된다.
아시아의 종교적 귀족들은 스스로 참고하기 위해서든, 기억의 괴로움을
덜기 위해서든, 후계자의 교육을 위해서든, 어쨌거나
그들의 법지식을 종국에는 법전의 형태로 구체화하기에 이른다.
그러나 자신들의 영향력을 확대하고 공고히하려는 유혹이 너무나 강해서
이에 저항하기 어려웠을 것이다. 즉,
법지식을 완전히 독점하고 있었기에 그들은
법전화를 되도록 미룰 수 있었을 것이다.
그들의 법전은 실제로 행해지는 규칙이 아니라,
준수하는 것이 마땅하다고 사제집단이 생각한 규칙들을 모은 것이다.
\wi{마누법전}\latin{Laws of Manu}이라 불리는 힌두법전은 브라만들이 집성한 것으로,
물론 인도인들이 실제로 준수한 것들을 다수 간직하고는 있지만,
오늘날 최고 가는 동양학자들의 견해에 따르면
전체적으로 그것은 인도에서 실제로 행해지던 규칙들의 집합이 아니다.
대체로 그것은 브라만들이 보기에 법\hemph{이어야 할} 것들을
이상적으로 그려놓은 것이다.
인간의 본성을 감안할 때, 그리고 그 저자들의 특별한 동기를 감안할 때,
마누법전 같은 것이 아주 오래 전의 것인 양 내세워지고
그 완전한 형태로 신에게서 유래한 것이라 주장되는 것은 당연한 일에 속한다.
힌두 신화에 따르면 마누는 최고 신의 화신이다.
하지만 그의 이름이 붙어있는 법전은, 비록 정확한 연대는 알 수 없지만,
힌두법의 진화 과정 중에 상대적으로 최근의 산물이다.

\para{타락}
\wi{12표법} 등의 법전이 그것을 획득한 사회에 가져다준 주요 이점은
특권 귀족들의 기만적 행태에 대한 보호막을,
그리고 국가 제도의 자연적 타락에 대한 보호막을 제공한 것이었다.
로마의 법전은 단순히 로마 인민의 기존 관습을 언어로 선언한 것이었다.
그것은 로마의 문명화 과정에서 상대적으로 무척 이른 시기에 법전화된 것이었고,
시민적 책무와 종교적 의무가 착종되어 있던 지적 상태를 아직
로마 사회가 거의 벗어나지 못했을 때에 공표된 것이었다.
그런데 이와 달리 여전히 관습을 준행하는 미개한 사회는
문명의 진보에 전적으로 치명적일 수 있는 어떤 특별한 위험에 노출된다.
공동체가 그 유년기에, 원시적 단계에 채택한 관행들은
대체로 그 물질적^^b7정신적 복리의 증진에 가장 적합한 경우가 일반적이다.
새로운 사회적 필요가 새로운 관행을 낳을 때까지 그것들이 순수하게 보존된다면
사회의 상승적 행진은 거의 확실해진다.
하지만 불행하게도 불문\hanja{不文}의 관행에 기초한 작동에는 그것에 위협이 되는
어떤 발전 법칙이 존재한다.
관습을 준수하는 대중들은 그 유용성의 진정한 근거를 알지 못한 채
당연한 듯 관습을 준수하거니와,
따라서 그들은 불가피하게 준수의 미신적 근거를 발명해낸다.
그리하여 합리적인 관행이 비합리적인 관행을 낳는다는 표현으로
간단히 묘사될 만한 어떤 과정이 시작된다.
유추\hanja{類推}는 성숙기 법학에서는 무엇보다 유용한 도구이지만,
유년기에는 무엇보다 위험한 덫이 된다.
어떤 합당한 이유로 애초에 특정한 하나의 행위에만 국한되던 명령과 금지가
동일한 유\hanja{類}의 다른 모든 행위들에도 적용되기 시작한다.
하나의 행위가 야기하는 신의 분노에 두려움을 느낀 인간은
그것과 조금밖에 비슷하지 않은 다른 행위에 관해서도
자연스레 공포를 느끼기 때문이다.
위생상의 이유로 어떤 음식이 금지되면,
그럴듯한 유추에 때로 의존하여
그 금지는 유사한 다른 모든 음식에도 확장된다.
또한, 일반적 청결을 보증하는 현명한 규칙 하나가 이윽고
판에 박힌 의례적 세정\hanja{洗淨}행위의 기나긴 목록을 명령하게 된다.
또한, 역사 과정의 특정한 위기 시에 국가의 존립을 위해 잠시 필요했던
계급의 구분이 인류의 제도 중에 가장 재앙적이고 파멸적인 것---카스트---으로
타락한다.
힌두법의 운명은 실로 로마 법전의 가치를 보여주는 척도다.
민족학은 로마인과 인도인이 원래 동일한 계통에서 발원했음을 알려준다.
사실 그들의 최초의 관습으로 여겨지는 것들 간에는
대단히 큰 유사성이 있다.
오늘날에도 힌두법의 밑바탕에는 선견지명과 건전한 판단이 깔려있다.
하지만 비합리적인 모방으로 인해 잔인하고 부조리한 거대한 장치가
힌두법에 접목되었다.
로마인들은 그들의 법전에 의해 이러한 타락으로부터 보호될 수 있었다.
그것은 그들의 관행이 아직 건강했을 때 편찬되었거니와,
만약 백년 후였다면 너무 늦었을지도 모른다.
힌두법은 그 대부분이 성문화되었다.
그러나,
산스크리트어로 전해지는 집성들은 일응 오래된 것이긴 하지만,
해악이 작용한 연후에 작성되었다는 풍부한 증거를 담고 있다.
만약 12표법이 공표되지 않았다면 로마인들도 인도인들처럼
허약하고 타락한 문명으로 전락할 운명이었을지에 관해
물론 우리는 아무 것도 말할 수 없다.
하지만 한 가지 확실한 점은 그들의 법전과 \hemph{더불어}
로마인들은 저 불행한 운명으로부터 벗어날 수 있었다는 것이다.


