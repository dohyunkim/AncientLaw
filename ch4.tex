\chapter{자연법의 근대사}

지금까지의 논의로부터, 로마법의 변화를 가져온 저 이론은 어떤 철학적 엄밀성을
주장한 것이 아니었음을 알 수 있을 것이다.
그것은 사실 일종의 ``혼합적 사고양식''이었거니와,
이러한 사고양식은 오늘날 최고의 정신을 제외한 모든 인간 정신의 유아기적 사고의
특징으로 인식되고 있으며 또한 현대인의 정신에서도 어렵지 않게
발견할 수 있는 것이다.
자연법이론은 과거와 현재를 혼동했다.
논리적으로는, 그것은 한때 자연법에 의해 통치되었던 자연상태를 상정한다.
하지만 로마의 법학자들은 그러한 자연상태의 존재를 분명하게 그리고
자신있게 말하지 않았다. 사실 황금시대를 상상하는 시적인 표현을
제외하면 고대인들은 그러한 상태에 대해 거의 언급하지 않았다.
실무적 목적에서는, 자연법은 현재에 속하는 어떤 것이고,
기존의 제도와 얽혀있는 어떤 것이며, 유능한 관찰자에 의해
기존 제도와 구분될 수 있는 어떤 것이다.
자연의 명령을 이와 함께 섞여있는 조잡한 요소들로부터 분리하는 기준은
단순성과 조화성의 감각이었다.
하지만 이들 더 세련된 요소가 애초 존중받은 것은
단순성과 조화성 때문이 아니라,
자연의 원초적 지배의 후예라는 데에 있었다.
이러한 혼동은 현대의 법학자들에 의해서도 성공적으로 설명되지 못했다.
실로
로마 법률가들이 받아야 할 비난보다 오히려
오늘날의 자연법사상이 인식의 불명료성을 훨씬 더 많이 노정하고 있으며
언어의 절망적인 모호성에 의해 더 많이 오염되어 있다.
이 주제에 관한 저자들 몇몇은, 자연법법전은 미래에 존재하는 것이고
모든 시민법들이 지향해야 할 목표라고 주장함으로써,
이러한 근본적인 난제를 피해가려고 시도하나,
이는 옛 이론이 근거하고 있던 가정을 순서만 뒤집는 것이거나,
아니면 서로 양립할 수 없는 두 이론을 뒤섞는 것에 불과할 것이다.
과거가 아니라 미래에서 완전성을 찾는 경향은 기독교에 의해
이 세상에 도입된 것이다.
사회의 진보가 더 나쁜 것에서 더 좋은 것으로 필연적으로 진행된다는 믿음은
고대 문헌에서는 거의 혹은 전혀 발견되지 않는다.

하지만 그 철학적 결함에 비해 이 이론이 인류에게 미친 영향은 훨씬 더 심대했다.
만약 자연법의 믿음이 고대세계에 보편적으로 퍼지지 않았다면
어떤 사상사적 전환이, 또 그에 따른 인류사적 전환이, 일어났을까는
실로 말하기가 쉽지 않다.

\para{자연법}
법, 그리고 법에 의해 결합되는 사회는 그 유아기에
두 가지 위험에 특히 취약하다.
하나는 법이 너무 빨리 발달할 수 있다는 것이다.
진보적인 그리스 공동체들에서 이런 일이 발생했거니와,
이들 공동체는 놀라운 능력으로 불편한 소송절차와 불필요한 법률용어의
질곡을 벗어던졌고, 곧이어 엄격한 규칙과 법규정들에 미신적 가치를 부여하는 일을
그만두었다.
이것으로 그 공동체의 시민들이 누린 직접적 혜택은 상당히 컸지만,
그것은 인류의 궁극적 이익에 기여하지는 못했다.
민족성의 드문 자질 중 하나는,
보다 높은 이상에 법을 일치시키려는 희망을 잃지 않으면서도,
법 자체의 적용과 운용에 있어
추상적 사법\hanja{司法}을 구현하는 데는 지속적으로 실패하는 능력이다.
유연성과 탄력성에 뛰어난
그리스의 지식인들은
엄격한 법형식의 틀 속에 스스로를 가둘 수가 없었던 것이다.
우리가 비교적 소상히 알고 있는 아테네의 인민법원을 두고 판단하건대,
그리스의 법원은 법률문제와 사실문제를 혼동하는 경향을 강하게 나타냈다.
아리스토텔레스의 <<수사학>>\latin{Treatise on Rhetoric}에 남아있는
웅변가\latin{orator}들과 법정 표현들의 흔적을 보건대,
순수한 법률문제의 변론은
판사들에게 영향을 줄 수 있는 모든 것을 끊임없이 고려하면서 이루어졌다.
이런 방식으로는 지속가능한 법학체계가 만들어질 수 없다.
특정 사건의 사실관계에 대한 완벽한 이상적인 결정에 성문법 규칙이 방해되는
경우라면 언제나 그 성문법 규칙을 완화하는 데 거리낌이 없었던 공동체는,
설령 후대에 어떤 법원리들을 물려준다 하더라도
오직 당대에 지배적이었던 옳고 그름의 관념에 기초한 것들만 물려줄 수 있을 뿐이다.
이러한 법은 후대의 보다 발달된 관념에 어울릴만한 틀을 전혀 제공할 수 없다.
기껏해야 그 법을 둘러싼 문명의 볼완전성을 드러내는 철학이 될 수 있을 뿐이다.

국가 사회 중에 그들의 법이
이러한 때이른 성숙과 때아닌 해체의 위험에 의해 위협받은 곳은 많지 않다.
로마인들이 이러한 위협에 심각하게 노출된 적이 있었는지는 모르겠으나,
어쨌든 그들은 자연법이론이라는 적절한 보호장치를 가지고 있었다.
분명 법학자들은 시민법을 점진적으로 흡수하는 체계로 자연법을 관념했으며,
시민법이 폐지되지 않는 한 자연법이 시민법을 대체할 수는 없다고 생각했다.
특정 소송사건을 감독하는 판사들이 자연법의 호소에 압도당할 정도로
그렇게 자연법이 신성하다는 인상은 유포되지 않았다.
이러한 관념의 가치와 유용성은 완벽한 유형의 법이 인간 정신의 눈 앞에
펼치지지 못하게 한 것이었고, 그러한 법에 무한히 가까이 다가갈 수 있다는
희망을 품지 못하게 한 것이었으며, 또한 아직 자연법에 조응하지 못한
기존 법이 부과한 의무를 실무가나 시민들이 거부하지 못하게 한 것이었다.
무엇보다 중요한 점은 이 모범적인 체계가,
후대에 인간의 희망을 꺾어놓았던 다른 많은 체계들과 달리,
결코 상상의 산물이 아니었다는 것이다.
그것은 결코 허황된 원리에 기초한 것으로 관념되지 않았다.
그것은 기존 법의 저변에 존재하며 기존 법을 통하여 추구되어야 한다고 생각되었다.
한마디로 그것의 기능은 구제수단을 제공하는 데 있었지, 혁명적이거나
무정부적인 것이 아니었다.
그리고 정확히 이 점에 있어 불행하게도 근대 자연법 사상은 고대의 그것을
닮지 않은 경우가 많다.

\para{벤담주의}
유아기의 사회에 나타나는 또 하나의 취약성은 훨씬 더 많은 민족들의 진보를
방해하고 가로막았다.
원시법의 엄격성은 대개 일찍이 종교와의 관련성 및 동일시에 의해 등장했거니와,
대부분의 민족들은 그 관행이 처음 체계적인 형태로 굳어질 당시 그들이
가지고 있던 인생관과 행위관에 얽매여왔다.
놀라운 운명으로 이러한 재난을 벗어난 민족이 한 둘 있거니와,
이들 나무줄기에 접목하여 몇몇 근대사회가 기름진 곳이 될 수 있었다.
하지만 여전히 세계의 더 많은 곳에서는 최초 입법자가 그려놓은 기본계획을
추종하는 것이 법의 완성이라고 여겨지고 있다.
만약 그런 곳에서 지식인이 법을 공부했다면, 한결같이 그들은
고대 텍스트에서 그 문자적 의미를 눈에 띄게 벗어나지 않고 이끌어낸 결론의
미묘한 고집스러움을 자랑스러워했을 것이다.
만약 자연법이론이 비범한 탁월함을 로마법에 주지 않았더라면
로마인들의 법이 인도인들의 법에 비해 우월하다고 할 것이
무엇이 있는지 나는 모르겠다.
이 유일한 예외적인 사례에서, 다른 여러 이유들로 인류에 막대한 영향을 끼치게 될
한 사회의 눈 앞에, 단순성과 조화성이 이상적이고 가장 완전한 법의 성질로
나타났던 것이다.
진보를 추구함에 있어 어떤 뚜렷한 목표를 가진다는 것이
한 민족이나 전문직업군에게 가지는 중요성은 아무리 강조해도 지나치지 않다.
지난 30년 동안 영국에서 벤담이 가졌던 막대한 영향력의 비밀은
이 나라 앞에 그러한 목표를 성공적으로 제시한 데 있다.
그는 우리에게 개혁의 뚜렷한 원칙을 제시했다.
지난 세기의 영국 법률가들은 아마도 명민했기에
영국법이 인간 이성의 완성이라는 흔해빠진 역설적 표현에 눈멀지는 않았겠지만,
일을 추진해나갈 다른 원리가 없었기 때문에 그것을 믿는 척 행동했다.
벤담은 다른 모든 목표 위에 공동체의 선\hanja{善}을 두었고,
그리하여 오랫동안 밖으로 빠져나갈 길만 찾고 있던 흐름에 나갈 길을 열어주었다.

우리가 기술해온 관념을 벤담주의의 고대적 대응물이라고 부른다면
그것은 그리 훌륭한 비교가 못 될 것이다.
로마인의 이론은 저 영국인의 이론과 마찬가지 방향으로 인간의 노력을 이끌었다.
그것의 실천적 결과도 공동체의 일반적 선\hanja{善}을 꾸준히 추구해온
일군의 법개혁가들이 달성한 것과 크게 다르지 않았다.
하지만 그것이 벤담의 원리를
의식적으로 예견한 것이었다고 보는 것은 잘못일 것이다.
로마인들의 대중문헌이나 법학문헌에서 분명 인류의 행복은
구제수단을 제공하는 입법의 목표로 때로 제시되곤 했지만,
자연법이라는 돋보이는 주장에 주어진 끊임없는 칭송에 비하면
벤담의 원리를 보여주는 증언은 거의 없거나 희미하다는 점을
주목해야 한다.
로마 법학자들이 기꺼이 수용한 것은 인간에 대한 사랑 같은 것이 아니라
단순성과 조화성---그들이 ``전아''\hanjalatin{典雅}{elegance}하다고 부르며
강조했던 것---의 감각이었다.
그들의 노력이 보다 정밀한 철학의 조언을 받아들인 자들의 노력과
우연히 일치했다는 것은 인류에게는 행운이었다.

\para{프랑스 법률가들}
자연법의 근대사로 전환하면, 우리는 그것의 영향력이 막대하다는 것은
말하기 쉬워도 그 영향이 좋은 것인지 나쁜 것인지를 자신있게
말하기는 어렵다는 것을 알고 있다.
근대 자연법 이론에서 나왔다고 할 수 있는 신조와 제도들은
우리 시대의 가장 첨예한 논쟁의 대상이거니와,
지난 백년 동안 프랑스가 서구 세계에 확산시킨 법, 정치, 사회에 관한
특수한 이념들 대부분이 자연법이론에 그 원천을 두고 있다고 말할 때
이것이 잘 드러난다.
프랑스 역사에서 법률가들의 역할은,
그리고 프랑스 사상에서 법사상의 비중은,
언제나 대단히 컸다.
근대 유럽의 법학이 발흥한 곳은 사실 프랑스가 아니라 이탈리아였지만,
이탈리아로 유학하고 돌아와 전 유럽대륙에 건설된,
그리고 {\small(허사로 돌아갔지만)} 우리 영국에 건설이 시도된,
학자군 중에서 프랑스에 건설된 학자군이 그 나라의 운명에 가장 큰
영향을 끼쳤다.
프랑스의 법률가들은 즉각 카페 왕조 및 발루아 왕조의 왕들과
강력한 동맹관계를 형성했다.
프랑스의 왕권이 여러 소국과 속령들의 집합체에서 떨어져나와
결국 그 위에 성장하게 된 것은 무력에 의한 것인 동시에
법률가들이 왕의 대권을 옹호해주고 봉건적 세습규칙을 해석해 준 데에도 기인했다.
왕과 법률가들의 동맹으로 프랑스 왕들이
강한 봉건제후들, 귀족들, 교회와의 투쟁과정에서 누린 이점은
중세를 거슬러 올라가 당시 유럽을 지배하던 이념을 고려하지 않으면
제대로 평가할 수 없다.
우선 일반화를 향한 강한 열정이 있었고 모든 일반적 명제를 향한 찬양이 높았다.
그리하여 법의 분야에서는, 여러 지방에서 관습적으로 사용되던 고립된
다수의 법규칙들을 하나로
포괄하고 요약하는 모든 일반적 공식\latin{formula}에 대한 무의식적 존중이 있었다.
이러한 공식은 로마법대전이나 표준주석\latin{the Glosses}\footnote{%
  주석학파를 집대성한
  아쿠르시우스(Accursius)의 표준주석(glossa ordinaria)을 말하는 듯.}에
익숙한 실무가라면 물론 얼마든지 제공할 수 있었다.
하지만 법률가들의 권력을 사뭇 증대시킨 또 다른 원인도 있었다.
우리가 말하는 이 시대에는 쓰여진 법 텍스트가 가진는 권위의 정도와 성질에 관한
관념이 보편적으로 퍼져있었다.
대체로 ``이것이 쓰여진 법이다''\latin{Ita scriptum est}라는 우선권을 가진
주장은 모든 항변을 침묵시키기에 충분했다.
우리 시대의 학자라면 인용된 공식을 조바심내며 조사하고,
그 출처를 따져묻고, {\small(필요하다면)} 인용이 들어있는 법령집이
지방관습을 대체할 만한 권위를 가지고 있지 않다고 부인하겠지만,
그 시대의 법학자들은 규칙의 적용가능성을 의문시하거나
기껏해야 학설휘찬이나 교회법에서 반대명제를 인용하는 것 외에
다른 것은 거의 시도하지 않았다.
법논쟁의 이러한 중요한 측면에 대해 사람들이 주저하는 관념을 가졌다는 것을
염두에 두는 것은 무척 중요하거니와,
그것은 법률가들이 국왕에게 힘을 실어준 것을 설명하는 데 도움이 될 뿐만 아니라,
몇몇 흥미로운 역사적 문제를 해명하는 데도 도움을 주기 때문이다.
위조된 교령\hanja{敎令}들\latin{forged decretals}\footnote{%
  `콘스탄티누스의 기증'을 포함한 `이시도르 위서'%
  \hyphlatin{(Pseudo-Isidore)}를 말한다.}을
만든 저자의 동기와 그의 특별한 성공은 이런 맥락에서
더 잘 이해될 수 있는 것이다.
보다 덜 흥미로운 예를 들자면, 브랙턴\latin{Bracton}의 표절을 이해하는 데도,
비록 부분적이지만, 도움을 준다.
헨리3세 시대의 저 영국 법률가는
순수한 영국법의 집성을 당대 영국인들에게 내놓을 수 있었는데,
이는 편제의 전부와 내용의 1/3을 로마법대전에서 직접 빌려온 논저였다.
로마법의 체계적인 연구가 공식적으로는 금지된 나라에서
이러한 작업을 감행했을 것이니, 이는 법학의 역사에서 영원히 풀리지 않는
수수께끼의 하나일 것이다.
그러나, 텍스트의 출처에 대한 고려는 차치하고라도,
쓰여진 텍스트가 가지는 구속력에 관한 당대의 여론 상황만 감안해도
우리의 놀라움은 다소간 완화된다.

프랑스의 왕들이 주권 확립을 위한 긴 투쟁을 성공적으로 종결지은 때,
즉 대체로 발루아^^b7앙굴렘 왕조의 재위기에 이르러,
프랑스 법률가들의 상황은 사뭇 특수한 것이었고 이 상태는
프랑스혁명 발발 시까지 지속된다.
한편으로, 그들은 프랑스에서 가장 식자층에 속했고
대단히 강력한 권세를 누리는 계급을 형성했다.
그들은 봉건귀족들과 나란히 특권계급의 신분을 가졌으며,
프랑스 전역에 걸쳐 분포한 조직을 통해 그들의 영향력을 행사했거니와,
이들 전문직업인이 속해있는 이 기구는
국왕의 특허로 각지에 설립되어
폭넓은 명시적 권한과 더 폭넓은 묵시적 권리를 행사했다.
변호사, 판사, 입법자의 권한을 모두 가진 그들은 유럽 전역의 다른 동료집단들을
훨씬 능가하는 권력을 누렸다.

















