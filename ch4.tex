\chapter{자연법의 근대사}

지금까지의 논의로부터, 로마법의 변화를 가져온 저 이론은 어떤 철학적 엄밀성을
주장한 것이 아니었음을 알 수 있을 것이다.
그것은 사실 일종의 ``혼합적 사고양식''이었거니와,
이러한 사고양식은 최고의 정신을 제외한 모든 인간 정신의 유년기적 사고의
특징으로 인식되고 있으며, 또한 현 시대의 정신에서도 어렵지 않게
발견할 수 있다.
\wi{자연법} 이론은 과거와 현재를 혼동했다.
논리적으로는, 그것은 한때 자연법에 의해 통치되었던 자연상태를 상정한다.
하지만 로마의 법학자들은 그러한 자연상태의 존재를 분명하게 그리고
자신있게 말하지 않았다. 사실 황금시대를 상상하는 시적인 표현을
제외하면 고대인들은 그러한 상태에 대해 거의 언급하지 않았다.
실무적으로는, 자연법은 현재에 속하는 어떤 것이고,
기존의 제도와 얽혀있는 어떤 것이며, 유능한 관찰자에 의해
기존 제도와 구분될 수 있는 어떤 것이다.
자연의 명령을 이와 함께 섞여있는 조잡한 요소들로부터 분리하는 기준은
단순성과 조화성의 감각이었다.
하지만 이들 더 세련된 요소가 애초에 존중받은 것은
단순성과 조화성 때문이 아니라,
자연의 원초적 지배의 후예라는 데에 있었다.\footnote{%
  가령 키케로의 <<법률론>> 1.19는 자연법이
  ``어떤 도시국가도 성립되기 이전에 아주 오랜 세월 전에 먼저
  생겨났''다고 말하고 있다.
  또한 가이우스는 만민법은 인간의 자연(본성)의 산물이기에
  개별 국가의 시민법보다 더 오래되었다고 말한다.
  \latin{Dig.\,41.1.1.pr.} }
이러한 혼동은 근대의 법학자들에 의해서도 성공적으로 해명되지 못했다.
실로
로마 법률가들이 받아야 할 비난보다 오히려
오늘날의 자연법사상이 인식의 불명료성을 훨씬 더 많이 노정하고 있으며
언어의 절망적인 모호성에 의해 더 많이 오염되어 있다.
이 주제에 관한 저자들 몇몇은, 자연의 법전은 미래에 존재하는 것이고
모든 시민법들이 지향해야 할 목표라고 주장함으로써,
이러한 근본적인 난제를 피해가려고 시도하나,
이는 옛 이론이 근거하고 있던 가정의 순서를 뒤집는 것이거나,
아니면 서로 양립할 수 없는 두 이론을 뒤섞는 것에 불과할 것이다.
과거가 아니라 미래에서 완전성을 찾는 경향은 기독교에 의해
이 세상에 도입된 것이다.
사회의 진보가 더 나쁜 것에서 더 좋은 것으로 필연적으로 진행된다는 믿음은
고대 문헌에서는 거의 혹은 전혀 발견되지 않는다.

하지만 그 철학적 결함에 비해 이 이론이 인류에게 미친 영향은 훨씬 더 심대했다.
만약 자연법의 믿음이 고대세계에 보편적으로 퍼지지 않았다면
어떤 사상사적 전환이, 또 그에 따른 인류사적 전환이, 일어났을까는
실로 말하기가 쉽지 않다.

\para{초기 사회의 위험}
법, 그리고 법에 의해 결합되는 사회는 그 유년기에
두 가지 위험에 특히 취약하다.
하나는 법이 너무 빨리 발달할 수 있다는 것이다.
진보적인 그리스 공동체들에서 이런 일이 발생했거니와,
이들 공동체는 놀라운 능력으로 불편한 소송절차와 불필요한 법률용어의
질곡을 벗어던졌고, 곧이어 엄격한 규칙과 규정들에 미신적 가치를 부여하는 일을
그만두었다.
이것으로 그 공동체의 시민들이 누린 직접적 혜택은 상당히 컸지만,
인류의 궁극적 이익에는 기여하지 못했다.
민족성의 드문 자질 중 하나는,
보다 높은 이상에 법을 일치시키려는 희망을 잃지 않으면서도,
법 자체의 적용과 운용에서는
추상적 사법\hanja{司法}을 구현하는 데 지속적으로 실패하는 능력이다.
유연성과 탄력성에 뛰어난
그리스의 지식인들은
엄격한 법형식의 틀 속에 스스로를 가둘 수가 없었다.
우리가 비교적 소상히 알고 있는 아테네의 인민법원을 두고 판단하건대,
그리스의 법원은 법률문제와 사실문제를 혼동하는 경향을 강하게 나타냈다.
아리스토텔레스의 <<수사학>>\latin{Treatise on Rhetoric}에 남아있는
웅변가\latin{orator}들의 법정 표현의 흔적을 보면,
순수한 법률문제의 변론은
판사들에게 영향을 줄 수 있는 모든 것을 끊임없이 고려하면서 이루어졌다.
이런 방식으로는 지속가능한 법학체계가 만들어질 수 없다.
특정 사건의 사실관계에 대한 완벽한 이상적인 결정에 성문법 규칙이 방해되는
경우라면 언제나 그 성문법 규칙을 완화하는 데 거리낌이 없었던 공동체는,
설령 후대에 어떤 법원리들을 물려준다 하더라도
오직 당대에 지배적이었던 옳고 그름의 관념에 기초한 것들만 물려줄 수 있을 뿐이다.
이러한 법은 후대의 보다 발달된 관념에 어울릴만한 틀을 전혀 제공할 수 없다.
기껏해야 그 법을 둘러싼 문명의 볼완전성을 드러내는 철학이 될 수 있을 뿐이다.

\para{자연법}
국가 사회 중에 그들의 법이
이러한 때이른 성숙과 때아닌 해체의 위험에 의해 위협받은 곳은 많지 않다.
로마인들이 이러한 위협에 심각하게 노출된 적이 있었는지는 의문스럽다.
어쨌든 그들의 \wi{자연법} 이론에는 적절한 보호장치가 들어있었다.
분명 법학자들은 \wi{시민법}을 점진적으로 흡수하는 체계로 자연법을 관념했으며,
시민법이 폐지되지 않는 한 자연법이 시민법을 대체할 수는 없다고 생각했다.
특정 소송사건을 감독하는 판사들이 자연법의 호소에 압도당할 정도로
그렇게 자연법이 신성하다는 인상은 유포되지 않았다.
이러한 관념의 가치와 유용성은 완벽한 유형의 법이 인간 정신의 눈 앞에
펼치지지 못하게 한 것이었고, 그러한 법에 무한히 가까이 다가갈 수 있다는
희망을 품지 못하게 한 것이었으며, 또한 아직 자연법에 조응하지 못한
기존 법이 부과한 의무를 실무가나 시민들이 거부하지 못하게 한 것이었다.
무엇보다 중요한 점은 이 모범적인 체계가,
후대에 인간의 희망을 꺾어놓았던 다른 많은 체계들과 달리,
결코 상상의 산물이 아니었다는 것이다.
그것은 결코 허황된 원리에 기초한 것으로 관념되지 않았다.
그것은 기존 법의 저변에 존재하며 기존 법을 통하여 추구되어야 한다고 생각되었다.
한마디로 그것의 기능은 구제수단을 제공하는 데 있었지, 혁명적이거나
무정부적인 것이 아니었다.
불행하게도 근대 자연법 사상은
정확히 이런 점에서
고대의 그것을
닮지 않은 경우를 자주 보여준다.

\para{벤담주의}
유년기의 사회에 나타나는 또 하나의 취약성은 훨씬 더 많은 민족들의 진보를
방해하고 가로막았다.
원시법의 엄격성은 대개 일찍이 종교와의 관련성 및 동일시에 의해 등장했으며,
대부분의 민족들은 그러한 관행이 처음 체계적인 형태로 굳어질 당시에 그들이
가지고 있던 인생관과 행위관에 얽매여왔다.
놀라운 운명으로 이러한 재난을 벗어난 민족이 한 둘 있거니와,
이들 줄기에 접목하여 몇몇 근대사회가 기름진 곳이 될 수 있었다.
하지만 여전히 세계의 더 많은 곳에서는 최초 입법자가 그려놓은 기본계획을
추종하는 것이 법의 완성이라고 여겨지고 있다.
만약 그런 곳에서 지식인이 법을 운용한다면, 그들은 한결같이
고대 텍스트의 문자적 의미에서 크게 벗어나지 않고 이끌어낸 결론의
미묘한 고집스러움을 자랑스러워할 것이다.
만약 자연법 이론이 비범한 탁월성을 로마법에 주지 않았더라면
로마인들의 법이 인도인들의 법에 비해 우월하다고 할 것이
무엇이 있는지 나는 모르겠다.
이 유일한 예외적인 사례에서, 다른 여러 이유들로 인류에 막대한 영향을 끼치게 될
한 사회의 눈 앞에, 단순성과 조화성이 이상적이고 가장 완전한 법의 성질로
나타났던 것이다.
진보를 추구함에 있어 어떤 뚜렷한 목표를 가진다는 것이
한 민족이나 전문직업군에게 가지는 중요성은 아무리 강조해도 지나치지 않다.
지난 30년 동안 영국에서 벤담이 가졌던 막대한 영향력의 비밀은
이 나라 앞에 그러한 목표를 성공적으로 제시한 데 있다.
그는 우리에게 개혁의 뚜렷한 원칙을 제시했다.
지난 세기의 영국 법률가들은 아마 명민했기에
영국법이 인간 이성의 완성이라는 흔해빠진 역설적 표현에 눈멀지는 않았겠지만,
일을 추진해나갈 다른 원리가 없었기 때문에 영국법을 믿는 척 행동했다.
\wi{벤담}은 다른 모든 목표 위에 공동체의 선\hanja{善}을 두었고,
그리하여 오랫동안 밖으로 빠져나갈 길만 찾고 있던 흐름에 나갈 길을 열어주었다.

우리가 기술해온 관념을 벤담주의의 고대적 대응물이라고 부른다면
그것은 그리 훌륭한 비유가 못 될 것이다.
로마인의 이론은 저 영국인의 이론과 마찬가지 방향으로 인간의 노력을 이끌었다.
그것의 실천적 결과도 공동체의 일반적 선\hanja{善}을 꾸준히 추구하는
일군의 법개혁가들이 달성할 만한 것과 크게 다르지 않았다.
하지만 그것이 벤담의 원리를
의식적으로 예견한 것이었다고 보는 것은 잘못일 것이다.
로마인들의 대중문헌이나 법학문헌을 보면
개혁 입법의 목표로
때로 인류의 행복이 제시되곤 했음이 분명하지만,
자연법이라는 돋보이는 주장에 주어진 끊임없는 칭송에 비하면
벤담의 원리를 보여주는 증거는 거의 없거나 희미하다는 점에
유의해야 한다.
로마 법학자들이 기꺼이 수용한 것은 인간에 대한 사랑 같은 것이 아니라
단순성과 조화성---그들이 ``전아''\hanjalatin{典雅}{elegance}하다고 부르며
강조했던 것---의 감각이었다.
그들의 노력이 보다 정밀한 철학의 조언을 받아들인 자들의 노력과
우연히 일치했다는 것은 인류에게 행운이었다.

\para{프랑스 법률가들}
자연법의 근대사로 전환하면, 우리는 그것의 영향력이 막대하다는 것은
말하기 쉬워도 그 영향이 좋은 것인지 나쁜 것인지를 자신있게
말하기는 어렵다는 것을 알고 있다.
근대 \wi{자연법} 이론에서 나왔다고 할 수 있는 신조와 제도들은
우리 시대의 가장 첨예한 논쟁의 대상이거니와,
지난 백년 동안 프랑스가 서구 세계에 확산시킨 법·정치·사회에 관한
특수한 이념들 대부분이 자연법 이론에 그 원천을 두고 있다는 주장을 둘러싸고
특히 그러하다.
프랑스 역사에서 법률가들의 역할은,
그리고 프랑스 사상에서 법사상의 비중은,
언제나 대단히 컸다.
근대 유럽 법학이 발흥한 곳은 사실 프랑스가 아니라 이탈리아였지만,
이탈리아로 유학하고 돌아와 전 유럽대륙에 건설된,
그리고 \paren{허사로 돌아갔지만} 우리 영국에 건설이 시도된,
학자군 중에서 프랑스의 학자군이 그 나라의 운명에 가장 큰
영향을 끼쳤다.
프랑스의 법률가들은 즉각 카페 왕조 및 발루아 왕조의 왕들과
강력한 동맹관계를 형성했다.
프랑스의 왕권이 여러 소국과 속령들의 집합체에서 떨어져나와
마침내 그 위에 성장하게 된 것은 무력에 의한 것인 동시에
법률가들이 왕의 대권을 옹호해주고 봉건적 세습규칙을 해석해 준 데에도 기인했다.
왕과 법률가들의 동맹으로 프랑스 왕들이
강력한 봉건제후들, 귀족들, 교회와의 투쟁과정에서 누린 우위는
중세를 거슬러 올라가 당시 유럽을 지배하던 이념을 고려하지 않으면
제대로 평가할 수 없다.
우선 일반화를 향한 강한 열정이 있었고 모든 일반적 명제를 향한 찬양이 높았다.
그리하여 법의 분야에서는, 여러 지방에서 관습적으로 사용되던 고립된
다수의 법규칙들을 하나로
포괄하고 요약하는 모든 일반적 공식\latin{formula}에 대한 무의식적 존중이 있었다.
이러한 공식은 로마법대전이나 표준주석\latin{the Glosses}\footnote{%
  주석학파를 집대성한
  아쿠르시우스(Accursius)의 <<표준주석>>(glossa ordinaria)을 말하는 듯.}에
익숙한 실무가라면 물론 얼마든지 제공할 수 있었다.
하지만 법률가들의 권력을 사뭇 증대시킨 또 다른 원인도 있었다.
우리가 말하는 이 시대에는 쓰여진 법 텍스트가 가지는 권위의 정도와 성질에 관한
관념이 보편적으로 퍼져있었다.
대부분의 경우 ``이렇게 쓰여져있다''\latin{Ita scriptum est}라는 우선권을 가진
주장은 모든 항변을 침묵시키기에 충분했다.
우리 시대의 학자라면 인용된 공식을 조바심내며 조사하고,
그 출처를 따져묻고, \paren{필요하다면} 인용이 들어있는 법령집이
지방관습을 대체할 만한 권위를 가지고 있지 않다고 부인하겠지만,
그 시대의 법학자들은 규칙의 적용가능성을 의문시하거나
기껏해야 학설휘찬이나 교회법에서 반대명제를 인용하는 것 외에
다른 것은 거의 시도하지 않았다.
법논쟁에서 이러한 자못 중요한 논점에 대해 사람들이 주저하는 관념을 가졌다는 것을
염두에 두는 것은 무척 중요하거니와,
그것은 법률가들이 국왕에게 힘을 실어준 것을 설명하는 데 도움이 될 뿐만 아니라,
몇몇 흥미로운 역사적 문제를 해명하는 데도 도움을 주기 때문이다.
위조된 교령\hanja{敎令}들\latin{Forged Decretals}\footnote{%
  이른바 `이시도르 위서'\latin{(Pseudo-Isidore)}를 말한다.}을
만든 저자의 동기와 그의 특별한 성공은 이런 맥락에서
더 잘 이해될 수 있는 것이다.
보다 덜 흥미로운 예를 들자면, \wi{브랙턴}\latin{Bracton}의 표절을 이해하는 데도,
비록 부분적이지만, 도움을 준다.\footnote{%
  브랙턴은 13세기에
  <<영국의 법과 관습>>(De Legibus et Consuetudinibus Angliae)이라는
  법서를 쓴 인물로 알려져있다. }
헨리3세 시대의 저 영국 법률가는
순수한 영국법의 집성을 당대 영국인들에게 내놓을 수 있었는데,
그것은 편제의 전부와 내용의 1/3을 로마법대전에서 직접 빌려온 논저였다.
로마법의 체계적인 연구가 공식적으로는 금지된 나라에서
이러한 작업을 감행했을 것이니,
이는 법학의 역사에서 영원히 풀리지 않는
수수께끼의 하나일 것이다.\footnote{%
  <<고대법>>에 대한 폴록의 주석에 따르면 영국에서 로마법의 연구가
  체계적으로 금지된 적은 없었다고 한다. 한때 런던에 국한해서
  그러한 금지---아마도 성직자들이 교회법 연구에 매진하도록 하기 위하여---가
  있었지만, 옥스포드나 케임브리지에서는 로마법 교수가 중단없이
  계속되었다. }
그러나, 텍스트의 출처에 대한 고려는 차치하고라도,
쓰여진 텍스트가 가지는 구속력에 대한 당대의 호의적 여론을 감안하면
우리의 놀라움은 다소간 완화된다.

프랑스의 왕들이 주권 확립을 위한 긴 투쟁을 성공적으로 종결지은 때,
즉 대체로 발루아·앙굴렘 왕조의 재위기에 이르러,
프랑스 법률가들의 상황은 사뭇 특수한 것이었고 이 상태는
\wi{프랑스혁명} 발발 시까지 지속된다.
한편으로, 그들은 프랑스에서 가장 식자층에 속했고
대단히 강력한 권세를 누리는 계급을 형성했다.
그들은 봉건귀족들과 나란히 특권계급의 신분을 가졌으며,\footnote{%
  혁명 이전 프랑스에서
  전통적 봉건귀족은 `대검(帶劍)귀족'이라 불렸던 반면, 이들 신흥 법률가층은
  `법복(法服)귀족'이라 불렸다. }
프랑스 전역에 걸쳐 분포한 기구를 통해 그들의 영향력을 행사했거니와,
이 기구는 국왕의 특허로 각지에 설립되어
폭넓은 명시적 권한과 더 폭넓은 묵시적 권리를 행사했다.\footnote{%
  이 기구란 프랑스혁명 전까지 각 지역의 최고법원이자 왕령등록기구였던
  파를르망(\latin{Parlement; Parliament})을 말한다. }
변호사, 판사, 입법자의 권한을 모두 가진 그들은 유럽 전역의 다른 동료집단들을
훨씬 능가하는 권력을 누렸다.
그들의 재판 기술, 표현의 능란함, 유추와 조화에 대한 세련된 감각,
그리고 \paren{가장 뛰어난 인물들로 판단하건대}
정의관에 대한 열정적 헌신
따위는 그들 중에 특출난 재능을 보였던 인물들의 다양성만큼이나 두드러졌다.
이들 인물의 다양성은
퀴자\latin{Jacques Cujas}와 몽테스키외,
다그소\latin{Henri François D'Aguesseau}와
\wi{뒤물랭}\latin{Charles Dumoulin}처럼 서로 대척점에 위치한 이들 사이의
전 영역을 아우르는 것이었다.
하지만, 다른 한편으로, 그들이 집행해야 했던 법체계는
그들이 훈련받은 법학의 정신과는 현저히 다른 것이었다.
상당 부분 그들의 노력으로 만들어진 당대의 프랑스는
유럽의 다른 어떤 나라 이상으로 법의 변칙성과 불일치라는 저주에 휩싸여 있었다.
프랑스를 가로지르는 큰 구획선이 그 나라를
\wi{성문법지역}\latin{pays de droit écrit}과
\wi{관습법지역}\latin{pays de droit coutumier}으로 갈라놓고 있었으니,
전자는 쓰여진 로마법을 그들 법의 토대로 받아들이고 있었고,
후자는 지방적 관습과 조화되는 한도 내에서
단지 표현의 일반적 양식으로만, 그리고
법적 추론의 수단으로만, 로마법을 인정하고 있었다.
이러한 분열 아래에는 계속적으로 하위 분열이 존재했다.
관습법지역에서는 지방\latin{province}마다,
군현\latin{county}마다,
성읍\latin{municipality}마다 그 관습의 성질이 서로 달랐다.
성문법지역에서는
로마법 층위 위에 봉건규칙들의 층위가 대단히 잡다한 양상으로
펼쳐져 있었다.
이러한 혼란상은 영국에는 존재한 적이 없었다.
독일에서는 존재했지만, 그것은 이 나라의 정치적·종교적 분열상에
어울리는 것이었기에 한탄의 대상도, 심지어 감지의 대상도 되지 못했다.
국왕의 중앙권위가 부단히 강화되고 있었음에도,
행정의 통일성을 달성하기 위한 노력이 빠르게 진행되고 있었음에도,
인민들 사이에 뜨거운 민족정신이 발달하고 있었음에도,
법의 비상한 다양성이 특별한 변화없이 지속된 곳은 프랑스가 유일했다.
이러한 현저한 차이는 여러 가지 심각한 결과를 낳았는데,
그 가운데 첫째로 꼽아야 할 것은 프랑스 법률가들의 정신에 미친 영향일 것이다.
그들의 사변적 의견과 지성적 성향은 그들의 이해관계나 직업적 관행과
크게 상반되는 것이었다.
단순성과 통일성에 기초한 완전한 법을 민감하게 느끼고 완전히
수용하고 있음에도 불구하고,
그들은 프랑스법을 감싸고 있는 악덕을 근절 불가능하다고 믿었거나
혹은 그렇게 믿는 듯이 보였다.
실제로 그들은 덜 계몽된 프랑스인들 사이에서는 볼 수 없는 고집스러움으로
악습의 개혁에 저항하곤 했다.
그러나 이러한 자기모순을 조화시키는 길이 있었다.
그들은 열렬한 자연법론자들이었다.
그 \wi{자연법}은 지방 간의 경계를, 성읍 간의 경계를 뛰어넘는 것이었고
귀족과 도시민의 구별을, 도시민과 농민의 구별을
알지 못하는 것이었으며
명료성, 단순성, 체계성에 최고의 지위를 부여하는 것이었으나,
그 신봉자들에게 어떤 특정한 진보도 의무지우지 않는 것이었고
존경받고 돈벌이되는 법기술을 직접 위협하지도 않는 것이었다.
자연법은 프랑스의 보통법이 되었다고 말할 수 있을 것이다.
아니면, 여하튼 자연법의 존엄과 가치는
모든 프랑스 법실무가들이 한결같이 승인하는 유일한 신조였다.
혁명 이전의 법률가들의 언어에서 자연법의 찬미는 자못 무조건적이었다.
특히,
순수한 로마법을 폄하하는 것을 의무로 여기곤 했던
관습법 연구자들이,
\wi{학설휘찬}과 \wi{칙법휘찬}만 존중하는 당대 로마법 학자들보다
훨씬 더 자연과 자연법을 열성적으로 이야기했던 것이다.
옛 프랑스 \wi{관습법}의 최고 권위자였던 \wi{뒤물랭}은
자연법에 대해서 몇몇 과도한 진술들을 했거니와,\footnote{%
  <<고대법>>에 대한 폴록의 주석에 따르면,
  뒤물랭의 저술의 정본에서는 이러한 자연법의 찬사를 찾을 수 없었다고 한다.
  그러면서 메인이 어떤 이본(異本)을 참조한 것이 아닌가 추정하고 있다.}
그의 찬사가 담고 있는 특유의 수사학적 표현은
그것이 로마시대의 법학자들의 조심성과는 사뭇 거리가 먼 것이었음을 알려준다.
자연법의 가설은 이제 법실무를 이끄는 이론이 아니라
사변적 믿음의 규약이 되었다.
그리하여, 곧 언급하겠지만, 자연법의 최근의 변화에서는
지지자들에게 가장 약하게 존중받던 부분이 가장 강하게 존중받는 지위로
올라섰던 것이다.

\para{루소와 그의 이론}
18세기가 절반이 지났을 때 \wi{자연법}의 역사에서 가장 중대한 시기가 도래한다.
그 이론과 결과에 대한 논의가 법전문가들 사이에서만 계속되었다면,
자연법이 누리던 신망은 감소되어갈 가능성이 있었다.
바로 이때 <<법의 정신>>\latin{Esprit des Lois}이 등장했던 것이다.
아무런 심사없이 무사통과되던 가정\hanja{假定}들에 격렬하게 반발하는
특징을 조금은 과장되게 보여주는,
그러면서도 기존의 편견과 타협하려는 욕망의 흔적을
조금은 모호하게 보여주는,
\wi{몽테스키외}의 이 책은,
그 모든 결함에도 불구하고,
자연법이 한 순간도 발붙인 적이 없었던 저 \wi{역사적 방법}\latin{historical method}에
기초하여 논의를 전개한다.
이 책의 인기만큼이나 그 사상적 영향력도 마땅히 컸어야 했으나,
실은 꽃피울 시간이 허락되지 않았으니,
이 책이 파괴하고자 했던 반대가설이 갑자기 법정에서 거리로 뛰쳐나가
과거 법정이나 대학을 뒤흔들었던 때보다 훨씬 더 격렬한 논쟁의
한복판에 서게 된 것이다.
논쟁을 새롭게 촉발시킨 저 비범한 인물은
배운 것도 없고, 그리 유덕하지도 않으며, 강한 성격의 소유자도 아니었으나,
그럼에도 불구하고 생동하는 상상력과
그의 단점 대부분을 용서하고도 남을
인간에 대한 진정어린 불타는 사랑의 힘에 의해
역사에 지울 수 없는 각인을 남겼다.
1749년부터 1762년 사이에 \wi{루소}가 출간한 문헌들만큼
인간의 정신에, 지성적 사고의 모든 형태와 색조에,
막대한 영향력을 행사한 예는
우리 시대에는 전혀 볼 수 없으며, 실로 세계 전체의 역사를 통틀어 한 두 번
있을까 말까 한 것이다.
그것은 벨\latin{Pierre Bayle}과
부분적으로 영국의 \wi{로크}에 의해
시작되고 \wi{볼테르}에 의해 완성된 저 순수한 우상파괴적인 노력 이래
처음으로 인간의 믿음의 건축물을 새로 건설하려는 시도였다.
그것은, 단지 파괴만 일삼는 노력에 비해 모든 건설적인 노력이
언제나 갖는 우월성 외에도,
사변적인 문제에 관하여 과거의 모든 지식의 건전성을 의심하는
거의 보편적인 회의주의의 한복판에서 출현했다는 매우 큰 장점을 지닌다.
루소의 사상 가운데 핵심적인 상징은,
\wi{사회계약}의 체약자라는 영국적 옷을 입은 존재이든,
일체의 역사적 특성에서 벗어난 벌거벗은 존재이든,
가상의 자연상태에서 살고 있는 인간인 점에는 변함이 없다.
이러한 이상적 상황 하의 상상적 존재에 어울리지 않는
모든 법과 제도는 원초적 완전성으로부터 타락한 것으로 낙인찍힌다.
저 자연의 피조물이 지배했던 세상의 모습에 가까이 다가가는 모든 사회적 변화는
칭송받을 만하고 어떤 대가를 지불하더라도 달성할 가치가 있다.
이 이론은 여전히 로마 법률가들의 그것과 일치하거니와,
자연상태를 채우고 있는 일련의 환영\hanja{幻影}들 중에서
로마 법학자들을 매료시킨 단순성과 조화성을 제외한 일체의
속성과 특징들은 받아들이지 않기 때문이다.
하지만 이 이론은 말하자면 아래위가 뒤집힌 이론이다.
이제 `\wi{자연법}'이 아니라 `자연상태'가 숙고의 제일가는 주제이기 때문이다.
로마인들은 기존 제도들을 주의깊게 관찰하면 그 가운데 일부는,
그들이 머뭇거리며 인정했던 저 자연의 지배의 흔적을 이미 보여주고 있는 것으로,
혹은 사법\hanja{司法}적 정화를 거쳐서 보여줄 수 있는 것으로,
골라낼 수 있다고 생각했다.
\wi{루소}의 믿음은 완전한 사회질서는 오직 자연상태를 고려함으로써만
도출될 수 있다는 것이었으니, 그 사회질서는
현실세계의 상태와 전적으로 무관한 것이고 그것과 전혀 닮지 않은 것이었다.
두 견해 간의 큰 차이는, 하나는
이상적인 과거와 닮지 않았다는 이유로 현재를 몹시 그리고 노골적으로 부정하는 반면,
다른 하나는 과거 못지 않게 현재도 필요한 것으로 보고 현재를 무시하거나
비난하려 하지 않는다는 데 있다.
자연상태에 기초하여 건설된 저 정치철학, 예술철학, 교육철학,
윤리학, 사회철학을 여기서 일일이 분석할 필요는 없을 것이다.
이 철학은 지금도 여러 나라의 저급한 사상가들을 매혹하는 힘을 가지고 있으며,
또한 분명 \wi{역사적 방법}에 기초한 탐구를 방해하는
대부분의 선입견들을 낳은 다소 먼 조상이기도 하다.
하지만 오늘날 수준 높은 지성인들 사이에는 이 철학에 대한 불신이 무척 깊어,
사변적 오류의 비상한 생명력에 익숙한 이들도 놀라게 할 정도이다.
아마 오늘날 가장 빈번하게 제기되는 질문은 이 철학의 가치가 무엇이냐의 문제가
아니라, 백 년 전에 이 철학이 지배적 영향력을 가졌던 원인이
무엇이었냐의 문제일 것이다.
생각건대, 그에 대한 답은 간단하다.
고대법에만 관심을 둘 때 초래될 법한 오해를 교정하는 데에
가장 적합했을 지난 세기의 연구분야는 종교에 관한 연구였다.
그러나 그리스 종교는, 당시에 이해된 바로는, 가공의 신화에 불과한 것으로
타기시되었다.
동양의 종교는, 설령 관심을 받았을지라도, 허황된 우주생성론에
빠져있는 것으로 여겨졌다.
연구의 가치가 있는 원시적 기록은 단 하나밖에 없었다.
바로 유대인들의 초기 역사였다.
하지만 이것에 관한 연구는 당대의 편견으로 인해 저지당했다.
\wi{루소} 학파와 \wi{볼테르} 학파의 공통점이 하나 있다면 그것은
일체의 고대종교에 대한, 무엇보다 히브리 민족의 종교에 대한,
철저한 경멸이라 할 것이다.
주지하듯이 당대의 지식인들에게는, \wi{모세}에게서 유래했다는 제도들이
실은 신이 명령한 것도 아니고, 그렇다고 모세 이후에 성문화된 것도 아니며,
오히려 그 제도들과 모세오경 전체가
바빌론 유수에서 귀환한 후에 날조된 것에 불과하다고
주장하는 것이 일종의 명예로운 일이었다.
그리하여 사변적 망상을 방지하는 담보장치 하나를 이용할 수 없게 된
프랑스 철학자들은, 성직자들의 미신이라 여겨진 것에서 탈출하려는 열망에서,
법률가들의 미신에로 앞뒤 가리지 않고 뛰어들었던 것이다.

\para{프랑스의 자연법}
자연상태 가설에 기초한 철학에 대한 존중이 일반적으로 감소했다고는 하나,
보다 거칠고 보다 쉽게 닿을 수 있는 측면에 관한 한,
뒷마당에서는 여전히 그것의 설득력과 인기와 권력이 사라지지 않고 있다고
해야할 것이다.
전술했듯이 여전히 그것은 \wi{역사적 방법}의 적대자이다.
역사적 탐구 방법에 \paren{종교적 반대는 제쳐놓고} 저항하거나
이를 비난하려 하는 자들은
대체로 사회나 개인의 비역사적·자연적 상태에 대한 의식적·무의식적
믿음에 기인하는 선입견이나 악의에 찬 편견에 의해 영향받고 있다.
그러나 자연의 교리와 \wi{자연법}의 교리가 에너지를 잃지 않고 있는 것은
주로 정치적·사회적 경향과의 동맹관계 덕분이다.
저 교리들은 이러한 경향의 일부를 자극했고, 다른 일부를 실제로 만들었으며,
대다수에게 표현과 형식을 제공했다.
그것들은 프랑스로부터 다른 문명세계로 지속적으로 퍼져나가는 뚜렷한 관념이
되었고, 그리하여 문명을 변화시키는 일반적 사상체계의 일부가 되었다.
당연히 이것이 인류의 운명에 행사하는 영향력의 가치는 우리 시대에
뜨거운 논쟁의 대상이 되고 있고, 또한 본 논저가 논의대상으로 삼는
목적이기도 하다.
하지만, 자연상태 이론이 최대의 정치적 중요성을 가졌던 때를 돌이켜볼 때,
제1차 \wi{프랑스혁명}이 무수히 낳았던 엄청난 실망감의 원인을 제공하는 데
그것이 크게
기여했음을 부인할 이는 거의 없을 것이다.
그것은 당대에 거의 보편적이었던 나쁜 정신적 습관, 즉
실정법에 대한 경멸, 경험에 대한 조급증,
다른 모든 추론에 앞선 선험\latin{à priori}의 우선성 등을 낳거나
강하게 자극했다.
또한 이 철학이 생각이 짧고 관찰이 부족한 사람들의 마음을
사로잡아가는 것에 비례하여, 그것은 확실히 무정부주의적으로 되는 경향을 보였다.
뒤몽\latin{Étienne Dumont}이 \wi{벤담}을 위해 출간했으며,
특히 프랑스적인 오류를 폭로한 벤담의 문건을 담고 있는
<<무정부적 궤변>>\latin{Sophismes Anarchiques}\footnote{%
  프랑스어로 먼저 출간된 이 책은
  벤담 사후에 <<무정부적 오류>>(Anarchical Fallacies)라는 제목으로
  영어판이 나왔다. 부제는
  ``프랑스 제헌의회가 선포한 인간과 시민의 권리 선언에 대한 검토''다.}의
얼마나 많은 부분이
프랑스어로 번안된 로마인들의 가설에서 유래한 것인지, 그리고
그 가설을 참조하지 않고는 이해되지 않는 것인지를 알면 놀라지 않을 수 없다.
또한 이 점에 관하여 혁명의 절정기에 발간된 <<모니퇴르>>\latin{Moniteur}지를
찾아보면 흥미로울 것이다.
자연법과 자연상태에 대한 호소는 시대가 어두워질수록 점점 짙어졌다.
제헌의회 시절에는 비교적 드문 편이었으나,
입법의회 시절에는 훨씬 빈번했고,
음모와 전쟁에 관한 논쟁으로 시끄러웠던 국민공회 시절에는
항시적으로 나타났다.\footnote{%
  이 마지막 문장은 초판에는 있었으나 그후 저자에 의해 삭제되었다.}

\para{인간의 평등}
자연법 이론이
근대사회에 끼친 영향을 여실히 보여주는,
그리고 이러한 영향이 얼마나 소진되기 어려운가를 보여주는
한 가지 예가 있다.
생각건대,
인간의 근본적 평등의 원리가
\wi{자연법}이라는 가정\hanja{假定}에
빚지고 있음에는 의문의 여지가 없다.
바로 이 ``모든 인간은 \wi{평등}하다''야말로
시간의 흐르면서 법적 명제가 정치적 명제가 된
대표적 예인 것이다.
안토니누스 황조 시대의 로마의 법학자들은
``모든 인간은 자연적으로 평등하다''\latin{omnes homines naturâ
aequales sunt}라고 단언했지만,
그들의 눈에 이것은 어디까지나 법적 공리\hanja{公理}였다.
그들이 의도한 것은
가설적인 자연법 하에서는, 그리고 실정법이 그것에 근접하는 한에서는,
로마 \wi{시민법}이 가지고 있던 사람의 신분 간의 자의적 구별이
법적으로 사라진다는 것이었다.
이 규칙은 로마의 실무가들에게 엄청나게 중요했거니와,
로마법이 자연의 법전을 따른다고 생각될 때면 언제나
시민과 외인\hanja{外人}, 자유인과 노예,
종족\hanja{宗族}과 혈족\hanja{血族} 간의 구별이
로마의 법정에서 사라졌다.
이와 같이 자기 의견을 명확히 밝힌 로마의 법학자들은
시민법이 사변적 법유형에 비해 모자란다고 해서 사회제도를
결코 비난하지 않았고, 자연의 질서에 완전히 일치하는
인간사회가 이 세상에 존재할 수 있으리라고 믿지도 않았다.
하지만 인간의 평등 원리가 근대의 옷을 입고 등장했을 때
그 옷은 확실히 전혀 새로운 색조의 의미를 띠고 있었다.
로마 법학자들이 ``\wi{평등}하다''\latin{aequales sunt}라고 썼을 때는
그 의미가 글자 그대로였지만,
근대 로마법 학자가 ``모든 인간은 평등하다''라고 썼을 때
그 의미는 ``모든 인간은 평등해야 한다\latin{ought to be equal}''였던 것이다.
\wi{자연법}은 \wi{시민법}과 공존하는 것이고 차츰 시민법을 흡수하는 것이라는
로마 특유의 자연법 관념은 이제 확실히 망각되었거나
적어도 이해할 수 없는 것이 되었다.
기껏해야 인간 제도의 기원, 구성, 발달에 관한 이론을 말하던 단어들이
이제
인류가 겪고 있는 기존의 커다란 해악을 가리키는 표현이 되기 시작했다.
일찍이 14세기 초에, 인간의 출생시 상태에 관해 말하는 당시의 언어는,
분명 \wi{울피아누스}나 그의 동료들의 언어를 그대로 따라하려 했으나,
실은 완전히 다른 형태와 의미를 띠게 되었다.
왕실의 농노들을 해방시킨
완고왕\hanja{頑固王} 루이의 유명한 왕령의 서문은
로마인들의 귀에는 생경하게 들렸을 것이다.
``자연법에 따르면 모든 사람은 자유롭게
태어나야 한다\latin{ought to be born free}.
그런데 아주 오래 전에 우리 왕국에 도입되어
지금까지 이어지고 있는 관행과 관습으로 인하여,
그리고 어쩌면 그들 선조들이 저지른 범죄행위로 인하여,
우리 평민들 가운데 다수가 예속상태에 떨어져 있다. 그리하여 우리는 \ldots'' 등.
이것은 법규칙이 아니라 정치적 도그마의 선언이었다.
그리고 이때부터 프랑스 법률가들은 인간의 평등을
그것이 마치 그들 학문의 저장고에 보관되어온 정치적 진리인 양 말해왔다.
자연법의 가설에서 연역되어 나온 모든 다른 것과 마찬가지로,
그리고 자연법 그 자체에 대한 믿음과 마찬가지로,
그것은 맥없이 승인되었고 여론이나 실무에 거의 영향을 끼치지 못했다.
그러나 그것이
법률가들의 점유에서 벗어나 18세기 문필가들과
그들에게 감화된 대중들의 점유로 넘어가면서,
이들의 신념을 표현하는 가장 두드러진 교리가 되었고
나아가 모든 신념을 요약하는 교리로 간주되었다.
하지만 그것이 1789년 사건 이후 마침내 권력을 획득하게 된 것은
프랑스 안에서의 인기에만 기인한 것이 아니었을 것이다.
18세기 중엽에 그것은 미국으로 건너갔던 것이다.
당시 미국 법률가들, 특히 버지니아 주의 법률가들은
당대 영국인들의 것과는 사뭇 다른 지식 계통을 가지고 있었던 것으로 보인다.
대륙 유럽의 법문헌들에서 유래한 것일 수밖에 없는 것들이
다수 포함되어 있었던 것이다.
제퍼슨의 저술을 조금만 들여다보더라도
프랑스에서 유행하던 반쯤은 법적이고 반쯤은 대중적인 견해들로부터
그가 강하게 영향받고 있었음을 알 수 있을 것이다.
의심할 여지 없이, 미국에서의 일련의 사건들을 이끌었던
그와 기타 식민지 법률가들은
프랑스 법률가들의 특유한 관념에 공감했고,
``모든 인간은 \wi{평등}하게 태어난다''라는 특히 프랑스적인 가정\hanja{假定}을
영국인들에게 보다 친숙한 ``모든 인간은 자유롭게 태어난다''는 가정과
결합시켰으니,
이는 독립선언문의 첫 몇 줄에 잘 나타나있다.
독립선언문의 이 문장은 우리가 다루는 교리의 역사에서
가장 중요한 문장 중의 하나이다.
미국의 법률가들은 이렇게 인간의 근본적 평등을 무엇보다 강하게 긍인함으로써
그들 조국의 정치적 운동에 추동력을 부여했다.
영국에서는 영향력이 덜 했으나, 영국은 아직
그 힘을 다 써버린 상태가 아니다.
그런데 그밖에도 그들은 저 교리를 수용한 본국인 프랑스에 그것을 되돌려주어
훨씬 더 큰 에너지를 만들어냈고 그것의 일반적 수용과 존중이 훨씬 더 강하게
주장되도록 만들었다.
제1차 제헌의회의 보다 신중한 정치가들조차
저 \wi{울피아누스}의 명제를,
마치 그것이 인류의 직관과 본능에 동시에 기초하고 있다는 듯이,
반복하여 외쳤다.
``1789년의 원리들'' 가운데 그것은 가장 덜 공격받은 것이고,
근대의 여론에 가장 큰 영향을 끼쳤으며,
여러 사회의 헌정과 여러 국가의 정치에 가장 근본적인 변화를
약속하고 있다.

\para{국제법}
자연법의 가장 큰 기여는 근대 국제법과 근대 전쟁법을 탄생시키는 데서 수행되었다.
하지만 여기서는 이 영역에 대한 자연법의 영향을
그 중요성에 비해 훨씬 소략하게 고려하는 것으로 만족할 수밖에 없다.

\wi{국제법}의 기초를 이루는 공준\hanja{公準} 중에,
또는 국제법의 초기 건설자들에서 유래한 상징을 많이 담고 있는 공준 중에,
무척 중요한 것들이 두 세 가지 있다.
그 중 첫 번째는 결정가능한 \wi{자연법}이 존재한다는 입장이다.
\wi{그로티우스}\latin{Hugo Grotius}와 그 후계자들은 로마인들에게서 직접 이 가정을 가져왔으나,
결정의 방식에 관해서는
로마 법학자들과 차이가 크고 또한 그들 상호 간에도 차이가 크다.
문예부흥 이후 대거 등장한 공법학자\latin{publicist}들의 대다수는
자연과 자연법에 대한 정의\hanja{定義}를 다루기 쉽게 새로 제공하려는
야심을 가지고 있었다.
공법학자들의 긴 행렬이 이어지면서 저 개념에는 첨가물이 대거 덧붙여졌거니와,
이는 윤리학의 거의 모든 이론들에서 따온 관념의 조각들로 이루어진 것이었으니,
이제 윤리학이 공법의 학파들을 장악하기에 이르렀음에 틀림없다.
그렇지만 저 개념이 기본적으로 역사적 성격의 개념이란 것은,
자연상태의 필수적 성격으로부터 자연의 법전을 도출해내려는
그 모든 노력에도 불구하고, 그 결과물이란 것이
로마 법률가들의 진술을 묻지도 따지지도 말고 그대로 수용했더라면
얻을 수 있었을 결과물과 별반 다르지 않다는 점에서도 뚜렷이 드러난다.
조약에 관한 국제법을 제쳐둔다면,\footnote{%
  약정은 지켜져야 한다(pacta sunt servanda)는
  그로티우스의 계약법 이론은 전체적으로 볼 때
  로마법보다는 오히려 교회법의 영향을 더 강하게 받았다.}
국제법 체계의 얼마나 많은 부분이 순수한 로마법으로 만들어져 있는지
놀라울 지경이다.
로마 법학자들의 법리가 \wi{만민법}\latin{ius gentium}과 조화된다고 생각되면
언제나,
그것이 아무리 순수히 로마적 기원을 가진 것이라 할지라도,
공법학자들은 그것을 빌려올 구실을 발견했던 것이다.
또한 우리는
이렇게 파생된 이론들이
원래의 관념이 가지고 있던 약점을
그대로 안고 있다는 점도 관찰할 수 있다.
대부분의 공법학자들의 사고양식은 여전히 ``혼합적''인 것이었다.
이들의 저술을 연구함에 있어서 항상 부딪치는 큰 어려움은
그들이 논하는 것이 법인지 아니면 도덕인지,
그들이 기술하는 국제관계의 상태가 현실의 것인지 아니면 이상적인 것인지,
그들의 진술이 존재에 관한 것인지 아니면
그들이 생각하는 바람직한 당위에 관한 것이지를 판가름하는 일이다.

\wi{자연법}이 국가들 사이에서\latin{inter se} 구속력을 가진다는 가정이
\wi{국제법}의 근저에 놓여있는 두 번째 공준이다.
이 원리에 대한 일련의 주장과 수용은 근대 법학의 유년기로 거슬러올라가며
추적할 수 있을 것인데,
일견 그것은 로마인들의 가르침에서 직접 추론한 것으로 보인다.
사회의 국가적 상태와 자연적 상태의 차이는
전자에는 입법자가 뚜렷이 존재하지만 후자에는 없다는 것이므로,
만약 다수의 \hemph{단위들}\latin{units}이
어떤 공통의 주권자나 정치적 상급자에게도
복종하지 않는다고 인정되면 그들은 자연법의 지배 상태로 되돌아간다고 볼 수 있다.
국가들이 바로 그러한 단위들이다.
국가의 독립성 가설은 공통의 입법자 관념을 배제하거니와,
몇몇 이론에 의하면,
따라서 국가들은 자연의 원시적 질서에 복종한다는 관념이 도출되는 것이다.
그에 대한 대안은 독립된 공동체들 간에는 어떠한 법도 존재하지 않는다는 것이나,
이러한 무법\hanja{無法} 상태야말로
로마 법학자들의 성정이 끔찍히도 싫어했던 진공상태인 것이다.
물론, 로마 법률가들은 시민법이 추방당한 어떤 영역에 맞닥뜨리면
즉시 그 빈 공간을 자연의 명령으로 채워넣었을 것이라고 추정할 만한
외견상의 이유는 존재한다.
하지만 어떤 결론이 우리의 눈에 아무리 확실하고 자명해 보일지라도,
역사의 어느 순간에
실제로 그러한 결론이 도출되었을 것이라고 가정하는 것은 위험한 일이다.
현존하는 로마법 텍스트 가운데,
로마 법학자들이 자연법을 독립된 국가들 간에 구속력을 갖는 것으로
믿었다는 증거는, 내가 알기로는, 전혀 발견된 바 없다.
로마 제국의 시민들은
자기들 국가의 통치영역이 문명의 영역과 경계를 같이 한다고 생각했기에,
국가들이 모두 동등하게 자연법에 복종한다는 것은,
설령 그런 생각을 해봤다 할지라도,
기껏해야 유별난 사변의 극단적 결과 쯤으로 치부했을 것이 분명하다.
사실 근대 국제법은,
로마법의 후손임에는 틀림없으나,
비정상적인 계통을 거쳐 로마법에 연결될 뿐이라고 해야할 것이다.
로마 법학의 근대 초기 해석자들은
\wi{만민법}\latin{ius gentium}의 뜻을 잘못 이해하여,
로마인들이 국제 거래를 규율하는 법체계를 그들에게 물려주었다고
서슴없이 믿었다.
이 ``만민법''\latin{law of nations}은 처음에는 강력한 경쟁상대들과
권위를 두고 싸워야 했고,
유럽의 상황은 오랫동안 그것의 보편적 수용을 방해했다.
그러나 차츰 서구 세계는 로마법 학자\latin{civilian}들의 저 이론에
보다 우호적인 형태로 재편되어갔고,
상황의 변화와 더불어 경쟁적 이론들의 신망은 땅에 떨어졌다.
마침내, 특별한 행운이 겹쳐,
아얄라\latin{Balthazar Ayala}와 \wi{그로티우스}는 그것에 대한 유럽의 열광적인 동의를
얻어낼 수 있었고, 다양한 유형의 엄숙한 계약이 체결될 때마다 이 동의는
계속해서 갱신되어갔다.
승리의 주역이라 할 수 있는 저 위인들이
그것을 완전히 새로운 기초 위에 놓으려 시도했음은 말할 필요도 없거니와,
이러한 재배치 과정에서 그 구조를
많이 바꾸었음---그러나 일반적으로 알려진 것보다는 훨씬 덜 바꾸었다---도
의문의 여지가 없다.
안토니누스 황조 시대 법학자들이 만민법과 자연법이 동일하다고 보았던 점을
수용한
그로티우스는 그의 직접적 선학들 및 후학들과 더불어
자연법에 특별한 권위를 부여하였으니,
그 권위는 만약 ``만민법''이 당시 모호한 의미를 갖지 않았다면
아마 결코 주장될 수 없었을 것이다.
그들은 자연법이 국가들의 법전임을 스스럼없이 주장했다.
그리하여
오직 자연 개념에 대한 숙고로부터 도출된다고 여겨진 규칙들을
국제법 체계에 접목시키는 과정이 시작되었고,
이 과정은 거의 우리 시대까지 지속되고 있다.
또한 이는 인류에게 대단히 중요한 현실적 결과 하나를 낳았으니,
그것은 근대 초기 유럽의 역사에 전혀 알려지지 않은 것은 아니나
그로티우스 학파의 법리가 지배적 위치를 차지하면서 비로소
명백하게 그리고 보편적으로 인식된 것이다.
만약 국가들의 사회가 \wi{자연법}의 지배를 받는다면,
그 사회를 구성하는 원자들은 절대적으로 평등해야 한다.
자연의 홀\hanja{笏} 아래서 모든 인간이 평등하듯이,
국가들 간의 상태가 일종의 자연상태라면 국가들도 평등하다.
크기와 힘이 서로 다르더라도 독립된 공동체들은
국제법의 관점에서 모두 \wi{평등}하다는 이 명제는,
각 시대의 정치적 상황에 의해 위협받아온 것도 사실이지만,
대체로 인류의 행복에 기여해왔다.
문예부흥 이후 공법학자들이
자연의 존엄하신 주장으로부터 국제법을 도출하지 않았더라면,
저 법리는 결코 굳건한 반석 위에 설 수 없었을 것이다.

전체적으로 볼 때, 전술했듯이,
단순히 로마 만민법이라는 고대 지층에서 가져온 요소들에 비해
그로티우스 시대 이래 \wi{국제법}에 새로 추가된 것이
얼마나 작은 비율인지 놀라울 정도이다.
영토의 취득은 언제나 국가의 야심을 자극해왔거니와,
이러한 취득을 규율하는 규칙들은,
그 야심이 너무나 자주 불러오는 전쟁을 억제하는 규칙들과 더불어,
\wi{만민법}상의\latin{jure gentium} 물건의 취득 방식에 관한 로마법을
단순히 옮겨적은 것에 지나지 않는다.
앞서 설명했듯이,
옛날 법학자들은
로마 인근의 여러 부족들을 관찰하여 그들 사이에 지배적인 관행에서
공통의 요소를 추출함으로써
이러한 취득 방식들을
획득했다.
그 기원에 따라
``모든 민족들에 공통적인 법''으로 분류된 이 방식들을
후대의 법률가들은
그 단순성에 착목하여 \wi{자연법}이라는 보다 최근의 개념과 어울린다고 생각했다.
그리하여 그들은 근대 \wi{만민법}\latin{law of nations}으로 이어지는
길을 열었으니, 결과적으로
\hemph{영토}\latin{dominion}와 그것의 성격, 한계, 취득방식 및
안전하게 지키는 방식에 관한 국제법 분야는
순수한 로마 물권법---즉,
안토니누스 황조 시대 법학자들이 자연상태와의 모종의 일치를 보인다고
생각했던 바로 그 로마 물권법---인 것이다.
국제법의 이 분야가 적용될 수 있으려면,
주권자들 사이의 관계가
로마의 소유권자 집단의 성원들처럼 될 필요가
있었다.\footnote{%
  로마에서는 원칙적으로 가부장(pater familias)들만이 소유권자가 될 수 있었다. }
이것이 국제법 법전의 초입에 놓여있는 또 하나의 공준인 것이다.
또한 이것은 근대 유럽 역사의 첫 몇 세기 동안은 지지받지 못한 공준이었다.
이것은 두 개의 명제로 분해될 수 있거니와,
``주권은 영토적이다,'' 즉
지구 표면의 한정된 부분에 대한 소유권을 갖는다는 명제와,
``주권자들 사이에서는 주권자가 당해 국가 영토의,
\hemph{최고}\latin{paramount} 소유자가 아니라,\footnote{%
  봉건제 하의 중층소유권 이론을 부정한다는 의미일 것이다. }
\hemph{절대적}\latin{absolute} 소유자로 간주된다''는 명제가
그것이다.

오늘날 국제법 학자들은
형평과 상식에 기초한
그들의 국제법 법리들이
근대 문명의 모든 단계에 쉽게 적용될 수 있다고
암묵적으로 전제한다.
이 전제는, 국제법 이론의 몇몇 현실적 결함을 감추고 있기는 하지만,
근대 역사의 대부분의 시기에 대해 결코 주장될 수 없는 전제이다.
국가들의 문제에 관하여 만민법\latin{ius gentium}의 권위가
전혀 도전받지 않았다는 것은 사실이 아니다.
오히려 그것은 오랫동안 몇몇 경쟁적인 이론들과 투쟁해야 했다.
또한 주권의 영토적 성격이 항상 인정되어왔다는 것도 사실이 아니다.
로마의 영토가 해체된 이후 오랫동안 인간의 정신은 그러한 이론과
조화될 수 없는 관념에 의해 지배되었던 것이다.
사물의 옛 질서와 그에 기초한 견해가 쇠퇴하고 나서야,
새로운 유럽과 그에 부합하는 새로운 관념이 등장하고 나서야,
국제법의 저 두 가지 공준은 보편적으로 받아들여질 수 있었다.

\para{영토주권}
근대사라고 불리는 것의 대부분의 기간 동안
``\wi{영토주권}''\latin{territorial sovereignty}이라는 관념이
존재하지 않았다는 점을 명심할 필요가 있다.
주권은 지구의 일부분 또는 더 세분된 영역에 대한 영유권과
아무런 관련이 없었다.
이 세계는 너무나 오랜 기간 동안 로마 제국의 그림자 아래 존재해왔기 때문에,
제국의 영토로 편입된 광대한 지역이 한때는
외부의 간섭으로부터 면제되고 국가 간에 평등한 권리를 요구하는
다수의 독립국가들로 나뉘어져 있었다는 사실을 망각해버렸다.
만족\hanja{蠻族}들의 침입이 진정된 이후,
주권의 개념에는 다음과 같이 양면성이 있었던 것으로 보인다.
우선 그것은 ``\wi{부족주권}''\latin{tribe-sovereignty}이라고 부를 수 있는
형태를 띠고 있었다.
물론
프랑크족, 부르군드족, 반달족, 롬바르드족, 서고트족은
그들이 차지한 영토의 주인이었거니와,
이는 몇몇 지역의 지리적 명칭으로도 남아있다.
하지만 그들은 영토적 점유에 기초한 어떤 권리도 주장하지 않았으니,
사실 영토적 점유를 중요하게 여기지도 않았다.
그들은 삼림과 초원에서 가져온 전통을 계속 유지했던 것으로 보이며,
여전히 가부장적 사회의 유랑 무리로서
단지 생계 수단을 제공하는 토지 위에 잠시
캠프를 치고 있을 뿐이라는 견해를 가지고 있었던 듯하다.
알프스 너머 갈리아 지방의 일부와 게르마니아 지방의 일부는
이제 프랑크족이 사실상 지배하는 나라---오늘날의 프랑스---가 되었다.
하지만 클로비스의 후손인 메로빙거 왕조의 군장\hanja{君長}들은
프랑스의 왕이 아니었다. 그들은 프랑크족의 왕이었던 것이다.
영토적 권리를 뜻하는 용어가 알려져 있지 않았던 것은 아니나,
처음에는 단지 부족이 점유한 땅의 \hemph{일부}를
통치하는 통치자를 지칭하는 편리한 수단의 하나로만 사용되었던 듯하다.
부족 \hemph{전체}의 왕은 그의 백성들의 왕이었지,
그의 백성들이 차지하고 있는 여러 토지의 왕은 아니었다.
이러한 특수한 주권 관념에 대한
대안으로---이 논점은 매우 중요한데---보편적 지배의 관념이
존재했던 것으로 보인다.
군주가 부족원들의 군장이라는 관계를 청산하고
자기자신을 위해 새로운 주권 형태를 만들어내고자 원했을 때,
받아들일 만한 선례로서 그에게 주어진 것은 로마 황제들의 지배형태였다.
흔히 쓰이는 인용구를 차용하자면, 그는
``황제가 아니면 아무 것도 아닌''\latin{aut Caesar aut nullus} 것이
된 것이다.\footnote{`전부 아니면 전무'(all or nothing)의 뜻으로 종종 쓰인다.}
비잔틴 황제의 완전한 대권\hanja{大權}을 주장하거나, 아니면 아무런 정치적 지위를
갖지 않는다는 것이다.
우리 시대에는 새로운 왕조가 폐위된 왕조의 기존 권리를 지워버리고자 할 때,
\hemph{영토}가 아닌 \hemph{인민}을 지칭하는 용어를 사용한다.
그리하여 오늘날에는 프랑스인의 황제나 왕이 존재하고,
벨기에인의 왕이 존재한다.
그러나 우리가 다루는 저 시대에는 다른 대안이 사용되었다.
더 이상 부족의 왕으로 불리고 싶지 않은 군장은 세계의 황제를 자처해야 했다.
따라서, 세습 궁재\hanja{宮宰}들은 그들이 이미 오래 전부터 사실상
무력화시켰던 국왕들과 더 이상 타협하고 싶지 않았을 때,
스스로를 단순히 프랑크족의 왕이라고 부르길 원하지 않았다.
이 호칭은 폐위된 메로빙거 왕조에 속하던 것이기 때문이다.
그렇다고 프랑스의 왕이라는 호칭도 쓸 수 없었다.
이 호칭은,
비록 알려져 있지 않은 것은 아니었으나, 존엄성을 갖지 못하던 것이기 때문이다.
그리하여 그들은 보편 제국을 지향하는 호칭을 사용했다.
그들의 동기는 크게 오해의 대상이 되었다.
최근의 프랑스 학자들은 샤를마뉴를 시대를 앞서간 인물로 그려내는 것을
당연시하거니와,
계획을 추진하는 에너지에 있어서는 물론이고
그의 계획의 성격 또한 그러하다는 것이다.
어떤 사람이 그의 시대를 앞서갈 수 있는지의 여부는 차치하고라도,
한 가지 분명한 것은, 무한한 영토를 추구했던 샤를마뉴는
그 시대의 특유한 관념이 그에게 허락한 유일한 길을 따랐을 뿐이라는 점이다.
지성을 중시하는 그의 탁월한 능력에는 이론\hanja{異論}이 없지만,
이는 그의 행위 때문에 그러한 것이지, 그의 이론\hanja{理論} 때문에 그러한 것이 아니다.

이러한 독특한 견해는 샤를마뉴의 세 명의 손자들 사이에서
상속재산이 분할되었을 때에도 그대로 유지되었다.
대머리 샤를, 루이, 그리고 로타르는 이론적으로는 여전히---이 용어를
사용하는 것이 적절하다면---로마 제국의 황제들이었다.
동로마황제와 서로마황제가 각각 법적으로는 세계 전체의 황제이지만
사실은 그 절반씩을 통치했던 것처럼,
저 세 명의 카롤링거 황제들도 권력은 제한되어 있지만
법적 타이틀은 무제한적이라 여겼던 것으로 보인다.
비만왕\hanja{肥滿王} 샤를의 죽음으로 또다시 분할이 이루어진 이후에도%
\footnote{이 문단의 처음부터 여기까지는 초판에 있었으나 그후 저자에 의해 삭제되었다.}
이러한 주권의 보편성 관념은
오랫동안 황제의 지위와 관련되어 있었고,
실로 신성로마제국이 존속하는 한 그것과 완전히 단절될 수 없었다.
\wi{영토주권}---주권을 지구 표면의 한정된 부분의 점유와 관련짓는 견해---은
명백히 \hemph{\wi{봉건제}도}\latin{feudalism}의 자손, 그것도 뒤늦은 자손이었다.
이는 연역적으로도 예상할 수 있거니와,
봉건제도는 역사상 최초로 인적\hanja{人的} 의무를, 따라서
인적 권리를, 토지 소유와 연결지었던 것이다.
그것의 기원과 법적 성격에 관한 적절한 견해가 무엇이건 간에,
봉건 구조를 생생하게 묘사하는 가장 좋은 방법은
그 밑바닥부터 시작하는 것이다.
우선 봉신\hanja{封臣}의 봉사\hanja{奉仕}의무를 설정하고 제한하는 한 조각 토지에 대한
봉신의 관계를 고려하고,
그 다음 차츰 상위의 \wi{수봉}\hanja{授封}관계로 올라가면서 원의 반경을 좁혀나가,
마침내 체제의 정점에 이르는 방법인 것이다.
암흑시대 후기 동안 그 정점이 정확히 어디에 위치했는지는 확인하기가 쉽지 않다.
아마도, \wi{부족주권}의 관념이 실제 쇠퇴한 곳이라면 어디서나,
그 정점은 서구 세계의 황제로 여겨지던 자들에게 언제나 주어졌을 것이다.
그러나 머지않아 제국의 권위가 먹혀드는 영역이 대폭 축소되자,
그리고 황제들이 얼마 남지 않은 그들의 권력을
독일 지역과 북이탈리아 지역에 집중시키자,
과거 카롤링거 제국의 나머지 모든 지역에서 최고 봉건 수장들은 사실상
상급자가 없는 상태가 되었다.
차츰 그들은 새로운 상황에 적응해갔고,
불입\hanjalatin{不入}{immunity}의 사실상태는
마침내 종속\hanja{從屬}의 법이론을 덮어서 가려버렸다.
그러나 이러한 변화가 쉽게 일어나기 어려웠을 것을 알려주는 여러 징후가 존재한다.
사실, 사물의 본성상 어딘가에 최고 권력이 반드시 존재해야 한다는 관념 탓에,
세속적 최고성을 로마 교황청에 부여하는 경향이 점점 커지고 있었던 것이다.
관념 혁명의 최초의 단계는
프랑스의 카페 왕조에 의해 완성된다.
그 전까지는
이제 카롤링거 제국에서 갈라져나온 몇몇 대\hanja{大}영지의 보유자들이
스스로를 공작이나 백작이 아닌 왕으로 자처하기 시작하고 있었다.
그런데 파리와 그 인근에 한정된 영토를 가진 저 봉건군주가
옛 왕가로부터 \hemph{프랑스인의 왕}이라는 타이틀을 찬탈하면서
중요한 변화가 일어나기 시작했다.
위그 카페와 그 후계자들은 전혀 새로운 의미의 왕들이었으니,
영주의 그의 영지에 대한 관계, 봉신의 그의 \wi{자유보유지}\latin{freehold}\footnote{%
  영국 보통법의 개념으로 `자유보유지'는
  단순토지보유권(estate in fee simple),
  한정승계토지보유권(estate in fee tail),
  생애토지보유권(estate for life),
  그리고 과부산(寡婦産 dower)과 홀아비산(鰥夫産 curtesy)을 통칭한다.
  예속적 토지보유 및 부동산 임대차에 대비되는 말로서,
  일단 우리의 토지소유권에 대응한다고 보아도 무방하다.
}에 대한 관계와
동일한 관계에서 프랑스 토지에 대한 주권자였던 것이다.
비록 오랫동안 저 옛 부족적 호칭이 통치왕가의 공식 라틴어 호칭으로 남아있었으나,
고유어 호칭에서는 빠르게 \hemph{프랑스의 왕}으로 변모되어갔다.
프랑스에서의 국왕 지위의 형식은 다른 곳에서 동일한 방향으로 일어나고 있던
변화를 뚜렷이 촉진시키는 결과를 가져왔다.
앵글로색슨 왕가들의 왕은 부족적 군장과 영토적 주권자 사이의
중간지대에 머물렀으나,
노르만 왕조의 군주들의 권력은 프랑스 왕의 그것을 본받아 명백히
영토적 주권자의 모습을 띠었다.
이후 건설되거나 공고화된 모든 영토는 이러한 후대의 모델에 입각하여 형성되었다.
스페인, 나폴리, 그리고 이탈리아 자유도시들의 폐허 위에 건설된
공국\hanjalatin{公國}{principality}들은 모두 영토적 주권을 가진 통치자들의 지배에 놓였다.
부연컨대, 베네치아가 이 견해에서 저 견해로 옮겨가면서 점진적으로
타락해간 것만큼 이상한 일도 별로 없을 것이다.
해외정복을 시작할 때의 베네치아 공화국은
다수의 피지배 속주들을 통치하는
로마 국가 체제의 예시의 하나로
스스로를
간주했었다.
그로부터 한 세기가 지난 후의 베네치아는
이탈리아와 에게해의 점유지들에 대해
봉건영주의 권리를 주장하는 주권체로 보여지기를
바라게 되었던 것이다.

\para{국제법}
주권이라는 주제에 관한 대중의 관념이 이러한 근본적인 변화를 겪고 있던 동안,
우리가 오늘날 \wi{국제법}이라고 부르는 것을 대신하던 체계는
오늘날의 그것과 형식에 있어서도 달랐고 원리에 있어서도 불일치했다.
유럽 중에서도 신성로마제국에 속하는 드넓은 지역에서
국가들 간의 연합관계는 제국칙령\latin{imperial constitution}이라는 복잡하고 불완전한 메커니즘에 의해
규율되었다.
우리에게는 놀라울지 몰라도, 당시 독일지역 법률가들 사이에서는
국가들 간의 관계가 제국 내부에서든 바깥에서든,
만민법\latin{ius gentium}에 의해서가 아니라,
황제를 중심으로 하는 순수한 로마법에 의해
규율되어야 한다는 생각이 널리 선호되었다.
이 법리는, 우리의 예상과는 달리,
제국 바깥의 나라들에서도
그다지 확신에 찬 거부의 대상이 되지 못했다.
그러나 실제로는, 유럽의 나머지 지역에서는
\wi{봉건제}의 지배복종관계가 공법의 대체물을 제공하고 있었다.
그리고 봉건제가 쇠퇴하거나 모호해지자 그 배후에서,
적어도 이론적으로는 최고 규율권력이 교회 수장의 권위에 속한다는 이론이
모습을 드러냈다.
하지만 봉건권력도 교회권력도 15세기에 이르면, 아니 이미 14세기부터,
빠르게 쇠퇴하고 있었음이 확실하다.
또한
당시에 전쟁의 구실이나 동맹의 동기로 공언된 것들을 살펴보면,
조금씩 옛 원리들이 추방되고 있었고,
그 대신 나중에 아얄라와 그로티우스에 의해 조화되고 공고화될 견해들이
비록 조용하고 느리지만
괄목할만한 진전을 이루고 있었음을 알 수 있다.
저 모든 권위가 융합되었다면 모종의 국제관계 체계가 진화되어 나올 수 있었을까,
그리고 그 체계는 그로티우스의 체계와 중대한 차이를 갖는 것이었을까, 따위는
오늘날 우리로서는 알 수 없거니와, 실로 종교개혁이 하나를 제외하고는
모든 잠재적 가능성을 파괴해버렸기 때문이다.
독일지역에서 시작된 종교개혁으로 제국의 제후들은 건널 수 없는 깊은 골을
사이에 두고 분열되었고, 최고 권력의 황제라도
이 골을 메울 수 없었다.
비록 황제가 중립적이었다 할지라도 그러했을진대,
하물며 황제는 종교개혁에 반대하는 교회의 입장에 동조해야 했다.
교황도 동일한 곤경에 처해 있었음은 말할 나위도 없다.
그리하여 분쟁당사자들 사이에서 중재 역할을 담당해야 할 저 두 권위는
그들 스스로가 국가들 간의 분열에서 한쪽 당파를 대표하는 수장들이 되어버렸다.
이미 허약해진 \wi{봉건제}도는 공법관계의 원리로서의 신망을 잃어버려,
종교적 당파성에 대항할만한 어떠한 안정적 결합도 제공할 수 없었다.
거의 카오스에 가까운 이러한 공법 상황 하에서,
로마 법학자들이 지지했을 법한 그러한 국가체제에 관한 견해만이
유일하게 남게 된 것이다.
\wi{그로티우스}가 보여준 저 견해의 외양과 조화성과 탁월성은
사실 당대 지식인이라면 누구나 알고 있었다.
그러나 <<전쟁과 평화의 법>>\latin{De Jure Belli et Pacis}의 경이로움은
그것이 신속하고도 완전하게 그리고 보편적으로 성공을 거두었다는 데 있다.
30년전쟁이 가져온 전율, 고삐풀린 군인들의 방종이 불러온 무한한 공포와 연민,
이런 것들도 분명 그것의 성공을 어느 정도 설명할 수 있겠지만,
이것만으로는 충분한 설명이 되지 못한다.
만약 그로티우스의 저 위대한 저서에서 스케치된 국제관계의 건축물 설계도가
이론적으로 완전한 모습을 띠지 않았다면,
저 저서는 법률가들에 의해 버림받고
정치인들과 군인들에 의해 무시당했을 것이라는 점은
당대의 관념을 깊이 천착해들어가지 않더라도 쉽게 알 수 있다.

\para{그로티우스의 체계}
말할 것도 없이,
\wi{그로티우스}의 체계가 갖는 사변적 완전성은 우리가 논의해온
\wi{영토주권}의 관념과 밀접하게 관련되어 있다.
\wi{국제법} 이론은 국가들이 상호 간의 관계에서 자연상태에 있다고 가정한다.
그러나,
그 근본전제에 의하면,
자연적 사회를 구성하는 원자들은
상호 간에 고립되어 있어야 하고
독립되어 있어야 한다.
만약
약하게라도 그리고 가끔씩이라도
그들을 연결시켜주는 상위 권력이 존재하여
공통의 주권자임을 주장한다면,
바로 그 공통의 주권자 개념으로부터 실정법 관념이 도입될 것이고
자연법 관념은 배제될 것이다.
따라서 제국 수장의 보편적 종주권\hanja{宗主權}이
순 이론적으로라도 받아들여졌다면, 그로티우스의 노력은 헛수고가 되었을 것이다.
근대 공법이론과 지금까지 그 발달과정을 서술해온 주권개념 간의 접점이
이것만 있는 것은 아니다.
전술한 대로, 국제법의 분야 중에는
그 전부를 로마 물권법에서 가져온 분야들이 있다.
이로써 무엇을 추론할 수 있는가?
주권의 관념에 내가 서술했던 변화가 일어나지 않았다면---주권이
지구의 한정된 부분에 대한 소유권과 관련하여 관념되지 않았다면,
다시 말해, 영토주권이 되지 않았다면---그로티우스 이론의 세 부분은
적용 불가능한 것이 되었으리라는 것이다.\footnote{그로티우스의
<<전쟁과 평화의 법>>은 모두 세 권으로 구성된다.}

