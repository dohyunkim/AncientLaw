\chapter{물권법의 초기 역사}

로마의 법학제요 저서들\footnote{가이우스의 법학제요와
유스티니아누스의 법학제요를 일컫는다.}은
소유권의 여러 형태들과 변종들을 정의한 후,
자연법상의 물건취득 방식들에 대하여 논한다.
법제사를 잘 모르는 이들은
취득의 이러한 ``자연법상의 방식들''이
일견
사변적으로나 실무적으로나 큰 관심의 대상이 아닐 것이라고
생각하기 쉽다.
야생동물을 덫으로 잡거나 사냥해서 죽이는 것,
토양이 강물에 의해 충적되어 부지불식간에
내 땅에 부합\hanja{附合}하는 것,
나무가 내 땅에 뿌리를 내리는 것 따위를
로마 법률가들은 모두 \hemph{자연적으로} 취득한다고 말했다.
옛 법학자들은
그들 주위의 여러 작은 사회의 관행에서
이들 취득이
보편적으로 인정되는 것을 분명 관찰했을 것이다.
후대의 법률가들은
이들이 옛 만민법\latin{jus gentium}에 분류되어 있고
단순명쾌하게 기술\hanja{記述}되어 있는 것을 보았고, 그리하여
이들에게
자연법의 자리를
내주었을 것이다.
이들에게 부여된 존엄성은 근대에 이르러 점점 커져,
이제는 원래 가졌던 중요성을 훨씬 능가하는 것이 되었다.
자연법 이론은 이들을 가장 즐기는 음식으로 삼았고,
실무에 사뭇 심각한 영향력을 행사할 수 있도록 만들었다.

\para{선점}
이러한 ``자연법적 취득방식들'' 가운데
한 가지만은 반드시 짚고 넘어갈 필요가 있거니와,
선점\hanjalatin{先占}{occupatio}이 그것이다.
선점은
취득 당시 누구의 물건도 아닌 것을
\paren{법기술적 정의\hanja{定義}가 이어진다}
당신의 물건으로 삼고자 하는 의사로써
점유하는 것을 말한다.
로마 법률가들이 무주물\hanjalatin{無主物}{res nullius}---소유주가
없거나 있어본 적이 없는 물건---이라 불렀던
것이 무엇인지는 열거함으로써만 알 수 있을 뿐이다.
소유주가 \hemph{있어본 적이 없는} 물건에는
야생 동물, 물고기, 야생 조류\hanja{鳥類}, 최초로 캐낸 보석,
새로 발견했거나 경작된 적 없는 토지 따위가 속한다.
소유주가 \hemph{없는} 물건에는
포기된 동산, 버려진 토지,
\paren{특이한 그러나 가공스러운 항목인데}
적\hanja{敵}이 소유한 물건 따위가 속한다.
이 모든 것들은
자기 것으로 삼으려는 의사---일정한 경우 이 의사는
특정한 행위에 의해 명시적으로 드러나야 한다---를 가지고서 처음 점유한
\hemph{선점자}가 완전한 소유권\latin{dominion}을 취득한다.
생각건대,
선점 관행의 보편성으로 인해
한 세대의 로마 법률가들이 그것을 모든 민족에 공통인 법으로
자리매김한 것,
그리고 그 단순성으로 인해
다음 세대의 로마 법률가들이 그것을 자연법에 귀속시킨 것은
그리 어렵지 않게 이해할 수 있다.
그러나 근대 법사\hanja{法史}에서 그것이 누린 행운은
선험적인 고찰로는 얼른 이해되지가 않는다.
로마법의 선점 원리, 그리고 이를 둘러싸고 로마 법학자들이 전개한 법규칙들은
근대 국제법 중에서도
전쟁시 포획에 관한 법과
새로 발견한 땅에 대한 주권 획득에 관한 법의
원천이 되었다.
또한 소유권의 기원에 관한 어떤 이론의 근거가 되었거니와,
이 이론은 대중적으로 인기있는 이론인 동시에,
다수의 위대한 사변적 법학자들이
이러저러한 형태로
널리 수긍하고 있는 이론이다.

\para{적의 소유물, 발견의 법리}
방금 나는 로마법의 선점 원리가
전쟁시 포획에 관한 국제법의 흐름을 결정했다고 말했다.
전쟁시 포획법의 법규칙들은,
적대관계의 발발에 의해 국가들은 일종의 자연상태로 환원되고
이렇게 만들어진 의제적\latin{artificial} 자연상태 하에서
교전국 간에는
사적 소유권 제도가
중지된다는
가정\hanja{假定}에 기초한다.
후기의 국제법 학자들은
그들이 설명하는 법체계에서도
사적 소유권이 어떤 의미에서는 인정된다는 주장을
유지하려고 했기 때문에,
적의 재산이 무주물이라는 가설은 그들에게
정도를 벗어난 충격적인 것으로 여겨졌고,
따라서 그들은 이 가설을 단지 법적인 의제\latin{fiction}에 불과하다고
내세우는 신중함을 보였다.
그러나 자연법이 만민법에 그 기원을 두고 있음을 잘 아는 우리는
어떻게 적의 재산이 무주물로 취급되고 그리하여
최초의 점유자에 의해 취득될 수 있었는지 금방 이해할 수 있다.
고대적 형태의 전쟁을 수행하는 사람들은
승전으로 정복군의 군대가 해산되고
해산된 군인들이 무차별적인 약탈을 자행했을 때
저 관념을 자동적으로 떠올렸을 것이다.
하지만
이때 포획자가 취득하도록 허용된 재산은
원래는 동산에 국한하였을 것으로 보인다.
우리는
고대 이탈리아에서
피정복 국가의 토지에 대한 소유권의 취득에 관해서는
전혀 다른 규칙이 지배했음을 별도의 전거를 통해 알고 있다.
따라서 토지에 대해 선점의 원리가 적용되기 시작한 것은
\paren{항상 어려운 문제이지만}
만민법이 자연법으로 전환되는 시기였을 것으로,
그리하여 황금시대의 법학자들이 행한 일반화의 결과였을 것으로,
짐작할 수 있다.
이에 관한 법리는 유스티니아누스의 학설휘찬에 보존되어 있거니와,
그것은 모든 종류의 적의 재산은 교전 상대방에게 무주물이라는,
그리고 포획자가 그것을 자기 것으로 만드는 선점은 자연법상의 제도라는,
무제한적 주장으로 나아간다.
이러한 명제로부터 국제법이 이끌어낸 규칙들은
때로 군인들의 만행과 탐욕을 필요 이상으로 부추긴다고
비판받았지만,
생각건대 이 비판은
전쟁의 역사를 잘 모르는 사람들에 의해,
그리하여 어떤 종류의 규칙이든 규칙에 대한 복종을 명하는 것이
얼마나 위대한 업적인지 잘 모르는 사람들에 의해 가해진 비판이다.
선점에 관한 로마법 원리가 전쟁시 포획에 관한 근대법에 수용되어 들어왔을 때,
그 남용을 제한하고 정밀함을 부여하는
많은 부수적인 법규칙들도 함께 들어왔으니,
만약 그로티우스의 저서가 권위를 획득한 후에 수행된 전쟁들을
그 이전의 전쟁들과 비교해본다면,
로마법의 규칙들이 수용되자마자 이제 전쟁은 그나마 어느 정도
인내할 만한 성질의 것이 되었음을 알 수 있을 것이다.
선점에 관한 로마법이 근대 만민법\latin{law of nations}에
어떤 해로운 영향을 끼쳤다고 비난받아야 한다면,
해로운 영향을 입었다고 자신 있게 말할 수 있는
분야는
근대 만민법의 다른 영역에 존재한다.
보석의 발견에 로마인들이 적용한 원리를 새로운 땅의 발견에도 적용함으로써,
공법학자\latin{publicist}들은
원래 기대되는 용도와 전혀 맞지 않는 곳에다 억지로
어떤 법리를
가져다 썼다.
15, 16세기의 위대한 항해자들의 발견으로 극히 중요한 것으로 부상한
저 법리는
문제를 해결하기보다는 오히려 야기시켰다.
확실성이 무엇보다 요청되는 두 가지 사항에 관하여
커다란 불확실성이 존재한다는 것이 당장 드러났거니와,
하나는 발견자가 주권자를 위해 취득한 영토의 범위에 관한 것이고,
다른 하나는 `집지'\hanjalatin{執持}{adprehensio},
즉 주권적 점유의 확보\latin{assumption}에
필요한 행위가 무엇이냐에
관한 것이다.
더욱이,
약간의 행운의 결과치고 엄청난 이득을 가져다주는 저 원리는
유럽의 가장 모험적인 몇몇 국가들, 즉 네덜란드, 영국, 그리고 포르투갈에 의해
본능적으로 거부되었던 것이다.
우리 영국인들은,
저 국제법 규칙을 대놓고 부인하지는 않았지만,
실제로는
멕시코만 이남의 아메리카 대륙을 전부 독점한다는 스페인의 주장을
결코 받아들이지 않았다.
오하이오강 유역과 미시시피강 유역을 독점한다는 프랑스왕의 주장도 마찬가지였다.
엘리자베쓰 1세의 등극부터 찰스 2세의 등극에 이르기까지
아메리카의 수역\hanja{水域}에는 완전한 평화가 깃든 날이
하루도 없었다고 할 수 있고,
프랑스왕의 영토에 대한
뉴잉글랜드의 식민가들의
잠식은
그로부터도 한 세기 이상 계속되었다.
저 법리의 적용을 둘러싼 혼란상에 충격을 받은 벤담은
아조레스 제도 서쪽 100리그\latin{league} 지점에 그은 선을 기준으로
스페인과 포르투갈 간에
이 세상의 미발견된 땅을
나누어갖도록 한
저 유명한 알렉산데르 6세 교황의 칙서를 짐짓 칭송하기까지 했다.\footnote{%
  1493년 알렉산데르 교황의 칙서는
  아조레스 제도 서쪽 100리그 지점 자오선의 서쪽을
  아라곤^^b7카스티야 왕국에 주었다.
  이에 불만을 품은 포르투갈은 스페인과의 협상 끝에
  다음 해인 1494년 저 유명한 `토르데시야스 조약'을 맺어
  교황의 자오선을 조금 더 서쪽으로 옮겼다.
}
그의 칭송이 일견 생뚱맞아 보이기는 하지만,
손으로 잡을 수 있는 귀중품의 취득 요건으로
로마 법학자들이 내건 조건을
어떤 군주의 신민이
수행했다고 해서 그 군주에게
대륙의 절반을 내주는 공법학자들의 법규칙보다
과연
저 알렉산데르 교황의 조치가
원칙적으로 더 불합리한 것인지는 의문의 대상일 수 있다.

\para{소유권의 기원}
본 저서의 주제를 연구하는 모든 사람에게
선점은
그것이
사변적 법학에
사적 소유권의 기원에 관한 가상의 설명을
제공하고 있다는 점에서
특히 관심의 대상이다.
애초 공유의 대상이었던 대지와 그 열매가
개인적 소유권의 대상으로 허용되는 과정이
선점이 이루어지는 과정과 동일하다고 한때 널리 믿어졌다.
자연법에 관한 고대적 관념과 근대적 관념 간의 미묘한 차이를 포착한다면,
이러한 가정\hanja{假定}을 이끌어내는 사고방식을
그리 어렵지 않게 이해할 수 있다.
로마 법률가들은 선점을 자연법적 물건취득의 한 방식이라고 주장했고,
만약 인류가 자연의 제도 하에 살고 있다면
선점도 인류의 관행의 일부일 것이라고 그들은 분명 믿었을 것이다.
인류가 실제로 그러한 상태에서 살았던 적이 있다고 그들이 과연 믿었는지는,
전술한 바처럼, 남아있는 자료로는 확인하기가 어렵다.
그러나 확실히 그들은
소유권 제도가 인류의 존재만큼 오래된 것은 아니라고 생각했던 것으로 보이며,
이런 생각은 시대를 막론하고 상당한 설득력을 가지는 것이다.
그들의 모든 도그마를 유보 없이 수용한 근대법학은
가상의 자연상태를 강조하는 열성에 있어서만큼은
그들보다 훨씬 멀리 나아갔다.
그리하여 근대법학은
대지와 그 열매가 한때 무주물이었다는 명제를
수용했을 뿐만 아니라,
자연에 대한 특유한 견해로 인해
국가사회가 형성되기 오래 전부터
인류가 무주물의 선점을 실제로 관행했었다고
서슴없이
가정하기에 이르렀다.
그리고 이로부터
원시 시대의 ``누구의 것도 아닌 물건''\latin{no man's goods}이
역사 시대의 개인의 사적 소유권으로 되는 과정이
바로 선점이었다는 추론이
즉시
도출되었다.
이런 이론을
이런저런 형태로
지지하는 법학자들을 일일이 열거하는 것은
지루한 일이 될 터이고,
그다지 필요하지도 않을 것이다.
언제나
당대의 평균적 의견의 충실한 지표 역할을 하는
블랙스톤이 그의 저서 제2권 제1장에서
그것을 잘 요약해놓았기 때문이다.

\para{블랙스톤의 이론}
그는 이렇게 쓰고 있다.
``대지와 대지 위의 모든 것은 창조주의 직접적 증여로서
인류 공동의 재산이었다.
물론
최초의 시기에도
물건의 공유성은
물건의 본질에만 적용될 수 있을 뿐이었고,
그것의 사용에까지 확장될 수 없었다.
왜냐하면, 자연법과 이성법에 따르면,
물건을 처음 사용하기 시작한 사람은
일종의 일시적 소유권을 취득하고
그것을 계속 사용하고 있는 동안은 그 일시적 소유권도 계속되기 때문이다.
보다 정확히 말하자면,
점유 행위가 지속되는 동안은 점유권도 지속되는 것이다.
그리하여 토지는 공유였고,
토지의 그 어떤 일부도 특정인의 영구적 소유권의 대상일 수 없었으나,
누군가가
휴식을 위해, 그늘을 위해, 또는 다른 이유로
특정 장소를 선점\latin{occupation}하면,
그는 당분간 일종의 소유권을 취득하고,
그에게서 강제로 그 소유권을 빼앗는 것은 부정의하고
자연법에 반하는 일이 될 것이다. 하지만
그가 사용이나 점유\latin{occupation}를 그치는 순간,
다른 사람이 그 장소를 차지하는 것은 아무런 부정의가 아니다.''
그리고 이렇게 주장을 이어간다.
``인류의 인구가 증가하면서,
보다 영구적인 소유권 관념이 필요하게 되었고,
개인에게
일시적인 사용을 넘어
물건의 본질을 사용할 수 있도록
허용할 필요가 생겨났다.''


