\chapter{물권법의 초기 역사}

로마의 \wi{법학제요} 저서들\footnote{%
  가이우스의 법학제요와 유스티니아누스의 법학제요를 일컫는다.
}은
소유권의 여러 형태들과 변종들을 정의한 후,
자연법상의 물건취득 방식들에 대하여 논한다.
법제사를 잘 모르는 이들은
취득의 이러한 ``자연법상의 방식들''이
일견
사변적으로나 실무적으로나 큰 관심의 대상이 아닐 것이라고
생각하기 쉽다.
야생동물을 덫으로 잡거나 사냥해서 죽이는 것,
토양이 강물에 의해 충적되어 부지불식간에
내 땅에 부합\hanja{附合}하는 것,
나무가 내 땅에 뿌리를 내리는 것 따위를
로마 법률가들은 모두 \hemph{자연적으로} 취득한다고 말했다.
옛 법학자들은
그들 주위의 여러 작은 사회의 관행에서
이들 취득이
보편적으로 인정되는 것을 분명 관찰했을 것이다.
후대의 법률가들은
이들이 옛 \wi{만민법}\hanjalatin{萬民法}{jus gentium}에 분류되어 있고
단순명쾌하게 기술\hanja{記述}되어 있는 것을 보았고, 그리하여
이것들에
자연법의 자리를
내주었을 것이다.
이것들에 부여된 존엄성은 근대에 이르러 점점 커져,
이제는 원래 가졌던 중요성을 훨씬 능가하는 것이 되었다.
자연법 이론은 이것들을 가장 즐기는 음식으로 삼았고,
실무에 사뭇 심각한 영향력을 행사할 수 있도록 만들었다.

\para{선점}
이러한 ``자연법적 취득방식들'' 가운데
한 가지만은 반드시 짚고 넘어갈 필요가 있거니와,
선점\hanjalatin{先占}{occupatio}이 그것이다.
선점은
취득 당시 누구의 물건도 아닌 것을
\paren{법기술적 정의\hanja{定義}가 이어진다}
당신의 물건으로 삼고자 하는 의사로써
점유하는 것을 말한다.
로마 법률가들이 무주물\hanjalatin{無主物}{res nullius}---소유주가
없거나 있어본 적이 없는 물건---이라 불렀던
것이 무엇인지는 열거함으로써만 알 수 있을 뿐이다.
소유주가 \hemph{있어본 적이 없는} 물건에는
야생 동물, 물고기, 야생 조류\hanja{鳥類}, 최초로 캐낸 보석,
새로 발견했거나 경작된 적 없는 토지 따위가 속한다.
소유주가 \hemph{없는} 물건에는
포기된 동산, 버려진 토지,
\paren{특이한 그러나 가공스러운 항목인데}
적\hanja{敵}이 소유한 물건 따위가 속한다.
이 모든 것들은
자기 것으로 삼으려는 의사---일정한 경우 이 의사는
특정한 행위에 의해 명시적으로 드러나야 한다---를 가지고서 처음 점유한
\hemph{선점자}가 완전한 소유권\latin{dominion}을 취득한다.
생각건대,
선점 관행의 보편성으로 인해
한 세대의 로마 법률가들이 그것을 모든 민족에 공통인 법으로
자리매김한 것,
그리고 그 단순성으로 인해
다음 세대의 로마 법률가들이 그것을 자연법에 귀속시킨 것은
그리 어렵지 않게 이해할 수 있다.
그러나 근대 법사\hanja{法史}에서 그것이 누린 행운은
선험적인 고찰로는 얼른 이해되지가 않는다.
로마법의 선점 원리, 그리고 이를 둘러싸고 로마 법학자들이 전개한 법규칙들은
근대 국제법 중에서도
전쟁시 포획에 관한 법과
새로 발견한 땅에 대한 주권 획득에 관한 법의
원천이 되었다.
또한 소유권의 기원에 관한 어떤 이론의 근거가 되었거니와,
이 이론은 대중적으로 인기있는 이론인 동시에,
다수의 위대한 사변적 법학자들이
이러저러한 형태로
널리 수긍하고 있는 이론이다.

\para{적의 소유물, 발견의 법리}
방금 나는 로마법의 선점 원리가
전쟁시 포획에 관한 \wi{국제법}의 흐름을 결정했다고 말했다.
전쟁시 포획법의 법규칙들은,
적대관계의 발발에 의해 국가들은 일종의 자연상태로 환원되고
이렇게 만들어진 인위적인 자연상태 하에서
교전국 간에는
사적 소유권 제도가
중지된다는
가정\hanja{假定}에 기초한다.
후기의 국제법 학자들은
그들이 설명하는 법체계에서도
사적 소유권이 어떤 의미에서는 인정된다는 주장을
유지하려고 했기 때문에,
적의 재산이 무주물이라는 가설은 그들에게
정도\hanja{正道}를 벗어난 충격적인 것으로 여겨졌고,
따라서 그들은 이 가설을 단지 법적인 의제\latin{fiction}에 불과하다고
내세우는 신중함을 보였다.
그러나 자연법이 만민법에 그 기원을 두고 있음을 잘 아는 우리는
어떻게 적의 재산이 무주물로 취급되고 그리하여
최초의 점유자에 의해 취득될 수 있었는지 금방 이해할 수 있다.
고대적 형태의 전쟁을 수행하는 사람들은
승전으로 정복군의 군대가 해산되고
해산된 군인들이 무차별적인 약탈을 자행했을 때
저 관념을 자동적으로 떠올렸을 것이다.
하지만
이때 포획자가 취득하도록 허용된 재산은
원래는 동산에 국한하였을 것으로 보인다.
우리는
고대 이탈리아에서
피정복 국가의 토지에 대한 소유권의 취득에 관해서는
전혀 다른 규칙이 지배했음을 별도의 전거를 통해 알고 있다.
따라서 토지에 대해 선점의 원리가 적용되기 시작한 것은
\paren{항상 어려운 문제이지만}
만민법이 자연법으로 전환되는 시기였을 것으로,
그리하여 황금시대의 법학자들이 행한 일반화의 결과였을 것으로
짐작할 수 있다.
이에 관한 법리는 유스티니아누스의 \wi{학설휘찬}\latin{Pandects}에 보존되어 있거니와,
그것은 모든 종류의 적의 재산은 교전 상대방에게 무주물이라는,
그리고 포획자가 그것을 자기 것으로 만드는 선점은 자연법상의 제도라는,
무제한적 주장으로 나아간다.\footnote{%
  \latin{D.\,41.1.5.7; D.\,41.2.1.1.}
  }
이러한 명제로부터 국제법이 이끌어낸 규칙들은
때로 군인들의 만행과 탐욕을 필요 이상으로 부추긴다고
비판받았지만,
생각건대 이 비판은
전쟁의 역사를 잘 모르는 사람들에 의해,
그리하여 어떤 종류의 규칙이든 규칙에 대한 복종을 명하는 것이
얼마나 위대한 업적인지 잘 모르는 사람들에 의해 가해진 비판이다.
선점에 관한 로마법 원리가 전쟁시 포획에 관한 근대법에 수용되어 들어왔을 때,
그 남용을 제한하고 정밀함을 부여하는
많은 부수적인 법규칙들도 함께 들어왔으니,
만약 \wi{그로티우스}의 저서가 권위를 획득한 후에 수행된 전쟁들을
그 이전의 전쟁들과 비교해본다면,
로마법의 규칙들이 수용되자마자 이제 전쟁은 그나마 어느 정도
인내할 만한 성질의 것이 되었음을 알 수 있을 것이다.
선점에 관한 로마법이 근대 \wi{만민법}\latin{law of nations}에
어떤 해로운 영향을 끼쳤다고 비난받아야 한다면,
해로운 영향을 입었다고 자신 있게 말할 수 있는
분야는
근대 만민법의 다른 영역에 존재한다.
보석의 발견에 로마인들이 적용한 원리를 새로운 땅의 발견에도 적용함으로써,
공법학자\latin{publicist}들은
원래 기대되는 용도와 전혀 맞지 않는 곳에다 억지로
저 법리를
가져다 썼다.
15, 16세기의 위대한 항해자들의 발견으로 극히 중요한 것으로 부상한
저 법리는
문제를 해결하기보다는 오히려 야기시켰다.
확실성이 무엇보다 요청되는 두 가지 사항에 관하여
커다란 불확실성이 존재한다는 것이 당장 드러났거니와,
하나는 발견자가 주권자를 위해 취득한 영토의 범위에 관한 것이고,
다른 하나는 `집지'\hanjalatin{執持}{adprehensio},
즉 주권적 점유의 확보에
필요한 행위가 무엇이냐에
관한 것이다.
더욱이,
약간의 행운의 결과치고 엄청난 이득을 가져다주는 저 원리는
유럽의 가장 모험적인 몇몇 국가들, 즉 네덜란드, 영국, 그리고 포르투갈에 의해
본능적으로 거부되었던 것이다.
우리 영국인들은,
저 \wi{국제법} 규칙을 대놓고 부인하지는 않았지만,
실제로는
멕시코만 이남의 아메리카 대륙을 전부 독점한다는 스페인의 주장을
결코 받아들이지 않았다.
오하이오강 유역과 미시시피강 유역을 독점한다는 프랑스왕의 주장도 마찬가지였다.
엘리자베쓰 1세의 등극부터 찰스 2세의 등극에 이르기까지
아메리카의 수역\hanja{水域}에는 완전한 평화가 깃든 날이
하루도 없었다고 할 수 있고,
프랑스왕의 영토에 대한
뉴잉글랜드 식민지인들의
잠식은
그로부터도 한 세기 이상 계속되었다.
저 법리의 적용을 둘러싼 혼란상에 충격을 받은 \wi{벤담}은
아조레스 제도 서쪽 100리그\latin{league} 지점에 그은 선을 기준으로
스페인과 포르투갈 간에
이 세상의 미발견된 땅을
나누어갖도록 한
저 유명한 알렉산데르 6세 교황의 칙서를 짐짓 칭송하기까지 했다.\footnote{%
  1493년 알렉산데르 교황의 칙서는
  아조레스 제도 서쪽 100리그 지점 자오선의 서쪽을
  아라곤^^b7카스티야 왕국에 주었다.
  이에 불만을 품은 포르투갈은 스페인과의 협상 끝에
  다음 해인 1494년 저 유명한 `토르데시야스 조약'을 맺어
  교황의 자오선을 조금 더 서쪽으로 옮겼다.
  }
그의 칭송이 일견 생뚱맞아 보이기는 하지만,
손으로 잡을 수 있는 귀중품의 취득 요건으로
로마 법학자들이 내건 조건을
어떤 군주의 신민이
수행했다고 해서 그 군주에게
대륙의 절반을 내주는 공법학자들의 법규칙보다
과연
저 알렉산데르 교황의 조치가
원칙적으로 더 불합리한 것인지는 의문의 대상일 수 있다.

\para{소유권의 기원}
본 저서의 주제를 연구하는 모든 사람에게
선점은
그것이
사변적 법학에
사적 소유권의 기원에 관한 가상의 설명을
제공하고 있다는 점에서
특히 관심의 대상이다.
애초 공유의 대상이었던 대지와 그 열매가
개인적 소유권의 대상으로 허용되는 과정이
선점이 이루어지는 과정과 동일하다고 한때 널리 믿어졌다.
자연법에 관한 고대적 관념과 근대적 관념 간의 미묘한 차이를 포착한다면,
이러한 가정\hanja{假定}을 이끌어내는 사고방식을
그리 어렵지 않게 이해할 수 있다.
로마 법률가들은 선점을 자연법적 물건취득의 한 방식이라고 주장했고,
만약 인류가 자연의 제도 하에 살고 있다면
선점도 인류의 관행의 일부일 것이라고 그들은 분명 믿었을 것이다.
인류가 실제로 그러한 상태에서 살았던 적이 있다고 그들이 과연 믿었는지는,
전술한 바처럼, 남아있는 자료로는 확인하기가 어렵다.
그러나 확실히 그들은
소유권 제도가 인류의 존재만큼 오래된 것은 아니라고 생각했던 것으로 보이며,
이런 생각은 시대를 막론하고 상당한 설득력을 가지는 것이다.
그들의 모든 도그마를 유보 없이 수용한 근대법학은
가상의 자연상태를 강조하는 열성에 있어서만큼은
그들보다 훨씬 멀리 나아갔다.
그리하여 근대법학은
대지와 그 열매가 한때 무주물이었다는 명제를
수용했을 뿐만 아니라,
자연에 대한 특유한 견해로 인해
국가사회가 형성되기 오래 전부터
인류가 무주물의 선점을 실제로 관행했었다고
서슴없이
가정하기에 이르렀다.
그리고 이로부터
원시 시대의 ``누구의 것도 아닌 물건''\latin{no man's goods}이
역사 시대의 개인의 사적 소유권으로 되는 과정이
바로 선점이었다는 추론이
즉시
도출되었다.
이런 이론을
이런저런 형태로
지지하는 법학자들을 일일이 열거하는 것은
지루한 일이 될 터이고,
그다지 필요하지도 않을 것이다.
언제나
당대의 평균적 의견의 충실한 지표 역할을 하는
\wi{블랙스톤}이 그의 저서 제2권 제1장에서
그것을 잘 요약해놓았기 때문이다.

\para{블랙스톤의 이론}
그는 이렇게 쓰고 있다.
``대지와 대지 위의 모든 것은 창조주의 직접적 증여로서
인류 공동의 재산이었다.
물론
최초의 시기에도
물건의 공유성은
물건의 본질에만 적용될 수 있을 뿐이었고,
그것의 사용에까지 확장될 수 없었다.
왜냐하면, 자연법과 이성법에 따르면,
물건을 처음 사용하기 시작한 사람은
일종의 일시적 소유권을 취득하고
그것을 계속 사용하고 있는 동안은 그 일시적 소유권도 계속되기 때문이다.
보다 정확히 말하자면,
점유 행위가 지속되는 동안은 점유권도 지속되는 것이다.
그리하여 토지는 공유였고,
토지의 그 어떤 일부도 특정인의 영구적 소유권의 대상일 수 없었으나,
누군가가
휴식을 위해, 그늘을 위해, 또는 다른 이유로
특정 장소를 선점\latin{occupation}하면,
그는 당분간 일종의 소유권을 취득하고,
그에게서 강제로 그 소유권을 빼앗는 것은 부정의하고
자연법에 반하는 일이 될 것이다. 하지만
그가 사용이나 점유\latin{occupation}를 그치는 순간,
다른 사람이 그 장소를 차지하는 것은 아무런 부정의가 아니다.''
그리고 이렇게 주장을 이어간다.
``인류의 인구가 증가하면서,
보다 영구적인 소유권 관념이 필요하게 되었고,
개인에게
일시적인 사용을 넘어
물건의 본질을 사용할 수 있도록
허용할 필요가 생겨났다.''\footnote{%
  \latinmarks
  William Blackstone,
  \textit{Commentaries on the Laws of England: In Four Books},
  Philadelphia: George W. Childs, 1866,
  Book 2, p.\,2. }

위 인용문에 나타난 몇몇 모호한 표현을 볼 때,
\wi{블랙스톤}은
그가 참조한 전거들에 나오는,
\hemph{선점자}가 자연법에 의하여 지구 표면에 대한 소유권을 취득한다는
명제를 제대로 이해하지 못한 것이 아닌가 한다.
그러나
의도적이든 오해에 의해서든
저 이론에 이처럼 제한을 가함으로써
그는
흔히들 상정되어온 형태를 그대로 따르고 있다.
언어의 정확한 구사에 있어
블랙스톤보다 더
유명한 많은 학자들이
태초에는
우선
선점에 의해
배타적이지만 일시적인 향유권이 대세\hanja{對世}적으로
부여되었고,
그후 이 권리가 배타성은 유지한 채 영구적인 것이 되었다고
주장해왔다.
그들이 이렇게 이론을 전개한 목적은
자연상태에서는
무주물이
선점에 의해
소유권의 대상이 된다는 법리와
가부장들이
양떼와 소떼에게 풀을 먹이던 토지를
처음에는
영구적으로 차지하지 않았다는
성서의 이야기에서
추론한 결과를
조화시키려는 것이었다. %\footnote{창세기 13.}

\wi{블랙스톤}의 이론에 직접 적용될 수 있는 한 가지 비판은
그가 묘사하는 원시사회의 상태가
똑같이 쉽게 상상할 수 있는 다른 상태들보다 과연 더 설득력이 있는 것이냐
하는 데 있다.
이 문제를 탐구하기 위해
우리는
토지의 특정 장소를 휴식이나 그늘을 위해
\hemph{선점한}\latin{occupied}
\paren{블랙스톤은 이 단어를
일상적인 의미로 사용하고 있음이 분명하다}
사람이 아무런 방해 없이 그것을 보유할 수 있겠는가를
질문해보는 것이 좋겠다.
그의 점유권은 그것을 지킬 수 있는 힘에 정확히 비례할 것이고,
똑같이 그 장소를 갈망하고 있고
점유자를
충분히
힘으로
내쫓을 수 있다고 생각하는
경쟁자들에 의해 끊임없이 방해를 받을 것임에 틀림없다.
그러나 사실,
이 모든 트집잡기는 저 이론 자체의 근거없음에 비하면 한가한 이야기에 불과하다.
원시 상태의 사람들이 무엇을 했는가 탐구하는 것은
전혀 희망없는 일은 아닐 수 있지만,
그들 행위의 동기를 안다는 것은 도저히 불가능한 일이다.
태초의 사람들에 대한 저 이론의 묘사는,
오늘날 우리가 처해있는 상태와는 사뭇 다른 상태에
그들이 놓여있었다고 우선 가정함으로써,
그리고는
이런 가상의 상황에서도
지금 우리가 가지고 있는 감정과 선입견을
그들도
똑같이 가지고 있었다고 상정함으로써 이루어진다.
그 감정이 실은 가설상의 그들의 상태와는 전혀 다른 상태에서
만들어진 감정일 수 있음에도 말이다.

\para{사비니의 금언}
소유권의 기원에 관하여
블랙스톤이 요약한 것과 비슷한 견해를 지지하는 것으로
때로 여겨져온 사비니의 금언\hanja{金言}이 있다.
저 위대한 독일 법학자는
모든 소유권의 기초가
취득시효\hanjalatin{取得時效}{prescription}에 의해 완성되는
적대적 점유\latin{adverse possession}에 있다고 주장했다.
사비니의 이러한 진술은 오직 로마법에만 근거한 것이고,
진술에 사용된 표현들을 설명하고 정의하는 데 충분한 노력을 들여야만
완전히 이해될 수 있는 것이다.
하지만
로마인들이 채용한 소유권 관념을 아무리 깊게 탐구하더라도,
법의 유년기에 이르기까지 그 관념을 아무리 멀리 추적하더라도,
저 금언에 포함된 세 가지 요소---점유, 적대적 점유, 그리고 취득시효---로
구성된 것 이상의
다른 소유권 개념은
얻을 수는 없다는 주장으로
그의 주장을
이해한다면,
우리는 그가 말하고자 한 바를 충분히 정확하게 이해했다고 할 수 있다.
여기서
적대적 점유란 허락받은 점유나 종속적인 점유가 아닌,
온 세상을 상대로 배타성을 주장하는 점유를 말한다.
취득시효란 적대적 점유가 평온하게 지속되어온 시간의 경과를 말한다.
이 금언은 사비니가 의도한 것을 넘어 더 일반적으로 적용될 수도 있다고 믿는다.
그리하여 어떠한 법체계를 조사해보더라도
이들 세 가지가 결합된 소유권 개념 이상의
건전하고 안정적인 결론을 찾아내기란 불가능하리라고 믿는다.
동시에,
소유권의 기원에 관한 대중적인 이론을 지지하기는커녕,
사비니의 금언은 그것의 가장 약한 고리를 드러내는 특별한 가치를 가지고 있다.
블랙스톤 및 그가 추종하는 사람들의 견해에서는,
배타적 점유를 획득하는 방식이
인류의 조상들의 정신에 어떤 신비로운 영향을 주었었다.
그러나 사비니의 금언에는 이러한 신비로움이 없다.
적대적 점유에서 소유권이 시작한다는 데는 놀라울 것이 전혀 없다.
최초의 소유자는 자신의 재산을 안전하게 지켜내는,
무장\hanja{武裝}한 힘있는 사람이었을
것이라는 데는 놀라울 것이 전혀 없다.
하지만 어째서 시간의 경과가 그의 점유를 존중하는 감정---이것이야말로
장기간 사실상\latin{de facto} 존재해온 것에 대한 인류 보편의 존중심의
원천이다---을 만들어내는지는
진정 깊이 연구해볼 가치가 있는 문제이지만,
이는 지금 우리의 탐구 범위를 훨씬 넘어선다.

\para{소유권의 추정}
드물고 불확실한 정보에 불과하지만
소유권의 초기 역사에 관한 약간의 정보를 얻을 수 있을 법한 지역을 다루기에 앞서,
우선 나는
문명의 초기 단계에서 선점이 수행한 역할에 주목하는 대중적 견해가
실은 진실을 거꾸로 뒤집은 것이라는 점을 감히 지적하고자 한다.
선점은 의사\hanja{意思}에 의한 물리적 점유의 획득이다.
이런 유\hanja{類}의 행위가 ``무주물''에 권리를 수여한다는 관념은,
초기 사회의 특징이기는커녕,
세련된 법학과 확립된 법상태의 결과물일 공산이 농후하다.
소유권이
오랜 관행을 통해
그 불가침성을
인정받은 후에야,
대부분의 향유의 객체가 사적 소유권의 대상이 되고 난 연후에야,
이전에 소유권이 주장된 바 없는 물건에 대한 소유권을
최초의 점유자가
단순한 점유에 의해
수여받도록
비로소
허용되는 것이다.
이러한 원리를 만들어낸 감정은
문명의 시초를 특징짓는
저 희소하고 불확실한 소유권과는 전혀 조화되지 않는 것이다.
그 감정의 진정한 기초는
소유권 제도를 지향하는 본능적인 선입견이 아니라,
소유권 제도의 오랜 지속으로 등장한,
\hemph{모든 것은 주인이 있어야 한다}는 추정\hanja{推定}인 것이다.
``무주물'', 즉
소유권의 대상이 \hemph{아닌}
또는 소유권의 대상인 적이 \hemph{없는}
객체가 점유될 때,
점유자가 소유권자로 허용되는 것은
모든 가치있는 물건은 당연히 배타적 향유의 대상이라는,
그리고
당해 사례에서는
선점자 외에 소유권을 수여받을 사람이 없다는 감정에
기인한다.
요컨대,
모든 물건은 누군가의 소유물이어야 하기에,
그리고
특정 물건의 소유권자로 선점자보다 더 나은 권리를 갖는
사람을 찾을 수 없기에,
선점자가 소유권자가 되는 것이다.

\para{대중적 이론의 반박}
우리가 논의해온 자연상태 사람들에 대한 기술\hanja{記述}에 대해
다른 반박이 없다 할지라도,
적어도 한 가지 점에 있어서만은
그것은 우리가 가진 믿을 만한 증거에 결정적으로 배치되고 있다.
저 이론이 상정하는 행위와 동기는 개인의 행위와 동기라는 점을 주목하자.
사회계약 체결의 당사자는 각 개인들이다.
홉스의 이론에 의하면
개인이라는 모래알로 구성된 어떤 움직이는 모래더미가
완전한 강제력에 의해 사회적 바위로 굳어지는 것이다.
\wi{블랙스톤}의 묘사에서
``휴식을 위해, 그늘을 위해, 또는 다른 이유로 특정 장소를 선점하는''
것도 개인이다.
로마인들의 자연법에서 유래한 모든 이론이
이 결함으로부터 자유로울 수 없거니와,
로마인들의 자연법은 개인을 취급함에 있어 그들의 시민법과 근본적으로 달랐고,
초기사회의 권력으로부터 개인을 해방시킴으로써
문명의 진보에 크게 기여하였다.
그러나 고대사회는, 반복하여 말하지만,
개인을 거의 알지 못한다.
그것의 관심은 개인이 아니라 가족에 있었고,
단독의 인간이 아니라 집단에 있었다.
국가법이 친족집단이라는 작은 원\hanja{圓}들을
원래는 전혀 뚫지 못하다가
마침내 뚫고 들어갔을 때도,
그것이 바라보는 개인은 훗날 성숙한 단계의 법이 바라보는 개인과
사뭇 달랐다.
각 시민의 생애는
출생과 사망에 의해 한계지워지지 않았다.
그는 그의 선조들의 존재의 계속이었을 뿐이고,
또한 그의 후손들의 존재 속에 계속 살아갈 것이었다.

\para{인법과 물법}
편리하기는 하지만 전적으로 인위적인,
\wi{신분법}\latin{law of persons}과 \wi{재산법}\latin{law of things} 간의
로마인들의 구별은
지금 우리 앞에 놓인 주제에 대한 탐구를 올바른 경로에서
벗어나게 하는 데 분명 크게 기여했다.
\index{인법|see{신분법}}인법\hanjalatin{人法}{jus personarum}에서 배운 지식은
\index{물법|see{재산법}}물법\hanjalatin{物法}{jus rerum}에 이르러서는 완전히 망각되었다.\footnote{%
  인법과 물법은 각각 신분법과 (물권법을 포함하는) 재산법을 지칭하는
  로마인들의 용어.
  }
그리하여
인법의 원초적 상태에 관해 알게된 사실로부터
물권법, 계약법, 불법행위법의 기원에 관한 힌트를 전혀 얻을 수가 없는 것처럼
지금까지 생각되어왔다.
이런 사고방식이 잘못되었다는 것은
순수한 고법\hanja{古法}체계 하나를 가져다 놓고
그것에 로마법의 분류를 적용하는 실험을 해볼 수 있다면
분명해질 것이다.
법의 유년기에는
재산법으로부터 신분법을 분리하는 것이 무의미하다는 것을,
두 영역에 속하는 법규칙들이 불가분 서로 엉켜있다는 것을,
후대 법학자들의 구분은 후대의 법에만 적합하다는 것을
알게 될 것이다.
본 저서의 앞 부분에서 말한 것들을 종합하면,
우리가 개인의 소유권에만 관심을 국한하면
초기 소유권의 역사에 대한 어떠한 단서도 얻을 수 없을 것이라는
강한 선험적 개연성을 감지할 수 있을 것이다.
개인적 소유가 아니라 공동소유가 초기법의 진정한 모습이라는 것은,
우리에게 시사점을 주는 소유 형태는 가족의 권리, 친족집단의 권리와
관련된다는 것은 그저 그럴 수도 있겠다는 정도를 넘어선다.
여기서 로마법은 우리를 깨우치는 데 별 도움을 주지 못하거니와,
자연법 이론에 의해 변형되어
우리에게 전달된 바로 그 로마법이
개인적 소유가 소유권의 정상적 상태라는 인상을,
인간집단이 공유하는 소유권은 원칙에 대한 예외에 불과하다는 인상을,
우리에게 심어주기 때문이다.
하지만 원초적 사회의 잃어버린 제도를 탐구하는 연구자라면
반드시 주의깊게 살펴봐야 할 공동체가 하나 있다.
오래 전부터 인도에 정착해 살아온,
인도^^b7유럽 계통의 한 갈래인 사람들 사이에서
그 제도가
어떤 변천을 겪어왔든 간에,
그것은
자신을 배태한 껍질을 완전히 벗어버리지 못했다는 것을 알 수 있을 것이다.
소유의 원초적 형태에 관하여
우리의 신분법 연구로부터
얻을 수 있는
아이디어의
형태에 정확히 들어맞는다는 점에서
즉시
우리의 눈길을 끄는 그러한 소유 형태가
인도인들 사이에서
발견되는 것이다.
인도의 \wi{촌락공동체}\latin{village community}는 조직화된 \wi{가부장제} 사회이자
공동소유자들의 연합체이다.
그것을 구성하는 사람들 간의 인적 관계는
그들의 소유권과 불가분 결합되어 있어서,
영국인 관리들이 이 둘을 분리하려고 하면 이는
영국의 인도 통치에 있어 가장 치명적인 실책이 될 것이다.
이 촌락공동체는 무한히 오래된 것으로 알려져있다.
인도 역사를 어느 방향에서 접근하든 간에,
일반 역사든 지방 역사든 간에,
이 공동체가 진보의 초기부터 존재했음이 항상 발견되어왔다.
대부분 그것의 성격과 기원에 대한 특별한 이론을 갖지 않는
무수한 지식인들과 관찰자들이
이구동성으로 말하기를,
이 사회가
어떤 혁신에도 좀처럼 굴하지 않고 지켜온 관행 중에서도
그것이야말로 가장 파괴되기 어려운 관행이라고 한다.
정복과 혁명이 수차례 휩쓸고 지나갔지만
그것을 어지럽히거나 없애지 못했으며,
인도에서 가장 유익한 통치체제는 언제나
그것을 행정의 기초로 인정하는 통치체제였던 것이다.

\para{공동체와 분할}
성숙한 로마법과 그 자취를 따른 근대법은
공동소유를 소유권의 예외적이고 일시적인 상황으로 바라본다.
이런 견해는
``누구도 자신의 의사에 반하여 공동소유에 묶이지 않는다''%
\latin{Nemo in communione potest invitus detineri}는,
서유럽에서 보편적으로 받아들여지는 법언에 명료하게 드러나 있다.
그러나 인도에서는
관념의 순서가 거꾸로이며,
개별 소유권은 언제나 공동소유권으로 되돌아가는 경향이 있다고 할 수 있다.
그 과정에 대해서는 이미 언급한 바 있다.
아들이 태어나자마자
그는 아버지의 재산에 대해 확정적 권리를 취득한다.
결정권을 행사할 수 있는 나이가 되면,
일정한 경우
가족 재산의 분할을 요구할 권리가
법문\hanja{法文}에 의해 주어진다.
하지만 사실
아버지의 사망시에도 분할은 잘 일어나지 않는다.
재산은 수 세대에 걸쳐 분할되지 않은 채 계속 유지되거니와,
다만
각 세대의 각 구성원들이 미분할된 지분에 대해 법적 권리를 가질 뿐이다.
이러한 공동소유의 토지는 때로는 선출된 관리자에 의해 관리되지만,
일반적으로는, 그리고 일부 지역에서는 언제나,
가장 나이 많은 종친\hanja{宗親}, 다시 말해
가장 손윗 계통의 가장 나이 많은 대표자에 의해 관리된다.
이러한 공동소유자들의 연합체, 즉
토지를 공동소유하는 친족집단은
인도 \wi{촌락공동체}의 가장 단순한 형태이다.\footnote{%
  하지만 <<고대법>>에 대한 폴록의 주석에 따르면
  이러한 공동소유적 촌락공동체보다
  각 가(家)의 개별 소유권을 인정하는
  공동체 유형(특히 인도 북부에서 지배적이다)이
  더 오래된 유형임이 이후의 인도 연구에서 밝혀졌다고 한다.
  }
그러나
이 공동체는 친족으로 구성된 동족집단 그 이상이고
조합원들로 구성된 조합 그 이상이다.
그것은 하나의 조직화된 사회이다.
공동재산을 관리하는 것 외에도,
거의 항상 그것은 다수의 스태프들을 통하여
내치\hanja{內治}를,
치안을,
사법\hanja{司法}을,
그리고 조세와 부역\hanja{賦役}의 할당을
수행한다.

\para{인도의 촌락공동체}
내가 기술한 촌락공동체의 형성과정은 전형적인 것으로 간주해도 좋다.
하지만
인도의 모든 촌락공동체가
그러한 단순한 방식으로 결합되어 있다고 생각해서는 안 된다.
인도 북부에서는,
기록에 의하면,
공동체는
거의 항상
혈연관계에 기초한 단일한 연합체의 모습이라지만,
같은 기록은
때로 외부인이 접목되어 들어가는 일이 늘 있어왔다는 것도
알려준다.
일정한 조건 하에서는
단순히 지분을 매수한 것에 불과한 자가 동족집단에 받아들여지는 것이다.
인도 반도 남부에서는
하나의 가족이 아니라 둘 이상의 가족에서 유래한 것으로 보이는
공동체도 다수 존재하거니와,
어떤 공동체는 그 구성이 전적으로 인위적인 것으로 알려져있다.
사실,
서로 다른 카스트에 속하는 사람들이
동일한 사회로
결합하는 것은
공통 조상의 후손이라는 가설에 전혀 부합하지 않는 것이다.
그럼에도 불구하고 이 모든 동족집단에는
최초의 공통 조상에 관한 전승\hanja{傳承}이 내려오거나
혹은 그러한 가정\hanja{假定}이 만들어져있다.
남부의 \wi{촌락공동체}를 집중 연구한
마운트스튜어트 엘핀스톤\latin{Mountstuart Elphinstone}은
이렇게 말한다\paren{<<인도사>>\latin{History of India}, 71쪽. 1905년판}:
``대중적 견해에 따르면,
마을의 지주들은 모두가 그 마을에 정착한 하나 이상의 개인들의 후손이다.
유일한 예외는 원주민 혈통의 구성원에게서 매수 등을 통해
권리를 취득한 사람들뿐이다.
이런 견해는
오늘날
작은 마을에는 지주 가족이 하나만 존재하고
큰 마을에도 몇 안 되는 가족만 존재한다는
사실에서도 확인된다.
그러나 각 마을은 수많은 구성원들로 분기\hanja{分岐}되어왔기에,
농사에 필요한 노동은
소작인이나 노무자의 도움 없이
전적으로 지주들에 의해 행해지는 경우가 적지 않다.
지주들의 권리는 그들에게 집단적으로 속한다.
그들은 거의 항상 그 권리에 대한 다소간의 완전한 지분을 갖지만,
완전한 분리가 일어나는 일은 결코 없다.
가령 어떤 지주가 그의 권리를 매각하거나 저당잡힐 수는 있다.
그러나 그러려면 그는 우선 마을의 동의를 얻어야 한다.
또한 매수인은 정확히 매도인의 지위를 대신하고 그의 모든 의무도 넘겨받는다.
만약 어떤 가족이 소멸하게 되면, 그 지분은 공동재산으로 되돌려진다.''

\para{공동체의 전형}
본서 제5장에서 살펴본 고찰이 엘핀스톤의 인용문을 이해하는 데
독자들에게 도움을 주리라 믿는다.
원시 세계의 어떤 제도도
생동력 있는 법적의제를 통해
원래의 성질에는 없는 유연함을
얻지 못했다면
오늘날까지 전해지지 못했을 것이다.
그리하여 \wi{촌락공동체}는 반드시 혈연관계의 연합체인 것이 아니라,
그러한 연합체\hemph{이거나 아니면} 친족관계의 모델에 기초하여 형성된
공동소유자 집단인 것이다.
이것에 비견되어야 할 유형은 로마의 가족이 아니라,
로마의 \wi{씨족}\latin{gens}임에 틀림없다.
씨족 또한 가족의 모델에 기초한 집단이었다.
그것은 다양한 의제를 통해 확대된 가족이었거니와,
그 의제의 정확한 성질이 무엇인지는 아주 오래 전에 잊혀져버렸다.
역사 시대에 이르렀을 때, 그것의 주된 성질은
촌락공동체에 관한 엘핀스톤의 언급에 나타난 바로 그 두 가지였다.
공통의 기원에 관한 가정\hanja{假定}이 항상 있었거니와,
다만 그 가정은 때로는 실제 사실과 노골적으로 배치되기도 한다.
그리하여, 저 역사학자의 말을 반복하자면,
``만약 어떤 가족이 소멸하게 되면, 그 지분은 공동재산으로 되돌려진다.''
옛 로마법에서도 상속인 없는 상속재산은
씨족원들\latin{gentiles}에게 복귀하였던 것이다.
나아가, 로마사 연구자라면 누구나
씨족과 같은 공동체는 외부인을 수용함으로써 수시로 불순물이 혼입되었다고,
그러나 수용의 정확한 방식은 지금으로서는 알 수 없다고 믿고 있다.
이제 인도에서는, 엘핀스톤이 알려주듯이,
동족집단의 동의를 얻어 매수인이 받아들여짐으로써
외부인이 유입되는 것이다.
하지만 수용된 자의 취득의 성질은
\wi{포괄적 승계}\latin{universal succession}에 해당한다.
매수한 지분뿐 아니라,
그는
전체 집단에 대해 매도인이 부담하고 있던 책임도
함께
승계하는 것이다.
\index{가의 매수인}%
그는 바로 가\hanja{家}의 매수인\latin{emptor familiae}으로서,
그가 대체하게 될 사람의 법적인 옷\hanja{[衣服]}을 물려입는 것이다.
그를 수용하는 데 필요한 전체 동족집단의 동의는,
쿠리아 민회\latin{comitia curiata}, 즉
동일한 이름을 가진 친족집단이 모인 보다 큰 동족집단인 고대 로마 국가의
입법기구가
\wi{입양}의 허가나 유언의 확인에 반드시 필요하다고 강하게 주장했던
그 동의를 상기시킨다.

\para{러시아의 촌락공동체, 크로아티아의 촌락공동체}
인도 촌락공동체의 거의 모든 특징들에서
그것이 대단히 오래된 것임을 알려주는 징후를 발견할 수 있다.
이 법의 유년기에는 공동소유가 지배적이었음을,
신분권과 재산권이 서로 엉켜있었음을,
공적 의무와 사적 의무가 혼재되어 있었음을 알려주는
수많은 근거들이 있어서,
이들 공동소유의 동족집단을 관찰함으로써 여러 중요한 결론들을
이끌어내도 무리가 없다 하겠거니와,
유사한 구조를 가진 사회가 세계 어디서도 발견되지 않는다 해도 그러할 것이다.
하지만,
봉건제에 의한 소유권의 격변을 그다지 겪지 않았고
여러 중요한 면에서 동양과도 서양과도 밀접한 친화성을 갖는
유럽의 한 지역에 존재하는 유사한 구조의 현상들이
최근 많은 진지한 관심의 대상이 되고 있다.
학스타우젠\latin{August von Haxthausen} 씨,
텡고보르스키\latin{Ludwig Tengoborski} 씨 등의 연구자들이
러시아의 촌락은 사람들의 우발적인 결합도 아니고
그렇다고 계약에 기초한 결합도 아님을 보여주었다.\footnote{%
  러시아의 촌락공동체는 흔히 `미르'(mir) 혹은 `옵쉬나'(obshchina)라고 불린다. }
그것은 인도의 공동체와 마찬가지로 자연적으로 조직화된 공동체인 것이다.
물론,
이들 촌락은 이론적으로는 언제나 어떤 귀족의 소유지이며,
역사 시대 내내
농부들은
그 영주의 토지에 예속된 농노\latin{predial serf}로,
혹은 보다 일반적으로는
그에게 신분적으로 예속된 농노\latin{personal serf}로 전락해갔다.
그러나
이러한 상급소유권의 압력도
촌락의 고대적 구조를 파괴시키지 못했다.
농노제를 도입한 것으로 여겨지는 러시아 짜르의 입법도
기실 옛 사회질서를 유지하는 데 불가결한 저 협력관계를
농부들이 버리지 못하도록 막기 위해 만들어진 것이었다.
마을 사람들 간의 종족\hanja{宗族}적 관계를 고려할 때,
\wi{신분법}과 \wi{재산법}의 혼재를 고려할 때,
또한 다양한 자발적 자치규범들을 고려할 때,
러시아의 촌락은 인도 \wi{촌락공동체}의 거의 정확한 반복으로 보인다.
그러나 한 가지 중대한 차이점이 있으니,
크게 관심을 둘 만하다.
인도 촌락의 공동소유자들은,
비록 그들의 소유권이 통합되어 있기는 하나,
각자 자기만의 권리를 가지며,
이러한 권리들이 분할되면 그 분리는 완전하고 또한 무한히 지속된다.
러시아 촌락에서도 이론적으로 권리들의 분할은 완전하지만,
그러나 여기서는 그것이 일시적인 데 그친다.
일정한, 그러나 모든 경우에 다 동일하지는 않은,
시간이 경과하면
분리된 소유권들은 소멸하여,
그 촌락의 토지가 하나의 덩어리로 합쳐지고, 그후
공동체를 구성하는 가족들 간에 식구 수에 따라 재분배된다.
이러한 재분할이 행해지면,
가족들의 권리와 개인들의 권리는 다시
다수의 계통으로 가지를 칠 수 있거니와,
이러한 가지치기는 또 다른 분할의 시기가 도래할 때까지 계속된다.
이러한 소유권 유형의 변종인 더욱 특이한 형태가
오랫동안
투르크 제국과 오스트리아 왕가의 영토 사이에 분쟁지역이 되어온
몇몇 나라들에서 발견된다.
세르비아, 크로아티아, 오스트리아령 슬라보니아에서도
촌락들은 공동소유자이자 친족관계인 사람들로 구성된 동족집단이다.\footnote{%
  이들 남슬라브 지역의 촌락공동체는 흔히 `자드루가'(zadruga)라고 불린다. }
그러나 여기서는 공동체의 내부 구조가
앞서 살펴본 두 사례와 상이하다.
여기서는 공동소유의 재산이 실제로도 분할되지 않고
이론적으로도 분할될 수 없다고 간주된다.
토지 전부가 마을 사람 모두의 공동의 노동으로 경작되고,
수확물은 매년 가구별로 분배되거니와,
때로는 각 가구의 필요에 따라서,
때로는 특정인에게 \wi{용익권}\latin{usufruct}의 일정 몫을 주는 규칙에 따라서,
분배된다.
동유럽의 법학자들은 이 모든 관행이
초기 슬라보니아법에 기초한 원리에서 나왔다고 주장하거니와,
그것은 가족의 재산은 영원토록 분할될 수 없다는 원리인 것이다.

\para{공동체의 다양성, 소유권 기원에 관한 문제}
우리의 탐구에서 이러한 현상들이 큰 관심의 대상이 되는 이유는
원래 공동으로 재산을 소유하던 집단 \hemph{속에서}
어떻게 해서 개별적 소유권이 발달했는가에 대해
그 현상들이 실마리를 던져주기 때문이다.
개인도 아니고,
독립된 가족도 아니고,
가부장적 모델에 기초한 더 큰 사회단위에
재산이 한때 속해 있었다고 믿을 만한 강력한 근거가 있다.
그러나
고대로부터 근대로의 소유권의 변화 양상은,
그 자체로도 모호하지만,
서로 다른 여러 형태의 \wi{촌락공동체}들이 발견되어 조사되지 않았다면
무한히 더 모호해졌을 것이다.
인도^^b7유럽 혈통의 민족들 사이에서
현재 관찰되는, 혹은 최근까지 관찰되었던,
가부장적 집단들의 내부 구조의 다양성은 충분히 주목할 가치가 있다.
스코틀랜드 산악지대의 미개한 \wi{씨족}들의 씨족장들은
그들이 관할하는 가\hanja{家}의 수장들에게
아주 짧은 간격으로, 때로는 매일,
식량을 분배해주었다고 한다.
오스트리아와 투르크의 변방 지역의 슬라보니아 촌락에서도
촌장들에 의한 주기적 분배가 행해지고 있거니와, 다만
이 경우에는 일년에 한번씩 수확물 전체를 분배하는 것이다.
하지만 러시아의 촌락에서는
재산은 불가분이라는 관념이 존재하지 않아서
개별 소유권이 성장할 수 있도록 기꺼이 허용되지만,
그러다가 일정 시간이 경과하면 소유권 분리의 진행이 단호히 중단된다.
인도에서는 공유재산의 불가분성이 부재할 뿐만 아니라,
거기서 분리된 소유권은 영구히 존속할 수 있고
무한히 많은 파생적 소유권으로 가지치기를 할 수 있지만,
재산의 분할은
뿌리깊은 관행에 의해,
그리고 동족집단의 동의 없는 이방인의 유입을 막는 규칙에 의해,
사실상\latin{de facto}
제한되고 있다.
물론,
이러한 다양한 형태의 촌락공동체들이
어디서나 똑같은 방식으로 이루어지는 진화 과정의
각 단계들을 대표한다고 주장하려는 것은 아니다.
이렇게까지 주장하기에는 증거가 부족하지만,
그러나
이들 증거에 의해,
공동체의 공동의 권리로부터
개인의 분리된 권리가 점차 해방되어나옴으로써
우리에게 익숙한 형태의 사적 소유권이
주로 형성되었을 것이라는 추측이
그나마 덜 뻔뻔한 주장이 되는 것은 사실이다.
신분법에 관한 우리의 연구로부터,
가족이 \wi{종족}\hanja{宗族}집단으로 확장되고
그후 종족집단이 개별 가\hanja{家}들로 해체된다는 것을,
그리고 종국에는
그 가\hanja{家}가 개인에 의해 대체되는 것을
알 수 있었을 것이다.
그런데
이제 우리는 이 변화의 각 단계에 상응하여
소유권의 성질도 함께 변화한다고 시사받고 있는 것이다.
이런 생각에 일말의 진실이 들어있다면,
소유권의 기원에 관한 이론가들이 널리 제기했던 문제에
그것이
중대한 영향을 줄 수 있다는 것을 알아챌 수가 있다.
그들이 주로 불러일으킨 문제---아마도 해결불가능한 것일텐데---는
사람들이 처음 서로의 점유를 존중하도록 만든 동기는 무엇이었는가? 라는 것이다.
이를 달리 표현하면---그런다고 답을 발견할 희망이 그리 많이 커지는 것은
아니지만---하나의 복합집단이 다른 복합집단의 소유물에 대해
무심해지게 되는 이유는 무엇인가라는 질문 형태로 바꿀 수 있을 것이다.
그러나,
사적 소유권의 역사에서 가장 중요한 과정이
친족의 공동소유로부터 사적 소유권이 점차 분리되어 나오는 과정이라고 한다면,
저 중대한 질문은 역사 시대의 모든 법의 초입에 놓여있는 문제, 즉
애초에 사람들을 가족의 결합으로 묶어주었던 동기는 무엇이었는가? 라는 질문과
동일한 것이 된다.
다른 학문의 도움 없이 법학만으로는 이러한 질문에 대한 답을 찾을 수 없다.
사실만 알 수 있을 뿐이다.

\para{고대의 양도 곤란성}
고대사회의 미분할된 재산 상태는
집단의 재산으로부터 하나의 지분이 완전히 분리되자마자 나타나는
특유의 날카로운 분할과 모순되지 않는다.
물론 분할이라는 현상은
그 재산이 어떤 새로운 집단의 소유물이 된다고 상정되는 상황에서
생겨나는 것이므로,
분리된 상태에서의 그것의 거래는
두 개의 대단히 복합적인 집단 간의 거래가 된다.
앞서 나는 고대법을 근대 \wi{국제법}에 비유한 바 있거니와,
그것이 다루는 권리와 의무의 주체인 단체의 크기와 복합성에 비추어 그리하였다.
고대 세계에 알려진 계약과 양도는
개인들이 당사자가 되는 것이 아니라
사람들이 조직화된 단체들이 당사자인 계약과 양도이기에,
그것은 사뭇 의례\hanja{儀禮}적일 수밖에 없다.
거기에는
참석자 모두의 기억에 거래를 각인시키기 위한
다양한 상징적 행동과 단어들이 요구되고,
또한 지나치게 많아 보이는 증인들의 참석이 요구된다.
이러한 특징들 및 기타 부수적인 특징들로부터
재산의 고대적 형태에 보편적으로 나타나는 경직성이 생겨난다.
슬라보니아의 경우처럼
때로 가족의 재산은 전혀 양도불가능하다.
좀 더 흔하기로는,
대부분의 게르만 부족법에서처럼
양도가 완전히 불법은 아니지만
수많은 사람들의 동의가 요구되어
사실상 양도가 거의 불가능한 경우도 있다.
이러한 장애물이 없는
또는 극복될 수 있는 경우에도,
미세한 잘못 하나조차 허용하지 않는 철두철미한 의례성이
양도 행위 자체에 널리 부담으로 작용한다.
한결같이
고대법에서는
아무리 이상하게 보이는 몸짓 하나라도,
아무리 그 의미가 망각된 음절 하나라도,
아무리 쓸모없어 보이는 증인 하나라도
빠뜨려서는 안 된다.
엄숙한 의례 하나하나가
그것을 수행할 법적 권리를 가진 사람들에 의해
정확하게 수행되어야 하고,
그렇지 못하면
양도가 무효가 되어,
매도인은 그가 헛되이 내주려 했던 그 권리를 그대로 가지게 된다.

\para{물건의 분류}
이용과 향유의 객체인 물건의 자유로운 유통에 대한 이러한 다양한 장애는
사회가 조금이라도 활기를 얻게 되면 즉시
고통으로 느껴지기 시작한다.
진보하는 사회가 이를 극복하기 위해 애써 강구한 수단들은
물권법의 역사의 주요 주제를 이룬다.
그러한 수단 중에
그 고대성과 보편성에 있어서 다른 것들을 능가하는 한 가지가 있다.
대다수 초기 사회에서 자생적으로 생겨난 것으로 보이는 이 관념은
바로 물건을 종류에 따라 분류하는 것이다.
어떤 종류의 물건은 다른 종류의 물건보다 낮은 가치의 지위에 놓이지만,
동시에 옛 법이 부과한 족쇄로부터 면제된다.
그후,
낮은 등급의 물건을 규율하는 양도 및 상속 규칙의 편리함이
널리 인식되고,
점진적인 혁신을 통해 낮은 가치의 물건 유형이 갖는 유연성이
전통적으로 높은 지위에 있던 물건 유형에도 전파되어간다.
로마 물권법의 역사는 \wi{악취물}\hanjalatin{握取物}{res mancipi}이
비악취물\hanjalatin{非握取物}{res nec mancipi}에 동화되어가는
역사이다.
대륙 유럽의 물권법의 역사는
봉건적 토지법이 로마법을 이어받은 동산\hanja{動産}법에 의해
대체되어가는 역사이다.
영국의 소유권의 역사는 아직 완성되지 않았지만,
\wi{인적재산}\hanjalatin{人的財産}{personalty}법이
\wi{물적재산}\hanjalatin{物的財産}{realty}법을
흡수하고 폐기시킬 공산이
농후하다.

\para{고대의 분류들, 상급재산과 하급재산}
향유의 객체인 물건의 유일한 \hemph{자연법적} 분류는,
물건의 본질적 차이에 따른 유일한 분류는,
동산\latin{movables}과 부동산\latin{immovables}의 구분뿐이다.
법학에서 이 분류는 익숙한 것이지만,
이것은 로마법에 의해 사뭇 느리게 발달하였거니와,
결국 로마법의 마지막 단계에 가서야 그것에 수용된 것을
우리가 물려받은 것이다.
고대법의 분류들은 때로 이 분류와 피상적인 유사성을 가질 뿐이다.
고대법의 물건의 종류 중에는 부동산을 포함하는 것이 있지만,
부동산과 아무 관련 없는 다수의 물건을
부동산과 함께
묶어 분류하거나,
아니면
부동산과 무척 가까운 권리들을 부동산과 따로 떼어 분류하는 경우가 흔하다.
그리하여
로마법의 \wi{악취물}은 토지뿐만 아니라 노예, 말, 소를 포함한다.
스코틀랜드법은 토지를 몇몇 다른 담보권들과 함께 분류한다.
힌두법은 토지를 노예와 함께 묶는다.
한편, 영국법은 정기\hanja{定期}부동산임차권\latin{lease of land for years}을
토지에 대한 다른 권리들과 분리하여
`부동산에 관한 인적재산'\latin{chattel real}이라 이름 하에
\wi{인적재산}에 포함시킨다.
더욱이 고대법의 분류는 상급과 하급의 우열을 나누는 분류이다.
동산과 부동산의 구분은,
적어도 로마법에 관한 한,
그러한 가치의 차이를 상정하지 않는 것이었다.
그러나
악취물은 비악취물에 비해서 처음에는 확실히 우월한 지위를 누렸다.
스코틀랜드의 세습재산\latin{heritable property}과
영국의 \wi{물적재산}도 이것들에 대비되는 인적재산에 비해 그러했다.
모든 법체계에서 법률가들은
이러한 분류를 어떤 합리적인 원리로 설명해보려는
노력을 아끼지 않았다.
그러나 구별의 근거를 법철학적으로 찾으려는 노력은 허사로 끝날 수밖에
없거니와,
그것은 철학이 아니라 역사에 속하는 문제이기 때문이다.
대부분의 경우를 포괄할 수 있을 만한 설명은,
다른 것들보다 우대받는 향유의 객체는
각 공동체의 초창기에 가장 먼저 알려져있던 유형의 물건들이었고,
따라서 \hemph{재산}\latin{property}이라는 이름으로 강조되어
불리면서 존종받았다는 것이다.
한편, 우대받는 객체에 포함되지 못하는 물건들은
상급재산의 목록이 정착되고 나서 나중에야
그 가치가 알려졌기 때문에 낮은 지위에 자리매김되었다는 것이다.
그리하여 로마법의 악취물은 가치가 큰 여러 동산들을 포함하지만,
아주 값나가는 보석들은
초기 로마인들에게 알려져있지 않았기 때문에
악취물로 분류되지 못했다는 것이다.
마찬가지로 영국법의 `부동산에 관한 인적재산'은
봉건 토지법 시대에는 그러한 부동산권\latin{estate}이 흔하지도 않았고
별 가치도 없었기 때문에 \wi{인적재산}의 지위로 떨어졌다는 것이다.
그러나
무엇보다 주목할 점은
중요성이 커지고 숫자가 늘어난 뒤에도
이들 물건과 권리들이 계속해서 낮은 지위에 머물렀다는 것이다.
왜 우대받는 향유의 객체에 계속 포함되지 못했을까?
고대법의 분류가 갖는 완고함에서
한 가지 이유를
찾을 수 있다.
무지한 사람들과 초기 사회들의 공통된 특징은
관행을 통해 익숙해진 것의 특정한 적용에 매몰되어
일반 원리를 거의 발견하지 못한다는 것이다.
그들은 일상 경험에서 만나는 특수한 사례들로부터
일반적 공준\hanja{公準}을 분리하지 못한다.
그리하여,
잘 알려진 유형의 물건에 대한 명칭을
그것과 정확하게 닮은 향유의 객체이자 권리의 대상인 물건에
붙이기를 거부하는 것이다.
그러나
법의 힘만큼이나 안정적인 힘을 대상에 가하는
이러한 영향력 외에도,
그후
계몽주의와 일반적 공리\hanja{功利} 개념의 진보에 유사한
다른 영향력이 더해졌다.
법원과 법률가들은
우대받는 물건의 양도, 회수, 상속에 필요한 성가신 형식요건들의
불편함에 마침내 눈을 뜨게 되어,
새로운 유형의 물건들에
유년기 법의 특징인 법기술적 속박을 씌우기를 점점 꺼리게 된다.
그리하여
법학 체계에서
이 후자의 것들을 계속 낮은 등급에 머물러둠으로써,
그것의 양도가
신의성실에 걸림돌이 되고 기망행위에 디딤돌이 되는
옛 양도방식보다 간편한 과정으로 이루어질 수 있도록
하려는 경향이 일어난다.
우리에게는 고대 양도방식의 불편함을 과소평가하는 위험이 있는 것 같다.
우리의 양도수단은 서면에 의한 것이기에,
전문가에 의해 신중하게 작성되면 그 문언에는 흠결이 별로 없다.
그러나 고대의 양도는 서면이 아니라 \hemph{행위}에 의한 것이었다.
몸짓과 말이 서면의 법기술적 문언을 대신했고,
공식\hanja{公式}을 조금이라도 잘못 발음하면,
상징적 행위를 하나라도 빠뜨리면,
그 절차는 치명적인 흠결이 있는 것으로 간주되었다.
마치 2백년 전\footnote{%
  1677년 사기방지법(Statute of Frauds) 제정 이전을 말하는 듯하다.
  이 법률로 부동산권의 양도나 임대차, 일정 유형의 계약이나 유언 등은
  서면으로 행하지 않으면 효력을 인정받을 수 없게 되었다.
  }
영국에서 유스\latin{use}\footnote{%
  유스---사용수익(benefit)의 뜻이다---는 원래 보통법에서는
  효력이 인정되지 않았다. 그러나 형평법법원이 이를 보호하기 시작하자,
  이를 이용한 각종 탈법을 막기 위해
  유스금지법(Statute of Uses, 1536)이 제정되었다. 이로써
  ``to A, to the use of B'' 같은 부동산권 설정에서 B는 동법에 의해
  완전한 보통법상의 권리를 얻게 된다.
  그러나 동법의 헛점을 파고든, 가령
  ``to A, to the use of B, to the use of C'' 같은 경우의 C에게
  이후 형평법법원에 의한 보호가 다시 주어지고
  C는 형평법상의 소유자로 불리게 된다.
  그런데 이러한 설정은 흔히 `신탁'(信託\,trust)이라 부른다.
  그렇다면 본문은 이것이 아니라 유스금지법에 의해 보통법적 효력이
  인정되는 유스, 가령
  ``to A and his heirs, to the use of B for his life'' 따위를
  말하는 것이 아닐까 한다.
}의 진술이나
잔여권\hanjalatin{殘餘權}{remainder}\footnote{%
  가령 ``to A for life, and then to B and his heirs''라며
  부동산을 양도할 때,
  B가 갖는 장래의 권리를 `잔여권'이라 부른다.
}의 설정에
중대한 실수가 있으면 날인증서가 무효로 되었던 것처럼 말이다.
사실, 이것으로는 원시적 의례적 절차가 갖는 문제점을 절반만 말한 것에 불과하다.
서면이든 행위든 복잡한 양도요건이 \hemph{토지}의 양도에만 요구되는 한,
급하게 거래할 일이 적은 유형의 재산의 양도인지라
실수할 가능성은 그리 크지 않다.
그러나 고대 세계의 상급재산의 범주에는
토지뿐만 아니라 몇몇 아주 평범한, 그리고 몇몇 아주 값나가는 동산들이
들어있었다.
사회의 수레바퀴가 빠르게 굴러가기 시작하자,
말이나 소의 양도에, 혹은
고대 세계에서 가장 값나가는 동산---노예---의
양도에 대단히 복잡한 방식을 요구하는 것은
큰 불편을 초래하였을 것이 틀림없다.
분명
이러한 물건들을 불완전한 방식으로 양도하는 일이,
따라서 불완전한 권리가 보유되는 일이
지속적으로, 심지어 일상적으로 벌어졌을 것이다.

\para{악취물과 비악취물}
옛 로마법에서 \wi{악취물}은 토지---역사 시대에는 이탈리아의 토지---와
노예와 짐을 끄는 가축, 예컨대 말이나 소를 의미했다.
의심할 여지 없이 이러한 유형의 객체는 농사를 짓기 위한 주요 수단이었고,
원시시대의 사람들에게는 무엇보다 중요한 물건이었을 것이다.
처음에는 이들 물건이 `재물' 또는 `재산'이라며 강조하여 불렸을 테고,
이들을 양도하는 방식이 바로 \wi{악취행위}\latin{mancipium; mancipation}였으나
이들을 ``악취행위가 요구되는 물건''이란 뜻의
`악취물'이라는 명칭으로 부르게 된 것은 나중에 가서였을 것이다.
그런데 이들 외에도,
악취행위의 복잡한 의례를 전부 거칠 필요는 없다고 생각되는
유형의 객체들이 존재했거나 발달하게 되었을 것이다.
이 후자의 물건들의 소유권 양도에는
통상적으로 요구되는 형식적 요건 중 일부,
즉 현실적인 교부, 물리적인 이전만 행해지면 충분하다고 생각되었다.
이것이 바로 \hemph{\wi{인도}}\hanjalatin{引渡}{tradition}이거니와,
이는 소유권 변동의 가장 명백한 지표인 것이다.
이런 물건들을 고법\hanja{古法}에서는
``악취행위가 필요치 않은 물건''이란 뜻의
`비악취물'이라고 불렀으니,
처음에는 그다지 값어치 없는 것들이었고
한 집단에서 다른 집단으로 양도될 일도 별로 없는 것들이었을 것이다.
하지만, 악취물의 목록은 전적으로 폐쇄적이었으나,
비악취물의 목록은 개방적이었고 무한히 확장되어갔다.
그리하여 인간이 물질적 자연을 하나씩 정복해감에 따라
비악취물은 항목이 하나씩 늘어나거나
기존의 항목에 개선이 이루어졌다.
결과적으로 부지불식간에
이들이 악취물과 동등한 가치를 갖는다고 여겨지고, 따라서
본질적으로 낮은 등급의 물건이라는 인상이 사라져가면서,
사람들은 복잡하고 장엄한 의례절차보다
이들의 양도에 수반되는 간편한 요건이 여러 모로 장점을 갖는다는 것을
알아채기 시작했다.
로마 법률가들은
법 개선의 두 가지 장치,
즉 \wi{법적의제}\latin{fiction}와 \wi{형평법}\latin{equity}을
열심히 활용하여
\wi{인도}에 사실상 악취행위와 동일한 효과를 부여하려했다.
비록 로마의 입법자들은
단순한 교부에 의해
악취물의 소유권이
즉시 이전한다는 입법에는
오랫동안 몸을 사려왔으나,
마침내 유스티니아누스에 의해 이러한 조치가 단행되어
\wi{악취물}과 비악취물의 차이가 사라졌고,
인도는 법이 인정하는 유일한 양도방식이 되었다.
로마 법률가들이 일찍부터
인도를 높이 평가하였기 때문에
근대 법학자들은 그것의 진짜 역사를 잘 모르는 경향이 있다.
\wi{인도}는 ``\wi{자연법}적'' 취득방식으로 분류되었거니와,
그것이 이탈리아 부족들 사이에서 널리 행해지는 방식이었을 뿐만 아니라,
또한 물건을 취득하는 가장 단순한 방식이었기 때문이다.
로마 법학자들의 표현들을 미루어 짐작하건대,
그들은 자연법에 속하는 인도가 \wi{시민법}상의 제도인 악취행위보다
더 오래된 것이라고 생각했을 것이 분명하다.
이것은, 말할 것도 없이, 진실과 정반대인 것이다.

\para{세습재산과 취득재산, 부동산과 동산}
악취물과 비악취물의 구분은 문명 세계가 크게 빚지고 있는 구분 유형이거니와,
모든 물건을 그중 일부는 그 자체로 가치있는 것으로 자리매김하고
다른 것들은 낮은 등급의 범주에 집어넣는 구분 방식이다.
멸시받고 무시당한
낮은 등급의 물건들은
원시법이 즐겨 채용한 복잡한 의례절차에서 처음 면제된 것들이었으나,
그후 지성의 상태가 진보하자
단순한 방법의 이전 및 회복 방식이 널리 사용되었고 이는
그 편리성과 단순성으로 인해
고대로부터 내려온 성가신 의례절차를 폐기하는 데 모범으로 작용했다.
그러나 몇몇 사회에서는
`재산'을 얽어매는 질곡이 무척 복잡하고 강고해서,
그것이 그리 쉽게 완화되지 못하는 경우가 있다.
인도에서 남자 아이가 출생하면,
전술했듯이,
인도의 법은 그에게 재산권을 완전히 부여하고,
재산의 양도에 그의 동의가 필수요건이 된다.
마찬가지 정신에서,
게르만 민족들의 일반적 관습---앵글로색슨의 관습이 예외였던 것은
주목할 만하다---은
아들들의 동의 없는 양도를 금지했다.
또한 슬라보니아의 원시법은 양도 자체를 아예 금지했다.
분명, 이러한 장애가
모든 종류의 물건에 확대적용되는 것인 한, 이는
물건의 종류를 구분함으로써 극복될 수 있는 장애가 아니었다.
따라서, 일단 진보의 물결이 일어나자,
고대법은 또 다른 성격의 구분을 행함으로써 장애에 대처했거니와,
이는 물건의 성질이 아니라 물건의 기원에 따라 구분하는 분류인 것이다.
두 가지 분류체계의 흔적을 다 가지고 있는
인도에서는, 지금 우리가 다루는 분류를
세습재산\latin{inheritances}과
취득재산\latin{acquisitions} 간의 힌두법상의 분류에서 볼 수 있다.
아버지의 세습재산은 자식들이 태어나자마자 그들과 공유된다.
그러나 대부분 지방의 관습에 따르면
아버지가 그의 생애 동안 획득한 취득재산은 전적으로 그의 것이며
그는 마음대로 그것을 양도할 수 있다.
유사한 구분이 로마법에서도 없지 않거니와,
가부장권에 대한 최초의 혁신은
아들이 군복무 중 취득하는 재산은 아들 자신의 것으로 삼도록
허용하는 형태를 취했던 것이다.
그러나 이러한 방식의 분류를 가장 광범위하게 사용한 것은
게르만인들이었을 것이다.
누차 언급했듯이,
\wi{자유소유지}\latin{allod}는 양도불가능은 아니었지만
대체로 양도가 무척 어려웠다.
더욱이 종친\hanja{宗親}들만이 그것을 상속받을 수 있었다.
그리하여 여러 특별한 분류방식이 인정되기에 이르렀으니,
그 모두가 자유소유지와 불가분 결합된 불편함을 줄이려는 것이었다.
예컨대,
게르만법의 큰 부분을 차지하는
속죄금\hanjalatin{贖罪金}{wehrgeld}, 즉
친족이 살해되어 받은 배상금은
가족재산의 일부를 구성하지 않았기에
전혀 다른 상속규칙에 따라 상속되었다.\footnote{%
  렉스 살리카(Lex Salica) 제59장에 따르면 자유소유지는
  직계비속--부모--형제자매--고모--이모--최근친 부계혈족 순으로 상속한다.
  하지만 제62장에서는
  살인에 기한 속죄금은 망자의 자식들이 그 절반을 가져가고,
  나머지 절반은 부계와 모계의 최근친 혈족들이 나누어가진다고 규정한다. }
마찬가지로,
과부가 재혼할 때 부과되는 벌금인
레이푸스\latin{reipus}도
지불받는 자의 자유소유재산에 속하지 않았고,
따라서 그것의 상속에서는 종친의 특권이 무시되었다.\footnote{%
  렉스 살리카 제44장에 의하면, 과부와 혼인하고자 하는 남자는
  소집된 법정에서 3솔리두스를 지불해야 한다. 특이하게도
  이 벌금은 죽은 남편의 여계혈족에게 귀속되었다.
  1순위는 전남편의 누이의 장자, 즉 조카(생질)이고,
  2순위는 조카딸의 장자,
  3순위는 이모의 아들,
  4순위는 모계 사촌의 아들,
  5순위는 전남편의 어머니의 남자형제, 즉 외삼촌이며,
  그 다음으로 비로소 남계혈족에게 넘어간다. }
또한, 인도인들과 마찬가지로,
게르만법도
가\hanja{家}의 수장의 취득재산을 그의 세습재산과 구별하여,
취득재산은 그가 보다 자유롭게 처분할 수 있었다.
다른 유형의 분류들도 인정되었거니와,
가장 친숙한 것은 부동산과 동산의 구분일 것이다.
그러나 동산은 몇 가지 하위범주로 다시 세분되었고
각각에는 서로 다른 규칙이 적용되었다.
이렇게 분류가 많은 것은,
로마제국을 정복한 게르만인들의 미개한 특성이라고 느껴질 수도 있겠으나,
실은
로마의 국경 부근에서 오랫동안 체류하는 동안
로마법적 요소가 그들 법에 상당히 많이 유입되어 들어갔다는 것으로
설명할 수 있을 것이다.
\wi{자유소유지}를 제외한 물건의 양도와 상속을 규율하는 법규칙의 대부분은
로마법에서 유래한 것으로 그 기원을 어렵지 않게 추적할 수 있거니와,
그것들은 아마도 장기간에 걸쳐
조금씩 로마법에서 빌려왔을 것이다.
재산의 자유로운 유통에 대한 장애가
이러한 방법들로
얼마는 극복되었을 것인가는 우리로서는 추측조차 할 수 없으니,
근대사에서는 내가 말한 그러한 구분들이 존재하지 않기 때문이다.
전술했듯이,
자유소유 형태의 재산은 \wi{봉건제}의 와중에서 망각되었고,
봉건제가 완전히 공고화된 이후에는
서구 세계에 알려졌던 모든 구분 중에 오직 한 가지 구분만이
사실상 남게 되었다.
부동산과 동산의 구분이 그것이다.
표면적으로 이 구분은 로마법이 마침내 채택한 그것과 동일한 것이었으나,
중세의 법은 부동산을 동산보다 훨씬 높게 평가했다는 점에서
로마법과 달랐다.
하지만 이 한 가지 예만 가지고도
이를 포함하는 분류 장치들의 중요성을 보여주는 데 충분하다.
프랑스 법전들에 기초한 법체계를 가진 모든 나라들에서는,
다시 말해 대륙 유럽의 대부분의 지역에서는,
언제나 로마법적이었던 동산법이 봉건 토지법을 대체하고 무효화시켰다.
주요국 중에
이러한 변화가 어느 정도 진행되었으나 아직 완성되지 못한
유일한 국가가 바로 영국이다.
게다가 주요 유럽 국가 중에 영국은
자연법적으로 용인되는 유일한
분류\footnote{%
  동산과 부동산의 구분을 말한다.
}로부터
고대법의 분류를 이탈하게 만들었던
바로 그 영향력에 의해
동산과 부동산의 구분이
상당 정도 방해받은
유일한 국가인 것이다.
영국법의 구분도 대체로 부동산과 동산에 일치하지만,
어떤 종류의 동산은 세습동산\latin{heirloom}으로서 부동산과 함께 상속되고,
어떤 종류의 부동산권은 역사적 이유에서 \wi{인적재산}으로 분류되어왔다.
영국법이
법발달의 주류와 동떨어져
고법\hanja{古法}의 현상을 재현한 예는 이것만이 아니다.

\para{시효의 이론들}
소유권에 대한 고대적 질곡을 비교적 성공적으로 완화시킨 장치들 가운데
한 두 가지를 더 다루고자 한다.
그러나 본 저서의 구도상 아주 오래된 것들만 언급하는 것을
양해해 주시길 바란다.
그 중 하나는 조금 자세히 들여다볼 필요가 있거니와,
초기법의 역사를 잘 모르는 사람들에게는
근대법이 아주 천천히 그리고 대단히 어렵게 겨우 인정하게된
어떤 원리가 유년기의 법학에는 실로 친숙한 원리였다는 것이
쉽게 믿기지 않을 것이기 때문이다.
법의 원리들 중에
근대인들이 그 유용성에도 불구하고
수용하기를 꺼리고 그 합당한 결론들을 관철시키기를 꺼린
원리로
로마인들이 `\hypertarget{usucapio}{사용취득}'\hanjalatin{使用取得}{usucapion}이라고
불렀고
근대법에서는
이것을 물려받아
`취득시효'\latin{prescription}라고 부르는 제도에
비할 만한 것이 또 있을까 싶다.
가장 오래된 로마법 규칙으로
\wi{12표법}보다도 더 오래된
이 실정규칙은
일정 기간 동안 중단 없이 점유상태가 지속된 물건은
그 점유자의 소유로 된다는 규칙이었다.
점유의 기간은 대단히 짧았으며---물건의 성질에 따라
1년 또는 2년\footnote{%
  부동산은 2년, 동산은 1년.
}---역사
시대에는 특정한 방식으로 점유가 개시된 경우에만 작동이 허용되었다.\footnote{%
  정당한 원인(iusta causa, 가령 매매)에 의해, 그리고
  선의로(bona fide, 가령 매도인이 무권리자임을 모르는 상태)
  점유가 개시되어야 한다.
  단, 도품(盜品)이나 강탈된 물건은 대상에서 제외된다. }
그러나 생각건대 그전에는
지금 우리가 전거들에서 보는 것보다 훨씬 덜 엄격한 요건으로
점유가 소유권으로 전환되었던 것 같다.
전술했듯이,
나는 사실상의 점유에 대한 사람들의 존중이
법학 그 자체만으로 설명될 수 있는 현상이라고 주장하지 않는다.
단지 사용취득의 원리를 채택함에 있어 원시사회는
근대인들 사이에 그것의 수용을 방해했던
어떤 사변적인 의심이나 망설임을 전혀 갖지 않았음을 말하고 싶을 뿐이다.
근대 법률가들 사이에서 취득시효는
처음에는 반감의 대상이었고
나중에는 어쩔 수 없이 수용하는 것으로 여겨졌다.
영국을 포함한 몇몇 나라에서는
과거 특정 시점---일반적으로 몇몇 선임 국왕들의 치세 원년---이전에 입은
손해에 기해서 제소가
이루어지는 것을 막는 것 이상으로
입법이 나아가지를 못했다.
중세가 마침내 마감하고
제임스 1세가 영국왕에 오른 이후에
비로소
사뭇 불완전한 것이었으나 진정한
출소기한법\hanjalatin{出訴期限法}{statute of limitation}이
제정되었다.
대다수 유럽 법률가들이 계속 읽어왔음이 분명한
로마법의 가장 유명한 분야 하나를 근대 세계에 재현하는 데
이렇게 오래 걸린 것은
무엇보다 \wi{교회법}의 영향 탓이다.
교회법의 기원이 된 교회의 관습은
성스러운 권리 또는 그에 준하는 권리로 여겨지는 것을 취급하므로,
교회가 인정한 특권은
아무리 오랫동안 불사용\hanja{不使用}되더라도
상실될 수 없다고
자연스레
간주되었다.
이런 견해에 따라,
후대의 안정된 교회법도 취득시효를 배척하는 특징을 나타냈다.
교회법학자들에 의해 세속 입법이 따를 본보기로 치켜세워지자
교회법이 세속 입법의
핵심 원리들에 특유의 영향을 미친 것은 당연한 일이었다.
유럽 전역에 걸쳐 형성되어가던 \wi{관습법} 체계들에게 교회법은
비록 로마법보다 적은 수의 명시적 법규칙들만 가져다 주었지만,
놀랄 정도로 많은 근본적 문제에 관하여
전문가들에게 어떤 선입견을 심어준 것으로 보이며,
이렇게 해서 형성된 경향은 각 법체계가 발달하면서 점점 강화되었다.
그렇게 만들어진 성향 중 하나가 취득시효에 대한 혐오였다.
그러나 만약
세속 세계의
스콜라주의 법학자들\footnote{%
  주해학파(Commentators)를 말하는 듯하다.
}의
법리와 일치하지 않았다면
그러한 편견이
과연 그렇게 강력하게 작용했을 것인지는 의문이다.
이들의 가르침에 따르면,
실제 입법이 아무리 반복되더라도,
\hemph{권리}는 아무리 오래 방치되어도
사실상 파괴될 수 없는 것이었다.\footnote{%
  여기서 `입법'은 제국 아래 영방이 주권을 사실상 행사하는 것을,
  `권리'는 황제의 주권을 뜻한다고 읽으면 혹시 이해에 도움이 될지도 모르겠다.
  }
이런 상황의 유산은 오늘날까지도 남아있다.
법철학이 진지하게 논의되는 곳이라면 어디서나
취득시효의 사변적 기초에 관한 문제는 항상 열띤 논쟁의 대상인 것이다.
여전히 프랑스와 독일에서는,
수년간 계속해서 점유하지 않은 자가
오랫동안 방치한 벌로
소유권을 박탈당할 수 있느냐,
또는 소송의 종료\latin{finis litium}를 바라는 법의 개입만으로
소유권을 상실할 수 있느냐가
큰 관심의 대상이 되고 있다.
그러나 초기 로마 사회의 사람들은 이러한 망설임으로 구애되지 않았다.
그들의 고대 관행은
일정 조건 하에 1년이나 2년 동안 점유를 상실한 자로부터
바로 소유권을 빼앗았다.
초기 형태의 사용취득 규칙이 정확히 어떤 취지에서 만들어진 것인지는
알기 어렵다.
그러나 전거들에 나타난 사용취득의 조건들을 살펴보건대,
그것은 지나치게 복잡한 양도방식의 해악에 대한 사뭇 유용한 안전장치였음이
드러난다.
사용취득이 인정되려면
적대적 점유가 선의\hanjalatin{善意}{good faith}로,
즉 점유자가 그 물건을 합법적으로 취득한다고 믿으면서
시작되어야 한다.
또한 그 물건이
당해 사안의 권리양도에 필요한 완전한 양도방식은 아니더라도
적어도 법이 인정하는
어떤 양도방식을 통하여 그에게 이전되었어야 한다.
따라서 \wi{악취행위}가 요구되는 사안에서
아무리 절차가 날림으로 수행되었더라도
적어도 \wi{인도}\hanja{引渡}가 행해졌다면,
길어야 2년이면 사용취득에 의해 권리의 하자가 치유되는 것이다.
로마인들의 관행 중에 사용취득의 관행보다
그들의 법적 천재성을 강하게 입증하는 것은 없다고 나는 생각한다.
로마인들을 괴롭힌 문제는 영국 법률가들을 괴롭혔고
지금도 괴롭히고 있는 문제와 거의 동일한 것이었다.
그들이 재건축할 용기도 힘도 아직 갖고 있지 않은
영국법의 복잡성 탓에,
실질적 권리와 법기술적 권리가,
\wi{형평법}적 소유권과 보통법적 소유권이,
계속 분리된 상태로 남아있다.
그러나 로마 법학자들이 솜씨있게 요리했던 사용취득은
소유권의 흠결이 계속해서 치유되어가는,
소유권들 간에 일시적 분리가 생기더라도
최소한의 지체 후에는 다시 재빨리 결합되어가는,
일종의 자동기계 같은 것이었다.
사용취득은 유스티니아누스의 개혁 전까지
그 장점을 잃지 않았다.
그러나 시민법과 형평법이 완전히 통합되자마자,
악취행위가 로마법의 양도방식이기를 그치자마자,
고대적 장치는 더 이상 필요하지 않게 되었다.
기나긴 생애를 마감한
사용취득은
이제
취득시효가 되었거니와,
이것이 마침내 거의 모든 근대 법체계에 수용된 것이다.

\para{법정양여}
방금 살펴본 것과 동일한 목적을 갖는 또 다른 수단 하나를
간단히 언급하고자 한다.
그것은 영국법사에서는 초기부터 등장한 것이 아니었으나,
로마법에서는 먼 옛날부터 있었던 오래된 제도이다.
영국법을 유추하여 이 제도를 조명할 능력이 부족한
독일의 몇몇 로마법 학자들은 심지어 악취행위보다도 오래되었다고
생각할 정도로 오래된 제도이다.
내가 말하려는 것은 바로
양도하려는 물건을
법정양여\hanjalatin{法廷讓與}{cessio in jure}하는 것으로,
일종의 공모회수소송\latin{collusive recovery}이다.
원고는 통상적인 소송 방식을 통해 객체에 대한 권리를 주장하고,
피고는 불출석한다. 그러면 당연히 판결을 통해 그 물건의 권리는
원고에게 주어진다.
영국 법률가들에게는
이 수단이 그들 선조들에게 어떤 의미를 가졌는지 굳이 설명할 필요가 없을 것이다.
봉건 토지법의 가혹한 질곡에서 벗어나기 위해
영국인들은
저 유명한
\wi{종국화해}\latin{fine}와 \wi{공모회수소송}\latin{recovery}을 만들어냈던
것이다.\footnote{%
  종국화해(final concord)와 공모회수소송(common recovery)에 대해서는
  본서 제6장 \hyperlink{finerecovery}{고대 유언의 비서면성 부분}의 각주 참조.
  }
이들 로마법의 장치와 영국법의 장치는 공통점이 많고
사뭇 유익한 사례를 서로에게 제공한다.
그러나 그것들 사이에는 차이점이 있으니,
영국법의 장치는
이미 취득한 권리에 붙어있는 골칫거리를 제거하는 데 목적이 있었던
반면,\footnote{%
  간단한 예를 들자면,
  가령 한정승계부동산권(fee tail)을 단순부동산권(fee simple)으로
  만드는 데 사용될 수 있었다.
  }
로마법의 장치는
자칫 잘못 수행하기 쉬운 양도방식을 대신하여
일종의 탄핵불가능한 양도방식을 제공함으로써 골칫거리를 미리 막는 데
목적이 있었다.
사실 이 장치는 법원이 안정적으로 작동하는 단계에 이르면
언제든 등장할 수 있지만,
그래도 역시 원시적 관념의 제국에 속하는 것이다.
진보된 상태의 법에서는 법원이
공모소송을 소권\hanja{訴權}의 남용이라고 간주한다.
그러나 방식만 정확히 갖추어진다면
그 이상은 따지지 않는 그런 시절이 반드시 있었던 것이다.

\para{소유권과 점유권}
법원과 소송절차가 물권법에 끼친 영향은 광범위한 것이었지만,
이 주제는 너무나 방대해서 본서에서 다 다룰 수 없을 뿐만 아니라
본서의 기획보다 훨씬 더 많이 법사\hanja{法史}를 거슬러내려가야 한다.
하지만 소유권과 점유권이라는 중요한 구분이 이 영향에서
유래하는 것임은 언급해둘 필요가 있겠다.
사실, 구분 그 자체---이는
\paren{어느 저명한 영국의 로마법 학자의 말에 의하면}
물건에 대하여 결정할 법적 권리와
그렇게 할 사실적\latin{physical} 권리 간의 구분과
같은 것이라 한다---가 아니라,
이 구분이 법철학에서 가지는 특별한 중요성에 대해 말하려는 것이다.
교양있는 사람이라면 법문헌을 별로 읽어보지 않았더라도,
점유라는 주제에 관한
로마 법학자들의 언어가 오랫동안 큰 혼란을 만들어냈다는 것,
그리고 사비니의 천재성이 이 수수께끼를 해결한 데서
주로 입증되었다는 것을
들어본 적이 있을 것이다.
실로 로마 법률가들은
쉽게 설명할 수 없는 다양한 의미로
점유라는 말을
사용했던 것으로 보인다.
점유라는 말의 어원을 따져보면, 이 말은 본디
물리적 접촉 또는
원한다면 언제든 회복할 수 있는 물리적 접촉을 뜻했음이 거의 확실하다.
그러나 실제로 사용될 때는,
수식어가 따로 붙지 않는 한,
그것은 단순한 물리적 소지\hanjalatin{所持}{detention}가 아니라,
물리적 소지에 덧붙여
물건을 자기 것으로\latin{as one's own} 보유하려는 의사가 결합된 것을
의미한다.\footnote{%
  사실적 소지로 점유를 파악하는 우리 민법의 점유 개념과 다름에 유의할 것.
  }
니부르의 견해를 수용한 사비니는
이러한 특이한 개념이 오로지 역사의 산물이라고 보았다.
사비니에 따르면,
명목상의 차임\hanja{借賃}만 내면서
국유지의 상당 부분을
보유하게 된
로마의 귀족시민들은
옛 로마법에 의하면 단순한 점유자에
불과했지만, 그러나
그들의 점유는
모든 도전자들을 상대로 자기 땅을 지키고자 하는 의사를 가진 점유였다.
실로 그들의 주장은 최근 영국에서 교회토지의 임차인들이 내세운 주장과
거의 동일한 것이었다.
이론적으로 보면 그들은 국가의
임의\hanja{任意}부동산임차인\latin{tenants-at-will}\footnote{%
  기한의 보장이 없이
  당사자 일방의 의사에 의해 언제든지 종료될 수 있는 부동산임대차의 임차인.
}에 불과했으나,
그들은 시간의 경과와 평온한 향유로
자신들의 보유가 일종의 소유권으로 성숙하였으며
토지의 재분배를 위해 자신들을 퇴거시키는 것은 부당하다고 주장했다.
이러한 주장은 귀족들이 토지를 보유한다는 점과 결합하여
``점유'' 개념에 항구적인 영향을 끼쳤다.
이 경우
토지보유자들이
퇴거당하거나 방해의 위협을 받을 때
이용할 수 있는 법적 구제수단은
점유보호특시\hanja{特示}명령\latin{possessory interdicts}이 전부였다.
로마법의 약식절차였던 점유보호특시명령은
\wi{법무관}이 그들을 보호하기 위해 특별히 고안해낸 수단이었거나,
또는 다른 이론에 의하면
법적 권리를 다투는 동안 임시로 점유를 유지하도록 옛날부터 사용되어온
수단이었다.
그리하여 물건을 \hemph{자기 것으로}\latin{as his own}
점유하는 모든 이들이 이 특시명령을 신청할 자격이 있다고
여겨지게 되었고,
사뭇 특별한 변론 절차에 기초하여
이 특시명령 절차는 점유를 둘러싼 분쟁을 재판하는 데 적합한 형태를
갖추어나갔다.
그러자, 존 \wi{오스틴}\latin{John Austin} 씨가 지적한대로,
영국법에서도 똑같이 반복된 어떤 흐름이 시작되었다.
소유권자들\latin{domini}이
지루하고 복잡한 대물소송\hanjalatin{對物訴訟}{real action} 대신에
간편하고 신속한 특시명령 절차를 선호하기 시작한 것이다.
점유권적 구제수단을 이용하려는 목적에서
그들은 소유권에 내포되어 있는 점유를 원용하기 시작한 것이다.
협의의 점유권자가 아닌 소유권자인 사람들에게
점유권적 구제수단에 의한 권리 주장을 허용한 것은,
처음에는 은혜로운 일이었을지 몰라도,
결국에는 영국법과 로마법 모두에 심각한 퇴행을 낳았다.
로마법은
이로 인해 생긴
점유 개념을 둘러싼 온갖 복잡미묘한 주장들로
불신을 초래했으며,
영국법은
\wi{물적재산}\latin{real property}의 회수를 위한 소송들이
절망적 혼란상태에 빠져들자
결국 영웅적인 결단으로
저 혼란한 덩어리 전체를 잘라내버렸던 것이다.
누구도 거의 30년 전에 단행된
영국 물적소송\hanjalatin{物的訴訟}{real action}의 사실상의 폐지\footnote{%
  1833년 물적재산출소기한법(Real Property Limitation Act).
  물적소송의 출소기한(이 법률에서는 20년, 그후 12년으로 단축된다)만
  정한 것이 아니라, 본문에서 언급하고 있듯이
  각종 부동산권 소송을 부동산점유회복소송(ejectment)으로 거의 단일화했다.
}가 공공에 이익이었다는 점을 의심할 수 없을 것이다.
그러나 여전히
법학의 조화를 중시하는 사람들은
물적소송을
정비하고 개선하고 단순화하는 대신에
물적소송 전부를 부동산점유회복소송\latin{ejectment}\footnote{%
  부동산에 관한 것이지만 법적으로는 인적소송(personal action)에
  해당한다. 인적재산인 정기부동산임차권(term of years)의 임차인이
  퇴거당했을 때 성립하는 소송이었기 때문이다.
  지금은 널리 부동산 일반의 점유 회복을 위한 소송이 되었다.
}에 갖다바친 것은
부동산 회수 소송의 체계 전체를 \wi{법적의제}에 기초하게 만든 것이라며
한탄할 것이다.

\para{형평법상의 소유권}
또한 법원은
재판권 간의 최초의 구별일 수밖에 없는
보통법과 \wi{형평법} 간의 구별을 통해
소유권 개념을 형성하고 수정하는 데
크게 기여했다.
영국의 형평법상의\latin{equitable} 소유권은 단지
형평법법원\latin{Court of Chancery}의 재판권에 의해 인정된 소유권일 뿐이다.
로마에서는
\wi{법무관}의 고시\latin{edict}에 의해,
일정한 경우 일정한 소송이나 신청을 허용하겠다는 약속의 형태로
새로운 원리들이 도입되었다.
따라서 `법무관법상의 소유물'\latin{property \textit{in bonis}},
다시 말해 로마법의 형평법상의 소유권은
고시에서 기원한 구제수단들에 의해서만 보호되었다.
형평법상의 권리가 보통법상의 소유권에 의해
무효화되지 않게 된 방법에는 양 법체계가 다소 차이가 있다.
영국에서 그것의 독립성은 형평법법원의
금지명령\latin{injunction}에 의해 보장되었다.
하지만 로마법에서는
시민법과 형평법이
아직 완전히 통합된 것은 아니나
동일한 법원에 의해 다루어졌기에,
금지명령 같은 것이 필요치 않았고
정무관은 보다 간단한 방법을 사용할 수 있었다.
그 방법은 형평법상 타인에게 속하는 물건을 회수하려는
\wi{시민법}상의 소유자에게 정무관이 소송이나 신청을 허가해주지 않는 것이었다.
그러나 양 법체계의 실제 작동은 거의 동일했다.
양자 모두 절차의 구별을 통해,
나중에 법 전체에 의해 인정될 때까지,
일종의 잠정적인 존재로 새로운 소유권 형태를
보호할 수 있었다.
그리하여
로마 법무관은
단순한 \wi{인도}로 \wi{악취물}을 취득한 자에게
사용취득 기간이 완성되기를 기다리지 않고
즉각 소유권을 부여했다.\footnote{%
  이른바 푸블리키우스 소권(actio Publiciana).
  사용취득을 의제하여 소권을 부여했다. }
마찬가지로 때로 그는
애초 ``보관인''\latin{bailee} 또는
수치인\hanjalatin{受置人}{depositary}에 불과했던
질권자\hanjalatin{質權者}{mortgagee}에게,\footnote{%
  이른바 세르비우스 소권(actio Serviana).
  농지 임대인이 차임의 담보물에 대해 갖는 대물소권이다.
  \latin{Inst.\,4.6.7.}}
그리고 정액지료\hanja{定額地料}를 정기적으로 내는 영구적 토지임차인인
영차권자\hanjalatin{永借權者}{emphyteuta}에게
일종의 소유권을 인정했던 것이다.
유사한 진보 과정을 거쳐,
영국의 형평법법원도
양도저당권\hanja{讓渡抵當權}설정자\latin{mortgagor}에게,\footnote{%
  우리의 저당권 제도와 달리 양도저당(mortgage)에서는
  채권자인 저당권자에게 담보물의 부동산권이 귀속한다. }
신탁\hanja{信託}의 수익자\latin{cestui que trust}에게,
특정한 유형의 재산설정보다 우선하여 기혼여성에게,\footnote{%
  혼인한 여자의 재산은 법적으로 남편의 것이었으나,
  차츰 아내에게 일정한 재산권을 부여하는 재산설정이 가능하게 되었다.
  그런데 가령 남편의 채무를 아내의 재산으로 변제하도록 하는 재산설정 같은 것이
  문제되자, 형평법이 나서서 그러한 설정의 효력을 제한한 것이다.
  }
완전한 보통법적 소유권을 취득하지 못한 매수인에게,\footnote{%
  매매계약이 체결되면
  아직 양도(날인증서의 작성 및 교부, 오늘날은 등기)를 경료하지 않았더라도
  매수인이 형평법상의 소유권자가 된다. }
특수한 소유권을 부여했다.
이 모두는 분명 새로운 형태의 소유권을 인정하고 보호한 사례들이다.
그러나
간접적으로는
영국의 물권법도 로마의 물권법도
형평법에 의해 수천 가지 방법으로 영향받았다.
법학자들이 구사하는 강력한 수단이
법학의 어떤 분야에든
밀고 들어가면,
법학자들은
반드시
물권법을 만나고, 건드리고, 어느 정도 변경하게 되어있다.
지난 몇 페이지에 걸쳐
내가
이런저런 고대법적 구별과 수단들이
소유권의 역사에 큰 영향력을 행사했다고
말했다면,
그것은
법학자들이
시대정신에
불어넣은 개선의 힌트와 제안들이
형평법 체계의 담당자들에 의해 호흡되어
그 영향력의 대부분이 생겨났다는
의미였다고 이해해주시길 바란다.

\para{로마법과 봉건법, 만족들의 법전}
그러나 소유권에 대한 형평법의 영향력을 기술하는 것은
현대에 이르기까지의 역사를 기술하는 것이 될 터이다.
앞서 내가 이런 말을 잠깐 언급한 이유는,
오늘날의 몇몇 저명한 학자들에 따르면,
로마제국의 법과 중세의 법이 소유권의 개념에서 차이를 보이는 것의 단서를
로마인들이 형평법상의 소유권과 \wi{시민법}상의 소유권을 분리한 데서
찾을 수 있다고 하기 때문이다.
\wi{봉건제}적 소유권 개념의 주요 특징은
이중\hanja{二重} 소유권\latin{double proprietorship},
즉 봉토 주군의 상급소유권과
토지보유자의 하급소유권 또는 보유권의
병존을 인정하는 것이다.
이제 이러한 소유권의 이중성이
\hemph{시민법}상의\latin{quiritarian} 또는 보통법상의 소유권과
\paren{나중에 생겨난 용어를 쓰면}
\hemph{법무관법}상의\latin{bonitarian} 또는 \wi{형평법}상의 소유권 간의
로마인들의
구별을
일반화한 형태에 대단히 유사하다는 것이다.
\wi{가이우스}는
\hemph{소유권}\latin{dominion}이 두 부분으로 분리되는 것을
로마법에 특유한 현상이라고 보았고,
이를 다른 민족들의 관습인
단일한 또는 완전한\latin{allodial} 소유권과 뚜렷이
대비시키고 있다.\footnote{%
  \latin{Gai.\,2.40.} }
물론
유스티니아누스가 소유권을
하나로 재통합했지만,\footnote{%
  \latin{C.\,7.25.1.}}
만족\hanja{蠻族}들이 수 세기 동안 접촉했던 것은
서로마제국의 부분적으로 개량된 법체계였지
유스티니아누스의 법이 아니었다.
로마제국의 경계선 근처에 자리잡으면서,
그들은
나중에 중요한 결실을 낳은
저 구분을 알게 되었을 가능성이 농후하다.
이 이론에 유리하게도,
만족들의 관습을 모은 법전들에
로마법적 요소가 얼마나 들어있는지가
제대로 조사되지 않았음을
어쨌든 인정하지 않을 수 없는 것이다.
봉건제도를 설명하는 잘못된 또는 불충분한 이론들은
봉건제도라는 직조물에 들어있는
이 특별한 요소를 무시하는 경향을 보인다는 공통점이 있다.
과거에 연구자들은,
그 추종자들이 주로 영국에 많았거니와,
봉건제가 형성되던 시기의 혼란스런 상황에만 중점을 두었다.
이러한 오류에 후에 새로운 오류가 더해졌으니,
독일 학자들은
민족적 자부심에서
그들 선조들이 로마 세계에 등장하기 전부터
가지고 있던 사회 조직의 완전성을 강조했다.
한 두 명의 영국 학자는
봉건제의 토대에 관해
올바른 방향으로 연구를
시도했으나
만족할 만한 결과를 얻는 데 실패하고 말았거니와,
유스티니아누스 법전에서 유사한 것을 찾는 데만 너무 몰두했거나,
혹은 현존하는 몇몇 만족\hanja{蠻族}들의 법전에 추가된
로마법 집성\hanja{集成}들에 관심을 한정했기 때문이었다.\footnote{%
  가령 서고트 왕국은 게르만인을 위한 에우릭법전(Codex Euricianus)과
  로마인들을 위한 알라릭약전(Breviarium Alaricianum)을 편찬했다. 후에
  서고트법전(Lex Visigothorum)으로 통합된다. }
그러나,
로마법이 만족들의 사회에 어떤 영향을 끼쳤다면,
그것은
대부분
유스티니아누스의 입법 이전의 일이었을 테고,
또한 저 집성들을 준비하기 이전의 일이었을 것이다.
생각건대
만족\hanja{蠻族}들의 관습이라는 골격에 살과 근육을 붙인 것은
유스티니아누스의 개혁되고 정화된 법이 아니라,
동로마제국의 로마법대전이 결코 완전히 대체하지 못한,
서로마제국에서 지배적이었던
정돈되지 못한 법이었다.
게르만 부족들이
정복자로서
로마 영토의 일부라도 본격적으로 차지하기 전에,
따라서 게르만 왕들이 로마인 백성들을 위해
로마법 약전\hanjalatin{略典}{breviary}들을 편찬하도록 명하기 한참 전에,
이미 변화가 일어났던 것으로 보아야 한다.
원시적인 법과 발달된 법 사이의 차이를 잘 아는 사람이라면
이 가설을 지지하지 않을 수 없을 것이다.
만족\hanja{蠻族}들의 법전\latin{leges barbarorum}은
비록 미개한 모습으로 우리에게 전해지지만,
오직 만족들의 법에서만 기원한다는 이론을 충족시킬 만큼
그렇게 미개하지는 않다.
또한
성문\hanja{成文}의 기록으로 우리에게 남겨진 규칙의 전부가
정복 부족의 구성원들 사이에서 관행되던 규칙이라고
믿을 만한 근거도 없다.
비속\hanja{卑俗}로마법의 상당수가
이미 만족들의 법체계에 들어있었다고 확신할 수 있다면,
우리는 커다란 난제 하나를 제거할 수 있다.
정복자들의 게르만법과
그들의 백성의 로마법이
세련된 법과 야만인의 관습 간에 통상 존재하는 정도를 넘어
서로 친화성을 갖지 않았다면
이들은
결합할 수 없었을 것이기 때문이다.
만족들의 법전은,
비록 원시적으로 보일지라도,
진정한 원시적 관행과 반쯤 이해된 로마법의 복합체일 뿐이며,
그것이
서로마제국 아래서의 비교적 세련된 형태로부터 이미 다소 후퇴한
로마법과 융합할 수 있었던 것은
이러한 외부적 요소 덕분이었을
가능성이 대단히 크다.

\para{영차권, 콜로누스, 봉건적 봉사}
그러나,
이 모든 것을 인정하더라도,
봉건적 소유 형태가 로마법의 이중 소유권에서 직접
유래했다고 보기 어렵게 만드는 몇 가지 고려사항이 있다.
시민법상의 소유권과 형평법상의 소유권의 구분은
만족\hanja{蠻族}들이 이해하기 힘들 정도로 복잡미묘한 것으로 보인다.
더욱이 그것은 법원이 정상적으로 작동하는 곳에서가 아니면 이해되기 어려운 것이다.
그러나 이 이론을 반박할 수 있는 무엇보다 강력한 근거는
로마법에 존재하는 어떤 재산권 형태 하나---분명 \wi{형평법}의 산물이다---가
하나의 관념 체계로부터 다른 것으로의 전이\hanja{轉移}를 훨씬 더
간단하게 설명할 수 있게 한다는 것이다.
영차권\hanjalatin{永借權}{emphyteusis}이 그것이니,
봉건적 소유권을 탄생시키는 데
그것이 기여한 지분이 얼마인지 그다지 정확한 지식 없이
중세의 봉토권이 바로 여기서 유래했다고
종종
주장되어왔다.
실로
영차권은,
어쩌면 아직 저 그리스식 명칭으로 불리기도 전에,
나중에 봉건제를 만들어낸 관념의 흐름에 중대한 획을 그었다.
로마 역사에서
가부장이 자기 가\hanja{家}의 아들들과 노예들을 데리고
농사를 지을 수 없을 만큼의 넓은 소유지에 대한 언급은
로마 귀족들의 토지에서 최초로 만나게 된다.
이들 대\hanja{大}소유주들은
자유차지인\hanja{借地人}들을 이용해 농사짓는 체제를 알지 못했던 듯하다.
그들의 라티푼디움은 어디서나
노예집단을 부려 경작되었고,
이들을 감독하는 관리자도 노예이거나 해방노예\latin{freedman}였다.
유일하게 시도된 조직형태로,
하급 노예들을 작은 집단으로 나누고
이들 집단을 상급의 보다 믿을 만한 노예에게
\wi{특유재산}\hanjalatin{特有財産}{peculium}으로 맡기는
형태가 있었던 것으로 보이며,
이로써 특유재산을 가진 노예는 노동의 효율성에 대해
일종의 이해관계를 갖게 된다.
하지만 이 체제는 토지소유자의 한 부류, 즉
지방정부\latin{municipality}에게는
특히 불리했다.
이탈리아의 관리들은 교체가 사뭇 빈번하여,
로마의 행정 중에서도 우리를 자주 놀라게 하는 점이다.
그러므로 이탈리아의 지방정부가 소유한 넓은 토지를
감독하는 일은 대단히 불완전했을 것이 틀림없다.
그리하여 지방정부들은
`전세\hanja{田稅}징수지'\latin{agri vectigales}를 세놓는,
다시 말해 특정한 조건 하에
정액지료를 받으며
자유차지인에게
영구적으로
토지를
임대하는
관행을 형성하기 시작했다고 한다.
이 체제는 그후 개인 소유자들도 대거 모방하였다.
토지보유자들은
원래는 계약에 따라
소유주와의 관계가
정해졌으나, 후에
\wi{법무관}에 의해 제한적 소유권을 갖는 것으로 인정받았고,
시간이 흐르면서 이것이 `영차권'이라는 명칭으로 불리게 된 것이다.
이때부터
\wi{토지보유권}의 역사는 두 갈래로 갈라진다.
로마 제국의 남아있는 역사기록이 사뭇 불완전한 긴 기간동안,
로마의 대토지 소유주의 노예집단들은
\wi{콜로누스}\latin{coloni}로 전환되어갔다.
콜로누스의 기원과 지위에 관한 문제는
역사학 전체에서 가장 모호한 문제의 하나이다.
부분적으로는 노예에서 신분이 상승하고,
부분적으로는 자유농에서 신분이 하강하여
형성되었을 것이라고 짐작해볼 수 있다.
그들은 로마의 부유한 계층에게
경작자들이 토지의 산물에 대해 이해관계를 가질 때
부동산의 생산성이 증가한다는 사실을
깨닫게 해주었다.
그들이 토지에 예속되어 있었다는 점,
완전한 노예의 속성 중 다수가 그들에게는 없다는 점,
매년 수확물의 일정 부분을 지주에게 바침으로써 그들의 봉사의무를
다했다는 점 등을
우리는 알고 있다.
또한 우리는 고대로부터 근대에 이르는 온갖 변화의 소용돌이 속에서도
그들이 살아남았다는 것을 안다.
봉건구조의 하층에 편입되어서도,
그들은 로마의 소유주\latin{dominus}에게 지불하던 소작료와 정확히
똑같은 것을 지주들에게 내면서 여러 나라에서 그 존재를 이어갔다.
그들 중 일부 계층인
분익\hanja{分益}\wi{콜로누스}\latin{coloni medietarii}는
수확물의 절반을 소유주에게 지불했거니와,
이는
오늘날 남유럽의 대부분의 토지를 경작하고 있는
분익농\hanjalatin{分益農}{metayer} \wi{토지보유권}으로 이어지고 있다.
다른 한편,
로마법대전에 언급된 것으로부터 이해하건대,
영차권은
소유권의 변종 중에서도
가장 선호되는 유익한 변종이었을 것이다.
자유농이 존재하는 곳이라면 어디서나,
그들의 부동산권을 규율하는 것은 바로 이 토지보유권이었을 것으로 추정할 수 있다.
전술했듯이, 법무관은 영차권자\latin{emphyteuta}를
진정한 소유권자의 하나로 취급했다.
퇴거당하면, 그는
소유권의 독특한 표장\hanja{標章}인
대물소송\hanjalatin{對物訴訟}{real action}을 통해 토지를 회복할 수 있었다.
`정조'\hanjalatin{定租}{canon},
즉 정액지료\latin{quit-rent}만
제때 납부하면 임대인의 방해로부터도 보호되었다.
그렇지만 임대인의 소유권이
소멸했다거나 휴면 중이라고 생각해서는 안 된다.
그것은 여전히 살아서,
지료 지불 해태의 경우 점유회복권,
매매의 경우 선매권\hanjalatin{先買權}{pre-emption},
경작방식에 대한 일정한 통제권 등을 가지고 있었다.
그리하여 우리는 영차권에서
봉건 소유권의 특징이었던
이중 소유권의 현저한 예를 보게 된 것이다.
더욱이 이는
시민법상의 권리와 형평법상의 권리의 병렬보다
훨씬 간단하고 훨씬 모방하기 쉬운 것이다.
하지만 로마 보유권의 역사는 여기서 끝나지 않는다.
라인강과 다뉴브강 줄기를 따라
배치되어 있으면서
오랫동안
인접 만족\hanja{蠻族}들을 상대로
국경선을 지켜온
큰 요새들 사이로
`국경지'\hanjalatin{國境地}{agri limitrophi}라고 불리는
기다란 땅뙈기들이
연속적으로 펼쳐져 있어서
로마 군대의 퇴역 군인들이 이를 영차권에 기하여
점유하고 있었다는
명백한 증거가 있다.
역시 이중 소유권이 있었다.
지주는 로마 국가였으나,
국경 상황이 요구할 때면 군역\hanja{軍役}에 소환될 채비가 갖추어져 있는 한
아무런 방해도 받지 않고
군인들이
경작하고 있었던 것이다.
사실,
오스트리아^^b7투르크 국경지대의 군사식민지 체제와 사뭇 유사한
이런 종류의 주둔지 군역은
통상적인 영차권의 대가였던 정액지료를 대체하는 것이었다.
의심할 여지 없이,
이것이야말로
봉건제의 기초를 놓은
만족\hanja{蠻族}의 군주들이 모방한 선례였을 것이다.
그들은 수백년 동안 그것을 지켜보았을 것이고,
또한 국경을 수비하는 퇴역 군인들 중에는
게르만어를 구사할 줄 아는 만족 출신 군인도 다수 있었다는 점을
기억해야 한다.
이렇게 쉽게 따라할 수 있는 모델이 근처에 있었다는 것은,
프랑크와 롬바르드 군주들이
공유지\hanja{公有地}를 나누어주고 종자\hanja{從者}들의 군사적 봉사를 확보한다는
아이디어를 어디서 얻었겠는가를
설명해줄 수 있다.
뿐만 아니라,
\wi{은대지}\hanjalatin{恩貸地}{benefice}가
곧장 세습화되어간
경향을
설명할 수도 있을 것이니,
비록 원래의 계약조건에 따라 달라질 수 있다 해도
일반적으로 영차권은 수혜자의 상속인에게 상속되는 것이었기 때문이다.
사실,
은대지의 보유자는,
그리고 나중에 은대지에서 전환된 봉토의 영주는,
군사식민지 주민들은 제공하지 않았던 것 같은,
그리고 영차권자도 제공하지 않았음이 분명한,
제반 봉사를
제공할 의무가 있었던 것으로 보인다.
봉건 상급자에게 존경과 감사를 바칠 의무,
그의 딸의 혼인지참금과 그의 아들의 군장비 조달에 조력할 의무,
미성년자인 경우 그의 \wi{후견}을 받아야할 의무,
기타 토지보유에 수반되는 여러 부담들은
로마법상의 보호자\latin{patron}와 해방노예\latin{freedman}의 관계,
즉 전\hanja{前}주인과 전\hanja{前}노예 간의 관계를
그대로 빌려온 것임에 틀림없다.
그런데 초기의 은대지 수혜자는
주군의 인간적 동료였다고 하나,
이 지위는,
보기에는 화려해도,
처음에는 일말의 예속상태로의 강등을 수반하는 것이었음을
부인할 수 없다.
주군의 궁정의 가신\hanja{家臣}이 된 사람들은
\wi{자유소유지} 소유자의 자랑스런 특권이었던
절대적인 신분상의 자유를 일부 포기했던 것이다.

