\chapter{물권법의 초기 역사}

로마의 법학제요 저서들\footnote{가이우스의 법학제요와
유스티니아누스의 법학제요를 일컫는다.}은
소유권의 여러 형태들과 변종들을 정의한 후,
자연법상의 물건취득 방식들에 대하여 논한다.
법제사를 잘 모르는 독자들은 일견
이러한 ``자연법상의 방식들''이
그다지 투기\hanja{投機}적 또는 실제적 이익을
갖지 않는 것으로 생각하기 쉽다.
야생동물을 덫으로 잡거나 사냥해서 죽이는 것,
토양이 강물에 의해 충적되어 부지불식간에
내 땅에 부합\hanja{附合}하는 것,
나무가 내 땅에 뿌리를 내리는 것 따위를
로마 법률가들은 모두 \hemph{자연적으로} 취득한다고 말했다.
옛 법학자들은
그들 주위의 여러 작은 사회의 관행에서
이들 취득이
보편적으로 인정되는 것을 분명 관찰했을 것이다.
후대의 법률가들은
이들이 옛 만민법\latin{jus gentium}에 분류되어 있고
단순명쾌하게 기술\hanja{記述}되어 있는 것을 보았고, 그리하여
이들에게
자연법의 자리를
내주었다.
이들에게 부여된 존엄성은 근대에 이르러 점점 커져,
이제는 원래 가졌던 중요성을 훨씬 능가하는 것이 되었다.
자연법론은 이들을 가장 즐기는 음식으로 삼았고,
실무에 사뭇 심각한 영향력을 행사할 수 있도록 만들었다.

