\chapter{물권법의 초기 역사}

로마의 법학제요 저서들\footnote{가이우스의 법학제요와
유스티니아누스의 법학제요를 일컫는다.}은
소유권의 여러 형태들과 변종들을 정의한 후,
자연법상의 물건취득 방식들에 대하여 논한다.
법제사를 잘 모르는 이들은
취득의 이러한 ``자연법상의 방식들''이
일견
사변적으로나 실무적으로나 큰 관심의 대상이 아닐 것이라고
생각하기 쉽다.
야생동물을 덫으로 잡거나 사냥해서 죽이는 것,
토양이 강물에 의해 충적되어 부지불식간에
내 땅에 부합\hanja{附合}하는 것,
나무가 내 땅에 뿌리를 내리는 것 따위를
로마 법률가들은 모두 \hemph{자연적으로} 취득한다고 말했다.
옛 법학자들은
그들 주위의 여러 작은 사회의 관행에서
이들 취득이
보편적으로 인정되는 것을 분명 관찰했을 것이다.
후대의 법률가들은
이들이 옛 만민법\latin{jus gentium}에 분류되어 있고
단순명쾌하게 기술\hanja{記述}되어 있는 것을 보았고, 그리하여
이들에게
자연법의 자리를
내주었을 것이다.
이들에게 부여된 존엄성은 근대에 이르러 점점 커져,
이제는 원래 가졌던 중요성을 훨씬 능가하는 것이 되었다.
자연법 이론은 이들을 가장 즐기는 음식으로 삼았고,
실무에 사뭇 심각한 영향력을 행사할 수 있도록 만들었다.

\para{선점}
이러한 ``자연법적 취득방식들'' 가운데
한 가지만은 반드시 짚고 넘어갈 필요가 있거니와,
선점\hanjalatin{先占}{occupatio}이 그것이다.
선점은
취득 당시 누구의 물건도 아닌 것을
\paren{법기술적 정의\hanja{定義}가 이어진다}
당신의 물건으로 삼고자 하는 의사로써
점유하는 것을 말한다.
로마 법률가들이 무주물\hanjalatin{無主物}{res nullius}---소유주가
없거나 있어본 적이 없는 물건---이라 불렀던
것이 무엇인지는 열거함으로써만 알 수 있을 뿐이다.
소유주가 \hemph{있어본 적이 없는} 물건에는
야생 동물, 물고기, 야생 조류\hanja{鳥類}, 최초로 캐낸 보석,
새로 발견했거나 경작된 적 없는 토지 따위가 속한다.
소유주가 \hemph{없는} 물건에는
포기된 동산, 버려진 토지,
\paren{특이한 그러나 가공스러운 항목인데}
적\hanja{敵}이 소유한 물건 따위가 속한다.
이 모든 것들은
자기 것으로 삼으려는 의사---일정한 경우 이 의사는
특정한 행위에 의해 명시적으로 드러나야 한다---를 가지고서 처음 점유한
\hemph{선점자}가 완전한 소유권\latin{dominion}을 취득한다.
생각건대,
선점 관행의 보편성으로 인해
한 세대의 로마 법률가들이 그것을 모든 민족에 공통인 법으로
자리매김한 것,
그리고 그 단순성으로 인해
다음 세대의 로마 법률가들이 그것을 자연법에 귀속시킨 것은
그리 어렵지 않게 이해할 수 있다.
그러나 근대 법사\hanja{法史}에서 그것이 누린 행운은
선험적인 고찰로는 얼른 이해되지가 않는다.
로마법의 선점 원리, 그리고 이를 둘러싸고 로마 법학자들이 전개한 법규칙들은
근대 국제법 중에서도
전쟁시 포획에 관한 법과
새로 발견한 땅에 대한 주권 획득에 관한 법의
원천이 되었다.
또한 소유권의 기원에 관한 어떤 이론의 근거가 되었거니와,
이 이론은 대중적으로 인기있는 이론인 동시에,
다수의 위대한 사변적 법학자들이
이러저러한 형태로
널리 수긍하고 있는 이론이다.

\para{적의 소유물, 발견의 법리}
방금 나는 로마법의 선점 원리가
전쟁시 포획에 관한 국제법의 흐름을 결정했다고 말했다.
전쟁시 포획법의 법규칙들은,
적대관계의 발발에 의해 국가들은 일종의 자연상태로 환원되고
이렇게 만들어진 의제적\latin{artificial} 자연상태 하에서
교전국 간에는
사적 소유권 제도가
중지된다는
가정\hanja{假定}에 기초한다.
후기의 국제법 학자들은
그들이 설명하는 법체계에서도
사적 소유권이 어떤 의미에서는 인정된다는 주장을
유지하려고 했기 때문에,
적의 재산이 무주물이라는 가설은 그들에게
정도를 벗어난 충격적인 것으로 여겨졌고,
따라서 그들은 이 가설을 단지 법적인 의제\latin{fiction}에 불과하다고
내세우는 신중함을 보였다.
그러나 자연법이 만민법에 그 기원을 두고 있음을 잘 아는 우리는
어떻게 적의 재산이 무주물로 취급되고 그리하여
최초의 점유자에 의해 취득될 수 있었는지 금방 이해할 수 있다.
고대적 형태의 전쟁을 수행하는 사람들은
승전으로 정복군의 군대가 해산되고
해산된 군인들이 무차별적인 약탈을 자행했을 때
저 관념을 자동적으로 떠올렸을 것이다.
하지만
이때 포획자가 취득하도록 허용된 재산은
원래는 동산에 국한하였을 것으로 보인다.
우리는
고대 이탈리아에서
피정복 국가의 토지에 대한 소유권의 취득에 관해서는
전혀 다른 규칙이 지배했음을 별도의 전거를 통해 알고 있다.
따라서 토지에 대해 선점의 원리가 적용되기 시작한 것은
\paren{항상 어려운 문제이지만}
만민법이 자연법으로 전환되는 시기였을 것으로,
그리하여 황금시대의 법학자들이 행한 일반화의 결과였을 것으로,
짐작할 수 있다.
이에 관한 법리는 유스티니아누스의 학설휘찬에 보존되어 있거니와,
그것은 모든 종류의 적의 재산은 교전 상대방에게 무주물이라는,
그리고 포획자가 그것을 자기 것으로 만드는 선점은 자연법상의 제도라는,
무제한적 주장으로 나아간다.
이러한 명제로부터 국제법이 이끌어낸 규칙들은
때로 군인들의 만행과 탐욕을 필요 이상으로 부추긴다고
비판받았지만,
생각건대 이 비판은
전쟁의 역사를 잘 모르는 사람들에 의해,
그리하여 어떤 종류의 규칙이든 규칙에 대한 복종을 명하는 것이
얼마나 위대한 업적인지 잘 모르는 사람들에 의해 가해진 비판이다.
선점에 관한 로마법 원리가 전쟁시 포획에 관한 근대법에 수용되어 들어왔을 때,
그 남용을 제한하고 정밀함을 부여하는
많은 부수적인 법규칙들도 함께 들어왔으니,
만약 그로티우스의 저서가 권위를 획득한 후에 수행된 전쟁들을
그 이전의 전쟁들과 비교해본다면,
로마법의 규칙들이 수용되자마자 이제 전쟁은 그나마 어느 정도
인내할 만한 성질의 것이 되었음을 알 수 있을 것이다.
선점에 관한 로마법이 근대 만민법\latin{law of nations}에
어떤 해로운 영향을 끼쳤다고 비난받아야 한다면,
해로운 영향을 입었다고 자신 있게 말할 수 있는
분야는
근대 만민법의 다른 영역에 존재한다.
보석의 발견에 로마인들이 적용한 원리를 새로운 땅의 발견에도 적용함으로써,
공법학자\latin{publicist}들은
원래 기대되는 용도와 전혀 맞지 않는 곳에다 억지로
어떤 법리를
가져다 썼다.
15, 16세기의 위대한 항해자들의 발견으로 극히 중요한 것으로 부상한
저 법리는
문제를 해결하기보다는 오히려 야기시켰다.
확실성이 무엇보다 요청되는 두 가지 사항에 관하여
커다란 불확실성이 존재한다는 것이 당장 드러났거니와,
하나는 발견자가 주권자를 위해 취득한 영토의 범위에 관한 것이고,
다른 하나는 `집지'\hanjalatin{執持}{adprehensio},
즉 주권적 점유의 확보에
필요한 행위가 무엇이냐에
관한 것이다.
더욱이,
약간의 행운의 결과치고 엄청난 이득을 가져다주는 저 원리는
유럽의 가장 모험적인 몇몇 국가들, 즉 네덜란드, 영국, 그리고 포르투갈에 의해
본능적으로 거부되었던 것이다.
우리 영국인들은,
저 국제법 규칙을 대놓고 부인하지는 않았지만,
실제로는
멕시코만 이남의 아메리카 대륙을 전부 독점한다는 스페인의 주장을
결코 받아들이지 않았다.
오하이오강 유역과 미시시피강 유역을 독점한다는 프랑스왕의 주장도 마찬가지였다.
엘리자베쓰 1세의 등극부터 찰스 2세의 등극에 이르기까지
아메리카의 수역\hanja{水域}에는 완전한 평화가 깃든 날이
하루도 없었다고 할 수 있고,
프랑스왕의 영토에 대한
뉴잉글랜드의 식민가들의
잠식은
그로부터도 한 세기 이상 계속되었다.
저 법리의 적용을 둘러싼 혼란상에 충격을 받은 벤담은
아조레스 제도 서쪽 100리그\latin{league} 지점에 그은 선을 기준으로
스페인과 포르투갈 간에
이 세상의 미발견된 땅을
나누어갖도록 한
저 유명한 알렉산데르 6세 교황의 칙서를 짐짓 칭송하기까지 했다.\footnote{%
  1493년 알렉산데르 교황의 칙서는
  아조레스 제도 서쪽 100리그 지점 자오선의 서쪽을
  아라곤^^b7카스티야 왕국에 주었다.
  이에 불만을 품은 포르투갈은 스페인과의 협상 끝에
  다음 해인 1494년 저 유명한 `토르데시야스 조약'을 맺어
  교황의 자오선을 조금 더 서쪽으로 옮겼다.
}
그의 칭송이 일견 생뚱맞아 보이기는 하지만,
손으로 잡을 수 있는 귀중품의 취득 요건으로
로마 법학자들이 내건 조건을
어떤 군주의 신민이
수행했다고 해서 그 군주에게
대륙의 절반을 내주는 공법학자들의 법규칙보다
과연
저 알렉산데르 교황의 조치가
원칙적으로 더 불합리한 것인지는 의문의 대상일 수 있다.

\para{소유권의 기원}
본 저서의 주제를 연구하는 모든 사람에게
선점은
그것이
사변적 법학에
사적 소유권의 기원에 관한 가상의 설명을
제공하고 있다는 점에서
특히 관심의 대상이다.
애초 공유의 대상이었던 대지와 그 열매가
개인적 소유권의 대상으로 허용되는 과정이
선점이 이루어지는 과정과 동일하다고 한때 널리 믿어졌다.
자연법에 관한 고대적 관념과 근대적 관념 간의 미묘한 차이를 포착한다면,
이러한 가정\hanja{假定}을 이끌어내는 사고방식을
그리 어렵지 않게 이해할 수 있다.
로마 법률가들은 선점을 자연법적 물건취득의 한 방식이라고 주장했고,
만약 인류가 자연의 제도 하에 살고 있다면
선점도 인류의 관행의 일부일 것이라고 그들은 분명 믿었을 것이다.
인류가 실제로 그러한 상태에서 살았던 적이 있다고 그들이 과연 믿었는지는,
전술한 바처럼, 남아있는 자료로는 확인하기가 어렵다.
그러나 확실히 그들은
소유권 제도가 인류의 존재만큼 오래된 것은 아니라고 생각했던 것으로 보이며,
이런 생각은 시대를 막론하고 상당한 설득력을 가지는 것이다.
그들의 모든 도그마를 유보 없이 수용한 근대법학은
가상의 자연상태를 강조하는 열성에 있어서만큼은
그들보다 훨씬 멀리 나아갔다.
그리하여 근대법학은
대지와 그 열매가 한때 무주물이었다는 명제를
수용했을 뿐만 아니라,
자연에 대한 특유한 견해로 인해
국가사회가 형성되기 오래 전부터
인류가 무주물의 선점을 실제로 관행했었다고
서슴없이
가정하기에 이르렀다.
그리고 이로부터
원시 시대의 ``누구의 것도 아닌 물건''\latin{no man's goods}이
역사 시대의 개인의 사적 소유권으로 되는 과정이
바로 선점이었다는 추론이
즉시
도출되었다.
이런 이론을
이런저런 형태로
지지하는 법학자들을 일일이 열거하는 것은
지루한 일이 될 터이고,
그다지 필요하지도 않을 것이다.
언제나
당대의 평균적 의견의 충실한 지표 역할을 하는
블랙스톤이 그의 저서 제2권 제1장에서
그것을 잘 요약해놓았기 때문이다.

\para{블랙스톤의 이론}
그는 이렇게 쓰고 있다.
``대지와 대지 위의 모든 것은 창조주의 직접적 증여로서
인류 공동의 재산이었다.
물론
최초의 시기에도
물건의 공유성은
물건의 본질에만 적용될 수 있을 뿐이었고,
그것의 사용에까지 확장될 수 없었다.
왜냐하면, 자연법과 이성법에 따르면,
물건을 처음 사용하기 시작한 사람은
일종의 일시적 소유권을 취득하고
그것을 계속 사용하고 있는 동안은 그 일시적 소유권도 계속되기 때문이다.
보다 정확히 말하자면,
점유 행위가 지속되는 동안은 점유권도 지속되는 것이다.
그리하여 토지는 공유였고,
토지의 그 어떤 일부도 특정인의 영구적 소유권의 대상일 수 없었으나,
누군가가
휴식을 위해, 그늘을 위해, 또는 다른 이유로
특정 장소를 선점\latin{occupation}하면,
그는 당분간 일종의 소유권을 취득하고,
그에게서 강제로 그 소유권을 빼앗는 것은 부정의하고
자연법에 반하는 일이 될 것이다. 하지만
그가 사용이나 점유\latin{occupation}를 그치는 순간,
다른 사람이 그 장소를 차지하는 것은 아무런 부정의가 아니다.''
그리고 이렇게 주장을 이어간다.
``인류의 인구가 증가하면서,
보다 영구적인 소유권 관념이 필요하게 되었고,
개인에게
일시적인 사용을 넘어
물건의 본질을 사용할 수 있도록
허용할 필요가 생겨났다.''

위 인용문에 나타난 몇몇 모호한 표현을 볼 때,
블랙스톤은
그가 참조한 전거들에 나오는,
\hemph{선점자}가 자연법에 의하여 지구 표면에 대한 소유권을 취득한다는
명제를 제대로 이해하지 못한 것이 아닌가 한다.
그러나
의도적이든 오해에 의해서든
저 이론에 이처럼 제한을 가함으로써
그는
흔히들 상정되어온 형태를 그대로 따르고 있다.
언어의 정확한 구사에 있어
블랙스톤보다 더
유명한 많은 학자들이
태초에는
우선
선점에 의해
배타적이지만 일시적인 향유권이 대세\hanja{對世}적으로
부여되었고,
그후 이 권리가 배타성은 유지한 채 영구적인 것이 되어다고
주장해왔다.
그들이 이렇게 이론을 전개한 목적은
자연상태에서는
무주물이
선점에 의해
소유권의 대상이 된다는 법리와
가부장들이
양떼와 소떼에게 풀을 먹이던 토지를
처음에는
영구적으로 전유\hanja{專有}하지 않았다는
성서의 이야기에서
추론한 결과를
조화시키려는 것이었다.

블랙스톤의 이론에 직접 적용될 수 있는 한 가지 비판은
그가 묘사하는 원시사회의 상태가
똑같이 쉽게 상상할 수 있는 다른 상태들보다 과연 더 설득력이 있는 것이냐
하는 데 있다.
이 문제를 탐구하기 위해
우리는
토지의 특정 장소를 휴식이나 그늘을 위해
\hemph{선점한}\latin{occupied}
\paren{블랙스톤은 이 단어를
일상적인 의미로 사용하고 있음이 분명하다}
사람이 아무런 방해 없이 그것을 보유할 수 있겠는가를
질문해보는 것이 좋겠다.
그의 점유권은 그것을 지킬 수 있는 힘에 정확히 비례할 것이고,
똑같이 그 장소를 갈망하고 있고
점유자를
충분히
힘으로
내쫓을 수 있다고 생각하는
경쟁자들에 의해 끊임없이 방해를 받을 것임에 틀림없다.
그러나 사실,
이 모든 트집잡기는 저 이론 자체의 근거없음에 비하면 한가한 이야기에 불과하다.
원시 상태의 사람들이 무엇을 했는가 탐구하는 것은
전혀 희망없는 일은 아닐 수 있지만,
그들 행위의 동기를 안다는 것은 도저히 불가능한 일이다.
태초의 사람들에 대한 저 이론의 묘사는,
오늘날 우리가 처해있는 상태와는 사뭇 다른 상태에
그들이 놓여있었다고 우선 가정함으로써,
그리고는
이런 가상의 상황에서도
지금 우리가 가지고 있는 감정과 선입견을
그들도
똑같이 가지고 있었다고 상정함으로써, 이루어진다.
그 감정이 실은 가설상의 그들의 상태와는 전혀 다른 상태에서
만들어진 감정일 수 있음에도 말이다.

\para{사비니의 금언}
소유권의 기원에 관하여
블랙스톤이 요약한 것과 비슷한 견해를 지지하는 것으로
때로 여겨져온 사비니의 금언\hanja{金言}이 있다.
저 위대한 독일 법학자는
모든 소유권의 기초가
취득시효\hanjalatin{取得時效}{prescription}에 의해 완성된
적대적 점유\latin{adverse possession}에 있다고 주장했다.
사비니의 이러한 진술은 오직 로마법에만 근거한 것이고,
진술에 사용된 표현들을 설명하고 정의하는 데 충분한 노력을 들여야만
완전히 이해될 수 있는 것이다.
하지만
로마인들이 채용한 소유권 관념을 아무리 깊게 탐구하더라도,
법의 유년기에 이르기까지 그 관념을 아무리 멀리 추적하더라도,
저 금언에 포함된 다음 세 가지 요소로 구성된 것 이상의
다른 소유권 개념은
얻을 수는 없다는 주장으로
그의 주장을
이해한다면,
우리는 그가 말하고자 한 바를 충분히 정확하게 이해했다고 할 수 있다.
점유, 적대적 점유, 그리고 취득시효가 그것이다.
적대적 점유란 허락받은 점유나 종속적인 점유가 아닌,
온 세상을 상대로 배타성을 주장하는 점유를 말한다.
취득시효란 적대적 점유가 평온하게 지속되어온 시간의 경과를 말한다.
이 금언은 사비니가 의도한 것을 넘어 더 일반적으로 적용될 수도 있다고 믿는다.
그리하여 어떠한 법체계를 조사해보더라도
이들 세 가지가 결합된 소유권 개념 이상의
건전하고 안정적인 결론을 찾아내기란 불가능하리라고 믿는다.
동시에,
소유권의 기원에 관한 대중적인 이론을 지지하기는커녕,
사비니의 금언은 그것의 가장 약한 고리를 드러내는 특별한 가치를 가지고 있다.
블랙스톤 및 그가 추종하는 사람들의 견해에서는,
배타적 점유를 획득하는 방식이
인류의 조상들의 정신에 어떤 신비로운 영향을 주었었다.
그러나 사비니의 금언에는 이러한 신비로움이 없다.
적대적 점유에서 소유권이 시작한다는 데는 놀라울 것이 전혀 없다.
최초의 소유자는 자신의 재산을 안전하게 지켜내는,
무장을 한 힘있는 사람이었을
것이라는 데는 놀라울 것이 전혀 없다.
하지만 어째서 시간의 경과가 그의 점유를 존중하는 감정---이것이야말로
장기간 사실상\latin{de facto} 존재해온 것에 대한 인류 보편의 존중심의
원천이다---을 만들어내는지는
진정 깊이 연구해볼 가치가 있는 문제이지만,
이는 지금 우리의 탐구 범위를 훨씬 넘어선다.

\para{소유권의 추정}
드물고 불확실한 정보에 불과하지만
소유권의 초기 역사에 관한 약간의 정보를 얻을 수 있을 법한 지역을 다루기에 앞서,
우선 나는
문명의 초기 단계에서 선점이 수행한 역할에 주목하는 대중적 견해가
실은 진실을 거꾸로 뒤집은 것이라는 점을 감히 지적하고자 한다.
선점은 의사\hanja{意思}에 의한 물리적 점유의 획득이다.
이런 유\hanja{類}의 행위가 ``무주물''에 권리를 수여한다는 관념은,
초기 사회의 특징이기는커녕,
세련된 법학과 확립된 법상태의 결과물일 공산이 농후하다.
소유권이
오랜 관행을 통해
그 불가침성을
인정받은 후에야,
대부분의 향유의 객체가 사적 소유권의 대상이 되고 난 연후에야,
이전에 소유권이 주장된 바 없는 물건에 대한 소유권을
최초의 점유자가
단순한 점유에 의해
수여받도록
비로소
허용되는 것이다.
이러한 원리를 만들어낸 감정은
문명의 시초를 특징짓는
저 희소하고 불확실한 소유권과는 전혀 조화되지 않는 것이다.
그 감정의 진정한 기초는
소유권 제도를 지향하는 본능적인 선입견이 아니라,
소유권 제도의 오랜 지속으로 등장한,
\hemph{모든 것은 주인이 있어야 한다}는 추정\hanja{推定}인 것이다.
``무주물'', 즉
소유권의 대상이 \hemph{아닌}
또는 소유권의 대상인 적이 \hemph{없는}
객체가 점유될 때,
점유자가 소유권자로 허용되는 것은
모든 가치있는 물건은 당연히 배타적 향유의 대상이라는,
그리고
당해 사례에서는
선점자 외에 소유권을 수여받을 사람이 없다는 감정에
기인한다.
요컨대,
모든 물건은 누군가의 소유물이어야 하기에,
그리고
특정 물건의 소유권자로 선점자보다 더 나은 권리를 갖는
사람을 찾을 수 없기에,
선점자가 소유권자가 되는 것이다.

\para{대중적 이론의 반박}
우리가 논의해온 자연상태 사람들에 대한 기술\hanja{記述}에 대해
다른 반박이 없다 할지라도,
적어도 한 가지 점에 있어서만은
그것은 우리가 가진 믿을 만한 증거에 결정적으로 배치되고 있다.
저 이론이 상정하는 행위와 동기는 개인의 행위와 동기라는 점을 주목하자.
사회계약 체결의 당사자는 각 개인들이다.
홉스의 이론에 의하면
개인이라는 모래알로 구성된 어떤 움직이는 모래더미가
완전한 강제력에 의해 사회적 바위로 굳어지는 것이다.
블랙스톤의 묘사에서
``휴식을 위해, 그늘을 위해, 또는 다른 이유로 특정 장소를 선점하는''
것도 개인이다.
로마인들의 자연법에서 유래한 모든 이론이
이 결함으로부터 자유로울 수 없거니와,
로마인들의 자연법은 개인을 취급함에 있어 그들의 시민법과 근본적으로 달랐고,
고대사회의 권력으로부터 개인을 해방시킴으로써
문명의 진보에 크게 기여하였다.
그러나 고대사회는, 반복하여 말하지만,
개인을 거의 알지 못한다.
그것의 관심은 개인이 아니라 가족에 있었고,
단독의 인간이 아니라 집단에 있었다.
국가법이 친족집단이라는 작은 원\hanja{圓}들을
원래는 전혀 뚫지 못하다가
마침내 뚫고 들어갔을 때도,
그것이 바라보는 개인은 훗날 성숙한 단계의 법이 바라보는 개인과
사뭇 달랐다.
각 시민의 생애는
출생과 사망에 의해 한계지워지지 않았다.
그는 그의 선조들의 존재의 계속이었을 뿐이고,
또한 그의 후손들의 존재 속에 계속 살아갈 것이었다.

\para{인법과 물법}
편리하기는 하지만 전적으로 인위적인,
신분법\latin{law of persons}과 재산법\latin{law of things} 간의
로마인들의 구별은
지금 우리 앞에 놓인 주제에 대한 탐구를 올바른 경로에서
벗어나게 하는 데 분명 크게 기여했다.
인법\hanjalatin{人法}{jus personarum}에서 배운 지식은
물법\hanjalatin{物法}{jus rerum}에 이르러서는 완전히 망각되었다.\footnote{%
  인법과 물법은 각각 신분법과 재산법을 지칭하는 로마인들의 용어.}
그리하여
인법의 원초적 상태에 관해 알게된 사실로부터
물권법, 계약법, 불법행위법의 기원에 관한 힌트를 전혀 얻을 수가 없는 것처럼
지금까지 생각되어왔다.
이런 사고방식이 잘못되었다는 것은
순수한 고대법 체계 하나를 가져다 놓고
그것에 로마법의 분류를 적용하는 실험을 해볼 수 있다면
분명해질 것이다.
법의 유년기에는
재산법으로부터 신분법을 분리하는 것이 무의미하다는 것을,
두 영역에 속하는 법규칙들이 불가분 서로 엉켜있다는 것을,
후대 법학자들의 구분은 후대의 법에만 적합하다는 것을
알게 될 것이다.
본 저서의 앞 부분에서 말한 것들을 종합하면,
우리가 개인의 소유권에만 관심을 국한하면
초기 소유권의 역사에 대한 어떠한 단서도 얻을 수 없을 것이라는
강한 선험적 개연성을 감지할 수 있을 것이다.
개인적 소유가 아니라 공동소유가 고대법의 진정한 모습이라는 것은,
우리에게 시사점을 주는 소유 형태는 가족의 권리, 친족집단의 권리와
관련된 것이라는 것은 그저 그럴 수도 있겠다는 정도를 넘어선다.
여기서 로마법은 우리를 깨우치는 데 별 도움을 주지 못하거니와,
자연법 이론에 의해 변형되어
우리에게 전달된 바로 그 로마법이
개인적 소유가 소유권의 정상적 상태라는 인상을,
인간집단이 공유하는 소유권은 원칙에 대한 예외에 불과하다는 인상을,
우리에게 심어주기 때문이다.
하지만 원초적 사회의 잃어버린 제도를 탐구하는 연구자라면
반드시 주의깊게 살펴봐야 할 공동체가 하나 있다.
오래 전부터 인도에 정착해 살아온,
인도^^b7유럽 계통의 한 갈래 사이에서
그 제도가
어떤 변천을 겪어왔든 간에,
그것은
자신을 배태한 껍질을 완전히 벗어버리지 못했다는 것을 알 수 있을 것이다.
우리의 신분법 연구로부터
소유의 원초적 형태에 관한 아이디어를 얻을 수 있는
형태에 정확히 들어맞는다는 점에서
즉시
우리의 눈길을 끄는 그러한 소유 형태가
인도인들 사이에서
발견되는 것이다.
인도의 촌락공동체\latin{village community}는 조직화된 가부장제 사회이자
공동소유자들의 연합체이다.
그것을 구성하는 사람들 간의 인적 관계는
그들의 소유권과 불가분 결합되어 있어서,
영국인 관리들이 이 둘을 분리하려고 하면 이는
영국의 인도 통치에 있어 가장 치명적인 실책이 될 것이다.
이 촌락공동체는 무한히 오래 된 것으로 알려져있다.
인도 역사를 어느 방향에서 접근하든 간에,
일반 역사든 지방 역사든 간에,
이 공동체가 진보의 초기부터 존재했음이 항상 발견되어왔다.
대부분 그것의 성격과 기원에 대한 특별한 이론을 갖지 않는
무수한 지식인들과 관찰자들이
이구동성으로 말하기를,
이 사회가
어떤 혁신에도 좀처럼 굴하지 않고 지켜온 관행 중에서도
그것이야말로 가장 파괴되기 어려운 관행이라고 한다.
정복과 혁명이 수차례 휩쓸고 지나갔지만
그것을 어지럽히거나 없애지 못했으며,
인도에서 가장 유익한 통치체제는 언제나
그것을 행정의 기초로 인정하는 통치체제였던 것이다.

\para{공동체와 분할}
성숙한 로마법과 그 자취를 따른 근대법은
공동소유를 소유권의 예외적이고 일시적인 상황으로 바라본다.
이런 견해는
``누구도 자신의 의사에 반하여 공동소유에 묶이지 않는다''%
\latin{Nemo in communione potest invitus detineri}는,
서유럽에서 보편적으로 받아들여지는 법언에 명료하게 드러나 있다.
그러나 인도에서는
관념의 순서가 거꾸로이며,
개별 소유권은 언제나 공동소유권으로 되돌아가는 경향이 있다고 할 수 있다.
그 과정에 대해서는 이미 언급한 바 있다.
아들이 태어나자마자
그는 아버지의 재산에 대해 기득권을 취득한다.
결정권을 행사할 수 있는 나이가 되면,
일정한 경우
가족 재산의 분할을 요구할 권리가
법문\hanja{法文}에 의해 주어진다.
하지만 사실
아버지의 사망시에도 분할은 잘 일어나지 않는다.
재산은 수 세대에 걸쳐 분할되지 않은 채 계속 유지되거니와,
다만
각 세대의 각 구성원들이 미분할된 지분에 대해 법적 권리를 가질 뿐이다.
이러한 공동소유의 토지는 때로는 선출된 관리자에 의해 관리되지만,
일반적으로는, 그리고 일부 지역에서는 언제나,
가장 나이 많은 종친\hanja{宗親}, 다시 말해
가장 손윗 계통의 가장 나이 많은 대표자에 의해 관리된다.
이러한 공동소유자들의 연합체, 즉
토지를 공동소유하는 친족집단은
인도 촌락공동체의 가장 단순한 형태이다.
그러나
이 공동체는 친족으로 구성된 동족집단 그 이상이고
조합원들로 구성된 조합 그 이상이다.
그것은 하나의 조직화된 사회이다.
공동재산을 관리하는 것 외에도,
거의 항상 그것은 다수의 스태프들을 통하여
내치\hanja{內治}를,
치안을,
사법\hanja{司法}을,
그리고 조세와 부역\hanja{賦役}의 할당을
수행한다.

\para{인도의 촌락공동체}
내가 기술한 촌락공동체의 형성과정은 전형적인 것으로 간주해도 좋다.
하지만
인도의 모든 촌락공동체가
그러한 단순한 방식으로 결합되어 있다고 생각해서는 안 된다.
인도 북부에서는,
기록에 의하면,
공동체는
거의 항상
혈연관계에 기초한 단일한 연합체의 모습이라지만,
같은 기록은
때로 외부인이 접목되어 들어가는 일이 늘 있어왔다는 것도
알려준다.
일정한 조건 하에서는
단순히 지분을 매수한 것에 불과한 자가 동족집단에 받아들여지는 것이다.
인도 반도 남부에서는
하나의 가족이 아니라 둘 이상의 가족에서 유래한 것으로 보이는
공동체도 다수 존재하거니와,
어떤 공동체는 그 구성이 전적으로 인위적인 것으로 알려져있다.
사실,
서로 다른 카스트에 속하는 사람들이
동일한 사회로
결합하는 것은
공통 조상의 후손이라는 가설에 전혀 부합하지 않는 것이다.
그럼에도 불구하고 이 모든 동족집단에는
최초의 공통 조상에 관한 전승\hanja{傳承}이 내려오거나
혹은 그러한 가정\hanja{假定}이 만들어져있다.
남부의 촌락공동체를 집중 연구한
마운트스튜어트 엘핀스톤\latin{Mountstuart Elphinstone}은
이렇게 말한다\paren{<<인도사>>\latin{History of India}, 71쪽. 1905년판}:
``대중적 견해에 따르면,
마을의 지주들은 모두가 그 마을에 정착한 하나 이상의 개인들의 후손이다.
유일한 예외는 원주민 혈통의 구성원에게서 매수 등을 통해
권리를 취득한 사람들뿐이다.
이런 견해는
오늘날
작은 마을에는 지주 가족이 하나만 존재하고
큰 마을에도 몇 안 되는 가족만 존재한다는
사실에서도 확인된다.
그러나 각 마을은 수많은 구성원들로 분기\hanja{分岐}되어왔기에,
농사에 필요한 노동은
소작인이나 노무자의 도움 없이
전적으로 지주들에 의해 행해지는 경우가 적지 않다.
지주들의 권리는 그들에게 집단적으로 속한다.
그들은 거의 항상 그 권리에 대한 다소간의 완전한 지분을 갖지만,
완전한 분리가 일어나는 일은 결코 없다.
가령 어떤 지주가 그의 권리를 매각하거나 저당잡힐 수는 있다.
그러나 그러려면 그는 우선 마을의 동의를 얻어야 한다.
또한 매수인은 정확히 매도인의 지위를 대신하고 그의 모든 의무도 넘겨받는다.
만약 어떤 가족이 소멸하게 되면, 그 지분은 공동재산으로 되돌려진다.''

\para{공동체의 전형}
본서 제5장에서 살펴본 고찰이 엘핀스톤의 인용문을 이해하는 데
독자들에게 도움을 주리라 믿는다.
원시 세계의 어떤 제도도
생동력 있는 법적의제를 통해
원래의 성질에는 없는 유연함을
얻지 못했다면
오늘날까지 전해지지 못했을 것이다.
그리하여 촌락공동체는 반드시 혈연관계의 연합체인 것이 아니라,
그러한 연합체\hemph{이거나 아니면} 친족관계의 모델에 기초하여 형성된
공동소유자 집단인 것이다.
이것에 비견되어야 할 유형은 로마의 가족이 아니라,
로마의 씨족\latin{gens}임에 틀림없다.
씨족 또한 가족의 모델에 기초한 집단이었다.
그것은 다양한 의제를 통해 확대된 가족이었거니와,
그 의제의 정확한 성질이 무엇인지는 아주 오래 전에 잊혀져버렸다.
역사 시대에 이르렀을 때, 그것의 주된 성질은
촌락공동체에 관한 엘핀스톤의 언급에 나타난 바로 그 두 가지였다.
공통의 기원에 관한 가정\hanja{假定}이 항상 있었거니와,
그 가정은 때로는 실제 사실과 노골적으로 배치되기도 했다.
그리하여, 저 역사학자의 말을 반복하자면,
``만약 어떤 가족이 소멸하게 되면, 그 지분은 공동재산으로 되돌려진다.''
옛 로마법에서도 상속인 없는 상속재산은
씨족원들\latin{getiles}에게 복귀하였던 것이다.
나아가, 로마사 연구자라면 누구나
씨족과 같은 공동체는 외부인을 수용함으로써 수시로 불순물이 혼입되었다고,
그러나 수용의 정확한 방식은 지금으로서는 알 수 없다고 믿고 있다.
이제 인도에서는, 엘핀스톤이 알려주듯이,
동족집단의 동의를 얻어 매수인이 받아들여짐으로써
외부인이 유입되는 것이다.
하지만 수용된 자의 취득의 성질은
포괄적 승계\latin{universal succession}에 해당한다.
매수한 지분뿐 아니라,
그는
전체 집단에 대해 매도인이 부담하고 있던 책임도
함께
승계하는 것이다.
그는 바로 가\hanja{家}의 매수인\latin{emptor familiae}으로서,
그가 대체하게 될 사람의 법적인 옷\hanja{[衣服]}을 물려입는 것이다.
그를 수용하는 데 필요한 전체 동족집단의 동의는,
쿠리아 민회\latin{comitia curiata}, 즉
동일한 이름을 가진 친족집단이 모인 보다 큰 동족집단인 고대 로마 국가의
입법기구가
입양의 허가나 유언의 확인에 반드시 필요하다고 강하게 주장했던
그 동의를 상기시킨다.

\para{러시아의 촌락공동체}
인도 촌락공동체의 거의 모든 특징들에서
그것이 대단히 오래된 것임을 알려주는 징후를 발견할 수 있다.
이 법의 유년기에는 공동소유가 지배적이었음을,
신분권과 재산권이 서로 엉켜있었음을,
공적 의무와 사적 의무가 혼재되어 있었음을 알려주는
수많은 근거들이 있어서,
이들 공동소유의 동족집단을 관찰함으로써 여러 중요한 결론들을
이끌어내도 무리가 없다 하겠거니와,
유사한 구조를 가진 사회가 세계 어디서도 발견되지 않는다 해도 그러할 것이다.
하지만,
봉건제에 의한 소유권의 격변을 그다지 겪지 않았고
여러 중요한 면에서 동양과도 서양과도 밀접한 친화성을 갖는
유럽의 한 지역에 존재하는 유사한 구조의 현상들이
최근 많은 진지한 관심의 대상이 되고 있다.
학스타우젠\latin{August von Haxthausen} 씨,
텡고보르스키\latin{Ludwig Tengoborski} 씨 등의 연구자들이
러시아의 촌락은 사람들의 우발적인 결합도 아니고
그렇다고 계약에 기초한 결합도 아님을 보여주었다.
그것은 인도의 공동체와 마찬가지로 자연적으로 조직화된 공동체인 것이다.
물론,
이들 촌락은 이론적으로는 언제나 어떤 귀족의 소유지이며,
역사 시대 내내
농부들은
그 영주의 토지에 예속된 농노\latin{predial serf}로,
혹은 보다 일반적으로는
그에게 신분적으로 예속된 농노\latin{personal serf}로 전락해갔다.
그러나
이러한 상급 소유권의 압력도
촌락의 고대적 구조를 파괴시키지 못했다.
농노제를 도입한 것으로 여겨지는 러시아 짜르의 입법도
기실 옛 사회질서를 유지하는 데 불가결한 저 협력관계를
농부들이 버리지 못하도록 막기 위해 만들어진 것이었다.
마을 사람들 간의 종족\hanja{宗族}적 관계를 고려할 때,
신분법과 재산법의 혼재를 고려할 때,
또한 다양한 자발적 자치규범들을 고려할 때,
러시아의 촌락은 인도 촌락공동체의 거의 정확한 반복으로 보인다.
그러나 한 가지 중대한 차이점이 있으니,
크게 관심을 둘 만하다.
인도 촌락의 공동소유자들은,
비록 그들의 소유권이 통합되어 있기는 하나,
각자 자기만의 권리를 가지며,
이러한 권리들이 분할되면 그 분리는 완전하고 또한 무한히 지속된다.
러시아 촌락에서도 이론적으로 권리들의 분할은 완전하지만,
그러나 여기서는 그것이 일시적인 데 그친다.
일정한, 그러나 모든 경우에 다 동일하지는 않은,
시간이 경과하면
분리된 소유권들은 소멸하여,
그 촌락의 토지가 하나의 덩어리로 합쳐지고, 그후
공동체를 구성하는 가족들 간에 식구 수에 따라 재분배된다.
이러한 재분할이 행해지면,
가족들의 권리와 개인들의 권리는 다시
다수의 계통으로 가지를 칠 수 있거니와,
이러한 가지치기는 또 다른 분할의 시기가 도래할 때까지 계속된다.
이러한 소유권 유형의 변종인 더욱 특이한 형태가
오랫동안
투르크 제국과 오스트리아 왕가의 영토 사이에 분쟁지역이 되어온
몇몇 나라들에서 발견된다.
세르비아, 크로아티아, 오스트리아령 슬라보니아에서도
촌락들은 공동소유자이자 친족관계인 사람들로 구성된 동족집단이다.
그러나 여기서는 공동체의 내부 구조가
앞서 살펴본 두 사례와 상이하다.
여기서는 공동소유의 재산이 실제도로 분할되지 않고
이론적으로도 분할될 수 없다고 간주된다.
토지 전부가 마을 사람 모두의 공동의 노동으로 경작되고,
수확물은 매년 가구별로 분배되거니와,
때로는 각 가구의 필요에 따라서,
때로는 특정인에게 용익권\latin{usufruct}의 일정 몫을 주는 규칙에 따라서,
분배된다.
동유럽의 법학자들은 이 모든 관행이
초기 슬라브법에 기초한 원리에서 나왔다고 주장하거니와,
그것은 가족의 재산은 영원토록 분할될 수 없다는 원리인 것이다.

\para{공동체의 다양성}
우리의 탐구에서 이러한 현상들이 큰 관심의 대상이 되는 이유는
원래 공동으로 재산을 소유하던 집단 \hemph{속에서}
어떻게 해서 개별적 소유권이 발달했는가에 대해
그 현상들이 실마리를 던져주기 때문이다.

