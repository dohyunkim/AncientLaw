\chapter{주권과 제국}

% 371
\para{힘과 질서, 그들 간의 우선성}
`법'이라는 단어는
\hemph{질서}라는 관념과
\hemph{힘}라는 관념의 두 관념과 밀접한 관련을 갖는 것으로
우리에게 주어져왔다.\footnote{%
  부록 B는 \latin{Henry Sumner Maine,
  \textit{Lectures on the Early History of Institutions}, 4th ed.,
  London: John Murray, 1885(1874), 제13장 `Sovereignty and Empire'}를
  우리말로 옮긴 것이다. }
이 관련은 아주 오래된 것이고
아주 많은 언어들에서 나타난다.
그리고 이렇게 해서 연결된 두 관념 가운데
어느 것이 우선권을 갖는가,
어느 것이 정신의 개념에서 먼저 오는 것인가?
하는 질문이 끊임없이 제기되어왔다.
분석법학 이전에는 대체로
`법'이 무엇보다 질서를 함의한다고 여겨져왔다.
``아주 일반적이고 포괄적인 의미에서 법은 작용의 규칙이다.
그것은 영혼 있는 것이든 아니든,
이성적인 것이든 아니든,
모든 종류의 작용에 무차별적으로 적용된다.
그리하여 우리는 운동의 법칙, 중력의 법칙, 광학이나 역학의 법칙을,
나아가 자연법이나 만민법도 이야기하는 것이다.''
블랙스톤은 `법 일반의 성질'에 관한 장\hanja{章}을 이런 말로 시작한다.
이것에 대한 전적인 거부감에서
% 372
벤담과 오스틴은 법학자가 되었다고
할 수도 있을 것이다.
한편, 분석법학자들은 서슴없이 힘의 관념을 질서의 관념보다 우선시한다.
그들에 따르면
진정한 법은 저항할 수 없는 주권자의 명령이고
일군의 작위 또는 부작위를 한 사람 또는 다수의 백성들에게 명한다.
그리고
이러한 명령에 의해 그들은
모두 함께 무차별적으로
어떤 법적 의무 아래 놓인다.
그리하여
진정한 법에
사실문제로서
부여된 특성,
무차별적으로
다수의 사람들을
일반적으로 정해진
다수의 작위 또는 부작위에 속박하는 특성에 의해
`법'이라는 용어가
물리 세계의, 정신작용 세계의, 혹은 인간 행동의
획일적이고 불변적인 연속적 현상에
은유적으로 확장될 수 있는 것이다.
중력의 법칙, 관념연관의 법칙, 혹은 지대의 법칙과 같은 표현에 사용된
법은 분석법학자들에 따르면
그 진정한 의미에서 벗어나 부정확한 상징으로써 확장된 것이다.
또한 그들이 이런 법을 말할 때 일종의 경멸의 뜻이 담겨있음을 부인할 수 없다.
그러나 생각건대,
어떤 단어에 존엄성이나 중요성이 담길 수 있다면,
오늘날에는 물리적, 정신적, 혹은 정치경제적 현상의 불변적 연속성을
뜻할 때의 법이라는 단어보다 더 존엄하고 중요한 것이 또 있을까 싶다.
이런 의미에서의 `법'은 대부분의 근대사상에 들어와있고,
% 373
근대사상이 수행되는 조건이 되었다고 할 수 있을 정도이다.
아가일 공작\latin{Duke of Argyll}의 책으로 유명해진
`법칙의 지배'\latin{Reign of Law} 같은 표현\footnote{%
  <<법칙의 지배>>라는
  책의 저자 아가일 공작은
  \latin{George Campbell(1823--1900)}을 말한다.
}을 오스틴이 경멸했을 것이라고 처음에는 믿기 어렵겠지만,
이 점에 관해
오스틴의 언어는
의심의 여지가 없다. 그는
또한, 그의 주된 저술이 40년 전으로 거슬러올라가는 것이 없지만,
인간의 관념이 오늘날처럼 실험과 관찰의 과학에 의해 세례를 받기 전에
저술활동을 했다고 누차 스스로 언급했다.

\para{`법'의 원초적 의미, 법학적 의미의 법, 법의 초기 개념}
모든 언어에서 법이 우선 주권자의 명령을 뜻하고
여기서 파생되어 자연의 질서있는 연속성에 적용되었다는 진술은
그 진위를 가리기가 무척 어렵다.
설령 그것이 진실이라 하더라도 진실성을 확인하는 데 들이는
노력에 값하는 가치가 있는지도 의문스럽다.
우리에게 알려진 철학적·법학적 사변의 역사를 돌아보면
법에 관련되는 사실문제인 저 두 관념이 서로에게 작용과 반작용을 가해왔음이
드러나기에 그 어려움은 배가된다.
의심의 여지 없이
자연의 질서는 주권자의 명령에 의해 결정된다고 관념되어왔다.
근대사상의 계보에 속하는 많은 이들이
우주를 구성하는 물질의 입자가 인격적 신의 명령에 복종한다고 생각했다.
마치 형사 제재의 두려움 때문에 주권자의 명령에 백성들이 복종하듯이 말이다.
한편,
% 374
외부세계의 질서에 대한 사고는
문명세계의 다수 인류가 갖는 진정한 법에 대한 견해에 강한 영향을 주었다.
로마인들의 자연법 이론은
법의 역사 전체에 영향을 주었던 것이다.
실로 이 유명한 이론은 두 가지 요소로 구성되어 있거니와,
하나는 그리스에서 유래한,
물리 세계의 어떤 질서나 규칙성의 초기 관념에 기초한 것이고,
다른 하나는 로마에서 기원한,
인류가 준행하던 것 중에 들어있는 어떤 질서나 유사성의 초기 관념에 기초한 것이다.
몇 년 전에 출간된 내 저서에서 이에 관한 증거를 제시해두었으므로
여기서 이를 반복할 필요는 없을 것이다.
어떤 사람들이나 공동체들이 어떤 단어를 그들이 원하는 의미로 사용한다고 해서,
또는 그들이 원하는대로 여러 가지 의미로 사용한다고 해서,
그들을 비난할 수 있는 권리는 누구에게도 없다.
하지만 과학적 연구자의 책무는
어떤 중요한 단어의 의미들을 서로 구분하고,
자신의 목적에 적합한 의미를 선택하고,
연구과정에서 그 단어를 일관되게 이런 의미에서만 사용하는 것이다.
오늘날 법학도의 관심 대상인 법은
분명
공동체의 일부분으로서 저항할 수 없는 강제력을 가진
주권자의 실제적 명령이거나, 아니면
`주권자가 허락하는 것은 주권자가 명령하는 것'이라는 공식에 의해
`법은 명령이다'라는 공식에 포섭되는 인류의 관습일 것이다.
% 375
저 법학자의 관점에서는,
일군의 작위 또는 부작위를 규정한다는,
일반적으로 정해진 다수의 작위 또는 부작위를 규정한다는,
진정한 법의 필수요건을 통해서만 법이 질서와 연결될 뿐이다.
단일한 행위만 규정하는 법은 진정한 법이 아니라
`임시의' 또는 `특별한' 명령이고 법과 구분된다.
이렇게 정의되고 한계지워진 법이 바로 분석법학자들이 생각하는
법학의 대상인 것이다.
지금 우리는 그들 체계의 기초에 대해서만 관심이 있거니와,
본 강의에서 내가 제기하고자 하는 질문은 이런 것이다:
법에의 복종을 강제하는 힘은
주권자의 강제력과 동일시할 수 있는 성질을 언제나 가지고 있었는가?
그리고 자연현상을 기술하는 물리법칙이나 일반공식과 유일하게 연결된다는
저 일반성을 언제나 가지고 있었는가?
이들 질문은 일견 우리를 너무 먼 곳으로 데려가는 듯이 보일 수 있겠지만,
틀림없이 여러분들은 결국 그것들이 흥미롭고 중요한 질문이란 것을,
어떤 경우에는 우리가 논의해온 사변의, 이론적 가치가 아닌,
현실적 가치의 한계를 밝혀준다는 것을
깨닫게 될 것이다.

\para{정부들의 복잡성 증가}
분석법학자들이 말하는 주권을 다시 살펴보자.
오스틴의 논저를 읽은 독자들은
그가 다수의 기존 정부들을,
\paren{그의 용어로는} 상급의 정치형태들과 하급의 정치형태들을
% 376
분석했음을 기억할 것이다.
그들 각각에서 주권의 정확한 소재를 파악하려는 목적에서였다.
이것은 그의 저술 중에 가장 흥미로운 부분이며,
그의 지성과 독창성이 유감없이 발휘된 부분이다.
이 문제는 홉스 시절보다 복잡해져 있었고, 심지어
벤담의 초기 저술활동 시절보다도 복잡해져 있었다.
영국의 당파주의자 홉스는 대륙의 정치현상에 대한 날카로운 과학적 관찰자였고,
거기서 그가 관찰한 정치상황은 \paren{영국을 제외하면}
사실상 전제정과 무정부상태 양자에 국한되었다.
그러나 오스틴의 시대에 이르면,
홉스가 그의 원리를 관철시키는 투쟁의 장으로 여겼을 영국은
오래 전부터 `제한군주정'\latin{limited monarchy}이 되어있었는데,
홉스의 후계자들은
이 표현\latin{expression}을 혐오했다.
홉스 자신이 그러한 \hemph{상태}\latin{thing}를 싫어했듯이 말이다.
게다가 제1차 프랑스혁명의 영향력이 나타나기 시작하고 있었다.
프랑스는 근래에 제한군주정이 되었고,
거의 모든 다른 대륙 국가들도 제한군주정이 될 조짐을 보였다.
대서양 건너편에서는 미국의 복잡한 정치체제가 등장했고,
대륙 유럽에서는 독일연맹과 스위스연맹의 더 복잡한 정치체제가 등장했다.
주권의 소재를 파악하기 위한 정치사회들의 분석이
확실히 훨씬 더 어려워진 것이다.
% 377
그러나
오스틴이 기존 사례들을 분석해낸 것을 능가할 수 있는 것은 아무 것도 없다.

\para{무정부 상태}
그럼에도 불구하고 오스틴은
아무리 살펴보아도 주권에 관한 자신의 정의에 부합하는 사람 또는 집단을
발견할 수 없는 공동체나 사람들의 집합체가 존재한다는 것을 전적으로 인정한다.
우선, 홉스처럼 그도 무정부상태가 존재한다는 것을 인정한다.
그런 상태가 발견되는 곳이라면 어디나
주권의 문제는 격렬한 투쟁의 대상이다.
오스틴이 든 예는 홉스의 정신을 사로잡았던 것이기도 한데,
바로 찰스 1세와 의회 간의 투쟁이었다.
홉스와 오스틴을 날카롭게 비판한 피츠제임스 스티븐\latin{Fitzjames Stephen}
같은 이는 \hemph{휴면 중인}\latin{dormant} 무정부상태가 있다고 주장한다.
이러한 유보는 남북전쟁 직전의 미국의 상태를 염두에 둔 것이 틀림없다.
당시 주권의 소재는 수년간 언론이나 신문을 통한 격렬한 투쟁의 대상이었고,
다수의 저명한 미국인들이 첨예하게 대립하는 원칙들의 차이를 봉합하여
그럼에도 불구하고 불가피했던 투쟁을 연기하는 조치로 명성을 얻었다.
해결되지 않은 문제를 두고 싸우기를 의도적으로 자제하는 일이 충분히 있을 수
있거니와, 이렇게 해서 만들어진 일시적 평정상태를 휴면 중인 무정부상태라고
부르지 못할 까닭은 없을 것이다.
나아가 오스틴은 자연상태의 이론적 가능성까지 인정한다.
그는 홉스 등의 사변가가 자연상태에 부여하는 중요성 같은
큰 중요성을 자연상태에 부여하지는 않는다.
그러나 다수의 사람들이, 또는
정치사회를 이룰 정도로 크지 않은 다수의 집단들이,
아직 공통의 또는 습관적으로 복종하는 권위자를 갖고 있지 않을 때
자연상태가 존재한다고 인정한다.
그런데
정치사회를 이룰 정도로 크지 않은 집단들을 말한 직전의 문장에서
나는 인류에게 주권이 보편적으로 존재한다는 규칙에 대해 오스틴이 중대한 예외를
인정했음을 소개한 셈이다.
관련 대목은 그의 저서 제3판 제1권 237쪽에 등장한다:

\para{자연상태}
``야만인 가족 하나가 다른 모든 공동체로부터 고립되어 살고 있다고 가정해보자.
또한 이 고립된 가족의 수장인 아버지가 어머니와 자식들로부터
습관적 복종을 받고 있다고 가정해보자.
다른 더 큰 공동체의 일부가 아니므로,
부모와 자식들로 구성된 이 사회는 분명 독립된 사회이다.
또한 나머지 구성원들이 습관적으로 수장에게 복종하고 있으므로,
구성원 숫자가 너무 적지 않다면, 이 독립된 사회는 정치적 사회를
구성할 수 있을 것이다.
그러나 구성원의 숫자가 너무 \hemph{적기} 때문에, 자연상태의 사회라고
부르는 것이 마땅할 것이다.
즉, 복종의 상태에 놓여있지 않은 사람들로 구성된 사회인 것이다.
% 379
부조리한 듯한 용어를 사용하지 않으려면,
우리는 이 사회를 \hemph{정치적인} 독립된 사회라고 부를 수 없을 것이고,
명령하는 아버지와 수장을 \hemph{군주} 또는 \hemph{주권자}라고
부를 수 없을 것이며,
또한 복종하는 어머니와 자식들을 \hemph{백성}이라고 부를 수 없을 것이다.''

그러고는 오스틴은 몽테스키외의 원칙을 인용한다:
``정치권력은 여러 가족들의 결합을 반드시 전제한다.''

\para{아주 작은 집단에는 주권자가 없다}
이 대목이 의미하는 바는 아주 작은 사회에는 그의 이론을 적용할 수
없다는 것이다.
이런 경우에 그의 용어를 적용하는 것은 부조리하다고 오스틴은 말한다.
나중에 나는 일반적으로는 가장 위험한 기준인 우리의 부조리 감각에
이처럼 호소하는 것의 의미에 대해
여러분에게 지적할 기회가 있을 것이다.
당장은 이러한 인정이 대단히 중대한 것이라는 점만 주목해주시기 바란다.
여기에 등장하는 권위의 형태, 즉
가족에 대한 가부장\latin{patriarch; paterfamilias}의 권위야말로,
적어도 오늘날의 어떤 한 이론에 따르면,
인간의 인간에 대한 모든 상설적 권력이 점차 발달해나오는
요소 혹은 맹아이기 때문이다.

\para{펀잡 주}
하지만 \paren{비록 오스틴이 근대적 저술가이긴 해도}
그가 저술할 당시에는 거의 알려지지 않았다고 해야 할 지식의 원천에서 유래하는
또 다른 사례들이 있다.
이들은 그의 원리를 적용하기가 적어도 곤란한 또는 의심스러운 사례들이다.
그 중 하나를 인도에서 가져올 것인데,
인도를 특별히 애정해서가 아니라
% 380
이 주제에 관한 가장 최근의 사례이기 때문이다.
그것은 인도에서도
다섯 개의 강이 흐르는
펀잡 주의 사례로,
영국령 인도에 편입되기 사반세기 전의 상태에 관한 것이다.
상상할 수 있는 모든 형태의 무정부상태와 휴면 중의 무정부상태를 거친 후,
펀잡은
시크교도들의
반쯤은 군사적이고 반쯤은 종교적인
과두정에 의해 어느 정도 안정적인 지배체제를 수립한다.
그후 시크교도들은 그들 신분에 속하는 한 사람의 수장에게
복속하게 되는데, 이 사람이 바로 란지트 싱\latin{Runjeet Singh}이다.
언뜻 보면 란지트 싱만큼 오스틴의 주권자 개념에
딱 들어맞는 경우도 없을 것 같다.
그는 절대적 전제군주였다.
가끔씩 멀리 변방 지역을 제외하면 그는 자못 완벽하게 질서를 유지했다.
그는 무엇이든 명령할 수 있었다.
그의 명령을 조금이라도 어기면 사형이나 절단형이 뒤따랐다.
그리고 그의 백성이라면 누구나 이것을 잘 알고 있었다.
하지만 그가 오스틴이 법이라고 부르는 것을 평생동안 한번이라도
선포한 적이 있는지 의심스럽다.
그는 많은 양의 곡물을 세금으로 거두어갔다.
세금을 잘 내지 않는 마을들을 약탈했고 무수한 사람들을 처형했다.
그는 다수의 군인들을 징집했다.
그는 모든 권력수단을 장악했고 다양한 방법으로 행사했다.
하지만 그는 법을 만들지는 않았다.
백성들의 생활을 규율하는 규칙은
% 381
기억할 수 없는 옛날부터 전해진 관행이었다.
이 규칙들은 가족이나 촌락공동체 내부의 법정에서 집행되었다.
다시 말해,
오스틴의 원리가 그 스스로 인정하는 부조리함 없이 적용될 수 없는 집단보다
크지 않은 혹은 조금만 더 큰 집단의 내부에서 집행되었던 것이다.

\para{란지트 싱}
이러한 정치사회의 존재로 인해 오스틴의 이론이 이론으로서 오류가 된다고는
조금도 주장할 생각이 없다.
그 이론에 대한 반박에 대처하는 위대한 격률이 있거니와,
그것은 내가 누차 말했던 `주권자가 허락하는 것은 주권자가 명령하는 것'이라는
격률이다.
저 시크교도 전제군주는 가\hanja{家}의 수장과 마을의 장로들에게
규칙을 정하라고 허락했고, 따라서
이들 규칙은 그의 명령이고 진정한 의미의 법인 것이다.
그런데,
영국의 주권자는 보통법을 명령한 적이 없다고 주장하는
영국 법률가에게
이런 종류의 대답이
주어진다면 그것은 어느 정도 유효한 대답이 될 것이다.
국왕과 의회가 보통법을 허락하기에 국왕과 의회는 보통법을 명령한다.
그리고 국왕과 의회가 보통법을 허락하는 증거는 그것을 변경할 수 있다는 데 있다.
실제로, 저 반박이 처음 개진된 후,
보통법은 의회 입법에 의해 상당 부분 잠식당했다.
오늘날에는 보통법의 구속력이 제정법에 의존한다고 할 수도 있을 정도이다.
하지만 나의 동양 사례는
옛 법률가들이 보통법에 대해 느꼈던 곤란함이
홉스와 그 후계자들이 말한 것보다 한때는 더 존중받을 만한 것이었음을 보여준다.
% 382
란지트 싱은 백성들이 생활하는 민사규칙을 결코 변경하지 않았고
변경할 꿈도 꾸지 못했다.
어쩌면 그는 마을 장로들이 적용하는 규칙들이
독자적인 구속력을 가진다고 믿었을 수 있다.
어떤 동양의 또는 인도의 법이론가가
란지트 싱이 이들 규칙을 명령했다는 주장을 접하면
그는 오스틴이 정당한 호소라고 인정했던 바로 그 부조리 감각에 의해
자극을 받을지도 모른다.
이런 경우 이론은 여전히 진리이겠으나,
그것의 말뿐인 진리에 지나지 않는다.

언어를 잡아늘이지 않고는
주권이론과 이에 기초한 법이론이
적용되지 않는 몇몇 특수한 사례를 가지고 유별난 사변을
내가 전개하고 있다고 생각해서는 안 된다.
우선, 란지트 싱 통치 하의 펀잡은
토착적 상태에 있는 모든 동양 공동체의 전형이라고 볼 수 있다.
평화와 질서가 수립된 드문 기간 동안은 말이다.
그것들은 전제정이었고,
수장인 전제군주의 명령은 엄격하고 잔인했으나
언제나 절대적으로 복종되었다.
그러나 이들 명령은,
세금 징수를 위한 행정기구를 조직하는 경우를 제외하면,
진정한 의미의 법인 적이 없다.
그것은 오스틴이 말한 임시의 또는 특별한 명령이었을 뿐이다.
사실,
우리가 알고 있는 세계에서
지방적·가족적 관행을 희석시키는 유일한 용매는
주권자의 명령이 아니라 신의 명령으로 여겨지는 것이었다.
인도에서 법과 종교의 혼합물을 다룬 브라만의 문헌은
옛 관습법을 무너뜨리는 데 있어 언제나 커다란 영향력을 발휘했다.
특히, 앞선 강의에서 설명하였듯이, 영국 통치 하에서
그것의 영향력은 더욱 커졌다.\footnote{%
  인도를 통치하게 된 영국은 인도에도
  획일적인 법규칙이 존재할 것이란 생각에서
  브라만들을 동원하여 마누법전 등 옛 문헌에서 법규칙들을
  발견하도록 했고, 영국에 의해
  새로 설치된 법원들은 이들 법규칙을 각 지방에 강제했다. }

\para{고대세계의 상태, 고대 제국}
지금 우리의 탐구를 위해서는,
우리 눈앞에 보이는 현대 서구의 사회조직보다
내가 인도 혹은 동양의 정치사회로 기술한 사회상태가
세계 대부분의 과거 상황을 알 수 있게 해주는 더 믿을 만한 단서가 된다는 점을
명심할 필요가 있다.
현대세계보다 고대세계에서 주권은 더 단순했고 더 쉽게 발견된다는 것은
터무니없는 생각이 아닐 것이다.
홉스와 오스틴의 비판자로 내가 언급했던 이는 이렇게 말한다.
``그리스든, 페니키아든, 이탈리아든, 아시아든,
내가 읽어서 알고 있는 모든 국가에는
일종의 주권자가 있어서 그것이 존속하는 동안 절대적 권위를 행사했다.''
또 이어서 말하기를,
``홉스는 가상의 인류역사를 쓰고자 했지만
로마 제국의 창설과 확립의 역사만큼 그의 목적에 더 들어맞는 것을
만들어낼 수는 없었다.''
로마 제국에 관한 것은 잠시 미뤄두겠는데,
그렇게 하는 이유는 나중에 밝혀질 것이다.
그런데,
영토의 범위에서 로마 제국에 맞먹는 제국들을 살펴보면
그것들은, 올바르게 이해한다면,
홉스가 상상한 리바이어던과 전혀 비슷하지 않았음을
알 수 있다.
우리는 유대인의 기록에서 아시리아 제국과 바빌로니아 제국에 대해
어느 정도 알고 있다.
또한 그리스인의 기록에서 메디아 제국과 페르시아 제국에 대해
어느 정도 알고 있다.
이를 통해 우리는 그들이 조세징수제국\latin{tax-taking empire}이었음을 알고 있다.
그들이 막대한 양의 세금을 백성들에게서 징수했음을 알고 있다.
가끔씩 정복전쟁을 위해서 그들은 방대한 지역에 살고 있는 인구로부터
대규모 군사력을 징집했음을 알고 있다.
그들의 임시적 명령에 대해서는 절대적 복종을 강요했고, 위반에 대해서는
극도로 잔인한 처벌을 가했음을 알고 있다.
그들의 수장인 군주는 지속적으로 소규모 왕들을 폐위시켰고,
심지어 공동체 전부를 이식하기도 했음을 알고 있다.
하지만 이런 와중에도,
백성들이 속하는 집단들의 종교적·세속적 일상생활은
대체로 거의 간섭하지 않았음이 확실하다.
`고칠 수 없는 메디아와 페르시아의 법'의 한 예로
우리에게 전해지는 `왕법'\hanja{王法}과 `금령'\hanja{禁令}은
근대법학이 말하는 법이 아니었다.\footnote{다니엘서 \latin{6:7--8.} }
그것은 오스틴이 `특별한 명령'이라 부른 것으로
갑작스럽고 돌발적이고 일시적인 개입이었으나,
고대의 다양한 형태의 관행들은 대체로 그대로 유지했다.
더욱 시사적인 것은
% 385
유명한 아테네 제국이
저 대왕의 제국과 동일한 주권 유형에 속한다는 점이다.
아테네 민회는 아티카 지역의 주민들을 위해서는 진정한 법을 만들었지만,
아테네 지배 하에 있는 도시와 섬들에 대해서는 분명
조세징수적이었다.
그것은 입법제국\latin{legislaing empire}이 아니었다.

\para{분석적 체계의 한계}
이 위대한 정치체들에 오스틴의 용어를 적용하는 것의 곤란함은 충분히 명백하다.
유대법이 수사에 거처하는 대왕에 의해 명령되었다고 말하는 것이
과연 명석한 사고를 낳을 수 있겠는가?
분석법학의 중추가 되는 규칙, `주권자가 허락하는 것은 주권자가 명령하는 것'이란
규칙은 말로는 여전히 진리이다.
하지만 이런 사례에 그것을 적용하면
오스틴이 인정한 상위의 법정, 즉 부조리의 감각에 상소가 허용되는 것이다.

이제 나는
분석법학의 체계가
갖는 현실적 한계에 관한 나의 의견을 편하게 말할 수 있는
지점에 도달했다.
그것은 이론적 진리가 아니라 현실적 가치를 갖는다고
주장될 때의 한계이다.
분석법학의 유일한 관심대상인
서구세계는 두 가지 큰 변화를 겪었다.
근대 유럽의 국가들은 \paren{하나를 제외한} 고대의 위대한 제국들 및
오늘날 동양의 제국이나 왕국들과는 다른
방식으로 구성되었다.
또한 \hemph{입법}이라는 주제에 관한 새로운 관념 질서가
로마 제국을 통하여 이 세계에 전해졌다.
이러한 변화가 없었다면 분석법학의 체계는 그 저자들의 머리에
들어서지 않았을 것이라고 나는 확신한다.
이러한 변화가 일어나지 않은 곳에서는
분석법학의 체계는 무가치할 것이라고 나는 확신한다.

\para{초기 공동체들, 근대국가의 형성}
국가라는 정치공동체의 기원에 관해 주장할 수 있는 거의 보편적인 사실은
그것이 집단들의 결합으로 구성되었고, 그 초기 집단은
가부장적 가족보다 작지 않은 집단이었다는 것이다.
그러나 로마 제국 이전에 존재한 공동체에서는,
그리고 로마 제국의 영향을 조금만 받았거나 전혀 받지 못한 곳에서는
이러한 결합이 곧 정지되었다.
이 과정의 흔적은 모든 곳에서 발견된다.
아티카의 촌락들이 결합하여 아테네 국가를 형성한다.
초기 로마 국가는 일곱 언덕의 작은 공동체들의 결합한 것이다.
다수의 인도 촌락공동체에서는 더 작은 요소들이 결합해 구성된 흔적이 보인다.
그러나 초기의 결합은 곧 정지된다.
후대에 이르러,
로마 제국과 외면적 유사성을 갖고 또한 대체로 무척 큰 영토를 갖는
정치공동체들은
한 공동체가 다른 공동체를 정복함으로써,
또는 한 공동체나 부족의 족장이
% 387
대규모 인구를 정복함으로써
만들어졌다.
그러나, 로마 제국과 그 영향권을 제외하면,
이 거대 국가에 포함된 작은 사회들의 독립된 지방적 삶은
사라지지 않았고 약해지지도 않았다.
그들은 인도의 촌락공동체가 존속해왔듯이 그대로 존속했다.
국가가 가장 번영했을 때조차 그들은 기본적으로 동일한 사회유형을 유지했다.
하지만 근대세계의 국가들이 형성될 때의 변화과정은 전혀 달랐다.
작은 집단들이 훨씬 더 철저하게 해체되었고 더 큰 집단에 흡수되었다.
더 큰 집단은 더욱 큰 집단에 흡수되었으며,
다시 후자는 더 넓은 지역으로 통합되었다.
지방적 삶과 마을의 관습이 모든 곳에서 동일한 정도로 쇠퇴한 것은 아니다.
독일보다 러시아에서 더 많이 살아남았다.
영국보다 독일에서 더 많이 살아남았다.
프랑스보다 영국에서 더 많이 살아남았다.
그러나 대체로 볼 때,
근대국가가 형성될 때면, 그것은
초기 제국을 구성하던 것들보다 훨씬 더 작은 파편들이 모인 것이다.
그리고 구성부분들 사이에는 유사성이 훨씬 더 컸다.

\para{마을회의}
어느 것이 원인이고 어느 것이 결과인지 확신하기는 어렵지만,
한때 독립된 삶을 살았던 집단들이 근대사회에서 더 철저히 분쇄되는 과정은
더 활발한 입법활동과 병행하여 진행되었다.
아리아인의 원시적 상황이 역사 기록이나 고대법의 유산을 통해
드러난 곳이면 어디서나,
원초적 집단 안에서 오늘날의 입법부에 해당하던 기구가 확인된다.
그것은 마을회의\latin{village council}이다.
그것은 때로는 마을 주민 전체에 대해 책임을 졌고
때로는 그렇지 않았으며
때로는 세습족장의 권위의 그림자에 가려지기도 했지만,
결코 완전히 사라지지는 않았다.
이 맹아로부터 세계의 모든 유명한 입법기구들이 출현했다.
아테네의 에클레시아\latin{Ekklesia}가, 로마의 민회, 원로원, 황제가,
그리고 영국의 의회가 출현했다.
이는 근대세계의 모든 \paren{오스틴의 용어로}
`동료 주권자 집단'\latin{collegiate sovereignties}의,
다시 말해 주권이 인민에 의해 또는 인민과 왕의 협치로 행사되는
모든 정부의 전형이자 모태였다.
하지만,
이 국가기구의 저발전된 형태를 살펴보면,
그것의 입법능력은 무척 불명확하고 무척 미약하다.
사실, 다른 곳에서 내가 말했듯이,
고유한 관념의 제국 아래에서
마을회의에 주어진 권한의 다양한 색깔들은 서로 잘 구분되지 않는다.
법의 작성, 법의 선포, 그리고 법 위반에 대한 처벌 사이에도
뚜렷한 차이를 찾아보기 어렵다.
이 기구의 권한을 근대적 이름으로 굳이 표현한다면,
가장 배경에 놓여있는 것은 입법권한이고,
가장 뚜렷하게 드러나는 것은 재판권한이라 할 수 있다.
복종의 대상인 법은 항상 존재해온 것으로 간주되고,
새로운 관행은 옛 관행과 잘 구분되지 않는다.

\para{원시 집단과 입법, 국왕의 법}
따라서 아리아인의 촌락공동체는, 그 원시적 상태에 머물러있는 한,
진정한 입법권을 행사하지 않는다.
또한,
원시적인 지방적 집단들을 거의 그대로 보존하는
동양의 대규모 국가들에서 주권자들이 행사하는 입법권한은
용어의 지성적 의미에서의 입법권이라 부르기 어렵다.
전술했듯이 입법과 지방적 삶의 해체는 일반적으로 서로 병행했다.
인도인의 촌락공동체와 영국의 튜턴족 촌락공동체를 비교해보라.
전자는, 근대적이 아닌 그리고 영국에 의해 만들어진 것이 아닌
모든 제도 중에서,
가장 명확하고 가장 강하게 눈에 띄고 가장 잘 조직되어 있는 것이다.
영국의 옛 공동체인 후자의 흔적은, 물론 추적할 수는 있지만,
비교방법을 동원해야만 그리고
여러 세기에 걸친 성문법과 성문의 역사를 조사해야만
비로소 그 의미가 이해될 수 있고
해체된 윤곽이 복원될 수 있다.
동일한 제도의 이렇게 서로 다른 생명력을 두 나라의 어떤 다른 현상과
관련짓지 않는다는 것은 불가능하다.
인도에서는
수많은 초기 정복자들에 이어
% 390
무굴 제국과 마라타 제국이
촌락공동체들을 휩쓸고 지나갔지만,
촌락공동체들을 명목적 제국에 포함시키고 나서도
그들은 세금과 공물을 바치는 것말고는 어떤 항구적인 의무도 부과하지 않았다.
몇몇 예외적인 경우에
피정복민들에게 종교의 개종을 강요하긴 했지만,
마을에서는 기껏해야 사원과 종교의례가 바뀌었을 뿐,
민사적 제도는 그대로 유지되었다.
영국에서는 중앙권력과 지방권력 간의 투쟁이 사뭇 다른 양상을 보였다.
국왕의 법과 국왕 법원이 지방의 법과 지방 법원에 대항하여
지속적으로 강화되어왔음을 우리는 잘 안다.
그리고 국왕 법의 승리가 그후 그것의 원리에 기초한
일련의 의회입법의 긴 목록을 이끌어냈음을 우리는 잘 안다.
이 모든 과정은 입법이 점점 강화되는 과정이라고 할 수 있다.
결과적으로 고대의 다양한 지방법은 거의 전적으로 폐지되었으며,
독립된 공동체들의 옛 관행들은 장원의 관습으로 전락하거나
아니면 법적 제재가 주어지지 않는 단순한 습관으로 전락했다.

\para{로마의 입법}
로마 제국이 중앙집권적이고 적극적으로 입법활동을 하는 국가의 형성에
직·간접적으로 영향을 준 원천이었다는 것은 충분히 믿을 만한 명제이다.
로마 제국은 조세부과뿐만 아니라 입법활동도 행한 최초의 대국\hanja{大國}이었다.
이 과정은 여러 세기에 걸쳐 확대되어갔다.
% 391
그것의 시작과 완성의 시점을 굳이 지적해야 한다면,
나는 대체로
최초의 속주고시\hanjalatin{屬州告示}{edictum provinciale}가 발해진 때와
로마 시민권을 제국의 모든 신민에게 확장한 때를 지적하고 싶다.
하지만 분명 이 변화는 시작 시점 훨씬 이전에 그 토대가 놓여졌고,
또한 종료 시점 이후에도 다방면으로 오랫동안 이어졌다.
결과적으로
관습법의 방대하고도 잡다한 덩어리가 해체되었고
새로운 제도에 의해 대체되었다.
이런 점에서 로마 제국은 다니엘서를 가지고 정확하게 묘사될 수 있다고 생각한다.
그것은
먹이를 잡아먹고, 으스러뜨리고, 그 나머지를 발로 짓밟아 버렸다.\footnote{%
  다니엘서 \latin{7:19.} }

\para{법의 힘}
만족\hanja{蠻族}들의 로마 제국 침공은
잃어버렸던 다양한 원시적 부족 관념과 촌락 관념을
제국 내의 공동체들에 확산시켰다.
그렇지만
로마 제국에 의해 직·간접적으로 영향받은 사회는
동양의 정체성\hanja{停滯性}으로 인해 오늘날에도 관찰할 수 있는
고대적 체제에 기초한 사회와 더 이상 같지 않았다.
전자의 사회에서 주권은 입법권과 어느 정도 분명하게 관련되어 있다.
그리고
대다수 국가에서 입법권이 행사되는 방향은 제국이 남겨놓은
법에 의해 분명하게 그어져 있었다.
고대적 법관념을 거의 전부 추방해버린 로마법은
% 392
어디서나 분명 지방적 관행을 희석시키는 강력한 용매였다.
그리하여 두 가지 유형의 정치사회가 존재하게 된다.
그 중 고대적 유형에서는
대다수 사람들이 자신의 마을이나 도시의 관습을 생활규칙으로 삼는다.
하지만 그들은 조세는 징수하지만 입법은 하지 않는 절대적 주권자의 명령에
가끔씩, 그러나 절대적으로 복종한다.
우리에게 친숙한 다른 유형에서는
주권자는 자신의 원리에 따라 더욱더 적극적으로 입법을 행한다.
그러는 사이 지방적 관습과 지방적 관념은 급속히 사라져간다.
하나의 정치체제에서 다른 정치체제로 변화하는 동안
법의 성격도 뚜렷이 변화되었다고 생각한다.
예컨대, 법의 배후에 존재하는 힘은
언어를 잡아늘임으로써만 동일한 언어로 부를 수 있게 되었다.
관습법---이 주제는 오스틴의 이론이 별로 도움이 되지 않는다---이 복종되는
방식은 제정법이 복종되는 방식과 같지 않다.
관습법이 좁은 지역과 작은 자연적 집단에 적용될 때,
그것의 제재는 부분적으로 여론에, 부분적으로 미신에,
그러나 주로는 우리의 신체운동 일부가 일어날 때와 비슷한 맹목적이고
무의식적인 본능에 의존한다.
관행에의 복종을 확보하는 데 직접적인 통제는 거의 필요치 않다.
하지만, 작은 자연적 집단의 외부에서 권위자에 의해 부과되는 규칙이
% 393
복종을 요구할 때, 그 규칙은 관습법과 전혀 다른 성질을 띤다.
미신의 도움은 사라지고 여론의 도움도 별로 받지 못하며
자발적 충동의 도움은 확실히 받지 못한다.
따라서 법 배후의 힘은 원시적 유형의 사회에서는 볼 수 없는 정도로
순수하게 강제력이 된다.
더욱이 다수의 국가에서
이 힘은 그 대상인 사람들로부터 아주 멀리 떨어져서 행사되어야 하기에,
이 힘을 행사하는 주권자는
개별적 행위와 개인들이 아니라
아주 광범위한 행위와 아주 광범위한 사람들을 다루어야 한다.
이러한 요구의 결과의 하나가
흔히 법과 불가분 결합되어 있다고 여겨지는 성질,
즉 법의 무차별성, 법의 냉혹성, 그리고 법의 일반성인 것이다.

\para{법과 질서}
법과 관련된 힘의 개념이 변화함에 따라
질서의 개념도 변화했다고 생각한다.
아리아인의 초기 사회집단에서
마을 관습만큼 보편적으로 나타나는 현상도 없을 것이다.
그렇지만, 촌락공동체를 구성하는 가\hanja{家} 안에서는
가부장의 전제정이
관행의 전제정을 대신한다.
경계선 밖에서는 기억할 수 없을 만큼 오래된 관습이 맹목적으로 복종되지만,
내부에서는
가부장권이
반쯤 문명화된 남자에 의해
아내와 자식과 노예에 대해 행사되었다.
법이 명령이라면,
이 단계의 법은 불변의 질서보다는 예견불가능한 변덕과 관련되어 있었다.
또한
당시의 사람들은
낮과 밤, 여름과 겨울 등
자연현상의 연속성에서
규칙성을 기대했을 뿐,
강제력을 가진 상급자의 말과 행위에서는 규칙성을 기대하지 않았을 것이다.

\para{힘과 질서의 변화}
따라서 법의 배후에 존재하는 힘이 항상 동일한 것은 아니었다.
법에 수반하는 질서도 항상 동일한 것은 아니었다.
대중의 견해와 분석법학자의 예리한 눈에
법의 본질적 속성이라 여겨지는 것들은 오직 서서히 법에 부착되어갔다.
법의 일반성과 주권자의 강제력에의 의존성은
광대한 영토를 가진 근대국가의 결과물이고,
국가를 구성하는 하위집단들이 해체된 결과물이며,
무엇보다 민회·원로원·황제가 통치하는 로마 국가의 모범과 영향력의 결과물이다.
아주 일찍부터
로마 국가는
그것이 잡아먹은 것을 철저히 으스러뜨린 점에서
다른 어떤 통치체제나 권력체제와도 현저히 다른 것이었다.

\para{홉스와 벤담}
위대한 사상체계는
그것이 탄생하기 오래 전에 일어난
우연한 사건이 아니면
아무 것도 그 등장을 막을 수 없다는 말이 있다.
분석법학에는
이러한 주장이
타당하지 않다.
% 395
분석법학은 시간이 완전히 무르익은 후에야 비로소 그 저자들의 머리에
들어설 수 있었다.
홉스의 위대한 원리는 분명
당시 그가 가졌던 특별한 기회를 이용한 일반화의 결과일 것이다.
지력이 한창일 때 그는 영국뿐 아니라 대륙에도 장기간 체류했는데,
처음에는 가정교사로서 여행을 다녔고
나중에는 국내정세의 혼란을 피해 망명자로서 지냈다.
분명 그 자신 강한 당파성을 가졌던 영국의 사태와는 별개로,
그가 관찰한 현상은 빠르게 중앙집권화되고 있는 정치체들이었다.
지방의 특권과 재판권은 완전히 소멸해가고 있었다.
프랑스의 파를르망\latin{Parliament} 같은 낡은 역사적 기구는
무정부상태의 용광로가 되고 있었고,
질서를 위한 유일한 희망은 왕권에서 찾을 수밖에 없었다.
이들은 베스트팔렌 조약을 낳은 전쟁의 명백한 결과물이었다.
봉건적 또는 준\hanja{準}봉건적 사회의 다양한 옛 지방세력들은
어디서나 쇠퇴했거나 파괴되어 있었다.
그것이 지속되었다면 저 위대한 사상가의 이론체계는 분명
탄생하지 못했을 것이다.
우리는 햄든\latin{Hampden} 마을에 대해서는 들어보았지만
홉스 마을은 상상할 수도 없다.
벤담이 저술활동을 하던 시대에 이르면
분석법학을 연상시키는 상황은 훨씬 더 명확해졌다.
프랑스의 법전편찬 사업이
민주정이었던 주권자에 의해 시작되어
전제정이었던 주권자에 의해 완성되었던 것이다.
% 396
주권자가 묵시적으로 허락한 것을
그는 언제라도 명시적 명령으로 대체할 수 있기에
주권자가 허락하는 것은 주권자가 명령하는 것이다, 라는 명제를
이처럼 선명하게 보여준 사례는 근대세계에서
이전에 존재한 적이 없었다.
또한 주권자의 진정한 의미의 입법활동의 증가로부터 기대되는
광범위하고 대단히 유익한 결과의 면에서
이처럼 인상적인 본보기는 존재한 적이 없었다.

\para{분석법학과 역사}
동일한 반열에 속하는 천재 중에서
홉스와 벤담만큼 역사로부터 완전히 결별한 인물도 없을 것이다.
어쨌든 내가 보기에는,
그들만큼 자신들이 보고 있는 대로 항상 이 세상이 존재해왔다고 생각한 사람도
없을 것이다.
벤담은 자신의 원리의 불완전한 또는 왜곡된 적용이 낳은 많은 것들은
자신의 원리와 아무 상관이 없다는 생각에서 벗어나지 못했다.
홉스는
특권 단체와 조직화된 지방 집단을
당시 점차 인기를 얻고 있던 생리학에 의해
인간 신체의 내부 조직 안에 살고 있다고 증명된
기생충에 비유했는데,
\paren{당시에는 자연스러운 일이었으나}
이것만큼 역사에 대한 오해를 분명히 보여주는 것이 또 있는지 의문이다.
오늘날의 우리라면,
생리학의 비유를 채용해야 한다면,
저 집단들을
인간 신체를 구성하는 기본 세포들에
비유할 것이다.

\para{분석법학의 영향}
역사의 도움으로써만 설명될 수 있는 많은 것들을
분석법학자들은 보지 못했지만,
말하자면
역사의 바다에서 표류하는 사람들이
오늘날에도
불완전하게만 보는 많은 것들을
그들은 보았다.
사실로서의 주권과 법은 오직 서서히
홉스, 벤담, 오스틴이 형성한 개념에 부합하는 형태를 띠어갔다.
그러나 그들 시대에 이르면 그러한 부합이 실제로 존재했고
계속해서 더 완전해지고 있었다.
그리하여 그들은
내적으로 엄격한 일관성을 갖는다는 장점을 가진 법이론을 만들 수 있었다.
그것의 또 하나의 장점은,
그것이 사실을 정확히 표현하지 못하더라도,
이러한 정확성의 한계가 그것의 가치를 박탈할 정도로
심각한 것이 되지 않을 뿐만 아니라
시간이 흐르면서 점점 덜 중요해진다는 것이다.
법과 사회에 관한 어떠한 개념도
의심할 여지 없는 망상의 덩어리를 제거한 적이 없었다.
주권자가 가지는 힘은 이들 법학자가 이해한 법을 통해 주로 행사된 것이 사실이나,
그것은 당황하며, 주저하며, 많은 실수와 방대한 누락을 품은 채 행사되었다.
만약 그것이 대담하게 그리고 일관되게 적용된다면
이루어낼 수 있는 모든 것들을
그들은 처음으로 보았다.
그후 그들의 현명함을 증거하는 모든 일들이 일어났다.
벤담 시대 이래 이루어진 법개혁 중에 그의 영향을 받지 않은 것이 하나라도 있는지
의문이다.
그러나 이 이론체계가 산출한
명석함의 더욱 놀라운 증거는, 앞선 시대에 속하지만, 홉스에게서 찾을 수 있다.
홉스는 <<보통법의 대화>>\latin{Dialogue of the Common Laws}\footnote{%
  1666년 쓰여졌으며 1681년 출간된
  <<철학자와 보통법학자의 대화>>(A Dialogue between
  a Philosopher and a Student of the Common Laws of England)를 말한다.
}에서
보통법과 형평법의 통합, 부동산권 등기 제도, 체계적인 형법전을 주장했다.
이들 세 가지 조치는 지금 이 순간 막 실현되려 하고 있다.\footnote{%
  그러나 체계적인 형법전은 2022년 현재까지도 실현되지 않고 있다. }

\para{근대국가의 입법, 공리주의 철학, 윤리론자로서의 벤담}
근대국가의 구조에서 가장 핵심적 사실은 활발한 입법기구이다.
전술했듯이 이 사실이 존재하기 전까지는
홉스, 벤담, 오스틴의 체계는 생각해낼 수 없을 것이다.
이 사실이 불완전하게 발현되는 곳에서는 그들의 체계는 올바르게
평가받지 못할 것이다.
독일 학자들이 이것을 상대적으로 소홀히 취급하는 것은
독일에서는 입법활동이 상대적으로 최근의 일이기 때문일 것이다.
그러나
벤담과 오스틴이 홉스의 사변에 추가한 저 유명한 이론을
고려하지 않고는 입법과 분석법학 간의 관련을 논의하는 것이 불가능하다.
그들이 추가한 것은 바로 공리\hanja{功利}---법과 도덕의 기초로서의
최대다수의 최대행복---의 이론이다.
그렇다면 공리주의와 분석법학 간의, 본질적인 것이든 역사적인 것이든,
관계는 무엇인가?
강의를 마쳐야 할 시점에 와서 그렇게 폭넓고 어려운 주제를
상세히 논할 생각은 없지만, 그래도 몇 마디 이야기하겠다.
내가 보기에 공리주의에 관하여 가장 흥미로운 점은
그것이 평등 이론을 전제한다는 것이다.
`최대다수'는 한 사람을 단위로 하여 계산하는 최대다수를 말한다.
``하나는 하나로만 취급되어야 한다''고 벤담은 누차 강조했다.
사실, 이 이론에 대한 가장 결정적인 반박은 이러한 평등성을 부인하는 것이
될 것이다.
나는 인도의 브라만에 속하는 어떤 이가 이를 부인하는 것을 들은 적이 있는데,
그들 종교의 분명한 가르침에 따르면
브라만의 행복은 다른 사람의 행복의 스무 배의 가치가 있다는 것이 근거였다.
평등성이라는 근본 가정은 \paren{내가 보기에}
순전히 이기주의에 기초한다는 비판을 함께 받을 만한 다른 이론으로부터
벤담의 이론을
뚜렷이 구별시켜준다.
자, 이 평등성의 근본 가정은 어떻게 해서 벤담의 머리에 떠오르게 되었을까?
분명 그는---누구보다도 명확히---인간은 사실상 평등하지 않다고 보았다.
인간이 자연적으로 평등하다는 명제를 그는
무정부적 궤변\latin{anarchical sophism}이라며 명시적으로 비난했던 것이다.
그렇다면
최대다수의 최대행복이라는 그의 유명한 원리의 공준인
평등성은 어디에서 온 것인가?
최대다수의 최대행복의 원리는 단지 입법의 원리에 불과하다고 감히 나는 생각한다.
애초 벤담은 바로 이런 형태로 그것을 파악했다.
수많은 사람들로 구성된 비교적 동질적인 공동체를 가정해보라.
입법의 형태로 명령하는 주권자를 가정해보라.
이 입법기구가, 현실적인 것이든 잠재적인 것이든,
왕성한 활력을 가지고 있다고 가정해보라.
이런 곳에서
대규모 입법을 안내할 수 있는 유일하게 가능한, 유일하게 생각할 수 있는
원리는 최대다수의 최대행복이다.
그것은 실로 입법의 조건이다.
이 조건은,
법의 어떤 특성과 마찬가지로,
근대 정치사회에서 주권자의 권력이 백성들로부터 멀리 떨어져서 행사된다는 데서
유래한다.
그리하여 사회를 구성하는 단위들 간의 차이를, 진정한 차이까지도,
무시해야 한다는 필요성에서 유래한다.
벤담은 사실 법학자도 아니었고 진정한 의미의 도덕이론가도 아니었다.
그는 법이 아니라 입법에 관한 이론을 전개했다.
면밀히 살펴보면, 그는 심지어 도덕에 관해서도 입법자였다고 볼 수 있다.
물론 그의 언어는 때로 도덕현상을 설명하고 있는 것으로 보인다.
하지만 실제로는 입법에 관한 그의 생각들에서 수집한 규칙에 따라
도덕현상을 바꾸거나 재배열하고자 했다.
이렇게 입법에서 윤리로 그의 규칙이 전환된 점은
도덕적 사실의 분석가로서의 벤담에게 마땅히 주어져야 할
비판의 진정한 근거가 될 것이다.

