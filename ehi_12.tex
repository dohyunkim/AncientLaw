\chapter{주권}

\para{영국의 법학, 분석법학자들}
영국 법률가들이 널리 받아들인 역사 이론은
법학뿐만 아니라 역사학에도
큰 피해를 끼쳤다.\footnote{%
  부록 A는 \latin{Henry Sumner Maine,
  \textit{Lectures on the Early History of Institutions}, 4th ed.,
  London: John Murray, 1885(1874), 제12장 `Sovereignty'}를
  우리말로 옮긴 것이다. }
그리하여
새로운 자료의 조사와 옛 자료의 재조사를 통하여
우리 법체계의 기원과 발달을 설명하는 것은
영국의 지식체계에 추가되어야 할 것들 중에 가장 시급한 것이다.
그러나 새로운 법제사 다음으로 우리에게 가장 필요한 것은
새로운 법철학이다.
우리나라가 새로운 법철학의 탄생에 기여한다면,
그것은 두 가지 장점 덕분일 것이다.
첫째,
우리는 여러 가지 면에서 토착적이라 할 수 있는 법체계를 가지고 있다.
우리는 민족적 자부심에서---때로 이것이 법학의 발달을 늦추고
제한해왔다---로마법대전\latin{Corpus Juris}이라는 위대한 원천에서
흘러나오는 법규칙의 강물에 섞이지 않고 특이하게도 순수한 형태로
우리 법을 유지해왔다. 그리하여
우리 법을 유럽의 다른 법체계와 나란히 놓고 비교하여 얻어지는 결과는
대륙의 법체계들끼리 서로 비교하는 것보다
훨씬 유익한 시사점을 제공해준다.
둘째,
제레미 벤담과 존 오스틴으로 대표되는
이른바 분석법학자들의 연구에 영국인들이 점점 익숙해지고 있다는 것도
장점이라 믿는다.
이 장점은 우리만이 독점하고 있다.
프랑스와 독일에서는
벤담은
인기 없는 도덕이론의 저자로만 알려져 있는 듯하다.
오스틴은 사실 전혀 알려져 있지 않다.
하지만
벤담은, 그리고 오스틴은 훨씬 더 높은 수준으로,
선험적 가정이 아니라 다양한 법적 개념들의 관찰과 비교와 분석에 기초하여
엄격한 과학적 방법으로 법학의 체계를 건설하려 시도한 인물들이다.
이 위대한 학자들의 결론을 맹목적으로 존중하여
그 모두를 받아들일 필요는 조금도 없으나,
그 결론들이 무엇인지 알 필요는 대단히 크다.
단지 명석하게 사고하는 법을 배우기 위해서라도 그것들을 아는 것은 불가결하다.

\para{벤담과 오스틴}
벤담과 오스틴 간의 중요한 차이가 자주 간과되고 있다.
벤담은 주로 입법에 관한 논의를 전개했다.
오스틴은 주로 법학에 관한 논의를 전개했다.
벤담은 있어야 할 법, 있으면 좋은 법이 주된 관심사이다.
오스틴은 있는 그대로의 법이 주된 관심사이다.
둘 다 때로 상대방의 영역을 넘나든다.
벤담이 <<정부론 단편>>\latin{Fragment on Government}이라는 논저를 쓰지 않았다면
오스틴이 자신의 체계의 기초를 확립한
<<법학의 영역 확정>>\latin{Province of Jurisprudence Determined}도
분명 쓰여지지 못했을 것이다.
다른 한편, 오스틴은 신법\hanja{神法}의 지표로서
특이하게도 공리\hanja{功利}의 이론을 들고 나옴으로써
벤담의 탐구영역에 들어갔다.
그렇지만 각자의 목표로 내가 제시한 것은 일반적으로는 충분히 정확하다 할 것이다.
그들의 목표는 서로 상당히 다르다.
벤담은 그 이름과 이제 불가분 결합된 원리를 적용하여 법을 개선하고자 한다.
그가 제시한 중요한 제안들의 대부분이 영국 의회에 의해 수용되었지만,
앞으로도 각 세대는 진보로 생각하는 것들을 법에 접목시키는 과정을
계속 이어갈 것이고 아마도 인류가 존속하는 한 계속될 것이다.
오스틴의 기획은 보다 온건하다.
편제에 있어 완전히 논리적이고 규칙의 진술에 있어 완전히 명료한 법전의
편찬이 이루어진다면 그의 기획은 달성되는 것이다.
법학, 즉 실정법의 과학에 대하여
마치 그것이 법의 내용을 무한히 완전한 상태로 만들 수 있을 것처럼 주장하는
말들이 오늘날 가끔 들려온다.
물론 법학을 철저히 추구하면
불명확과 망상을 추방함으로써
간접적으로 법의 개선에도 크게 기여할 수 있을 것이다.
하지만 법규칙의 내용을 직접적으로 개선하는 원리를 탐구하는 것은
법학의 이론이 아니라 입법의 이론에 속한다.

\para{법학의 영역 확정, 오스틴의 주장의 성격, 주권}
오스틴의 강의들 중
자신의 체계의 기초를 확립한 부분, 몇 년 전
<<법학의 영역 확정>>이라는 이름으로 출간된 부분은
우리 대학에서 오랫동안 상급반 교재로 사용되어왔다.
보다 최근에 \paren{불행히도 단편적인 형태로} 출간된 강의들과 함께,
그것은 앞으로도 오랫동안 우리 학과의 주된 공부 대상이 될 것이다.
이 책의 가치에 대해 충분히 인정하면서도,
초심자에게 그것이 대단히 어렵다는 점을 나는 지적하지 않을 수 없다.
문체의 특징에서 연유하는 어려움,
그리고 벤담과 홉스 같은 그의 선학들의 사상과 끊임없이 연계된다는 데서
연유하는 어려움은 차라리 덜 심각하다.
진짜 어려움은 오스틴의 분석에 나타난 법, 권리, 의무 등의 개념 형태가
우리의 정신에서 거부감을 불러일으킨다는 데서 연유한다.
물론 이러한 거부감이 불쾌한 진리에 기인하는 것이라면
그것에 예민하게 반응하는 것은 시간낭비일 것이다.
그렇다 하더라도 그것은 불행한 일인데,
진술의 방법이나 편제의 방법처럼 피할 수 있는 원인으로 촉발된 것이라면
그것을 없애려 노력하는 고통은 들이지 않을 수 있기 때문이다.
적극적인 정신과 근면한 습성을 가진 학생들에게
어떤 이유로든 그들이 불쾌해하는 체계나 주제를 강요하게 되면
그들은 그것을 일종의 도그마로, 저자의 이름이 갖는 인격적 권위에 기초한 것으로,
간주하게 될 공산이 크다.
<<법학의 영역 확정>>의 체계가 단지 오스틴의 체계로---블랙스톤의
체계나, 헤겔의 체계나, 또는 어느 누구의 체계와 나란히 서있는 체계로,
이들과 대체가능한 또는 대등한 체계로---간주되는 것보다 법철학에게
더 불행한 일은 없을 것이다.
어떤 가정이나 공준이 제시되었을 때
거기서 도출된 오스틴의 입장 대다수는 당연한 것이자 정규의 논리적인 과정을
따른 것이라고 나는 전적으로 확신한다.
그런데
이들 가정은
내가 보기에
오스틴에 의해 충분히 완전하게 진술 또는 기술되지
못했는데, 이는 아마 비록 그가 근대적인 저자이기는 해도
그러한 진술에 필요한 탐구가 당시에는 제대로 시작되지 않았기 때문일 것이다.
그러나, 원인이야 어떻든 간에
그것의 결과는,
정치경제학의 몇몇 위대한 저자들이 애초에 그들 학문의 대상을
명확히 선언해두지 않아서 불가피하게 많은 편견을 불러일으켰다는 비난과
동일한 비난을 그도 받을 만하게 되었다는 것이다.
본 강의는 이러한 가정이나 공준 가운데 몇 가지를 살펴보려는 시도이다.
이하에서 나는 앞선 강의들\paren{제1강부터 제11강까지}에서
초기 사회의 역사를 탐구하면서
얻은 결론이 이들 가정에 어떤 영향을 미치는지 보여주도록 하겠다.
우리의 목적을 위해서는 `주권'의 정의에 주목하는 데서 시작하는 것이 좋겠다.
의심의 여지 없이 이것은 오스틴의 논의의 논리적 순서일 것이다.
한 가지 가설을 제외하면,
왜 그가 홉스의 구성을 버리고는 법, 권리, 의무를 분석하면서
논의를 시작했는지, 그리고 처음에 와야 할 주권에 대한 설명을
마지막에 배치했는지,
이해하기 어렵다.
나는 블랙스톤의 영향이라고 생각한다.
벤담도 블랙스톤으로부터, 이를테면 거부감에 의해, 영향을 받았듯이 말이다.
블랙스톤은 로마법의 법학제요 저술 방식에 따라
법의 정의로부터 시작해서 여러 법개념들의 관계에 관한 이론으로 나아간다.
<<영국법 주해>>의 이 부분의 오류를 드러내는 것이
벤담이 <<정부론 단편>>을 저술하게 된 주된 동기였고,
오스틴이 <<법학의 영역 확정>>을 저술하게 된 주된 동기였다.
오스틴은 블랙스톤의 논의 순서에 따라
그가 제시한 명제들을 반박하는 것이 효과적이라고 판단했던 것 같다.
어쨌거나
나는 오스틴이 첫 강의를 주권의 성질에 관한 탐구로 시작했더라면
그의 분석이 어떻게 달라졌을까 하는 문제를 먼저 다루어보고자 한다.
이 주제 분야를 오스틴은 <<법학의 영역 확정>>에서 제6강, 즉
마지막 강의에서 취급한다.

\para{오스틴의 정의}
여러분들은 오스틴의 저 논저에 개진된 탐구의 일반적 성격에 대해서
잘 알고 계시리라 믿는다.
그러나 그의 정의를 완전한 형태로 기억하기란 쉽지 않기 때문에
`독립된 정치사회'와 `주권'에 대한 그의 기술\hanja{記述}을 인용할 것이다.
이 두 가지 개념은 상호의존적이고 서로 분리불가능하다.

``다른 상급자에게 복종하는 습관을 갖고 있지 않은
어떤 특정의 상급자가
어떤 사회의 대다수로부터 습관적인 복종을 받는다면,
그러한 특정의 상급자는 그 사회의 주권자이고,
그 상급자를 포함한 당해 사회는 정치적이고 독립된 사회이다.''

그러고는 이렇게 이어진다:
``그 특정의 상급자에게 당해 사회의 다른 구성원들은
백성이다. 즉, 그 특정의 상급자에게 당해 사회의 다른 구성원들은
종속한다. 그 특정한 상급자에 대해 다른 구성원들은
복종하는 상태에 놓이고 종속적인 상태에 놓인다.
그 상급자와 다른 구성원들 사이에 존재하는 상호관계는
주권자와 백성의 관계, 즉 주권과 복종의 관계로 표현할 수 있다.''

\para{독립된 정치공동체, 정체의 형태들}
오스틴의 주권 개념을 어떤 다른 방식---정확성을 그다지 해치지 않으면서도
보다 대중적인 방식---으로 진술한다면,
이들 인용문이 들어있는 장\hanja{章}에 나오는 정의들을 부연설명할
필요를 줄일 수 있을 것이다.
그 다른 방식은 이런 것이다:
모든 독립된 정치공동체---자기 위의 어떤 상급자에게 복종하는 습관을
갖고 있지 않은 모든 정치공동체---에는
어떤 한 사람 또는 사람들의 집단이 있어 그 공동체의 다른 구성원들을
원하는 대로 강제할 수 있는 권력을 가진다.
이 한 사람 또는 집단---\paren{오스틴의 용어로는} 이 개인 또는
이 주권자 집단---은 모든 독립된 정치공동체에서
발견될 수 있다.
마치 물체에는 무게중심이 있게 마련이듯이 말이다.
만약 그 공동체가 폭력적으로 또는 자발적으로
다수의 부분들로 분리된다면,
각 부분들이 안정되어
\paren{어쩌면 일정한 무정부적 시기가 지난 후에}
평정상태를 이루자마자, 주권자가 존재할 것이고
잘 살펴보면
각각의 독립된 부분들마다 발견될 것이다.
북미 대륙의 영국 식민지들에 대한 주권은 미국이 성립되기 이전과 이후에
그 소재가 달라졌지만, 두 경우 모두 어딘가에서 주권자를 발견할 수 있는 것이다.
이 주권자, 모든
독립된 정치공동체에 보편적으로 존재하는 이 사람 또는 이 집단은
주권의 다양한 형태에도 불구하고
모두 한 가지 특징을 가진다. 저항할 수 없는 힘의 보유가 그것인데,
이 힘은 반드시 행사되지는 않더라도 행사될 수 있는 것이면 족하다.
오스틴이 선호하는 용어에 의하면,
주권자가 한 사람이면 군주정이고, 작은 집단이면 과두정이고,
상당히 규모가 큰 집단이면 귀족정이고,
규모가 대단히 크고 다수로 이루어지면 민주정이다.
오스틴은 제한군주정---오늘날보다 오스틴 시대에 더 유행했던 용어이다---을
혐오했거니와, 그에 따르면 영국의 정체는 귀족정이다.
주권의 모든 형태가 가지는 공통점은 백성들 또는 시민들을
무제한적으로 강제하는
권력\latin{power}\paren{권력이지만
반드시 의지\latin{will}일 필요는 없다}에 있다.
때로 어떤 사회에서 주권자를 발견하기 어려운 경우가 있고
발견된다 하더라도 누구인지 지목하기 곤란한 경우도 있다.
하지만 무정부상태가 아닌 독립된 정치사회가 존재하는 곳이라면
반드시 주권자가 존재한다.
주권자가 누구인지 결정하는 문제는 아시다시피 언제나 사실문제이지 결코
법적인 문제나 도덕의 문제가 아니다.
특정 공동체에서 특정 사람이나 집단이 주권자로 드러날 때,
그 주권이 찬탈이라거나 헌법원리를 위반했다면서 이 명제를 거부하는 자는
오스틴의 논점을 완전히 오해한 것이다.

오스틴의 제6강에서 읽어낸 이러한 정의는 독립된 국가에서 주권자를 발견하는
심사기준을 제공한다.
그 심사기준 가운데 중요한 몇 가지를 간략하나마 좀 더 살펴보고자 한다.

\para{주권의 특정성}
첫째, 주권자는 \hemph{특정의}\latin{determinate} 상급자이다.
주권자가 한 사람일 필요는 없다.
현대 서구 사회에서 그런 경우는 무척 드물다.
하지만 주권자는 한 사람의 속성을 가질 정도로 \hemph{특정}되어야 한다.
한 사람이 아니더라도,
하나의 단체 또는 동료집단으로 행동할 수 있는 다수의 사람들이어야 한다.
정의에서의 이 부분은 반드시 필요한데,
주권자는 명확한 의지의 표명으로
권력을 행사해야 하고 명령을 발해야 하기 때문이다.
역사적으로 볼 때
주권의 특징의 하나인 물리력의 보유가
특정되지 \hemph{않은}, 의지를 행사할 수 있을 정도로 결합되지 않은,
사람들의 수중에 당분간 존재한 적이 많았으나,
이러한 상태는 오스틴에 의하면, 비록 혁명기의 통상적 징후를 다 갖추지
않았다 하더라도, 무정부상태이다.
또한
주권자가 한 개인이 아닌 경우 주권을 특정의 집단에 한정하는 것은
더더욱 중요하거니와,
그럼으로써만 어떤 단체로부터 나오는 의지의 행사가 여러 가지 인공적 장치에
따르도록 주권 관념을 제약할 수 있기 때문이다.
다수의 의견을 집단 전체의 의견으로 삼는 관행은 우리에게 친숙하고 자연스러워
보이는데, 이것만큼 인공적인 것도 없을 것이다.

\para{백성의 복종}
또한 사회의 대다수 구성원은 주권자로 불리는 상급자에게 복종해야 한다.
전체 사회가 복종하는 게 아니다. 그렇게 되면 주권은 불가능해지기 때문이다.
대다수가 복종해야 하는 것이다.
하노버 왕조가 영국에 들어섰을 때
일부 자코바이트\latin{Jacobite}들과
다수의 스코틀랜드 산악지대 사람들은
영국 국왕과 의회에 습관적으로 불복종했고 그 명령을 무시했다.
하지만 분명 다수의 자코바이트를 포함한 국민 대부분은 그 명령에 관행적으로
복종했다.
따라서 오스틴의 원리에 의할 때,
조지 1세 및 2세와 이들에 의해 소집된 의회의 주권성을 의문시할
근거는 조금도 없다.
하노버 왕조의 왕은 하노버 공국의 주권자일 뿐이라는
자코바이트의 견해를 오스틴은 즉각 거부하는데,
그의 이론체계에서 유일한 논의거리인 저 사실문제를 제기하지 않기 때문이다.

\para{습관적 복종}
다음으로
주권자는 공동체의 다수 구성원들의 습관적인 복종을 받아야 한다.
로마가톨릭을 신봉하는 유럽 국가에서는 대다수의 사람들이
개인의 행동지침에 관하여
교황청으로부터
직·간접적으로 다양한 지시를 받는다.
하지만 그들이 거주하는 나라의 법에 복종하는 횟수에 비해
이러한 외부의 명령에 복종하는 것은 가끔일 뿐이고 습관적이지 않다.
오스틴이 밝혀놓은 원리에 대한 어렴풋한 인식은
교회에 관련된 몇몇 유명한 논쟁에서도 발견된다.
실제로 교황청에 복종하는 것이
습관적이라 할 수 있을 만큼 자주 있는 일인가 아닌가를 둘러싸고 논쟁이
벌어지곤 하는 것이다.

주권의 또 다른 성질은
다른
모든
상급자의 통제로부터 면제되어 있다는 것이다.
이 제한은 분명 필요한데,
그렇지 않으면 영국령 인도의 총독은 주권자가 될 것이기 때문이다.
이 점을 제외하면
실로 그는 지구상의 어떤 실력자보다 주권의 특징을 더 많이 가진다고
볼 수 있다.

\para{홉스}
역사와 정치의 영역에서는 개념의 발달에 오랜 시간이 걸린다는 것을
잘 아는 사람이라면,
주권의 성질에 관한 견해가 오스틴의 작품 이전으로 멀리 거슬러올라가리라고
쉽게 짐작할 수 있을 것이다.
그러나 홉스 이전으로 거슬러올라가는 자료를 나는 알지 못한다.
홉스의 <<리바이어던>>에서, 그리고 애초 라틴어로 출간된
<<철학의 기본요소>>\latin{Elementa Philosophiae}라는 논저의 일부인
<<시민론>>\latin{De Cive}에서
정부 및 사회의 분석과 주권의 확정 이론은 거의 완성되어 있어서
벤담과 오스틴에 의해 추가되어야 할 것이 별로 남아있지 않았다.
벤담과 오스틴의, 특히 오스틴의, 독창성은
주권 관념에 기초한 개념들---실정법, 실정 의무, 제재, 권리---을 좀 더
철저하게 탐구한 것,
이들 개념과 이들과 피상적으로 닮은 다른 개념들의 관계를 설명한 것,
이 모든 관념들을 연결하는 이론에 대한 비판과 싸운 것,
이 이론을 홉스 이후에 등장한 복잡한 사실관계에 적용한 것에 있다.
하지만 홉스와 그의 최근 계승자 간에는 한 가지 큰 차이가 있다.
홉스의 이론 전개 과정은 과학적인 것이었으나,
그의 목표는 과학적이기보다는 정치적이었다.
타의 추종을 불허하는
예리한 직관과 명료한 진술로
홉스가 주권의 보편적 이론적 존재를 위한 주장을 개진했을 때
그는 귀족정이나 민주정보다 대체로 군주정을,
\paren{그가 기초한 학파의 용어를 사용하면}
단체적 주권보다 개인적 주권을,
강하게 선호했음이 분명하다.
그의 추종자들 중에
그의 정치학을 수용하지 않은 자들은
그가 오해를 당했다고 주장하기도 한다.
분명 피상적으로 그를 읽은 독자들은
그가
주권의 형태와 무관하게 주권자의 무제한적 권력을 이야기했을 때
전제정을 주장했다고 생각하기도 했다.
그러나 솔직히 말해서
스튜어트 왕조에 저항하는 주요 수단인
장기의회와 영국 보통법에 대한 강한
혐오가 주권, 법, 무정부의 성질에 관한 그의 언어를 물들이고 있다는 것은
부인할 수 없는 사실인 것 같다.
또한 그가
호국경의 비위를 맞추려는 은밀한 의도에서
그의 체계를 만들었다고 생전에 비난을 받았던 것도
놀라운 일이 아니다.
그러나 오스틴의 목표는 엄격히 과학적인 것이다.
그가 오류에 빠졌다면 그것은 그의 철학 때문이다.
그의 언어는 자신의 정치적 의견의 색조를 거의 드러내지 않는다.

\para{홉스의 사회기원론}
또 다른 중대한 차이도 있다.
주지하듯이 홉스는 정부와 주권의 기원에 관한 사변을 전개했다.
누군가 그에 대해 배웠다면 바로 이것을 배웠을 것이며,
이것 때문에 그의 철학이 비난받아 마땅하다고 생각할 것이다.
그러나 오스틴은 이런 탐구로는 거의 나아가지 않는다.
어쩌다 무의식적으로 주권과 그것에 기초한 개념들이 선험적인 존재를
갖는다는 함의를 내비치기는 하지만 말이다.
그런데
이 문제에 관하여 나 자신은 홉스의 방법이 옳았다고 생각한다.
물론 사회와 정부의 기원에 관한 홉스의 추측보다 더 무가치한 것은 없을 것이다.
그는 태초의 인류가 전쟁상태에 있었다고 주장한다.
그리고 모든 사람이 공격 권한을 포기하는 약정을 체결했다고 한다.
그 결과물이 주권이고 주권을 통해서 법, 평화, 질서가 등장한다는 것이다.
이 이론은 모든 면에서 반박될 수 있다.
가상의 역사 단계의 증거가 존재하지 않으며,
우리가 조금이나마 알고 있는 원시인의 상태는 그의 추정과 모순된다.
인류 초기의 보편적 무질서 상태는 부족과 부족, 가족과 가족의 투쟁에서는
사실일 수 있으나, 개인과 개인의 관계에서는 사실이 아니다.
오히려 우리가 아는 한 그들은, 현대적 용어를 사용하자면,
초법적 상태\latin{ultra-legality}라고 부를 만한 체제 하에서 살고 있었다.
더욱이 로크의 반대가설에 가해지는 비판과 동일한 비판을 가할 수 있는데,
근대적 법개념인 계약 개념이 등장하기 이전의 상태라는 비판이 그것이다.
그렇지만,
비록 홉스가 문제를 해결하지는 못했지만,
나는 그가 문제에 올바르게 접근하기는 했다고 생각한다.
주권이 어떻게 발생했는가는 아니더라도
주권이 어떤 단계를 거쳐왔는가 하는 문제에 대한 탐구는
반드시 필요하다는 것이 내 판단이다.
이를 통해서만 우리는
오스틴의 분석 결과가 사실과 얼마나 부합하는지 알 수 있다.

\para{분석법학자들의 주장, 사회의 힘, 주권의 추상화}
사실,
인간 본성과 사회의 관찰된 사실들이
주권에 관한
분석법학자들의 주장이나 그들이 주장했음직한 것들과
얼마나 부합하는지 신중하게 살펴보는 것보다
법학도들에게
더 중요한 일은 없을 것이다.
우선 이러한 주장들을 서로 간에 분리할 필요가 있다.
그 가운데 첫째는 인간의 모든 독립된 공동체에는
저항할 수 없는 힘을
그 공동체의 여러 구성원들에게 행사하는
권력이 존재한다는 것이다.
이것은 사실이라고 인정할 수 있다.
공동체의 모든 구성원들이 동일한 육체적 힘을 갖고 있고
무장하지 않고 있다면
권력은 단지 숫자의 우위에 기초할 것이지만,
기실 다양한 원인이, 특히 공동체의 일부가 강한 육체적 힘과
우월한 무기를 가지는 등의 원인이,
저항할 수 없는 힘을
공동체 전체에 대해 행사하는
권력을
소수자들에게
가져다 주었다.
그 다음 주장은
모든 독립된 \hemph{정치}공동체---자연상태에
머물러있지도 무정부상태에 빠져있지도 않은 모든
독립된 공동체---에서는
그 사회가 품고 있는 저항할 수 없는 힘을 사용하고 지휘할 권력이
그 사회에 속한
어떤 사람 또는 사람들의 집단에게 주어진다는 것이다.
이 주장은 일군의 사실들, 특히 서양 세계나 근대 세계의 정치적 사실에 의해
진리일 것이 강하게 추정된다.
하지만 관련된 모든 사실이 모두 관찰된 것은 아니란 것을 명심해야 한다.
세계 전체---인간 본성에 관한 이론가들은 그 가운데 절반 이상을
간과하는 경향이 있다---가,
세계 전체의 역사 전체가
조사된 이후라야
우리는 사실에 관하여 확신할 수 있을 것이다.
이것이 행해진 이후에는
다수의 사실들이 저 결론을 그리 강하게 지지하지는 않는 것으로 드러날 수 있다.
또는, 내가 짐작하고 있는 것처럼,
저 주장이 거짓으로 드러나기보다는
단지 말로만 진실인 것으로, 그리하여
우리가 속한 사회의 유형에서 갖는 가치를 갖지는 않는 것으로 드러날 수 있다.
그러나
분석법학의 위대한 창시자들에게 비난을 돌릴 수는 없지만
그 추종자들의 일부가 내걸고 있는 어떤 주장, 즉
주권자인 사람 또는 집단은 무제한적 의지의 행사를 통해
사회가 품고 있는 힘을 현실적으로 휘두른다는 주장은
확실히 사실과 전혀 부합하지 않는다.
왜곡된 정신을 가진 폭군만이 그러한 주권자의 사례에 해당할 것이다.
도덕적이라 부를 수 있는 무척 큰 영향력이
주권자가 행사하는 사회적 힘의 실제 작용을 상시적으로 틀지우거나
제한하거나 금지한다.
다른 무엇보다 이 점을 염두에 두는 것은 자못 중요한데,
주권에 관한 오스틴류의 견해가 실제로 무엇인지를 잘 보여주기 때문이다.
그것은 추상화의 결과이다.
정부와 사회의 모든 특징과 속성을 한 가지만 빼놓고 모두
사상해버리면 도달하는 것,
힘의 보유라는 공통점을 가지고서 다양한 형태의 정치적 상급자를 하나로 묶어버리면
도달하는 것이 그것이다.
이 과정에서 무시된 요소들이 언제나 중요하고, 때로는 대단히 중요하다.
그 요소들이야말로
힘의 직접적 적용이나 직접적 장악을 제외한 모든 인간 행위를
통제하는 모든 영향력을 이루기 때문이다.
하지만
분류를 위하여 이것들을 사상하는 것은
철학적으로 물론 정당하며,
통상의 과학적 방법의 적용일 뿐이다.

\para{역사적 영향들의 제거}
달리 표현하면,
주권 개념에 도달하는 추상화 과정에서 사상되는 것들은
각 공동체의 전체 역사이다.
우선,
각 사회의 어디에, 어떤 사람 또는 집단에,
사회적 힘을 사용할 권력이 존재하는가를 결정하는
각 사회의 역사를, 모든 역사적 선행사실들을 사상해버린다.
주권의 이론은 어떻게 현재의 결과에 도달했는지를 무시한다.
그리하여 페르시아 왕의, 아테네 민중의, 후기 로마의 황제의,
러시아 짜르의, 영국 국왕과 의회의, 강제적 권위를 모두 하나로 묶어버린다.
다음으로,
주권자가 그의 저항할 수 없는 강제력을 어떻게 행사하는지 또는 행사하지 않는지를
결정하는
각 공동체의 역사를, 역사적 선행사실들 전체를 사상해버린다.
이것을 구성하는 모든 것---여론, 감정, 믿음, 미신, 편견의 거대한 집적물,
전래된 것이든 새로 얻은 것이든, 제도에 의해 만들어진 것이든
인간 본성에 따른 것이든,
모든 종류의 관념의 집적물---을 분석법학자들은 무시한다.
그리하여,
주권의 정의에 들어있는 제약에 따르면,
우리나라의 국왕과 의회는 모든 허약한 아이들을 죽이라고 명할 수도 있고
구체제 하의 프랑스 국왕이 발부하던 무제약적인 구속영장\latin{lettres de cachet}
체제를 도입할 수도 있게 된다.

\para{추상적 과학}
분석법학자의 논리과정은 수학이나 정치경제학의 그것과 자못 닮았다.
그것은 엄격하게 철학적이다.
하지만 추상화에 기초하는 모든 과학의 현실적 가치는
추상화 과정에서 사상된 요소들과 남겨진 요소들 간의 상대적 중요성에 달려있다.
이 기준에 의할 때, 수학은 정치경제학보다 더 가치가 크고,
이 두 가지 모두는 내가 비판하고 있는 저자의 법학보다 더 가치가 크다.
마찬가지로,
오스틴의 분석이 야기하는 오해들은
응용수학\latin{mixed mathematics}의 학생들을 혼란스럽게 할 수 있는
오해들과 유사하고,
정치경제학의 학생들을 실제로 혼란스럽게 하는 오해들과도 유사하다.
자연에서의 마찰력의 존재를 망각할 수 있듯이,
부자가 되려는 욕망을 제외한 다른 사회적 동기의 존재를 망각할 수 있듯이,
오스틴 학도는 현실의 주권에는 힘\latin{force}말고도 더 많은 것들이 들어있음을,
주권자의 명령인 법에는 명령을 단지 규칙적인 힘으로만 보아서 얻는 것말고도
더 많은 것들이 들어있음을 망각하는 경향이 있다.
물론 나는 오스틴이 때로, 홉스가 자주, 그들의 체계가
그 근저에 놓여있는 한계에 의해 완전히 제약되는 것은 아니라고 말했음을
부인하지 않는다.
사실, 추상화의 대가\hanja{大家}들은
순수한 정신적 과정에서 버려지는 자료들은 부스러기에 지나지 않는다는
취지의 말이나 글을
가끔 내놓는다.

\para{주권에 종속적인 법, 관습법, 허락과 명령}
하지만, 오스틴의 체계에서 주권의 확정이 법의 확정에 선행해야 함을
인식한다면,
오스틴의 주권 개념은
강제력 이외의 나머지 속성들을 사상하여
모든 형태의 정부를 하나로 묶는 정신 과정을 통해 도달됨을
이해한다면,
추상적 원리로부터의 연역은
사실들에서
완전히
예시된 사례의 성질로부터 도출되는 것이 아님을
명심한다면,
생각건대,
오스틴을 읽는 학생들이 느끼는 주된 어려움들이 사라질 뿐만 아니라,
초심자들이 받아들이기 힘든 그의 몇몇 주장들이 이제 자명한 명제들로
보이기 시작할 것이다.
여러분들은 그의 논저에 충분히 익숙할 것이므로
나는 이러한 명제들 중 몇몇을, 완전히 정확하게 진술하기 위한 부연설명 없이,
언급하겠다.
법학은 실정법의 과학이다.
실정법은 주권자가 백성들에게 발하는 명령으로서,
책무를, 또는 의무의 조건을, 또는 의무를 백성들에게 부과하고,
명령에 불복종하는 경우 제재 또는 처벌이 가해질 것이라고 위협한다.
권리는 주권자가 공동체의 특정 구성원들에게 수여하는 능력 또는 권한으로서,
동료시민의 의무 위반에 대해 제재를 가할 수 있게 한다.
이렇게 보면 법, 권리, 의무, 처벌에 관한 이 모든 개념들이
주권이라는 근본적 개념에 의존하고 있는 것이다.
마치 사슬의 아래쪽 고리가 제일 위쪽 고리에 의존하여 매달려있듯이 말이다.
그러나 오스틴의 체계에서 주권은 힘 이외에는 아무 속성도 갖지 않는다.
결과적으로 여기서 `법' `의무' `권리'에 관한 견해는 오직
이들을 강제력의 산물로만 바라보는 견해일 뿐이다.
그리하여 일련의 관념들 중에 가장 중요한 일차적 관념은 `제재'가 되고 이것이
다른 모든 관념들을 물들인다.
공식적인 입법부에 의해 선포된 법에 관한 한,
그것이 오스틴이 말한 성격의 법이라는 데 누구도 이의를 달지 않을 것이다.
하지만 많은 이들은, 그중에서도 특히 강한 정신을 가진 이들은,
입법부라고 흔히 불리는 국가기관에 의해 선포된 적이 없는 방대한 법규칙들도
주권자의 명령이라는 입장에 의문을 제기해왔다.
법전에 포함되지 않은 여러 나라의 관습법\latin{customary law}은,
특히 영국의 보통법은,
주권자와 무관한 기원을 가져왔던 것이 일반적이다.
그리하여 오스틴이 주창한 주제가 모호하고 이해불가능하다는 이론들이 개진되어왔다.
홉스나 오스틴이 보통법과 같은 규칙집합을 그들의 체계로 포섭하는 방식은
그들 체계에 대단히 중요한 격률 하나를 주장하는 것에 기초한다.
``주권자가 허락하는 것은 주권자가 명령하는 것이다''가 그것이다.
% 364
법원에 의해 관습이 강제되기 전까지는
관습은 여론에 의해 강제되는 `실증도덕'\latin{positive morality}일 뿐이다.
법원에 의해 강제되는 즉시 관습은,
주권자의 수임인 또는 대리인인 판사를 통하여,
주권자의 명령이 된다.
이런 이론에 대하여 오스틴이 인정했을 대답보다 더 나은 대답은,
그것이 단지 언어의 농간에 기초하고 있다는 것이고, 또한
자신이 전혀 의식하지 못하는 동기나 방식으로 법원이 행위한다고
가정하고 있다는 것이다.
그러나,
그의 체계에서 주권자는 힘이나 권력에만 관계된다는 것을 명확히 인식한다면,
`주권자가 허락하는 것은 주권자가 명령하는 것'이라는 입장은 보다 쉽게 이해된다.
그것이 주권자의 명령인 이유는,
주권자는
무제한적 힘을 가진다고 가정되기에
언제든 제약 없이 혁신을
가져올 수 있기 때문이다.
보통법이 주권자의 명령으로 구성되는 이유는,
주권자가
그것을
마음대로
폐지하거나 변경하거나 재진술할 수 있기 때문이다.
이 이론은 이론으로서는 완전히 옹호할 수 있지만,
그것의 현실적 가치는, 그리고 진실에 접근하는 정도는,
시대에 따라 나라에 따라 큰 차이를 보인다.
지금까지 존재했던 독립된 정치공동체 중에는,
세계 전체를 철저히 조사해보면 지금도 독립된 정치공동체 중 일부는,
주권자가 저항할 수 없는 권력을 가지고 있으면서도
혁신은 결코 꿈꾸지 않는 공동체가,
법을 선언하고 적용하는 사람들 또는 집단들이
% 365
주권자 자신만큼이나 그 사회의 필수적인 구성요소라고 믿어지는 공동체가
분명히 있을 것이다.
또한 독립된 정치공동체 중에는,
주권자가 저항할 수 없는 강제력을 가지고 있고
혁신을 최대 한도로 수행하지만,
법을 주권자의 명령으로 간주해서는
법에 관련된 모든 관념연관이 왜곡되어버리는 공동체가
분명히 있을 것이다.
그리스 도시국가의 참주들은 종종 오스틴의 주권자 기준 전부를 충족하지만,
참주에 관한 널리 승인된 정의는 `그가 법을 전복시켰다'는 뜻을 내포하고
있는 것이다.
설령 이런 경우에도 저 이론이 들어맞는다고 볼 수 있다 하더라도,
그것은 그저 언어를 지나치게 잡아늘인 것에 불과하다.
단어나 명제들을 이들과 통상적으로 관련된 관념의 영역에서 이탈하게 함으로써
겨우 가능해지는 것이다.

\para{오스틴 이론의 한계, 오스틴의 도덕이론, 오스틴의 윤리적 신조}
오스틴의 이론의 현실적 가치의 역사적 한계를
다음 장에서 자세히 논하기에 앞서,
지금까지의 나의 의견을 요약해두고자 한다.
내가 옳다고 생각하는 논의방식을 오스틴이 채용했다면,
주권의 탐구가 그것에 의존하는 다른 개념들의 탐구에 선행했다면,
이 후자의 개념들에 관한 그의 진술 중 많은 것들이
무해할 뿐만 아니라 자명한 것으로 드러날 것이다.
법이 규칙적인 힘으로 간주되는 이유는 단지
다른 모든 개념이 의존하는 일차적 관념에 진입할 수 있는
하나의 요소가 바로 힘이기 때문일 뿐이다.
% 366
이론적으로는 반박할 수 없으나 역사가 전개됨에 따라 실제적 진실에 접근해가는
어떤 가정---주권자는 그가 변경할 수 있지만 변경하지 않는 것을
명령한다는 가정---이 행해진다면,
법률가들에게 거부감을 불러일으키는 분석법학자들의 원리 하나가
더 이상 역설적으로 여겨지지 않을 것이다.
생각건대, 이러한 논의구조는 또 다른 장점도 가질 수 있거니와,
오스틴의 도덕이론에 대해 필요한 수정을 제공하는 것이다.
물론 이 주제를 여기서 자세하게 다룰 수는 없을 것이다.
다수의 독자들이 이해에 어려움을 겪는 명제는---나는 대중적인 언어로만
진술하겠다---도덕 규칙의 제재란 그 위반에 대해 동료시민들이 표출하는
불승인이라는 것이다.
때로 이것은 도덕 규칙에 복종하는 유일한 동기가 그러한 불승인에 대한
두려움이라고 해석되기도 한다.
오스틴의 언어의 이러한 해석은 그의 뜻을 완전히 오해한 것이다.
하지만 내가 주장하는 논의의 순서를 따른다면,
이런 해석은 들어설 여지가 전혀 없을 것이다.
오스틴이 주권의 분석을, 그리고 그것에 직접 의존하는 개념들의 분석,
즉 법, 법적 권리, 법적 의무 등의 분석을 완성했다고 가정해보자.
그러면 그는 사람들이 사실상 복종하고 있는 방대한 규칙들을 탐구해야 할 것이다.
즉, 어느 정도 법의 성질을 가지고 있지만
% 367
그 자체로 주권자의 백성에 대한 명령은 아닌 규칙들을,
그 자체로 주권자가 제공하는 제재에 의해 강제되지는 않는 규칙들을,
탐구해야 할 것이다.
물론 이들 규칙을 탐구하는 것은 저 법철학자의 임무이다.
왜냐하면 그의 가설에 의하면 주권자는
\hemph{인간}인 상급자이고, 인간으로서 그 규칙들에 복종하기 때문이다.
사실 오스틴은 이런 관점에서
그 규칙들을
사뭇 흥미로운 방식으로
탐구한다.
주권은 그 본성상 법적 제한에 복하지 않는다고 주장하면서도,
주권자는 어떤 명령은 발해서는 안 되고 다른 명령은 반드시 발해야 한다고
전적으로 인정하고 있는 것이다.
이때 적용되는 규칙은, 법이 아니라, 엄격한 합당성\latin{cogency}의 규칙이다.
영국의 국왕과 의회는
그의 견해에 따르면
주권자---그의 용어로는 주권자 귀족---이지만,
그리고
이 귀족들은 이론적으로는 무엇이든 마음대로 할 수 있지만,
그러나
이렇게 주장하는 것은 경험칙에 완전히 반한다.
헌법적 격률에 구체화되어 있는 방대한 규칙들이 어떤 일을 하지 못하도록 막는다.
언어관용상 도덕이라 불리는 방대한 규칙들이 다른 일을 하지 못하도록 막는다.
이렇게 일반인은 물론이고 주권자에게도 작용하는 규칙들이 갖는
공통점은 무엇인가?
주지하듯이 오스틴은 그것을 `실증도덕'이라 부른다.
그리고 그것의 제재는 여론, 즉 위반시 수반되는 공동체 대다수의 불승인이라고
말한다.
% 368
제대로 이해된다면,
이 마지막 명제는 참으로 진실이다.
공동의 불승인이야말로 이 모든 규칙들이 공통적으로 갖는 제재이다.
국왕과 의회로 하여금 살인을 합법화하지 못하게 하는 규칙과
국왕으로 하여금 각료 없이 통치하지 못하도록 하는 규칙은
위반시에 뒤따르는 처벌, 즉 영국인 다수의 강한 불승인에 의해
하나로 연결되는 것이다.
이 두 규칙을 진정한 법과 근본적으로 연결짓는 것은
일종의 제재를 가진다는 점이다.
하지만,
여론에 대한 두려움이 저 두 규칙에 복종하는 동기이기는 하지만,
그렇다고 저 두 규칙에 복종하는 유일한 동기가 여론에 대한 두려움인 것은 아니다.
대부분의 사람들이
이 두려움을
헌법규칙에 복종하는
유일한 동기는 아니더라도 주된 동기로 인정할 것이지만,
이렇게 인정한다고 해서 도덕규칙의 제재에 관한
어떤 필연적 주장이 반드시 도출되는 것은 아니다.
진실은 오스틴의 체계가 \hemph{어떠한} 윤리이론과도 양립할 수 있다는 것이다.
오스틴이 이에 반대되는 주장을 한다면,
생각건대 그것은 자신의 윤리적 신조의 진리치에 대한 그의 확신에서
기인할 것이다.
그의 신조는 말할 것도 없이 초기 형태의 공리주의\hanja{功利主義}였다.
사실,
오스틴을 주의깊게 연구함으로써 학생들의 도덕관이 바뀔 수 있다는 것을
부인할 생각은 내게 조금도 없다.
많은 다른 학문들처럼, 윤리학 논의는
% 369
사고의 불명료 속에서 진행된다.
그리고 그러한 불명료를 떨쳐버리는 데는,
탐구대상인 주요 용어들을 완전히 일관된 의미로 연결짓고
이러한 의미의 용어들을 가지고 모호한 표현을 탐지하는 심사기준으로
삼는 것보다 더 나은 것이 없다.
엄격하게 일관된 용어로써 이러한 것을 제공한다는 점에서
분석법학은 법학과 윤리학에 대해 더없이 소중한 가치를 지닌다.
하지만 그 체계를 잘 이해하고 평가할 수 있는 학생들이라고 해서
반드시 공리주의자가 되어야 한다고 생각할 이유는 조금도 없다.

\para{오스틴의 신법}
이하에서 나는 오스틴의 체계와 공리주의 철학 간의 진정한 연결지점이라
생각되는 것에 관해 말하겠다.
잘못된 논의구성과 더불어 공리주의 철학에 대한 강한 신념이
<<법학의 영역 확정>>에 무엇보다 심각한 오점을 남겼다.
제2강, 제3강, 제4강에서
신법과 자연법\paren{이 용어들이 어떤 의미라도 가질 수 있다면}을
공리주의의 규칙과 동일시하는 시도가 행해졌다.
이 강의들은 정당하고 흥미롭고 가치있는 관찰을 다수 포함하고 있다.
그러나 이들 강의의 목표인 저 동일시는
어떤 목적에서든
전혀 불필요하고 무가치한 것이다.
편견들을 피하고 제거하는 데 도움이 된다---나는 아니라고 생각하지만---는
% 370
성실한 믿음에서 쓰여진
저 강의들은,
오스틴의 체계에 신학과 철학 양쪽으로부터
전적인 편견들의 덩어리를 도입했다.
하지만,
내가 제안한 순서에 따라
오스틴이 주권의 성질에 관한 탐구를 마친 후
신법의 성질에 대한 탐구로 들어갔다면,
그것은
주권자라 불리는 인간 상급자의 특성이 얼마나
전능한 비인간 통치자에게도 부여될 수 있을 것인가 하는 문제,
인간 주권에 의존하는 많은 개념들이 얼마나
신의 명령에도 통용될 수 있을 것인가 하는 문제의 형태를 띠었을 것이다.
나는 오스틴의 논저와 같은 논저에서 이러한 탐구가 필요한 지 의문이다.
기껏해야
그러한 논의는
법의 철학이 아닌 입법의 철학에나
속할 것이다.
진정한 의미의 법학자는 법이나 도덕의 이상적인 기준에 대해서
아무런 관심이 없다.

