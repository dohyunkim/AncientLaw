\chapter{계약법의 초기 역사}

우리가 속한 시대에 관한 명제로,
오늘날의 사회가
지난 시대의 사회와 차이나는 주요 특징은
계약법이 차지하는 영역이 대폭 증가했다는 데 있다는
주장만큼
일견 쉽게 수긍할 수 있을 법한 것도 없을 것이다.
이 명제가 근거하고 있는 현상들 중 일부는
대단히 빈번하게 선택되어 관심과 논평과 칭송의 대상이 되고 있다.
우리들 중에서
옛 법이 사람의 출생에 따라 그의 사회적 지위를
불가역적으로 고정시켰던 수많은 사안들에서
근대법은 합의에 의해 그 스스로 자신의 지위를 만들어갈 수 있도록
허용하고 있음을
알아차리지 못할 정도로
무감한 사람은
별로 없을 것이다.
실로 이 원칙에 대한 예외로 남아있는 소수의 몇몇 것들은
열정적 분노에 찬 비난을 지속적으로 받고 있다.
가령 흑인노예제를 둘러싸고 여전히 진행 중인 열띤 논쟁에서
실로 다투어지고 있는 논점은
노예제가 지난 시대의 제도가 아니냐는 것,
그리고
근대적 도덕성에 부합하는
고용주와 노동자 간의 관계는
오직 계약에 의해 정해지는 관계뿐이지 않겠느냐는 것이다.
과거와 현재 간의 이러한 차이의 인정은
현대의 가장 유명한 사변적 논의의 핵심으로 우리를 끌고 들어간다.
확실히,
명령법\latin{imperative law}이
한때 장악하고 있던 영역의 많은 부분을
포기하지 않았다면,
그리고
최근까지 허용되지 않던 자유를 누리며
사람들이 스스로의 행위규칙을 정하도록 허용하지 않았다면,
도덕에 관한 연구 분야 중에
우리 시대에 비약적인 진보를 보인 유일한 분야인
정치경제학은
생활 현실에 부응하지 못하고 실패할 것이다.
정치경제학의 훈련을 받은 사람들의 대다수가
실로 가지고 있는 선입견은
그들 학문이 의지하고 있는 일반적 진리가
보편적인 것이 될 권리가 있다고 보는 것이다.
그리하여 그들이 그것을 학문으로 적용할 때면,
그들의 노력은 대개 계약법의 영역을 확장하고
명령법의 영역은 축소하는 방향을 지향하거니와,
단지 계약의 이행을 강제하는 데 필요한 한에서만
명령법을 용인하는 것이다.
이러한 관념의 영향을 받은 사상가들이 불러일으킨 충동은
바야흐로 서구 세계에서 사뭇 강력하게 느껴지기 시작하고 있다.
입법은
발견과 발명과 축적된 부\hanja{富}의 사용에 관한 사람들의 행동을
따라잡을 능력이 없음을
거의 자백했다.
가장 덜 진보된 공동체의 법조차
점점 단지 껍데기에 불과한 것이 되어가고 있거니와,
그 아래에는
지속적으로 변화하는 계약적 규칙들의 연합이 존재하여,
여기에 법이 개입하는 경우는
약간의 근본 원리들의 준수를 강제하거나
신의\hanja{信義} 위반을 벌하기 위해 소환되는 경우 외에는
거의 없는 실정이다.

\para{계약의 강제}
법현상을 고려해야 하는 것인 한,
사회적 탐구는 그 상황이 매우 낙후되어 있는지라,
사회의 진보에 관하여 널리 통용되는
통속적인 견해에서 저 진리가
발견되지 않더라도 놀라울 것이 없다.
이들 통속적인 견해는
우리의 신념보다는 우리의 편견에 더 잘 부응한다.
도덕의 진보를 인정하기를 꺼리는 강한 경향성은
계약의 기초가 되는 미덕을 의문시할 때
특히 더 강력해지는 듯하다.
우리들 중 다수는
신의와 성실이 옛날보다 오늘날에 더 널리 퍼져있음을,
또는 적어도 고대 세계의 충실성에 비견할 만한 풍속이 오늘날에도 존재함을,
인정하는 것에 대한
거의 본능적인 거부감을 가지고 있다.
때로 이러한 선입견은
예전에는 들려오지 않던
사기행각이 만연함을 보면서,
그리고 그것의 범죄성이 가져오는 커다란 혼란과 충격을 보면서
더욱 강화된다.
그러나 바로 이러한 사기행위의 범죄성으로부터 우리는
그것을 범죄로 취급할 수 있기 위해서는
우선
그것이 위반하는 도덕적 의무가 더 크게 성장해야 한다는 것을
뚜렷이 알 수 있다.
다수가 믿고 따르는 신뢰가 있어야만
소수의 신의 위반도 생길 수 있는 법이므로,
아주 큰 부정직의 사례들이 발생한다면 이는
다수의 평균적 거래에서는 성실한 정직이 지배적이어서
예외적인 경우 범죄자들에게 기회가 주어졌다고
결론짓지 않을 수 없는 것이다.
계약법에서 형법으로 눈을 돌려
법에 반영된 도덕의 역사를 읽어야 한다면,
우리는 그것을 오독\hanja{誤讀}하지 않도록 주의해야 한다.
로마 고법\hanja{古法}에서 부정직한 행위로 취급된 형태는
절도가 유일했다.
이 글을 쓰는 순간,
영국 형법에 추가된 최신 영역은
수탁자\hanja{受託者}의 사기행위를 처벌대상으로 삼으려는 것이다.
이러한 대비에서 얻을 수 있는 올바른 추론은
원시 로마인들이 우리보다 더 높은 도덕성을 지녔다는 것이 아니다.
오히려 그들 시대에서 우리 시대로 시간이 흐르면서
사뭇 미개한 도덕성으로부터 대단히 세련된 도덕성으로
도덕성 관념이 진보했다는 것을 알 수 있는 것이다.
소유권만을 신성한 것으로 여기던 관념에서
단지 일방적 신뢰의 수여만으로 발생하는 권리까지도
형법에 의해 보호되는 권리로 보는 관념으로 진보가 이루어진 것이다.

\para{사회계약}
이 점에 관하여 법학자들의 정연한 이론이라고 해서 대중들의 의견보다 더
진리에 가까운 것도 아니다.
로마 법률가들의 견해부터 말하자면,
그것은 도덕과 법의 진보에 관한 참된 역사와 일치하지 않았다.
계약 당사자들이 약속한 신의가 유일하게 중요한 요소인
계약의 한 유형을 그들은 만민법상의\latin{juris gentium} 계약이라고
지칭했거니와,\footnote{%
  `낙성계약'(contractus consensu)을 말하고 있다.
  }
이 유형의 계약은 로마법에 나중에야 편입되어 들어간 것이 확실함에도
불구하고,
그들이 사용한 표현으로부터 어떤 확정적 의미를 추출해보면
그들은 그것을 로마법이 인정하는 다른 유형의 계약, 즉
법기술적 방식요건이 하나만 잘못되어도 오늘날의 착오나 사기만큼이나
계약의무의 성립에 치명적이었던 다른 유형의 계약들보다
더 오래된 것으로 보았음을 알 수 있다.
하지만 그들이 말하는 옛 것은 모호하고 희미한 것이었고
현재를 통해서만 이해될 수 있는 것이었다.
그리하여 ``만민법\latin{law of nations}상의 계약''을
자연상태의 사람들 사이의 계약으로 간주하게 된 것은
로마 법률가들의 언어가
그러한 사고양식에 진입하는 열쇠를 이미 상실해버린 시대의 언어로 된
이후의 일이었다.
루소는 법률가들의 오류와 대중들의 오류를 모두 이어받았다.
관심을 끈 첫 작품이자
그를 한 분야의 선구자로 만든 의견이 사뭇 기탄없이 개진된 논문인
예술과 학문이 도덕에 끼친 영향을 논하는 논문에서,
그는 고대 페르시아인들이 지녔던 정직함과 신의성실이야말로
문명의 등장과 더불어 점차 망각되어간 원시적 순수성의 특징이라고
누차 지적하고 있다.
그리고 나중에 그는
그의 모든 사변\hanja{思辨}의 토대를
원초적 사회계약의 교리에서 발견하게 된다.
사회계약론은 우리가 논하고 있는 오류를 지닌 것 가운데 가장 체계적인 형태이다.
비록 정치적 열정에 의해 그 중요성이 고양되었지만
이 이론은 법률가들의 사변으로부터
모든 수액\hanja{樹液}을 채취한 이론이다.
처음 이 이론에 감화된 영국의 저명인사들은
주로 정치적 유용성의 면에서 그것의 가치를 높이 평가한 것이 사실이지만,
곧이어 설명하겠으나
만약 정치가들이 법적인 용어로 논쟁을 해오지 않았더라면
영국인들은 결코 이 이론에 다가서지 못했을 것이다.
그리하여 이 이론을 주창한 영국인 학자들도
그들로부터 그것을 물려받은 프랑스인들에게 강한 호소력을 가졌던
저 사변적 깊이를 모르지 않았다.
그들의 저서는 이 이론이 정치적 현상뿐만 아니라
사회적 현상까지 모두 설명할 수 있다고 그들이 인식했음을 보여준다.
사람들이 준수하는 실정규칙 가운데
계약으로 만들어진 것이 점점 많아지고
명령법으로 만들어진 것이 점점 줄어들고 있는 현상,
그들 시대에도 이미 현저하게 나타나고 있던 이 현상을
그들은 관찰을 통해 알고 있었다.
그러나 법학의 저 두 구성부분의 역사적 관계에 대해서는
그들은 무지했거나 주의를 게을리했다.
그리하여
그들은 모든 법은 계약에서 기원한다는 이론을 창안하였거니와, 이는
모든 법의 기원을 단일한 원천에 둠으로써 그들의 사변적 취향을
만족시키기 위한 것이었으며,
또한
명령법은 신에게서 기원한다는 교리를 피하려는 견해에서 나온 것이었다.
한 단계 더 사고가 진보한다면,
그들은 기꺼이
그들의 이론을
어떤 기발한 가설이나 편려한 언어 공식\hanja{公式}에 불과했다고
치부했을 것이다.
그러나 당시는 법적 미신\hanja{迷信}이 지배하던 시대였다.
자연상태에 관한 논의는 그것이 역설적이 아니라고 여겨지는 한 계속되었고,
따라서
사회계약을 역사적 사실로 주장함으로써
법의 계약적 기원이라는 거짓 현실과 확신을
쉽게 심어줄 수 있었던 것으로 보인다.

\para{몽테스키외의 혈거인}
우리 세대는 이러한 잘못된 법이론을 떨쳐버렸다.
그것은 부분적으로는 저 이론이 속했던 지적 상태를 벗어났기 때문이고,
또 부분적으로는 그러한 주제를 이론화하는 일을 거의 그만두었기 때문이다.
오늘날 적극적인 연구를 수행하는 학자들이 선호하는 작업은,
그리고 사회의 기원에 관한 우리 선조들의 사변에 대해 답할 수 있는 작업은,
사회의 존재를 있는 그대로, 사회의 운동을 운동하는 그대로 분석하는 것이다.
그러나 역사의 도움을 받지 못하면,
이런 분석은 단순한 호기심의 발로로 전락하기 일쑤이거니와,
특히
연구자가 익숙해있는 사회상태와 자못 다른 사회상태를 이해하는 데
장애물로 작용할 공산이 크다.
우리 시대의 도덕성을 가지고 다른 시대의 사람들을 판단하는 잘못은
현대사회라는 기계장치의 바퀴 하나, 볼트 하나까지
원초적 사회에 그 대응물이 있을 것이라고 가정하는 잘못에 견줄 만하다.
이러한 인상\hanja{印象}은
근대적 양식으로 쓰여진 역사학 저술들에서
사뭇 다양하게 가지를 치고 있으며
사뭇 미묘하게 모습을 숨기고 있다.
그러나
나는
몽테스키외의
``페르시아인의 편지''에 삽입된
혈거인\hanjalatin{穴居人}{Troglodytes}의 우화에 대해
흔히 주어지는 찬사에서
법학 영역에서의 그러한 인상의 흔적을 발견한다.
혈거인들은 계약을 항상 위반하는 사람들이었으며, 그래서 완전히 멸망해버렸다.
만약 이 이야기에 저자가 의도한 도덕이 담겨있고,
그것이
금세기와 지난 세기를 위협해온 반사회적 이단\hanja{異端}을
폭로하기 위해 사용되었다면,
그것은 전혀 나무랄 데가 없는 것이다.
그러나
성숙한 문명이 보여주는 것과 같은 정도로
약속과 합의에 신성함을 부여하지 않는 한
어떤 사회도 결속을 유지할 수 없다는
주장이
저 이야기로부터
추론되어 나온다면,
그것은 법사\hanja{法史}의 어떤 건전한 이해와도 상반되는 치명적인 오류가
될 것이다.
사실,
혈거인들은 계약적 의무를 아주 조금 준수함으로써
번성할 수 있었고 강력한 국가를 건설할 수 있었던 것이다.
원시사회의 헌정\hanja{憲政}에 관하여
무엇보다 먼저 이해해야 할 것은
개인은 자신을 위해 권리나 의무를 거의 혹은 전혀 만들지 못한다는 점이다.
개인이 지켜야 할 규칙은 우선은 출생에 따르는 지위에서 나오고,
다음으로는 그가 속하는 가\hanja{家}의 수장이 그에게 부과하는
명령에서 나온다.
이러한 체제는 계약을 위한 여지를 거의 남겨두지 않는다.
동일한 가\hanja{家}의 구성원들은
\paren{증거로부터 해석하건대}
서로 간에 전혀 계약을 체결할 수 없으며,
가\hanja{家}는 그 구성원이 가를 구속시키려고 맺은 계약을
무시할 수 있는 권리를 가진다.
물론 가와 가 사이, 가부장과 가부장 사이의 계약은 있을 수 있지만,
그 거래는 물건의 양도와 마찬가지 성격을 지니므로
수많은 방식요건들이 부과되어
실행에 있어
사소한 실수라도 계약의무의 성립에 치명적인 것이 된다.
타인의 말을 신뢰하는 것에서 생겨나는 적극적 의무는
진보된 문명이 아주 나중에야 성취하게 되는 것이다.

\para{초기 로마의 계약들}
어떤 고대법도, 다른 어떤 전거도,
계약의 개념을 전혀 알지 못하는 사회가 있음을 보여주지 못한다.
그러나 이 개념이 처음 나타났을 때
그것은 분명 아주 원초적인 것이었다.
어떤 믿을 만한 원시 기록에서도
약속을 유효하게 만드는 인간의 정신이 아직 미숙하였음을,
그리고
노골적인 배신행위가 비난 없이, 때로는 칭송의 대상으로, 언급되고 있음을
읽을  수 있다.
가령 호메로스의 문헌에서
오디세우스의 기망적인 교활함은
네스토르의 현명함, 헥토르의 지조,
아킬레우스의 용기와 동급의 미덕으로 나타난다.
고대법은 계약의 원시적 형태가 그것의 성숙한 형태로부터
멀리 떨어져있었음을 훨씬 더 분명히 보여준다.
처음에는 단순히 약속의 이행을 강제하기 위해
법이 개입하지는 않았던 것으로 보인다.
법이 제재로써 강제하는 것은 단순한 약속이 아니라,
엄숙한 의례\hanja{儀禮}를 수반하는 약속이었다.
요식성\hanja{要式性}은 약속과 마찬가지로 중요했을 뿐만 아니라,
어쩌면 약속 이상으로 훨씬 더 중요했다.
성숙한 법학이 구두\hanja{口頭}의 승인\hanja{承認}이 행해진 상황에 적용하는
섬세한 분석이
고대법에서는
그것의 실행에 수반되는 말과 몸짓에 전가된 듯하다.
사소한 방식\latin{form} 하나라도 빠뜨리거나 잘못 실행되면 어떠한 서약도
강제될 수 없었다.
한편, 방식이 정확히 준수되었음이 입증된다면,
사기나 강박으로 약속하였다는 변론은 아무 소용이 없었다.
법제사에서는
이러한 고대적 관념으로부터 우리에게 친숙한 계약 관념으로의 이행이
명백히 드러난다.
처음에는 의례의 한 두 단계가 건너뛸 수 있는 것이 되고,
그후 일정 조건 하에서 다른 것들도 단순화되거나 생략이 허용되며,
마침내 몇몇 특수한 계약들이 다른 것들로부터 분리되어
방식의 구애를 받지 않고 체결할 수 있게 되거니와,
이들 특수한 계약은
사회적 거래의 활동성과 에너지가
이에
의존하는 계약인 것이다.
서서히, 그러나 사뭇 명백하게,
법기술적 요소들로부터 정신적 요소가 분리되어 나오고,
차츰 법학자들의 관심을 한몸에 받는 유일한 요소가 된다.
외부적 행위를 통해 표현되는
이러한 정신적 요소를 로마인들은
`약정'\hanjalatin{約定}{pact; convention}이라 불렀다.
그리고 약정이 계약의 핵심으로 인정되자,
곧이어
방식과 의례의 껍질을 부수어버리는 것이
진보하는 법의 경향성이 된다.
그후 방식들은 진정성을 보증하는 한에서만,
그리고 주의와 숙고를 담보하는 한에서만
보존될 뿐인 것으로 된다.
이로써 계약의 관념은 완전한 발달을 보이게 되거니와,
로마법의 용어를 사용하자면,
계약은 약정에 흡수되어버리는 것이다.

\para{양도와 계약}
로마법이 보여주는 이러한 변화 과정의 역사는 자못 시사적이다.
로마법의 여명기에
계약에 해당되는 말로 쓰인 용어는
고대 라틴어를 연구하는 학자들에게는 무척 익숙한 용어이다.
그것은 바로 넥숨\latin{nexum}, 즉 구속행위\hanja{拘束行爲}로서,
이 계약의 당사자들은 `피구속자'\hanja{被拘束者}들\latin{nexi}이라 불렸다.
이 표현들은 그 근저에 놓인 은유의 이례적인 지속성으로 인해
특히 주목할 필요가 있다.
계약관계에 놓인 사람들이 강력한 \hemph{속박}\latin{bond}
또는 \hemph{사슬}\latin{chain}로 연결되어 있다는
관념은 마지막까지 계속해서 로마 계약법에 영향을 주었고,
거기서 흘러나와 근대적 관념에도 섞여들어갔다.
그렇다면 이 구속행위 혹은 속박이란 무엇을 의미하는 것이었을까?
라틴어에 관한 고문헌을 통해 우리에게 전해진 바에 따르면
구속행위는 ``구리와 저울로써 행해지는
모든 것''\latin{omne quod geritur per aes et libram}이라고
정의되어 있거니와,\footnote{%
  Varro. De Lingua Latina. 7.105.
  }
이 단어들은 상당히 큰 혼란을 불러있으켰다.
구리와 저울은
악취행위에 수반되는 것들로 잘 알려져있다.
악취행위는
앞 장에서 서술한 고법\hanja{古法}상의 엄숙한 행위로서,
로마 물권법에서 높은 등급의 물건의 소유권이
한 사람에게서 다른 사람에게 양도되는 방식이었다.
이렇게 악취행위는 \hemph{양도}\latin{conveyance}의 방식이기에
어려운 문제가 부상하게 된다.
위에 인용한 저 정의는
계약과 양도를 혼동하고 있거니와,
법철학에서는 이 두 가지가 단지 구분될 뿐만 아니라
사실상 서로 대립하는 것이기 때문이다.
성숙한 법학의 분석가들은
물\hanja{物}에 대한 직접적 권리\latin{jus in re},
대세적\hanja{對世的} 권리\latin{right \textit{in rem}},
``온 세상에 대하여 주장할 수 있는'' 권리,
즉 물권\hanjalatin{物權}{proprietary right}과
물\hanja{物}에 대한 간접적 권리\latin{jus ad rem},
대인적\hanja{對人的} 권리\latin{right \textit{in personam}},
``특정인이나 특정집단에 대하여 주장할 수 있는'' 권리,
즉 채권\hanjalatin{債權}{obligation}을
날카롭게 구별한다.
그런데 양도는 물권을 이전하고, 계약은 채권을 창설한다.
어떻게 이 두 가지가 동일한 이름 아래, 동일한 일반개념 아래
포섭될 수 있다는 말인가?
다른 유사한 난제들과 마찬가지로 이 문제도
미발달된 사회의 정신적 상태에
진보된 지적 단계에 특별히 속하는 능력을,
현실에서는 혼재되어 있는 것을 사변적 관념들로 구별하는 능력을,
끼워맞추려는
오류 탓에 발생한 것이다.
여기서
우리는
양도와 계약이 현실적으로 혼재되어 있는 사회상태에 관하여
오인하지 말아야 한다는 시사를 받는다.
계약과 양도에 관하여 독자적인 실무관행이 채택되기 전까지는
저 개념들 간의 차이는 인식될 수 없었던 것이다.

\para{구속행위}
로마 고법\hanja{古法}에 관한 우리의 지식으로부터
법의 여명기에 법적인 개념과 법적인 용어가 어떻게 변해갔는지
그 변화의 양상에 대한 약간의 관념을 얻을 수 있을 것이다.
이 변화는 일반적인 것에서 특수적인 것으로의 변화라고 할 수 있다.
다시 말해 고법상의 개념과 고법상의 용어는 점진적 특수화의 과정을
겪었던 것이다.
고법상의 개념은 하나가 아니라 다수의 근대적 개념에 대응된다.
고법상의 법기술적 표현은 근대법이 여러 개의 이름으로 나누어놓은
다수의 것들을 지칭한다.
하지만 법사\hanja{法史}의 다음 단계에 이르면,
하위 개념들이 점차 서로 분리되어,
예전의 일반적 이름은 특수적 명칭들로 바뀌어가는 것이다.
그렇다고 옛 개념이 사라지는 것은 아니고,
단지 원래 포섭하던 관념의 일부만 포섭하게 된다.
그리하여 예전의 법기술적 이름은 여전히 존재하지만,
한때 수행했던 기능들 중에 하나만 수행할 뿐이다.
이러한 현상의 예로는 여러 가지를 들 수 있겠다.
가령 여러 종류의 가부장권은 한때
그 성격이 모두 동일했고,
따라서 하나의 이름으로 불리었을 것이 틀림없다.
존속친\hanja{尊屬親}에 의해 행사되던 가부장권은
가족에 대해 행사되든 물질적 재산에 대해 행사되든---양떼나 소떼, 노예,
자식, 아내를 불문하고---모두 동일했다.
그것의 옛 로마식 명칭에 대해 완전히 확신할 수는 없지만,
가부장\hemph{권}\latin{power}을 지칭하는 여러 명칭들에
\hemph{마누스}\latin{manus}\footnote{%
  `마누스'는 흔히 `수권'(手權)으로 번역되나 맥락상 원어를 살렸다.
  이하 관련 단어들도 마찬가지다.
}라는 단어가 들어가 있는 것으로 볼 때,
옛 일반적 명칭은 `마누스'였을 것으로 믿을 만한 근거는 충분해보인다.
그러나 로마법이 좀 더 진보하면서,
저 이름도 저 관념도 특수화되었다.
가부장권은
그것이 행사되는 대상에 따라
단어에서도 개념에서도 분화되어갔다.
물건이나 노예에 대해 행사될 때는
`도미니움'\latin{dominium},
자식들에 대해서는 `포테스타스'\latin{potestas},
존속친에 의해 다른 사람의 권력에 제공된 자유인에 대해서는
`만키피움'\latin{mancipium}이 되었고,
아내에 대해서는 여전히 `마누스'로 남았다.
여기서 알 수 있듯이,
원래의 단어가 전혀 쓰이지 않게 된 것이 아니라,
과거에 지칭하던 권력행사 중 특수한 한 가지 권력행사에 국한하게 된 것이다.
이 사례를 모범삼아 계약과 양도 간의 역사적 결합관계의 성질에 대해서도
이해를 도모할 수 있을 것이다.
처음에는 모든 엄숙한 거래에 오직 하나의 엄숙한 의례\hanja{儀禮}만
존재했을 것이니,
로마에서는 그것의 명칭이 `구속행위'\latin{nexum}였던 것으로 보인다.
물건의 양도에 사용되던 바로 그 방식이
계약의 체결에도 사용되었던 것으로 보인다.
그러나 양도 관념으로부터 계약 관념이 분리되어 나오는 데는
그다지 긴 시간이 필요치 않았다.
그리하여 이중\hanja{二重}의 변화가 일어났다.
``구리와 저울에 의한'' 거래가
물건의 이전을 의도하는 경우에는
`악취행위'\latin{mancipation}라는 새롭고 특수한 이름으로 불리게 된다.
옛 이름인 `구속행위'는 여전히 동일한 의례절차를 지칭하지만,
이제
오직 계약을 엄숙하게 체결하는 특수한 목적에만 쓰이게 된다.

\para{변화}
두 세 가지 법개념이 고대에는 하나로 혼재되어있었다고 해서,
거기에 포함된 관념 중 하나가 다른 것들보다 더 오래된 것이 아니라는
말은 아니다.
혹은 그 하나가, 다른 것들이 형성된 후,
이것들보다 크게 우세하거나 우선하지 않는다는 말도 아니다.
하나의 법개념이 오래 계속해서 여러 법개념들을 포괄할 수 있는 이유는,
그리고 하나의 법기술적 용어가 여러 용어들을 대신할 수 있는 이유는,
원시사회의 법에 실무관행의 변화가 일어나더라도
오랫동안 사람들은 그것에 주목하거나 이름붙일
필요를 느끼지 못하기 때문일 것이 분명하다.
비록, 전술했듯이,
처음에는 가부장권에 행사대상에 따른 구별이 없었다 할지라도,
자식들에 대한 권력이 옛 가부장권의 근본이었다고
나는 믿어 의심치 않는다.
또한
`구속행위'라는 말의 최초의 사용은,
그리고 그것을 사용했던 사람들이 주로 염두에 둔 것은,
물건의 양도에 엄숙한 형식을 부여하려는 것이었음을
나는 의심치 않는다.
구속행위가 원래의 기능으로부터 아주 조금 벗어나기 시작했을 때
그것은 바로 계약의 체결에 사용되었을 것이나,
아주 조금의 변화였기에 그 변화는 오랫동안 인정되거나 감지되지 못했다.
새로운 것을 원한다는 것을 사람들이 자각하지 못했기 때문에
옛 이름은 그대로 남았다.
아무도 수고스럽게 새로운 것을 검토해볼 필요를 느끼지 못했기 때문에
옛 관념은 그대로 남았다.
우리는 이러한 과정의 사례를 유언법의 역사에서 명료하게 살펴본 바 있다.
유언은 처음에는 단순히 재산의 양도였다.
점차 이러한 특수한 양도와 다른 모든 양도 간에 커다란 실무상의 차이가
나타나고 나서야 비로소
이들이
서로 다른 것으로 간주되기 시작했고,
그러고도 수 세기가 흐른 뒤에야
법의 개량에 나선 사람들이
허울뿐인 악취행위에 붙어있던 복잡한 절차를
청소했고 마침내
유언에 있어
유언자의 명시적 의사 외에는 다른 어떤 것도 중요하지 않다는
합의가 이루어졌던 것이다.
유언법의 초기 역사만큼의
절대적 확신을 가지고
계약법의 초기 역사를 추적할 수가 없다는 것은 유감스런 일이지만,
구속행위가 새로운 사용에 놓여짐으로써
계약이 처음 등장했고
이어서
이 실험의 중차대한 실무적 결과로써
계약이 독자적 거래형태로 승인되었다는 것을 암시하는
힌트마저 얻을 수 없는 것은 아니다.
다음과 같은 과정을 대체로 따랐을 것이라는 추측이,
그러나 그리 억지스럽지만은 않은 추측이 가능하다.
구속행위의 통상적인 방식에 의해
일정한 대금을 받고 매매가 행해진다고 가정하자.
매도인은 처분하고자 하는 목적물---가령 노예 한 명---을 가지고 온다.
매수인은 매매대금을 해당하는 구리 덩어리를 가지고 참석한다.
필수적 보조인인 저울소지자\latin{libripens}도 저울을 들고 나와있다.
노예는 정해진 요식절차에 따라 매수인에게 건네진다.
저울소지자는 구리 조각을 저울에 달고는 매도인에게 넘겨준다.
이러한 거래행위가 지속되는 한 그것은 `구속행위'이고,
당사자들은 `피구속자'들\latin{nexi}이다.
그러나 그것이 완료되자마자,
구속행위는 끝나고,
매도인과 매수인도 그들의 일시적 관계에서 유래하는 이름으로 불리기를 그친다.
이제 여기서
거래의 역사를 한 걸음 진척시켜보자.
노예는 양도되었으나,
대금은 지불되지 않았다고 가정해보자.
\hemph{이} 경우,
매도인에 관한 한 구속행위는 종료된다.
이미 자기 물건을 넘겨주었으므로 그는 더 이상 `피구속자'\latin{nexus}가 아니다.
그러나 매수인에 관해서는 구속행위가 계속된다.
매수인 쪽에서는 거래가 끝나지 않았고 그는 여전히 `피구속자'로 남는다.
따라서 동일한 용어가 물권을 이전하는 양도를 기술\hanja{記述}함과 동시에
미지불된 매매대금에 관한 채무자의 채무도 기술하고 있음을 알 수 있다.
다시 한 걸음 더 나아가,
완전히 형식적인 거래, 즉 아무 것도 건네지지 \hemph{않고}
아무 것도 지불되지 \hemph{않는} 절차를 상상해보자.
우리는 사뭇 발달된 상거래 행위의 하나, 바로
\hemph{미이행}\hanjalatin{未履行}{executory} \hemph{매매계약}에
도달하게 되는 것이다.

\para{양도와 계약}
대중적 견해에서나 전문가적 견해에서나
\hemph{계약}이라는 것이
오랫동안 \hemph{미완의 양도}\latin{incomplete conveyance}라고
간주된 것이 사실이라면,
이 사실은 여러 모로 의미심장하다.
자연상태의 인류에 관한 지난 세기의 사변적 이론을
``원시사회에서는 물권은 아무 것도 아니었고 채권이 모든 것이었다''는
교리로 요약하는 것이 그다지 부당하지는 않을 것이다.
그러나 이제 우리는
저 명제를 거꾸로 뒤집으면
그것이 오히려 진실에 가깝다는 것을 알게 되었다.
다른 한편,
역사적으로 보면,
양도와 계약의 원시적 결합은
학자들과 법률가들에게 특별히 수수께끼로 여겨지곤 했던 어떤 것을
설명할 수 있을 것이다.
초기 고대법은 어디서나 \hemph{채무자들}을 무척 가혹하게 처우했으며,
\hemph{채권자들}에게는 막강한 권한을 주었다는 수수께끼 말이다.
구속행위가 채무자에게는 인위적으로 긴 시간 동안 지속되었음을 알고 나면,
대중들과 법이 바라보는 그가 지위가 어떤 것이었을지
더 잘 이해할 수 있는 것이다.
그의 채무상태는 틀림없이 비정상적인 것으로 여겨졌을 것이고,
지불의 해태\hanja{懈怠}는 일반적으로
간교한 책략이자 엄격법의 왜곡으로 비춰졌을 것이다.
반대로,
거래에서의 자신의 의무를 성실하게 완수한 사람은
특별한 호의로써 대우받았을 것이니,
엄격법에 따르면 연장되거나 지체되어서는 안 될
어떤 절차를 강제로 완성시킬 권한을 그에게 주는 것보다
더 당연한 일은 없어보인다.

\para{로마법의 합의 분석}
따라서 구속행위는 원래 재산의 양도를 의미했지만,
부지불식간에 계약도 의미하게 되었고,
구속행위 개념과 계약 관념 간의 결합이 오랫동안 지속되었기에
마침내
악취행위\latin{mancipatio}라는 특별한 용어가
진정한 구속행위, 즉 실제로 재산이 양도되는 거래를 지칭하는 데
사용되게 되었다.
그리하여 계약은 이제 양도와 분리되었고
이로써 계약법 역사의 첫 단계가 마무리되었으나,
계약당사자의 약속이 이를 둘러싼 요식성보다 더 신성\hanja{神聖}하게 평가되는
단계에 이르기까지는 아직 한참 멀리 떨어져있었다.
그 사이 기간 동안 진행된 변화의 성격을 알아보려면,
지금 우리가 다루고 있는 주제의 범위를 살짝 넘어갈 필요가 있거니와,
바로 로마 법학자들이 합의\latin{agreement}를 어떻게 분석했는지
살펴보는 것이다.
그들의 재능이 만들어낸 가장 아름다운 기념비인
이 분석에 관하여, 나는
그것이 채권채무관계\latin{obligation}를 약정\latin{pact}으로부터
이론적으로 분리하는 데 기초하고 있다는 것 이상을 말할
필요를 느끼지 않는다.
벤담과 오스틴 씨는 이렇게 주장했다.
``계약의 주요 성질은 다음 두 가지이다:
첫째,
하기로 약속하는 작위를 하겠다는,
또는
하지 않기로 약속하는 부작위를 하지 않겠다는
낙약자\hanja{諾約者}의
\hemph{의사}의 표시.
둘째,
이 주어진 약속을 낙약자가 이행할 것이라는 데 대한
요약자\hanja{要約者}의
\hemph{기대}의 표시.''
이것은
로마 법률가들의 법리와 거의 동일하지만,
그러나 로마 법률가들은
이러한 ``표시''의 결과를 `계약'이 아니라
`약정'으로 보았다.
약정은 개인들 간의 합의로 맺어지는 약속의 최종 산물이지만,
그렇다고 그것이 바로 계약인 것은 아니다.
약정이 계약이 되는가 여부는
법이 그것에 채권채무관계를 덧붙이느냐 여부에 달려있다.
계약이란 `약정' \hemph{더하기} `채권채무관계'인 것이다.
약정이 채권채무관계의 옷을 입고 있지 않는 한,
그것은 \hemph{나}약정\hanja{裸約定}, 즉 \hemph{벌거벗은} 약정이라 불렸다.

\para{로마법의 채권채무관계}
채권채무관계\latin{obligation}란 무엇인가?
로마 법률가들의 정의에 따르면
``누군가에게 급부\hanja{給付}를 할 것이 필히 강제되는
법의 사슬''\latin{juris vinculum, quo
necessitate adstringimur alicujus solvendae rei}이다.
이 정의는
채권채무관계를 구속행위와 연결짓거니와,
이들의 배경에 놓인 공통의 은유를 통해서 그러하다.
또한 이 정의는
특정 개념의 계보를 사뭇 명료하게 보여준다.
채권채무관계는 ``속박'' 또는 ``사슬''이거니와,
이로써
사람들 혹은 사람들의 집단들은
어떤 자발적 행위의 결과로서
법에 의해 하나로 결속되는 것이다.
채권채무관계를 이끌어내는 행위들은 주로
합의와 위법행위, 즉
계약과 불법행위라는 표제 아래 분류되지만,
정확하게 분류되기 힘든 여러 다른 행위들도 유사한 결과를 낳는다.
하지만 유의할 점은
어떤 도덕적 필요성도 약정을 바로 채권채무관계로 만들지는 못한다는 것이다.
약정에 채권채무관계의 힘을 완전히 부여하는 것은 법이다.
이 점 더욱 유의할 필요가 있거니와,
도덕적 또는 형이상학적 이론을 지지하는 근대 대륙법학자들에 의해
때로 이와 다른 법리가 주창되어왔기 때문이다.
`법의 사슬'\latin{vinculum juris}이라는 이미지는
로마 계약법과 불법행위법의 모든 부분을 물들이고 있고 지배하고 있다.
법은 당사자들을 속박하거니와,
이 \hemph{사슬}은 `변제'\hanjalatin{辨濟}{solutio}라고 불리는 과정을
통해서만 풀 수 있다.
`변제'라는 표현도 은유적이거니와,
``지불''\latin{payment}이라는 일상용어는 가끔씩 그리고 우연히
여기에 들어맞을 뿐이다.
이들 은유적 이미지의 일관성은
이것이 없었다면 혼란을 초래했을
로마법 용어의 특별한 의미를 이해할 수 있게 해준다.
즉, ``채권채무관계''\latin{obligation}는 의무뿐만 아니라
권리도 의미하는 것이다.
이를테면 빌린 돈을 지불할 의무뿐만 아니라
빌려준 돈을 지불받을 권리도 의미한다.
실로 로마인들의 눈 앞에는 ``법의 사슬''의 전체 그림이
펼쳐져있었으며,
사슬의 한쪽 끝을 다른 쪽 끝보다 더 많이 혹은 더 적게
바라보지 않았다.

\para{약정과 계약}
발달된 로마법에서는
약정이 체결되자마자 거의 모든 경우
즉시
채권채무관계의 관\hanja{冠}이 씌워지고, 따라서 계약이 된다.
이것은 분명 계약법이 지향하는 결과이다.
그러나 우리의 탐구의 목적을 위해서는
그 중간 단계, 즉 채권채무관계가 되려면 완전한 합의 이상의 어떤 것이
요구되는 단계에 주목해야 한다.
이 단계는
네 종류의 계약---언어계약, 문서계약, 요물계약, 낙성계약---을 구분한,
저 유명한 로마법상의 분류가 사용되던 시기와 일치한다.
이 시기 동안 법에 의해 강제된 약속은 저 네 가지에 국한되었다.
채권채무관계를 약정과 분리하는 이론을 알고 있다면
저 네 가지 항목의 의미는
쉽게 이해될 수 있다.
사실,
계약들의 각 항목은
계약당사자들의 단순한 합의 이외에 어떤 요식성이 필요한가에 따라
이름붙여진 것이다.
언어\latin{verbal}계약에서는 약정이 체결되는 순간
일정한 방식의 말들이 발설되어야 ``법의 사슬''이 부여된다.
문서\latin{literal}계약에서는
원장\hanja{元帳}, 즉 회계장부에 기입되어야
약정에 채권채무관계의 옷이 입혀진다.
요물\latin{real}계약의 경우
예비적 약속의 목적물인
물건의 인도가 있어야 동일한 결과가 뒤따른다.
요컨대,
이 모든 경우
계약당사자들 간에는 의사합치가 있어야 하지만,
거기에만 그친다면 그들은 서로에게 \hemph{채권채무}를 갖지 못하고,
따라서 이행을 강제할 수도,
신의\hanja{信義} 위반을 이유로 배상을 청구할 수도 없다.
그러나 그들이 어떤 정해진 요식성을 충족시킨다면,
계약은 바로 체결되고,
그 계약의 이름은 그들이 채택한 특정한 방식에 따라 붙여지는 것이다.
이러한 관행에 대한 예외는 조금 있다 살펴보겠다.

\para{언어계약}
나는 네 가지 계약들을 역사적 순서에 따라 열거했으나,
로마의 법학제요 저자들이 이 순서를 반드시 따른 것은 아니다.
언어계약이 네 가지 중에 가장 오래된 것임에는 의심의 여지가 없다.
그것은 원시적 구속행위의 후손으로 알려진 것 중에 가장 먼저 나타났다.
언어계약에 속하는 몇몇 종\hanja{種}이 옛날에는 사용되었으나,
가장 중요한 것이자 우리의 전거들이 다루었던 유일한 것은
질문과 답변으로 이루어진 `문답계약'\latin{stipulation}이다.
요약자가 질문을 하고 낙약자가 답변을 한다.
이러한 질문과 답변이야말로, 전술했듯이,
원시적 관념이
당사자들 간의 단순한 합의를 넘어 추가적으로 요청하는 요소인 것이다.
이들은 채권채무관계가 부여되기 위한 매개체이다.
옛 구속행위는
보다 성숙한 법학에게 무엇보다
계약당사자들을 결속시키는 사슬의 개념을 물려주었으니,
이것이 이제 채권채무관계가 되었다.
그것은 또한 약속에 수반하면서 약속을 성별\hanja{聖別}하는
의례행위의 개념도 물려주었으니,
이 의례행위가 이제 질문과 답변으로 변형된 것이다.
초기 구속행위의 특징이었던 엄숙한 양도행위가
어떻게 단순한 질문과 답변으로 전환되었을까 하는 것은
이와 유사한 로마 유언법의 역사가 우리에게 가르쳐준 바가 없었다면
더욱 미스테리로 남았을 것이다.
유언법의 역사를 돌아보면,
실질적 관심 대상에 직접 관련되는 절차 부분\footnote{%
  `양도'와 대비되는 `언명'(nuncupatio)을 말하는 듯하다.
}으로부터
어떻게
형식적 양도가
처음 분리되었는지를,
그리고 어떻게 그후 이것을 완전히 생략하게 되었는지를
이해할 수 있다.
그렇다면
문답계약의 질문과 답변은 단순화된 형태의 구속행위였을 것이 분명하고,
따라서
그것은 오랫동안 법기술적 형태의 성질을 가졌을 것이라고 쉽게 추정할 수 있다.
옛 로마 법률가들이 문답계약을 옹호했던 이유를
합의에 임한 당사자들에게 숙고하고 성찰할 기회를 제공하는
유용성때문이라고 본다면 이는 잘못일 것이다.
물론 이런 유\hanja{類}의 가치가 있었고 점점 중요하게
인식되었다는 것을 부인할 수는 없다.
그러나, 우리의 전거들에 나타난 증거에 비추어볼 때,
계약에 관련된 그것의 기능은 처음에는 형식적이고 의례적인 것이었다.
문답계약을 구성할 수 있는 오래된 질문과 답변은
특정한 경우에 적합한 법기술적 용어들을 사용한
질문과 답변에만 국한되었고,
아무 질문이나 답변이라고 해서 다 되는 것은 아니었다.

\para{문답계약}
그러나,
비록 문답계약이 유용한 안전장치로 인식되기 이전에
엄숙한 형식으로 인식되었다고 이해하는 것이
계약법의 역사를 올바르게 평가하는 데 필수적이라 할지라도,
다른 한편
그것의 현실적 유용성에 눈을 감아버리는 것도
잘못된 일일 것이다.
언어계약은, 비록 고법\hanja{古法}상에서 누리던 중요성을
상당 부분 상실해갔지만, 로마법의 마지막 시기까지 계속 살아남았다.
당연한 말이겠지만,
로마법의 어떤 제도도
어떤 현실적 유용성이 없었다면
그렇게 오래 유지될 수 없었을 것이다.
놀랍게도
어떤 영국 학자의 말에 따르면,
로마인들은
초창기부터도
숙고없이 서둘러 계약을 맺는 것에 대한 방비가 거의 없어도
괘념치 않았다고 한다.
그러나 문답계약을 면밀히 조사해보면,
그리고 서면증거를 만들기 쉽지 않았던 당시의 사회상태를 감안하면,
문답계약의 질문과 답변은,
만약 그것이 실제 기여한 목적을 위해 의도적으로 고안되었다면,
그야말로 천재적인 방책이었다고 평가해도 좋다고 생각한다.
\hemph{요약자}\latin{promisee}가
계약의 모든 조항을 질문의 형태로 만들어 질문하면,
\hemph{낙약자}\latin{promisor}가 답변을 한다.
``당신은 이러이러한 노예를 이러이러한 장소에서 이러이러한 날짜에
나에게 인도할 것을 약속하는가?''
``약속하노라.''
잠깐만 생각해보면,
이렇게 질문 형태로 구성되는 채권채무관계는
당사자들의 자연스런 입장을 거꾸로 뒤집고,
대화의 통상적인 흐름을 깨뜨리는 효과를 가져와,
위험한 약속에 빠지지 않도록 주의를 환기시키는 기능을 함을 알 수 있다.
우리 영국인들이 행하는 구두의 약속은
오직 약속자\latin{promisor}의 말로 구성되는 것이 일반적이다.\footnote{%
  원어로는 똑같이 `promisor'이지만,
  로마법의 맥락에서는 `낙약자'로,
  영국법의 맥락에서는 `약속자'로 번역하고 있음을 유의할 것.
  사실 `요약자'니 `낙약자'니 하는 우리의 법률용어는
  바로 로마법의 문답계약에서 유래하는 말이다.
  }
옛 로마법에서는 또 하나의 단계가 반드시 요구된다.
합의가 이루어지고 나면
요약자가 엄숙한 질문의 형태로 이 합의의 모든 조항들을 요약해야 하는 것이다.
재판에서 증거로 제출되는 것은,
구속력 없는 약속 자체가 \hemph{아니라},
바로 이 질문과 그에 대한 답변인 것이다.
일견 사소해보이는 이 차이가
계약법 용어에 얼마나 큰 차이를 만들어내는지는
로마법 입문자들이 입문 후 금세 깨닫게 되는 것이니,
그들은 첫 번째 걸림돌을 거의 언제나 여기서 만나고 있다.
우리가 영어로 어떤 계약에 관해 말하면서
그것을 편의상 한쪽 당사자와 결부시키는 경우---가령
어떤 계약의 당사자에 대해 일반적으로 말하려는 경우---우리의
말이 지시하는 것은 언제나
\hemph{약속자}\latin{promis\textit{or}} 쪽이다.
그러나 로마법의 일반적 언어는 방향이 반대이다.
로마법은 계약을 언제나, 이런 용어를 쓸 수 있다면,
\hemph{수약자}\hanjalatin{受約者}{promis\textit{ee}} 쪽에서 바라본다.
계약의 당사자 중에서
주로 언급되는 대상은 언제나 요약자\latin{stipulator}, 즉
질문을 하는 사람이다.
하지만
문답계약의 유용성이 자못 생생하게 드러나는 예를
라틴 희극작가들의 희곡의 몇몇 페이지에서도 찾아볼 수 있다.
이 대목들이 나오는 장면 전체를 읽어보면
\paren{예컨대, Plautus, \textit{Pseudolus}, 1막 1장;
4막 6장; \textit{Trinummus}, 5막 2장},
질문하는 것이 계약에 임한 사람의 주의를 얼마나 많이 이끌어내는지,
그리고
즉흥적인 합의에 이르지 않게 할 가능성이 얼마나 커지는지
알 수 있을 것이다.

\para{문서계약}
문서계약에서 약정에 채권채무관계가 입혀지기 위해 필요한
요식행위는,
채무액이 확정될 수 있는 경우,
채무의 총액을
원장\hanja{元帳}의 차변\hanja{借邊}에 기입하는 것이었다.
이 계약은 로마의 가\hanja{家}의 관행에 의해 설명될 수 있거니와,
고대에는 그것이 체계적인 성격을 띠었고 무척 규칙적으로 장부작성이
이루어졌던 것이다.
로마 고법\hanja{古法}과 관련하여
가령 노예의 특유재산\latin{peculium}의 성격 같은
몇몇 작은 문제들이 있거니와,
이는
로마의 가\hanja{家}가 가부장에게 엄격히 책임지는 다수의 사람들로
구성되었고,
가의 수입과 지출의 모든 항목은,
일단 일지\hanja{日誌}에 기재된 후,
정해진 시기에
가의 총\hanja{總}원장에 이기\hanja{移記}되었음을 상기할 때
비로소 해소될 수 있다.
하지만 문서계약에 관해 남아있는 기술\hanja{記述}에는
몇 가지 모호한 점이 있거니와,
사실
나중에는
장부작성의 습관이
보편적이지 않게 되었고,
``문서계약''이라는 표현은 이제 원래 가졌던 의미와 완전히
다른 형태의 계약을 지칭하게 되었던 것이다.
따라서 우리는
초기의 문서계약에서
채권채무관계가 단순히 채권자 측의 기입만으로 성립했는지,
아니면
그것이 법적 효력을 가지려면
채무자의 동의나 채무자 측 장부의 상응하는 기입도 필요했는지
말할 수 있는 입장에 있지 않다.
하지만
이 계약의 경우
한 가지 조건만 충족되면 다른 모든 요식성은 필요치 않다는
핵심적 성격만은 확실히 알려져있다.
이것은 계약법의 역사에서 또 한 걸음의 진전이었던 것이다.

\para{요물계약}
역사적 순서에 따라 다음에 등장하는 계약인 요물계약은
도덕 개념의 큰 진전을 보여준다.
어떤 합의가 물건의 인도를 목적으로 한다면---이는
대다수의 단순한 계약을 포괄한다---그 인도가 실제로 행해지는 즉시
채권채무관계가 성립하는 것이다.
이러한 결과는 초기 계약 관념에 큰 혁신을 의미했다.
의심의 여지 없이 초창기에는,
계약의 당사자가 자신의 합의에 문답계약의 옷을 입히지 못했다면,
계약의 이행으로써 무엇을 했던지 간에
법은 아무 것도 인정해주지 않을 것이기 때문이다.
공식적으로 \hemph{문답계약}을 체결하지 않았다면
돈을 빌려주었더라도 빌린 돈을 갚으라고 소송할 수 없었던 것이다.
그러나 요물계약에서는
일방의 이행이 상대방에게 법적 의무를---분명 도덕적 근거에서---부과한다.
그리하여 도덕적 고려가 계약법의 요소로 처음으로 등장한 것이다.
요물계약이
앞서 살펴본 두 가지와 다른 점은,
법기술적 방식이나 로마의 가\hanja{家}의 습관에 대한 존중이 아니라,
도덕적 고려에 기초한다는 데 있다.

\para{낙성계약}
이제 네 번째 유형, 즉 낙성계약에 이르렀거니와,
이것은 가장 흥미롭고 가장 중요한 유형이다.
여기에는 네 가지 계약들이 속했고, 그 이름은 다음과 같다:
위임\latin{mandatum}, 조합\latin{societas},
매매\latin{emtio-venditio}, 임약\hanjalatin{賃約}{locatio-conductio}.\footnote{%
  `임약'은 우리 민법의 `임대차' `고용' `도급'을 포괄하는 개념이다.
  }
몇 페이지 앞에서
계약이란 약정에 채권채무관계가 덧붙여진 것이라고 말한 후,
나는
약정이 채권채무관계로 되기 위해 법이 요구하는
어떤 행위나 요식성에 관해 이야기했다.
나는 일반적 표현의 장점을 활용해 이 말을 하였으나,
적극적인 것 외에 소극적인 것까지 포괄하는 것으로 이해하지 않으면
전적으로 옳은 말이 되지는 못한다.
기실,
낙성계약의 특이성은 약정 외에 그 어떤 요식성도
\hemph{전혀} 요구되지 않는다는 것이기 때문이다.
낙성계약에 관하여 많은 옹호될 수 없는 것들이,
더 많은 모호한 것들이 주장되어왔거니와,
심지어 낙성계약에서는 당사자들의 \hemph{동의}\latin{consent}가
다른 어떤 합의 유형들보다 더 강하게 주어진다는 주장까지 있었다.
그러나 저 `낙성'\latin{consensual}이라는 용어는
여기서는
단지 \hemph{합의}\latin{consensus}만 있으면 바로 채권채무관계가
부가된다는 것을 의미할 뿐이다.
합의, 즉 당사자들의 상호 동의는
약정에 있어 최종의 그리고 최고의 요소이다.
당사자들의 동의가 이 요소를 제공하자마자
\hemph{즉시} 계약이 성립하는 것은
매매, 조합, 위임, 임약의 네 가지 표제 중 하나에 속하는
합의의 특유한 성질이다.
이 합의는 바로 채권채무관계를 끌고 들어오기에,
특정 종류의 거래에서 그것이 행하는 기능은
다른 종류의 계약에서 물건이, 문답의 언어가,
문서, 즉 장부에의 기입이 행하는 기능과 정확히 일치한다.
따라서 `낙성'은
조금도 이상할 것이 없는 용어이며,
`요물' `언어' `문서'에 정확히 대응한다.

일상생활에서
가장 흔하고 가장 중요한 계약은, 말할 것도 없이,
네 가지 이름의 계약을 포괄하는 낙성계약이다.
어떤 공동체든 집단생활의 대단히 큰 부분이
사고 파는 거래, 세\hanja{貰}놓고 세드는 거래,
공동사업을 위해 사람들이 결합하는 거래,
업무처리를 타인에게 맡기는 거래로 구성된다.
바로 그 때문에
다른 많은 사회들과 마찬가지로 로마도
이들 거래에서 법기술적 장애물을 제거하여,
사회적 운동의 효율적 동력이
가능한 한
방해받지 않도록 했을 것이다.
물론 이러한 동기는 로마에만 국한된 것이 아니었다.
로마인들과 이웃 민족들 간의 거래는
우리가 말한 계약들이 어디서나 \hemph{낙성계약},
즉 상호 동의의 의사표시만으로 구속력이 부여되는 계약이
되어가는 것을 관찰할 수 있는
풍부한 기회를
로마인들에게
제공했을 것이다.
그리하여 그들의 통상적인 관행에 따라
로마인들은 이들 계약을
만민법상의\latin{juris gentium} 계약으로 분류했다.
하지만 나는
아주 초기부터 이렇게 불리지는 않았다고 생각한다.
만민법\latin{jus gentium}이라는 관념은
외인법무관\latin{praetor peregrinus}이 임명되기 오래 전부터
로마 법률가들의 정신에 이미 들어있었다.
그러나 다른 이탈리아 공동체들의 계약 관행에 로마인들이 익숙해진 것은
수많은 거래가 일상적으로 이루어지면서부터일 것이고,
이러한 거래는 이탈리아가 완전히 평정되어
로마의 패권이 확고해지고 나서
비로소 대규모로 이루어질 수 있었을 것이다.
하지만, 비록
낙성계약이 가장 늦게 도입된 것일 확률이 무척 크다고 할지라도,
그리고
`만민법상의'\latin{juris gentium}라는 수식어가 그것의 뒤늦은 도입을
나타내는 것이라 하더라도,
낙성계약을 ``만민법''\latin{law of nations}에 귀속시키는
바로 이 표현이 근대에 들어서는
그것이 아득한 옛날의 것이라는 관념을 만들어냈다.
``만민법''\latin{law of nations}이
``자연법''\latin{law of nature}으로
전환되자,
저 표현은
낙성계약이 자연상태에 가장 부합하는 합의 유형임을
의미한다고 여겨졌던 것이다.
그리하여 문명의 나이가 어릴수록
계약의 형태는 더 단순할 것이라는 이상한 믿음이 형성되었다.

\para{자연법적 채무와 시민법적 채무}
전술했듯이 낙성계약에 속하는 계약들은 그 수가 대단히 제한적이었다.
그러나
낙성계약으로 대표되는 계약법 발달의 단계가
계약에 관한 모든 근대적 관념의 출발점이었음은 의심할 여지가 없다.
이제 합의를 구성하는 의사\latin{will} 작용은
다른 것들과 완전히 분리되어 독립적 고찰의 대상이 되었다.
계약 관념에서 방식은 완전히 제거되었고,
외적 행위는 오직 내적 의사의 징표로만 간주되었다.
더욱이 낙성계약은 만민법\latin{jus gentium}으로 분류되었으니,
이러한 분류는
얼마 안 가
낙성계약이야말로
자연에 의해 승인되고 자연의 법전에 포함된
계약을 대표하는 합의 유형이라는 추론을 형성시켰다.
이 지점에 이르러, 우리는
로마 법률가들의 몇몇 유명한 법리와 구분들을 만나게 된다.
그중 하나가 자연법상의 채무\latin{natural obligation}와
시민법상의 채무\latin{civil obligation}의 구분이다.
지적으로 완전히 성숙한 어떤 사람이
자신의 의사에 기해
어떤 약속을 맺었다면,
비록 필요한 방식을 다 갖추지 못했더라도,
또는 어떤 법기술적 장애로 인해
유효한 계약을 체결할 법적 능력이 결여되어 있었다 할지라도,
그는 \hemph{자연채무}\latin{natural obligation}를 진다.
법은
\paren{바로 이것이 저 구분이 의미하는 바이다}
이런 채무를 강제하지 않는다.
그러나 법이 이런 채무를 전혀 인정하지 않은 것은 아니다.
\hemph{자연채무}는
단순히 무효인 채무와는 여러 모로 달랐거니와,
특히 계약 체결 능력을 사후적으로 취득한다면
시민법적으로도 인정될 수 있었던 것이다.\footnote{%
  가령 노예가 진 빚은 자연채무였다. 따라서 노예가 해방된 뒤
  스스로 빚을 갚으면 되돌려받을 수 없었다.
  Dig.\,12.6.13.pr.
  }
법학자들의 또 하나의 특유의 법리는 약정이
계약의 법기술적 요소로부터 분리된 후에 비로소 등장할 수 있었다.
그들에 따르면,
계약만이 \hemph{소권}\hanjalatin{訴權}{action}을 근거지울 수 있었지만,
단순한 약정은 \hemph{항변권}\hanjalatin{抗辯權}{plea}의 기초가 될 수 있었다.
그리하여,
누구도
계약을 성립시키는 데 필요한
방식을 갖추지 못한
합의에 기초하여 소송을 제기할 수 없었지만,
유효한 계약에 기초한 주장이라도
단순한 약정 상태에 불과한 다른 합의가 있었음을 입증함으로써
이를 물리칠 수 있었다.
가령 금전채무의 회수를 구하는 소송은
채무 면제나 유예의
단순한 비공식적 합의를 입증하여 이에 대항할 수 있었다.

\para{계약법의 변화}
방금 언급한 법리는 법무관들이 궁극적 혁신을 이루어내는 데
방해물로 작용했을 것이다.
그들의 자연법 이론은
낙성계약과 이를 포함하는 약정 일반에 대해
특별한 호의를 갖도록 그들을 이끌었을 것이 틀림없다.
그러나 그들이 즉시
모든 약정에 낙성계약의 효력을 확대적용하는 모험을 감행한 것은 아니었다.
그들은 로마법 초기부터 그들에게 주어졌던
소송절차에 대한 감독권한을 이용하였으니,
방식을 갖추지 못한 계약에 근거한 소송은 여전히 허가하지 않았지만,
합의에 관한 새로운 이론을 적극 활용하여
이후의 발달 단계로 향하는 길을 텄다.\footnote{%
  이른바 `법무관법상의 약정'(pacta praetoria)을 말하는 듯하다.
  특정 기일에 자기 또는 타인의 기존 채무를 변제하겠다는 약정,
  해운업자^^b7여관주인^^b7마굿간주인 등의 고객 물건에 대한 인수(引受)약정
  따위가 여기에 속한다.
  }
그러나
여기까지 나아가자
계속 더 나아가는 것이 불가피해졌다.
고대 계약법의 혁명이 달성된 것은,
어느 해인가 법무관의 고시\hanja{告示}가
계약의 옷을 입지 못한 약정이라도
그 약정이 대가관계\latin{consideration}\paren{원인\latin{causa}}에
기초하는 것인 한
형평법상의 소송을 허가하겠노라고 공표했을 때였다.\footnote{%
  이른바 `무명요물계약'(無名要物契約)을 말하는 듯하다.
  쌍무성(synallagma) 있는 계약의 당사자라면
  자신의 급부를 이행한 경우---따라서 일종의 요물계약이었다---상대방의
  이행을 소구할 수 있었다.
  혹은 자신의 급부를 돌려달라는 부당이득반환청구소송을 제기할 수 있었다.
  }
이런 종류의 약정은 발달된 로마법에서는 항상 강제되었다.
이 원리는
낙성계약의 원리가 그 합당한 결과에 도달한 것에 불과했다.
사실, 로마인들의 법기술적 용어가 그들의 법이론만큼이나 유연했다면,
법무관에 의해 강제된 이들 약정은
새로운 계약, 새로운 낙성계약이라고 불렸을지도 모른다.
하지만 법용어는 가장 바뀌기 어려운 법이어서,
형평법적으로 강제된 저 약정들은
여전히 `법무관법상의 약정'\latin{praetorian pacts}이라고만 지칭되었다.
만약 약정에 대가관계가 없다면,
새로운 법에서도 계속 \hemph{나}약정\hanja{裸約定}이었음을
유의해야 한다.
이것에 법적 효력을 부여하려면,
문답계약을 통해 언어계약으로 전환시켜야 했다.

\para{계약법의 진화}

