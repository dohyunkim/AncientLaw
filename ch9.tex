\chapter{계약법의 초기 역사}

우리가 속한 시대에 관한 명제로,
오늘날의 사회가
지난 시대의 사회와 차이 나는 주요 특징은
계약법이 차지하는 영역이 대폭 증가했다는 데 있다는
주장만큼
일견 쉽게 수긍할 수 있을 법한 것도 없을 것이다.
이 명제가 근거하고 있는 현상들 중 일부는
대단히 빈번하게 선택되어 관심과 논평과 칭송의 대상이 되고 있다.
%우리들 중에서
옛 법이 사람의 출생에 따라 그의 사회적 지위를
불가역적으로 고정시켰던 수많은 사안들에서
근대법은 합의에 의해 그 스스로 자신의 지위를 만들어갈 수 있도록
허용하고 있음을
알아차리지 못할 정도로
무감한 사람은
별로 없을 것이다.
실로 이 원칙에 대한 예외로 남아있는 소수의 몇몇 것들은
열정적 분노에 찬 비난을 지속적으로 받고 있다.
가령 흑인\wi{노예제}를 둘러싸고 여전히 진행 중인 열띤 논쟁에서
실로 다투어지고 있는 논점은
노예제가 지난 시대의 제도가 아니냐는 것,
그리고
근대적 도덕성에 부합하는
고용주와 노동자 간의 관계는
오직 계약에 의해 정해지는 관계뿐이지 않겠느냐는 것이다.
과거와 현재 간의 이러한 차이의 인정은
현대의 가장 유명한 사변적 논의의 핵심으로 우리를 끌고 들어간다.
확실히,
명령법\latin{imperative law}이
한때 장악하고 있던 영역의 많은 부분을
포기하지 않았다면,
그리고
최근까지 허용되지 않던 자유를 누리며
사람들이 스스로의 행위규칙을 정하도록 허용하지 않았다면,
도덕에 관한 연구 분야 중에
우리 시대에 비약적인 진보를 보인 유일한 분야인
\wi{정치경제학}\latin{political economy}은
생활 현실에 부응하지 못하고 실패할 것이다.
정치경제학의 훈련을 받은 사람들의 대다수가
실로 가지고 있는 선입견은
그들 학문이 의지하고 있는 일반적 진리가
보편적인 것이 될 권리가 있다고 보는 것이다.
그리하여 그들이 그것을 학문으로 적용할 때면,
그들의 노력은 대개 계약법의 영역을 확장하고
명령법의 영역은 축소하는 방향을 지향하거니와,
단지 계약의 이행을 강제하는 데 필요한 한에서만
명령법을 용인하는 것이다.
이러한 관념의 영향을 받은 사상가들이 불러일으킨 충격은
바야흐로 서구 세계에서 사뭇 강력하게 느껴지기 시작하고 있다.
입법은
발견과 발명과 축적된 부\hanja{富}의 사용에 관한 사람들의 행동을
따라잡을 능력이 없음을
거의 자백했다.
가장 덜 진보된 공동체의 법조차
점점 단지 껍데기에 불과한 것이 되어가고 있거니와,
그 아래에는
지속적으로 변화하는 계약적 규칙들의 연합이 존재하여,
여기에 법이 개입하는 경우는
약간의 근본원리들의 준수를 강제하거나
신의\hanja{信義} 위반을 벌하기 위해 소환되는 경우 외에는
거의 없는 실정이다.

\para{계약의 강제}
법현상을 고려해야 하는 것인 한
사회 탐구는 그 상황이 매우 낙후되어 있는지라,
사회의 진보에 관하여 널리 통용되는
통속적인 견해에서 저 진리가
발견되지 않더라도 놀라울 것이 없다.
이들 통속적인 견해는
우리의 신념보다는 우리의 편견에 더 잘 부응한다.
도덕의 진보를 인정하기를 꺼리는 강한 경향성은
계약의 기초가 되는 미덕을 의문시할 때
특히 더 강력해지는 듯하다.
우리들 중 다수는
신의와 성실이 옛날보다 오늘날에 더 널리 퍼져있음을,
또는 적어도 고대 세계의 충실성에 비견할 만한 풍속이 오늘날에도 존재함을,
인정하는 것에 대한
거의 본능적인 거부감을 가지고 있다.
때로 이러한 선입견은
예전에는 들려오지 않던
사기행각이 만연함을 보면서,
그리고 이들 범죄가 가져오는 커다란 혼란과 충격을 보면서
더욱 강화된다.
그러나 바로 이러한 사기행위의 범죄성으로부터 우리는,
그것을 범죄로 취급할 수 있기 위해서는
우선
그것이 위반하는 도덕적 의무가 더 크게 성장해야 한다는 것을
뚜렷이 알 수 있다.
다수가 믿고 따르는 신뢰가 있어야만
소수의 신뢰 위반도 생길 수 있는 법이므로,
아주 큰 부정직의 사례들이 발생한다면 이는
다수의 평균적 거래에서는 성실한 정직이 지배적이어서
예외적인 경우 범죄자들에게 기회가 주어졌다고
결론짓지 않을 수 없는 것이다.
계약법에서 형법으로 눈을 돌려
법에 반영된 도덕의 역사를 읽어야 한다면,
우리는 그것을 오독\hanja{誤讀}하지 않도록 주의해야 한다.
로마 고법\hanja{古法}에서 부정직한 행위로 취급된 형태는
\wi{절도}가 유일했다.
이 글을 쓰는 순간,
영국 형법의 최신 영역은
수탁자\hanjalatin{受託者}{trustee}의 사기행위를 처벌대상으로 삼으려는 시도이다.
이러한 대비에서 얻을 수 있는 올바른 추론은
원시 로마인들이 우리보다 더 높은 도덕성을 지녔다는 것이 아니다.
오히려 그들 시대에서 우리 시대로 시간이 흐르면서
사뭇 미개한 도덕성으로부터 대단히 세련된 도덕성으로
도덕성 관념이 진보했음을 알 수 있다.
소유권만 신성시하던 관념에서
단지 일방적 신뢰의 수여만으로 발생하는 권리까지도
형법으로 보호하는 관념으로 진보가 이루어진 것이다.

\para{사회계약}
이 점에 관하여 법학자들의 정연한 이론이라고 해서 대중들의 의견보다 더
진리에 가까운 것도 아니다.
로마 법률가들의 견해부터 말하자면,
그것은 도덕과 법의 진보에 관한 참된 역사와 일치하지 않았다.
계약 당사자들이 약속한 신의가 유일하게 중요한 요소인
계약의 한 유형을 그들은 \wi{만민법}상의\latin{juris gentium} 계약이라고
지칭했거니와,\footnote{%
  `낙성계약'(contractus consensu)을 말하고 있다.
  }
이 유형의 계약은 로마법에 나중에야 편입되어 들어간 것이 확실함에도
불구하고,
그들이 사용한 표현으로부터 어떤 확정적 의미를 추출해보면
그들은 그것을 로마법이 인정하는 다른 유형의 계약, 즉
법기술적 방식요건이 하나만 잘못되어도 오늘날의 착오나 사기만큼이나
계약의무의 성립에 치명적이었던 다른 유형의 계약들보다
더 오래된 것으로 보았음을 알 수 있다.
하지만 그들이 말하는 옛 것은 모호하고 희미한 것이었고
현재를 통해서만 이해될 수 있는 것이었다.
그리하여 ``만민법\latin{law of nations}상의 계약''을
자연상태의 사람들 사이의 계약으로 간주하게 된 것은
로마 법률가들의 언어가
그러한 사고양식에 진입하는 열쇠를 이미 상실해버린 시대의 언어로 된
이후의 일이었다.
\wi{루소}는 법률가들의 오류와 대중들의 오류를 모두 이어받았다.
관심을 끈 첫 작품이자
그를 한 분야의 선구자로 만든 의견이 사뭇 기탄없이 개진된 논문인
예술과 학문이 도덕에 끼친 영향을 논하는 논문에서,\footnote{%
  <<학문예술론>>(Discours sur les sciences et les arts)을 말한다. }
그는 고대 페르시아인들이 지녔던 정직함과 신의성실이야말로
문명의 등장과 더불어 점차 망각되어간 원시적 순수성의 특징이라고
누차 지적하고 있다.
그리고 나중에 그는
그의 모든 사변\hanja{思辨}의 토대를
원초적 \wi{사회계약}의 교리에서 발견하게 된다.
<<사회계약론>>은 우리가 논하고 있는 오류를 지닌 것 가운데 가장 체계적인 형태이다.
비록 정치적 열정에 의해 그 중요성이 고양되었지만
이 이론은 법률가들의 사변으로부터
모든 수액\hanja{樹液}을 채취한 이론이다.
처음 이 이론에 감화된 영국의 저명인사들은
주로 정치적 유용성의 면에서 그것의 가치를 높이 평가한 것이 사실이지만,
뒤에서 설명하겠으나
만약 정치가들이 법적인 용어로 논쟁을 해오지 않았더라면
영국인들은 결코 이 이론에 다가서지 못했을 것이다.
그리하여 이 이론을 주창한 영국인 학자들도
그들로부터 그것을 물려받은 프랑스인들에게 강한 호소력을 가졌던
저 사변적 깊이를 모르지 않았다.
그들의 저서는 이 이론이 정치적 현상뿐만 아니라
사회적 현상까지 모두 설명할 수 있다고 그들이 인식했음을 보여준다.
사람들이 준수하는 실정규칙 가운데
계약\latin{contract}으로 만들어진 것이 점점 많아지고
명령법\latin{imperative law}으로 만들어진 것이 점점 줄어들고 있는 현상,
그들 시대에도 이미 현저하게 나타나고 있던 이 현상을
그들은 관찰을 통해 알고 있었다.
그러나 법학의 저 두 구성부분의 역사적 관계에 대해서는
그들은 무지했거나 주의를 게을리했다.
그리하여
그들은 모든 법은 계약에서 기원한다는 이론을 창안했거니와, 이는
모든 법의 기원을 단일한 원천에 둠으로써 그들의 사변적 취향을
만족시키기 위한 것이었으며,
또한
명령법은 신에게서 기원한다는 교리\footnote{%
  필머(Robert Filmer)로 대표되는 왕권신수설을 말하는 듯하다.
}를 피하려는 견해에서 나온 것이었다.
한 단계 더 사고가 진보한다면,
그들은 기꺼이
그들의 이론을
어떤 기발한 가설이나 편리한 언어 공식\hanja{公式}에 불과했다고
치부했을 것이다.
그러나 당시는 법적 미신\hanja{迷信}이 지배하던 시대였다.
자연상태에 관한 논의는 그것이 역설적이 아니라고 여겨지는 한 계속되었고,
따라서
\wi{사회계약}을 역사적 사실로 주장함으로써
법의 계약적 기원이라는 거짓 현실과 확신을
쉽게 심어줄 수 있었던 것으로 보인다.

\para{몽테스키외의 혈거인}
우리 세대는 이러한 잘못된 법이론을 떨쳐버렸다.
그것은 부분적으로는 저 이론이 속했던 지적 상태를 벗어났기 때문이고,
또 부분적으로는 그러한 주제를 이론화하는 일을 거의 그만두었기 때문이다.
오늘날 적극적으로 연구를 수행하는 학자들이 선호하는 작업은,
그리고 사회의 기원에 관한 우리 선조들의 사변에 대해 답할 수 있는 작업은,
사회의 존재를 있는 그대로, 사회의 운동을 운동하는 그대로 분석하는 것이다.
그러나 역사의 도움을 받지 못하면,
이런 분석은 단순한 호기심의 충족으로 전락하기 일쑤이거니와,
특히
연구자가 익숙해있는 사회상태와는 자못 다른 사회상태를 이해하는 데
장애물로 작용할 공산이 크다.
우리 시대의 도덕성을 가지고 다른 시대의 사람들을 판단하는 잘못은
현대사회라는 기계장치의 바퀴 하나, 볼트 하나까지
원초적 사회에 그 대응물이 있을 것이라고 가정하는 잘못에 견줄 만하다.
이러한 인상\hanja{印象}은
근대적 양식으로 쓰여진 역사학 저술들에서
사뭇 다양하게 가지를 치고 있으며
사뭇 미묘하게 모습을 숨기고 있다.
그러나
나는
\wi{몽테스키외}의
<<페르시아인의 편지>>에 삽입된
혈거인\hanjalatin{穴居人}{Troglodytes}의 우화\footnote{%
  \latinmarks
  Montesquieu, \textit{Persian Letters}, 11--14.
}에 대해
흔히 주어지는 찬사에서
법학 영역에서의 그러한 인상의 흔적을 발견한다.
혈거인들은 계약을 항상 위반하는 사람들이었으며, 그래서 완전히 멸망해버렸다.
만약 이 이야기에 저자가 의도한 도덕이 담겨있고,
그것이
금세기와 지난 세기를 위협해온 반사회적 이단\hanja{異端}을
폭로하기 위해 사용되었다면,
그것은 전혀 나무랄 데가 없는 것이다.
그러나
성숙한 문명이 보여주는 것과 같은 정도로
약속과 합의에 신성함을 부여하지 않는 한
어떤 사회도 결속을 유지할 수 없다는
주장이
저 이야기로부터
추론되어 나온다면,
그것은 법사\hanja{法史}의 어떤 건전한 이해와도 상반되는 치명적인 오류가
될 것이다.
사실,
혈거인들은 계약적 의무를 아주 조금 준수함으로써
번성할 수 있었고 강력한 국가를 건설할 수 있었던 것이다.
원시사회의 헌정\hanja{憲政}에 관하여
무엇보다 먼저 이해해야 할 것은
개인은 자신을 위해 권리나 의무를 거의 혹은 전혀 만들지 못한다는 점이다.
개인이 지켜야 할 규칙은 우선은 출생에 따르는 지위에서 나오고,
다음으로는 그가 속하는 가\hanja{家}의 수장이 그에게 부과하는
명령에서 나온다.
이러한 체제는 계약을 위한 여지를 거의 남겨두지 않는다.
동일한 가\hanja{家}의 구성원들은
\paren{증거로부터 해석하건대}
서로 간에 전혀 계약을 체결할 수 없으며,
가\hanja{家}는 그 구성원이 가를 구속시키려고 맺은 계약을
무시할 수 있는 권리를 가진다.
물론 가와 가 사이, 가부장과 가부장 사이의 계약은 있을 수 있지만,
그 거래는 물건의 양도와 마찬가지 성격을 지니므로
수많은 방식요건들이 부과되어
실행에 있어
사소한 실수라도 계약의무의 성립에 치명적인 것이 된다.
타인의 말을 신뢰하는 것에서 생겨나는 적극적 의무는
진보적 문명이 아주 나중에야 성취하게 되는 것이다.

\para{초기 로마의 계약들}
어떤 고대법도, 다른 어떤 전거도,
계약의 개념을 전혀 알지 못하는 사회가 있음을 보여주지 못한다.
그러나 이 개념이 처음 나타났을 때
그것은 분명 아주 원시적이었을 것이다.
어떤 믿을 만한 원시 기록에서도
약속을 유효하게 만드는 인간의 정신이 아직 미숙했음을,
그리고
노골적인 배신행위가 비난 없이, 때로는 칭송의 대상으로, 언급되고 있음을
읽을  수 있다.
가령 \wi{호메로스}의 문헌에서
오뒷세우스의 기망적인 교활함은
네스토르의 현려\hanja{賢慮}, 헥토르의 지조,
아킬레우스의 용기와 동급의 미덕으로 나타난다.
고대법은 계약의 원시적 형태가 그것의 성숙한 형태로부터
멀리 떨어져있었음을 훨씬 더 분명히 보여준다.
처음에는 단순히 약속의 이행을 강제하기 위해
법이 개입하지는 않았던 것으로 보인다.
법이 제재로써 강제하는 것은 단순한 약속이 아니라,
엄숙한 의례\hanja{儀禮}를 수반하는 약속이었다.
요식성\hanja{要式性}은 약속과 마찬가지로 중요했을 뿐만 아니라,
어쩌면 약속 이상으로 훨씬 더 중요했다.
성숙한 법학이 구두\hanja{口頭}의 승인\hanja{承認}이 행해진 상황에 적용하는
섬세한 분석이
고대법에서는
그것의 실행에 수반되는 말과 몸짓에 전가된 듯하다.
사소한 방식\latin{form} 하나라도 빠뜨리거나 잘못 실행되면 어떠한 서약도
강제될 수 없었다.
한편, 방식이 정확히 준수되었음이 입증된다면,
사기나 강박으로 약속했다는 항변은 아무 소용이 없었다.
법제사에서는
이러한 고대적 관념으로부터 우리에게 친숙한 계약 관념으로의 이행이
명백히 드러난다.
처음에는 의례의 한 두 단계가 건너뛸 수 있는 것이 되고,
그후 일정 조건 하에서 다른 것들도 단순화되거나 생략이 허용되며,
마침내 몇몇 특수한 계약들이 다른 것들로부터 분리되어
방식의 구애를 받지 않고 체결할 수 있게 되거니와,
이들 특수한 계약은
사회적 거래의 활동성과 에너지가
이에
의존하는 계약인 것이다.
서서히, 그러나 사뭇 명백하게,
법기술적 요소들로부터 심적\hanja{心的}인 요소가 분리되어 나오고,
차츰 법학자들의 관심을 한몸에 받는 유일한 요소가 된다.
외부적 행위를 통해 표현되는
이러한 심적 요소를 로마인들은
`\wi{약정}'\hanjalatin{約定}{pact; convention}이라 불렀다.
그리고 약정이 계약의 핵심으로 인정되자,
곧이어
방식과 의례의 껍질을 부수어버리는 것이
진보적 법학의 경향성이 된다.
그후 방식들은 진정성을 보증하는 한에서만,
그리고 주의와 숙고를 담보하는 한에서만
보존될 뿐인 것으로 된다.
이로써 계약의 관념은 완전한 발달을 보이게 되거니와,
로마법의 용어를 사용하자면,
계약은 약정에 흡수되어버리는 것이다.

\para{양도와 계약}
로마법이 보여주는 이러한 변화 과정의 역사는 자못 시사적이다.
로마법의 여명기에
계약에 해당되는 말로 쓰인 용어는
고대 라틴어를 연구하는 학자들에게는 무척 익숙한 용어이다.
그것은 바로 넥숨\latin{nexum}, 즉 \wi{구속행위}\hanja{拘束行爲}로서,
이 계약의 당사자들은 `피구속자'\hanja{被拘束者}들\latin{nexi}이라 불렸다.
이 표현들은 그 근저에 놓인 은유의 이례적인 지속성으로 인해
특히 주목할 필요가 있다.
계약관계에 놓인 사람들이 강력한 \hemph{속박}\latin{bond}
또는 \hemph{사슬}\latin{chain}로 연결되어 있다는
관념은 마지막까지 계속해서 로마계약법에 영향을 주었고,
거기서 흘러나와 근대적 관념에도 섞여들어갔다.
그렇다면 이 구속행위 혹은 속박이란 무엇을 의미하는 것이었을까?
라틴어에 관한 고문헌을 통해 우리에게 전해진 바에 따르면
구속행위는 ``구리와 저울로써 행해지는
모든 것''\latin{omne quod geritur per aes et libram}이라고
정의되어 있거니와,\footnote{%
  \latinmarks
  Varro, \textit{De Lingua Latina}, 7.105.
  }
이 단어들은 상당히 큰 혼란을 불러일으켰다.
구리와 저울은
\wi{악취행위}에 수반되는 것들로 잘 알려져있다.
악취행위는
앞 장에서 서술한 고법\hanja{古法}상의 엄숙한 행위로서,
로마 물권법에서 높은 등급의 물건의 소유권이
한 사람에게서 다른 사람에게 양도되는 방식이었다.
이렇게 악취행위는 \hemph{양도}\latin{conveyance}의 방식이기에
어려운 문제가 부상하게 된다.
위에 인용한 저 정의는
계약과 양도를 혼동하고 있거니와,
법철학에서는 이 두 가지가 단지 구분될 뿐만 아니라
사실상 서로 대립하는 것이기 때문이다.
성숙한 법학의 분석가들은
물\hanja{物}에 대한 직접적 권리\latin{jus in re},
대세적\hanja{對世的} 권리\latin{right \textit{in rem}},
``온 세상에 대하여 주장할 수 있는'' 권리,
즉 물권\hanjalatin{物權}{proprietary right}과
물\hanja{物}에 대한 간접적 권리\latin{jus ad rem},
대인적\hanja{對人的} 권리\latin{right \textit{in personam}},
``특정인이나 특정집단에 대하여 주장할 수 있는'' 권리,
즉 채권\hanjalatin{債權}{obligation}을
날카롭게 구별한다.
그런데 양도는 물권을 이전하고, 계약은 채권을 창설한다.
어떻게 이 두 가지가 동일한 이름 아래, 동일한 일반개념 아래
포섭될 수 있다는 말인가?
다른 유사한 난제들과 마찬가지로 이 문제도
미발달된 사회의 정신적 상태에
진보된 지적 단계에 특별히 속하는 능력을,
현실에서는 혼재되어 있는 것을 사변적 관념들로 구별하는 능력을,
끼워맞추려는
오류 탓에 발생한 것이다.
여기서
우리는
양도와 계약이 현실적으로 혼재되어 있는 사회상태에 관하여
오인하지 말아야 한다는 시사를 받는다.
계약과 양도에 관하여 독자적인 실무관행이 채택되기 전까지는
저 개념들 간의 차이는 인식될 수 없었던 것이다.

\para{구속행위}
로마 고법\hanja{古法}에 관한 우리의 지식으로부터
법의 여명기에 법적 개념과 법적 용어가 어떻게 변해갔는지
그 변화의 양상에 대한 약간의 관념을 얻을 수 있을 것이다.
이 변화는 일반적인 것에서 특수적인 것으로의 변화라고 할 수 있다.
다시 말해 고법상의 개념과 고법상의 용어는 점진적 특수화의 과정을
겪었던 것이다.
고법상의 개념은 하나가 아니라 다수의 근대적 개념에 대응된다.
고법상의 법기술적 표현은 근대법이 여러 개의 이름으로 나누어놓은
다수의 것들을 지칭한다.
하지만 법사\hanja{法史}의 다음 단계에 이르면,
하위 개념들이 점차 서로 분리되어,
예전의 일반적 이름은 특수적 명칭들로 바뀌어가는 것이다.
그렇다고 옛 개념이 사라지는 것은 아니고,
단지 원래 포섭하던 관념의 일부만 포섭하게 된다.
그리하여 예전의 법기술적 이름은 여전히 존재하지만,
한때 수행했던 기능들 중에 하나만 수행할 뿐이다.
이러한 현상의 예로는 여러 가지를 들 수 있겠다.
가령 여러 종류의 \wi{가부장권}은 한때
그 성격이 모두 동일했고,
따라서 하나의 이름으로 불렸을 것이 틀림없다.
존속친\hanja{尊屬親}에 의해 행사되던 가부장권은
가족에 대해 행사되든 물질적 재산에 대해 행사되든---양떼나 소떼, 노예,
자식, 아내를 불문하고---모두 동일했다.
그것의 옛 로마식 명칭에 대해 완전히 확신할 수는 없지만,
가부장\hemph{권}\latin{power}을 지칭하는 여러 명칭들에
\hemph{마누스}\latin{manus}라는 단어가 들어가 있는 것으로 볼 때,
옛 일반적 명칭은 `마누스'였을 것으로 믿을 만한 근거는 충분해보인다.\footnote{%
  `마누스'는 흔히 `수권'(手權)으로 번역되나 여기서는 본문의 의미맥락상
  원어를 살렸다.
  이하 관련 단어들도 마찬가지다.
}
그러나 로마법이 좀 더 진보하면서,
저 이름도 저 관념도 특수화되었다.
\wi{가부장권}은
그것이 행사되는 대상에 따라
단어에서도 개념에서도 분화되어갔다.
물건이나 노예에 대해 행사될 때는
`도미니움'\latin{dominium},
자식들에 대해서는 `포테스타스'\latin{potestas},
존속친에 의해 다른 사람의 권력에 제공된 자유인에 대해서는
`만키피움'\latin{mancipium}이 되었고,
아내에 대해서는 여전히 `마누스'로 남았다.\footnote{%
  \latin{Gai.\,1.49} 참조. }
여기서 알 수 있듯이,
원래의 단어가 전혀 쓰이지 않게 된 것이 아니라,
과거에 지칭하던 권력행사 중 특수한 한 가지 권력행사에 국한하게 된 것이다.
이 사례를 모범삼아 계약과 양도 간의 역사적 결합관계의 성질에 대해서도
이해를 도모할 수 있을 것이다.
처음에는 모든 엄숙한 거래에 오직 하나의 엄숙한 의례\hanja{儀禮}만
존재했을 것이니,
로마에서는 그것의 명칭이 `\wi{구속행위}'\latin{nexum}였던 것으로 보인다.
물건의 양도에 사용되던 바로 그 방식이
계약의 체결에도 사용되었던 것으로 보인다.
그러나 양도 관념으로부터 계약 관념이 분리되어 나오는 데는
그다지 긴 시간이 필요치 않았다.
그리하여 이중\hanja{二重}의 변화가 일어났다.
``구리와 저울에 의한'' 거래가
물건의 이전을 의도하는 경우에는
`\wi{악취행위}'\latin{mancipation}라는 새롭고 특수한 이름으로 불리게 된다.
옛 이름인 `구속행위'는 여전히 동일한 의례절차를 지칭하지만,
이제
오직 계약을 엄숙하게 체결하는 특수한 목적에만 쓰이게 된다.

\para{변화}
두 세 가지 법개념이 고대에는 하나로 혼재되어있었다고 해서,
거기에 포함된 관념 중 하나가 다른 것들보다 더 오래된 것이 아니라는
말은 아니다.
혹은 그 하나가, 다른 것들이 형성된 후,
이것들보다 크게 우세하거나 우선하지 않는다는 말도 아니다.
하나의 법개념이 오래 계속해서 여러 법개념들을 포괄할 수 있는 이유는,
그리고 하나의 법기술적 용어가 여러 용어들을 대신할 수 있는 이유는,
원시사회의 법에 실무관행의 변화가 일어나더라도
오랫동안 사람들은 그것에 주목하거나 이름붙일
필요를 느끼지 못하기 때문일 것이 분명하다.
비록, 전술했듯이,
처음에는 가부장권에 행사대상에 따른 구별이 없었다 할지라도,
자식들에 대한 권력이 옛 \wi{가부장권}의 근본이었다고
나는 믿어 의심치 않는다.
또한
`\wi{구속행위}'라는 말의 최초의 사용은,
그리고 그것을 사용했던 사람들이 주로 염두에 둔 것은,
물건의 양도에 엄숙한 형식을 부여하려는 것이었음을
나는 의심치 않는다.
구속행위가 원래의 기능으로부터 아주 조금 벗어나기 시작했을 때
그것은 바로 계약의 체결에 사용되었을 것이나,
아주 조금의 변화였기에 그 변화는 오랫동안 인정되거나 감지되지 못했다.
새로운 것을 원한다는 것을 사람들이 자각하지 못했기 때문에
옛 이름은 그대로 남았다.
아무도 수고스럽게 새로운 것을 검토해볼 필요를 느끼지 못했기 때문에
옛 관념은 그대로 남았다.
우리는 이러한 과정의 사례를 유언법의 역사에서 명료하게 살펴본 바 있다.
유언은 처음에는 단순히 재산의 양도였다.
점차 이러한 특수한 양도와 다른 모든 양도 간에 커다란 실무상의 차이가
나타나고 나서야 비로소
이들이
서로 다른 것으로 간주되기 시작했고,
그러고도 수 세기가 흐른 뒤에야
법의 개량에 나선 사람들이
허울뿐인 악취행위에 붙어있던 복잡한 절차를
청소했고 마침내
유언에 있어
유언자의 명시적 의사 외에는 다른 어떤 것도 중요하지 않다는
합의가 이루어졌던 것이다.
유언법의 초기 역사만큼의
절대적 확신을 가지고
계약법의 초기 역사를 추적할 수가 없다는 것은 유감스런 일이지만,
구속행위가 새로운 사용에 놓여짐으로써
계약이 처음 등장했고
이어서
이 실험의 중차대한 실무적 결과로써
계약이 독자적 거래형태로 승인되었다는 것을 암시하는
힌트마저 얻을 수 없는 것은 아니다.
다음과 같은 과정을 대체로 따랐을 것이라는 추측이,
그러나 그리 억지스럽지만은 않은 추측이 가능하다.
구속행위의 통상적인 방식에 의해
일정한 대금을 받고 매매가 행해진다고 가정하자.
매도인은 처분하고자 하는 목적물---가령 노예 한 명---을 가지고 온다.
매수인은 매매대금을 해당하는 구리 덩어리를 가지고 참석한다.
필수적 보조인인 \wi{저울소지인}\latin{libripens}도 저울을 들고 나와있다.
노예는 정해진 요식절차에 따라 매수인에게 건네진다.
저울소지인은 구리 조각을 저울에 달고는 매도인에게 넘겨준다.
이러한 거래행위가 지속되는 한 그것은 `\wi{구속행위}'이고,
당사자들은 `피구속자'들\latin{nexi}이다.
그러나 그것이 완료되자마자,
구속행위는 끝나고,
매도인과 매수인도 그들의 일시적 관계에서 유래하는 이름으로 불리기를 그친다.
이제 여기서
거래의 역사를 한 걸음 진척시켜보자.
노예는 양도되었으나,
대금은 지불되지 않았다고 가정해보자.
\hemph{이} 경우,
매도인에 관한 한 구속행위는 종료된다.
이미 자기 물건을 넘겨주었으므로 그는 더 이상 `피구속자'\latin{nexus}가 아니다.
그러나 매수인에 관해서는 구속행위가 계속된다.
매수인 쪽에서는 거래가 끝나지 않았고 그는 여전히 `피구속자'로 남는다.
따라서 동일한 용어가 물권을 이전하는 양도를 기술\hanja{記述}함과 동시에
미지불된 매매대금에 관한 채무자의 채무도 기술하고 있음을 알 수 있다.
다시 한 걸음 더 나아가,
완전히 형식적인 거래, 즉 아무 것도 건네지지 \hemph{않고}
아무 것도 지불되지 \hemph{않는} 절차를 상상해보자.
우리는 사뭇 발달된 상거래 행위의 하나, 바로
\hemph{미이행}\hanjalatin{未履行}{executory} \hemph{매매계약}에
도달하게 되는 것이다.

\para{양도와 계약}
대중적 견해에서나 전문가적 견해에서나
\hemph{계약}이라는 것이
오랫동안 \hemph{미완의 양도}\latin{incomplete conveyance}라고
간주된 것이 사실이라면,
이 사실은 여러 모로 의미심장하다.
자연상태의 인류에 관한 지난 세기의 사변적 이론을
``원시사회에서는 물권은 아무 것도 아니었고 채권이 모든 것이었다''는
교리로 요약하는 것이 그다지 부당하지는 않을 것이다.
그러나 이제 우리는
저 명제를 거꾸로 뒤집으면
그것이 오히려 진실에 가깝다는 것을 알게 되었다.
다른 한편,
역사적으로 보면,
양도와 계약의 원시적 결합은
학자들과 법률가들에게 특별히 수수께끼로 여겨지곤 했던 어떤 것을
설명할 수 있을 것이다.
초기 고대법은 어디서나 \hemph{채무자들}을 무척 가혹하게 처우했으며,
\hemph{채권자들}에게는 막강한 권한을 주었다는 수수께끼 말이다.
구속행위가 채무자에게는 인위적으로 긴 시간 동안 지속되었음을 알고 나면,
대중들과 법이 바라보는 그가 지위가 어떤 것이었을지
더 잘 이해할 수 있는 것이다.
그의 채무상태는 틀림없이 비정상적인 것으로 여겨졌을 것이고,
지불의 해태\hanja{懈怠}는 일반적으로
간교한 책략이자 엄격법의 왜곡으로 비춰졌을 것이다.
반대로,
거래에서의 자신의 의무를 성실하게 완수한 사람은
특별한 호의로써 대우받았을 것이니,
엄격법에 따르면 연장되거나 지체되어서는 안 될
어떤 절차를 강제로 완성시킬 권한을 그에게 주는 것보다
더 당연한 일은 없어보인다.

\para{로마법의 합의 분석}
따라서 \wi{구속행위}는 원래 재산의 양도를 의미했지만,
부지불식간에 계약도 의미하게 되었고,
구속행위 개념과 계약 관념 간의 결합이 오랫동안 지속되었기에
마침내
\wi{악취행위}\latin{mancipatio}라는 특별한 용어가
진정한 구속행위, 즉 실제로 재산이 양도되는 거래를 지칭하는 데
사용되게 되었다.
그리하여 계약은 이제 양도와 분리되었고
이로써 계약법 역사의 첫 단계가 마무리되었으나,
계약당사자의 약속이 이를 둘러싼 요식성보다 더 신성\hanja{神聖}하게 평가되는
단계에 이르기까지는 아직 한참 멀리 떨어져있었다.
그 사이 기간 동안 진행된 변화의 성격을 알아보려면,
지금 우리가 다루고 있는 주제의 범위를 살짝 넘어갈 필요가 있거니와,
바로 로마 법학자들이 합의\latin{agreement}를 어떻게 분석했는지
살펴보는 것이다.
그들의 재능이 만들어낸 가장 아름다운 기념비인
이 분석에 관하여, 나는
그것이 \wi{채권채무관계}\latin{obligation}를 \wi{약정}\latin{pact}으로부터
이론적으로 분리하는 데 기초하고 있다는 것 이상을 말할
필요를 느끼지 않는다.
\wi{벤담}과 \wi{오스틴} 씨는 이렇게 주장했다.
``계약의 주요 성질은 다음 두 가지이다:
첫째,
하기로 약속하는 작위를 하겠다는,
또는
하지 않기로 약속하는 부작위를 하지 않겠다는
\wi{약속자}\hanja{約束者}의
\hemph{의사}의 표시.
둘째,
이 주어진 약속을 약속자가 이행할 것이라는 데 대한
\wi{수약자}\hanja{受約者}의
\hemph{기대}의 표시.''
이것은
로마 법률가들의 법리와 거의 동일하지만,
그러나 로마 법률가들은
이러한 ``표시''의 결과를 `계약'이 아니라
`\wi{약정}'으로 보았다.
약정은 개인들 간의 합의로 맺어지는 약속의 최종 산물이지만,
그렇다고 그것이 바로 계약인 것은 아니다.
약정이 계약이 되는가 여부는
법이 그것에 채권채무관계를 덧붙이느냐 여부에 달려있다.
계약이란 `약정' \hemph{더하기} `채권채무관계'인 것이다.
약정이 채권채무관계의 옷을 입고 있지 않는 한,
\index{나약정}%
그것은 \hemph{나}약정\hanja{裸約定}, 즉 \hemph{벌거벗은} 약정이라 불렸다.

\para{로마법의 채권채무관계}
\wi{채권채무관계}\latin{obligation}란 무엇인가?
로마 법률가들의 정의에 따르면
``누군가에게 급부\hanja{給付}를 할 것이 필연적으로 강제되는
법의 사슬''\latin{juris vinculum, quo
necessitate adstringimur alicujus solvendae rei}이다.\footnote{%
  \latin{Inst.\,3.13.pr.} }
이 정의는
채권채무관계를 \wi{구속행위}와 연결짓거니와,
이들의 배경에 놓인 공통의 은유를 통해서 그러하다.
또한 이 정의는
특정 개념의 계보를 사뭇 명료하게 보여준다.
채권채무관계는 ``속박'' 또는 ``사슬''이거니와,
이로써
사람들 혹은 사람들의 집단들은
어떤 자발적 행위의 결과로서
법에 의해 하나로 결속되는 것이다.
채권채무관계를 이끌어내는 행위들은 주로
합의와 위법행위, 즉
계약과 불법행위라는 표제 아래 분류되지만,
정확하게 분류되기 힘든 여러 다른 행위들도 유사한 결과를 낳는다.\footnote{%
  `준(準)계약'과 `준(準)불법행위'를 말한다.
  준계약은 오늘날의 부당이득, 사무관리 등의 법리에 해당한다.
  \latin{Inst.\,3.27.}
  또한 절도(furtum), 강도(rapina), 재산침해(아퀼리우스법\,Lex Aquilia),
  인격침해(iniuria) 등 통상의 불법행위에 해당하지 않지만,
  재판을 자기 것으로 만든 심판인,
  무언가를 집 밖으로 내던지고 쏟아부은 경우,
  무언가를 위험하게 세워두고 걸어놓은 경우,
  선박주인·여관주인·마구간주인이 피용인의 불법행위에 책임을 지는 인수(引受)
  등은 준불법행위로 취급되었다. \latin{Inst.\,4.5.} }
하지만 유의할 점은
어떤 도덕적 필요성도 \wi{약정}을 바로 채권채무관계로 만들지는 못한다는 것이다.
약정에 채권채무관계의 힘을 완전히 부여하는 것은 법이다.
이 점 더욱 유의할 필요가 있거니와,
도덕적 또는 형이상학적 이론을 지지하는 근대 대륙법학자들에 의해
때로 이와 다른 법리가 주창되어왔기 때문이다.
`법의 사슬'\latin{vinculum juris}이라는 이미지는
로마 계약법과 불법행위법의 모든 부분을 물들이고 있고 지배하고 있다.
법은 당사자들을 속박하거니와,
이 \hemph{사슬}은 `변제'\hanjalatin{辨濟}{solutio}라고 불리는 과정을
통해서만 풀 수 있다.
`변제'라는 표현도 은유적이거니와,
``지불''\latin{payment}이라는 일상용어는 가끔씩 그리고 우연히
여기에 들어맞을 뿐이다.
이들 은유적 이미지의 일관성은
이것이 없었다면 혼란을 초래했을
로마법 용어의 특별한 의미를 이해할 수 있게 해준다.
즉, ``\wi{채권채무관계}''\latin{obligation}는 의무뿐만 아니라
권리도 의미하는 것이다.
이를테면 빌린 돈을 지불할 의무뿐만 아니라
빌려준 돈을 지불받을 권리도 의미한다.
실로 로마인들의 눈 앞에는 ``법의 사슬''의 전체 그림이
펼쳐져있었으며,
사슬의 한쪽 끝을 다른 쪽 끝보다 더 많이 혹은 더 적게
바라보지 않았다.

\para{약정과 계약}
발달된 로마법에서는
\wi{약정}이 체결되자마자 거의 모든 경우
즉시
채권채무관계의 관\hanja{冠}이 씌워지고, 따라서 계약이 된다.
이것은 분명 계약법이 지향하는 결과이다.
그러나 우리의 탐구의 목적을 위해서는
그 중간 단계, 즉 채권채무관계가 되려면 완전한 합의 이상의 어떤 것이
요구되는 단계에 주목해야 한다.
이 단계는
네 종류의 계약---\wi{언어계약}, \wi{문서계약}, \wi{요물계약}, \wi{낙성계약}---을 구분한,
저 유명한 로마법상의 분류가 사용되던 시기와 일치한다.
이 시기 동안 법에 의해 강제된 약속은 저 네 가지에 국한되었다.
채권채무관계를 약정과 분리하는 이론을 알고 있다면
저 네 가지 항목의 의미는
쉽게 이해될 수 있다.
사실,
계약들의 각 항목은
계약당사자들의 단순한 합의 이외에 어떤 요식성이 필요한가에 따라
이름붙여진 것이다.
언어\latin{verbal}계약에서는 약정이 체결되는 순간
일정한 방식의 말들이 발설되어야 ``법의 사슬''이 부여된다.
문서\latin{literal}계약에서는
원장\hanja{元帳}, 즉 회계장부에 기입되어야
\wi{약정}에 \wi{채권채무관계}의 옷이 입혀진다.
요물\latin{real}계약의 경우
예비적 약속의 목적물인
물건의 \wi{인도}\hanja{引渡}가 있어야 동일한 결과가 뒤따른다.
요컨대,
이 모든 경우
계약당사자들 간에는 의사합치가 있어야 하지만,
거기에만 그친다면 그들은 서로에게 \hemph{채권채무}를 갖지 못하고,
따라서 이행을 강제할 수도,
신의\hanja{信義} 위반을 이유로 배상을 청구할 수도 없다.
그러나 그들이 어떤 정해진 요식성을 충족시킨다면,
계약은 바로 체결되고,
그 계약의 이름은 그들이 채택한 특정한 방식에 따라 붙여지는 것이다.
이러한 관행에 대한 예외는 조금 있다 살펴보겠다.

\para{언어계약}
나는 네 가지 계약들을 역사적 순서에 따라 열거했으나,
로마의 \wi{법학제요} 저자들이 이 순서를 반드시 따른 것은 아니다.
\wi{언어계약}이 네 가지 중에 가장 오래된 것임에는 의심의 여지가 없다.
그것은 원시적 \wi{구속행위}의 후손으로 알려진 것 중에 가장 먼저 나타났다.\footnote{%
  <<고대법>>에 대한 폴록의 주석에 따르면,
  문답계약의 기원은 구속행위(nexum)가 아니라
  서약(oath)과 같은 초기의 종교적인 방식에 의한 채권채무관계에서 찾는 것이
  오늘날
  일반적이라 한다.
  }
언어계약에 속하는 몇몇 종\hanja{種}이 옛날에는 사용되었으나,
가장 중요한 것이자 우리의 전거들이 다루었던 유일한 것은
질문과 답변으로 이루어진 `\wi{문답계약}'\latin{stipulation}이다.
\wi{요약자}가 질문을 하고 \wi{낙약자}가 답변을 한다.
이러한 질문과 답변이야말로, 전술했듯이,
원시적 관념이
당사자들 간의 단순한 합의를 넘어 추가적으로 요청하는 요소인 것이다.
이들은 채권채무관계가 부여되기 위한 매개체이다.
옛 구속행위는
보다 성숙한 법학에게 무엇보다
계약당사자들을 결속시키는 사슬의 개념을 물려주었으니,
이것이 이제 \wi{채권채무관계}가 되었다.
그것은 또한 약속에 수반하면서 약속을 성별\hanja{聖別}하는
의례행위의 개념도 물려주었으니,
이 의례행위가 이제 질문과 답변으로 변형된 것이다.
초기 구속행위의 특징이었던 엄숙한 양도행위가
어떻게 단순한 질문과 답변으로 전환되었을까 하는 것은
이와 유사한 로마 유언법의 역사가 우리에게 가르쳐준 바가 없었다면
더욱 미스테리로 남았을 것이다.
유언법의 역사를 돌아보면,
실질적 관심 대상에 직접 관련되는 절차 부분\footnote{%
  `양도'와 대비되는 `언명'(nuncupatio)을 말하는 듯하다.
}으로부터
어떻게
형식적 양도가
처음 분리되었는지를,
그리고 어떻게 그후 이것을 완전히 생략하게 되었는지를
이해할 수 있다.
그렇다면
\wi{문답계약}의 질문과 답변은 단순화된 형태의 \wi{구속행위}였을 것이 분명하고,
따라서
그것은 오랫동안 법기술적 형태의 성질을 가졌을 것이라고 쉽게 추정할 수 있다.
옛 로마 법률가들이 문답계약을 옹호했던 이유를
합의에 임한 당사자들에게 숙고하고 성찰할 기회를 제공하는
유용성때문이라고 본다면 이는 잘못일 것이다.
물론 이런 유\hanja{類}의 가치가 있었고 점점 중요하게
인식되었다는 것을 부인할 수는 없다.
그러나, 우리의 전거들에 나타난 증거에 비추어볼 때,
계약에 관련된 그것의 기능은 처음에는 형식적이고 의례적인 것이었다.
문답계약을 구성할 수 있는 오래된 질문과 답변은
특정한 경우에 적합한 법기술적 용어들을 사용한
질문과 답변에만 국한되었고,
아무 질문이나 답변이라고 해서 다 되는 것이 아니었다.

\para{문답계약}
그러나,
비록 \wi{문답계약}이 유용한 안전장치로 인식되기 이전에
엄숙한 형식으로 인식되었다고 이해하는 것이
계약법의 역사를 올바르게 평가하는 데 필수적이라 할지라도,
다른 한편
그것의 현실적 유용성에 눈을 감아버리는 것도
잘못된 일일 것이다.
언어계약은, 비록 고법\hanja{古法}상에서 누리던 중요성을
상당 부분 상실해갔지만, 로마법의 마지막 시기까지 계속 살아남았다.
당연한 말이겠지만,
로마법의 어떤 제도도
어떤 현실적 유용성이 없었다면
그렇게 오래 유지될 수 없었을 것이다.
놀랍게도
어떤 영국 학자의 말에 따르면,
로마인들은
초창기부터도
숙고없이 서둘러 계약을 맺는 것에 대한 방비가 거의 없어도
괘념치 않았다고 한다.
그러나 문답계약을 면밀히 조사해보면,
그리고 서면증거를 만들기 쉽지 않았던 당시의 사회상태를 감안하면,
\wi{문답계약}의 질문과 답변은,
만약 그것이 실제 기여한 목적을 위해 의도적으로 고안되었다면,
그야말로 천재적인 방책이었다고 평가해도 좋다고 생각한다.
\hemph{\wi{요약자}}\latin{promisee}가
계약의 모든 조항을 질문의 형태로 만들어 질문하면,
\hemph{\wi{낙약자}}\latin{promisor}가 답변을 한다.
``당신은 이러이러한 노예를 이러이러한 장소에서 이러이러한 날짜에
나에게 인도할 것을 약속하는가?''
``약속하노라.''
잠깐만 생각해보면,
이렇게 질문 형태로 구성되는 \wi{채권채무관계}는
당사자들의 자연스런 입장을 거꾸로 뒤집고,
대화의 통상적인 흐름을 깨뜨리는 효과를 가져와,
위험한 약속에 빠지지 않도록 주의를 환기시키는 기능을 함을 알 수 있다.
우리 영국인들이 행하는 구두의 약속은
오직 약속자\latin{promisor}의 말로 구성되는 것이 일반적이다.\footnote{%
  원어로는 똑같이 `promisor'이지만,
  로마법의 맥락에서는 `낙약자'(promissor)로,
  영국법의 맥락에서는 `약속자'로 번역하고 있음을 유의할 것.
  마찬가지로 `promisee'는 로마법의 맥락에서는 `요약자'(stipulator)로,
  영국법의 맥락에서는 `수약자'로 번역된다.
  사실 `요약자'니 `낙약자'니 하는 우리의 법률용어는
  바로 로마법의 문답계약에서 유래하는 말이다.
  }
옛 로마법에서는 또 하나의 단계가 반드시 요구된다.
합의가 이루어지고 나면
\wi{요약자}가 엄숙한 질문의 형태로 이 합의의 모든 조항들을 요약해야 하는 것이다.
재판에서 증거로 제출되는 것은,
구속력 없는 약속 자체가 \hemph{아니라},
바로 이 질문과 그에 대한 답변인 것이다.
일견 사소해보이는 이 차이가
계약법 용어에 얼마나 큰 차이를 만들어내는지는
로마법 입문자들이 입문 후 금세 깨닫게 되는 것이니,
그들은 첫 번째 걸림돌을 거의 언제나 여기서 만나고 있다.
우리가 영어로 어떤 계약에 관해 말하면서
그것을 편의상 한쪽 당사자와 결부시키는 경우---가령
어떤 계약의 당사자에 대해 일반적으로 말하려는 경우---우리의
말이 지시하는 것은 언제나
\hemph{약속자}\latin{promis\textit{or}} 쪽이다.
그러나 로마법의 일반적 언어는 방향이 반대이다.
로마법은 계약을 언제나, 이런 용어를 쓸 수 있다면,
\hemph{수약자}\hanjalatin{受約者}{promis\textit{ee}} 쪽에서 바라본다.
계약의 당사자 중에서
주로 언급되는 대상은 언제나 \wi{요약자}\latin{stipulator}, 즉
질문을 하는 사람이다.
하지만
\wi{문답계약}의 유용성이 자못 생생하게 드러나는 예를
라틴 희극작가들의 희곡의 몇몇 페이지에서도 찾아볼 수 있다.
이 대목들이 나오는 장면 전체를 읽어보면
\paren{예컨대, Plautus, \textit{Pseudolus}, 1막 1장;
4막 6장; \textit{Trinummus}, 5막 2장},
질문하는 것이 계약에 임한 사람의 주의를 얼마나 많이 이끌어내는지,
그리고
즉흥적인 합의에 이르지 않게 할 가능성이 얼마나 커지는지
알 수 있을 것이다.

\para{문서계약}
\wi{문서계약}에서 \wi{약정}에 \wi{채권채무관계}가 입혀지기 위해 필요한
요식행위는,
채무액이 확정될 수 있는 경우,
채무의 총액을
원장\hanja{元帳}의 차변\hanja{借邊}에 기입하는 것이었다.
이 계약은 로마의 가\hanja{家}의 관행에 의해 설명될 수 있거니와,
고대에는 그것이 체계적인 성격을 띠었고 무척 규칙적으로 장부작성이
이루어졌던 것이다.
로마 고법\hanja{古法}과 관련하여
가령 노예의 \wi{특유재산}\latin{peculium}의 성격 같은
몇몇 작은 문제들이 있거니와,
이는
로마의 가\hanja{家}가 가부장에게 엄격히 책임지는 다수의 사람들로
구성되었고,
가의 수입과 지출의 모든 항목은,
일단 일지\hanja{日誌}에 기재된 후,
정해진 시기에
가의 총\hanja{總}원장에 이기\hanja{移記}되었음을 상기할 때
비로소 해소될 수 있다.
하지만 문서계약에 관해 남아있는 기술\hanja{記述}에는
몇 가지 모호한 점이 있거니와,
사실
나중에는
장부작성의 습관이
보편적이지 않게 되었고,
``문서계약''이라는 표현은 이제 원래 가졌던 의미와 완전히
다른 형태의 계약을 지칭하게 되었던 것이다.\footnote{%
  문답계약 따위를 문서로써 확정적 증거를 남기는 경우가 흔해졌는데,
  이러한 서면계약을 뜻하게 되었다. \latin{Inst.\,3.21.} }
따라서 우리는
초기의 \wi{문서계약}에서
채권채무관계가 단순히 채권자 측의 기입만으로 성립했는지,
아니면
그것이 법적 효력을 가지려면
채무자의 동일한 행위, 즉 채무자 측 장부의 상응하는 기입도 필요했는지
말할 수 있는 입장에 있지 않다.
하지만
이 계약의 경우
한 가지 조건만 충족되면 다른 모든 요식성은 필요치 않다는
핵심적 성격만은 확실히 알려져있다.
이것은 계약법의 역사에서 또 한 걸음의 진전이었던 것이다.

\para{요물계약}
역사적 순서에 따라 다음에 등장하는 계약인 \wi{요물계약}은
윤리 개념의 큰 진전을 보여준다.
어떤 합의가 물건의 \wi{인도}를 목적으로 한다면---이는
대다수의 단순한 계약을 포괄한다---그 인도가 실제로 행해지는 즉시
\wi{채권채무관계}가 성립하는 것이다.\footnote{%
  요물계약에는 소비대차(mutuum),
  사용대차(commodatum), 임치(depositum), 입질(pignus) 등이 속했다.
  현행 민법에서는 입질(질권설정계약)을 제외하면
  모두 낙성계약으로 되어있다. }
이러한 결과는 초기 계약 관념에 큰 혁신을 의미했다.
의심의 여지 없이 초창기에는,
계약의 당사자가 자신의 합의에 문답계약의 옷을 입히지 못했다면,
계약의 이행으로써 무엇을 했던지 간에
법은 아무 것도 인정해주지 않을 것이기 때문이다.
공식적으로 \hemph{\wi{문답계약}}을 체결하지 않았다면
돈을 빌려주었더라도 빌린 돈을 갚으라고 소송할 수 없었던 것이다.
그러나 요물계약에서는
일방의 이행이 상대방에게 법적 의무를---분명 윤리적 근거에서---부과한다.
그리하여 도덕적 고려가 계약법의 요소로 처음으로 등장한 것이다.
요물계약이
앞서 살펴본 두 가지와 다른 점은,
법기술적 방식이나 로마의 가\hanja{家}의 습관에 대한 존중이 아니라,
도덕적 고려에 기초한다는 데 있다.

\para{낙성계약}
이제 네 번째 유형, 즉 \wi{낙성계약}에 이르렀거니와,
이것은 가장 흥미롭고 가장 중요한 유형이다.
여기에는 네 가지 계약들이 속했고, 그 이름은 다음과 같다:
위임\latin{mandatum}, 조합\latin{societas},
매매\latin{emtio-venditio}, 임약\hanjalatin{賃約}{locatio-conductio}.\footnote{%
  `임약'은 우리 민법의 `임대차' `고용' `도급'을 포괄하는 개념이다. }
몇 페이지 앞에서
계약이란 약정에 채권채무관계가 덧붙여진 것이라고 말한 후,
나는
약정이 채권채무관계로 되기 위해 법이 요구하는
어떤 행위나 요식성에 관해 이야기했다.
나는 일반적 표현의 장점을 활용해 이런 말을 했으나,
적극적인 것 외에 소극적인 것까지 포괄하는 것으로 이해하지 않으면
전적으로 옳은 말이 되지는 못한다.
기실,
낙성계약의 특이성은 \wi{약정} 외에 그 어떤 요식성도
\hemph{전혀} 요구되지 않는다는 것이기 때문이다.
낙성계약에 관하여 많은 옹호될 수 없는 것들이,
더 많은 모호한 것들이 주장되어왔거니와,
심지어 낙성계약에서는 당사자들의 \hemph{동의}\latin{consent}가
다른 어떤 합의 유형들보다 더 강하게 주어진다는 주장까지 있었다.
그러나 저 `낙성'\hanjalatin{諾成}{consensual}이라는 용어는
여기서는
단지 \hemph{합의}\latin{consensus}만 있으면 바로 \wi{채권채무관계}가
부가된다는 것을 의미할 뿐이다.
합의, 즉 당사자들의 상호 동의는
약정에 있어 최종의 그리고 최고의 요소이다.
당사자들의 동의가 이 요소를 제공하자마자
\hemph{즉시} 계약이 성립하는 것은
매매, 조합, 위임, 임약의 네 가지 표제 중 하나에 속하는
합의의 특유한 성질이다.
이 합의는 바로 채권채무관계를 끌고 들어오기에,
특정 종류의 거래에서 그것이 행하는 기능은
다른 종류의 계약에서 물건이, 문답의 언어가,
문서, 즉 장부에의 기입이 행하는 기능과 정확히 일치한다.
따라서 `낙성'은
조금도 이상할 것이 없는 용어이며,
`요물' `언어' `문서'에 정확히 대응한다.

일상생활에서
가장 흔하고 가장 중요한 계약은, 말할 것도 없이,
네 가지 이름의 계약을 포괄하는 \wi{낙성계약}이다.
어떤 공동체든 집단생활의 대단히 큰 부분이
사고 파는 거래, 세\hanja{貰}놓고 세드는 거래,
공동사업을 위해 사람들이 결합하는 거래,
업무처리를 타인에게 맡기는 거래로 구성된다.
바로 그 때문에
다른 많은 사회들과 마찬가지로 로마도
이들 거래에서 법기술적 장애물을 제거하여,
사회적 운동의 효율적 동력이
가능한 한
방해받지 않도록 했을 것이다.
물론 이러한 동기는 로마에만 국한된 것이 아니었다.
로마인들과 이웃 민족들 간의 거래는
우리가 말한 계약들이 어디서나 \hemph{낙성계약},
즉 상호 동의의 의사표시만으로 구속력이 부여되는 계약이
되어가는 것을 관찰할 수 있는
풍부한 기회를
로마인들에게
제공했을 것이다.
그리하여 그들의 통상적인 관행에 따라
로마인들은 이들 계약을
\wi{만민법}상의\latin{juris gentium} 계약으로 분류했다.
하지만 나는
아주 초기부터 이렇게 불리지는 않았다고 생각한다.
만민법\latin{jus gentium}이라는 관념은
\index{외인담당법무관}%
외인\hanja{外人}담당법무관\latin{praetor peregrinus}이 임명되기 오래 전부터
로마 법률가들의 정신에 이미 들어있었다.
그러나 다른 이탈리아 공동체들의 계약 관행에 로마인들이 익숙해진 것은
수많은 거래가 일상적으로 이루어지면서부터일 것이고,
이러한 거래는 이탈리아가 완전히 평정되어
로마의 패권이 확고해지고 나서
비로소 대규모로 이루어질 수 있었을 것이다.
하지만, 비록
낙성계약이 가장 늦게 도입된 것일 확률이 무척 크다고 할지라도,
그리고
`만민법상의'\latin{juris gentium}라는 수식어가 그것의 뒤늦은 도입을
나타내는 것이라 하더라도,
낙성계약을 ``만민법''\latin{law of nations}에 귀속시키는
바로 이 표현이 근대에 들어서는
그것이 아득한 옛날의 것이라는 관념을 만들어냈다.
``만민법''\latin{law of nations}이
``자연법''\latin{law of nature}으로
전환되자,
저 표현은
낙성계약이 자연상태에 가장 부합하는 합의 유형임을
의미한다고 여겨졌던 것이다.
그리하여 문명의 나이가 어릴수록
계약의 형태는 더 단순할 것이라는 이상한 믿음이 형성되었다.

\para{자연법적 채무와 시민법적 채무}
전술했듯이 낙성계약에 속하는 계약들은 그 수가 대단히 제한적이었다.
그러나
낙성계약으로 대표되는 계약법 발달의 단계가
계약에 관한 모든 근대적 관념의 출발점이었음은 의심할 여지가 없다.
이제 합의를 구성하는 의사\latin{will} 작용은
다른 것들과 완전히 분리되어 독립적 고찰의 대상이 되었다.
계약 관념에서 방식은 완전히 제거되었고,
외적 행위는 오직 내적 의사의 징표로만 간주되었다.
더욱이 \wi{낙성계약}은 \wi{만민법}\latin{jus gentium}으로 분류되었으니,
이러한 분류는
얼마 안 가
낙성계약이야말로
자연에 의해 승인되고 자연의 법전에 포함된
계약을 대표하는 합의 유형이라는 추론을 형성시켰다.
이 지점에 이르러, 우리는
로마 법률가들의 몇몇 유명한 법리와 구분들을 만나게 된다.
그 가운데 하나가 자연법상의 채무\latin{natural obligation}와
\wi{시민법}상의 채무\latin{civil obligation}의 구분이다.
지적으로 완전히 성숙한 어떤 사람이
자신의 의사에 기해
어떤 약속을 맺었다면,
비록 필요한 방식을 다 갖추지 못했더라도,
또는 어떤 법기술적 장애로 인해
유효한 계약을 체결할 법적 능력이 결여되어 있었다 할지라도,
그는 \hemph{\wi{자연채무}}\latin{natural obligation}를 진다.
법은
\paren{바로 이것이 저 구분이 의미하는 바이다}
이런 채무를 강제하지 않는다.
그러나 법이 이런 채무를 전혀 인정하지 않은 것은 아니다.
\hemph{자연채무}는
단순히 무효인 채무와는 여러 모로 달랐거니와,
특히 계약 체결 능력을 사후적으로 취득한다면
시민법적으로도 인정될 수 있었던 것이다.\footnote{%
  가령 노예가 진 빚은 자연채무였다. 따라서 노예가 해방된 뒤
  스스로 빚을 갚으면 반환받을 수 없었다.
  \latin{Dig.\,12.6.13.pr.}
  마찬가지로 후견인의 조성(助成) 없이 미성숙자가 돈을 빌려 이득을 본 경우,
  그가 성숙기에 달한 후 갚으면 반환받을 수 없다.
  \latin{Dig.\,12.6.13.1.}
  }
법학자들의 또 하나의 특유의 법리는 약정이
계약의 법기술적 요소로부터 분리된 후에 비로소 등장할 수 있었다.
그들에 따르면,
계약만이 \hemph{소권}\hanjalatin{訴權}{action}을 근거지울 수 있었지만,
단순한 \wi{약정}은 \hemph{\wi{항변권}}\hanjalatin{抗辯權}{plea}의 기초가 될 수 있었다.
그리하여,
누구도
계약을 성립시키는 데 필요한
방식을 갖추지 못한
합의에 기초하여 소송을 제기할 수 없었지만,
유효한 계약에 기초한 주장이라도
단순한 약정 상태에 불과한 다른 합의가 있었음을 입증함으로써
이를 물리칠 수 있었다.
가령 금전채무의 회수를 구하는 소송은
채무 면제나 유예의
단순한 비공식적 합의를 입증하여 이에 대항할 수 있었다.

\para{계약법의 변화}
방금 언급한 법리는 법무관들이 궁극적 혁신을 이루어내는 데
방해물로 작용했을 것이다.
그들의 자연법 이론은
낙성계약과 이를 포함하는 약정 일반에 대해
특별한 호의를 갖도록 그들을 이끌었을 것이 틀림없다.
그러나 그들이 즉시
모든 약정에 낙성계약의 효력을 확대적용하는 모험을 감행한 것은 아니었다.
그들은 로마법 초기부터 그들에게 주어졌던
소송절차에 대한 감독권한을 이용했거니와,
방식을 갖추지 못한 계약에 근거한 소송은 여전히 허가하지 않았지만,
합의에 관한 새로운 이론을 적극 활용하여
이후의 발달 단계로 향하는 길을 텄다.\footnote{%
  이른바 `법무관법상의 약정'(pacta praetoria)을 말하는 듯하다.
  특정 기일에 자기 또는 타인의 기존 채무를 변제하겠다는 약정,
  중재인, 은행업자, 선박주인·여관주인·마구간주인 등의 인수(引受)약정
  따위가 여기에 속한다. }
그러나
여기까지 나아가자
계속 더 나아가는 것이 불가피해졌다.
고대 계약법의 혁명이 달성된 것은,
어느 해인가 \wi{법무관}의 \wi{고시}\hanja{告示}가
계약의 옷을 입지 못한 \wi{약정}이라도
그 약정이 \wi{대가관계}\latin{consideration}\paren{원인\latin{causa}}에
기초하는 것인 한
\wi{형평법}상의 소송을 허가하겠노라고 공표했을 때였다.\footnote{%
  이른바 `무명요물계약'(無名要物契約)을 말한다.
  쌍무성(synallagma) 있는 계약의 당사자라면
  자신의 급부를 이행한 경우---따라서 일종의 요물계약이었다---상대방의
  이행을 소구할 수 있었다.
  혹은 자신의 급부를 돌려달라는 부당이득반환청구소송을 제기할 수 있었다.
  로마 법학자들에 따르면 무명요물계약은 다음 네 가지 유형을 포괄했다.
  `네가 주도록 내가 준다'(do ut des)
  `네가 하도록 내가 준다'(do ut facias)
  `네가 주도록 내가 한다'(facio ut des)
  `네가 하도록 내가 한다'(facio ut facias).
  결국 `주는 채무'과 `하는 채무'가 쌍무적으로 견련되는 모든 유형의 약정에
  소권이 주어질 수 있었다.
  \latin{Dig.\,19.5.5.} }
이런 종류의 약정은 발달된 로마법에서는 항상 강제되었다.
이 원리는
\wi{낙성계약}의 원리가 그 합당한 결과에 도달한 것에 불과했다.
사실, 로마인들의 법기술적 용어가 그들의 법이론만큼이나 유연했다면,
법무관에 의해 강제된 이들 약정은
새로운 계약, 새로운 낙성계약이라고 불렸을지도 모른다.
하지만 법용어는 가장 바뀌기 어려운 법이어서,
형평법적으로 강제된 저 약정들은
여전히 `\wi{법무관법상의 약정}'\latin{praetorian pacts}이라고만 지칭되었다.
만약 약정에 \wi{대가관계}가 없다면,
\index{나약정}%
새로운 법에서도 계속 \hemph{나}약정\hanja{裸約定}이었음을
유의해야 한다.
이것에 법적 효력을 부여하려면,
문답계약을 통해 언어계약으로 전환시켜야 했다.

\para{계약법의 진화}
수많은 오해에 대한 방패막이로서 큰 중요성을 가지기에
계약법의 역사를
이렇게
길게 논하고 있는 것도 이해해주시리라 믿는다.
그것은 하나의 중대한 법관념으로부터
다른 중대한 법관념으로의 관념들의 행진을 완전히 설명해준다.
우리는 구속행위로부터 시작했으니,
여기서는 계약과 양도가 혼재되어 있고,
합의에 수반되는 요식성이 합의 자체보다 훨씬 중요하다.
구속행위 다음에 오는 \wi{문답계약}은 옛 의례행위의 단순화된 방식이다.
그 다음의 \wi{문서계약}에서는
로마의 가\hanja{家}의 엄격한 관행에 의해
합의가 입증되기만 하면 다른 모든 요식성은 포기된다.
\wi{요물계약}에서는 도덕적 의무가 처음으로 인정되니,
계약의 일부 이행을 수령하거나 묵인한 자는
방식의 흠결을 이유로 계약을 부인하는 것이 금지된다.
끝으로 \wi{낙성계약}이 등장함으로써,
계약 당사자들의 내심의 의사만 고려대상이 되고
외적 격식은 내적 의사의 증거로서만 의미를 갖는다.
조야한 개념에서 세련된 개념으로 나아가는 로마법의 이러한 관념의 진보가
계약에 관한 인간 사고의 필연적 진보과정을 얼마나 예시하고 있는지는
물론 확인할 수 없다.
로마를 제외한 다른 모든 고대사회는
계약법이 너무 부족하여 정보를 얻을 수 없거나,
아니면 계약법 자체가 아예 없다.
또한 근대법은 철저히 로마법의 관념을 이어받은 것이어서
가르침을 구할 만한 비교대상이 되지 못한다.
하지만 우리가 살펴본 고대 로마계약법의 역사에는
억지스럽거나 놀랍거나 불가해한 것이 전혀 없기에,
어느 정도는 그것이
다른 고대사회의 계약법 개념의 역사에도 통용된다고
보아도 불합리하지 않을 것이다.
그러나 로마법의 진보가 다른 법체계의 진보를 대표하더라도,
그것은 어느 정도까지만 그렇다는 것이지 그 이상은 아니다.
자연법 이론은 로마법에만 있었다.
`법의 사슬'이라는 관념도, 내가 알고 있는 한,
로마법에만 있었다.
로마의 성숙한 계약법과 불법행위법의 많은 특징들은
이 두 가지  관념이 따로 혹은 함께 작용한 결과이거니와,
따라서 특정한 사회 하나만의 산물인 것이다.
이 후대의 법관념이 갖는 중요성은,
어떤 상황에서도 진보적 사고의 필연적 결과를 대표한다는 데
있는 것이 아니라,
근대 세계의 지적 기질에 엄청나게 큰 영향력을 행사했다는 데 있다.

\para{로마제국의 법적 사고, 동방과 서방의 관념}
로마법, 특히 로마계약법이
다양한 학문의
사고양식, 추론과정, 전문용어에 기여한 것보다
더 대단한 일이 또 있는지 모르겠다.
자연과학을 제외하고,
근대인의 지적 욕구를 자극한 대상 중에
로마법이라는 여과장치를 통과하지 않은 것은 거의 없다.
순수한 형이상학은 물론 로마보다는 그리스의 후예이지만,
정치학, 도덕철학, 심지어 신학까지,
모두가 로마법에서 표현수단을 발견했을 뿐만 아니라,
거기서 학문 발달을 위한 깊이있는 탐구가 배양되는 거점도 발견했다.\footnote{%
  이 단락부터 이 챕터의 거의 끝까지는
  정치학, 도덕철학(윤리학), 신학에 끼친
  로마법의 영향, 특히 로마계약법의 영향에 대한 논의가 이어진다.
  }
이런 현상을 설명하기 위해,
말과 관념 간의 불가사의한 관계를 논할 필요는 전혀 없을 것이며,
또한
적절한 언어의 저장고와 적절한 추론의 장치가 미리 주어지지 않으면
인간 정신은 어떠한 사고 주제도 다룰 수 없었다는 것을 설명할
필요도 전혀 없을 것이다.
동방과 서방의 철학적 관심이 분리되었을 때,
서방 사상의 기초자들은 라틴어로 말하고 라틴어로 사고하는 사회에
속해있었다는 것만 말해도 충분할 것이다.
그런데
서방에서는
철학적 목적을 충족하는 정확성을 갖는 유일한 언어가
로마법의 언어였거니와,
비속\hanja{卑俗}라틴어가 불길한 만족\hanja{蠻族}들의 방언으로 전락해가는 동안,
로마법의 언어는 특별한 행운으로
\wi{아우구스투스} 시대의 순수함을 거의 그대로 보존할 수 있었다.
로마법이 언어의 정확성을 위한 유일한 수단이었다면,
사고의 정확성과 명석함과 깊이를 위한 유일한 수단은 더더욱 로마법이었다.
서방에서는
적어도 3백녁 동안 철학과 학문이 자리잡지 못하고 있었다.
비록 형이상학과 형이상학적 신학이 로마인 백성들의 정신적 에너지를
독점하고 있었지만,
이러한 열렬한 탐구에 사용된 언어는 오직 그리스어였고,
그러한 탐구의 무대는 동로마제국이었다.
사실,
때로 동로마의 논쟁들의 결과는 사뭇 중요해져서
그것에 찬성하고 반대하는 모든 이들의 의견이 기록되어졌다.
그후 이러한 동방의 논쟁의 결과가 서방에 소개되었으니,
대체로 그것은
감흥없이 그리고 저항없이 받아들여졌다.
그러는 동안,
가장 근면한 자에게도 어렵고,
가장 명석한 자에게도 멀고,
가장 치밀한 자에게도 까다로운
학문 분야 하나가 서방의 식자층 사이에서
매력을 잃지 않고 있었다.
아프리카, 에스파니아, 갈리아, 북이탈리아의 교양있는 시민들에게
그것은 법학, 오직 법학이었으니,
그것은 시와 역사, 철학과 학문을 대신하는 것이었다.
서방 사상의 초기 성과의 명백히 법적인 양상에는
불가사의한 것이 거의 없었으므로,
그것이 다른 의미로 다가왔다면 오히려 놀라운 일이었을 것이다.
나로서는
어떤 새로운 요소의 존재로 인해 생겨난
서방과 동방의 관념의 차이가,
서방의 신학과 동방의 신학의 차이가,
그동안 거의 주목을 받지 못했다는 것이
놀라울 따름이다.
이렇게 법학의 영향이 강력해지기 시작하기 때문에,
콘스탄티노플의 건설과 이후 서로마제국과 동로마제국의 분리가
철학의 역사에서 획을 긋는 사건이 되는 것이다.
그러나 대륙의 사상가들은
분명 이 중차대한 국면의 중요성을 인식하기 어려운 위치에 있으니,
그들은 로마법에서 유래한 관념들에 친숙하고
그것이 일상적 관념들에 섞여들어가 있기 때문이다.
반면, 영국인들은 놀라울 정도로 그것을 알지 못하니,
근대 지식의 가장 풍부한 원천,
로마 문명의 유일한 지적 성과로부터 스스로를 유폐한 것이다.
동시에,
고전기 로마법에 친숙해지는 데 많은 노력을 들이는
영국인이라면,
지금까지 영국인들이 이 분야에 무심했다는 바로 그 사실 덕분에,
내가 감히 내놓는 주장의 가치에 관해
프랑스인이나 독일인보다
더 나은 판관이 될 수 있을 것이다.
로마인들이 실제 관용한 로마법이 무엇인지 아는 사람은,
그리고 초기 서방의 신학과 철학이 그 이전의 사상 국면과 어떻게
달랐는지 아는 사람은,
사변\hanja{思辨}을 지배하기 시작한 새로운 요소가 무엇이었는지
선언할 수 있는 위치에 있다고 할 것이다.

\para{준계약}
로마법의 여러 영역 중에 다른 학분 분야에 가장 큰 영향을 끼친 것은
채권법, 혹은, 거의 같은 말이지만, 계약법과 불법행위법이었다.
로마인들도
이 법영역에 속하는 용어가 감당하게 될 역할을 모르지 않았거니와,
그것은 그들이 특유의 \hemph{준}\hanjalatin{準}{quasi}이라는
수식어를 `\wi{준계약}'\latin{quasi-contract}과
`\wi{준불법행위}'\latin{quasi-delict} 같은 표현에 사용한 것을 보면 알 수 있다.
여기서 ``준''이라는 말은 분류를 위한 용어일 뿐이다.
흔히 영국의 학자들은 준계약을 \hemph{묵시적}\latin{implied} 계약과 동일시해왔으나,
이는 잘못이다.
묵시적 계약은 진짜로 계약이지만 준계약은 그렇지 않기 때문이다.
명시적 계약에서 말로써 상징되는 것과 동일한 요소가
묵시적 계약에서는 행위와 상황에 의해 상징되거니와,
어떤 이가
어느 쪽 상징집합을 사용하든
합의의 이론에 관한 한
아무런 차이가 없다.
그러나 준계약은 계약이 아니다.
이 유형에 해당하는 가장 흔한 사례는
한 사람이 다른 사람에게 착오로 돈을 지불한 경우 두 사람 간의 관계에서 발견된다.
법은, 도덕의 관점에서,
반환할 채무를 수령자에게 지운다.
그러나 그 성질은 계약에 기초하는 것이 아니니,
계약의 본질적 요소인 \wi{약정}\latin{convention}이 결여되어 있기 때문이다.
로마법의 어떤 용어에 붙는
``준''이라는 말은 그것이 지시하는 개념이
비교대상이 되는 개념과 강한 외관상의 유비\hanja{類比} 혹은 유사성으로
연결되어 있다는 것을 의미한다.
두 개념이 동일하다거나, 혹은 동일한 유\hanja{類}에 속한다는 말이 아니다.
오히려
그것들 간의 동일성을 부정하는 의미가 들어있다.
그러나 그것들은 충분히 유사해서
하나가 다른 하나의 속편\hanja{續篇}으로
분류되고,
하나의 법영역의 용어를 다른 법영역에도 쓸 수 있어서,
그렇지 않으면 불완전하게 표현될 수밖에 없는 법규칙의 진술에
과도한 왜곡 없이 사용할 수 있다는 뜻이다.

\para{준계약과 사회계약}
진짜 계약인 묵시적 계약과 계약이 아닌 \wi{준계약} 간의 혼동이
정치적 권리와 의무의 원천을
통치자와 피치자 간의 원초적 계약에서
찾는
저 유명한 오류와 공통점이 많다는 예리한 지적이 있다.
이 이론이 확립되기 오래 전부터도
주권자와 백성들 간에 존재한다고 여겨지는
권리와 의무의 상호성을 기술하는 데
로마계약법의 용어가 자주 사용되어왔다.
무조건적인 복종을 요구할 수 있는 왕의 권리를 적극적으로 내세우는
격률들---신약성서에서 기원한다고 주장되었으나 실은
황제들의 전제정에 대한 기억이 지속된 데서 유래한 격률들---은
세상에 가득했지만,
피치자들이 갖는 상응하는 권리의 인식은,
만약
아직 제대로 발달하지 못한 관념을 암시하는 언어를
로마채권법이
제공해주지 않았다면,
그것을 표현할 수단이 전혀 없었을 것이다.
왕의 특권과 백성들에 대한 그의 의무 간의 대립은
서양 역사가 시작된 이래 한번도 잊혀진 적이 없다고 믿지만,
봉건제도가 굳건히 존속하는 동안은
아주 예외적인 극소수의 사상가를 제외하고는
이 문제에 관심을 두지 않았다.
\wi{봉건제}의 명백한 관습에 의해
유럽의 대부분 주권자들의 터무니없는 이론적 주장이
효과적으로 통제되었기 때문이다.
하지만,
주지하듯이,
봉건체제의 붕괴로 중세의 헌정질서가 혼란에 빠지자,
그리고
종교개혁으로 교황의 권위가 추락하자,
\wi{왕권신수설}\hanja{王權神授說}은
과거에 한번도 누려보지 못한
중요한 이론의 지위로
급부상했다.
이 이론이 얻은 인기는
로마법 용어에 상시 의존하는 경향을 심화시켰고,
원래 신학적 옷을 입고 있던 논쟁은
점점
법적인 논쟁의 분위기를 띠어갔다.
그리하여 여론의 역사에서 반복적으로 나타나던 현상 하나가 등장했다.
군주의 권력을 옹호하는 주장이 \wi{필머}\latin{Robert Filmer}의 교리로
확립되자,
피치자의 권리를 방어하는 데 사용되었던,
계약법에서 빌려온
용어가
왕과 신민 간의 원초적 계약이 실재한다는 이론으로
구체화되었던 것이다.
이 이론은 처음에는 영국인들의 손에서,
나중에는 특히 프랑스인들의 손에서,
모든 사회현상과 법현상을 포괄적으로 설명하는 이론으로 확장되었다.
그러나
정치학과 법학의 진정한 결합은
후자가 전자에게
특유의 유연한 용어를 제공한 것이 전부였다.
로마계약법은,
주권자와 백성의 관계에 대해서도,
보다 소박한 영역에서
``\wi{준계약}''의 \wi{채권채무관계}로 묶인 사람들의 관계에 대해 수행하던 것과
정확히 똑같은 기능을 수행했다.
그것은
정치조직이라는 주제에 관하여 수시로 형성되고 있던 관념들에
사뭇 잘 들어맞는
일군의 용어와 문장들을 제공했다.
원초적 계약의 교리는,
비록 부당한 것이지만,
휴얼\latin{William Whewell} 박사의
찬사보다 더 높은 찬사를 받을 수는 없을 것이다.
``그것은 도덕적 진리를 표현하는 \hemph{유용한} 형식일 것이다.''\footnote{%
  \latinmarks
  William Whewell,
  \textit{The Elements of Morality Including Polity},
  Vol.\,2,
  London: John W. Parker, 1848,
  p.\,113.
  }

\para{윤리학과 로마법}
우선
정치적 주제에 관한 법적 용어가
원초적 계약의 발명에
광범위하게 사용되어 들어간 것, 그리고
이후 이 가정\hanja{假定}이 강력한 영향력을 행사한 것은
정치학에는
용어와 개념이
왜 그렇게 풍부한가를 넉넉히 설명할 수 있거니와,
그것은 오로지 로마법의 산물이었다.
\wi{도덕철학}\latin{moral philosophy}에서 용어와 개념이 풍부한 것에는
조금 다른 설명이 주어져야 한다.
정치적 사변\hanja{思辨}에 비해
윤리학 저술들에서는 로마법의 기여가
훨씬 더 직접적이었으며,
윤리학 저자들도
그 은혜의 크기를 훨씬 더 잘 알고 있었다.
내가 도덕철학이 로마법에 크게 빚졌다고 말하는 것은
칸트에 의해 도덕철학의 역사에 단절이 일어나기 이전의
도덕철학을 대상으로 하는 것임을 알아야 한다.
그것은 인간의 행위를 규율하는 규칙들과
그 규칙들의 적절한 해석,
그리고 그 규칙들의 한계를
다루는 학문이었다.
비판철학이 등장한 이후
도덕철학은 옛 의미를 완전히 상실했거니와,
로마 가톨릭 신학자들이 여전히 가꾸고 있는
\wi{결의론}\hanjalatin{決疑論}{casuistry}에서
저급한 형태로 보존되어 있는 것을 제외하면,
도덕철학은 거의 예외 없이 존재론의 한 분야로 간주되고 있는 듯하다.
형이상학에 흡수되기 이전의 도덕철학,
규칙 자체보다 규칙의 근본원리가 더 중요하게 고려되기 이전의 도덕철학을
이해하는
현대 영국 학자는
내가 알기로,
휴얼 박사를 제외하면,
한 사람도 없다.
하지만,
오랫동안
윤리학은
실천적 행위준칙을 다루어왔기에,
그것은 어느 정도 로마법에 물들어 있었다.
근대 사상의 다른 모든 주요 분야들과 마찬가지로,
원래
그것은
신학과 한 몸이었다.
처음에는 `\wi{도덕신학}'\latin{moral theology}이라 불렸고
지금도 로마 가톨릭 신학자들 사이에서는 이렇게 불리고 있는
이 학문은 분명, 그 저자들도 잘 알고 있었듯이,
행위의 원리를
교회체계로부터
가져오는 것, 그리고
이를 표현하고 전개하는 데 법학의 언어와 방법을 사용하는 것으로
구성되어 있었다.
이런 과정이 지속되면서,
사고\hanja{思考}의 운송수단에 불과했어야 할 법학이
사고 그 자체에도 자신의 색깔을 전달하는 일이 불가피하게 일어났다.
법개념들과의 접촉에서 얻은 이러한 색조는
근대 세계의 초창기 윤리학 문헌에서 쉽게 감지할 수 있거니와,
생각건대
만약 계약법이 없었더라면
도덕적 의무를
신국\hanja{神國}의 시민의 공적 의무로만
바라보려는
저자들의
경향을,
철저한 상호성과 권리·의무의 확고한 결속에 기초하는 계약법이
건강한 방향으로 교정했음에
틀림없다.
그러나 \wi{도덕신학}에서 로마법이 차지하는 비중은
스페인의 도덕론자들이 이 학문을 키우면서부터
눈에 띄게 줄어들게 된다.\footnote{%
  이른바 살라망카 학파를 말하고 있는 듯하다.
  }
박사들에 의해 주석에 주석이 달리는 법학적 방법으로 발달되던
도덕신학이 자신만의 용어를 스스로 만들어냈다.
또한
학파들의 도덕 논쟁에서 상당 부분 흡수한 것이 분명한
아리스토텔레스적 추론과 표현의 특색들이
로마법에 정통한 사람이라면 결코 틀릴 수 없는
사고와 언어의 특수한 문체를 대신하게 된다.
스페인 학파의 도덕신학자들이 계속해서 신망을 유지했다면
윤리학에서 법학적 요소는 하찮은 수준으로 쪼그라들었을 것이다.
그러나
그들의 영향력은
다음 세대 로마 가톨릭 저술가들이
이 학문 영역에서
그들의 성과를 이용한 방식에 의해
거의 전적으로 파괴되어버렸다.
\wi{결의론}\hanja{決疑論}으로 전락한
도덕신학은
유럽의 사변\hanja{思辨}을 선도\hanja{先導}하는 자들의
관심을 상실했으며,
전적으로 프로테스탄트의 손에 들어간
새로운 \wi{도덕철학}은
\wi{도덕신학}자들이 추종하던 길을 크게 벗어났다.
결과적으로 윤리학에 대한 로마법의 영향은 대폭 증가했다.

\para{그로티우스 학파, 결의론의 쇠퇴}
``종교개혁 이후,\origfootnote{%
  이 인용문은
  1856년 <<케임브리지 논문집>>(Cambridge Essays)에 기고한
  저자의 논문의 일부를 조금 수정하여 가져온 것이다.
}\,\footnote{%
  인용된 논문의 제목은 ``로마법과 법교육''(Roman Law and Legal Education)이다.
  저자의 저서 <<동·서양의 촌락공동체>>(Village Communities
  in the East and West) 제3판(1876년)에도 재수록되었다.
}
이 학문 영역에서는
사상을 달리하는 두 개의 큰 학파 간의 대립이 나타났다.
둘 중 더 영향력 있는 쪽은 애초
\wi{결의론}자\latin{casuist}로 우리에게 알려진 분파 혹은 학파였거니와,
그들 모두는 로마 가톨릭 교회를 신앙했고,
그들의 거의 모두는 이런저런 가톨릭 수도회에 소속되어 있었다.
다른 쪽은
<<전쟁과 평화의 법>>의 위대한 저자
후고 \wi{그로티우스}의 지적 후예라는 공통점을 갖는
일군의 학자들이었다.
후자 쪽의 거의 모두는 종교개혁의 추종자들이었으니,
그들이 공식적·공개적으로 결의론자들과 갈등했다고 할 수는 없을지라도,
그들 체계의 기원과 대상은 결의론자들의 것과 근본적으로 달랐다.
이러한 차이는 주목할 필요가 있거니와,
이들 양 체계의 사상 영역에 끼친 로마법의 영향 문제와 관련되기 때문이다.
그로티우스의 저 저서는,
비록 모든 페이지마다 순수한 윤리학 문제를 다루고 있지만,
또한 비록 수많은 공식적인 윤리학 저서의 직·간접적 선조이지만,
주지하듯이 \wi{도덕철학}에 관한 논저임을 자처하지는 않는다.
그것은 자연의 법\latin{law of nature},
즉 자연법\latin{natural law}을 명확히 하려는 시도이다.
자연법이라는 개념이 로마 법학자들의 배타적 창안이었는지의
문제를 따질 필요 없이,
그로티우스 본인이 스스로 인정한 것에 근거하여,
실정법의 어느 부분을 자연법의 일부로 보아야 하는가에 관한
로마 법학의 언명은,
그것이 오류가 아닌 한,
언제나
사뭇 깊은 존경을 받으며 수용되었다고
볼 수 있다.
그리하여 그로티우스의 체계는
로마법과 근본적으로 얽혀있는 것이다.
이러한 연결로 인해 불가피---저자가 법학으로 교육받았던 것의
결과이기도 하겠지만---단락마다
법기술적 용어가 자유자재로 구사되고 있고,
추론과
정의\hanja{定義}와 예시의 방식도 마찬가지이다.
이들이 어디서 유래했는지 출처를 모르는 독자들은 틀림없이,
때로는 그 논증의 의미를 이해하기 어렵고,
거의 항상은 그 논증의 힘과 설득력을 파악하기 어려울 것이다.
다른 한편,
결의론은 로마법에서 빌려온 것이 거의 없고,
무엇이 도덕적이냐의 견해도 그로티우스의 그것과 공통점이 없다.
\wi{결의론}의 이름 아래 유명해진, 혹은 악명높아진, 옳고 그름에 관한 저 모든 철학은
대죄\hanjalatin{大罪}{mortal sin}와
소죄\hanjalatin{小罪}{venial sin} 간의 구분에 기초한다.
어떤 행위를 대죄로 판정하는 끔찍한 결과를 피하려는 자연스런 염려와,
프로테스탄티즘과 대결하고 있는 로마 가톨릭 교회에게서
부담스런 이론의 짐을 덜어주려는, 역시 이해할 만한, 열망에서,
결의론 철학의 저자들은
비도덕적 행위를 가능하면 대죄의 영역에서 제외하여
소죄의 영역에 편입시키려는 복잡한 행위기준의 체계를 발명하게 되었다.
이러한 실험의 결과는 일반 역사의 영역이다.
주지하듯이 결의론의 저 구분은,
사제들의 영적\hanja{靈的} 통제가 다종다양한 인간성에 부응할 수 있도록 만들어,
실로
군주·정치인·장군들에 대한
사제들의
영향력을
종교개혁 이전에는 들어본 적이 없는 수준으로
키워주었으니, 이는
프로테스탄티즘의 초기 성공을 견제하고 축소시킨
저 반\hanja{反}종교개혁에 크게 기여했던 것이다.
그러나 무언가를 세우려는 것이 아니라 피하려는 시도로,
원리를 발견하는 것이 아니라 공준\hanja{公準}을 피하려는 시도로,
옳고 그름의 본성을 정하는 것이 아니라
특정 본성의 무엇이 그르지 않은 지를 정하려는 시도로
출발한 결의론은
교묘한 복잡함을 더해간 결과,
행위의 도덕적 성격을 감소시키고
인간의 도덕적 본능을 배반하는 지경에 이르렀으니,
마침내 그것에 거역하여 인류의 양심이 일거에 들고일어나
그 체계와 그 박사들을 모두 공통의 파멸로 몰아넣었다.
오래도록 유예되었던 일격이 마침내 \wi{파스칼}의
<<시골 친구에게 보내는 편지>>\latin{Provincial Letters}에 의해
가해졌다.
이 주목할 만한 저서가 등장한 이후,
조금이라도 영향력이나 신망이 있는 윤리학자라면
자신의 사변을 공공연히 \wi{결의론}에 기초하여 전개할 수는 없게 되었다.
윤리학의 전 영역은 이제 전적으로
그로티우스의 추종자들의 손에 남겨지게 되었다.
그리하여 지금도 윤리학은,
때로는 \wi{그로티우스} 이론의 흠의 원인으로 평가되기도 했고
때로는 그의 이론에 최고의 상찬을 가져다주기도 했던
로마법과의 연루의 흔적을
비상한 정도로 보여주고 있다.
그로티우스 시대 이래 많은 연구자들이 그의 원리를 수정했고,
비판철학의 등장 이후로는 많은 이들이 그의 원리를 포기했지만,
그의 근본 가정\hanja{假定}으로부터 가장 멀리 떠나온 사람들조차
그의 진술 방법, 그의 사고 순서, 그의 예시 방식의 많은 것을
물려받았다.
그리고 이런 것들은 로마법에 무지한 사람들에게는 거의 혹은 전혀
이해될 수 없는 것들이다.''

\para{형이상학과 로마법}
전술했듯이,
자연과학을 제외하면,
형이상학만큼 로마법의 영향을 적게 받은 학문 영역도 없다.
그 이유는 형이상학적 주제의 논의는 언제나 그리스어로 이루어졌다는 데 있다.
정확히 말하면 처음에는 순수한 그리스어로,
나중에는 그리스어 개념을 표현하기 위해 특별히 만들어진 라틴어 방언으로
이루어졌던 것이다.
현대 언어들은 이 라틴어 방언을 채용함으로써,
혹은 그것의 형성기의 과정을 모방함으로써,
비로소
형이상학적 탐구에 적합한 언어가 될 수 있었다.
근대에 들어 형이상학적 논의에 항상 사용되어온 용어의 출처는
라틴어로 번역된 아리스토텔레스였거니와,
그것이 아랍어판을 번역한 것이든 아니든,
번역자의 의도는
라틴어 문헌에서 유사한 표현을 찾는 것이 아니라,
그리스 철학 관념을 표현하는 일군의 용어들을
라틴어 어근으로부터
새롭게 구성하는 것이었다.
이러한 과정에 로마법 용어가 줄 수 있는 영향은 거의 없었다.
기껏해야 몇몇 라틴어 법률용어가 변형된 형태로
형이상학의 언어에 포함되었을 뿐이다.
동시에 언급하고 싶은 점은,
서유럽을 자못 크게 뒤흔든 형이상학의 문제는 어느 것이든,
그 언어는 몰라도,
그 사상은 법학적 기원을 드러낸다는 것이다.
사변\hanja{思辨}의 역사에서 아마도 가장 인상적인 것은,
그리스어를 말하는 민족은
자유의지\latin{free will}와 필연성\latin{necessity}\footnote{%
  `necessity'는 법률용어인 `긴급피난'으로 번역될 수도 있다.
}이라는 중대한 문제로 심각하게 고민해본 적이 없다는 사실일 것이다.
나는 이 문제를 간략하게라도 감히 설명할 생각이 전혀 없다.
그러나 그리스인들도, 그리고 그리스어로 말하고 생각하는 어떤 사회도,
법철학을 생산할 일말의 능력도 보여준 적이 없다는 사실은
이와 무관하지 않다고 생각한다.
법학은 로마인들의 산물이며,
자유의지의 문제는 형이상학적 개념을 법적인 측면에서 숙고할 때 등장하는
문제이다.
어떻게 해서 이 문제가
불변의 사건 연쇄는 필연적\latin{necessary} 관계와 동일한 것인지 어떤지의
문제로 되었는가?
내가 말할 수 있는 것은,
로마법의 경향은,
시간이 갈수록 강해진 그 경향은,
법적 원인과 법적 효과가
흔들림없는
필연성으로 결합된다고
보았다는 것뿐이다.
앞서 인용한 채권채무관계의 정의가 이러한 경향의 현저한 사례이다:
``누군가에게 급부\hanja{給付}를 할 것이 필연적으로 강제되는
법의 사슬''\latin{juris vinculum quo necessitate adstringimur alicujus
solvendae rei}.

\para{교회에서의 로마법}
그런데 자유의지의 문제는
철학이기 이전에 신학의 문제였으며,
그 용어가 법학의 영향을 받았다면
그것은
법학이 신학에 의해 감지되어 받아들여졌기 때문일 것이다.
여기서 내가 제시하는 주요 논점은 한번도 만족스럽게 해명된 적이 없는 것이다.
우리가 확인해야 할 점은,
법학이
신학적 원리에 접근하는 매개체로
기능했는가,
특유의 언어를, 특유의 추론양식을, 여러 세상사에 대한 특유의 해결책을
제시함으로써 법학은 신학적 사변이 흘러나오고 확장되어가는
새로운 통로를 열었는가 하는 것이다.
답을 구하기 위해서는,
초기에 신학이 흡수한 지적 양식\hanja{糧食}이 무엇이었는가에 관해
최고의 학자들 간에 이미 합의된 것을 상기할 필요가 있다.
기독교 교회의 초창기 언어는 그리스어였으며,
기독교 교회가 초기에 대처한 문제들도
후기 그리스 철학이 그 길을 닦아놓았던 문제였음이
널리 인정되고 있다.
신의 위격\hanjalatin{位格}{persons},
신의 본체\latin{substance},
신의 본성\latin{natures} 같은 심오한 논쟁에 인간 정신이 참여할 수 있도록
해주는 언어와 관념의 유일한 창고는
그리스의 형이상학적 문헌들이었다.
라틴어와 소박한 라틴 철학은 이러한 임무를 감당할 능력이 사뭇 모자랐고,
따라서 서방, 즉 라틴어를 사용하는 유럽 지역은
동방의 성과를 따지지도 검토하지도 않고 그대로 받아들였다.
밀만\latin{Henry Hart Milman} 주임사제에 따르면,
``라틴 기독교는 자신의 협소하고 빈약한 어휘로는 적절하게 표현하기 어려운
저 신조를 받아들였다.
그런데 로마와 서방의 동의는 어디까지나
동방 신학자들의
심오한 신학에 의해 형성된 교리체계를 수동적으로 묵인한 것이었을 뿐,
신학적 난제들을 스스로 열성적으로 그리고 독창적으로 검토한 것이 아니었다.
라틴 교회는 아타나시우스\latin{Athanasius}의 제자였으며
충성스런 지지자였다.''\footnote{%
  \latinmarks
  Henry Hart Milman, \textit{History of Latin Christianity},
  London: John Murray, 1854, p.\,61. }
그러나 동로마와 서로마의 분리가 더욱 확고해지고
라틴어를 쓰는 서로마제국이 스스로의 지적 삶을 살기 시작하면서,
동방에 대한 존경은 갑자기
동방적 사변\hanja{思辨}에는 전적으로 생경한
다수의 문제들에 관한 격론으로 변모했다.
``그리스 신학이
\paren{밀만, <<라틴 기독교>>, 서문, 5쪽}
훨씬 세련된 섬세함으로 삼위일체와 그리스도의 본성을 계속 정의해가는 동안''
``끝없는 논쟁이 여전히 길게 이어지고
허약해진 공동체로부터 이런저런 분파들이 계속 분리되어나가는 동안''\footnote{%
  위의 책, p.\,5. }
서방 교회는
새로운 종류의 논쟁들에 열정적으로 뛰어들었으니,
이는 그때부터 지금까지 라틴 교파에 속하는 사람들이라면 한시도
관심을 놓지 않았던 것들이다.
원죄와 그것의 대물림,
인간의 진 빚과 그것의 대속\hanja{代贖},
속죄\latin{Atonement}의 필연성과 충분성,
특히 자유의지와 신의 섭리 간의 표면적 대립관계,
서방은
이런 것들을
동방이 특정한 신경\hanja{信經}의 조항을 두고 논쟁했던 것 못지않게
가열차게 논쟁하기 시작했다.
그렇다면
그리스어를 쓰는 지역과
라틴어를 쓰는 지역 간에
신학적 문제의 종류가 서로 그렇게 달랐던 까닭은 무엇이었을까?
교회사가\hanja{史家}들은
동방 기독교를 갈라놓았던 문제들보다
새로운 문제들이
더 ``실제적인''\latin{practical},
즉 전적으로 사변적이지는 않은 것이었다고 말하여
어느 정도 정답에 가까이 다가갔으나,
내가 아는 한 어느 누구도 정답에 도달하지는 못했다.
나는
두 신학체계 간의 차이는,
동방에서 서방으로 넘어오면서
신학적 사변의 풍토도
그리스의 형이상학에서 로마법으로 바뀌었다는 사실로
설명된다고
서슴없이 주장하고 싶다.
이들 논쟁이 압도적으로 중요한 논쟁으로 부상하기
수 세기 전부터
서로마인들은 그들의 지적 활동을 전적으로 법학에 쏟아부었다.
그들은
세상사가 조합해낼 수 있는 온갖 상황에
특유의 법원리들을
적용하는 일에 몰두해왔다.
다른 어떤 업무나 취미도
그 일에서 그들의 관심을 멀어지게 할 수 없었으며,
그것을 수행하기 위해 그들은
정확하고 풍부한 어휘,
엄격한 추론방법,
경험에 의해 어느 정도 실증된 일반적 행위 명제의 저장고,
그리고 엄정한 \wi{도덕철학}을
보유하고 있었다.
기독교 기록에 나타난 문제들 중에서
그들에게 친숙한 사고 유형에 가까운 것들을
그들이
발견하지 못한다는 것은 불가능한 일일 것이다.
또한 그것들을 취급하는 방식을
그들의 법학적 습관에서 가져오지 않는다는 것도 불가능한 일일 것이다.
로마법에 대한 지식이 충분해서
로마의 형법체계를,
계약과 불법행위로 성립되는 로마의 채권채무관계 이론을,
채무 및 그것을 부담하고 소멸시키고 이전하는 방식에 관한
로마인의 견해를,
포괄적 승계에 의해 개인의 존재가 계속 이어진다는 로마인의 관념을
이해할 수 있는 사람이라면 거의 누구나,
서방 신학의 저 문제들과 잘 어울리는 것으로 드러난 사고의 틀이 어디서 온 것인지,
이들 문제를 진술하는 용어가 어디서 온 것인지,
그 문제의 해결책에 사용된 추론의 유형이 어디서 온 것인지
자신있게 말할 수 있을 것이다.
다만,
서방 사상에 작용하여 들어간 로마법은
옛 로마시의 고법\hanja{古法}도 아니고,
비잔틴 황제들에 의해 잘려나가 축약된 법도 아니며,
그렇다고 근대의 사변적 교리의 기생\hanja{寄生}적 과대성장 속에 거의 파묻힌,
근대 대륙법이라고 불리는 법규칙의 덩어리도 아니었다는
점만은 유념해야 한다.
나는 바로 안토니누스 황조 시대의 위대한 법학자들이
일구어낸 법철학을 말하고 있거니와,
그것은 \wi{유스티니아누스}의 학설휘찬\latin{Pandects}을 통해 지금도 부분적으로
재구성할 수 있는 것이다.
그 체계의 흠을 굳이 들라면,
인간의 법이 추구할 수 있을 것으로 보이는 한계를 넘어선
고도의 우아함, 확실함, 정확함을
목표로 했다는 점
정도가 아닐까 한다.

\para{로마에서 법학의 우위}
영국인들이 자진해서 고백하는,
때로 부끄러워하기는커녕 자랑스러워하는,
로마법에 대한 무지로 인해,
다수의 저명하고 신망있는 영국 학자들조차
제정기 로마의 지적 상태에 관해 도저히 지지할 수 없는
역설적 주장을 내놓는 특이한 결과가 생겨났다.
\wi{아우구스투스} 시대가 마감된 때부터
기독교 신앙에 대한 대중적 관심이 일어나기 전까지
문명 세계의 정신적 에너지가 마비상태에 빠졌다는 명제가,
그 명제의 주장에 아무런 무모함도 없다는 듯이,
서슴없이
주장되어왔다.
하지만
인간 정신이 보유한 모든 힘과 능력을 사용할 수 있도록 하는
사고 영역에는 두 가지---아마도 자연과학을 제외하면
이 두 가지뿐일 것이다---가 있다.
하나는 형이상학으로,
인간 정신이 스스로 즐겨 작동하는 한 한계가 없는 영역이고,
다른 하나는 법학으로,
인간사의 일들과 외연을 같이하는 영역이다.
전술한 바로 그 시기 동안,
그리스어를 말하는 지역에서는 전자가,
라틴어를 말하는 지역에서는 후자가,
몰두의 대상이었다.
알렉산드리아와 동방에서의 사변의 결실에 대해서는 모르겠으나,
로마와 서방은
다른 모든 지적 훈련의 부재를 보상하고도 남을 만한
직업 하나를 수중에 쥐고 있었다고
자신있게 말할 수 있다.
또한 우리가 아는 한,
그것이 이룩한 성취는
그것을 만드는 데 들어간 지속적이고도 배타적인 노력에
충분히
값하는 것이었다.
어쩌면
전문 법률가가 아니라면
법학이 흡수할 수 있는 개인의 지적 능력이 얼마나 큰지
완전히 이해할 수 없을지도 모른다.
그러나 일반인이라도
로마의 집단지성 가운데 이례적으로 큰 몫이
어째서 법학에 의해
독점되었는지
어렵지 않게 이해할 수 있을 것이다.
``장기적으로 볼 때,\origfootnote{%
  앞의 1856년도 <<케임브리지 논문집>>. }
어떤 공동체의
법학적 능숙함은
다른 어떤 학문 분야의 진보와도 동일한 조건에 달려있다.
그중 중요한 것은 한 나라의 지식인 중에 거기에 투입되는 비율과
투입되는 시간의 길이이다.
그런데
학문을 진보시키고 완성시키는 데 기여하는
직·간접적인 원인들이 모두 함께,
12표법부터 두 제국의 분리에 이르기까지 줄곧
로마의 법학에 작용하였거니와,
그것은 불규칙적이거나 간헐적이 아니라
꾸준히 힘이 증가하고 지속적으로 수가 많아지는 양상이었다.
초창기의 지적 훈련이 법의 연구에 바쳐지고 있는 젊은 나라를 상상해보라.
일반화를 위한 의식적 노력이 행해지면서,
일상생활의 관심은 가장 먼저
그것을 일반적 규칙과 포괄적 공식에 포섭하는 것이 된다.
젊은 공동체의 모든 에너지가 바쳐지고 있는 이 분야의 인기는
처음에는 무제한적이다.
하지만 시간이 흐르면서 그것도 시들해진다.
법학이 인간 정신을 독점하는 상황도 깨져간다.
위대한 로마 법학자의 대기실에 아침부터 몰려들던 고객들도 줄어든다.
영국의 법조원\hanjalatin{法曹院}{inns of court}의 학생 수도
수천명대에서 수백명대로 줄어든다.
예술, 문학, 과학, 정치가 그 나라의 지식인 중의 일정 몫을 가져간다.
법실무는 전문가 그룹 내의 것으로 국한된다.
그러나 쪼그라들거나 하찮은 것이 되지는 않거니와,
보수\hanja{報酬}의 측면에서도 그들의 학문의 고유한 매력의 측면에서도
여전히 사람들을 끌어들인다.
이러한 변화의 과정은 영국보다 로마에서 더 현저하게 나타났다.
공화정 말기에 이르면
군대를 통솔하는 특별한 재능을 제외하면
모든 재능 있는 사람들은
법학을 공부한다.
그러나,
영국의 엘리자베스 1세 시대가 그러했듯이,
\wi{아우구스투스} 시대와 더불어
지성의 진보는 새로운 단계를 맞이한다.
주지하듯이 시와 산문에 있어 그 시대의 업적은 대단했지만,
장식용에 불과한 문학의 번영 외에도
자연과학을 정복하려는 새로운 경향도 막 등장하려 했음에
유의해야 한다.
하지만 이 시기는 로마 국가의 정신의 역사가
그후 추구되어온 정신 진보의 일반적인 경로와 달라지는 시기였다.
이른바, 그러나 정확한 묘사인, 로마 문학의 짧은 수명은
여러 가지 요인으로 갑자기 종말을 맞았거니와,
여기서 그 요인들을 분석하는 것은,
비록 부분적으로 추적가능하다 할지라도,
적합치 않을 것이다.
고대 지식인들은 급격히
옛 상태로 되돌아갔고,
로마인들이 철학과 시를 유치한 민족의 장난감으로 경멸하던 시절만큼이나
배타적으로 다시 법학이
재능 있는 사람들에게 적합한 영역으로 각광받았다.
제정기 동안,
법학 분야에 적합한
타고난 능력을 가진 사람들을 끌어들인
외적 요인을 이해하기 위해서는
그의 앞에 놓인 직업의 선택지를 생각해보는 것이
가장 적절할 것이다.
그는 수사학 교사,
전선의 사령관,
또는 온갖 찬사를 쏟아내는 전문 작가가 될 수 있었다.
하지만 그에게 열려있는 그밖의 활동영역으로는
법실무에 종사하는 것이 유일했다.
\hemph{이것}을 통하여 그는
부, 명예, 관직에 접근할 수 있었고,
황제의 자문단\latin{council chamber}\footnote{%
  원수정 시기에 존립한 황제자문단(consilium principis)을 말하는 듯.
  전주정 시기에는 추밀원(consistorium)으로 확대개편된다.
  제10장 중 추밀원 관련 부분 및 각주(\pageref{consiliumprincipis}쪽) 참조.
}---어쩌면 황제의 자리 자체---에도
오를 수 있었다.''

\para{서방 신학에서의 로마법}
법학이 갖는 장점이 그렇게 컸기에
제국의 모든 지역에, 심지어 형이상학이 번성한 지역에도,
법학교들이 존재했다.
비록 황제의 거처가 비잔티움으로 옮아가
동방에서 법학이 발달할 뚜렷한 계기가 되었음에도,
법학은 거기서 경쟁관계에 있는 다른 학문들을 결코 몰아내지 못했다.
법학의 언어는 라틴어였으니,
제국의 동부에서는 외래 방언이었던 것이다.
오직 서방에서만
법학이 야심과 포부를 가진 사람들의 정신적 양식이었을 뿐만 아니라
지적 활동의 유일한 자양분이기도 했다.
로마의 식자층 사이에서는 그리스 철학이
일시적인 유행 이상의 것이 되지 못했다.
동방에 새로운 수도가 건설되고 그후 제국이 둘로 갈라지자,
서방은
그리스적 사변으로부터
더없이 결정적으로
결별하게 된다.
이제 그리스의 문하생에서 벗어나
독자적으로 신학을 궁구하기 시작하자,
그들의 신학은 법적인 관념에 물들고 법적인 용어로 표현되었다.
확실히 서방 신학에서 이러한 법학적 토대는 대단히 깊은 것이다.
그후 아리스토텔레스 철학이라는 새로운 그리스적 이론이
서방에 유입되었고 서방의 고유한 원리들을 거의 전부 덮어버렸다.
그러나
종교개혁 이후 서방은
그것의 영향력을
부분적으로 떨쳐버렸고,
그 자리에 즉각 법학을 가져다 앉혔다.
칼뱅\latin{Calvin}의 종교체계와
아르미니우스파\latin{Arminians}의 종교체계 중
어느 것이 더 법학적 성격이 강한 지는 판가름하기 어렵다.

\para{계약법과 봉건제도}
로마인들이 생산한 이러한 계약법이
근대 계약법에 끼친 막대한 영향력은
성숙한 법학의 역사에 해당하므로
본 논저의 논의대상을 벗어난다.
그것은
볼로냐 대학이 근대 유럽 법학의 기초를 다지면서
비로소
감지되기 시작했다.
그러나
제국이 몰락하기 전에 이미
로마인들에 의해
계약 개념이
완전히 발달했다는 사실은
그보다 훨씬 이른 시기에
중요한 의미를 갖게 된다.
누차 강조했듯이
\wi{봉건제}도는 옛 만족\hanja{蠻族}들의 관습과 로마법이
결합한 것이었다.
다른 설명은 지지될 수 없거나 심지어 이해조차 불가능하다.
봉건시대 초창기의 사회 형태는
원시 문명의 사람들이 어디서나 보여주는
결합의 형태와 별반 다르지 않았다.
봉건관계는 일종의 유기적으로 완전히 결합된 동료관계로서,
재산적 권리와 신분적 권리가 불가분 혼재되어 있었다.
그것은 인도의 \wi{촌락공동체}와 많은 공통점을 가지며,
스코틀랜드 산악지대의 씨족과도 많은 공통점을 가진다.
그러나 그것은 여러 문명의 초기에 자생적으로 형성된 결합관계와는 다른
특수한 성질도 가진다.
실로 원시적 공동체는 명시적 규칙이 아니라 감정에 의해,
아니 어쩌면 본능에 의해 결합된다.
또한 동료관계에 새로 들어오는 자는
이러한 본능에 부합하게
짐짓
자연적 혈연관계를 공유한다고 내세움으로써 편입되는 것이다.
그러나 초창기의 봉건적 공동체는 단순한 감정에 의해
결합되는 것도 아니었고
의제\hanja{擬制}에 의해 충원되는 것도 아니었다.
그들을 결속시키는 것은 계약이었으니,
그들은 계약을 맺음으로써 새로운 성원을 얻었던 것이다.
주군과 가신의 관계는 원래 명시적 계약을 통해 설정되었다.
\hemph{\wi{충성서약}}\latin{commendation}이나
\hemph{\wi{수봉}}\hanjalatin{受封}{infeudation}을 통해
동료관계에 편입되려는 자는
그가 받아들여지는 조건을 분명히 알 수 있었다.
따라서 \wi{봉건제}도가 원시 민족들의 순수한 관행과 다른
주된 차이점은 계약이 차지하는 부분에 의해서인 것이다.
주군은 가부장의 성격을 다분히 가지고 있었으나,
그의 대권\hanja{大權}은
수봉시 합의된 명시적 조건에서 기인하는 다양하게 설정된 관습에 의해
제한되었다.
그리하여 봉건사회를 진정한 원시 공동체로 분류할 수 없는
주요 차이들이 발생하게 된다.
봉건사회는 훨씬 더 지속적이었고 훨씬 더 다양했다.
명시적 규칙은 본능적 습관에 의해 파괴되기 어렵다는 점에서
그것은
훨씬 더 지속적이었다.
봉건사회의 기초인 계약은
세부적인 상황에 따라
그리고
토지를 맡기거나 양여하는 자의 원하는 바에 따라
얼마든지 달라질 수 있다는 점에서
그것은
훨씬 더 다양했다.
이 마지막 점은
근대 사회의 기원에 관한 오늘날 우리의 통속적인 견해가
얼마나 잘못된 것인지를 알려주는 데 도움이 될 수 있다.
근대 문명의 불규칙하고 다양한 모습이
게르만 민족들의 지나치게 많은 변칙적인 풍속 탓이라고 하면서,
이를 지루하리만치 틀에 박힌 로마제국의 그것과 대비시키는 일이 흔히 있다.
그러나 진실은
로마제국이 이 모든 불규칙성의 원인인 저 법개념을
근대 사회에 물려주었다는 데에 있다.
만족\hanja{蠻族}들의 관습과 제도들의 가장 두드러진 특징 하나를 들자면,
그것은
그것들이 무척 단조로웠다는 것이다.

