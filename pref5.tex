\chapter*{제5판 서문}

추가적으로 연구하고 숙고해보아도
본 저서에서 다룬 저자의 견해의 대부분은
변경할 필요가 없다는 결론에 이르렀지만,
제1장 가운데
관습법의 기원이라는, 난해하고도 여전히 모호한 주제에 관한 의견만은
교정과 수정이 필요하다는 확신을 가지게 되었다.
필요한 교정과 수정의 일부를
저자의
<<동·서양의 촌락공동체>>\latin{Village Communities in the East and West}라는
저서에서 제시해두었다.\footnote{%
  \latin{%
  Henry Sumner Maine,
  \textit{Village-communities in the East and West: With Other Lectures,
  Addresses, and Essays}, 3rd ed.,
  London: John Murray, 1876(1871),} 제3장을 말하는 듯하다.
  관련하여 벤담 및 오스틴의 명령설적 법개념으로는
  전통사회의 관습법을 설명하기 곤란하다는 주장으로
  \latin{Henry Sumner Maine,
  \textit{Lectures on the Early History of Institutions}, 4th ed.,
  London: John Murray, 1885(1874), 제12--13장}도 참조. }

\begin{flushright}
H. S. M.
\end{flushright}

\begin{footnotesize}
런던: 1878년 12월.
\end{footnotesize}

