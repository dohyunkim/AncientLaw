\chapter{제10판 서문}

본 저서에서 전개된 법발달의 이론은 일반적으로 널리 받아들여졌다.
그러나
제5장 ``원시사회와 고대법''에서
저자는,
\hyperlink{cyclops}{110페이지}에 인용된\footnote{%
  페이지 숫자는 원문의 숫자이며 본 번역본의 페이지와 같지 않다.
  이하 마찬가지다.
}
호메로스의 싯귀로써 생생하게 묘사된,
흔히 가부장제 국가로 불리는
사회상태보다
훨씬 더 원시적인 사회상태의 존재를 보여주는 연구에 대해서는
충분히 다루지 못했다.
\hyperlink{contemporary}{106페이지}에서 저자는
``당대의 관찰자들이 그들보다 문명의 진보 수준이 낮은 사회를 기술한 것''이
사회의 원초적 상태에 관해 특별히 귀중한 증거를 제공해줄 수 있다고
말한 바 있다.
그런데 실로,
본 저서가 1861년에 처음 출간된 이후,
야만적인 또는 아주 미개한 사회에 대한 관찰이
저자가 법의 시초라고 불렀던 것과는 사뭇 다른,
경우에 따라서는 그보다 훨씬 더 오래된
모습의 사회조직을 밝혀내고 있다.
이 주제는 엄밀히 말하면 본 저서의 대상을 넘어선다.
하지만
저자는
<<초기의 법과 관습>>\latin{Early Law and Custom (Murray, 1883)}이라는
저서\footnote{%
  \latinmarks
  Henry Sumner Maine,
  \textit{Dissertations on Early Law and Custom:
  Chiefly Selected from Lectures Delivered at Oxford},
  London: John Murray, 1883.
}에
포함된 논문
``원시사회의 이론들''\latin{Theories of Primitive Society}에서
이러한 보다 최근의 연구에 대한 저자의 견해를 제시해놓았다.

\begin{flushright}
H. S. M.
\end{flushright}

\begin{footnotesize}
런던, 1884년 11월.
\end{footnotesize}

