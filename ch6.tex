\chapter{유언상속법의 초기 역사}

\para{교회의 영향}
역사적 연구방법이
법에 관한
기존에 널리 퍼진 연구방법에 비해 우수하다는 것을
증명하는 시도가
영국에서
행해진다면,
유언법\latin{testament; will}보다 더 좋은 예를 보여주는 법분야는 없을 것이다.
그러한 능력은 유언법의 긴 역사와 오랜 지속성에 빚지고 있다.
역사의 초창기의 유언법에서
우리는
아주 유년기의 사회상태를 발견하거니와,
그것은 어느 정도 노력을 기울여야만
그 고대적 형태를 깨달을 수 있는 개념들로 둘러싸여있다.
반면, 진보의 반대편 극인 지금의 우리는
동일한 개념들이
현대적인 용어와 사고습관에 의해 감추어진 것에
불과한
법관념들 가운데에 서있거니와,
따라서
우리의 일상적 정신에 속하는 관념들을 분석하고 조사할 필요성을
인식하는 또 다른 종류의 어려움에 처한다.
이들 양 극 사이의 유언법의 발달을
우리는
자못 뚜렷하게 추적할 수 있다.
유언법의 역사는
다른 어떤 법분야의 역사보다
봉건제 탄생 시기의 단절을 훨씬 덜 보여준다.
물론
고대사와 근대사의 분리에 의해 촉발된 단절,
즉 로마제국의 붕괴에 의해 촉발된 단절이
모든 법영역에서
지나치게 과장되어온 것은 사실이다.
나태한
많은 학자들은
혼돈의 여섯 세기 동안 착종\hanja{錯綜}되고 희미해진 관련성의 실타래를
찾는 수고를 하려들지 않았고,
인내와 노력이 부족하지 않은 다른 학자들은
자기 나라 법체계에 대한 헛된 자부심에 가득차
로마법에 대한 감사의 뜻을 고백하지 않아 잘못된 길을 갔다.
그러나 이러한 좋지 못한 힘들이 유언법 영역에는 거의 영향력을 발휘하지 못했다.
스스로도 인정하듯이 만족\hanja{蠻族}들은 유언이라는 개념을 알지 못했다.
최고의 학자들이 이구동성으로 하는 말에 따르면,
만족들의 법전 중에
원래 거주지에서 따르던 관습과 그후
로마제국 변방의 정착지에서 따르던 관습을 모은 법전에서는
유언 개념이 흔적조차 발견되지 않는다.
그러나 곧이어 로마속주의 주민들과 혼합되면서, 그들은
로마법으로부터, 처음에는 부분적으로 나중에는 전면적으로,
유언 개념을 수용했다.
이러한 신속한 동화에는 교회의 영향력이 크게 작용했다.
일찍이 교회권력은 몇몇 이교도 사원들이 누리던,
유언을 보관하고 등록하는 특권을 물려받았거니와,
일찍부터 종교단체가 취득하는 세속적 재산은
거의 전적으로 사적인 유증\hanja{遺贈}에 기인한 것이었다.
그리하여 초기의 지방 종교회의들의 결정에는
유언의 신성함을 부인하는 자들에 대한 파문이 지속적으로 등장한다.
여기 영국에서도,
다른 법 영역에는 존재한다고 여겨지기도 하는
역사적 단절이
유언법의 역사에서는, 널리 인정되듯이, 방지되었으니, 그 원인 중에
가장 주된 원인은 교회였음이 확실하다.
유언 사건에 대한 재판권은 교회법원에 주어졌거니와,
교회법원은 유언 사건에, 항상 현명한 것은 아니지만, 로마법 원리를 적용하였다.
보통법법원도 형평법법원도,
교회법원의 결정에 따라야 할 적극적 의무는 없었지만,
교회법원의 법적용 과정에서 이미 형성된 법규칙 체계의 강력한 영향력으로부터
벗어날 수는 없었다.
인적재산\hanjalatin{人的財産}{personalty}%
\footnote{영미법에서 동산과 채권을 포괄적으로 지칭하는
`인적재산'(personal property; personalty)은
봉건적 토지보유권에서 유래하는 `물적재산'(real property; realty)과
대비되는 개념이다.}%
의 유언상속에 관한 영국법은
로마 시민들의 상속을 규율하던 제도의 변형된 형태가 된 것이다.

\para{고대의 유언}
이 주제의 역사적 탐구방법에 의한 결론과
역사의 도움없이 단지
일견\latin{primâ-facie} 우리가 받는 인상만을 분석하는 방법에 의한
결론 사이에
커다란 차이가 있음은 그리 어렵지 않게 지적할 수 있다.
유언이란 것의 대중적 개념에서 출발하여,
아니면 심지어 법적 개념에서 출발하더라도,
거기에 어떤 속성이 반드시 수반된다는 것을 생각해내지 못할 사람은
아무도 없을 것이다.
예컨대, 유언은 반드시 \hemph{사망시에만}\latin{at death only}
효력이 생긴다는 것, 유언은 \hemph{비밀}\latin{secret}이라서
그것에 이해관계를 가지는 사람에게 알려져서는 안 된다는 것,
유언은 \hemph{철회가능}\latin{revocable}해서
언제든 새로운 유언에 의해 번복될 수 있다는 것 따위를 거론할 수 있다.
하지만 나는 이러한 속성들 중 어느 것도 유언의 속성이 아니었던 시대가
있었다는 것을 보여줄 수 있을 것이다.
우리의 유언의 직접적 선조였던 유언은 처음에는
작성 즉시 효력이 발생했고, 비밀도 아니었고, 철회도 불가능했다.
사실,
법제도 중에서
인간의 의도를 담은 문서에 의해 사후\hanja{死後}의 그의 재산 처분이
좌우된다는 것만큼 복잡한 역사적 작용의 산물인 것은 거의 없을 것이다.
유언이
위에 언급한 속성들을 얻게 되는 것은
아주 천천히 그리고 점진적으로 이루어졌다.
그것은
우연이라 할 만한 사건들이,
적어도 법의 역사에 영향을 준 것이 아닌 한
어쨌든 지금 우리의 관심대상은 아닌 사건들이,
원인이 되어 그 압력으로 이루어졌다.

\para{유언이라는 자연권}
법이론이 지금보다 풍부했던 시절,
법이론이 대체로 근거없고 미성숙한 상태에 머물러 있는 것은 사실이지만
그럼에도 불구하고 어떤 일반화도 없이 단지
경험적으로만 법을 추구하던 열등하고 조야한 상태는 벗어난 시절,
유언의 속성에 대해 우리가 쉽게 가지는 표면적인 직관을 설명하기 위해
널리 사용된 방식은
그 속성들이 유언에 자연적이라고 말하는 것,
또는, 이 용어를 끝까지 밀고나가,
자연법에 의해 유언에 주어진다고 말하는 것이었다.
생각건대,
일단 이 모든 속성들이
역사의 기억 속에 그 기원을 가지고 있음을 알게 되면
아무도 이런 법리를 유지하려 하지 않을 것이다.
또한,
우리 모두가 사용하고 있고
그것 없이는 어찌할 바를 잘 모르는 표현 형태 안에는
이 법리를 낳은 이론의 흔적들이
여전히 남아있는 것이다.
나는 17세기 법문헌에 자주 등장하는 명제를 가지고 이를 보여주고자 한다.
당시의 법학자들은 흔히 유언권한을 자연법상의 것이라고,
자연법이 부여한 권리라고 주장했다.
그들의 가르침은,
비록 모든 사람들이 연관성을 한눈에 알아보는 것은 아니지만,
재산의 사후\hanja{死後}처분을 지시하고 통제하는 권한이
소유권 자체의 필연적이고도 자연적인 귀결이라고
인정하는 사람들의 주장으로 실제 이어졌다.
전문기술적인 법을 공부하는 법학도라면 누구나,
학파 간에 서로 다른 언어의 옷을 입고 있더라도,
이와 동일한 견해를 만나보았을 것이다.
그것은, 이 법분야의 논리에 따르면,
유언\latin{ex testamento}상속을
망자\hanja{亡者}의 재산이 일차적으로 따라야 할 이전 방식으로 취급하고,
이어서 무유언\hanjalatin{無遺言}{ab intestato}상속은
사망한 소유권자가 실수로 또는 불운으로
행하지 않은 것만을 처리하기 위한 입법자의 부수적 대응책으로
설명하는 것이다.
이러한 견해는 유언 처분이 자연법상의 제도라는 보다 간결한 법리가 확장된
형태에 불과한 것이다.
물론, 자연과 자연법을 숙고했던 근대 학자들의 관념 연관의 범위를
도그마틱하게 단정짓는 것은 결코 안전한 일이 아닐 것이다.
그러나 유언권이 자연권이라고 인정하는 사람들의 대다수는
그것이 사실상 보편적이라는 것을 의미하거나, 아니면
그것이 인간의 최초의 본능과 충동에 의해 인정되었다는 것을 의미하고 있다고
나는 믿는다.
전자의 입장에 관해서는,
나폴레옹법전\latin{Code Napoléon}에 의해 유언권이 심하게 제약되고 있고
프랑스법전을 모델로 삼은 법체계들이 속속 증가하고 있는 이 시대에,
그것을 명시적으로 주장한다면
이는 진지하게 내세울 수 있는 주장이 못된다는 것이 내 생각이다.
후자의 주장에 관해서는,
그것이 법의 초기 역사에서 충분히 확인된 사실에 반한다고 해야 할 것이다.
모든 토착적 사회에서는 유언이 허용되지 \hemph{않는},
아니, 생각조차 되어보지 못한 법상태가
소유권자의 단순한 의사가 다소간의 제약 하에
그의 혈연 친족들의 요구에 우선하게 되는 후대의 발달된 법상태에
일반적으로 선행하였다고
나는 감히 주장하고 싶다.

\para{유언의 성질}
유언의 개념은 그것 자체만으로 이해될 수 있는 것이 아니다.
그것은 일련의 개념들 중 하나일 뿐이며, 그들 중 첫 번째 것도 아니다.
유언 자체는 유언자의 의사가 선언되는 수단일 뿐이다.
생각건대,
이러한 수단을 논의하기에 앞서,
우선 예비적으로 몇 가지 논점을 밝혀둘 필요가 있다.
가령, 망자의 사망으로 그에게서 이전되는 것은 무엇이며 어떤 종류의
권리 혹은 이익인가? 그것은 누구에게 어떤 형태로 이전되는가?
망자가 자기 재산의 사후\hanja{死後} 처분을 통제할 수 있는 것은 어째서인가?
따위가 그것들이다.
유언이라는 관념에 기여하는 다양한 종속적 개념들은 다음과 같이 법기술적인
용어로 표현될 수 있다.
유언이란 상속재산의 이전\hanja{移轉}을 정하는 수단이다.
상속이란 포괄적 승계\latin{universal succession}의 일종이다.
포괄적 승계란 포괄적 재산\latin{universitas juris}의 승계, 즉
권리와 의무의 총체를 승계하는 것이다.
그리하여 우리는, 역순으로,
무엇이 포괄적 재산인지, 무엇이 포괄적 승계인지,
그리고 상속이라 불리는 포괄적 승계의 형식은 무엇인지를 탐구해야 한다.
나아가, 내가 말한 논점들과 어느 정도 무관한, 그러나
유언이라는 주제를 마치기 전에 해결해야 할, 두 가지 논점이 더 있다.
어떻게 해서 상속재산이 유언자의 의사의 통제대상이 되었을까,
그리고 상속재산을 통제하는 수단의 성질은 무엇인가, 하는 것이 그것들이다.

\para{포괄적 재산}
첫 번째 문제는 ``포괄적 재산'',
즉 권리와 의무의 총체\paren{혹은 묶음}에 관한 것이다.
포괄적 재산이란
특정한 한 시점에 특정한 한 사람에게 속한다는 단일한 상황으로 결합된
권리와 의무의 집합물을 말한다.
말하자면 그것은 특정 개인의 법적인 옷\hanja{[衣服]}인 것이다.
몇몇 권리와 몇몇 의무를 하나로 묶는다고 해서 되는 것이 아니다.
오직 특정인의 모든 권리와 모든 의무를 하나로 묶어 성립할 수 있을 뿐이다.
다수의 소유권, 통행권, 유증에 대한 권리, 특정한 급부 채무, 금전채무,
불법행위 손해배상 채무 따위의 모든 법적 권리와 의무를 하나로 묶어
포괄적 재산을 이루게 하는 힘은
이를 행사하고 이행할 수 있는 어떤 개인에게
이것들이
속해있다는
\hemph{사실}에 있다.
이러한 \hemph{사실}이 없으면 포괄적 재산은 존재할 수 없다.
`포괄적 재산'이라는 용어는 고전기의 것이 아니지만,
그 관념은 오로지 로마법에 빚지고 있다.
그것을 이해하는 것도 결코 어렵지 않다.
우리들 각자가 바깥 세상에 대해 갖는 모든 법률관계의 집합을
하나의 개념 아래에 끌어모으면 된다.
그 성격이나 성분이 어떠하든, 이들이 모여 포괄적 재산을 이루는 것이다.
권리뿐만 아니라 의무도 포함된다는 점만 명심한다면,
개념을 형성하는 데 있어 실수할 위험은 거의 없다.
의무가 권리보다 더 많을 수도 있다.
어떤 이는 적극재산보다 채무가 더 많아서,
그의 법률관계의 집합을 금전적으로 평가하면 소위 지급불능 상태인 것으로
판명될 수도 있다.
그렇다고 그를 둘러싸고 있는 권리와 의무의 총체가
``포괄적 재산''이 아닌 것은 아니다.

\para{포괄적 승계}
다음으로 ``포괄적 승계''가 문제된다.
포괄적 승계란 포괄적 재산을 승계하는 것이다.
어떤 사람이 다른 사람의 법적인 옷을 입어,
그의 모든 책임을 부담하고 그의 모든 권리를 가질 때 이런 일이 일어난다.
포괄적 승계가 진정하고 완전한 것이 되기 위해서는,
이전\hanja{移轉}이,
법학자들의 표현을 빌면, `일거\hanja{一擧}에'\latin{uni ictu}
일어나야 한다.
물론, 어떤 이가 다른 사람의 권리와 의무의 전부를 여러 번에 걸쳐,
가령 순차적 매수를 통해서, 취득할 수 있다.
혹은 서로 다른 자격으로, 가령 일부분은 상속인으로,
다른 일부분은 매수인으로, 나머지는 수유자\hanja{受遺者}로서 취득할 수도 있다.
그러나 이렇게 해서 얻은 권리와 의무의 집합이 사실상 특정인의 법인격 전부라
할지라도 이러한 취득은 포괄적 승계가 아니다.
진정한 포괄적 승계가 되기 위해서는,
권리와 의무의 총체의 이전이 \hemph{동일한} 시점에 이루어져야 하고
수령인이 \hemph{동일한} 법적 자격에서 넘겨받는 것이라야 한다.
``포괄적 재산'' 개념과 마찬가지로 포괄적 승계의 개념도
법학에서 항상 발견되는 개념이다.
다만 영국법에서는 권리를 취득하는 사뭇 다양한 자격으로 인해,
특히 ``물적재산''\latin{realty}과 ``인적재산''\latin{personalty}이라는
영국 재산법의 두 개의 큰 영역 간의 구분으로 인해, 그 개념이 흐려져있을 뿐이다.
하지만 파산관재인이 파산자의 전 재산을 승계하는 것은
일종의 포괄적 승계에 해당하거니와,
다만
파산관재인은 그 재산의 한도 내에서만 채무를 지불하므로
원래 개념의 변형된 형태일 뿐이라 할 것이다.
만약 어떤 이가 어떤 다른 사람의 \hemph{모든} 채무를 지불한다는 조건으로
그의 \hemph{모든} 재산을 양수하는 일이 영국에도 흔한 일이라면,
그러한 양수는 초기 로마법이 알고 있던 포괄적 승계와 정확히 일치할 것이다.
어떤 로마 시민이 아들을
\hemph{자권자입양}\hanjalatin{自權者入養}{adrogate}할 때,
즉 가부장권에 복속하지 않는 남자를 양자로 입양할 때,
그는 입양되는 아들의 재산을 \hemph{포괄적으로} 승계했다.
모든 재산을 취득했을 뿐만 아니라 모든 의무에 대해서도 책임을 졌던 것이다.
초기 로마법에는 포괄적 승계의 몇몇 다른 형태들도 있었으나,
무엇보다 가장 중요하고 가장 지속적인 행태는 지금 우리의 관심대상인 것,
즉 상속\latin{haereditas}이었다.
상속은 사망으로 발생하는 포괄적 승계였다.
이때의 포괄승계인은 `상속인'\latin{Haeres}이라고 불렸다.
그는 망자의 모든 권리와 모든 의무를 동시에 물려받았다.
그는 즉시 망자의 법인격 전체를 옷으로 입었다.
유언으로 지명된 상속인이든, 아니면
무유언 상속인이든, 상속인이라는 지위에는 아무런 차이가 없었다는 점은
부연할 필요가 없을 것이다.
`상속인'이라는 용어는 유언상속인에 대해서도, 무유언상속인에 대해서도
똑같이 사용되었다.
상속인이 되는 것은 그가 갖는 법적 성격이 무엇이냐와
무관하기 때문이다.
유언에 의해 상속인이 되든, 무유언상속의 상속인이 되든,
망자의 포괄승계인은 그의 상속인인 것이다.
그러나 상속인은 반드시 한 사람이어야 할 필요는 없었다.
법적으로 하나의 단위로 간주되는 일군의 사람들이
\hemph{공동상속인}이 되어 승계할 수 있었다.

\para{포괄승계인}
이제 로마인들의 상속의 정의를 인용해보자.
독자들은 각각의 용어를 완전히 이해할 수 있을 것이다.
``상속은 망자가 가졌던 포괄적 법적 지위를 승계하는 것이다.''\latin{Haereditas
est successio in universum jus quod defunctus habuit.}
비록 망자의 육신은 소멸하지만,
그의 법인격은 살아남아 그와의 동일성이
\paren{적어도 법의 관점에서는}
이어지는 상속인 또는 공동상속인들에게
전달된다는 뜻이다.
영국법에서 유언집행인\latin{executor} 또는
유산관리인\latin{administrator}이
인적재산에 관한 한 망자의 대표자로 이해되는 것은
여기서 유래한 이론으로 예시될 수 있을 것이다.\footnote{유언집행인은
유언장에 의해 지명되고, 유산관리인은 유언이 없는 경우 법원에 의해
선임된다. 인적재산은 이들이 일단 취득하여 관리^^b7집행하고,
물적재산은 상속인이나 수유자가 직접 승계한다.}
그러나 예시는 몰라도 설명까지 되는 것은 아니다.
후기 로마법의 견해에 의하더라도
망자와 상속인 간에는 긴밀한 견련\hanja{牽連}관계가 필요했거니와,
이는 영국법상의 대표자에게는 요구되지 않는 것이다.
또한 원시법에서는 모든 것이 승계의 연속성을 지향하고 있었다.
유언자의 권리와 의무를 즉시 넘겨받을 상속인이나 공동상속인들이
유언장에
지시되지 않았다면,
그러한 유언장은 전부 무효였다.

\para{고대의 상속}
후기 로마법과 마찬가지로 근대 유언법에서도
가장 중요한 목적은 유언자의 의사를 집행하는 일이다.
고\hanja{古}로마법에서 그에 상응하는 중요성을 띠는 것은
포괄적 승계를 수여하는 일이었다.
우리 눈에 전자는 상식이 명령한 원리로 보이지만,
후자는 한가한 변덕의 발로로 보이기 십상이다.
하지만 후자가 없었다면 전자도 존재할 수 없었을 것이라는 점은
유사한 다른 명제들만큼이나 여기서도 확실하다.

\para{원시 사회}
이 역설처럼 보이는 것을 풀기 위해서는,
그리고 내가 보이고자 하는 관념의 연쇄를 보다 명백히 보여주기 위해서는,
바로 앞 장에서 행했던 탐구의 결론을 빌어와야 할 것 같다.
거기서 우리는 사회의 유년기에 나타나는 보편적인 특징 하나를 발견했다.
사람들은 개인이 아니라 언제나 어떤 집단의 구성원으로
간주되고 취급되었다.
누구나 우선은 시민이었다.
다음으로 시민으로서의 그는
신분집단---그리스의 귀족\latin{aristocray}이나 평민\latin{democracy},
혹은 로마의 귀족\latin{patrician}이나 평민\latin{plebeian},
혹은 발달과정에서 불행한 운명이 할퀴고 간 타락한 사회에서는
카스트---에 소속되었다.
그 다음으로 그는 씨족\latin{gens; house; clan}의 구성원이었다.
그리고 마지막으로 \hemph{가족}의 구성원이었다.
이 마지막 것은 그가 서있는 가장 좁은 관계이자 가장 친밀한 관계였다.
역설적으로 보일지 몰라도 그는 \hemph{그 자신}으로,
독립적 개인으로, 간주되지 않았다.
그의 개인성은 가족에 의해 흡수되었다.
전술한 원시사회의 정의를 재차 강조하자면,
그것의 단위는 개인이 아니라,
실제의 또는 의제\hanja{擬制}의 혈연 관계에 기초한 사람들의 집단이었다.

\para{가족이라는 단체}
저발전 사회의 이러한 특성에서 우리는 포괄적 승계의 최초의 흔적을 발견한다.
근대국가의 구조와 달리, 원시시대의 국가는
다수의 작은 전제\hanja{專制}적 정부들로 구성되어 있었고,
그 각각은 다른 것들로부터 완전히 독립적이었으며,
각각은 단일한 군주의 대권\hanja{大權}에 의해 절대적으로 지배되고 있었다고
기술\hanja{記述}함이 마땅하다.
하지만, 비록 이 가부장---아직 로마의 가부장\latin{pater-familias}이라고
해서는 안 된다---이 광범위한 권리를 가지고 있었지만,
의심할 여지 없이 그는 수많은 의무도 마찬가지로 부담하고 있었다.
그가 가족을 지배했다면, 그것은 가족을 위한 것이었다.
그가 가족의 물건의 주인이었다면, 그것은 그의 자식들과 친족들을 위한
수탁자\hanja{受託者}로서 보유하는 것이었다.
그가 특권이나 높은 지위를 가졌다면, 그것은
그가 지배하는 작은 국가와의 관계에 의해 그에게 부여되는 것이 전부였다.
가족은 실로 단체였고, 그는 그 단체의 대표자였다.
어쩌면 그 단체의 공직자였다고도 말할 수 있을 것이다.
그는 권리를 향유하고 의무를 부담했으나,
동료시민들이 보기에는, 그리고 법의 관점에서는,
이들 권리와 의무는 그의 것인 동시에 집합체의 것이기도 했다.
이러한 대표자의 사망으로 어떤 일이 발생하는지 잠시 생각해보자.
법의 관점에서는, 국가 정무관의 관점에서는,
가내 권위자의 사망은 전혀 중요하지 않은 일이었다.
가족이라는 집합체를 대표하고 국가법정에서 일차적으로 책임지는 사람이
이제 다른 이름을 가진다는 것, 그것이 전부였다.
사망한 가\hanja{家}의 수장에게 주어졌던 권리와 의무는
연속성이 끊어지지 않은 채 그의 승계인에게 주어진다.
사실 이들 권리와 의무는 가족의 권리와 의무였고,
가족은 단체 특유의 성질---결코 죽지 않는 성질---을 가지기 때문이다.
채권자들은 과거의 수장\hanja{首長}에 대한 것처럼
새로운 수장에 대해서도 동일한 구제수단을 가진다.
가족은 여전히 존속하고, 가족의 책임도 완전히 동일하기 때문이다.
가족이 가지던 모든 권리는 수장의 사망 전과 똑같이 사망 후에도
가족에게 남는다.
다만, 이제 단체는 조금 다른 이름으로---이렇게 정확한 법기술적인 용어를
저 초기 사회에 대해서도 사용할 수 있다면---\hemph{소송}을 제기해야 할
따름이다.

\para{가족과 개인}
법제사의 전 역사를 추적해야만,
가족이 어떻게 해서 그것을 구성하는 원소들로 점차 느리게
해체되어갔는지---어떤 비가시적인 점진적 변화에 의해
개인의 가족에 대한 관계가,
그리고 가족의 가족에 대한 관계가,
개인의 개인에 대한 관계로 대체되어갔는지---를
이해할 수 있다.
지금 우리가 살펴볼 논점은,
혁명이 완수된 후에도,
가부장의 자리를 정무관이 거의 떠맡은 후에도,
국가법정이 가내법정을 대체한 후에도,
사법당국이 다루는 권리와 의무의 체계 전체에는 여전히
낡은 특권의 영향이 남아있었고 그 반향이 구석구석을 물들이고 있었다는 것이다.
로마법에서 유언상속이나 무유언상속의 첫 번째 요건으로 강조되었던
포괄적 재산의 이전은 옛 사회구조의 특징이었고
새로운 발달단계와는 진정한 또는 적합한 결합을 이루지 못함에도 불구하고,
인간의 정신은 새로운 사회형태에서 옛 형태를 떨쳐버릴 수 없었음이
거의 틀림없어 보인다.
어떤 사람의 법적 존재가 그의 상속인이나 공동상속인단\hanja{團}에게
연장된다는 것은
\hemph{가족}의 성격이 의제\hanja{擬制}에 의해 \hemph{개인}에게 부여된다는 것,
그 이상도 이하도 아니다.
단체의 승계는 어디서나 있을 수밖에 없거니와, 가족은 단체였다.
단체는 죽지 않는다.
개인 구성원의 사망은 집합체의 집단적 존재에는 아무런 차이도 가져오지 못하고,
그것의 법적 측면, 즉 그것의 권한과 책임에도 아무런 영향을 주지 못한다.
이제 로마법상의 포괄적 승계 개념에서는
단체의 이 모든 속성들이 개인 시민에게
부여된 것으로 보인다.
그의 물리적 죽음은 그가 가졌던 법적 지위에 아무런 영향도 주지 못했다.
이는 그의 지위를
가족이라는 것의 유추\hanja{類推}에 가능한 한 가까운 것으로 만드는
원리에 기초한 것으로 보인다.
물론 가족은 단체의 성격을 가지고 있어서 물리적으로 소멸하지 않았다.

\para{단독법인}
포괄적 승계를 구성하는 개념들 간의 관계의 본질을 이해함에 있어
적지 않은 수의 대륙의 법학자들이
큰 어려움을 겪고 있는 것으로 보이며,
그들의 법철학의 주제 중에 이것만큼
일반원리로서의 가치를 거의 갖고 있지 못한 것도 없는 것 같다.
하지만 영국의 법학자들은 지금 우리가 다루고 있는 관념을 분석하는 데
실패할 위험이 없다고 할 것이다.
모든 법률가들이 익히 알고 있는 영국법상의 의제\hanja{擬制} 하나를 가지고
그것을 해명할 수 있기 때문이다.
영국의 법률가들은 법인을 집합법인\latin{corporation aggregate}과
단독법인\latin{corporation sole}으로 구분한다.
집합법인은 진정한 단체이다.
하지만 단독법인은 의제를 통해 단체의 속성이
주어지는 개인에 불과하거니와,
이 개인은 연속적으로 등장하는 개인들의 일원이다.
국왕이나 교구목사를
단독법인의 예로 드는 일은 굳이 필요치 않을 것이다.
이 자리가 갖는 권한은 그 자리를 수시로 차지하는 특정인과 분리하여 취급된다.
또한 이 권한은 영구적이므로, 그 자리를 차치하는 일련의 개인들은
단체의 제일가는 속성---영구성---의 옷을 입는다.
옛 로마법 이론에서 개인의 가족에 대한 관계는
영국법의 법리에서 단독법인이 집합법인에 대해 갖는 관계와 정확히 일치한다.
관념들의 파생관계와 연합관계가 완전히 동일하다.
실로, 로마 유언법을 가르칠 목적으로
각 개인 시민은 단독법인이었다고 영국인들에게 말한다면,
영국인들은 상속의 개념을 완전히 이해할 뿐 아니라,
그것이 어떤 생각에서 기원했는지에 대한 실마리도 얻어낼 수 있을 것이다.
국왕은 단독법인이어서 죽지 않는다는 것이 영국인들의 공리\hanja{公理}이다.
국왕의 권한은 즉시 그의 계승자에 의해 채워지고,
통치권의 연속성은 단절되지 않는다.
로마인들에게도
권리와 의무의 이전으로부터 사망의 사실을 제거하는 것은
똑같이 단순하고 자연스러운 과정이었을 것이다.
유언자는 그의 상속인 또는 공동상속인단 속에 여전히 살아있었다.
법적으로 그는 그들과 동일인이었다.
만일 누군가의 유언장이, 그것의 해석에 의해서라도,
그의 현실적 존재와 사후\hanja{死後}적 존재를 결합시키는 원리를
위반하는 것이라면,
법은 그러한 흠결 있는 유언장을 무효로 선언하고
그의 혈연 친족들에게 상속권을 부여했다.
이 경우 혈연 친족들의 상속능력은 법 자체에 의해 주어진 것이지,
잘못 작성되었을 수 있는 유언장에 의한 것이 아니었다.

\para{무유언상속}
로마 시민이 유언 없이 사망하거나 유효한 유언을 남기지 못한 경우,
조금 뒤에 언급할 순위에 따라 그의 자손들이나 친족들이 상속인이 되었다.
상속인 또는 상속인단은 단순히 망자를 \hemph{대표하는} 것이 아니라,
좀 전에 서술한 이론에 따라 그의 시민적 삶, 그의 법적 존재를 계속
\hemph{이어갔다}.
이런 결과는 유언에 의해 상속의 순위가 정해지더라도 똑같이 발생한다.
그러나 망자와 그 상속인 간의 동일성 이론은 분명 그 어떤 유언의 형식보다도,
그 어떤 유언법의 단계보다도, 더 오래된 것이다.
실로, 이 주제를 파고들면 들수록 점점 더 강하게 우리를 압박해오는
의문 하나를 제기할 적절한 때가 된 것 같다.
포괄적 승계에 관련된 저 중차대한 관념이 없었더라면 도대체
\hemph{유언}이라는 것이 등장할 수 있었을 것인가 하는 의문이 그것이다.
오늘날 유언법에 적용되는 원리는
근거는 없지만 그럴 듯해 보이는 다양한 철학적 가설에 기초하여 설명될 수 있다.
그것은 근대사회의 모든 부분과 얽혀있고, 일반적 공리\hanja{功利}라는
사뭇 폭넓은 근거에서 정당화되고 있다.
하지만,
오늘날
기존 제도를 유지하게 하는 저 근거들이
그 제도의 기원을 가져왔던 감정과 반드시 같을 것이라는
인상이야말로
법학에 있어 잘못된 생각의 큰 원천이 되고 있다는
경고는 아무리 반복해도 지나치지 않을 것이다.
확실히,
옛 로마 상속법에서 유언의 관념은
사람이 상속인의 인격 속에서 사후에도 존재한다는 이론과
불가분 혼합되어 있었던, 아니 어쩌면 일체화되어 있었던 것이다.

\para{초기 로마의 유언, 로마와 인도의 사크라}
포괄적 승계의 개념은 법학에 굳건히 뿌리내렸지만
모든 법체계의 기초자들에게 자동적으로 주어졌던 것은 아니다.
오늘날 그것이 발견되는 어디서나 그것은 로마법에서 유래한 것임을
보일 수 있을 것이다.
또한 그것과 더불어 유언과 유증에 관한 일군의 법규칙들이 전해내려왔거니와,
오늘날의 법률가들은 저 원초적 이론과의 관계를 알지 못한 채 이들을 적용하고 있다.
그러나 순수한 로마법에서는 사람이 그의 상속인 속에서
계속 살아남는다---말하자면 사망의 사실이 제거된다---는
원리가 너무도 자명해서, 유언상속법과 무유언상속법 전체의 핵심이
무엇인지 도저히 오인할 수가 없었다.
지배적 이론을 따르도록 강제하는 로마법의 단호한 엄격함은
저 이론이 로마 사회의 초기 구조에서 자라났을 것임을 짐작하게 한다.
하지만 우리는 추정을 넘어 입증까지 나아가야 한다.
로마의 초기 유언 제도에서 기원하는
몇몇 법기술적인 표현들이 우연히도 우리에게 전해졌다.
가이우스의 저서에서는 포괄승계인을 지명하는 방식\latin{formula}이
발견된다.\footnote{고법상의 `구리와 저울에 의한 유언'을 말하고 있다.
  매수인은 ``당신의 가와 재산이 나의 책임과 보관에 넘어오도록 \ldots''
  이라고 언명한다. \latin{Gai.\,2.104.}}
나중에 상속인이라고 불리게 되는 사람을 애초에 지칭했던 옛 이름도
등장한다.\footnote{`구리와 저울에 의한 유언'에서
  상속인에 해당하는 자는 `가(家)의 매수인'(familiae emptor)이라고
  불리고 있다. \latin{Gai.\,2.103.}}
또한 우리는 유언권한을 명시적으로 인정하는
유명한 12표법 조항의 텍스트를 알고 있으며,\footnote{%
  ``가부장이 자신의 가(家)와 재산에 관하여 종의처분(終意處分)한 바가 있으면
  그대로 법으로 한다''(5.3). Cicero.\,De Inventione.\,2.148.}
무유언상속을 규율하는 조항들 역시 전해지고 있다.\footnote{``무유언으로
  사망하는 자에게 가내상속인이 없을 경우에는 가장 가까운 종친이
  가(家)를 상속한다''(5.4).
  ``그러한 종친이 없을 경우에는 씨족원들이 가(家)를 상속한다''(5.5). }
이 모든 고법\hanja{古法}상의 표현들은 두드러진 특징 하나를 가진다.
유언자로부터 상속인에게 넘어가는 것은 다름아닌 \hemph{가}\hanja{家}라는 것,
즉 가부장이 보유하고 그로부터 유래하는 권리와 의무의 총체라는 것이다.
물질적인 재산은 세 가지 경우에는 전혀 언급되지 않고 있으며,
나머지 두 가지 경우에는 가\hanja{家}의 부속물로서 거론되고 있을 뿐이다.
그리하여 원래의 유언은 \hemph{가}\hanja{家}의 이전을 규율하는 문서,
또는 \paren{처음에는 문서로 작성되지 않았을 것이므로} 절차였던 것이다.
그것은 유언자를 승계하여 누가 수장\hanja{首長}이 될 것인가를 선언하는
양식이었다.
유언의 원래의 목적이 이런 것임을 이해할 때,
우리는 그것이 고대종교와 고대법의 가장 진기한 유물의
하나---\hemph{사크라}\latin{sacra}, 즉 가족제의\hanja{祭儀}---와
어떻게 연결되는지 즉시 알 수 있다.
사크라는 원시의 옷을 완전히 벗어버리지 못한 사회라면 어디서나 보이는
제도의 로마적 형태였다.
그것은 가족의 동포애를 기념하는 희생제의였고,
가족의 영구성을 담보하고 증언하는 장치였다.
그 성격이 어떠하든---어떤 신화적인 조상에 대한 숭배이든 아니든---그것은
어디서나 가족관계의 신성함을 증명하기 위해 사용되었다.
따라서 수장의 인격이 바뀜으로써 가족의 연속성이 위협받는 곳이라면
그것은 특별한 의미와 중요성으로 다가왔다.
그리하여 우리는 그것을 가내 주권자의 사망과 관련하여 자주 듣게 되는 것이다.
인도인들 사이에서, 망자의 재산을 상속하는 권리는 그의 장례식을 치르는 의무와
정확히 궤를 같이했다.
제의가 제대로 거행되지 않거나 엉뚱한 사람에 의해 거행된다면,
죽은 자와 산 자 간에는 아무런 관계도 형성되지 않는 것으로 간주된다.
상속법이 적용되지 않고, 누구도 재산을 상속받을 수 없는 것이다.
인도인이 삶에서 겪는 중대한 사건들은 모두 이 장엄한 의례와 관련되고 이를
지향하는 것으로 보인다.
인도인이 혼인을 한다면, 그것은
그의 사후에 제의를 거행할 자식을 가지기 위해서이다.
그에게 자식이 없다면,
그는 다른 가족으로부터 양자를 들여야 한다는 강한 의무감을 갖는다.
인도인 학자에 따르면 ``장례식의 떡과 물과 신성한 제물을 염두에 두고''
그렇게 한다는 것이다.
키케로 시대에 로마의 사크라가 포괄하는 범위도 그에 못지 않았다.
그것은 상속과 입양을 다 포괄했다.
아들을 내어주는 가족의 사크라에 대한 적절한 대비\hanja{對備}가 없다면
입양은 효력을 발생하지 않았다.
공동상속인들 간에 장례식 비용을 엄격히 분배하지 않으면
유언에 의한 상속재산 분할은 일어날 수 없었다.
사크라를 마지막으로 엿볼 수 있는
이 시대의 로마법과 현존하는 힌두법 간의 차이는
시사하는 바가 크다.
힌두법에서는 종교적 요소가 전적으로 우세했다.
가족제의는 친족법 전부와 물권법 대부분의 초석이 되었다.
그것은 심지어 기괴하게 확장되기까지 했으니,
인도인들에 의해 역사시대에 이르기까지 지속된 관행이며
몇몇 인도^^b7유럽 민족의 전승\hanja{傳承}에도 남아있는,
남편의 장례식에서 과부가 스스로를 제물로 바치는 관행이,
인간의 피야말로 최고의 제물이라는, 희생제의에 언제나 동반되는 생각으로 인해
원시적 사크라에 접목되어 들어간 것이다.
반면, 로마인들에게는 법적 의무와 종교적 의무가 분리되기 시작했다.
사크라를 엄숙히 거행해야 한다는 요청은 세속법의 이론에 속하지 않았고,
대신 신관단\hanjalatin{神官團}{college of pontiffs}이 다루는
별도의 법역에 속했다.
물론,
키케로가 아티쿠스에게 보내는,
사크라에 대한 언급으로 가득한
편지들을 보면, 그것이 상속에 참기 힘든 부담이 되고 있었음을 알 수 있다.
하지만 법이 종교로부터 분리되는 발달지점을 지나,
후대의 법에서는 사크라가 완전히 사라진 것을 발견하게 된다.

\para{로마의 상속 관념}
힌두법은 진정한 유언에 해당하는 것을 알지 못한다.
유언의 자리를 대신 차지하고 있는 것은 입양이다.
이제 우리는 유언권한의 입양권능에 대한 관계를,
그리고 어째서 양자의 행사가 사크라의 거행에 대한 염려를 상기시키는지 그 이유를
알 수 있을 것이다.
유언과 입양은 둘 다 가계\hanja{家系}의 통상적인 진행을 왜곡시킬
위험을 안고 있지만, 명백히 이들은
계승할 친족이 존재하지 않을 때 가계가 단절되는 것을 막기 위한
장치들인 것이다.
이들 두 가지 수단 중에,
혈연 관계의 인위적 창설인
입양만이
대다수 고대 사회에서 발견된다.
사실 인도인들은 분명 고대의 관행인 것에서 한 걸음 더 나아가,
남편이 생전에 하지 못했다면 과부가 입양을 할 수 있도록 허용하였다.
또한 벵갈 지방의 관습에서는 유언권한의 희미한 흔적도 보이고 있다.
하지만,
계약 다음으로 인류 사회의 변화에 큰 영향을 끼친 제도인
유언을 발명한 것은 다름아닌 로마인들의 공적에 속한다.
그러나
보다 최근에 유언에 부여된 기능을
유언의 초기 형태에 부여하지 않도록 주의해야 한다.
초기의 유언은 망자의 재산을 분배하는 방법이 아니라,
가\hanja{家}의 대표권을 새로운 수장에게 전하는 여러 방법 중 하나였던 것이다.
물론 재산도 상속인에게 승계되지만,
그것은 이전되는 가\hanja{家}의 통치권에
공동재산을 처분하는 권한까지 포함되기 때문일 뿐이다.
유언의 역사에 있어
아직 우리는
유언이
재산의 유통을 자극하고
소유권에 유연성을 가져옴으로써
사회 변화의
강력한 도구가 되는 그러한 단계로부터 한참 멀리 있다.
사실 최후의 로마 법률가들조차도
유언권한에 그러한 결과가 부여되는 상태를 만들어내지 못한 것으로 보인다.
후술하겠지만,
로마 사회는 유언을
재산과 가족을 남에게 넘겨주는 장치나
잡다한 이해관계를 만들어내는 장치로
결코 생각한 적이 없으며, 오히려
무유언상속의 법규칙에 의한 것보다는 유언이
가의 구성원들을 위해 더 나은 대비\hanja{對備}를 할 수 있는 수단이라고
생각했던 것이다.
실로 로마인들이 유언 관행에 대해 가지는 관념은
오늘날 우리가 친숙하게 여기는 관념과는 완전히 달랐다고 보아도 좋을 것이다.
입양과 유언을 가\hanja{家}의 연속성을 위한 장치로 보는 습관은
주권\hanja{主權}의 상속에 관하여 로마인들이 가졌던 특유의 느슨한 관념과도
무언가 관련이 있음이 분명하다.
초기 로마 황제들 간의 승계가 꽤나 정상적이라고 여겨졌던 점,
그리고 테오도시우스나
유스티니아누스 같은 황제들이 카이사르와 아우구스투스의 전례\hanja{前例}를
따른다는 구실을 내세웠을 때
우여곡절은 있었지만
이를 비정상이라고 여기지 않았던 점 등은
보기 싫어도 보이는 사실들이다.\footnote{%
  379년 테오도시우스 1세는 선임인 서로마 지역 황제 그라티아누스에 의해
  동로마 지역 황제로 지명되었다.
  동로마제국의 유스티니아누스 1세는 전임 황제 유스티누스 1세에 의해
  양자로 입양되었다.}

\para{유언권한의 희소성}
원시사회의 현상들에 비추어볼 때,
17세기 법학자들이 의문시했던 명제, 즉
무유언상속이 유언상속보다 더 오래된 제도라는 명제를
논박하기는 불가능해 보인다.
이 문제가 해결되면 즉시 대단히 흥미로운 질문 하나가 떠오른다:
어떻게, 어떤 조건 하에서
유언의 지시가
가\hanja{家}에 대한 권위의 이전을,
따라서 재산의 사후적 분배를,
처음으로 규제할 수 있게 되었는가 하는 것이다.
이것을 규명하기 어려운 이유는
유언권한을 인정하는 고대 공동체가 드물었다는 데 있다.
로마를 제외하면 진정한 유언권한을 알고 있는 초기 사회가 있었는지 의심스럽다.
유언의 초보적 형태는 여기저기서 발견되지만,
그들 대부분은 로마의 것을 빌려왔다는 의심에서 자유롭지 못하다.
물론 아테네의 유언은 자생적인 것이었으나,
후술하듯이 그것은 미숙한 제도에 지나지 않았다.
로마제국을 정복한 만족\hanja{蠻族}들의 법전의 형태로 우리에게 전해지는
법체계들에서 인정되는 유언을 보면, 이것들은 거의 확실히 로마적이다.
이들 `만족\hanja{蠻族}들의 법'\latin{Leges Barbarorum}에 대한
사뭇 철처한 분석이 최근 독일에서 행해졌거니와,
그 목적은 각 법체계에서 어떤 부분이
그들의 원래 고향에서부터 부족 관습을 구성하던 부분이며
어떤 부분이 로마인들의 법에서 빌려온 부차적인 요소인지를
가려내는 것이었다.
이러한 과정을 거쳐 하나의 결과가 한결같이 도출되었으니,
각 법전의 오래된 핵심부분에는 유언의 흔적이 발견되지 않는다는 것이었다.
유언법이 존재한다면, 어느 것이나 로마법에서 가져온 것이었다.
마찬가지로, \paren{내가 알기로} 율법학자들이 유대법에 첨가한
초보적 유언도 로마인들과의 접촉에서 유래했다는 것이 정설이다.
로마나 그리스 사회에 속하지 않으면서
어떤 이유에서건 자생적이라고 할 만한 유언의 형태로는
벵갈 지방의 관행에서 인정되는 것이 유일하거니와,
인도에 사는 영국인 법률가들의 발명이라고까지 보는 이도 있는
이 벵갈의 유언은
기껏해야 초보적인 유언에 지나지 않는다.

\para{초기 유언의 작동}
그렇지만,
이러한 증거가 가리키는 것으로 보이는 결론은
유언이 처음에는
진정한 혹은 의제적 혈연권\hanja{血緣權}에 의해
상속자격을 갖는 사람이 없을 때에만 허용되었다는 것이다.
그리하여,
솔론의 법에 의해
아테네 시민들에게 처음
유언을 할 수 있는 권한이 주어졌을 때,
남성 직계비속을 상속에서 제외하는 것이 금지되었다.
마찬가지로 벵갈 지방의 유언도
가족의 어떤 우월적 권리와 부합하는 경우에만
상속을 통제할 수 있도록 허용된다.
또한, 유대인들의 초기 제도는 유언자의 특권을 전혀 알지 못했는데,
후대의 율법학자들의 법은
모세법에 누락이 있는 경우\latin{casus omissi}에만 보완한다는 명분으로
모세법상 상속자격 있는 친족들이 전혀 없거나 발견되지 않을 때에만
유언권한을 인정한다.
고대 게르만 법전들이 유언법을 수용하면서도 그것에 울타리를 쳐서 제한한 것도
의미심장하며, 동일한 방향을 지시한다.
우리에게 알려진 형태로만 볼 때 이들 게르만법 대다수는,
자유소유지\latin{allod}, 즉 가\hanja{家}의 소유지 외에,
여러 하위 유형 또는 하위 등급의 재산을 인정하는
특징을 갖고 있거니와,
이들 후자의 각각은 로마법 원리가 튜턴족의 원시적 관습체계에
개별적으로 융합되어 들어간 결과로 보인다.
원시 게르만법상의 재산, 즉 자유소유지는 전적으로 친족들에게만 주어진다.
그것은 유언 처분의 대상이 될 수 없을 뿐만 아니라,
살아있는 사람들 간\latin{inter vivos}에도 양도가 거의 불가능하다.
힌두법과 마찬가지로, 고대 게르만법에서도
아들들은 아버지와 함께 공동소유권자가 되며,
가족의 토지는 모든 구성원의 동의 없이는 처분할 수 없다.
그러나 보다 후대에 기원하며 자유소유지보다 등급이 낮았던
다른 종류의 재산은 훨씬 더 쉽게 양도될 수 있으며,
훨씬 더 느슨한 상속규칙에 따른다.
여자들과 그녀의 후손들도 그것을 상속할 수 있거니와, 이는
그것이 종족\hanja{宗族}관계의 성역\hanja{聖域} 바깥에 있다는
원리에 의한 것이 분명하다.
로마로부터 차용한 유언법이 처음 작동하도록 허용된 것은
바로 이 후자의 성격의 재산에 대해서였고, 오직 그것에 국한되었다.

\para{귀족의 유언}
이들 몇몇 사례가 주는 암시는
로마 유언법의 초기 역사에서 확인되는 사실을 사뭇 그럴 듯하게 설명하는 것으로
보이는 것에 추가적인 설득력을 제공할 수 있을 것이다.
풍부한 전거에 따르면,
로마 국가의 초기 역사 동안 유언은
코미티아 칼라타\latin{comitia calata}\footnote{굳이 번역하자면
`소집된 민회'라는 뜻이다.}에서 행하여졌다.
코미티아 칼라타는 로마 귀족 시민들의 입법기구인
쿠리아 민회\latin{comitia curiata}가 사적인 안건을 위해 소집된 것이다.
이러한 유언 방식은 대륙법학자들 사이에서 세대를 거쳐 전해져온 주장,
즉 로마 역사에서 한때 모든 유언은 장엄한 입법행위였다는 주장의
근거가 되었다.
그러나 고대 민회의 절차에 지나친 정확성을 부여하는 결함이 있는
설명에 굳이 의존해야 할 필요가 있는지 의문이다.
코미티아 칼라타에서의 유언에 관한 이야기를 풀이할 적절한 열쇠는
\hemph{무유언}상속에 관한 옛 로마법에서 찾아야 할 것이 분명하다.
친족 간의 상속을 규율하는 원시 로마법의 규칙은,
법무관의 고시법\hanja{告示法}에 의해 변경되기 전까지는,
다음과 같은 순서를 따랐다:
첫째, 가내상속인\hanjalatin{家內相續人}{sui}, 즉 부권면제되지 않은
직계비속이 상속한다.
가내상속인이 없으면 최근친 종족\latin{nearest agnate},
즉 망자와 동일한 가부장 아래 있었거나 있을 수 있었던 친족 중에
가장 가까운 사람 또는 사람들이
그 자리를 차지한다.
세 번째이자 마지막 순위로 상속재산은 씨족원들\latin{gentiles}, 즉
망자가 속한 씨족\latin{gens; house}의 공동의 구성원들에게 넘어간다.
전술했듯이 씨족은 가족의 의제\hanja{擬制}적 확장이었으니, 그것은
동일한 이름을 가진,
그리고 동일한 이름을 가졌기에 공통의 조상의 후손들이라고 믿어진,
모든 로마 귀족 시민들로 구성되었다.
쿠리아 민회라고 불린 귀족들의 집회는 그야말로 씨족들만을 대표하는 입법기구였다.
국가를 구성하는 단위가 씨족이라는 전제 위에 구성된,
로마 인민을 대표하는 집회였던 것이다.
사정이 이러하다면 다음과 같은 추론이 불가피해진다.
저 민회에 의한 유언의 확인은 씨족원들의 권리와 관련된다는 것,
그리고 그들의 최종적 상속권을 보호하려는 의도에서 이루어졌다는 것이다.
씨족원들을 발견할 수 없을 때에만,
혹은 씨족원들이 권리를 포기했을 때에만,
유언이 행해질 수 있었다고 가정한다면,
그리고 로마 씨족들의 총회에 제출된 모든 유언은
그 유언 처분에 의해 손해를 입을 사람들이 원한다면 거부권을 행사하거나
아니면 유언을 통과시킴으로써 그들의 상속권을 포기한 것으로 간주하기 위해서
제출되었다고 가정한다면,
일견 이상하게 보이던 모든 것들이 말끔히 해소된다.
12표법의 제정 직전에는 이러한 거부권이 대단히 축소되었거나
아니면 간헐적으로만 그리고 예외적으로만
행사되었을 수 있다.
하지만 코미티아 칼라타에 주어진 권한이 어떻게 점차 발달했는지 혹은
점차 쇠퇴했는지 추적하는 일보다 그것의 본래의 의미와 기원을 밝히는 일이
훨씬 더 수월하다.

하지만,
근대 유언법의 계보를 거슬러올라갈 때 만나는 유언은
코미티아 칼라타에서 행한 유언이 아니라,
그것과 경쟁하여 마침내 그것을 대체한 또 다른 유언이다.
로마 유언법의 초기 역사가 대단히 중요하고
또 그것을 통해 많은 고대 관념들을 해명할 수 있기에
다소 장황한 서술이 이어지더라도 양해하시길 바란다.

\para{평민의 유언}
유언권한이 법의 역사에 처음 등장했을 때,
로마의 거의 모든 위대한 제도들과 마찬가지로,
이것도 귀족들과 평민들 간의 투쟁의 대상이었다.
``평민은 씨족을 갖지 않는다''\latin{Plebs gentem non habet},
즉 평민은 씨족의 구성원이 될 수 없다는 정치적 격률이 뜻하는 바는
평민은 쿠리아 민회로부터 전적으로 배제되었다는 것이다.
그리하여 일부 학자들은, 평민은 귀족들의 집회에서 유언을
낭독하거나 구술할 수 없었고
따라서 유언의 특권을 완전히 박탈당했다고 주장했다.
다른 학자들은, 유언자가 대표되지 못하는 비우호적인 집회에
유언 안건을 제출해야 하는 고초를 지적하는 데 만족했다.
어느 견해가 진실이든,
불쾌한 어떤 의무를 회피할 의도에서 고안되었다고 볼 수밖에 없는
유언 형식 하나가 널리 사용되기에 이르렀다.
문제의 유언은 살아있는 사람 간의\latin{inter vivos} 양도로서,
유언자의 가\hanja{家}와 재산을 그가 상속인으로 점찍은 사람에게 양도하는
완전하고 철회불가능한 행위였다.
로마의 엄격법에 따라 이러한 양도행위는 언제나 허용되었으나,
그것이 사후\hanja{死後}적 효과를 의도하는 경우에는
귀족들의 입법기구의 공식적 승인을 받지 않고도 유효한 유언이 될 수 있는지
논란이 될 수 있었다.
이 점에 관하여 로마의 두 신분집단 간에 의견대립이 있었다 할지라도,
다른 많은 시기\hanja{猜忌}의 원인들과 함께
이것도 십인위원회\hanja{十人委員會}의\latin{decemviral}
타협에 의해 사라졌다.
``가부장이 금원과 그의 물건의 후견에 관하여 종의처분\hanja{終意處分}한 바
있으면, 그대로 법으로 한다''%
\latin{Pater familias uti de pecuniâ tutelâve rei suae legâssit, ita jus esto}는
12표법상의 텍스트가 우리에게 전해지거니와,
이 조항은
평민의 유언을 합법화하는 것 이외의 목적을 가졌다고 보기 어렵다.

귀족들의 집회가 로마 국가의 입법기구임을 그치고 수 세기가 지나서도
여전히 사적인 안건을 처리하기 위해 그것이 공식적으로 개최되었다는 것은
학자들 사이에 잘 알려져 있다.
결과적으로, 12표법이 공표되고 나서도 오랫동안
유언의 확인을 위해 코미티아 칼라타가 소집되었다고 믿을 이유가 있는 것이다.
아마도 그것의 기능은
그것이 유언등기소\latin{court of registration}였다는
말로써 가장 잘 표현될 수 있을 것이다.
하지만 이 말은 제출된 유언이 \hemph{대장에 기록}되었다는\latin{enrolled}
뜻이 아니라,
단지 참가자들에게 구술하여 그들이 그 취지를 이해하고 기억하도록 하는 데
그쳤음에 유의해야 한다.
이러한 형식의 유언은 문서로 작성되지 않았을 것이 거의 확실하며,
설령 유언이 애초 문서화되었다 할지라도, 민회의 임무는 분명
그것을 큰 소리로 낭독하는 것을 듣는 것에 그쳤을 것이니,
그후 그 문서는 유언자가 보관하거나 아니면
어느 신전\hanja{神殿}의 보호에 맡겨졌을 것이다.
코미티아 칼라타에서의 유언의 한 측면인
이러한 공개성은 대중들이 그것을 꺼리는 원인이 되었다.
제정 초기에도 저 민회는 개최되었으나,
단순히 형식적인 것으로 전락하였던 듯하고, 아마도
정기집회에 제출되는 유언은 거의 또는 전혀 없었을 것이다.

\para{악취행위}
근대 세계의 문명을 크게 바꾼 장기적 영향력을 가진 것은
고대 평민의 유언---방금 기술한 유언의 대체물---이었다.
그것은 코미티아 칼라타에 제출되는 유언이 상실한 인기를 고스란히 획득했다.
그것의 성격을 이해하는 열쇠는
고대 로마의 양도방식인
악취행위\hanjalatin{握取行爲}{mancipium}에서
그것이
유래했다는 데 있다.
악취행위는 근대사회에서는 하나로 연결지어 생각하기 힘든
두 가지 위대한 제도, 즉 계약과 유언의 모태라고 서슴없이 말할 수 있는 절차이다.
후대 라틴어에서 만키파티오\latin{mancipation}라고 불리게 되는
악취행위의 제반 측면들은 우리를 국가사회의 유년기로 이끌고 간다.
문자의 발명까지는 아니더라도 어쨌든 문자의 대중화 이전으로
그 기원을 소급하기에,
몸짓과 상징적 행위와 장엄한 어구\hanja{語句}가 문서의 형식을 대신했다.
길고 복잡한 의식\hanja{儀式}은 거래의 중요성에 대한
당사자들의 주의를 환기시키는 동시에 증인들의 기억에 각인을 남기려는 것이었다.
또한, 문서화된 증거에 비해 불완전할 수밖에 없는 구두\hanja{口頭}절차였기에,
후대 사람들이 적절하다고 생각하는 또는 한계라고 생각하는 선 이상으로 많은
증인들과 보조자들이 필요했다.

로마의 악취행위에는 우선 당사자 모두, 즉 매도인과 매수인이,
혹은 오늘날의 법률용어로는 양도인과 양수인이라고 불러야 할 사람들이
참가해야 했다.
또한 적어도 \hemph{다섯 명}의 증인들과 더불어 좀 특이한 인물인
저울소지자\latin{libripens}가 필요했다.
저울소지자는 고대 로마의 주조되지 않은 구리 화폐의 무게를 다는
천칭을 가지고 왔다.
우리가 다루는 유언---오랫동안
`구리와 저울에 의한\latin{per aes et libram} 유언'이라고
법기술적으로 불리어온 유언---은
통상적인 악취행위와 형식에 있어서 동일했고
언표하는 내용도 거의 다르지 않았다.
유언자가 양도인이 된다.
다섯 명의 증인과 저울소지자도 현장에 나와있다.
양수인의 자리에는 법기술적으로
`가\hanja{家}의 매수인'\latin{familiae emptor}이라고 불리던 사람이 선다.
이제 악취행위의 통상적인 의식이 거행된다.
어떤 형식적인 몸짓들이 행해지고 형식적인 문장들이 선언된다.
가의 매수인이
구리 화폐 조각으로 저울을 쳐서 대금을 지불하는 행위를 흉내낸다.
끝으로 유언자가
거래의 공표에 해당하는 ``언명''\hanjalatin{言明}{nuncupatio}이라
불리는 일련의 형식적인 말로써 지금까지 행하여진 것을 승인한다.
법률가들에게는 상기시킬 필요가 없겠지만, 이 언명은
유언법에서 장구한 역사를 가지고 있다.
특히 `가의 매수인'이라고 불리는 사람의 성격에 주목할 필요가 있다.
처음에는 그가 상속인 자신이었다는 데 의심의 여지가 없다.
유언자는 그에게 ``가\hanja{家}'' 전체, 즉
가에 대해 그리고 가를 통해 유언자가 향유하는 일체의 권리를 완전히 양도했다.
그의 재산, 그의 노예, 선조에게 물려받은 그의 모든 특권을,
다른 한편으로 그의 모든 의무 및 채무와 더불어, 함께 양도했던 것이다.

\para{양도로서의 유언}
이러한 자료를 앞에 두고,
우리는---이렇게 부를 수 있다면---`악취행위에 의한 유언'이
그 원시적 형태에 있어서 근대의 유언과 어떻게 다른지 몇 가지 주목할 점을
지적할 수 있다.
그것은 유언자의 가산을 아주 양도해버리는 것이므로
\hemph{철회가능}하지 않았다.
이미 소진해버린 권한을 새로이 행사할 수는 없었던 것이다.

또한 그것은 비밀성이 없었다.
가의 매수인은 자신이 상속인이면서도 그의 권리가 무엇인지 정확히 알았고,
상속재산에 대한 권원을 불가역적으로 가지게 되었음을 알았다.
가장 질서잡힌 고대사회라 하더라도 없을 수 없는 폭력이 이 지식을 대단히
위험한 것으로 만들었다.
그러나 아마도 양도에 대한 유언의 이러한 관계가 가져오는 가장 놀라운 결과는
상속인에게 상속재산이 즉시 주어진다는 점일 것이다.
적지 않은 대륙법학자들에게 이것은 너무도 믿을 수 없는 일이었기에,
그들은 유언자의 가산이 유언자의 사망을 조건으로 하여 주어졌다거나,
불특정 시점부터, 즉 양도인의 사망시부터 주어졌다고 말해왔다.
하지만 로마법의 마지막에 이를 때까지,
조건에 의해 직접 변경될 수 없는, 혹은
어떤 시점까지 또는 어떤 시점부터라는 제한이 있을 수 없는,
거래의 유형이 존재했다.
법기술적 용어로는 조건\latin{conditio}이나
기한\latin{dies}이 붙을 수 없는 거래들이 있었던 것이다.
악취행위가 바로 그런 거래의 하나였다.
따라서, 이상하게 보일지 몰라도, 우리는 초기 로마의 유언은,
비록 유언자가 자신의 유언행위 이후에 오래 산다고 하더라도,
즉시 효력을 발생했다고 결론짓지 않을 수 없다.
어쩌면 사실 로마 시민들은 원래 사망에 임박해서만 유언을 했을 것이고,
한창 나이의 남자가 가의 연속성을 위한 대비를 할 때는
유언이 아니라 입양의 형식을 취했을 것이라고 추정할 수 있다.
그렇지만, 만약 유언자가 건강을 회복하였다 해도,
그는 상속인의 묵인 하에서만 그의 가를 계속 지배할 수 있었을 것이다.

\para{고대 유언의 비서면성}
어떻게 해서 이러한 불편함이 치유되었는지,
어떻게 해서 유언이 오늘날 널리 부여되는 성격을 가지게 되었는지
설명하기 전에
두 세 가지 먼저 말해둘 것이 있다.
유언은 문서화될 필요가 없었다:
처음에는 언제나 구두\hanja{口頭}였던 것으로 보이며,
나중에도 유증을 선언하는 문서는 유언에 부수적인 요소였을 뿐,
본질적 구성요소를 이루지는 않았다.
그것의 유언에 대한 관계는
옛 영국법에서
종국화해\hanjalatin{終局和解}{fine}나
공모회수소송\hanjalatin{共謀回收訴訟}{recovery}의
이용을 이끄는 날인증서\latin{deed leading the uses}가
종국화해와
공모회수소송에
대해 가지는 관계,\footnote{%
  종국화해(final concord; fine)와 공모회수소송(common recovery; recovery)은
  확실하고 완전한 소유권을 양도하기 위한
  공모소송(collusive action)의 방식들로서,
  전자는 재판상화해의, 후자는 판결의 형식을 취한다.
  1833년 `종국화해 및 공모회수소송에 관한 법률'(Fines and Recoveries Act)에
  의해 모두 폐지되었다.
  한편, 이들 공모소송과 관련하여 몇몇 날인증서가 작성되었는데,
  `종국화해의 이용을 이끄는 날인증서'(deed to lead the uses of a fine)와
  `공모회수소송의 이용을 이끄는
  날인증서'(deed to lead the uses of a common recovery)가 대표적이다.
  이들 증서에는 왜 이러한 공모소송을 이용하려는지 그 목적이 제시된다.
}
또는
토지보유권양도날인증서\latin{charter of feoffment}가
토지보유권양도 자체에 대해 가지는 관계\footnote{%
  원래 토지보유권은 어떤 상징적 행동과 언명에 의해 양도되었는데,
  이를 확인하는 날인증서의 작성 또한 관행상 널리 행하여졌다.
  하지만 1677년 사기방지법(Statute of Frauds) 제정 이전에는
  날인증서의 작성이 필수요건은 아니었다.
}와
정확히 일치한다.
실로 12표법 이전에는 문서가 조금도 이용되지 않았을 것이니,
유언자에게는 유증\latin{legacy}할 권리가 없었고,
유언으로 이익보는 자는 상속인 또는 공동상속인들에 국한되었기 때문이다.
하지만 12표법 조문의 극단적 일반성으로 인해
곧이어
상속인은 유언자의 지시를 이행할 부담을 안고서,
다시 말해 유증의 부담을 안고서,
상속재산을 취득해야 한다는 법리가
형성되었다.
따라서 서면으로 작성된 유언장은
수유자\hanjalatin{受遺者}{legatee}의 권리를 침해하는 상속인의 기망행위를
방지한다는 새로운 가치를 띠게 되었다.
그러나
증인들의 증언에만 의존할 것인가, 즉
가의 매수인이 지불해야 할 유증의 선언을 말로써 할 것인가 여부는
마지막까지도 유언자의 재량에 맡겨져 있었다.

\para{가의 매수인}
`가의 매수인'\latin{emptor familiae}이라는 용어는 특별히 주목을 요한다.
``매수인''\latin{emptor}은 유언이 글자 그대로 매매였음을 의미한다.
``가\hanja{家}''라는 단어는,
12표법의 유언 관련 조문의 표현에 비추어볼 때,
시사하는 바가 적지 않다.
고전 라틴어에서 ``가\hanja{家}''는 항상 어떤 사람의 노예를 뜻한다.
하지만 여기서는, 그리고 고대 로마법의 일반적 용법에서는,
그것은 그의 가부장권에 복속하는 모든 사람을 포함하는 의미였으며,
유언자의 물질적 재산은 그의 가\hanja{家}에 부수하는 부속물로서
이전된다고 이해된다.
다시 2표법으로 돌아가면, ``그의 물건의 후견''\latin{tutela rei suae}이라는
표현이 등장하거니와, 이는 방금 설명한 용어를 정확히 거꾸로 뒤집은 표현형태이다.
따라서,
비교적 늦은 시기인
십인위원회의 타협의 시기에도,
``가''를 지칭하는 용어와 ``재산''을 지칭하는 용어가 당대의 용법에서
서로 혼재되어 쓰였다는
결론을 피해가기는 불가능해보인다.
만약 어떤 사람의 가\hanja{家}가 그의 재산이라고 말하여진다면,
이 표현은 가부장권의 범위를 가리키는 말로 이해할 수 있을 것이다.
그러나, 거꾸로 바꾸어 쓸 수도 있는 것이기에,
저 표현형태는
재산은 가족에 의해 소유되고 가족은 시민에 의해 지배되는,
그리하여 공동체의 구성원은 재산\hemph{과} 가족을 소유하는 것이 아니라,
가족을 \hemph{통하여} 재산을 소유하는,
그러한 원시적 시기를 우리에게 시사한다고
인정하지 않을 수 없는 것이다.

\para{법무관법의 유언}
정확히 언제부터인지는 알 수 없지만,
로마의 법무관들은
엄격한 형식요건을 요하는 유언을
법의 문언보다는 법의 정신에 더 부합하게
취급하기 시작했다.
그때그때의 처리가 어느새 확립된 관행이 되어갔고,
마침내 완전히 새로운 유언의 형태가 자라나
꾸준히 고시법\hanja{告示法}에 접목되어 들어갔다.
이 새로운, \hemph{법무관법의} 유언\latin{praetorian testament}은 그 견실함을
오직 명예관법\hanjalatin{名譽官法}{jus honorarium}, 즉 로마의 형평법에
빚지고 있었다.
어느 해, 신임 법무관은 취임시 선포되는 자신의 고시에
이러저러한 형식요건들을 갖춘 유언은 모두 인정하겠노라는 뜻을 담은
조항 하나를 삽입했을 것이다.
이 개혁조치가 유익한 것으로 판명되자, 관련 조항은
차기 법무관에 의해 재차 도입되었을 것이며,
후임자들에 의해서도 반복되어, 마침내
이러한 연속적 포함 덕분에 영구고시록\latin{Perpetual Edict}이라고 불리게 되는
법체계의 공인된 일부를 형성하게 되었다.
법무관법 유언이 유효하기 위한 요건을 조사해보면,
그것은 악취행위에 의한 유언의 요건에 기초하고 있었음이 명백히 드러날 것이다.
저 혁신가 법무관은 옛 형식요건들 가운데 진정성을 담보할 수 있거나
기망행위를 방지할 수 있는 것들만 보존하기로 결정하였을 것이 분명하다.
악취행위에 의한 유언에서는 유언자 외에 7명의 사람들이 현장에 나와야 했다.
따라서 법무관법 유언에도 7명의 증인이 요구되었다.
그중 두 명은 원래 저울소지자\latin{libripens}와
가의 매수인\latin{familiae emptor}이었으나 이제 이들의 상징적 성격은
제거되고 단지 증인으로서의 역할만 담당하게 되었다.
각종 상징적 의식절차도 사라졌다.
유언이 구술\hanja{口述}될 뿐이었다. 그렇지만 아마도
\paren{전적으로 확실한 것은 아니지만}
유언자의 처분에 관한 증거를 영구화하기 위해 서면이 필요했을 것이다.
어쨌거나, 서면이 유언자의 마지막 의사로서 읽혀지거나 제시된 경우,
7명의 증인들이 각자 그 겉봉에 자신의 인장을 날인하지 않았다면
법무관의 법정이
특별히 개입하여
그 효력을 인정하지 않았을 것임을 우리는 잘 알고 있다.
이것은
\hemph{날인}\latin{sealing}이
법의 역사에서
인증\hanja{認證}의 수단으로
처음 등장하는 사례이다.
하지만
단순히 잠금장치로서 날인이 사용된 것은 물론 훨씬 더 오래 전의 일이며,
히브리인들에게도 알려져있었던 듯하다.
로마인들에게 유언장 또는 다른 중요한 문서의 날인은
날인한 자의 참석과 동의의 지표로서 기능했을 뿐만 아니라,
또한
나중에 서면을 조사하기 전에 깨뜨려야 할, 말그대로 잠금장치이기도 했음을
알 수 있다.

\para{유산점유}
그리하여,
악취행위의 형식을 통해 거행되지 않고
단지 7명의 증인의 날인에 의해 입증되는 경우,
그러한 유언자의 처분은 고시법이 강제할 수 있게 되었다.
그러나
로마인의 재산의 주요 속성들은
시민법과 그 기원을 함께 한다고 여겨진 절차를 통하지 아니하고는
양도될 수 없었다는 것이 일반 명제로 제시될 수 있을 것이다.
따라서 법무관은 그 누구에게도 \hemph{상속재산}을 수여할 수는 없었다.
유언자가 자신의 권리와 의무에 대해 가졌던 관계와 동일한 관계를
상속인이나 공동상속인들에게 줄 수는 없었던 것이다.
법무관이 할 수 있는 것이라고는
물려받은 재산의 사실상의 향유권을
상속인으로 지명된 자에게
주는 것과
유언자의 채무에 대한 그의 변제에 법적인 효력을 인정하는 것이
전부였다.
이러한 목적을 위해 법무관이 권한을 행사했을 때,
법기술적으로는
`유산점유'\latin{bonorum possessio}를 수여했다고
표현되었다.
이 경우 상속인에 해당하는 자, 즉 유산점유자\latin{bonorum possessor}는
시민법상의 상속인이 누리는 모든 재산법상의 특권을 가지고 있었다.
과실\hanja{果實} 수취도 할 수 있었고, 양도도 할 수 있었다.
하지만 피해를 구제받기 위해서는 법무관 법정의,
이런 표현을 쓸 수 있다면, 보통법적 측면이 아니라
형평법적 측면에 호소해야 했다.
그를 상속재산에 대한 \hemph{형평법상의}\latin{equitable} 소유자라고
부르더라도 크게 잘못된 표현은 아닐 것이다.
그렇지만,
이러한 유추가 불러올 수 있는 오해를 불식시키기 위해
한 가지 반드시 유념해야 하는 점은,
1년이 지나면
유산점유가
로마법상 사용취득\latin{usucapion}이라 불린 원리의 적용을 받는다는 것이다.
그리하여
점유자는 상속재산에 속하는 모든 재산에 대해
로마시민법상의\latin{quiritarian} 소유권자가 되었다.

\para{옛 유언의 진화}
옛 민사소송법에 대해 우리가 가진 지식이 얕은 수준에 머물러 있기에,
법무관 법정이 제공한 구제수단의 다양한 유형들 간의 장점과 단점을
균형있게 파악하기가 쉽지 않은 것이 현실이다.
하지만 한 가지 확실한 점은,
포괄적 재산을 한꺼번에 고스란히 이전하는 악취행위에 의한 유언은
그 모든 결함에도 불구하고
새로운 유언에 의해 결코 완전히 대체되지 않았다는 것이다.
옛 방식에 대한 집착이 완화된 이후에도,
옛 방식의 의미가 어느 정도 생동감을 잃은 이후에도,
법학자들의 재능은 보다 유서 깊은 유언 수단을 개량하는 데
집중되었던 것으로 보인다.
가이우스의 시대, 즉 안토니누스 황조 시대에 이르면
악취행위에 의한 유언을 둘러싼 주요 결함들이 사라지게 된다.
전술했듯이, 원래 이 방식의 본질적 성격은
상속인 자신이 가의 매수인이 될 것을 요구했고,
따라서 상속인은
유언자의 재산에 속한 기존 권리의무를 즉시 취득했을 뿐만 아니라,
자신의 권리가 무엇인지도 공식적으로 알 수 있었다.
하지만 가이우스의 시대에는
이해관계 없는 사람이 가의 매수인의 역할을 할 수 있었다.
그리하여 실제 상속인에게는 그가 받게 될 상속에 대해 굳이 알려줄 필요가 없었으니,
이후로는 유언이 \hemph{비밀성}을 획득하게 되었다.
이렇게 실제 상속인 대신에 국외자가 ``가의 매수인'' 기능을 담당함으로써
먼 훗날 또 다른 결과도 생겨났다.
이것이 합법화되자 로마의 유언은 두 부분 또는 두 단계로
구성되는 것이 되었다.
하나는 순수한 형식이었던 양도\latin{conveyance}이고,
다른 하나는 언명\latin{nuncupatio}, 즉 공표이다.
후자의 절차단계에서 유언자는
자신의 사후에 무엇이 행해져야 할지 의사를 보조자들에게 구두로 선언하거나,
아니면 자신의 의사가 담겨있는 문서를 제출했다.
거래의 핵심 부분에 주어지는 관심이
가상의 양도로부터 멀어지고
언명에 집중되자,
이제 유언은 \hemph{철회가능한} 것이 될 수 있었다.

지금까지 법사\hanja{法史}를 따라 내려오면서 유언의 계보를 살펴보았다.
그것의 뿌리는 악취행위, 즉 양도에 기초한 ``구리와 저울에 의한'' 유언이다.
하지만 이 고대의 유언은 다수의 결함을 가지고 있었고,
그것은 법무관법에 의해 간접적으로만 교정될 수 있었을 뿐이다.
그러는 동안, 재능있는 법학자들은
법무관들이 형평법을 통해 동시대에 수행해온 것과 같은 개량을
보통법적 유언, 즉 악취행위에 의한 유언에 대해 수행했다.
하지만 이러한 개량은 단지 법적인 재간에 의존한 것이었기에,
가이우스나 울피아누스의 시대의 유언법은 과도기적인 것에 불과했다.
그후의 변화과정에 대해서 우리는 잘 알지 못한다.
그러나 마침내 유스티니아누스에 의한 법학의 재건이 있기 직전
동로마제국의 백성들이 사용하고 있던 유언의 형태는
그 계보를 한편으로는 법무관법의 유언에,
다른 한편으로는 ``구리와 저울에 의한'' 유언에 소급할 수 있는
것이었다.
법무관법의 유언처럼, 그것은 악취행위를 요구하지 않았고,
7명의 증인들의 날인이 없으면 무효였다.
악취행위에 의한 유언처럼, 그것은 단순히 유산점유가 아니라
상속재산을 이전하는 것이었다.
하지만 그것의 주요 특징의 일부는 실정적 입법에 의해 추가된 것이었다.
이렇게 법무관의 고시, 시민법, 그리고 황제의 칙법이라는
세 가지 기원을 가진다는 의미에서
유스티니아누스는 당시의 유언법을
`삼중의 법'(jus tripertium)이라고 불렀던 것이다.\footnote{Inst.\,2.10.3.}
방금 언급한 이 새로운 유언이
로마인의 유언이라고
일반적으로
알려져 있는 것이다.
그러나 그것은 동로마제국에 국한된 것이었다.
사비니의 연구가 밝혀놓은 것처럼,
서유럽에서는 옛 악취행위에 의한 유언이,
양도, 구리, 저울 등 그것의 모든 장치들과 함께,
중세에 들어서도 한동안 계속 사용되었다.

