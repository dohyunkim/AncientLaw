\chapter{고대와 근대의 유언 및 상속에 관한 관념}

근대 유럽의 상속법에는
역사 초기에 사람들 사이에 준행된 유언 처분의 규칙과
밀접한 관련된 것들이 많이 있지만,
유언과 상속이라는 주제에 대한 고대와 근대의 관념들 간에는
몇 가지 중요한 차이점들이 있다.
이번 장에서는 그러한 몇 가지 차이점들을 설명해보고자 한다.

\para{자녀의 상속제외}
12표법이 제정되고 몇 세기가 흐른 시점에 이르면,
자녀를 상속에서 제외하는 것을 제한하는
여러 법규칙들이
로마 시민법에
접목되어 들어가 있었다.
이러한 관심에서 법무관의 재판권이 적극적으로 행사되고 있었으며,
또한
아버지의 유언에 의해 상속에서 부당하게 제외된 자녀에게
상속재산을 회복시켜주는,
\index{배륜유언의 소}%
``배륜유언\hanja{背倫遺言}의
소\hanja{訴}''\latin{querela inofficiosi testamenti}라고 불리는,
성질이 특이하고 기원이 모호한 새로운 구제수단이 제공되고 있었다.\footnote{%
  이유 없이 상속제외되거나
  법정상속분의 $1/4$에 미달하는 재산을 물려받은 경우
  지정상속인들을 상대로 이 소송을 제기하여 자신의 법정상속분을
  회복할 수 있었다. }
이러한 법상태를
최대한의 유언 자유를 문언상 인정하는 12표법의 텍스트와 비교하면서,
몇몇 학자들은 로마 유언법에 어떤 커다란 극적인 변화가 있었다고
추정하기도 한다.
그들에 따르면,
가족의 수장들은 무제한인 상속제외의 권한을 즉각 남용하기 시작했고,
이 새로운 관행이 대중의 도덕감정을 침해하여 스캔들을 일으켰으며,
가부장의 악행을 저지하는 법무관들의 용기에 모든 선량한 사람들이
환호를 보냈을 것이라고 한다.
이러한 이야기는,
관련 주요 사실에 비추어 전혀 근거가 없지는 않으나,
법제사의 기본 원리에 대한 자못 심각한 오해를 노정한다고
일반적으로 평가된다.
\wi{12표법}은 제정될 당시의 시대적 성격에 비추어 설명되어야 한다.
12표법은 후대에 반작용을 불러올 어떤 경향을 허용하려고 만들어진 것이 아니다.
그것은 그러한 경향이 존재하지 않는다는 가정 하에,
아니 어쩌면 그러한 경향의 존재 가능성조차 인식하지 못한 채,
만들어진 것이다.
로마 시민들이 상속제외의 권한을 즉시 자유롭게 이용하기 시작했다는 것은
개연성이 없다.
가족이라는 굴레의 멍에가
여전히 끈질기게 추종되고 있고
가혹하게 강제되고 있는 곳에서,
오늘날에도 환영받지 못할 일로써
그 멍에를 벗어던진다는 것은
역사의 모든 합리성과 건전한 판단에 역행하는 것이다.
12표법은 유언이 실행될 수 있을 경우에만,
즉 자식들과 근친들이 없는 경우에만,
유언이
실행되도록 허용한 것이다.
직계비속의 상속제외를 금지하지 않은 것은
당대의 로마 입법자가 생각할 수 없었던 일에 대비한 입법을
하지 않은 것에 불과하다.
물론, 가족애의 의무가 점차 개인의 일차적 의무인 측면을 상실하면서,
자식의 상속제외도 이따금 행해졌을 것이다.
하지만 법무관의 개입은,
남용이 보편적이었기 때문에 촉발된 것이 아니라,
애초 그러한 부자연스런 변덕이 소수의 예외적인 사례였고
당대의 도덕감정에 반하였기 때문에 촉발되어졌음에 틀림없다.

\para{로마의 무유언상속}
로마 유언법의 이 부분이 시사하는 것은 전혀 다른 종류의 것이다.
로마인들은 유언을 가족을 상속제외하는 수단이나
재산을 불균등 분배하는 수단으로 보지 않았을 것이라는 데 주목해야 한다.
유언법이 전개되어감에 따라
그러한 목적에 유언이 사용되는 것을 방지하는 법규칙들이
증가하고 강화되었던 것이다.
이러한 법규칙들은,
개인의 일시적 감정 변화와는 구분되는,
로마 사회의 불변의 감정에 분명 부합하는 것이었다.
유언 권한의 주된 가치는 가족을 위해 \hemph{대비를 하는} 데에,
\wi{무유언상속}법이 나누어주는 것보다 더 균등하고 공정하게
상속재산을 나누어주는 데에,
기여하는 것이었다고 생각된다.
대중의 일반적 감정을 이렇게 읽는 것이 옳다면,
이것은 로마인들이 줄곧 가졌던 무유언상속에 대한 특유의 두려움을
어느 정도 설명할 수 있을 것이다.
유언의 특권을 박탈당하는 것보다 더 큰 불행은 없다고 생각되었던 것이다.
적에게 퍼부을 수 있는 저주 가운데
유언 없이 죽으라는 말보다 더 심한 저주는 없다고 생각되었던 것이다.
오늘날의 여론에는 이것에 해당하는 감정이 아예 없거나 찾아보기가 어렵다.
물론 어느 시대 어느 누구든
자신의 임무를 법이 대신 수행해주는 것보다
스스로 자신의 재산의 운명을 설계하는 것을
더 선호할 것이다.
그러나 유언을 향한 로마인들의 열정은 그 강도\hanja{强度}에 있어
단순히 자의로 처분하고자 하는 욕구와는 차원을 달리하는 것이었다.
또한 그것은
\wi{봉건제}의 산물에 불과한,
재산의 한 종류를 단일한 대표자의 수중에 모아두어
가족의 위신을 지키려는 것과도
물론 아무런 공통점이 없었다.
추론컨대,
무유언상속법에 들어있는 무언가가
법에 의한 재산분배보다
유언에 의한 재산분배를
이렇게 강렬하게
선호하게 만들었던 것 같다.
하지만 문제는,
근대 입법자들이 거의 보편적으로 받아들인
유스티니아누스의 상속법 정비 이전으로 수 세기를
거슬러올라간 시대의
무유언 상속에 관한 로마법을 일별해 보더라도,
눈에 띄게 불합리하다거나 불공정한 부분을 발견하기가 어렵다는 것이다.
오히려 법이 예정한 분배 규칙이 사뭇 공정하고 합리적이었고
근대사회가 널리 만족하고 있는 것과 별반 다르지 않았음에도,
무엇보다
보살필 자식을 둔 사람의 유언 권한이
법에 의해 이미 좁게 제한되고 있었음에도,
왜 그렇게 그것이 특별히 꺼리는 대상이 되어야 했는지 이해하기 어려운 것이다.
차라리, 오늘날의 프랑스처럼,
일반적으로 가족의 수장은 수고스럽게 유언장을 작성하느니
법이 예정한대로 재산이 분배되도록 내버려두는 현상까지도
기대할 수 있을 법한 지경이었다.
하지만 생각건대,
유스티니아누스 이전 시대의 \wi{무유언상속}의 척도를 좀 더 면밀히 들여다보면
이 미스테리를 해결할 열쇠를 발견할 수 있을 것이다.
당시의 법은 두 가지 독립된 부분으로 직조되어 있었다.
하나는 로마의 보통법이라 할 수 있는 \wi{시민법}\latin{ius civile}에서
유래한 법규칙들이고, 다른 하나는 \wi{법무관}의 고시에서 유래한 것들이었다.
앞서 다른 목적에서 언급했듯이,
시민법은 오직 세 단계의 상속순위를 차례로 인정할 뿐이다.
\wi{부권면제}\hanja{父權免除}되지 않은 자녀, 최근친 \wi{종족}\hanja{宗族},
그리고 \wi{씨족}원들\latin{gentiles}이 그것이다.
법무관은
이들 세 가지 상속인 집단들 사이에
시민법에서 전혀 고려되지 않는
여러 친족 집단들을 끼워넣는다.
나중에
이러한 고시법과 이러한 시민법이 결합하여
근대 법전들에 널리 전해진 것과 사실상 다르지 않은 상속순위를
형성하게 되는 것이다.

\para{유언과 자연적 감정}
한때 무유언상속이 시민법으로만 규율되고
고시법은 아직 존재하지 않던, 또는 일관되게 적용되지 않던
시기가 있었다는 것을 기억해야 한다.
유년기의 법무관법은 강력한 저항에 부딪쳐 고전하였을 것이 틀림없다.
또한
대중의 여론과 법감정이 법무관법을 용인한 이후로도 오랫동안,
그것이 정기적으로 도입하는 변경은
확고한 원리에 의해 지배되기보다는
개별 법무관들의 서로 다른 견해에 따라 동요하였을 것이 거의 확실하다.
이 시기의 로마인들이 적용한 \wi{무유언상속}법은, 생각건대,
로마 사회가 그렇게 오랫동안 유지했던 무유언상속에 대한 강렬한 혐오를
설명---아니, 설명 이상의 것을---할 수 있을 것이다.
상속의 순위는 다음과 같았다:
로마 시민이 유언 없이 또는 무효인 유언만 남기고 사망하면,
\wi{부권면제}되지 않은 자녀가 그의 상속인이 된다.
\hemph{부권면제된} 아들은 상속분이 전혀 없는 것이다.
사망시 살아있는 직계비속을 남기지 않았다면,
최근친 \wi{종족}\hanja{宗族}이 상속한다.
망자와 \paren{아무리 가깝더라도} 여계\hanja{女系}로 이어지는 친족은
한푼도 받지 못하는 것이다.
종족 이외의 다른 모든 가계도상의 가지\hanja{[枝]}들은 배제되고,
결국 상속재산은 최종적으로 씨족원들\latin{gentiles}, 즉
망자와 동일한 이름을 가진 로마 시민들의 집단 전체에게 귀속된다.
그리하여, 유효한 유언이 행해지지 않았다면,
지금 고찰대상인 시대의 로마인은 부권면제된 자식에게 생계를 위해
아무 것도 남겨주지 못하게 된다.
또한 만약 자식 없이 사망한다면,
그의 재산은 가족을 완전히 떠나, 단지
동일 \wi{씨족}\latin{gens}의 모든 구성원을 공통 조상의 후손으로 만드는
제사\hanja{祭祀}라는 의제\hanja{擬制}에 의해서만 연결될 뿐인
다수의 사람들에게 넘어가게 될 위험이 즉각 발생한다.
바로 이 문제점만으로도 저 대중적 감정에 대한 거의 충분한 설명이
될 수 있을 것이다.
그러나 실은,
독립된 가\hanja{家}들로 구성된 원시적 구조로부터
로마 사회가 막 벗어나기 시작한 바로 그 단계에
내가 서술해온 사회 상태가
놓여있었을 것이라는 점을
생각하지 못한다면
아직 우리는 절반만 이해한 것에 불과하다.
사실,
부권면제가 합법적 관행으로 인정받으면서
가부장의 제국에
최초의 일격이 가해졌다.
그러나 가부장을 여전히 가족관계의 뿌리로 간주했던 법은
\wi{부권면제}된 자식을 친족적 권리에 있어 남으로,
혈연에 있어 이방인으로 취급하고 있었다.\footnote{부권면제된 자식은
  종족(宗族)에 속하지 않고 단지 혈족(血族)에 속할 뿐이었다.}
하지만, 가족범위의 한계가 법기술적으로 부과되었다고 해서
부모의 자연적 감정에도 동일한 한계가 있었다고는 조금도 생각할 수 없는 것이다.
\wi{가부장제} 아래에는
가족애가 거의 무한한 신성함과 강력함을 유지한 채
흐르고 있었다.
부권면제 행위에 의해 가족애까지 소멸된다는 것은
상상하기 어려우며, 오히려 완전히 그 반대였을 확률이 높다.
부권으로부터의 해방이 친애\hanja{親愛}의 단절이기는커녕
표현---가장 사랑하고 가장 우대하는 자식에 대한 은혜와 호의의 표시---이라고
주저없이 인정할 수 있는 것이다.
이렇게 다른 자식들보다 존중받는 아들이 \wi{무유언상속}에 의해
유산에서 전적으로 배제된다면,
이것을 꺼리게 되는 것은 더 이상의 설명을 요하지 아니한다.
지금까지
우리는
무유언상속법이 초래하는 어떤 도덕적 부정의로 인해
\wi{유언상속}에 대한 열정이 생겨났다는 추론을 진행해왔다.
그리고 무유언상속법이 바로 저 초기사회를 묶어주었던 본능과
배치되고 있었음을 알게 되었다.
지금까지 주장한 것을 어떤 간명한 행태로 요약할 수 있을 것이다.
원시 로마인들을 지배한 모든 감정은
가족관계와 밀접하게 엮여있었다.
그런데 어떤 가족을 말하는가?
법은 이 가족을, 자연적 감정은 저 가족을 말하고 있었다.
양자 간의 갈등에서 우리의 분석대상인 열정이 자라난 것이다.
그것은
친애가 지시하는 바대로 목적 재산의 운명을 정할 수 있게 한 제도에 대한 열정의
형태로 나타났다.

따라서 나는 무유언상속에 대한 로마인들의 두려움을
가족에 관한 고대법과 서서히 변화할 뿐인 고대적 감정 간의 옛 갈등의 유물이라고
이해한다.
몇몇 로마 제정법들이, 그중에서도 특히
여자의 상속권을 제한하는 법률 하나가,\footnote{%
  기원전 169년의 보코니우스 법(Lex Voconia)을 말하는 듯하다.
  이는 10만 아스 이상의 재산을 가진 자는 여자를 상속인으로 삼을 수
  없도록 한 법률이다. \latin{Gai.\,2.274.}}
통과됨으로써
저 열정이 계속 유지되는 데 기여하였을 것이다.
그리고 이러한 제정법들이 부과한 금지를 회피하기 위해
\wi{신탁유증}\hanjalatin{信託遺贈}{fideicommissa}%
\footnote{시민법상의 유증(legatum)이 갖는 엄격한 제한을 회피하기 위해
  널리 이용된 것으로, 상속재산의 일부 또는 전부를 제3자에게 이전하도록
  상속인에게 지시했다.
  이로써 가령 시민이 아닌 자에게 유증하거나,
  상속인의 상속인을 지정할 수 있었다.
  원래는 법적 강제력이 없었으나,
  아우구스투스 시절부터는 비상심리절차를 이용해 강제할 수 있었다.
  나중에는 유언을 대체하는 비공식적 유언으로 기능하기도 했다.
  유스티니아누스에 의해 유증과 신탁유증은 하나로 통합된다. }%
이라는 제도가 고안되었다는 것이 정설이다.
하지만 저 열정의 특별한 강렬함에 비추어볼 때,
그것은 법과 여론 사이의 어떤 더 깊은 대립관계를 지시하고 있는 듯하다.
그러니 법무관법의 발달에 의해 저 열정이 완전히 소멸하지 않은 것도
놀라운 일이 아니다.
여론의 철학에 능통한 사람이라면
어떤 감정을 만들어낸 상황이 사라졌다고 해서 반드시
그 감정까지 소멸하는 것은 아님을 잘 알 것이다.
감정은 더 오래 살아남을 수 있다.
아니, 상황이 아직 지속될 때는 볼 수 없었던 최고도의 강렬함이
상황이 사라진 후에 나타날 수 있다.

\para{중세의 유언, 프랑스의 유언법}
재산을 가족 바깥으로 유출하는 권한,
혹은
유언자의 임의대로
불균등하게 재산을 분배하는 권한으로서의 유언 관념은
\wi{봉건제}가 완전히 공고해진 중세 후반기에 이르러 비로소 등장한다.
근대법이 처음 그 거친 모습을 드러내기 시작했을 때,
유언법은 망자의 재산 처분의 절대적 자유를 거의 인정하지 않았다.
이 시기, 유언에 따른 재산의 승계가 인정되는 지역 어디서나---유럽
대부분 지역에서 동산 또는 \wi{인적재산}\hanja{人的財産}은 유언 처분의 대상이 될 수
있었다---승계되는 재산 중 과부에게 주어지는 일정한 몫의 권리\footnote{%
  13세기 이래 영국에서 아내의 인적재산은 혼인하면 남편의 것이었다. 그러나
  남편의 유증에도 불구하고 과부는 수유자를 상대로
  자신의 신분에 걸맞는 개인 의복과 식기 등 생필품과 장신구를
  요구할 수 있었다.
  이러한 처의 특별재산을 `파라페르날리아'(paraphernalia)라고 부른다.
  다만 장신구는 남편의 금전채무에 대해 책임을 부담했다.
}와
자식들에게 주어지는 일정한 비율의 권리를
침해하여 유언 권한을 행사하는 것은 거의 허용되지 않았다.
자식들의 몫은, 그 양에서 알 수 있듯이,
로마법의 선례를 따른 것이었다.\footnote{%
  영국에서 인적재산의 유언---물적재산의 유언은
  1540년 유언법(Statute of Wills)에 이르러 비로소 인정된다---에 있어
  유류분은 지방마다 달랐으나 대체로
  과부와 자녀 모두 생존하고 있다면 각 $1/3$씩이고
  과부만 혹은 자녀만 생존하고 있다면 $1/2$이었다.
  자녀가 여러 명이면 $1/3$ 또는 $1/2$을 균분했다.
% : Polloc \& Maitland, The History of English Law before the Time of Edward I, V. 2, Ch. 6, § 3.
  한편 로마법의 유류분 비율은
  전술한 배륜유언의 소의 예에 따라
  원래 무유언상속분의 $1/4$이었다가
  유스티니아누스 신칙법에 의해 네 자녀까지는 $1/3$,
  다섯 자녀부터는 $1/2$이 되었다.
}
과부를 위한 대비책은 교회의 노력에 기인한 것이다.
교회는 과부의 이익을 위한 배려를 멈추지 않았고
마침내 가장 힘겨운 승리 중의 하나를 쟁취했다.
혼인식에 임한
남편에게
아내에게 나누어줄 것을 명시적으로 약속하도록
2, 3백년에 걸쳐
압박한 결과,
서유럽 전역의 \wi{관습법}에 \wi{과부산}\hanjalatin{寡婦産}{dower}의 원리를
집어넣는 데 성공하였던 것이다.
놀랍게도, 이러한 부동산의 과부산이
그에 대응하는 더 오래된,
인적재산에 대한 과부 및 자식들의 유류분\hanja{遺留分}보다
더 안정적인 제도였음이 드러났다.
프랑스 일부 지역의 관습법은 \wi{프랑스혁명}기까지 이 권리를 유지했고,
영국에서도 유사한 관행의 흔적이 남아있는 것이다.
그러나
동산은 유언에 의해 자유롭게 처분할 수 있다는 법리가
전반적으로 지배하게 되었고,
과부의 권리가 계속 존중되고 있었던 때에도
자식들의 권리는 법에서 사라져갔다.
이러한 변화의 원인을 \wi{장자상속제}\latin{primogeniture}의 영향에서
찾는 데 주저할 필요는 없을 것이다.
봉건 토지법이 한 명을 제외한 다른 자식들을 상속에서 사실상 제외함에 따라,
종전에 균등하게 분배되었던 종류의 재산조차도
더 이상 그것의 균등한 분배를 의무로 여겨지 않게 되었다.
유언은 불균등 분배를 만들어내는 주요한 수단이었고,
이러한 상황 속에서
고대와 근대의 유언 관념 간에 미묘한 차이가 나타났다.
유언을 통해 행사되는 유증의 자유는 이렇게 \wi{봉건제}의 우연한 산물이었지만,
그러나
유언에 의해 자유롭게 처분되는 재산의 체계와
봉건 토지법에서처럼
정해진 계통을 따라 강제적으로 상속되는 재산의 체계
간에 존재하는 차이만큼 큰 차이도 아마 없을 것이다.
프랑스 법전의 입안자들은 이 진리를 보지 못한 것 같다.
그들은
\wi{가족승계적 재산설정}\latin{family settlement}%
\footnote{재산이 가족 바깥으로 유출되지 못하도록 하는 부동산권 설정.
  한번의 설정에서 여러 개의 연속적 토지보유권을 창설함으로써 만들어진다.
  가령 \latin{`to W for life,
  remainder to X in fee tail,
  remainder to Y in fee tail,
  remainder to Z in fee simple'} (W는 아내, X, Y, Z는 아들들로 나이순) 같은 것.
  그러나 아래 한정승계부동산권도 넓은 의미의 가족승계적 재산설정으로 볼 수 있다.
}%
에 주로 기초한
장자상속제를
파괴해야 할 사회 구조의 일부라고
보았을 뿐만 아니라,
유언도
가장 엄격한 한정승계부동산권\latin{entail}%
\footnote{직계비속 또는 직계비속남성(또는 여성)에게만 상속될 수 있도록
  설정된 부동산권.
  \latin{`estate in fee tail'} 또는 간단히 \latin{`fee tail'}이라고도 부른다.
  \latin{``to A and the heirs (male) of his body''} 따위의 말로써 수여된다.
  }
하에서 장자에게 주어진 것과
동일한 우선권을
장자에게
주기 위해 자주 사용되는 것이라고 보았다.
따라서,
원하는 바를 이루기 위해 그들은
부부재산계약에서 다른 자식들에 비해 장자를 우대하는 것을 금지했을 뿐만 아니라,
상속재산이 자식들 사이에 균등하게 분배되도록 한 원칙을 회피하는 데
유언이
사용되지 않도록
\wi{유언상속}을 법에서 거의 추방해버렸다.
결과적으로 그들은
유증의 자유가 완전히 인정되는 체계보다는
봉건제 하의 유럽의 체계에 무한히 가까운
일종의 작은 영구적 한정승계부동산권 체계를 만들어냈다.
물론
``\wi{봉건제}의 헤르쿨라네움''인 영국의 부동산법은
대륙 국가들의 그것보다 중세 부동산법에 훨씬 가깝다.\footnote{%
  헤르쿨라네움은 베수비우스 화산의 폭발로 폼페이와 함께 매몰된 고대 도시.}
또한 영국에서는
부동산에 관한 부부재산계약에 거의 보편적으로 등장하는
장자 및 그의 계통이 갖는 우선권을 조장하거나 흉내내는 데
유언이
자주 사용된다.
그럼에도 불구하고 영국인들의 법감정과 여론은
자유로운 유언 처분의 관행에 의해 크게 영향을 받았다.
내가 보기에,
가족 내에 재산을 보존하는 문제에 관한
대부분의 프랑스 사회의 법감정의 상태는
현재 영국인들의 여론보다는 2, 3백년 전 유럽을 지배했던 것에
훨씬 더 가까운 것 같다.

\para{장자상속제}
장자상속제를 언급하였거니와,
이것은 법제사의 가장 어려운 문제의 하나를 제기한다.
자세히 설명하지는 않았지만,
로마 상속법과 관련하여
단일한 상속인과 나란히
다수의 ``공동상속인''을
수차
언급했던 것을 기억하실 것이다.
사실, 로마법 역사 전체에 걸쳐
공동상속인 집단이 상속인, 즉 포괄승계인의 지위를
가질 수 없었던 때는 한 번도 없었다.
이 집단은 하나의 단위로서 상속했고,
그후 상속재산은 별도의 법적 절차를 통해 그들 사이에 분할되었다.
\wi{무유언상속}의 경우,
이 집단이 망자의 자식들로 이루어진다면
그들은 균등한 몫으로 재산을 나누어가졌다.
남성들이 여성들에 비해 약간의 우대를 받는 때도 있었지만,
\wi{장자상속제}의 흔적은 조금도 발견되지 않는다.
이러한 분배방식은 초기법을 통털어 일관된다.
실로, 국가사회가 시작되고
여러 세대의 가족이 하나로 모여 살기를 그친 무렵
인간의 자연스러운 관념은
재산을 각 세대의 구성원들 간에 균등하게 분할하여
장자나 그의 계통에 어떠한 특권도 부여하지 않는 것이었다고 보여진다.
이러한 현상과
원시적 사고 간의
밀접한 관계에 관하여
특별히 중요한 힌트 몇 가지를
로마보다 더 오래된 법체계들에서 발견할 수 있다.
인도인들 사이에서는 아들이 태어나면
그는 아버지의 재산에 대해 확정적 권리를 가지지만,
공동소유자들의 승인 없이는 이것을 팔아버릴 수 없다.
아들이 성년에 이르면,
아버지의 반대에도 불구하고
그는 가산의 분할을 때로 강제할 수 있고,
아버지가 묵인한다면
한 아들은 다른 아들들의 반대에도 불구하고
항상 분할을 강제할 수 있다.
그러한 분할이 일어날 때,
아버지는
자식들 몫의 두 배를 가져가는 것 외에는
자식들보다 더 우대받지 않는다.
게르만의 고대 부족법도 대단히 유사하다.
\wi{자유소유지}\latin{allod}, 즉 가\hanja{家}의 소유지는
아버지와 아들들의 공동소유였다.
하지만 이것은 아버지의 사망시에도 쉬이 분할되지 않았던 것으로 보인다.
마찬가지로 인도인의 토지도,
이론적으로는 분할가능하지만,
실제로는 좀처럼 분할되지 않아서
수 세대 동안 분할 없이 상속되는 일이 흔하다.
그리하여
인도의 가족은 끊임없이 \wi{촌락공동체}\latin{village community}로
확장되는 경향을 보이거니와,
어떤 조건 하에서 그러한지는
추후 설명할 것이다.
이 모든 것들은
아버지의 사망시 아들들 간의 철저한 재산 균등 분배가
가족종속성이 해체되기 시작할 무렵의 사회에 나타나는
일반적 관행이었음을
사뭇 명료하게 지시하고 있는 것이다.
여기서 장자상속제라는 법제사적 난제가 등장한다.
봉건제가 형성되고 있을 무렵,
한편으로 로마 속주들의 법과 다른 한편으로 만족\hanja{蠻族}들의 옛 관습 외에는
세상 어디에도 \wi{봉건제}를 이루는 요소들의 원천이 될 만한 것이 없었음을
분명히 인식하면 할수록,
우리는 일견 더욱더 당혹스러움에 빠져들지 않을 수 없거니와,
로마인들도 만족\hanja{蠻族}들도 재산상속에서 장자나 그의 계통에
아무런 우선권도 주지 않고 있었다는 사실을 잘 알게 되었기 때문이다.

\para{은대지와 봉토, 자유소유지와 봉토}
만족\hanja{蠻族}들이 로마제국 내에 처음 정착했을 때
그들의 관습은 \wi{장자상속제}가 아니었다.
그것의 기원은 제국을 침공한 수장들이 나누어준
\wi{은대지}\hanjalatin{恩貸地}{benefice}에 있다는 것이 정설이다.
초기 이주민 국왕들에 의해 이따금 주어진,
그러나 샤를마뉴에 의해 대규모로 주어진,
이 은대지는 수혜자의 군사적 봉사를 대가로
로마 속주의 토지를 나누어준 것이었다.
\wi{자유소유지} 소유자들은 그들 군주의 멀고 험난한 원정에
잘 따라나서려 하지 않았던 것으로 보이며,
프랑크의 수장들이나 샤를마뉴의 대규모 원정은
모두
왕실에 복속되어있는 군인들이나
토지 보유의 대가로 봉사에 나서야했던 군인들로
군대를 구성하여 수행되었다.
하지만 은대지는 처음에는 결코 세습적인 것이 아니었다.
수여자가 원하면 언제든 되돌려주어야 하는 것이거나,
기껏해야 수혜자의 생애 동안만 보유할 수 있는 것이었다.
그러나 처음부터 수혜자들은 토지보유를 늘리는 데, 그리고
사망 후에도 가족들이 그 토지를 계속 보유토록 하는 데
모든 노력을 아끼지 않았다.
샤를마뉴 이후 허약한 후계자들이 속출하자,
그들의 노력은 예외 없이 성공을 거두었고
\wi{은대지}는 점차 세습봉토\latin{fief}로 변모해갔다.
그러나 세습적이었다고는 해도 그것이 반드시
장자에게 상속되었다는 말은 아니다.
상속의 규칙은 전적으로 수여자와 수혜자 간에 맺어진,
혹은 그들 중 강자가 약자에게 강요한,
약정에 의해 결정되었다.
따라서 애초에는 \wi{토지보유권}이 무척이나 다양했다.
물론 지금까지 서술한 로마의 상속방식과 만족\hanja{蠻族}들의 상속방식이
결합한 것이기에 완전히 무작위적인 것은 아니었으나,
그래도 대단히 잡다한 양상이었다.
어떤 곳에서는 분명 장자와 그의 계통이 우선하여 봉토를 상속했으나,
그러한 상속은 보편적이기는커녕 일반적이지도 않았던 것으로 보인다.
정확히 동일한 현상이
보다 후대에 일어난 유럽 사회의 변화,
즉 \paren{로마의} 완전소유지\latin{domainia}와
\paren{게르만의} \wi{자유소유지}\latin{allodial}가
봉건적 토지보유로 대체되는 과정에서도
반복되었다.
자유소유지는 완전히 봉토로 흡수되어갔다.
대규모 자유소유지 소유주들은
그들 땅의 일부를 종자\hanja{從者}들에게 조건적으로 양도함으로써
봉건영주가 되어갔다.
소규모 자유소유지 소유주들은
그들의 땅을 어떤 힘있는 수장에게 양도하고
전쟁시 복무한다는 조건으로 다시 되돌려받음으로써
험악한 시절의 압박으로부터 벗어나고자 했다.
그러는 동안,
서유럽 인구의 대다수를 차지하는
예속적 또는 반\hanja{半}예속적 신분의 사람들---로마와 게르만의 노예들,
그리고 로마의 콜루누스\latin{coloni}와 게르만의 리두스\latin{lidi}---도
동시에
봉건 조직에 흡수되어 갔으니,
그들 중 일부는 영주의 하인이 되었으나
대부분은 당시 굴욕적이라 여겨진 조건으로 땅을 하사받았다.
이렇게 보편적 봉건화가 진행되는 동안 형성된 토지보유권은
토지보유자가 그들의 새로운 수장들과 맺은, 또는
맻도록 강요당한 조건에 따라 사뭇 다양했다.
\wi{은대지}의 경우처럼,
전부가 아닌 일부 \wi{토지보유권}만이 장자상속의 규칙을 따랐다.
하지만 \wi{봉건제}가 서유럽 전역을 지배하게 되자,
다른 상속방식보다 \wi{장자상속제}가 큰 장점을 갖는 것임이 명백히 드러났다.
그것은 놀라운 속도로 전 유럽에 퍼져나갔거니와,
확산의 주요 수단은 영국의 \wi{가족승계적 재산설정}\latin{family settlement},
프랑스의 팍트 드 파미유\latin{pacte de famille},
독일의 하우스게제츠\latin{Hausgesetz} 따위였으니,
이들은 모두 기사\hanja{騎士} 봉사의 대가로 보유한 토지를
장자가 상속하도록 설정한 것이었다.
마침내 이 규칙은 만성적인 관행으로 굳어져,
서서히 형성되어온 모든 \wi{관습법} 체계에서
장자와 그의 계통이
자유민의
군역\hanja{軍役}토지보유권의 상속에서 우선권을 가지게 되었다.
예속적 토지보유의 경우 \paren{처음에는 보유자가 금전을 지불하거나
노역을 제공할 의무가 있는 모든 보유지가 예속적이었다}
관습법상의 상속체계는 나라에 따라 그리고 지방에 따라 사뭇 달랐다.
그나마 일반적이라 할 만한 규칙은 이런 성격의 토지는 자식들 간에
균분상속하는 것이었으나, 그래도 어떤 곳에서는
장자\hanja{長子}가 우대되었고 어떤 곳에서는 말자\hanja{末子}가 우대되었다.
그러나,
영국의 농역\hanja{農役}\wi{토지보유권}\latin{socage}\footnote{%
  일정한 지대의 정기적 제공을 조건으로 보유하는 토지보유양태.
  처음에는 주군의 토지에서 특정 농사일을 제공하는 것을 조건으로 했다.
}처럼,
비교적 나중에 등장하였고 완전히 자유롭지도 완전히 예속적이지도 않은
유형의 토지보유권의 상속은 가장 중요한 몇몇 측면에서
통상적으로 장자상속제에 의해 규율되었다.

\para{정치적 장자상속제}
\wi{장자상속제}의 확산을 설명하기 위해 일반적으로 거론되는 것은
이른바 `\wi{봉건제}적' 근거이다.
봉건관계의 주군\hanja{主君}의 입장에서는
최후 보유자의 사망으로
봉토가
여러 명에게 분산되는 것보다
한 명에게 상속되는 것이 군사적 봉사를 안정적으로 확보하는 데
더 낫다는 것이다.
이런 이유가 장자상속제의 점진적 확산에 대한 부분적인 설명이 될 있음은
부인할 수 없지만,
장자상속제가 유럽 전역의 관습이 된 데에는
주군이 누리는 이익보다 토지보유자들 사이에서의 인기가 더 크게
작용했다는 점을
지적하지 않을 수 없을 것이다.
더욱이, 전술한 이유는 장자상속제의 기원을 전혀 설명하지 못한다.
편의성에 대한 감각만으로는 어떤 법제도도 생겨나지 않는다.
반드시 미리 어떤 관념들이 존재하고, 여기에 편의성의 감각이 작용하여
어떤 새로운 결합이 형성될 수 있을 따름인 것이다.
이 관념들이 무엇인지 찾는 것이 지금 우리에게 주어진 과제이다.

이에 관한 암시가 풍부한 어떤 지역에서 유용한 힌트를 얻을 수 있겠다.
인도에서는
아버지의 사망시, 또는 아버지가 살아있을 때에도,
그의 재산이 아들들 간에 균분으로 분할될 수 있거니와,
이러한 \hemph{재산}의 균분 원칙은 힌두법의 모든 법제에
두루 퍼져있다.
하지만 최후 보유자의 사망으로 \hemph{공직} 또는 \hemph{정치권력}이
이양되는 경우, 상속은 거의 보편적으로 장자상속의 규칙을 따른다.
그리하여 통치권은 장자에게 세습된다.
또한 인도 사회를 구성하는 단위체인 \wi{촌락공동체}의 업무가
단일한 관리자에게 맡겨져있는 경우,
아버지의 사망시
일반적으로 장자가
관리업무를 이어받는다.
실로, 인도의 공직은 세습되는 경향이 있으며, 또한
성질상 허용된다면
가장 손윗 계통의 가장 연장인 구성원에게 주어지는 경향을 보인다.
이러한 인도의 상속제도를
유럽에 거의 오늘날까지 남아있는 몇몇 보다 미개한 사회조직과
비교해보면,
\wi{가부장권}이 \hemph{가족내부적}인 것을 넘어 \hemph{정치적}인 것일 때
그것은 아버지의 사후 모든 자식들에게 고루 분배되는 것이 아니라
장자의 생득권\hanja{生得權}이 된다는 결론이 자연스레 도출된다.
가령 스코틀랜드 산악지대의 \wi{씨족}장의 자리는
\wi{장자상속제}를 따른다.
거기서는, 사실, 조직된 국가사회의 초기 기록에서 발견되는 것보다
더 오래된 가족종속성의 한 형태가 남아있는 듯하다.
옛 로마법에서 보이는 저 종족\hanja{宗族}연합은,
그리고 다른 많은 유사한 징후들도,
가계도의 이리저리 뻗어있는 모든 가지들이 한때
하나의 단일한 유기체로 결합되어 있던 때가 있었음을 암시한다.
그리고 이렇게 결합된 친족단체가 그 자체로 하나의 독립된 사회였을 때
그것은 가장 손윗 계통의 가장 연장인 남자에 의해 통치되었다고 보더라도
지나친 억측은 아닐 것이다.
물론 우리는 그러한 사회에 대한 실제적 증거는 갖고 있지 않다.
우리가 알고 있는 가장 초보적인 사회에서도
가족 조직은 기껏해야 `통치권 안의 통치권'\latin{imperia in imperio}인 것이다.
그러나 그들 중 일부, 특히 켈트족 씨족의 상태는
역사 시대에 속하면서도 독립적 상태에 사뭇 가까워서
그것이 한때 독립된 `통치권'\latin{imperia}이었다는 확신을,
그리고 장자상속제가 그 씨족장의 상속을 규율했다는 확신을 우리에게 심어준다.
하지만 현대적 법률용어가 불러오는 인상은 주의할 필요가 있다.
지금 우리는 인도 사회나 고대 로마법을 통해 알고 있는 그 어떤 것보다
훨씬 친밀하고 훨씬 엄격한 가족결합의 형태를 말하고 있는 것이다.
로마의 가부장이 가족재산의 가시적 관리인이었다면,
인도의 아버지가 아들들과의 공동소유자에 불과하다면,
저 순수한 가부장은 더더욱 공동재산의 관리인에 불과했을 것이 분명하다.

\para{카롤링거 제국의 몰락}
따라서
은대지에서 보이는 장자상속의 예는
로마제국을 침공한 민족들이 가지고 있던, 그러나 보편적이었다고 할 수는 없는,
가족 통치권을 흉내낸 것이라 할 수 있다.
몇몇 보다 미개한 부족들이 여전히 그것을 행하고 있었기에,
혹은 더 그럴듯하기로는, 보다 원시적 상태를 거의 벗어나지 못하고 있었기에,
새로운 형태의 재산에 관한 상속규칙을 정해야 했을 때
몇몇 사람들은 자연스레 그것을 머리에 떠올렸을 터이다.
그러나 한 가지 문제가 아직 남아있다.
어째서 장자상속제가 점차 다른 상속원리들을 대체해갔는가? 하는 것이다.
생각건대, 그에 대한 답은
카롤링거 제국의 해체가 진행되는 동안 유럽 사회가 결정적으로 퇴보했다는 데 있다.
과거 만족\hanja{蠻族}들의 왕국 시절의
몹시도 저급한 수준보다 한 두 가지 점에서는 더 퇴보했던 것이다.
저 시대의 큰 특징은 왕의 권위가, 따라서 국가의 권위가,
허약했다는 것, 아니 차라리 부재했다는 것이다.
그리하여, 국가사회가 더 이상 결속하지 못하는 상황에서,
사람들은
국가공동체보다 더 오래된 사회조직에
너도나도 뛰어들었던 것으로 보인다.
9세기와 10세기 무렵, 봉신\hanja{封臣}들은 둔 주군은,
초기 사회에서와 같은 입양이 아니라
이제 \wi{수봉}\hanja{授封}에 의해 사람들을 충원하는,
일종의 가부장적 가\hanja{家}라고 보아도 좋을 것이다.
\wi{장자상속제}는
그러한 연합체제에
힘과 지속성의 원천이 되었다.
조직 전체가 의존하고 있는 토지가 하나로 결합되어 있는 한,
그것은 방어와 공격에서 큰 힘을 발휘했다.
땅을 분할하는 것은 안 그래도 작은 사회를 또 분할하는 것이고,
폭력이 만연한 시대에 스스로 공격을 불러오는 일이었다.
이러한 장자상속제의 선호에는
한 명을 위해 나머지 자식들을 모두 상속배제한다는 관념이
들어있지 않았다고
전적으로 확신할 수 있다.
봉토가 분할되면 모두가 고통받을 것이었다.
봉토의 결합으로 모두가 이득을 누렸다.
권력을 한 사람에게 집중시킴으로써 가족은 더 강해졌다.
그러하기에, 상속재산을 차지한 주군이
사용과 수익과 처분에 있어
그의 형제들과 친족들에 비해
더 큰 권한을 누렸다고 볼 수가 없는 것이다.
봉토의 상속인이 상속하는 특권을
영국의 엄격한 가족승계적 재산설정 하에서 장자가 누리는 것과
동일시하는 것은 완전히 시대착오적인 발상이라 하겠다.

\para{가부장이 소유권자가 되다}
나는 초기의 봉건 연합체를 원시적 가족 형태에서 유래한 것으로,
또
그것과 강한 유사성을 갖는 것으로 본다고 말했다.
그렇지만 고대 세계에서는,
그리고 봉건제의 시련을 겪지 않은 사회에서는,
\wi{장자상속제}가 지배적이었더라도 그것은 나중에 유럽 봉건제에서 보이는
장자상속제로 결코 전환되지 않았다.
친족집단이 각 세대마다 한 명의 세습 수장에 의해 통치되던 시대가 마감되자,
모두를 위해 관리되던 토지가 이제 모두에게 균등하게 분할되기
시작한 것으로 보인다.
왜 봉건제의 세상에서는 이런 일이 일어나지 않았을까?
봉건제 초기의 혼란 속에서는 장자가
전체 가족을 대신해서 토지를 보유했다 하더라도,
유럽에서 봉건제가 공고해진 이후에는,
정상적인 공동체가 다시 수립된 이후에는,
왜 가족들이
로마도 게르만도 다 가지고 있던 균분상속의 권리를
회복하지 못했을까?
봉건제의 계보를 추척해온 학자들은
이 난제를 해결할 열쇠를 거의 찾지 못했다.
그들은 봉건제를 구성하는 재료들은 발견했지만
그것들을 연결시키는 데 실패했다.
봉건제의 형성에 기여한 관념들과 사회형태들은
분명 미개하고 원시적인 것이었음에도,
법원과 법률가들이 그것을 해석하고 정의하는 일에 소환되자
그들이 적용한 해석원리는 최후의 로마법의 그것이었고
따라서 자못 세련되고 성숙한 것이었다.
가부장에 의해 통치되는 사회에서
장자는 종족\hanja{宗族}집단의 통치권을,
그리고 집단의 재산에 대한 절대적 권한을 상속할 수도 있을 것이다.
그러나 그렇다고 해서 그가 진정한 소유권자인 것은 아니다.
소유권 개념에는 들어있지 않은,
사뭇 불확정적이고 사뭇 정의 불가능한,
상응하는 의무도 그는 부담한다.
하지만 후기 로마법은, 오늘날의 법과 마찬가지로,
재산에 대한 무제약적 권한을 소유권과 등치시켰기에,
흔히들 법이라고 부르는 것이 등장하기 이전 시기에 속하는
그러한 책임 개념을 알지 못할 뿐만 아니라 사실 알 수도 없는 것이다.
세련된 관념과 미개한 관념의 접촉은 불가피
장자를 상속재산의 법적 소유권자로 전환시키는 결과를 낳았다.
교회법학자들과 세속법학자들은 장자의 지위를 처음부터 그렇게 정의했다.
그러나 부지불식간에 그의 연하 형제는
친족의 모든 위험한 일과 즐거운 일에 동등하게 참여하던 것에서
사제로, 용병으로, 영주 저택의 식객으로 그 지위가 조금씩 떨어져갔다.
이와 동일한 성격의 법적 혁명이
최근에
보다 작은 규모로
스코틀랜드 산악지역 대부분에 걸쳐 일어났다.
\wi{씨족}의 생계를 책임지던 토지에 대한 씨족장의 법적 권한을
결정하도록 소환되었을 때, 스코틀랜드법은
토지지배권의 완전성이
씨족원들의 권리에 의해 막연하게나마 제한된다는 것을
알아차릴 수 있는 시대를 한참 지나있었고,
따라서 다수의 재산은 일인\hanja{一人}의 재산으로 불가피 전환되었던 것이다.

\para{장자상속의 형태들}
설명을 단순화하기 위해,
나는
어떤 가\hanja{家}나 단체의 권위를
한 명의 아들 또는 자손이
상속할 때
이것을 \wi{장자상속제}라고 불러왔다.
하지만,
이런 상속유형의 고대 사례를 보여주는 소수의 남아있는 기록들 중에는
반드시 우리가 생각하는 그러한 장자가 대표자의 자리를 차지하는 것은
아닌 경우도 있음을 유의해야 한다.
서유럽에 널리 퍼진 장자상속의 형태가 인도인들 사이에서도 준행되어왔거니와,
그것이 통상적인 형태라는 것은 의심의 여지가 없다.
여기서는 장자뿐만 아니라 가장 손윗 계통이 항상 우대된다.
만약 장자가 먼저 죽고 없다면, 그의 장자가 다른 형제들뿐만 아니라
삼촌들에 대해서도 우선하는 것이다.
그러나 단순히 \hemph{민사적} 권력이 아니라 \hemph{정치적} 권력의 상속이
걸려있을 때는 곤란한 문제가 대두될 수 있거니와,
이 문제는 사회의 결속력이 불완전할수록 더 커질 것으로 짐작된다.
마지막으로 권력을 가졌던 수장이 그의 장자보다 오래 살았고,
일차적 상속권을 갖는
손자는 아직 너무 어리고 미성숙해서
공동체를 실제로 이끌고 관리할 수 있는 상태가 아닐 수 있는 것이다.
이런 경우, 어느 정도 안정된 사회에서 널리 사용되는 방식은
어린 상속인이 통치에 적합한 나이가 될 때까지 그를 \wi{후견} 아래 두는 것이다.
일반적으로 \wi{종족}\hanja{宗族} 남성이 후견인이 되지만,
다른 대안도 있을 수 있음에 유의해야 한다.
드물지만 어떤 고대 사회는 여자가 권력을 행사하는 데 동의했거니와,
이는 분명 어머니의 권리가 우선한다는 인식에서 나온 것이다.
인도에서는 통치자의 과부가 어린 아들의 이름으로 대신 통치한다.
또한 프랑스의 왕위계승에 관한 관습---그 기원이 무엇이건 이는
아주 오래된 고대의 유산임이 분명하다---에서는,
여성이 왕위에 오르는 것은 엄격히 배제하면서도,
다른 어느 누구의 섭정보다도
대비\hanja{大妃}의 섭정이
선호되었다는 것을
떠올리지 않을 수 없다.
하지만, 어린 상속인에게 통치권이 이전될 때 발생하는 문제점을 회피하는
또 다른 방법이 있거니와, 이는 미개한 구조를 가진 공동체에서
자연스럽게 일어나는 일이라고 보면 틀림없을 것이다.
어린 상속인을 완전히 제치고,
윗 세대의 가장 연장인 살아있는 남성이
수장 자리를 차지하는 것이 그것이다.
켈트족의 \wi{씨족}연합은
국가사회 또는 정치사회가 아직 초보적으로 분화되지도 못한 시대의
현상들을 다수 보존해왔거니와,
그 중에서도 특히
이러한 상속규칙을 역사 시대에 이르기까지 보존해왔다.
장자가 이미 죽고 없다면,
통치권 이양시 손자들의 나이를 전혀 불문한 채
손자들에 우선하여,
그 장자의 바로 밑 남동생이
상속한다는 것이
실정규범의 형태로 그들에게 존재해온 것으로 보인다.
혹자는
마지막 족장을 일종의 뿌리 또는 계통으로 파악하여
그에게 가장 가까운 후손에게 상속시키는 것이 켈트족의 관습이었다고
가정함으로써 이 원리를
설명한다.
삼촌이
손자보다
공통의 뿌리에 더 가까우므로 삼촌이 우선한다는 것이다.
이런 진술은 단지 상속규칙을 기술\hanja{記述}하는 목적만 가진다면
아무런 이의도 제기할 수 없을 것이다.
그러나
저 규칙을 처음 채용한 사람이
법률가들 사이에서
봉건제적 상속규칙이
논쟁의 대상이 되기 시작할 시대에
기원한 것이
분명한 추론과정을
적용했다고 본다면 이는
심각한 오류가 아닐 수 없다.
손자보다 삼촌이 선호되는 것의 진정한 기원은
어린 아이보다는 어른 족장이 다스리는 것이 더 낫다는,
그리고 장자의 자식들보다는
차자\hanja{次子}가 성숙기에 도달했을 확률이 더 높다는,
미개한 사회의 미개한 사람들의 단순한 계산이었을 것이 틀림없다.
뿐만 아니라,
우리가 잘 알고 있는 장자상속의 형태가 일차적 형태라는 것을 보여주는
증거도 있으니,
전승에 의하면
어린 상속인을 무시하고 그의 삼촌이 우대될 경우
씨족원들의 동의가 필요했던 것이다.
스코틀랜드의 맥도널드\latin{Macdonald} 씨족의 연대기에는
이러한 의식\hanja{儀式}의 사례가 전해지거니와,
그 진정성은 상당히 믿을 만하다.
또한 최근 연구자들의 해석에 따르면
아일랜드 켈트족의 유물들도 비슷한 관행의 흔적을 다수 보여준다고 한다.
인도의 \wi{촌락공동체}에서도
선거를 통해
``보다 훌륭한'' 종친\hanja{宗親}이
손윗 종친을
대체하는 일이 없지 않다.

\para{이슬람의 법}
아라비아 관습을 보존해온 것으로 보이는
\wi{이슬람 법}에 의하면,
상속재산은 아들들 간에 균등하게 분할되고,
딸들은 아들 몫의 절반을 갖는다.
그런데 상속재산 분할 전에 자식들 중 누군가가 자녀를 남기고 사망했다면,
이들 손자녀는 상속에서 배제되고 그들의 삼촌들이 재산을 독차지한다.
이 원리가 동일하게 적용되어,
통치권이 이양될 때의 상속도
켈트족 사회에서 행해져온 장자상속의 형태와
같은 형태를 따른다.
서양의 무슬림 가문 중 가장 큰 두 가문의 상속규칙은,
조카가 장자의 아들이라 할지라도, 조카에 우선하여 삼촌이 왕위를
계승하는 것이라고 믿어지고 있다.
이 규칙이 이집트와 투르크 양자 모두에서
최근까지 준수되어왔지만,
그러나
투르크의 통치권 이양에 관해서는 이 규칙은
늘 어떤 의문의 대상이 되어왔다고 한다.
술탄들의 정책으로 말미암아 그것의 적용이 사실상 널리 방해받았다는 것이다.
물론 연하의 남동생들을 대량 학살함으로써
왕좌를 둘러싼 위험한 경쟁자들을 제거할 수 있을 뿐만 아니라
자기 자식들의 이익도 확보할 수 있었을 것이다.
하지만 한 가지 확실한 것은
일부다처제 사회에서
\wi{장자상속제}는 반드시 다양한 형태를 띨 수밖에 없다는 것이다.
상속권 주장에는
가령 어머니의 순위라든가, 혹은 아버지의 총애를 얼마나 받는가 따위의
다양한 사항이 고려대상이 될 수 있다.
따라서 인도의 몇몇 무슬림 통치자들은,
유언 권한은 전혀 내세우지 않은 채,
자신을 계승할 아들을 지명할 권리가 자신에게 있다고 주장한다.
이삭과 그의 아들들 이야기와 관련하여 성서에 언급된
\hemph{축복}\latin{blessing}을 일종의 유언이라고 보는 이도 있지만,
그보다는 장자를 지명하는 방법의 하나였던 것으로 보인다.\footnote{%
  창세기 \latin{27:1--40}. }

