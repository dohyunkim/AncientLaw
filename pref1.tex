\chapter*{초판 서문}

본 저서의 주된 목적은
고대법에 반영된 인류의 몇몇 초기 관념들을 밝혀내고
그 관념들이 근대사상에 대해 가지는 관계를 지적하는 것이다.
로마법과 같은 법체계가 존재하지 않았더라면
본 저서에서 시도한 탐구의 많은 부분이 어떤 유용한 결실도
맺지 못했을 것이다.
그 법체계의
초기 부분은 아주 오래된 옛 것의 흔적을 담고 있고,
그것의 후기 규칙들은 오늘날에도 근대사회를 규율하고 있는
문명사회의 제도들의 근간을 제공해준다.
로마법을 전형적인 법체계로 삼아야했던 까닭에,
저자는 지나쳐 보일 정도로 많은 비율의 사례들을
로마법에서 가져왔다.
그러나 저자의 의도는 로마법에 관한 논저를 쓰는 것이 아니며,
또한 그런 외양을 보여줄 법한 논의는 되도록 피하고자 했다.
제3장과 제4장에서
로마 법학자들의 어떤 철학적 이론들에 많은 분량이 할애된 것은
두 가지 이유에서 그러하다.
첫째, 저자가 보기에 그 이론들은
이 세계의 사상과 실천에
흔히들 생각하는 것보다
훨씬 폭넓고 훨씬 항구적인 영향력을 끼쳤기 때문이다.
둘째, 이 저서가 다루는 주제들에 대하여
아주 최근까지 지배적이었던 견해의 대부분이
거기에서 유래한다고 생각되고 있기 때문이다.
이러한 사변\hanja{思辨}들의 기원과 의미와 가치에 관한
저자의 의견을 개진하지 않고는
저자의 작업은 조금도 앞으로 나아갈 수 없었다.

\begin{flushright}
H. S. M.
\end{flushright}

\begin{footnotesize}
런던: 1861년 1월.
\end{footnotesize}

