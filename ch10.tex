\chapter{불법행위법 및 형법의 초기 역사}

\para{고대 법전에서의 형법}
앵글로색슨 선조들의 법전을 포함한
튜턴족의 법전들은
초기의 법들 간의 비중을 정확히 알 수 있는 상태로
우리에게 전해지는
유일한 원시 세속법 법전들이다.
로마와 그리스 법전들의 현존하는 단편들은
그것들의 일반적 성격을 알려주기에는 충분하지만,
그 부분들 간의 정확한 양이나 비율을 파악하기에는 부족한 상태로 남아있다.
그럼에도 불구하고 전반적으로 보아
우리에게 전해지는 모든 고대법 집성들은
성숙한 법체계와 크게 다른 한 가지 특징을 가지고 있다.
민법 대비 형법의 비율이 대단히 큰 차이를 보이는 것이다.
게르만 법전들에서는 형법에 비해 민법 부분의 비중이 아주 작다.
드라콘의 법전이 규정한 포악한 형벌에 관한 전승\hanja{傳承}을 보더라도
이것 또한 마찬가지 성격이었던 것으로 보인다.
뛰어난 법적 재능과 애초 부드러운 풍속을 가졌던 사회의 작품인
12표법만이 유일하게
근대법과 비슷한 정도로 민법의 우위를 보여주지만,
불법행위의 구제방식이 차지하는 상대적 비중이
아주 크지는 않더라도 상당히 큰 편이다.
생각건대
오래된 법전일수록 형법이 더 풍부하고 더 상세하다고
주장할 수 있겠다.
이 현상은
처음으로 그들의 법을 성문화한 공동체들에 만연했던 폭력때문이라고
흔히
인식되어왔고 설명되어왔거니와,
대체로 정확하다고 할 수 있다.
입법자들은
그들 법전의 부분들의 비율을
야만적 생활에서 발생하는 특정 종류의 사건들의 빈도에 맞추었다는 것이다.
하지만
이런 설명은 불완전하다고 생각한다.
옛 집성들에서 민법이 상대적으로 황무지인 것은
이 논저에서 다룬 고대법의 다른 성격들과 밀접히 관련된다는 점을 기억해야 한다.
문명사회의 민법 부분의 십중팔구는
신분법, 물권 및 상속법, 그리고 계약법으로 이루어져있다.
그러나
이들 법분야는
사회적 결속의 유년기로 거슬러올라갈수록
더욱 좁은 범위로 축소될 수밖에 없음이 명백하다.
신분법\latin{law of status}에 다름아닌 인법\hanjalatin{人法}{law of persons}은
모든 형태의 신분이 가부장권에 함께 복속해 들어가있는 한,
아내가 남편에 대해,
아들이 아버지에 대해,
미성숙의 피후견인이 종친\hanja{宗親}인 후견인에 대해
아무런 권리도 갖지 않는 한,
아주 좁은 범위로 축소될 것이다.
마찬가지로
물건과 상속에 관한 법도
부동산과 동산이 가족 내에서 대물림되는 한,
설령 분배되더라도 가족 내에서 분배되는 한,
결코 풍부할 수 없을 것이다.
그러나 고대 민법의 가장 큰 빈틈은
언제나 계약법의 부재에 기인할 것이다.
몇몇 원시 법전은 계약법이 아예 없고,
다른 법전들에서는
선서\hanjalatin{宣誓}{oath}에 관한 복잡한 법이 계약법을 대신하고 있어서
계약 관련
도덕관념의 미성숙을 보여주고 있을 뿐이다.
그런데 이에 상응하는,
형법의 빈곤을 가져올 만한 이유는 없다.
따라서,
설령 제 민족들의 유년기가 언제나 무제약적 폭력의 시기였다고 말하는 것이
위험하다 할지라도,
왜 근대법의 민법 대비 형법의 관계가 고대 법전에서는 역전되는 것인지
우리는 여전히 이해할 수 있는 것이다.

\para{범죄와 불법행위}
나는 후대에 비해 원시법이 형법,
즉 \hemph{범죄}법\latin{\textit{criminal} law}에
큰 비중을 둔다고 말했다.
이 표현은 편의상 사용한 것이고,
실은
고대 법전을 살펴보면
비상한 양을 차지하는 저 법이
진정한 범죄법은 아님이 드러난다.
문명사회의 법은
국가나 공동체에 대한 침해와
개인에 대한 침해를 구분하는 데 일치하고 있다.
이렇게 구분된 두 종류의 침해를 우리는,
법학에서 항상 일관되게 이들 용어가 사용되고 있다고 자신할 수는 없지만,
범죄\latin{crimes; \textit{crimina}}와
불법행위\latin{wrongs; \textit{delicta}}라고
부를 수 있을 것이다.
그렇다면 고대 공동체의 형법은
범죄\latin{crimes}법이 아니라,
불법행위\latin{wrongs}---영국법 용어로는 토트\latin{torts}---법인 것이다.
피해자는 가해자를 상대로 통상의 민사소송을 제기하고,
승소하면 금전배상의 형태로 전보\hanja{塡補}받는다.
가이우스의 주해서 중에
12표법에 기초한 형법을 취급하는 부분을 펼쳐보면,
로마법이 인정하는 민사 불법행위의 첫 머리에
\hemph{절도}\latin{furtum}가 나오는 것을 볼 수 있다.
우리가 익히 \hemph{범죄}로만 취급된다고 여기는 침해가
\hemph{불법행위}로만 취급되고 있는 것이다.
절도뿐만 아니라
폭행 및 모욕\latin{assault}\footnote{%
  저자는 로마법상의 인격침해(iniuria)를 영어의 `assault'로 옮기고 있는 듯하다.
  로마법의 `인격침해'는 신체적 폭행뿐 아니라
  모욕, 명예훼손 등 인격적 침해도
  포괄하므로 본문에서는 `폭행 및 모욕'으로 번역했다.
}과 강도도
로마 법학자들은 영국법상의 불법침해\latin{trespass},
문서명예훼손\latin{libel}, 구두명예훼손\latin{slander}과
마찬가지로 취급한다.\footnote{%
  이들 세 가지 영미법상의 법개념들은 모두
  불법행위(tort)에 속하는 소송형식들이었다.}
이들 모두가 채권채무관계, 즉 `법의 사슬'\latin{vinculum juris}을
가져오고, 이들 모두가 금전배상으로 전보되는 것이다.
하지만 이 특징은 게르만 부족들의 법전에서 더욱 뚜렷하게 나타난다.
예외 없이 그것들은
살인에 대한 금전배상의 방대한 체계를 기술하고 있고,
거의 예외 없이
기타 덜 중대한 침해에 대한 방대한 배상체계를 기술하고 있다.
켐블\latin{John Mitchell Kemble} 씨에 따르면
``앵글로색슨법에서는 \paren{<<앵글로색슨>>, 1.177}
모든 자유인의 생명에 그의 신분에 따라 금액이 매겨져있었다.
또한 사람의 신체에게 가해질 수 있는 모든 상해에 대해,
그리고 그의 시민권, 명예, 평온에 대해 가해질 수 있는 거의 모든 침해에 대해
그에 상응하는 금액이 매겨져있었다.
그 금액은 우발적인 상황에 따라 가중된다.''\footnote{%
  John Mitchell Kemble,
  \textit{The Saxons in England: A History of the English Commonwealth
  Till the Period of the Norman Conquest},
  Vol. 1,
  London: Longman, Brown, Green, and Longmans, 1849,
  pp.\,276f.}
이들 배상금은 중요한 수입원이 되었을 것이 분명하고,
매우 복잡한 규칙이 그것에 대한 권리와 책임을 규율하고 있거니와,
전술했듯이, 귀속되는 사람의 사망으로 그것이 면책되지 않는다면,
어떤 특정한 상속규칙에 따라 상속되는 것이 일반적이다.
따라서,
국가가 아니라 개인을 피해자로 보는 것이
\hemph{불법행위}법의 기준이라면,
법의 유년기에는
형법이 아니라
불법행위법에 의존하여
시민들이
폭력이나 사기로부터 보호받았다고 주장할 수 있는 것이다.

\para{불법행위와 종교적 죄}
그리하여 원시법에서 불법행위는 방대한 양을 차지하고 있다.
또한 종교적 죄\latin{sin}도 원시법에 알려져있었음을 첨언해야 할 것이다.
튜턴족 법전들에 관해서는 이런 주장을 굳이 할 필요도 없거니와,
현존하는 이들 법전은 기독교도 입법자들에 의해
편찬되고 개정되었기 때문이다.
그러나 사실 비\hanja{非}기독교적인 옛 법전들도
일정한 작위 유형과 일정한 부작위 유형에 대해
신의 지시와 명령을 위반했다는 이유로 형벌을 부과한다.
아테네의 아레오파고스\latin{Areopagus} 원로회의가 집행한 법은
아마도 어떤 특별한 종교법전이었을 것이다.
로마에서도 일찍이
신관\hanjalatin{神官}{pontifical}법이
간통, 독신\hanja{瀆神}, 그리고 어쩌면 살인도
처벌했던 것이다.
따라서 아테네와 로마 국가에서는 \hemph{종교적 죄}를 벌하는 법이 있었다.
또한 \hemph{불법행위}를 벌하는 법도 있었다.
신에 대한 침해라는 개념이 전자의 법을 만들었고,
이웃에 대한 침해라는 개념이 후자의 법을 만들었다.
그러나 국가 또는 전체 공동체에 대한 침해라는 관념은
초기에는 진정한 의미의 형법을 만들지 못했다.

\para{범죄의 개념}
그렇다고 해서
국가에 대한 침해라는 간단하고도 기초적인 개념이 원시사회에
부재했다고 생각해서는 안 된다.
그보다는,
이 개념이 실현되는 특별한 방식이
초기에 형법의 성장을 가로막는 원인이었다고 보인다.
여하튼
로마 공동체는 자신이 침해당했다고 생각되면
개인이 침해당한 경우를 유추하여
완전히 똑같은 결과를 강제했거니와,
국가는 어떤 특별한 행위로써 개인 침해자를 응징했던 것이다.
즉, 로마 공동체의 유년기에는
국가의 안전과 이익을 중대하게 침해하는 모든 행위는
입법기관의 개별 입법에 의해 처벌했다.
그리고 이것이 범죄\latin{crimen}의 초창기 개념이었거니와,
국가가 사건을 민사법원이나 종교법정에 맡기지 아니하고
침해자에 대한 특별법\latin{privilegium}을 만들어 처벌할 정도로
중대한 문제를 야기하는 행위가
바로 그것이다.
따라서 모든 기소는
`형벌법안'\latin{bill of pains and penalties}의
형태를 띠었고,
\hemph{범죄자}의 재판은
정해진 규칙이나 정해진 절차로부터 전적으로 독립된
전적으로 특별하고 전적으로 비정규적인 절차였다.
결과적으로
재판을 담당하는 법원이 주권자인 국가 자신인 까닭에,
또한
미리 어떤 행위 유형을 지시하거나 금지하는 것이 불가능한 까닭에,
이 시대에는 형\hemph{법}이 존재하지 않았던 것이다.
그 절차는 통상적으로 법률이 통과되는 형태와 동일했다.
동일한 사람들에 의해 발의되었고
동일한 엄숙한 절차에 의해 진행되었다.
나중에 법원 및 사법관의 조직을 갖춘 통상의 형법이
들어서고 나서도,
옛 절차는,
이론상 모순될 것이 없다는 점에서 짐작할 수 있듯이,
여전히
가용한 상태로
엄연히
남아있었음을 유념해야 한다.
이러한 수단을 사용하는 것이 다분히 경원시되었음에도,
로마 인민은
이 권한을 항상 보유했고 이를 이용해
특별법으로 대역죄를 처벌하곤 했다.
고전학자들에게는
정확히 동양\hanja{同樣}으로
아테네의 형벌법안인
에이산겔리아\greek{εἰσαγγελία}가
정규 법원의 설치 이후에도 계속 유지되었음을 굳이 상기시킬 필요가 없을 것이다.
또한
주지하듯이
튜턴족의 자유민들이 입법을 위해 집회했을 때
그들은
특별히 사악한 범죄
또는
높은 신분의 범죄자가 저지른 범죄를 처벌할 권한도 행사했다.
앵글로색슨의 위테나게모트\latin{Witenagemot}\footnote{%
  `현자(賢者)들의 모임'이라는 뜻으로 앵글로색슨 왕국들에서
  일종의 국왕평의회(curia regis)의 역할을 했다.
}의 형사 재판권도
동일한 성격을 가졌다.




