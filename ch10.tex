\chapter{불법행위법 및 형법의 초기 역사}

\para{고대 법전에서의 형법}
앵글로색슨 선조들의 법전을 포함한
튜턴족의 법전들은
초기의 법들 간의 비중을 정확히 알 수 있는 상태로
우리에게 전해지는
유일한 원시 세속법 법전들이다.
로마와 그리스 법전들의 현존하는 단편들은
그것들의 일반적 성격을 알려주기에는 충분하지만,
그 부분들 간의 정확한 양이나 비율을 파악하기에는 부족한 상태로 남아있다.
그럼에도 불구하고 전반적으로 보아
우리에게 전해지는 모든 고대법 집성들은
성숙한 법체계와 크게 다른 한 가지 특징을 가지고 있다.
민법 대비 형법의 비율이 대단히 큰 차이를 보이는 것이다.
게르만 법전들에서는 형법에 비해 민법 부분의 비중이 아주 작다.
드라콘의 법전이 규정한 포악한 형벌에 관한 전승\hanja{傳承}을 보더라도
이것 또한 마찬가지 성격이었던 것으로 보인다.
뛰어난 법적 재능과 애초 부드러운 풍속을 가졌던 사회의 작품인
12표법만이 유일하게
근대법과 비슷한 정도로 민법의 우위를 보여주지만,
불법행위의 구제방식이 차지하는 상대적 비중이
아주 크지는 않더라도 상당히 큰 편이다.
생각건대
오래된 법전일수록 형법이 더 풍부하고 더 상세하다고
주장할 수 있겠다.
이 현상은
처음으로 그들의 법을 성문화한 공동체들에 만연했던 폭력때문이라고
흔히
인식되어왔고 설명되어왔거니와,
대체로 정확하다고 할 수 있다.
입법자들은
그들 법전의 부분들의 비율을
미개한 생활에서 발생하는 특정 종류의 사건들의 빈도에 맞추었다는 것이다.
하지만
이런 설명은 불완전하다고 생각한다.
옛 집성들에서 민법이 상대적으로 황무지인 것은
이 논저에서 다룬 고대법의 다른 성격들과 밀접히 관련된다는 점을 기억해야 한다.
문명사회의 민법 부분의 십중팔구는
신분법, 물권 및 상속법, 그리고 계약법으로 이루어져있다.
그러나
이들 법분야는
사회적 결속의 유년기로 거슬러올라갈수록
더욱 좁은 범위로 축소될 수밖에 없음이 명백하다.
\wi{신분법}\latin{law of status}에 다름아닌 인법\hanjalatin{人法}{law of persons}은
모든 형태의 신분이 가부장권에 함께 복속해 들어가있는 한,
아내가 남편에 대해,
아들이 아버지에 대해,
미성숙의 피후견인이 종친\hanja{宗親}인 후견인에 대해
아무런 권리도 갖지 않는 한,
아주 좁은 범위로 축소될 것이다.
마찬가지로
물건과 상속에 관한 법도
부동산과 동산이 가족 내에서 대물림되는 한,
설령 분배되더라도 가족 내에서 분배되는 한,
결코 풍부할 수 없을 것이다.
그러나 고대 민법의 가장 큰 빈틈은
언제나 계약법의 부재에 기인할 것이다.
몇몇 원시 법전은 계약법이 아예 없고,
다른 법전들에서는
선서\hanjalatin{宣誓}{oath}에 관한 복잡한 법이 계약법을 대신하고 있어서
계약 관련
도덕관념의 미성숙을 보여주고 있을 뿐이다.
그런데 이에 상응하는,
형법의 빈곤을 가져올 만한 이유는 없다.
따라서,
설령 제 민족들의 유년기가 언제나 무제약적 폭력의 시기였다고 말하는 것이
위험하다 할지라도,
왜 근대법의 민법 대비 형법의 관계가 고대 법전에서는 역전되는 것인지
우리는 여전히 이해할 수 있는 것이다.

\para{범죄와 불법행위}
나는 후대에 비해 원시법이 형법,
즉 \hemph{범죄}법\latin{\textit{criminal} law}에
큰 비중을 둔다고 말했다.
이 표현은 편의상 사용한 것이고,
실은
고대 법전을 살펴보면
비상한 양을 차지하는 저 법이
진정한 범죄법은 아님이 드러난다.
문명사회의 법은
국가나 공동체에 대한 침해와
개인에 대한 침해를 구분하는 데 일치하고 있다.
이렇게 구분된 두 종류의 침해를 우리는,
법학에서 항상 일관되게 이들 용어가 사용되고 있다고 자신할 수는 없지만,
범죄\latin{crimes; \textit{crimina}}와
불법행위\latin{wrongs; \textit{delicta}}라고
부를 수 있을 것이다.
그렇다면 고대 공동체의 형법은
범죄법이 아니라,
불법행위\paren{영국법 용어로는 토트\latin{torts}}법인 것이다.
피해자는 가해자를 상대로 통상의 민사소송을 제기하고,
승소하면 금전배상의 형태로 전보\hanja{塡補}받는다.
\wi{가이우스}의 주해서 중에
12표법에 기초한 형법을 취급하는 부분을 펼쳐보면,
로마법이 인정하는 민사 불법행위의 첫머리에
\hemph{\wi{절도}}\latin{furtum}가 나오는 것을 볼 수 있다.
우리가 익히 \hemph{범죄}로만 취급된다고 여기는 침해가
\hemph{불법행위}로만 취급되고 있는 것이다.
절도뿐만 아니라
폭행 및 모욕\latin{assault}\footnote{%
  저자는 로마법상의 인격침해(iniuria)를 영어의 `assault'로 옮기고 있는 듯하다.
  로마법의 `인격침해'는 신체적 폭행뿐 아니라
  모욕, 명예훼손 등 인격적 침해도
  포괄하므로 본문에서는 `폭행 및 모욕'으로 번역했다.
}과 강도도
로마 법학자들은 영국법상의 불법침해\latin{trespass},
문서명예훼손\latin{libel}, 구두명예훼손\latin{slander}과
마찬가지로 취급한다.\footnote{%
  이들 세 가지 영미법상의 법개념들은 모두
  불법행위(tort)에 속하는 소송형식들이었다.}
이들 모두가 채권채무관계, 즉 `법의 사슬'\latin{vinculum juris}을
가져오고, 이들 모두가 금전배상으로 전보되는 것이다.
하지만 이 특징은 게르만 부족들의 법전에서 더욱 뚜렷하게 나타난다.
예외 없이 그것들은
살인에 대한 금전배상의 방대한 체계를 기술하고 있고,
거의 예외 없이
기타 덜 중대한 침해에 대한 방대한 배상체계를 기술하고 있다.
켐블\latin{John Mitchell Kemble} 씨에 따르면
``앵글로색슨법에서는 \paren{<<앵글로색슨>>, 1.177}
모든 자유인의 생명에 그의 신분에 따라 금액이 매겨져있었다.
또한 사람의 신체에게 가해질 수 있는 모든 상해에 대해,
그리고 그의 시민권, 명예, 평온에 대해 가해질 수 있는 거의 모든 침해에 대해
그에 상응하는 금액이 매겨져있었다.
그 금액은 우발적인 상황에 따라 가중된다.''\footnote{%
  \latinmarks
  John Mitchell Kemble,
  \textit{The Saxons in England: A History of the English Commonwealth
  Till the Period of the Norman Conquest},
  Vol.\,1,
  London: Longman, Brown, Green, \& Longmans, 1849,
  pp.\,276f.}
이들 배상금은 중요한 수입원이 되었을 것이 분명하고,
매우 복잡한 규칙이 그것에 대한 권리와 책임을 규율하고 있거니와,
전술했듯이, 귀속되는 사람의 사망으로 그것이 면책되지 않는다면,
어떤 특정한 상속규칙에 따라 상속되는 것이 일반적이다.
따라서,
국가가 아니라 개인을 피해자로 보는 것이
\hemph{불법행위}법의 기준이라면,
법의 유년기에는
형법이 아니라
불법행위법에 의존하여
시민들이
폭력이나 사기로부터 보호받았다고 주장할 수 있는 것이다.

\para{불법행위와 종교적 죄}
그리하여 원시법에서 불법행위는 방대한 양을 차지하고 있다.
또한 종교적 죄\latin{sin}도 원시법에 알려져있었음을 첨언해야 할 것이다.
튜턴족 법전들에 관해서는 이런 주장을 굳이 할 필요도 없거니와,
현존하는 이들 법전은 기독교도 입법자들에 의해
편찬되고 개정되었기 때문이다.
그러나 사실 비\hanja{非}기독교적인 옛 법전들도
일정한 작위 유형과 일정한 부작위 유형에 대해
신의 지시와 명령을 위반했다는 이유로 형벌을 부과한다.
아테네의 \wi{아레오파고스}\latin{Areopagus} 원로회의가 집행한 법은
아마도 어떤 특별한 종교법전이었을 것이다.
로마에서도 일찍이
신관\hanjalatin{神官}{pontifical}법이
간통, 독신\hanja{瀆神}, 그리고 어쩌면 살인도
처벌했던 것이다.
따라서 아테네와 로마 국가에서는 \hemph{종교적 죄}를 벌하는 법이 있었다.
또한 \hemph{불법행위}를 벌하는 법도 있었다.
신에 대한 침해라는 개념이 전자의 법을 만들었고,
이웃에 대한 침해라는 개념이 후자의 법을 만들었다.
그러나 국가 또는 전체 공동체에 대한 침해라는 관념은
초기에는 진정한 의미의 형법을 만들지 못했다.

\para{범죄의 개념}
그렇다고 해서
국가에 대한 침해라는 간단하고도 기초적인 개념이 원시사회에
부재했다고 생각해서는 안 된다.
그보다는,
이 개념이 실현되는 특별한 방식이
초기에 형법의 성장을 가로막는 원인이었다고 보인다.
여하튼
로마 공동체는 자신이 침해당했다고 생각되면
개인이 침해당한 경우를 유추하여
완전히 똑같은 결과를 강제했거니와,
국가는 어떤 특별한 행위로써 개인 침해자를 응징했던 것이다.
즉, 로마 공동체의 유년기에는
국가의 안전과 이익을 중대하게 침해하는 모든 행위는
입법기관의 개별 \wi{입법}에 의해 처벌되었다.
그리고 이것이 범죄\latin{crimen}의 초창기 개념이었거니와,
국가가 사건을 민사법원이나 종교법정에 맡기지 아니하고
침해자에 대한 특별법\latin{privilegium}을 만들어 처벌할 정도로
중대한 문제를 야기하는 행위가
바로 범죄인 것이다.
따라서 모든 기소는
`\wi{처벌법안}'\latin{bill of pains and penalties}의
형태를 띠었고,
\hemph{범죄자}의 재판은
정해진 규칙이나 정해진 절차로부터 전적으로 독립된
전적으로 특별하고 전적으로 비정규적인 절차였다.
결과적으로
재판을 담당하는 법원이 통치자인 국가 자신인 까닭에,
또한
미리 어떤 행위 유형을 지시하거나 금지하는 것이 불가능한 까닭에,
이 시대에는 형\hemph{법}이 존재하지 않았던 것이다.
그 절차는 통상적으로 법률이 통과되는 형태와 동일했다.
동일한 사람들에 의해 발의되었고
동일한 엄숙한 절차에 의해 진행되었다.
나중에 법원 및 사법관의 조직을 갖춘 정규의 형법이
들어서고 나서도,
옛 절차는,
이론상 모순될 것이 없다는 점에서 짐작할 수 있듯이,
여전히
가용한 상태로
엄연히
남아있었음을 유념해야 한다.
이러한 수단을 사용하는 것이 다분히 경원시되었음에도,
로마 인민은
이 권한을 항상 보유했고 이를 이용해
특별법으로 \wi{대역죄}를 처벌하곤 했다.
고전학자들에게는
정확히 동양\hanja{同樣}으로
아테네의 `\wi{처벌법안}'인
\wi{에이산겔리아}\greek{εἰσαγγελία}가
정규 법원의 설치 이후에도 계속 유지되었음을 굳이 상기시킬 필요가 없을 것이다.
또한
주지하듯이
튜턴족의 자유민들이 입법을 위해 집회했을 때
그들은
특별히 사악한 범죄
또는
높은 신분의 범죄자가 저지른 범죄를 처벌하는 권한도 행사했다.
앵글로색슨의 \wi{위테나게모트}\latin{witenagemot}\footnote{%
  `현자(賢者)들의 모임'이라는 뜻. 앵글로색슨 왕국들에서
  일종의 국왕평의회(curia regis)였다.
}의 형사 재판권도
동일한 성격을 가졌다.

\para{중재로서의 재판}
내가 주장한 고대 형법관과 근대 형법관 사이의 차이가
말로만 존재하는 것이 아닐까 생각될 수도 있을 것이다.
가령 공동체는,
입법을 통해 범죄를 처벌하는 것 외에도,
일찍부터
자신의 법원을 통해
위법행위자에게 피해 배상을
강제해왔으며,
이렇게 할 경우
그의 행위로 공동체가 침해받았다고
어떤 식으로든
가정하지 않으면 안 된다고
말할 수 있는 것이다.
그러나,
이런 추론이
현대인에게
아무리 그럴듯해 보이더라도,
먼 옛날 태곳적 사람들이
실제로
이런 추론을
했을까는 대단히 의문스럽다.
공동체에 대한 침해라는 관념이
\hemph{자신의 법원을 통한}
초기 국가의 개입과
얼마나 무관한지는,
초창기 재판 절차가
사적 영역에서 분쟁하고 있는,
그러나 이후 그 분쟁을 진정시키는 데 동의하는
사람들이 행할 법한
일련의 행위를 거의 그대로 모방하는 것이었다는
흥미로운 사실에서도
알 수 있다.
정무관은 무심코 불려온 사적 중재인의 행위를 꼼꼼히 흉내냈던 것이다.

\para{법률소송}
이 진술이 그저 상상으로 꾸며낸 말이 아님을 보여주기 위해
이제 나는 그 근거가 되는 증거를 제시하고자 한다.
우리에게 알려진 가장 오래된 소송절차의 하나가
바로 로마의
\wi{신성도금법률소송}\hanjalatin{神聖賭金法律訴訟}{legis actio sacramenti}이거니와,
이것으로부터 모든 후대의 로마 소송법이 발달되어 나오는 것을 볼 수 있다.
\wi{가이우스}가 그 의례절차를 꼼꼼히 기술하고 있다.\footnote{%
  \latin{Gai.\,4.16.} }
일견 무의미하고 기이하게 보이지만,
조금만 주의를 기울이면 그것을 해독할 수 있고 해석할 수 있다.

소송의 목적물이 법정에 나와야 한다.
동산이면, 실제로 가지고 나오면 된다.
부동산이면, 그것의 일부분 또는 견본을 대신 가지고 온다.
예컨대, 토지라면 한줌의 흙덩이, 가옥이라면 벽돌 한 장으로 대신한다.
가이우스가 채택한 예에서는 노예가 소송의 목적물이다.
절차자 시작되면 원고가 막대기 하나를 들고 나서는데,
가이우스가 밝혀놓았듯이 막대기는 창\hanja{槍}을 상징한다.
원고는 노예에게 손을 얹고 다음과 같은 말로 권리를 주장한다.
``말했듯이
이 사람은 \wi{시민법}에 의해 정당한 권원에 따라 나의 것임을
주장하노라\latin{Hunc ego hominem ex jure quiritium meum esse dico
secundum suam causam sicut dixi}.''
그러고는 이어서 ``보라! 이 사람에게 나의 창을
두었노라\latin{Ecce tibi vindidam imposui}''고 말하면서
그에게 창을 갖다댄다.
이어서 피고도 동일한 언명과 몸짓을 수행한다.
그러고나면 \wi{법무관}이 개입하여
당사자들에게 손을 뗄 것을 명한다.
``둘 다 그 사람을 놓아주라\latin{Mittite ambo hominem}.''
당사자들은 명을 따른다.
이어 원고는 피고에게 주장의 근거를 요청한다.
``너는 어떤 권원에서 주장하였는지 진술해줄 것을
요청하노라\latin{Postulo anne dicas
quâ ex causâ vindicaveris}.''
이 질문에 대해 피고는 또 다시 자신의 권리를 주장함으로써 답한다.
``나의 창을 두었거니와 나는 나의 권리를 행사하였노라\latin{Jus peregi
sicut vindictam imposui}.''
그러면 원고는 이 사건 재판에 대해
`신성도금'\latin{sacramentum}이라 불리는 일정액의 금전을 걸자고 제안하다.
``너는 불법적으로 주장하였므로,
나는 500아스의 신성도금으로 너에게
도전하노라\latin{Quando tu injuriâ provocasti, D aeris sacramento te provoco}.''
피고는 ``나도 너에게 똑같이 도전하노라\latin{Similiter ego te}''는 말로
내기\latin{wager}를 받아들인다.
이후의 절차는 더 이상 요식적 성격을 띠지 않는다.
그러나 법무관은 신성도금 명목으로 보증금을 받았고,
이 돈은 언제나 국고에 귀속되었다는 점을 명심해야 한다.

\para{호메로스가 묘사한 고대 소송}
고대 로마의 모든 소송은 반드시 이러한 절차로 시작했다.
생각건대
재판의 기원이 드라마 형태로 각색되어 있는 것을
여기서
볼 수 있다는 데
동의하지 않을 수 없을 것이다.
두 명의 무장한 사람들이 분쟁 대상을 두고 다투고 있다.
큰 자비\hanja{慈悲}의 소유자\latin{vir pietate gravis}인 \wi{법무관}이 마침
그곳을 지나가다가 분쟁을 해결하기 위해 개입한다.
분쟁당사자들이 각자 자신의 사정을 말하고,
법무관이 중재해 줄 것에 동의한다.
즉, 패자\hanja{敗者}는 분쟁의 목적물만 잃는 것이 아니라
중재인에게 그의 노고와 시간에 대한 보답으로
일정액의 금전도 지불하기로 합의하는 것이다.
이런 해석은,
가이우스가 법률소송\latin{legis actio}의 필수과정으로 묘사한
의례절차가
헤파이스토스 신이
아킬레우스의 방패의 첫 번째 부분에 새겨넣은 것으로
\wi{호메로스}가 묘사한
두 가지 주제 중 하나\footnote{%
  아킬레우스의 방패에는 두 개의 도시가 새겨졌으니,
  한 도시에는 결혼식 장면과 재판 장면이 그려졌고,
  다른 도시에는 전쟁 장면이 그려졌다고 한다.
  첫 번째 도시에 대한 묘사는
  \latin{Hom. Il. 18.490.}
}와
놀라울 정도로 일치하지 않았다면,
설득력이 반감되었을 것이다.
\index{일리아스}%
호메로스가 묘사한 재판 장면은,
마치 원시사회의 성격을 드러내고자 의도하였다는 듯이,
물건에 대한 소송이 아니라
살인에 대한 \wi{속죄금} 소송이다.
한 사람은 이미 지불했다고 주장하고,
다른 사람은 받은 적이 없다고 주장한다.
하지만 상세하게 묘사된 부분은
옛 로마의 소송관행의 대응물을 그려내고 있거니와,
판관들에게 주어지는 보답 부분이다.
2달란트의 황금이 가운데 놓여있고,
판결의 근거를 청중들에게 가장 만족스럽게 설명하는 자에게
이것이 주어진다.
이 금액이 신성도금의 사소한 금액에 비해 무척 큰 것은
유동적인 관행과 법으로 고정된 관행 간의 차이를 시사하는 것으로 보인다.
저 시인에 의해
영웅시대의 도시생활의 두드러진 특징으로,
그러나 아직은 임시적인 성격의 것으로
소개된 이 장면은
문명의 역사가 열리면서
정규의 통상적인 소송 형식으로 굳어져간다.
따라서
법률소송에 이르면 판관의 보수는
당연히 합리적인 금액으로 줄어들고,
군중의 환호에 따라 여러 중재인 중 한 명에게 주어지는 대신
당연히 법무관으로 대표되는 국가에 귀속되는 것이다.
그러나
호메로스에 의해 생생하게 묘사된 장면과
가이우스에 의해
흔히 법기술적 언어에서 보이는 것 이상으로 날것 그대로
묘사된 장면은 의심할 여지 없이
사실상 동일한 것을 의미한다.
또한
근대 유럽의 초기 재판관행을 연구한 다수의 학자들은
법원이 법위반자에게 부과한 벌금이 원래는 신성도금\latin{sacramenta}이었다고
말하고 있거니와, 이 또한
우리의 견해를 뒷받침해준다.
국가는
피고인에게서
자신에게 가해진 침해에 대한 배상을
받아낸 것이 아니라,
원고에게 주어지는 배상금의 일정 비율을
단지
시간과 노고를 들인 데 대한 정당한 대가로서
요구했던 것이다.
켐블 씨는 앵글로색슨의
`반눔'\latin{bannum} 또는 `프레둠'\latin{fredum}에
명시적으로
이러한 성격을
부여하고 있다.\footnote{Kemble, 앞의 책, p.\,270.}

\para{로마 고법상의 절도}
초기의 재판관들이 사적 분쟁에 연루된 당사자들 간에 있음직한 행위를
모방했다는 것을 보여주는 다른 증거들도 고대법에서 발견된다.
구제수단을 정함에 있어
그들은 당해 사건의 상황 하에서 피해자가 감행했을 법한 복수의 양태를
감안했던 것이다.
이것은 현장에서 또는 범행 직후에 붙잡힌 범죄자와
상당 기간 경과 후에 발각된 범죄자를
고대법이
사뭇 다르게 처벌했던 점을 올바르게 설명해줄 수 있다.
이러한 차이를 보여주는 다소 특이한 예는 로마 고법\hanja{古法}상의 절도에서 볼 수 있다.
\wi{12표법}은 \wi{절도}를
\wi{현행도}\hanjalatin{現行盜}{manifest theft}와
비\hanja{非}현행도\latin{non-manifest theft}로
구분하였으며,
어느 쪽에 속하는가에 따라 완전히 다른 처벌을 부과했다.
현행도는 절도행위를 하던 중 그 집 안에서 붙잡힌 자,
혹은 훔친 물건을 가지고 안전한 장소로 달아나다가 붙잡힌 자를 말한다.
12표법에는 이런 자가 노예인 경우 \wi{사형}에 처하도록 하였고,
자유인인 경우 그 물건의 주인의 노예가 되도록 정해놓았다.\footnote{%
  \latin{XII Tab.\,8.14.}}
비현행도는 그밖의 상황 하에서 발각된 자를 말한다.
저 옛 법전은 이런 종류의 절도는 훔친 물건의 2배액을 배상한다고만
정해놓았다.\footnote{%
  \latin{XII Tab.\,8.16.}}
당연히
\wi{가이우스}의 시대에는
현행도에 대한 12표법의 가혹함이 상당 부분 완화되었다.
그러나
비현행도의 경우 여전히 2배액만 배상하는 데 비해
\wi{현행도}는 훔친 물건의 4배액을 배상하게 함으로써
\index{비현행도|see{현행도}}%
옛 원리를 계속 유지하고 있었다.\footnote{%
  \latin{Gai.\,3.189--190.}}
고대 입법자는
피해자인 소유주가,
스스로 처벌한다면,
격정에 휩싸여 있을 때 처벌하는 것과
시간이 한참 지나 절도범이 발견될 때 만족할 만한 것 간에
큰 차이가 있으리라는 점을 고려했음이 분명하고,
이에 맞추어 처벌 수위를 조정했을 것이다.
이러한 원리는 앵글로색슨 및 기타 게르만 법전들이 따르는 원리와
정확히 일치한다.
\wi{절도}범을 추격하여 도품과 함께 붙잡은 경우 그들은
그 자리에서 그를 교수형 또는 참수형에 처했던 반면,
추격이 중단된 후에는 살인자라도 살인에 대한 완전한 배상금을 받아내는 데
그쳤던 것이다.
옛 법상의 이러한 구분은 원시적 법과 세련된 법 사이에
간격이 얼마나 큰지 실감케 해준다.
근대의 재판관들은
법기술적으로 동일한 종류에 속하는 범죄들 간에
죄의 경중을 구별하는 일이
그들 업무 중 가장 고역이라고 고백하고 있다.
어떤 사람이 살인죄, 절도죄, 또는 중혼\hanja{重婚}죄의 유죄라고
말하기는 쉬워도, 그가 어느 정도로 도덕적 죄질이 나쁜지,
그리하여
어떤 형벌을 부과하는 것이 적절한지
판단하기란 무척 어렵다.
우리가 이런 사항을 정확히 정해놓으려 시도한다면,
동기를 분석하는 결의론\latin{casuistry}에 빠진다 해도
그다지 잘못이라 할 수 없을 것이다.
따라서 오늘날의 법은
이 문제에 관하여 세세한 실정규칙을 정해놓는 일을
가능하면 피하려는 경향을 보인다.
프랑스에서는
유죄로 판명된 범죄에 참작할 만한 정황이 있는지 결정하는 것이
배심원단에게 맡겨져있다.
영국에서는
형벌을 선택함에 있어 거의 무제한적 재량이
판사에게 주어져있다.
또한 모든 국가에서
법의 오판을 교정하는 수단으로서
어디서나 사면권이 최고 통치권자에게 유보되어 있다.
신기하게도
원시시대의 사람들은 이러한 망설임으로 고민하는 일이 거의 없었고,
전적으로
피해자의 충동이 그가 가할 수 있는 복수의 적절한 기준이라고 생각했으며,
또한 형벌의 척도를 정함에 있어
피해자의 격정의 있음직한 등락을 그대로 모방했던 것이다.
그들의 입법의 방법이 이제 완전히 사라졌다고 말할 수 있기를 나는 바란다.
하지만 몇몇 근대법체계에서는
중대한 침해의 경우
침해행위자를 행위로 나아가게 만든 사실이
피해자가 그에게 가한 과잉된 응징을
정당화\latin{justification}하는 항변 사유로 인정되고 있거니와,
겉보기에는 타당해보일지 몰라도,
이는
저급한 도덕성에 기초한 면죄부라고 나는 생각한다.

\para{로마의 사문회}
전술했듯이
고대사회로 하여금 결국 진정한 의미의 형법을 갖도록 이끈
고려사항은 무척 단순한 것이었다.
국가는 스스로를 피해자로 관념했고,
민회는 입법행위에 수반되는 것과 동일한 절차로
가해자를 직접 타격했다.
또한 실로 고대 세계에서---후술할 기회가 있겠지만,
현대 세계에서는 그렇지 않을 수 있다---초창기 형사 법정은
단지 입법기구의 하위 분과, 즉 위원회에 지나지 않았다.
어쨌든 이것이
법사\hanja{法史}에서 두 개의 위대한 고대 국가들에 관하여,
하나는 그럭저럭 명료하게, 다른 하나는 완전히 확실하게,
드러나는 결론이다.
아테네의 원시 형법은 범죄의 처결을
일부는 \wi{아르콘}\latin{archon}들에게 맡겨
\hemph{불법행위}로서 처벌하게 했고,
일부는 \wi{아레오파고스} 원로회의에 맡겨
\hemph{종교적 죄}로서 처벌하게 했다.
양자의 재판권은 결국 \wi{헬리아이아}\latin{Heliaea}, 즉
최고인민법원에 사실상 넘어갔고,
아르콘의 기능과 아레오파고스의 기능은 행정적인 것에 불과하게 되거나
아니면 거의 무의미한 것이 되어버렸다.
그런데 ``헬리아이아''는 애초에 단지 민회를 지칭하는 말이었다.
고전 시대의 헬리아이아는 사법\hanja{司法}적 목적을 위해
모인 민회였을 뿐이고,
아테네의 저 유명한 \wi{디카스테리온}\latin{dikastery}들은
민회의 하위 분과, 즉 배심원단들이었을 뿐이다.
로마에서 일어난 이와 유사한 변화는
훨씬 더 수월하게 해석될 수 있거니와,
로마인들은 배심원단에 관한 법만 수정했고,
아테네인들처럼 민사재판권과 형사재판권을 포괄하는 인민법원을
만들지는 않았기 때문이다.
로마 형법의 역사는
왕이 주재했다고 전해지는
옛 인민재판\latin{judicia populi}에서 시작한다.
이는 입법의 형태로 중범죄자들을 처벌하는 엄숙한 재판이었을 뿐이다.
하지만
일찍부터
민회\latin{comitia}는 자신의 형사재판권을
\wi{사문회}\hanjalatin{査問會}{quaestio}, 즉 위원회에
위임하곤 했었던 것으로 보인다.
사문회의 민회에 대한 관계는
영국 하원의 위원회가 하원에 대해 갖는 관계와 거의 동일하지만,
다만 로마의 사문관들은
민회에 \hemph{보고}\latin{report}만 한 것이 아니라
민회가 행사하던 모든 권한을 행사했으며,
따라서 피고인에게 판결을 선고할 수 있었다.
이런 종류의 사문회는
특정 범죄자를 재판하기 위해서만 임명되었으나,
두 세 개의 사문회들이 동시에 재판하는 일도 불가능하지는 않았다.
또한
공동체를 위협하는 여러 개의 중대한 범죄사건이 동시에 발생하면,
여러 개의 사문회가 동시에 임명되는 일도 있었던 것으로 보인다.
또한 사문회가
\hemph{상임}위원회의 성격을 띠는 경우도 가끔 있었거니와,
이는 중대한 범죄의 발생을 기다리지 아니하고
정기적으로 임명되는 사문회였다.
아주 오래된 업무집행과 관련하여 언급되는
\index{가부장살해사문관}%
옛 `가부장\hanja{家父長}살해사문관'들\latin{quaestores parricidii}은
모든 존속살해 및 일반 살인 사건의
재판\paren{수사와 재판을 모두 담당했다는 의견도 있다}을
위임받았거니와,
매년 정기적으로 임명되었던 것으로 보인다.
두 명으로 구성되어
국가에 대한 폭력적 위해를 재판하던
`\wi{반역이인관}'\hanjalatin{反逆二人官}{duumviri perduellionis}도
정기적으로 임명되었다고 보는 것이 통설이다.
이들 두 경우의 권한 위임은 우리를 한걸음 더 후대로 데려다준다.
국가에 대해 범죄가 \hemph{이미} 행해진 후 임명되는 것이 아니라,
\hemph{장래} 행해질 경우에 대비해
일시적이기는 해도 일반적인 재판권을 가졌던 것이다.
``가부장살해''\latin{parricidium}와
``반역''\latin{perduellio}이라는 일반적 용어도
정규의 형법에 다가가고 있음을 시사하거니와,
범죄의 분류에 유사한 것에 접근하고 있는 것이다.

\para{상설사문회}
하지만 진정한 의미의 형법은 기원전 149년에 비로소 등장한다.
칼푸르니우스 피소\latin{Calpurnius Piso}에 의해
이른바
`부당착취에 관한 칼푸르니우스 법'\latin{Lex Calpurnia de Repetundis}이
만들어졌던 것이다.
이 법률은 부당착취금회수\latin{repetundarum pecuniarum}소송,
즉 속주총독이 부당하게 수탈한 돈을 반환하라고 속주민들이 제기한
소송에 적용되었다.
그러나 이 법률이 항구적으로 큰 중요성을 갖게 된 것은
최초로 \wi{상설사문회}\hanjalatin{常設査問會}{quaestio perpetua}가 설치되었다는 데 있다.
상설사문회는 일종의 \hemph{상임}위원회로서
임시적인 또는 일시적인 위원회와는 그 성격을 달리했다.
그것은 정규의 형사법원으로서,
이를 창설하는 법률이 통과된 때부터 존재하여
폐지하는 법률이 통과될 때까지 존속하였던 것이다.
그것의 구성원들은
옛 \wi{사문회} 구성원처럼 특별히 임명되는 것이 아니라,
그것을 창설하는 법률의 규정에 의해
특정 신분의 사람들 중에서 직무를 맡을 심판인들이 선발되었고
또한 정해진 규칙에 따라 갱신되었다.
그것이 담당할 범죄들의 이름과 정의\hanja{定義}도
이 법률에
명시적으로 규정되어 있었다.
그리하여 이 새로운 사문회는
장래에
이 법률에 규정된 범죄의 정의에 해당하는 행위를 한
모든 사람들을 재판하고 선고를 내릴 권한을 가졌다.
그것은 따라서
진정한 형법을 집행하는
정규의 형사법원이었다.

\para{형법의 역사}
원시적 형법의 역사는 따라서 네 단계로 구분된다.
\hemph{범죄}의 개념을
\hemph{불법행위}와 구분하고
\hemph{종교적 죄}와도 구분하여
국가 혹은 전체 공동체에 대한 침해로 관념할 때,
이 개념에 글자 그대로 부합하게,
우선
국가는
개별적 행위에 의해 스스로 직접 개입하여
자신을 침해한 범죄행위자에게 보복했다.
이것이 출발점이다.
각각의 기소는 일종의 \wi{처벌법안}이고,
범죄자의 이름과 그에 대한 형벌이 개별 특별법에 의해 정해진다.
\hemph{두 번째} 단계는
다수의 범죄에 대해
입법기구가 자신의 권한을 특별한 사문회 또는 위원회에 위임할 때
달성된다.
각 사문회 또는 위원회는
특정한 하나의 기소에 대해 수사하고,
유죄로 입증되면 그 특정 범죄자를 처벌할 권한을 부여받는다.
\hemph{또 하나의} 단계는
입법기구가,
범죄의 실행을 기다려
사문회를 임명하는 것이 아니라,
가부장살해사문관이나 반역이인관처럼
일정 유형의 범죄가 저질러질 가능성과
\hemph{장래} 그것들이 실행될 것이라는 기대 하에
정기적으로 위원들을 임명할 때에 이루어진다.
\hemph{마지막} 단계는
정기적이거나 임시적이던 \wi{사문회}가
상설적인 판사단\latin{bench} 또는 법원\latin{chamber}이 될 때,
위원회를 임명하는 특별법에 의해 그때마다
심판인들을
지명하는 것이 아니라
장래를 향해 영구적으로
특정한 방식으로 특정한 신분 중에서
심판인들을
선발하도록 정해져있을 때,
일정한 행위가 일반적 언어로 기술되고 범죄로 선언되어 있어서
그 행위가 실행되면 각 유형에 적합한 예정된 형벌로 처벌될 때에
도달된다.

\para{사문회의 결과}
상설사문회의 역사가 더 오래 지속되었다면,
그것은 틀림없이 독자적인 제도로 관념되었을 것이고,
민회에 대한 그것의 관계도
영국의 보통법법원들이
모든 사법권의 이론적 원천인
국왕에 대해 갖는 관계 이상으로
가깝게 생각되지 않았을 것이다.
그러나
\wi{상설사문회}는 그 기원이 미처 망각되기 전에
제정기의 전제정치에 의해
파괴되고 말았다.
그리고 이들 상임위원회가 지속되는 동안
로마인들은
그것을 단지 위임된 권한의 담지자로만 생각했다.
그것이 담당하는 범죄는 본디 입법기구의 권한에 속하는 것으로 여겨졌고,
\wi{사문회}를 떠올릴 때면
로마인들의 머리에는
끊임없이 민회가,
자신의 불가양의 기능의 일부를 행사하도록 위임한 저 민회가
연상되었다.
사문회가
상설화된 이후에도 여전히
그것을 민회의 위원회로만---상위의 권위에 종속된 기구로만---간주하는 견해는
로마 형법의 마지막 시기까지 각인을 남긴 어떤 중요한 결과를 낳았다.
한 가지 직접적인 결과는
사문회들이 설치되고 나서도 오랫동안
민회가
\wi{처벌법안}을 통해
계속
형사재판권을 행사했다는 것이다.
입법기구가 편의상 다른 기구에게 권한을 위임했다고 해서
그 권한을 완전히 양도해버렸다고 할 수는 없다.
민회와 사문회들은 서로 나란히 범죄자들을 재판하고 처벌했다.
그리하여
민중의 분노가 끓어오르는 이례적인 사건이 발생하면,
공화국이 끝날 때까지는,
반드시 그 대상에 대한 기소가
\wi{트리부스 민회}\latin{assembly of the tribes}에 제출되었던 것이다.

\para{민회의 재판권, 사형}
민회에 대한 사문회의 종속성에서
로마 공화정의 또 하나의 중요한 특징 하나를 더
끌어낼 수 있다.
로마 공화국의 형벌체계에서 \wi{사형}이 소멸한 것은
지난 세기 학자들이 사뭇 선호하던 논의주제였다.
그들은 끊임없이 이를 이용해
로마인들의 심성에 관한 이론이나
근대적 사회경제에 관한 이론을 내세웠다.
그러나 자신있게 말할 수 있는 저 현상의 원인은
기실 순전히 우연적인 것이었다.
연속해서 나타난 로마의 입법기구들 중의 하나인
\wi{켄투리아 민회}\latin{comitia centuriata}는 주지하듯이
오로지 군사적 목적을 위한 로마 국가의 대표기구였다.
따라서 켄투리아 민회는
군대를 통솔하는 사령관에게 주어질 법한 모든 권한을 가졌고,
그중에서도
군율을 위반한 군인에게 가해지는 처벌과 동일한 처벌을 범죄자들에게
부과하는 권한을 가졌다.
그리하여 켄투리아 민회는 \wi{사형}을 언도할 수가 있었다.
하지만 \wi{쿠리아 민회}\latin{comitia curiata}나
\wi{트리부스 민회}\latin{comitia tributa}는 달랐다.
이 점에 관하여 이들 민회는
로마시 성벽 안의 시민들의 인격에는
종교와 법에 의해
신성함이 부여되어 있다는 관념에 의해 제약받았다.
이들 중 후자인 트리부스 민회에 관해 말하자면,
트리부스 민회는 기껏해야 벌금형만 부과할 수 있다는 것이
확고한 원칙이 되어갔음을 우리는 확실히 알고 있다.
형사재판권이 입법기구에 국한되어 있는 한,
그리고 켄투리아 민회와 트리부스 민회가
동등한 권한을 계속 행사하는 한,
중형을 과할 수 있는 입법기구에
중대한 범죄를
기소하는 것이
선호되었을 것임은 어렵지 않게 짐작할 수 있다.
그러나
보다 민주적인 기구, 즉 \wi{트리부스 민회}가
거의 전적으로 다른 민회들을 대체하여
공화정 후기의 통상적인 입법기구가 되는 일이 발생했다.
또한
공화정의 몰락기는
저 상설사문회들이 창설되던 시기와 정확히 일치한다.
그리하여
\wi{상설사문회}를 창설하는 법률들은 모두
정상적인 집회로써는
범죄자에게 사형을 부과할 수 없는 입법기구에 의해 통과되었다.
결과적으로,
위임받은 권한만 갖는
상설 사법위원회들의
속성과 능력은
위임하는 기구가 갖는 권한의 한계에 의해 제한되었다.
트리부스 민회가 할 수 없는 것은
그것들도 할 수가 없었고,
\wi{트리부스 민회}는 \wi{사형}을 과할 수 없었으므로
사문회들도 똑같이 사형을 과할 수 없었다.
이렇게 해서 생겨난 특별한 상황을 바라보는
고대인들의 관점은
근대인들 사이에 인기있는 어떤 관념과 전혀 다른 것이었다.
실로,
로마인들의 심성이 그것에 더 부합했는지는 모르겠으나,
로마의 헌정에는 확실히 훨씬 더 나쁜 결과를 가져왔다.
역사적으로 인류가 경험했던 다른 모든 제도들과 마찬가지로,
사형도 문명화과정의 일정 단계에서는 사회의 필수요소가 된다.
어떤 시대에는
사형을 없애려는 시도가
형법의 근저에 놓여있는 두 가지 중대한 본능을 거스르게 된다.
사형이 없이는,
공동체는 범죄자에게 충분히 복수했다는 느낌도 갖지 못하고,
그에 대한 처벌의 사례를 가지고
그를 모방하고자 하는 다른 사람들을 억지시킬 수 있다는 느낌도
갖지 못하는 것이다.
\wi{사형}을 언도할 수 없는 로마 법원의 무능력은
명백하게 그리고 직접적으로
`\wi{추방}'\hanjalatin{追放}{proscription} 시대라고 불리는
저 가공스런 혁명적 시간들을 낳았다.\footnote{%
  여기서 `추방'이란 법의 보호를 박탈(outlaw)하는 것을 의미한다.
  추방자 명부에 등재되면 재산도 몰수당하고 생명도 언제 누구에게 빼앗길지
  모를 상태에 놓인다.
  본문은 기원전 82년 독재관 술라에 의한 대규모 숙청과
  기원적 43년 옥타비아누스, 안토니우스, 레피두스의 삼두체제에 의한
  (키케로를 포함하는) 대규모 숙청을 말하고 있는 듯하다. }
당시
복수에 목말라하던
당파적 폭력이
어떤 다른 출구도 찾지 못했다는 단순한 이유로
모든 법이 공식적으로 정지되었던 것이다.
법의 이러한 일시적 정지상태보다 더 강력하게
로마 인민의 정치적 능력을 몰락하게 만든 것은 없었다.
법이 일단 다시 회복된 후에도
로마 공화국의 몰락은 이제 시간 문제에 불과하게 되었다고
우리는 주저없이 주장할 수 있다.
민중의 감정이 분출될 수 있는 적절한 출구를
로마 법원이
제공했다면,
사법과정의 모습은 분명
스튜어트 왕조 후기의 영국이 경험했던
도착\hanja{倒錯}적인 형태를 노골적으로 띠었겠지만,
국민성은 그렇게 철저히 손상받지 않았을 것이며,
제도의 안정성도 그렇게 심각하게 허약해지지 않았을 것이다.

\para{사문회의 결과, 범죄의 분류}
재판권에 관한 전술한 이론에 의해 생겨난
로마 형사사법체계의 특징을 두 가지 더 언급하고자 한다.
하나는 로마에서는 형사법원의 숫자가 대단히 많았다는 것이고,
또 하나는
범죄의 분류가 변칙적이고 부조화스러웠다는 것으로, 이는
로마 역사 전체에 걸쳐 형법의 성격을 규정했다.
상설이든 아니든, 모든 \wi{사문회}는
각각 개별적인 법률에 의해 탄생한다.
각 사문회는 그것을 만든 법률에 의해 권위가 부여되었고,
그 법률이 정한 한계를 엄격히 준수해야 했으며,
그 법률이 명시적으로 정의하지 않은 범죄 형태는 결코 취급할 수 없었다.
그런데 다양한 사문회를 만든 법률들은
모두 특정한 위기상황에 대응하는 것이었고,
각각은 실로 당시의 상황에서
특별히 혐오스럽거나 특별히 위험하다고 여겨진
일군의 행위를 처벌하기 위해 통과되었기에,
이들 제정법들은 조금도 서로를 고려하지 않았고,
따라서 공통의 원리에 의해 통합되어 있지 못했다.
20 내지 30개의 서로 다른 형법들이 공존하고 있었거니와,
정확히 같은 숫자의 사문회들이 난립하고 있었다.
또한 공화정 시기에는
이들 개별 사법조직들을 하나로 통합하거나,
이들을 창설하고 이들의 책무를 정한 법률조항들을 조화시키려는
시도가 전혀 행해지지 않았다.
당시 로마의 형사사법의 상태는
영국의 보통법법원들이
서로 간에 다른 법원의 관할을 넘나들 수 있도록 하는,
영장\latin{writ}에 기입하는
의제적 사실 진술\latin{fictitious averment}이
도입되기 이전\footnote{%
  본서 제2장 \hyperlink{commonlawfiction}{의제의 용도} 부분 참조.}
영국의 민사 구제수단들의 사법체계와
비슷한 면을 보이고 있었다.
사문회들처럼,
왕좌법원과 민소법원과 재무법원도 모두
이론적으로
상위의 권위에서 권위가 유래하며,
그 상위의 권위에 의해 재판권이 부여된
특별한 사건군\hanja{群}을 각각 관장하고 있었던 것이다.
그러나 로마의 사문회들은 그 수가 세 개보다 훨씬 많았고,
각 사문회가 관장하는 행위군을 구분하는 것은
웨스트민스터 홀의 세 법원 간에 관할을 구분하는 것보다
훨씬 더 어려웠다.\footnote{%
  엄밀히 말하면, 웨스트민스터 궁의 웨스트민스터 홀에서
  개정했던 법원은 왕좌법원, 민소법원, 그리고 형평법법원이었다.
  재무법원은 웨스트민스터 홀이 아니라 인근의 다른 방에서 개정했다.
  }
서로 다른 사문회 간에 정확하게 경계선 긋기가 어려웠던 점은
여러 로마 법원들에게 단순히 불편함을 주는 것 이상의 문제를 낳았다.
범죄로 여겨지는 누군가의 행위가 어떤 일반적 유형에 해당하는지
즉각 명료하지 않은 경우,
서로 다른 여러 위원회들이
동시에 또는 순차적으로 그를 기소할 수 있고,
그중 어느 하나가
그에게 유죄판결을 내릴지도 모른다는 데서
우리는 놀라지 않을 수 없다.
또한 비록 하나의 사문회가 유죄판결을 내리면
다른 사문회의 재판권이 배척된다 하더라도,
하나의 사문회가 무죄판결을 내려도 그것이
다른 사문회의 기소를 저지하는 항변사유가 될 수 없었다.
이는 로마 민사법의 규칙과는 완전히 다른 것이었다.
또한 로마인들처럼 법의 부조화\paren{혹은 그들이 쓰는
의미심장한 용어에 따르면,
\hemph{전아}\hanja{典雅}\hemph{하지 못함}\latin{inelegancy}}에 대해
그렇게 민감한 민족이
이런 상태를 오래도록 방치하지 않을 것이라는 것도
확신할 수 있다.
다만,
범죄를 교정하는 항구적인 제도로
사문회를
보기보다는,
사문회를 둘러싼 우울한 역사로 인해
이를 파벌들이 장악하는 일시적인 무기로 보고 있었을 뿐이다.
제정기의 황제들은 머지않아 다수의 재판권 간의 충돌을 없애버린다.
그러나 그들은 법원의 난립과 밀접하게 연관된 형법의 또 다른 특징은
제거하지 않았다는 데 유의해야 한다.
유스티니아누스의 로마법대전에 이르러서도
범죄의 분류는 무척 변칙적인 것이었다.
사실
각각의 \wi{사문회}는
법률에 의해 수권\hanja{授權}된 범죄들만 취급할 수 있었다.
하지만 이들 범죄는
그 법률의 제정 당시
우연히
동시에 처벌대상이 되었다는 이유로
함께 묶여있었을 뿐이다.
따라서
그것들 간에는 필연적인 공통성이 없었다.
그러나
그것들이 특정한 사문회의 재판 대상이라는 사실은
자연스레 대중의 머리에 각인되었고,
동일한 법률에 언급된 범죄들 간의 관념적 연관은
고정된 틀로 굳어져,
술라와
황제 아우구스투스에 의해
로마 형법을 통합하려는 공식적 노력이 행해질 때에도,
입법자들은 옛 범죄군을
그대로 유지했다.
술라와 아우구스투스의 법률들은
로마 제국의 형법의 원천이 되었거니와,
그 법률들이 후대에 물려준 범죄의 분류만큼
특이한 것도 없을 것이다.
한 가지 예만 들자면,
\hemph{위증}은 언제나
\hemph{절단상해}\latin{cutting and wounding} 및
\hemph{독살}과 함께 분류되었으니,
이는 분명
술라의 법률 중 하나인
`자살\hanja{刺殺}범과 독살범에 관한
코르넬리우스 법'\latin{Lex Cornelia de Sicariis et Veneficis}이
이 세 가지 범죄 형태 모두를 동일한 \wi{상설사문회}의 관할로 삼았기
때문인 것이다.\footnote{%
  이 코르넬리우스 법에서 말하는 `위증'은 일반적인 위증이 아니라
  이른바 사법살인, 즉 형사재판을 통해 누군가를 죽이려는 의도로
  행해진 위증을 뜻하던 것으로 보인다.
  \latin{D.\,48.8.1.1.} }
나아가
이러한 변칙적인 범죄 분류는 로마인들의 일상언어에도 영향을 끼친 것으로 보인다.
하나의 법률에 나열된 모든 범죄를
자연스레
목록의 첫 번째 이름을 가지고 지칭하는 습관이 생겼던 것으로 보이며,
이들 범죄 모두를 재판하는 법원도 이 이름으로  부르게 되었음이 분명하다.
그리하여
`간통사문회'\latin{quaestio de adulteriis}가 재판하는
모든 범죄가 `간통죄'로 불리게 되었던 것이다.

\para{이후의 형법}
지금까지 로마 사문회의 역사와 성격을 다루었거니와,
다른 곳에서는 형법의 형성과정을 알려주는 예를 찾을 수 없기 때문이다.
아우구스투스 황제에 의해
마지막 \wi{사문회}가 설치되었고,
그때부터 로마는 어느 정도 완비된 형법을 가진다고 말할 수 있게 되었다.
이러한 형법의 성장에 병행하여
이와 유사한 과정, 즉
내가 불법행위의 범죄로의 전환이라고 불렀던 과정이 진행되었다.
로마의 입법자들은,
비록 보다 극악한 범죄에 대해서는
민사적 구제수단을 없애지 않았지만,
피해자가 선호할 것이 분명한 구제수단을 그들에게 제공했던 것이다.
아우구스투스가 그의 입법을 마무리한 후에도,
근대사회라면 범죄로만 취급하는
몇몇 침해가 여전히 불법행위로 간주되고 있었다.
이들이 범죄로 되는 것은,
언제인지는 확인할 수 없으나, 후대의
법이 \wi{학설휘찬}\latin{Digest}에서
`\wi{비상심리절차에 의하는 범죄}'\latin{crimina extraordinaria}라고
불리는 새로운 유형의 범죄들을 인정하면서부터이다.
분명
이들은
로마법 이론상 불법행위로만 취급되던 행위군이었으나,
사회의 존엄성에 대한 관념이 성장하면서
이들을 위반한 자가 단지 돈으로만 배상하면 그만인 상태를
더는 용납할 수 없게 되었고,
따라서 피해자가 원한다면
비상심리절차에 의하는 범죄로,
즉 통상적인 소송절차와는 여러모로 구별되는 구제방법으로,
가해자를 소추할 수 있게 허용한 것으로 보인다.
비상심리절차에 의하는 범죄가 인정되면서부터
로마 국가의 범죄목록은 근대 세계의 여느 국가들 못지 않는 수준이
되었음에 틀림없다.

\para{주권은 사법권의 원천}
로마 제국 하에서의 형사사법체계를
상세하게 기술할 필요는 없겠으나,
그것의 이론과 실무가 근대사회에 강력한 영향을 끼쳤음은
지적해두어야 하겠다.
황제들은
\wi{사문회}들을 즉시 폐지하지는 않았으며,
처음에는
원로원에게
폭넓은 형사사법권을
주었다.
아무리
원로원이
사실상 굴종적이었다고 해도,
원로원에서 황제는
명목상
다른 원로원의원들과 대등한 관계였을 뿐이다.
그러나
황제는
처음부터
몇몇 부수적인 행사재판권을 행사했고,
자유롭던 공화정 시절의 기억이 서서히 사라지면서
이것이 옛 법원들을 꾸준히 대체해나가게 된다.
차츰
범죄의 처벌은
황제가 직접 임명한 관리들의 손에 넘어갔고,
원로원의 특권도 황제의 추밀원\hanjalatin{樞密院}{privy council}으로 넘어갔거니와,
결국 이 황제의 추밀원이 형사 최고법원이 되었다.\footnote{%
  여기서 `추밀원'은 `근위장관'(近衛長官\,praefectus praetorio)을
  뜻하는 듯하다. 형사뿐 아니라 민사재판도 담당했다.
  파피니아누스, 울피아누스, 파울루스 등이 이 직을 역임한 대표적 법학자들이다.}
이러한 영향력의 결과,
근대인들에게도 친숙한 원리가 부지불식간에 형성되어갔으니,
주권자가 모든 사법권의 원천이자 모든 은혜의 저장소라는 원리가 그것이다.
이것은 아첨과 굴종이 증가한 결과물이 아니라,
이 시기에 이르러 완성된 제국의 중앙집권화의 결과물이었다.
형법의 이론은 사실상 첫 출발점으로 되돌아갔다고 할 수 있다.
자신이 입은 피해에 대해 국가가 스스로 나서서 보복한다는 데서
형법의 역사가 출발했거니와,
범죄의 처벌이
인민의 대표자이자 수임자\hanja{受任者}인 주권자에게
특별한 방식으로 속한다는 원리에서 끝맺고 있는 것이다.
새로운 견해가 옛 것과 다른 점은
정의의 수호자라는 직무로 인해
장엄하고도 위엄있는 분위기가
주권자의 인격을 감싸게 되었다는 것 정도이다.

\para{교회의 영향}
사법권에 대한 주권자의 관계에 관한 후기 로마의 견해는
내가 사문회의 역사로써 예시한 일련의 변화 과정을
근대사회가
되풀이할 필요성을 덜어주었다.
서유럽에 정착한 거의 모든 민족의 원시법에서는
범죄의 처벌이 전체 자유민의 집회에 속한다는 옛 관념의 흔적이 발견된다.
몇몇 나라---스코틀랜드가 그중 하나라고 한다---의
현존하는 법원은 그 기원을 입법기구의 위원회로 소급할 수 있다.
그러나 형법의 발달은 널리 두 가지 원인에 의해 촉진되었거니와,
로마 제국에 대한 기억이 그 하나요,
교회의 영향이 다른 하나이다.
한편으로,
잠시였지만 샤를마뉴 가\hanja{家}의 재위로 인해 이어진
황제들의 존엄성이라는 전통이 주권자들을 감싸고 있었으니,
그 위신은 단순히 만족\hanja{蠻族}의 수장으로서는 도저히 얻을 수 없는 것이었다.
또한 그 전통은
사회의 수호자이자 국가의 대표자로서의 주권자의 성격을
봉건 위계의 최하위 유력자에게까지
전해주고 있었다.
다른 한편으로,
교회는
참혹한 폭력을 억제하려는 열망에서
중대한 범죄행위를 처벌할 권위를 찾고자 했고,
처벌권한이 세속권력에게 주어져있다는 성경구절에서 그것을 발견했다.
악행을 행하는 자를 두려움에 떨게 하기 위해
세속 통치자가 존재한다는 것을 증명하는 데
신약성서가 원용되었고,\footnote{%
  ``관원들은 악을 행하는 자에게나 두려운 존재이지 \ldots''
  로마서 \latin{13:3.}}
구약성서는 ``다른 사람의 피를 흘리면 그 사람의 피도 흘릴 것이니''라는
주장에 원용되었다.\footnote{%
  창세기 \latin{9:6.}}
생각건대,
의심할 여지 없이
범죄라는 주제에 관한 근대적 관념은
암흑시대에 교회가 내세운 두 가지 가정\hanja{假定}에 근거한다.
하나는
각 위계의 봉건 통치자는 사도 바울이 말한
로마 관원들에 비견될 수 있다는 것이다.
다른 하나는
그가 처벌해야 할 범죄는
모세의 십계명으로 금지된 것들,
정확히 말하면 교회가 자신의 관할로 유보하지 않은 것들이라는 것이다.
이단\hanja{異端}---이는 십계명의 첫 번째와 두 번째 계명에
포함된 것으로 볼 수 있다---과 간통과 위증은
교회재판이 관할하는 범죄였고,
\index{교회법}%
교회는
특별히 사안이 중대한 경우
보다 가혹한 처벌을 부과할 목적에서만
세속권력의 협력을 용인했다.
동시에
교회는
살인과 강도는
그 다양한 변종들과 함께
세속 통치자의 관할에 속한다고 가르쳤다.
다만 그것은 세속 통치자 지위의 우연한 속성이 아니라
신의 명시적 말씀에 의해 그러하다는 것이다.

\para{알프레드 대왕의 형법}
\wi{알프레드 대왕}의 법전 중 한 구절\paren{켐블, 2.209}은
형법의 기원에 관한
당대의 여러 관념들 간의 갈등을 사뭇 명료하게 보여주고 있다.
알프레드는 그것을
일부는 교회의 권위에,
일부는 현자\hanjalatin{賢者}{witan}들의 권위에
귀속시키고 있음을 볼 수 있다.
그러면서도
주군에 대한 반역은,
\wi{대역죄}\latin{majestas}에 관한
로마법이
황제에 대한 반역을 통상적인 법에서 면제시켰듯이,
통상적인 법에서 면제된다고 명시하고 있다.
알프레드는 이렇게 적고 있다.
``이후 많은 나라들이 기독교 신앙을 받아들였고,
땅 위의 모든 곳에서 종교회의가 개최되었다.
영국 민족 사이에서도 기독교 신앙을 받아들인 후
성스러운 주교들의 모임과 고귀한 현자들의 모임이 있었다.
그리하여 그들은 이렇게 선포했다.
그리스도께서 가르치신 자비 덕분에,
세속 주군들은,
그들의 허락 하에,
모든 범죄자들에게서 그들이 정한 \wi{속죄금}\latin{bot}을
죄지음 없이
취할 수 있거니와,
다만 주군에 대한 반역의 경우에는
그들은 어떠한 자비도 베풀지 아니할 것이니,
전능하신 주께서는 주를 능멸한 자에게 판결을 내려주지 않으셨고,
그리스도께서도 그분을 죽음에 팔아넘긴 자들에게 판결을 내려주지 않으셨으며,
또한 말씀하시기를 그분과 같이 주군을 사랑하라 하셨기 때문이노라.''\footnote{%
  \latinmarks
  John Mitchell Kemble,
  \textit{The Saxons in England: A History of the English Commonwealth Till the Period of the Norman Conquest}, Vol.\,2,
  London: Longman, Brown, Green \& Longmans, 1849,
  pp.\,208f.}




