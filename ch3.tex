\chapter{자연법과 형평법}

내재적인 탁월함을 가진
일군의 법원리가
낡은 법을 대체한다는 이론은
로마에서도 영국에서도 아주 일찍부터 통용되었다.
어떤 법체계에서도 발견되는
이러한 원리들을 앞 장에서 우리는 형평법\latin{equity}이라고 불렀다.
곧 살펴보겠지만, 이 용어는
로마 법학자들이 이러한 법변화 작인\hanja{作因}을 지칭하는
여러 명칭 가운데 하나, 오직 하나에 불과했다.
영국에서는 형평법법원\latin{Court of Chancery}의 법이
형평법이라는 이름으로 불리고 있거니와,
이것은 별도의 논저를 통해서만 제대로 논의될 수 있을 것이다.
그것의 구조는 대단히 복잡하고, 여러 다양한 원천에 기원을 두고 있다.
초기 챈슬러들은 성작자들이었기에 그들은 교회법으로부터
형평법의 바탕이 되는 법원리들을 이끌어냈다.
후대의 챈슬러들은
세속 사건에 적용할 수 있는 법규칙이 교회법보다 더 풍부한 로마법을
자주 원용했다.
그들의 판결문 중에는, 비록 출처는 밝히고 있지 않으나
로마법대전에서 따온 텍스트 전체가 토씨 하나 바뀌지 않고
들어가 있는 경우가 적지 않다.
더 최근에는, 특히 18세기 후반에는,
네덜란드의 학자들의 법학 및 윤리학의 혼합체계가 영국 법률가들에 의해
널리 연구되었거니와,
이들 연구는
탈보 경\latin{Lord Talbot}에서 엘던 경\latin{Lord Eldon}에 이르는
챈슬러들의 형평법 법원 판결에 큰 영향을 끼쳤다.
이렇게 다양한 기원의 요소들로 구성된 형평법 체계는,
보통법의 유추적용과 정합성을 가져야 한다는 요청으로 인해
그 성장이 크게 제한되었으나,
상대적으로 새로운 법원리들을 기술하는 일에 언제나 응답해왔다.
그 법원리들은 내재적인 윤리적 탁월함에 있어 영국의 옛 법을 능가한다고
주장되었다.

\para{로마의 형평법}
로마의 형평법은 구조가 훨씬 단순했고, 그 발달과정도 보다 쉽게 추적할 수 있다.
그것의 성질과 역사는 주의깊게 살펴볼 가치가 있다.
그것은 인간의 사고에 심대한 영향을 끼친 개념들을 창조하였고,
인간의 사고를 통해 인류의 운명에도 심대한 영향을 주었다.

로마인들은 그들의 법체계가 두 부분으로 구성된다고 보았다.
유스티니아누스 황제의 명에 의해 편찬된 법학제요는 이렇게 말한다.
``법과 관습에 의해 규율되는 모든 민족들은, 부분적으로는 그들 자신의
고유한 법에 의해, 부분적으로는 모든 인류에 공통되는 법에 의해,
통치된다. 당해 인민이 제정한 법은 그 민족의 시민법\latin{civil law}이라
불리고, 자연이성\latin{natural reason}이 모든 인류에게 지시한 법은
모든 민족이 사용하기 때문에
만민법\hanjalatin{萬民法}{Law of Nations}이라 불린다.''
여기서 ``자연이성이 모든 인류에게 지시한 법''은 법무관의 고시가
로마법에 엮어넣은 요소를 의미했다.
다른 곳에서는 이것을 단순히 자연법\latin{ius naturale}이라고 불렀는데,
자연법은 자연이성뿐 아니라 자연적 형평\latin{naturalis aequitas}에 의해서도
명령된다고 여겨졌다.
나는 만민법, 자연법, 형평법이라는 유명한 표현들의 기원을 탐구할 것이고,
또 이들이 지시하는 관념들의 상호 관련성을 탐구할 것이다.

\para{만민법}
로마의 역사를 조금만 살펴보아도,
여러 다른 이름으로 불리우며
로마의 영토 안에
살고 있는 외인\hanja{外人}들의 존재에 의해 공화국의 운명이
좌우되었음을 알고 놀라게 된다.
이러한 이주의 원인은 후대에 분명히 드러나게 되거니와,
왜 모든 민족의 사람들이 세상의 주인인 도시로 몰려드는지를
짐작하기란 그다지 어렵지 않다.
그러나 외인들과 준\hanja{準}시민\latin{denizen}들의 대규모 존재는
로마의 초기 역사에서도 그 기록이 발견된다.
말할 것도 없이,
다수의 약탈적 부족들로 구성된 고대 이탈리아의 사회적 불안정성은
사람들로 하여금, 공동체와 그 구성원들을 외적으로부터 보호할 수 있는
강력한 힘을 가진 공동체의 영토에 몰려가 살도록 하는 유인을 제공했다.
그 보호가 과중한 세금, 선거권 박탈, 사회적 신분 저하의 대가로
주어지는 것이라 할지라도 말이다.
하지만 이러한 설명은 불완전하며,
활발한 상거래 관계를 고려에 넣어야만 완전해질 수 있을 것이다.
이러한 상거래 관계는 공화국의 군사적 전승\hanja{傳承}에는
별로 반영되어 있지 않지만,
분명 로마는 카르타고와, 그리고 이탈리아 안에서,
선사시대부터 상거래 관계를 유지해온 것으로 보인다.
그 원인이야 무엇이든 간에,
국가 내에 외인들의 존재는
로마의 전체 역사 과정을 결정했으며,
그 모든 역사단계는 완고한 국수주의와 이방인 인구 간의 갈등의 이야기를
크게 벗어나지 않는다.
이와 같은 것이 현대에는 발견되지 않거니와,
우선 현대 유럽 국가들은 다수 국민들이 너무 많다고 여길 정도의
외국이민을 거의 혹은 전혀 받아들이지 않아왔기 때문이며,
또한
국왕이나 주권기구에 대한 충성으로 결합되는 현대국가들은
상당한 규모의 이민자 집단도
신속하게 흡수하기 때문이다.
고대 세계는 이러한 신속한 흡수를 알지 못했다.
고대사회에서 국가의 최초 시민들은 언제나 스스로를 혈연의 친족관계에 의해
결합되어 있다고 생각했고,
특권의 평등을 주장하는 것은 그들의 생래적 권리를 찬탈하려는 것이라
여기며 분개했다.
로마 공화정 초기에는
공법 영역은 물론이고 사법 영역에서도
외인들의 철저한 배제가 만연했다.
외인이나 준\hanja{準}시민은
국가 영역에 해당하는 어떠한 제도에도 참여할 수 없었다.
그들은 로마시민법\latin{Quiritarian Law}의 혜택도 누릴 수 없었다.
그들은 초창기 로마인들의 물권이전방식이자 계약방식이었던
구속행위\hanjalatin{拘束行爲}{nexum}의 당사자가 될 수 없었다.
그들은 문명의 유아기로 그 기원이 거슬러올라가는 소송방식인
신성도금소송\hanjalatin{神聖賭金訴訟}{sacramental action}도 제기할 수 없었다.
그럼에도 불구하고 로마의 이익도 로마의 안전도 그들이 법적 보호를 박탈당하는
상태를 허용하지 않았다.
어떤 고대 공동체도 약간의 평화교란으로도 전복될 수 있는 위험을 안고 있었다.
그리하여 단순한 자기보존의 본능에서 로마인들은
외인들의 권리와 의무를 조정하는 방법을 고안해냈거니와,
그렇지 않았다면---그리고 이것은 고대 세계에서는
진짜로 중대한 위험요인이었는데---외인들의 무장봉기가 일어났을 것이기 때문이다.
더욱이 로마 역사의 어느 시기에도 외인들의 상거래가 완전히 무시된 적은
한 번도 없었다.
따라서,
당사자 모두가 외인인 분쟁이나 시민과 외국인 간의 분쟁에 대해
재판권을 처음 인정한 것은
반쯤은 치안을 위한 조치였을 것이고, 반쯤은 상거래의 지속을 위해서였을 것이다.
이러한 재판권의 인정은, 재판의 대상이 된 문제들을 해결할 어떤 법원리들을
발견해야 할 필요성을 즉시 불러왔다. 그리고
로마 법률가들이 이들 대상에 적용한 법원리들은 그 시대의
두드러진 성격을 반영한 것이었다.
전술했듯이 그들은 이들 새로운 사건에 로마 시민법을 적용하기를 거부했다.
그들이 거부한 이유는 분명, 외인인 당사자의 출신 모국의 법을 적용하는 것은
일종의 체면손상이라고 여겼기 때문일 것이다.
그들이 채택한 방법은 로마를 비롯하여
그 이주민들이 태어난 다른 이탈리아 공동체들에
공통되는 법규칙을 찾아내 적용하는 것이었다.
다시 말해서, 그들은
모든 민족들에 공통되는 법, 즉 만민법\latin{ius gentium}의 원시적이고
문자적인 의미에 합치하는 법체계를 만들어냈다.
실로 만법법은 옛 이탈리아 부족들의 관습 가운데 공통된 요소의 총합이었다.
이들 부족이 로마인들이 관찰할 수 있었던 \hemph{모든 민족들}이었고,
로마의 영역에 지속적으로 이주민 무리를 보낸 민족들이었던 것이다.
어떤 특정 관행이 개별 민족들의 대다수에서 공통적으로 발견되면,
모든 민족들에 공통되는 법, 즉 만민법으로 선언되었다.
그리하여,
비록 물건의 양도는 로마 인근의 여러 다른 국가들에서 각기 다른 형식으로
수행되었으나, 그 실제적 이전인 인도\hanjalatin{引渡}{tradition}, 즉
양도할 목적물을 교부하는 것은 그들 모두에서 의례행위의 일부를 구성했다.
가령 인도는 로마에 특유한 양도방식인
악취행위\hanjalatin{握取行爲}{mancipation}의 일부분을, 비록
부차적인 부분에 불과했지만, 구성했다.
따라서,
법학자들이 관찰할 수 있었던 양도행위 방식들의 유일한
공통요소였을 인도는
만민법, 즉 모든 민족들에 공통되는 법의 규칙으로 선언되었다.
다른 수많은 관찰들도 마찬가지 방법으로 심사대상이 되었다.
공통의 대상을 가진 관찰들 모두에서
어떤 공통의 성질이 발견되면,
이러한 성질은 만민법에 속하는 것으로 분류되었던 것이다.
따라서 만민법은,
여러 이탈리아 부족들 사이의 지배적인 제도들에 공통적이라고
관찰에 의해
정해지는, 그러한
법규칙과 법원리의 총체였다.

