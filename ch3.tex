\chapter{자연법과 형평법}

내재적인 탁월함을 가진
일군의 법원리가
낡은 법을 대체한다는 이론은
로마에서도 영국에서도 아주 일찍부터 통용되었다.
어떤 법체계에서도 발견되는
이러한 원리들을 앞 장에서 우리는 형평법\latin{equity}이라고 불렀다.
곧 살펴보겠지만, 이 용어는
로마 법학자들이 이러한 법변화 작인\hanja{作因}을 지칭하는
여러 명칭 가운데 하나, 오직 하나에 불과했다.
영국에서는 형평법법원\latin{Court of Chancery}의 법이
형평법이라는 이름으로 불리고 있거니와,
이것은 별도의 논저를 통해서만 제대로 논의될 수 있을 것이다.
그것의 구조는 대단히 복잡하고, 여러 다양한 원천에 기원을 두고 있다.
초기 챈슬러들은 성직자들이었기에 그들은 \wi{교회법}으로부터
형평법의 바탕이 되는 법원리들을 이끌어냈다.
후대의 챈슬러들은
세속 사건에 적용할 수 있는 법규칙이 교회법보다 더 풍부한 로마법을
자주 원용했다.
그들의 판결문 중에는, 비록 출처는 밝히고 있지 않으나
로마법대전에서 따온 텍스트 전체가 토씨 하나 바뀌지 않고
들어가 있는 경우가 적지 않다.
더 최근에는, 특히 18세기 후반에는,
네덜란드 공법학자들의 법학 및 윤리학의 혼합체계가 영국 법률가들에 의해
널리 연구되었거니와,
이들 연구는
탈보트 경\latin{Lord Talbot}에서 엘던 경\latin{Lord Eldon}에 이르는
챈슬러들의 형평법법원 판결에 큰 영향을 끼쳤다.
이렇게 다양한 기원의 요소들로 구성된 형평법 체계는,
보통법의 유추적용과 정합성을 가져야 한다는 요청으로 인해
그 성장이 크게 제한되었으나,
상대적으로 새로운 법원리들을 기술하는 일에 언제나 응답해왔다.
그 법원리들은 내재적인 윤리적 탁월함에 있어 영국의 옛 법을 능가한다고
주장되었다.

\para{로마의 형평법}
로마의 형평법은 구조가 훨씬 단순했고, 발달과정도 보다 쉽게 추적할 수 있다.
그것의 성질과 역사는 주의깊게 살펴볼 가치가 있다.
그것은 인간의 사고에 심대한 영향을 끼친 개념들을 창조했고,
인간의 사고를 통해 인류의 운명에도 심대한 영향을 주었다.

로마인들은 그들의 법체계가 두 부분으로 구성된다고 보았다.
\wi{유스티니아누스} 황제의 명에 의해 편찬된 \wi{법학제요}는 이렇게 말한다.
``법과 관습에 의해 규율되는 모든 민족들은, 부분적으로는 그들 자신의
고유한 법에 의해, 부분적으로는 모든 인류에 공통되는 법에 의해,
통치된다. 당해 인민이 제정한 법은 그 민족의 \wi{시민법}\latin{civil law}이라
불리고, 자연이성\latin{natural reason}이 모든 인류에게 지시한 법은
모든 민족이 사용하기 때문에
\wi{만민법}\hanjalatin{萬民法}{law of nations}이라 불린다.''\footnote{%
  \latin{Inst.\,1.2.1.}}
여기서 ``자연이성이 모든 인류에게 지시한 법''은 \wi{법무관}의 \wi{고시}가
로마법에 엮어넣은 요소를 의미했다.
다른 곳에서는 이것을 단순히 \wi{자연법}\latin{ius naturale}이라고 불렀는데,
자연법은 자연이성뿐 아니라 자연적 \wi{형평}\latin{naturalis aequitas}에 의해서도
명령된다고 여겨졌다.
나는 여기서 만민법, 자연법, 형평법이라는 유명한 표현들의 기원을 탐구할 것이고,
또 이들이 지시하는 관념들의 상호 관련성을 탐구할 것이다.

\para{만민법}
로마의 역사를 조금만 살펴보아도,
여러 다른 이름으로 불리우며
로마의 영토 안에
살고 있는 외인\hanja{外人}들의 존재에 의해 공화국의 운명이
좌우되었음을 알고 놀라게 된다.
이러한 이주의 원인은 후대에 분명히 드러나게 되거니와,
왜 모든 민족의 사람들이 세상의 주인\latin{mistress of the world}인 도시로 몰려드는지를
짐작하기란 그다지 어렵지 않다.
그러나 외인들과 준\hanja{準}시민\latin{denizen}들의 대규모 존재는
로마의 초기 역사에서도 그 기록이 발견된다.\footnote{%
  여기서 `준시민'은 `라틴인'(latini)을 뜻하는 듯하다.
  라틴인의 신분을 가진 자는 로마시민은 아니었으나 적어도
  라틴 식민도시들은
  로마시민과 상거래를 할 수 있는 통상권(\latin{ius commercii; commercium})은
  가지는 것이 일반적이었다. }
말할 것도 없이,
다수의 약탈적 부족들로 구성된 고대 이탈리아의 사회적 불안정성은
사람들로 하여금, 공동체와 그 구성원들을 외적으로부터 보호할 수 있는
강력한 힘을 가진 공동체의 영토에 몰려가 살도록 하는 유인을 제공했다.
그 보호가 과중한 세금, 선거권 박탈, 사회적 신분 저하의 대가로
주어지는 것이라 할지라도 말이다.
하지만 이러한 설명은 불완전하며,
활발한 상거래 관계를 고려에 넣어야만 완전해질 수 있을 것이다.
이러한 상거래 관계는 공화국의 군사적 전승\hanja{傳承}에는
별로 반영되어 있지 않지만,
분명 로마는 카르타고와, 그리고 이탈리아 안에서,
선사시대부터 상거래 관계를 유지해온 것으로 보인다.
그 원인이야 무엇이든 간에,
국가 내에 외인들의 존재는
로마의 전체 역사 과정을 결정했으며,
그 모든 역사단계는 완고한 국수주의와 이방인 인구 간의 갈등의 이야기를
크게 벗어나지 않는다.
이와 같은 것이 현대에는 발견되지 않거니와,
우선 현대 유럽 국가들은 다수 국민들이 너무 많다고 여길 정도의
외국이민을 거의 혹은 전혀 받아들이지 않아왔기 때문이며,
또한
국왕이나 주권기구에 대한 충성으로 결합되는 현대국가들은
상당한 규모의 이민자 집단도
신속하게 흡수하기 때문이다.
고대 세계는 이러한 신속한 흡수를 알지 못했다.
고대사회에서 국가의 최초 시민들은 언제나 스스로를 혈연의 친족관계로
결합되어 있다고 생각했고,
특권의 평등을 주장하는 것은 그들의 생래적 권리를 찬탈하려는 것이라
여기며 분개했다.
로마 공화정 초기에는
공법 영역은 물론이고 사법 영역에서도
외인들의 철저한 배제가 만연했다.
외인이나 준\hanja{準}시민은
국가 영역에 속하는 어떠한 제도에도 참여할 수 없었다.
그들은 로마\wi{시민법}\latin{quiritarian law}의 혜택도 누릴 수 없었다.
그들은 초창기 로마인들의 물권양도방식이자 계약방식이었던
\wi{구속행위}\hanjalatin{拘束行爲}{nexum}의 당사자가 될 수 없었다.
그들은 문명의 유년기로 기원이 거슬러올라가는 소송방식인
\index{신성도금법률소송}신성도금소송\hanjalatin{神聖賭金訴訟}{sacramental action}도 제기할 수 없었다.
그럼에도 불구하고 로마의 이익도 로마의 안전도 그들이 법적 보호를 박탈당하는
상태를 허용하지 않았다.
어떤 고대 공동체도 약간의 평화교란으로도 전복될 수 있는 위험을 안고 있었다.
그리하여 단순한 자기보존의 본능에서 로마인들은
외인들의 권리와 의무를 조정하는 방법을 고안해냈거니와,
그렇지 않았다면---그리고 이것은 고대 세계에서는
정말 중대한 위험요인이었는데---외인들의 무장봉기가 일어났을 것이기 때문이다.
더욱이 로마 역사의 어느 시기에도 외인들의 상거래가 완전히 무시된 적은
한 번도 없었다.
따라서,
당사자 모두가 외인인 분쟁이나 시민과 외인 간의 분쟁에 대해
재판권을 인정한 것은 애초에는
반쯤은 치안을 위한 조치였을 것이고, 반쯤은 상거래의 지속을 위해서였을 것이다.
이러한 재판권의 인정은, 재판의 대상이 된 문제들을 해결할 어떤 법원리들을
발견해야할 필요성을 즉시 불러왔다. 그리고
로마 법률가들이 이들 대상에 적용한 법원리들은 그 시대의
두드러진 성격을 반영한 것이었다.
전술했듯이 그들은 이들 새로운 사건에 로마 \wi{시민법}을 적용하기를 거부했다.
그들이 거부한 이유는 분명, 외인인 당사자의 출신 모국의 법을 적용하는 것은
일종의 체면손상이라고 여겼기 때문일 것이다.
그들이 채택한 방법은 로마를 비롯하여
그 이주민들이 태어난 다른 이탈리아 공동체들에
공통되는 법규칙을 찾아내 적용하는 것이었다.
다시 말해서, 그들은
모든 민족들에 공통되는 법, 즉 \wi{만민법}\latin{ius gentium}의 원래의
문자적인 의미에 합치하는 법체계를 만들어냈다.
실로 만법법은 옛 이탈리아 부족들의 관습 가운데 공통된 요소의 총합이었다.
이들 부족이 로마인들이 관찰할 수 있었던 \hemph{모든 민족들}이었고,
로마의 영역에 지속적으로 이주민 무리를 보낸 민족들이었던 것이다.
어떤 특정 관행이 개별 민족들의 대다수에서 공통적으로 발견되면,
모든 민족들에 공통되는 법, 즉 만민법으로 선언되었다.
가령,
비록 물건의 양도는 로마 인근의 여러 다른 국가들에서 각기 다른 방식으로
수행되었으나, 그 실제적 이전인 \wi{인도}\hanjalatin{引渡}{tradition}, 즉
양도의 목적물을 교부하는 것은 그들 모두에서 의례행위의 일부를 구성했다.
예컨대 인도는 로마 특유의 양도방식인
\wi{악취행위}\hanjalatin{握取行爲}{mancipation}의 일부분을, 비록
부차적인 부분에 불과했지만, 구성했던 것이다.
따라서,
법학자들이 관찰할 수 있었던 양도행위 방식들의 유일한
공통요소였을 인도는
\wi{만민법}, 즉 모든 민족들에 공통되는 법규칙으로 선언되었다.
다른 수많은 관찰들도 마찬가지 방법으로 심사대상이 되었다.
공통의 대상을 관찰하여 모두에서
어떤 공통의 성질이 발견되면,
이러한 성질은 만민법에 속하는 것으로 분류되었던 것이다.
따라서 만민법은,
여러 이탈리아 부족들의 지배적 제도들에 공통적이라고
관찰로써 확인된,
그러한 법규칙과 법원리의 총체였다.

만민법의 기원에 관한 이러한 서술은
로마 법률가들이 만민법을 특별히 존중했을 것이라는 오해에 대한
좋은 방패막이가 될 것이다.
\wi{만민법}은 부분적으로는 일체의 외국법에 대한 경멸의 결과였으며,
부분적으로는 그들 고유의 \wi{시민법}의 혜택을 외인들에게 주기를 꺼려하는
마음의 결과였다.
물론, 로마 법학자들이 수행했던 역할을 오늘날의 우리가 수행한다면,
우리는 만민법을 사뭇 다르게 접근했을 것이다.
우리라면 그렇게 다양한 관행들을 관통하는 배경적 요소로 판별된 것에 대해
어떤 탁월성이나 우선성을 부여할 것이다.
우리라면 그렇게 보편적인 법규칙과 법원리에 어떤 존중심을 가질 것이다.
우리라면 그 공통의 요소를 당해 거래의 본질이라고 말할 것이다.
그리고 공동체마다 서로 다른 그밖의 의례적 장치들은
우연적이고 부수적인 것으로 폄하할 것이다.
혹은, 우리가 비교하고 있는 민족들이 한때 어떤 위대한 공통의 제도를
따랐고 만민법은 그것의 재현\hanja{再現}이라고 추론할 것이다.
그리고 개별 국가들의 복잡다기한 관행들은 한때 원시국가를 규율했던
보다 단순한 법제의 타락이고 퇴행일 뿐이라고 추론할 것이다.
하지만 근대적 관념이 이끌어낸 이러한 결과들은
초기 로마인들이 본능적으로 느끼고 있던 것들과
거의 항상 정반대이다.
우리가 존중하고 칭송하는 것을 그들을 싫어하고 질시하고 두려워한다.
그들의 법 중에 그들이 애정했던 부분---\wi{악취행위}의 엄숙한 몸짓,
\wi{언어계약}\latin{verbal contract}의 정연한 질문과 답변,
변론과 소송절차의 한없는 형식주의 등등---은
오늘날의 학자라면 모두 우연적이고 일시적인 것으로 무시할 것들뿐이다.
만민법은 단지 정치적 필요 때문에 어쩔 수 없이 용인한 법체계에 불과했다.
그들은 외인들을 사랑하지 않았듯이 만민법도 사랑하지 않았다.
만민법은 외인들의 법제도에서 추출한 것이고 외인들의 이익을 위한 것에 불과했던
것이다.
만민법이 그들의 존중을 받기 위해서는 근본적인 혁명이 필요했다.
그 일이 실제 발생했을 때 그것은 너무나 근본적이었으니,
만민법에 대한 현대적 평가가 방금 언급한 그들의 것과 다른 진정한 이유는
현대의 법학과 현대의 철학이
이 주제에 관한 후대 법학자들의 성숙한 관념을
물려받았기 때문이다.
시민법에 붙은 비천한 부속물에서
만민법은 일약 모든 법이 따라야할 위대한, 그러나 아직은 발달 중에 있는,
전범\hanja{典範}으로 간주되는 시대가 도래했다.
그 결정적 전기는
로마인들이
모든 민족들에 공통인 법의 실무적 집행에
그리스의 자연법 이론을
적용하기 시작하면서 도래했다.

\para{자연법}
\wi{자연법}\latin{ius naturale}은 \wi{만민법}을 특정한 이론의 관점에서 바라본 것에
지나지 않는다.
법률가의 특징인 분류 성향에 따라
법학자 울피아누스가
이 둘을 구분하려는 애처로운 시도를 했지만,\footnote{%
  ``만민법과 자연법이 다른 것은 쉽게 알 수 있거니와,
  자연법은 모든 동물에 공통적인 법이지만 만민법은 인간들 사이에서만
  공통적인 법이다.'' \latin{D.\,1.1.1.4.}}
훨씬 높이 평가되는 \wi{가이우스}의 말에 따르면, 그리고
앞서 인용한 \wi{법학제요}의 문구에 따르면,
이들 표현은 의심의 여지 없이 사실상 서로 바꾸어 쓸 수 있는
것들이었다.\footnote{%
  다만 노예제도에 관한 한 자연법과 만민법은
  서로 분기했다. 노예제도는 고대 모든 민족들에서 발견할 수 있었으나,
  자연법상으로는 모든 인간이 자유롭게 태어났다고 여겨졌다. \latin{Inst.\,1.2.2.}}
그들 간의 차이는 순전히 역사적인 것이었으며
본질적인 구별은 성립될 수 없었다.
\wi{만민법}\latin{ius gentium}, 즉 모든 민족에 공통인 법과
\hemph{\wi{국제법}}\latin{international law} 간의 혼동은 전적으로 근대적인 것임은
부연할 필요조차 없다.
국제법의 고전적 표현은
선전강화법\hanjalatin{宣戰講和法}{jus feciale},
즉 협상과 외교에 관한 법이었다.
하지만 만민법의 의미에 관한 모호한 인상은
독립 국가들 간의 관계가 자연법의 지배를 받는다는 오늘날의 이론을
낳는 데 크게 기여했을 것임에 틀림없다.

여기서
자연과 \wi{자연법}에 관한 그리스인들의 관념을 살펴볼 필요가 생긴다.
퓌시스\greek{φύσις}는 라틴어로 나투라\latin{natura}, 우리말로는
자연\latin{nature}이라 번역되는데,
확실히 원래는 물질적인 우주를 뜻하는 말이었다.
그러나 그것은 현대적 언어로 표현하기 힘든---고대와 현대의 지적인 거리가
그만큼 멀다---어떤 관점에서 사고된
물질적 우주였다.
자연은 어떤 근원적인 요소 또는 근원적인 법칙의 결과로서의
물리적 세계를 의미했다.
초기 그리스 철학자들은
창조의 과정을 어떤 단일한 원리의 발현으로 설명하곤 했거니와,
그 원리를 운동, 힘, 불, 습기, 생성 등으로 다양하게 주장했다.
가장 단순하고 가장 고대적인 의미의 자연은 다름 아니라
이렇게 어떤 원리의 발현으로
간주된 물리적 우주였다.
후대의 그리스인들은, 그동안 위대한 그리스 지식인들이 벗어났던 길을 되돌려,
자연 개념의 \hemph{물리적} 세계에 \hemph{정신적} 세계를 추가했다.
자연이라는 말이 확장되어 가시적인 피조물뿐만 아니라 인간의 사상, 관찰, 소망까지
포괄하게 된 것이다.
그럼에도 불구하고 여전히, \hemph{자연}이라는 단어로 그들이 이해한 것은
그저 인간사회의 정신적 현상만이 아니라,
이러한 현상이 어떤 일반적이고 단순한 법칙으로 환원된다는 것까지 포함했다.

\para{스토아 철학}
초기 그리스 이론가들은 물리적 우주가 단순한 원시적 형태에서 우연의 장난으로
오늘날의 이질적인 복잡한 상태로 변화했다고 생각했다.
마찬가지로 이제 그들의 지적인 후손들도 만약 불행한 사고가 없었다면
인류는 보다 단순한 행위규칙과 보다 고난이 덜한 삶에 만족하며
살았을 것이라고 상상했다.
\hemph{자연}에 따라 사는 것이 인간이 창조된 목적이자
탁월한 인간이 달성해야할 목적으로 간주되기 시작했다.
\hemph{자연}에 따라 사는 것은 난잡한 습관과 저속한 것에의 탐닉을 넘어서는
고차원적인 행위법칙으로 고양되었고, 자제와 극기만이
이 법칙을 따를 수 있게 해 준다고 생각했다.
이 명제---자연에 따라 사는 것---가 \wi{스토아 철학}의 핵심 신조였던 것은
너무도 유명하다.
그리스의 정복과 더불어 이 철학은 즉시 로마 사회로 흘러들어갔다.
이 철학에는 로마의 엘리트 계급을 사로잡는 매력이 있었다. 그들은, 적어도 이론적으로는,
고대 이탈리아 민족의 단순한 습관을 고수했고
외국풍의 혁신에 굴복하기를 경멸했던 것이다.
이런 사람들은 자연에 따른 삶이라는 스토아의 명제에 즉각 매료되었다.
세상을 약탈하고 가장 사치스런 민족의 대명사가 된 저 제국의 수도에 만연했던
무절제한 방종에 비추어볼 때, 참으로 감사한 매료요, 생각건대 참으로 고귀한 매료였다.
새로운 그리스 철학의 사도 무리의 맨 앞 열은,
역사적으로 증명할 수는 없을지라도, 로마 법률가들이 차지하고 있었음이 거의 확실하다.
여러 증거로 추정컨대,
로마 공화국에는 사실상 두 종류의 전문직만 있었거니와,
군인들은 일반적으로 변혁을 추진하는 당파에 속했고,
법률가들은 일반적으로 변혁에 저항하는 당파의 선두에 서 있었다.

\para{법무관의 고시}
법률가들과 \wi{스토아 철학}의 결합은 수 세기에 걸쳐 지속되었다.
저명한 법학자들의 몇몇 초기 이름들은 스토아주의와 관련되어 있다.
나중에는 안토니누스 황조\latin{Antonine Caesars} 시대로 널리 합의되어 있는
로마법학의 황금기가 도래하거니와,
이 시기 황제들은 저 철학을 생활의 규칙으로 삼았던 유명한 사도들이었다.
특정 전문직 구성원들 사이에 이 신조가 장기간 확산됨에 따라
그들이 실무에 활용하고 영향을 끼쳤던 학문도 영향을 받지 않을 수 없었다.
저 스토아적 신조를 열쇠말로 사용하지 않으면
로마 법학자들이 남긴 유산에 속하는 몇몇 견해들은 거의 이해가 불가능해진다.
그러나 그렇다고 해서,
스토아주의가 로마법에 끼친 영향을,
스토아 교리에서 기원했다고 생각되는 법규칙의 숫자를 세어 측정하는 것은,
매우 흔하지만 심각한 오류에 해당한다.
스토아주의의 강점은,
때로 거부감을 불러일으키는 어처구니없는 행위준칙들에 있는 것이 아니라,
정념에의 저항을 가르치는 모호하지만 위대한 원리에 들어있다고 널리 인정되어왔다.
마찬가지로, 스토아주의로 대표되는 그리스 철학이 법학에 끼친 영향도
그것이 로마법에 기여한 여러 특정 견해들의 숫자가 아니라
그것이 가져다준 특유의 근본적인 가정\hanja{假定}에서 찾아야 한다.
자연이라는 단어가 로마인들이 일상적으로 사용하는 말이 되면서,
로마 법률가들 사이에서는
옛 만민법이 사실은 잃어버린 자연의 법전이라는 믿음이
점차 확산되어갔다.
또한 \wi{만민법} 원리에 기초하여 \wi{고시}법\hanja{告示法}을 형성함으로써
쇠퇴하기 시작한 법을 법무관들이 점차 다시 되살리고 있다는 믿음도 확산되어갔다.
이러한 믿음으로부터,
고시를 통해 가능한 한 많이 \wi{시민법}을 대체하는 것이,
원시상태의 인간에게 자연이 가르쳐준 제도들을 가능한 한 많이 되살리는 것이,
법무관의 의무라는 생각이 즉각 추론되어 나온다.
물론 이러한 방법으로 법을 개선하는 데는 많은 장애가 따른다.
법전문직 내에서도 극복해야할 편견들이 있었고,
로마인들의 습관도 꽤나 끈질겨서 단순한 철학 이론에 당장 굴복하지는 않았다.
\wi{법무관}들이 고시를 가지고 몇몇 법기술적 장애들과 싸운 간접적인 방법들을 통해
우리는 그들이 신중하게 준수해야만 했던 것들을 엿볼 수 있다.
또한 \wi{유스티니아누스} 시대에 이르기까지도 고법\hanja{古法}의 일부는
이러한 영향력에 완고하게 저항했던 것이다.
하지만 법 개선에 있어 로마인들의 진보는
자연법 이론의 자극이 주어지자마자 신속하게 전개되었다.
단순화와 일반화의 관념이 자연의 개념에 밀접히 연관되어 있었다.
그리하여 단순성, 조화성, 명료성이 좋은 법체계의 특징으로 간주되었고,
복잡한 언어, 복잡한 의례, 무의미한 장애물들은
모두 사라져갔다.
로마법을 현존의 모습으로 되살리는 데는
유스티니아누스의 강력한 의지와 흔치않은 기회가 필요했지만,
로마법의 기초 계획도는 그가 제국 개혁에 착수하기 오래 전에
이미 수립되어 있었던 것이다.

\para{형평법의 기원}
옛 \wi{만민법}과 \wi{자연법}이 만나는 접점은 무엇인가?
나는 원래의 의미의 \wi{형평}\latin{aequitas}을 통해
이 둘이 만나고 결합된다고 생각한다.
여기서 우리는 \wi{형평법}\latin{equity}이라는 유명한 용어가
법학에 처음 등장함을 보게 된다.
이처럼 그 기원이 멀고 역사가 오래된 표현을 탐구할 때에는,
가능한 한,
일견 어렴풋이 개념의 그림자만 보여주는
단순한 은유나 상징을 파고드는 것이
언제나 가장 안전할 것이다.
흔히들 라틴어의 `형평'이 그리스어 `\wi{이소테스}'\greek{ἰσότης}와 동의어라고
하는데, 후자는 평등한 또는 비례적인 분배의 원리를 뜻한다.
숫자나 물리적 양을 평등하게 나누는 것은 분명 우리의 정의\hanja{正義} 관념과
밀접히 연관되어 있다.
인간의 정신에서 이처럼 강고하게 결합되어 있는 관념 연관을 찾기란 쉽지 않으며
가장 깊이있는 사상가들의 고된 작업을 통해서도 이것을 분리하기가 쉽지 않다.
하지만 이들의 연관을 역사적으로 추적해보면,
아주 초기의 사상에서는 이것이 나타나지 않거니와,
오히려 상대적으로 후대의 철학의 산물인 것으로 보인다.
또한 주목할 점은, 그리스 민주주의가 자랑하는
법의 ``\wi{평등}''\latin{equality}---칼리스트라토스\latin{Callistratus}의
아름다운 권주가에 따르면
하르모디오스\latin{Harmodius}와 아리스토게이톤\latin{Aristogiton}이
아테네인들에게 주었다고 전해지는 그 평등---이
로마인들의 ``\wi{형평}''\latin{equity}과 거의 공통점이 없다는 것이다.
전자는 시민들 사이에, 그 시민의 계급이 비록 낮다고 할지라도,
시민법의 집행이 평등해야 한다는 의미이다.
후자는 시민이 아닌 자를 포함하는 계급에게도 법이, 그러나 시민법은 아닌 법이,
적용될 수 있다는 의미이다.
전자는 폭군을 배제한다는 뜻이고, 후자는 외인을, 경우에 따라서는 노예를,
포함한다는 뜻이다.
대체로 여기서 방향을 약간 틀어 로마인들의 ``형평''이란 단어의
기원을 살펴볼 필요가 있겠다.
라틴어 ``아이쿠스''\latin{aequus}는 그리스어 ``이소스''\greek{ἴσος}보다
\hemph{평평하게 하기}\latin{levelling}라는 의미를 더 명백하게 가진다.\footnote{%
  `aequus'는 `평평한' `고른'이란 뜻으로, 라틴어 `aequitas'(형평)의 어원이 된다.}
이러한 평평하게 하는 경향이야말로 정확히 \wi{만민법}의 성격이었다.
초기 로마인들에게 만민법은 상당히 충격적이었을 것이다.
순수한 로마\wi{시민법}은 사람과 물건에 대해 여러 가지 자의적인 분류를 두고 있었지만,
여러 민족들의 관습에서 일반화된 만민법은 로마시민법상의 구분을
알지 못했다.
예컨대 옛 로마법은 ``\wi{종족}''\hanjalatin{宗族}{agnatic}인 친족과
``\wi{혈족}''\hanjalatin{血族}{cognatic}인 친족을 근본적으로 구분했다.
전자는 공통의 \wi{가부장권}\hanja{家父長權}에 복속하는 가족관계를
지칭하고,\footnote{%
  `종족'은 사실상 `남계혈족'(男系血族)과 거의 같은 뜻이다.
  남계혈족은 나와 상대방(이들은 여자라도 상관없다)을 이어주는
  가계도상의 연결점들이 모두 남자인 경우의 혈족관계를 의미한다. }
후자는 \paren{오늘날의 관념에 일치하는 것으로} 단순히 공통의 혈통으로
결합된 가족관계를 지칭한다.
이러한 구분은 ``모든 민족들에 공통인 법''에서는 존재하지 않았다.
또한 ``\wi{악취물}''\hanjalatin{握取物}{things \textit{mancipi}}과
``비악취물''\hanjalatin{非握取物}{things \textit{nec mancipi}}이라는
물건 분류의 고법\hanja{古法}상의 방식도 존재하지 않았다.
따라서 구분과 경계의 부재는 \wi{형평}\latin{aequitas}으로 묘사되는 만민법의
특징이라 할 수 있다.
나는 이 형평이라는 단어가 처음에는 단지
이러한 끊임없는 \hemph{평평하게 하기}, 즉
울퉁불퉁함의 제거를 뜻했다고 생각한다.
이것은 외인 당사자가 개재된 사건에 법무관법이 적용될 때면 지속적으로 일어났다.
처음에는 이 표현에 어떤 윤리적 의미도 들어있지 않았을 것이다.
또한 초기 로마인들은 이러한 과정을 무척 싫어했을 것이라고
추정하지 않을 이유도 전혀 없다.

\para{형평과 평등}
한편, \wi{형평}이란 말로써 로마인들이 이해한 만민법의 특징은
최초에 생생하게 감지된 가상의 자연상태의 성격과 완전히 일치했다.
자연은 처음에는 물리적 세계의, 나중에는 정신적 세계의, 균형잡힌 질서였고,
질서에 대한 최초의 관념은 분명 직선, 평면, 측정된 거리 같은 것과
관련되어 있었다.
인간 정신의 눈이 가상의 자연상태의 윤곽을 그려내려 할 때든,
혹은 ``모든 민족들에 공통인 법''의 실제 집행을 바라보고 받아들일 때든,
그 정신의 눈 앞에는 이러한 종류의 그림 혹은 상징이 무의식적으로 그려졌을 것이다.
그리고 원시적 사고에 대한 우리의 모든 지식으로 판단하건대,
이러한 관념적 유사성은 이들 두 개념 간의 동일성에 대한 믿음을
불러일으켰을 것이다.
그런데,
예전에 만민법은 로마에서 거의 혹은 전혀 권위를 인정받지 못하던 것이었으나,
\wi{자연법} 이론은 철학적 권위의 위신을 두른 채 들어왔을 뿐 아니라,
그것도 역사가 더 깊고 더 축복받은 민족의 것이라는 매력까지 품고 있었다.
이러한 관점의 차이가
옛 법원리의 작동과 새로운 이론의 결과를 동시에 기술하는 저 용어의 위엄에
어떤 영향을 주었을까는 쉽게 이해할 수 있다.
어떤 과정을 ``평평하게 하기''라고 묘사하는 것과
``잘못된 것의 시정\hanja{是正}''이라고 부르는 것 사이에는,
비록 그 은유는 완전히 똑같은 것이지만,
현대인들이 듣기에도 적지않은 차이가 있다.
`형평'\latin{aequitas}이 저 그리스 이론을 암시하는 것으로 이해되자,
이제 `\wi{이소테스}\index{이소테스|seealso{평등}}'\greek{ἰσότης}라는 그리스적 관념이 형평 개념을 둘러싸기 시작했음에 틀림없다.
키케로의 언어가 이런 일이 실제 일어났음을 보여주고 있거니와,
이는 형평 개념의 변용의 첫 번째 단계였던 것이다.
그리고 그때 이후 등장한 거의 모든 윤리체계는 이 형평 개념을 전승해왔다.

\para{영구고시록}
처음에는 모든 민족들에 공통인 법과 관련되고 나중에는 자연법과 관련되는
법원리와 법개념들이 차츰 로마법에 흡수되어간 형식적 도구에 대해
몇 마디 말해둘 것이 있다.
타르퀴니우스 왕조 축출 사건으로 대변되는 로마 역사상 최초의 위기 시에
많은 고대국가의 초기 연대기에 나타나는 것과 유사한 변화가 일어났지만,
이는 오늘날 우리가 혁명이라고 부르는 정치적 변화와는 거의 공통점이 없는
것이었다.
왕정이 계속 유지되었다고 하는 것이 보다 정확한 기술일 것이다.
지금까지 한 사람의 수중에 집중되었던 권력이
다수의 선출직 관리들 사이에 분할되었으나,
왕이라는 명칭은 나중에
\wi{제사왕}\hanjalatin{祭祀王}{rex sacrorum; rex sacrificulus}이라고
불리게 되는 사람에게 주어져 그대로 유지되었다.
변화의 일부로서 최고 사법관직의 기존 임무는 당시 국가의 최고 관리였던
\wi{법무관}\latin{praetor}에게 부여되었다. 또한
이러한 임무와 더불어
법과 입법에 관한 불명확한 대권\hanja{大權}도 그에게 이전되었거니와,
이는 고대의 통치자들이라면 누구나 가졌던 것이지만
한때 그들이 누렸던 가부장적이고 영웅적인 권위와의 희미한 연관성은
사라지고 없었다.
로마의 상황으로 인해 이렇게 이전된 기능 중에 보다 불명확한 부분이
더 큰 중요성을 가졌는데,
법기술상 본래의 로마인으로 분류할 수 없지만
그러나 로마 법역 안에 상주하고 있는 사람들을 다루는 어려운 문제를
안겨준 재판들이
공화정 수립 이후
지속적으로 제기되기 시작했기 때문이다.
이러한 사람들 간의 쟁송 및 이러한 사람들과 생래적 시민들 간의 쟁송은
법무관이 그 재판업무를 떠맡지 않았다면
로마법상 구제수단이 전혀 주어질 수 없는 것들이었다.
또한 곧이어 상거래가 확산되면서 로마 시민들과 자신을 외인이라고 진술한 사람들 사이에 발생한
보다 중대한 분쟁들에 대해서도 법무관이 대처하지 않으면 안 되었다.
제1차 포에니 전쟁을 전후하여 로마 법원에 이러한 소송이 대폭 증가하자,
후에 외인담당\wi{법무관}\latin{praetor peregrinus}이라 불리게 되는,
이런 종류의 사건만 전담하는 특별한 법무관이 임명되기에 이른다.
한편, 압제의 부활에 대한 로마 인민들의 두려움으로 인해,
업무영역이 확장되는 경향을 가진 모든 정무관은
매년 임기 초에
자신이 맡은 업무를 앞으로 어떻게 수행할 지를 선언하는
\wi{고시}\hanjalatin{告示}{edict}를 공표할 의무가 부과되었다.
다른 정무관들과 함께 법무관도 이 규칙의 적용대상이었다.
그런데 해마다 따로 다수의 법원칙들을 고안해내는 것은 사실상 불가능하므로
\wi{법무관}은 전임자의 고시를 거의 답습하여 재공표하고,
다만 그때그때의 상황에 따라 혹은 자신의 법적 견해에 따라
약간의 추가와 변경을 가하는 데 그쳤던 듯하다.
그리하여 장기간 매년 반복되는 법무관의 선포는
영구고시\latin{edictum perpetuum}라는 이름을 얻게 되었으니,
이는 \hemph{지속적인} 또는 \hemph{중단없는} 고시라는 뜻이다.
이것이 너무나 오랫동안 계속되자,
그리고 아마도 그 무질서해질 수밖에 없는 구조에 대한 염증 때문에,
하드리아누스 황제 재위기에 정무관직에 있었던
살비우스 \wi{율리아누스}\latin{Salvius Julianus}의 임기에 이르러
더 이상의 확장이 중단되게 된다.
그리하여 이 법무관의 \wi{고시}는 \wi{형평법}의 총체였거니와,
아마도 새롭고 체계적인 질서를 갖추었을 것이다.
이후 로마법에서 이 \wi{영구고시록}은 단순히
\index{율리아누스 고시|see{영구고시록}}%
율리아누스 고시\latin{Edict of Julianus}로 흔히 인용되곤 했다.

고시의 특수한 메커니즘을 고찰하는 영국인의 머리에 떠오르는
첫 번째 의문은 이런 것이리라: 법무관의 이러한 포괄적 권한을 통제하는
제약\hanja{制約}은 무엇이었을까? 어떻게 그렇게나 불명확한 권한이
기존의 사회상황 및 법상태와 조화될 수 있었을까?
이에 대한 답변은 우리의 영국법이 운용되는 상황을 면밀히 관찰함으로써만
주어질 수 있을 것이다.
법무관은 그 자신이 법학자이거나, 아니면 법학자인 조언자들의 수중에 있는
사람임을 상기할 필요가 있다.
또한 로마 법률가라면 누구나 저 위대한 사법정무관직에 취임하거나 아니면
그 직을 통제할 날을 손꼽아 기다렸을 것이다.
그 사이 기간동안 그의 취향, 감정, 편견, 그리고 계몽의 정도는
불가피 그의 동료집단의 그것이었으며,
또한 후에 공직에 취임하거나 그 직을 통제하게 될 때의 그의 자질은
그가 전문직으로서 실무와 연구에서 얻었던 것이었다.
영국의 챈슬러도 정확히 동일한 훈련을 거치며, 또한 동일한 종류의 자질을 가지고
챈슬러직을 수행한다.
그가 공직에 취임할 때는, 공직을 떠나기까지 어느 정도는
그가 법을 변경하리라는 것이 확실하다.
하지만 공직을 물러나고 그가 내린 판결들이 판례집에 수록되기
전에는, 그가 전임자에게서 물려받은 법원리를 얼마나 더 분명히 밝히고
또 새로운 것을 추가했는지 우리는 알 수 없다.
로마법에 대한 법무관의 영향도 단지 그 영향의 정도가 확인되는 시기에 있어서만
차이가 날 따름이었다.
전술했듯이 \wi{법무관}의 임기는 1년에 불과했다.
또한 임기 동안 그가 내린 결정은, 물론 소송당사자들에게는 불가역적인 것이었으나,
장래에 대해 구속력을 갖지 않았다.
따라서 그가 계획하는 변화를 선포하는 순간은 당연히
법무관직에 취임하는 순간일 수밖에 없었다.
그리하여 임기 시작 시에 그는,
후에 영국의 챈슬러가 부지불식간에 그리고 때로는 무의식적으로 행하는 것을
공개적이고 명시적으로 수행했다.
그의 외관상의 재량에 대한 통제는 영국 판사들에 대한 통제와
하등 다를 것이 없었다.
이론상으로는 양자의 권한에 거의 아무런 제한이 없는 것처럼 보이나,
실제적으로는 로마의 법무관도 영국의 챈슬러도
초기 훈련 과정에서 습득한 선이해에 의해, 그리고
전문직 그룹의 여론이라는 강력한 제약에 의해
엄격하게 한계지워진다.
이러한 제약의 엄격함은 직접 경험한 사람들만이 실감할 수 있는 것이다.
부연하건대, 움직임이 허락된 공간의 경계선, 넘어서는 안 되는 그 경계선은
영국만큼이나 로마에서도 분명히 그어져 있었다.
영국의 판사들은 고립된 사실관계에 대한 공표된 판결들의 유사성을 따라야 한다.
로마에서는, 법무관의 개입이 처음에는 국가의 안전이라는 단순한 고려에 의해
지배되었기에, 애초에는 제거하고자 하는 문제의 곤란함 정도에 비례해서만
개입이 이루어졌을 것이다.
후에, 법학자의 해답에 의해 법원리에 대한 애호가 확산되자,
법무관은 그러한 근본원리들을 더 폭넓게
적용하기 위한 수단으로 그의 \wi{고시}를 이용했을 것이 틀림없다.
이때 그와 그의 동시대인인 나머지 실무 법학자들은
법의 저변에 놓여있는 그 원리들을 발견했다고 믿었다.
더 시간이 흐른 후에,
법무관은 이제 전적으로 그리스 철학이론의 영향력 하에서
행동했거니와, 이 이론은 특정한 진화의 방향으로 그를 이끄는 동시에
그 방향으로 가도록 그를 한계지웠다.

\para{그후 로마 형평법}
살비우스 \wi{율리아누스}의 조치는 그 성격이 큰 논쟁의 대상이 되었다.
그 성격이 어떠하든 간에, 그것이 고시에 미친 효과는 자못 명백했다.
\wi{고시}는 이제 해마다 확장되기를 그쳤고, 이후로
로마의 \wi{형평법}은 하드리아누스 황제 치세와 알렉산데르 세베루스 황제 치세 사이에
활발하게 저술활동을 펼친 일련의 위대한 법학자들에 의해 발달하게 된다.
그들이 이룩한 경탄스런 체계의 일부가
\wi{유스티니아누스}의 \wi{학설휘찬}\latin{Pandects}에
남아있거니와, 이를 통해 우리는 그들의 작품이 로마법의 모든 영역에 관한
논저의 형태를,
그러나 주로 고시에 대한 주해서의 형태를, 띠고 있었음을 알 수 있다.
실로 이 시대의 어떤 법학자가 어떤 주제 하나를 다루었다 할지라도
그는 형평법의 해설자로 불릴 만하다.
고시에 담긴 법원리들은 이 시대가 끝나기 전에 로마법학의 모든 영역에
침투해 들어갔다.
로마의 형평법은, 비록 \wi{시민법}과 완전히 동떨어진 경우에도,
언제나 동일한 법원에 의해 재판되었다는 점을 잊지 말아야 한다.
\wi{법무관}은 형평법 수석판사인 동시에 보통법 수석판사이기도 했다.
그리하여 고시가 어떤 형평법규칙을 발달시키면,
법무관의 법원은 바로 옛 시민법규칙을 대체하여 혹은 그것과 병행하여
그 형평법규칙을 적용하기 시작했다. 이것은
입법기관의 명시적 법제정 없이 시민법이 직^^b7간접적으로 폐지되는 결과를 낳았다.
물론 이것은 시민법과 형평법의 완전한 통합에는 전혀 이르지 못하는 것이었다.
이 통합은 후에 \wi{유스티니아누스}의 개혁에 의해 비로소 달성된다.
두 영역의 법이 법기술상 분리되어 있다는 사실은
일말의 혼동과 일말의 불편함을 낳았다. 또한
시민법 법리 가운데 아주 완고한 것들은 \wi{고시}의 선포자들도 그 해설자들도
감히 건드리지 못하는 것들이 있었다.
하지만 법학 분야 가운데
형평법의 영향이 다소간이나마 휩쓸고 지나가지 않은 구석은 하나도 없었다.
그것은 법학자들에게 일반화를 위한 자료를,
해석의 방법을, 근본원리들의 해명을 제공했다. 또한
입법자의 개입이 거의 없는,
오히려 입법으로 정해진 법률의 적용에 중대한 통제를 가하는,
다량의 제한적 규칙들도 제공했다.

법학자의 시대는 알렉산데르 세베루스 황제와 더불어 종말을 고한다.
하드리아누스로부터 이 황제에 이르기까지 법의 발달은,
오늘날 대부분의 대륙법계 국가들에서와 마찬가지로,
부분적으로는 공인된 주해에 의해,
부분적으로는 직접적인 \wi{입법}에 의해 이루어졌다.
하지만 알렉산데르 세베루스의 치세에 로마 형평법의 성장력은 소진되었고,
법학자들의 잇따른 등장도 마감되었다.
로마법의 나머지 역사는 황제의 \wi{칙법}\latin{constitution}의 역사이고,
종국에는 오늘날 로마법의 거창한 집적물로 남겨진 것을 편찬하려는
시도의 역사이다.
이런 종류의 실험 중에 최후의 그리고 가장 칭송받는 것으로
\wi{유스티니아누스} 황제의 로마법대전\latin{Corpus Juris}이 우리에게 전해지고 있다.

\para{영국과 로마의 형평법}
영국과 로마의 형평법을 집요하게 비교하고 대비시키는 것이 지루하게 느껴질 수도
있겠다. 하지만 그들의 공통점 두 가지는 언급해둘 가치가 있다.
첫째는 이렇게 말할 수 있을 것이다:
그 둘은 모두, 모든 이러한 체계가 그러하듯이,
\wi{형평법}이 처음 개입했을 때의 옛 보통법의 상태와 정확히 같은 상태에
이르는 경향이 있었다.
최초에 도입된 도덕적 원리들이 모든 정당한 결과들을 낳으며 역할을 다한 후,
그들에 기초한 체계가 굳어지고, 더 이상 확장이 안 되고,
보통법이라 부르는 아주 엄격한 규칙체계와 마찬가지로
도덕적 진보에 뒤처지기 시작하는
시기가 반드시 도래한다.
로마에서는 그 시기가 알렉산데르 세베루스 재위기에 도래했다.
그후, 전체 로마 세계가 정신적 혁명에 휩싸였지만, 로마의 형평법은
더 이상 확장되지 못했다.
영국의 법제사에서는 동일한 시기가 엘던 경\latin{Lord Eldon}이
챈슬러직에 있을 때 도달했다.
간접적인 입법에 의해 형평법을 확장시키는 대신, 그는
형평법을 설명하고 조화시키는 데만 평생을 바친 최초의 형평법 판사였다.
법제사의 교훈이 영국에서 좀 더 잘 이해되었더라다면,
엘던 경의 업적은
당대 법률가들 사이의 평판보다
한편으로는 덜 과장되었을 것이고,
다른 한편으로는 더 나은 평가를 받았을 것이다.
실천적 결과에 영향을 주는 또 다른 오해도 불식되어야 한다.
영국의 형평법이 도덕 규칙들에 기초한 체계임을
영국 법률가라면 누구나 쉽게 이해한다.
하지만 이 규칙들이---현재가 아니라---수 세기 전 과거의 도덕임은 잊고 있다.
그동안 너무 많이 적용되어 능력이 거의 소진될 지경에 이르렀음은 잊고 있다.
그것들이 물론 오늘날의 윤리적 신조와 크게 다르지는 않다 할지라도
오늘날의 그것을 따라잡지 못하는 것일 수 있음은 잊고 있다.
이 주제에 관한 불완전한, 그러나 널리 받아들여지고 있는, 이론들이
서로 상반되는 종류의 오류를 생산해왔다.
형평법에 관한 논저의 다수는
현재 상태의 체계의 완전성에 매료되어
명시적^^b7묵시적으로 역설적인 주장을 펼치고 있거니와,
형평법의 창시자들이 처음 그 기초를 다졌을 때 이미 현재와 같은 고정된 형태를
만들어냈다는 주장이 그것이다.
또한 다른 이들은---법정 변론에서 자주 들리는 불평인데---형평법법원이
강제하는 도덕 규칙들이 오늘날의 윤리 기준에 미치지 못한다고 불평하고 있거니와,
그들은 영국 형평법의 창시자들이 옛 보통법에 대해
행하던 역할과 정확히 동일한 역할을
지금의 챈슬러들이
수행해주기를 형평법에 대해 요구하고 있는 것이다.
하지만 이것은 법의 발달이 진행되는 순서를 거꾸로 뒤집는 것이다.
형평법에는 자신만의 장소와 시간이 있다.
나는 다른 수단이 있음을,
형평법의 에너지가 소진되면 이 수단이 형평법을 대체할 것임을,
앞서 지적한 바 있다.\footnote{%
  물론 이 수단은 `입법'을 의미한다. }

영국과 로마의 형평법의 또 하나의 주목할만한 성격은
\wi{형평법}이 보통법이나 시민법보다 우월하다는 주장이 처음
개진될 때, 이 주장이 근거했던 가정들이 모두 허위였다는 점이다.
개인이든 집단이든 인간에게 있어 도덕적 진보를 실제적 현실로
받아들이는 것만큼 싫은 것이 없다.
이 거부감이 개인에게서는 일관성이라는 의심스런 덕목을 과장되이 존중하는
모습으로 통상 나타난다.
전체 사회의 수준에서도 집단적 여론의 움직임은 너무나 명백해서 무시할 수 없고
대체로 너무나 뚜렷이 더 좋은 것을 향하기에 대놓고 비난할 수 없으나
그것을 주요한 현상으로 인정하기를 꺼리는 경향이 강하게 존재하거니와,
보통은 잃어버린 완전성의 회복---인류가 타락하기 이전 상태로의 점진적
회귀---의 주장으로 나타난다.
이렇게 도덕적 진보의 목표를 앞을 바라보는 데서가 아니라
뒤를 돌아보는 데서 찾는 경향은, 전술했듯이,
고대 로마법에 가장 심각하고 영속적인 영향을 주었다.
로마 법학자들은, 법무관에 의한 법 발달을 설명하기 위해서,
실정법에 의해 통치되는 국가들이 조직되기 이전에 존재한
인간의 자연상태---자연적 사회---의 이론을 그리스로부터 빌려왔다.
한편, 영국에서는, 당시 영국인들의 취향에 특별히 부합하는 관념으로
보통법에 대한 형평법의 우월성을 설명했거니와,
국왕이 갖는 가부장적 권위의 당연한 결과로서 국왕에게는
사법\hanja{司法}을 감독할 일반적 권리가 있다는 것이 그것이었다.
동일한 견해가
``형평법은 국왕의 양심에서 유래한다''는
옛 법리에서
보다 고풍스런 형태로
등장했으니, 이는
실제로는 공동체의 도덕 기준에 진보가 일어난 것을
주권자의 내면적 도덕 감각의 상승으로 치환시키고 있는 것이다.
이후 영국 헌정의 발달로 이러한 이론은 부적합한 것이 되었지만,
형평법법원의 재판권이 확고하게 자리잡음에 따라 이 이론을 대체할 공식적 대체물을
고안할 필요도 없어졌다.
오늘날 형평법 교재들에서 발견되는 이론들은 참으로 다양하지만,
하나같이 유지될 수 없는 이론들뿐이다.
그 대부분은 자연법에 기초한 로마법 이론의 변용이거니와,
이는
자연적 정의와 시민적 정의의 구분으로써
형평법법원의 재판권에 대한 논의를 시작하는
저술가들에 의해
실제로 그대로
채용되고 있다.

